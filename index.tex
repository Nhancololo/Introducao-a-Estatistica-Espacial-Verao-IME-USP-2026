% Options for packages loaded elsewhere
% Options for packages loaded elsewhere
\PassOptionsToPackage{unicode}{hyperref}
\PassOptionsToPackage{hyphens}{url}
\PassOptionsToPackage{dvipsnames,svgnames,x11names}{xcolor}
%
\documentclass[
  brazilian,
  letterpaper,
  DIV=11,
  numbers=noendperiod]{scrreprt}
\usepackage{xcolor}
\usepackage{amsmath,amssymb}
\setcounter{secnumdepth}{5}
\usepackage{iftex}
\ifPDFTeX
  \usepackage[T1]{fontenc}
  \usepackage[utf8]{inputenc}
  \usepackage{textcomp} % provide euro and other symbols
\else % if luatex or xetex
  \usepackage{unicode-math} % this also loads fontspec
  \defaultfontfeatures{Scale=MatchLowercase}
  \defaultfontfeatures[\rmfamily]{Ligatures=TeX,Scale=1}
\fi
\usepackage{lmodern}
\ifPDFTeX\else
  % xetex/luatex font selection
\fi
% Use upquote if available, for straight quotes in verbatim environments
\IfFileExists{upquote.sty}{\usepackage{upquote}}{}
\IfFileExists{microtype.sty}{% use microtype if available
  \usepackage[]{microtype}
  \UseMicrotypeSet[protrusion]{basicmath} % disable protrusion for tt fonts
}{}
\makeatletter
\@ifundefined{KOMAClassName}{% if non-KOMA class
  \IfFileExists{parskip.sty}{%
    \usepackage{parskip}
  }{% else
    \setlength{\parindent}{0pt}
    \setlength{\parskip}{6pt plus 2pt minus 1pt}}
}{% if KOMA class
  \KOMAoptions{parskip=half}}
\makeatother
% Make \paragraph and \subparagraph free-standing
\makeatletter
\ifx\paragraph\undefined\else
  \let\oldparagraph\paragraph
  \renewcommand{\paragraph}{
    \@ifstar
      \xxxParagraphStar
      \xxxParagraphNoStar
  }
  \newcommand{\xxxParagraphStar}[1]{\oldparagraph*{#1}\mbox{}}
  \newcommand{\xxxParagraphNoStar}[1]{\oldparagraph{#1}\mbox{}}
\fi
\ifx\subparagraph\undefined\else
  \let\oldsubparagraph\subparagraph
  \renewcommand{\subparagraph}{
    \@ifstar
      \xxxSubParagraphStar
      \xxxSubParagraphNoStar
  }
  \newcommand{\xxxSubParagraphStar}[1]{\oldsubparagraph*{#1}\mbox{}}
  \newcommand{\xxxSubParagraphNoStar}[1]{\oldsubparagraph{#1}\mbox{}}
\fi
\makeatother

\usepackage{color}
\usepackage{fancyvrb}
\newcommand{\VerbBar}{|}
\newcommand{\VERB}{\Verb[commandchars=\\\{\}]}
\DefineVerbatimEnvironment{Highlighting}{Verbatim}{commandchars=\\\{\}}
% Add ',fontsize=\small' for more characters per line
\usepackage{framed}
\definecolor{shadecolor}{RGB}{241,243,245}
\newenvironment{Shaded}{\begin{snugshade}}{\end{snugshade}}
\newcommand{\AlertTok}[1]{\textcolor[rgb]{0.68,0.00,0.00}{#1}}
\newcommand{\AnnotationTok}[1]{\textcolor[rgb]{0.37,0.37,0.37}{#1}}
\newcommand{\AttributeTok}[1]{\textcolor[rgb]{0.40,0.45,0.13}{#1}}
\newcommand{\BaseNTok}[1]{\textcolor[rgb]{0.68,0.00,0.00}{#1}}
\newcommand{\BuiltInTok}[1]{\textcolor[rgb]{0.00,0.23,0.31}{#1}}
\newcommand{\CharTok}[1]{\textcolor[rgb]{0.13,0.47,0.30}{#1}}
\newcommand{\CommentTok}[1]{\textcolor[rgb]{0.37,0.37,0.37}{#1}}
\newcommand{\CommentVarTok}[1]{\textcolor[rgb]{0.37,0.37,0.37}{\textit{#1}}}
\newcommand{\ConstantTok}[1]{\textcolor[rgb]{0.56,0.35,0.01}{#1}}
\newcommand{\ControlFlowTok}[1]{\textcolor[rgb]{0.00,0.23,0.31}{\textbf{#1}}}
\newcommand{\DataTypeTok}[1]{\textcolor[rgb]{0.68,0.00,0.00}{#1}}
\newcommand{\DecValTok}[1]{\textcolor[rgb]{0.68,0.00,0.00}{#1}}
\newcommand{\DocumentationTok}[1]{\textcolor[rgb]{0.37,0.37,0.37}{\textit{#1}}}
\newcommand{\ErrorTok}[1]{\textcolor[rgb]{0.68,0.00,0.00}{#1}}
\newcommand{\ExtensionTok}[1]{\textcolor[rgb]{0.00,0.23,0.31}{#1}}
\newcommand{\FloatTok}[1]{\textcolor[rgb]{0.68,0.00,0.00}{#1}}
\newcommand{\FunctionTok}[1]{\textcolor[rgb]{0.28,0.35,0.67}{#1}}
\newcommand{\ImportTok}[1]{\textcolor[rgb]{0.00,0.46,0.62}{#1}}
\newcommand{\InformationTok}[1]{\textcolor[rgb]{0.37,0.37,0.37}{#1}}
\newcommand{\KeywordTok}[1]{\textcolor[rgb]{0.00,0.23,0.31}{\textbf{#1}}}
\newcommand{\NormalTok}[1]{\textcolor[rgb]{0.00,0.23,0.31}{#1}}
\newcommand{\OperatorTok}[1]{\textcolor[rgb]{0.37,0.37,0.37}{#1}}
\newcommand{\OtherTok}[1]{\textcolor[rgb]{0.00,0.23,0.31}{#1}}
\newcommand{\PreprocessorTok}[1]{\textcolor[rgb]{0.68,0.00,0.00}{#1}}
\newcommand{\RegionMarkerTok}[1]{\textcolor[rgb]{0.00,0.23,0.31}{#1}}
\newcommand{\SpecialCharTok}[1]{\textcolor[rgb]{0.37,0.37,0.37}{#1}}
\newcommand{\SpecialStringTok}[1]{\textcolor[rgb]{0.13,0.47,0.30}{#1}}
\newcommand{\StringTok}[1]{\textcolor[rgb]{0.13,0.47,0.30}{#1}}
\newcommand{\VariableTok}[1]{\textcolor[rgb]{0.07,0.07,0.07}{#1}}
\newcommand{\VerbatimStringTok}[1]{\textcolor[rgb]{0.13,0.47,0.30}{#1}}
\newcommand{\WarningTok}[1]{\textcolor[rgb]{0.37,0.37,0.37}{\textit{#1}}}

\usepackage{longtable,booktabs,array}
\usepackage{calc} % for calculating minipage widths
% Correct order of tables after \paragraph or \subparagraph
\usepackage{etoolbox}
\makeatletter
\patchcmd\longtable{\par}{\if@noskipsec\mbox{}\fi\par}{}{}
\makeatother
% Allow footnotes in longtable head/foot
\IfFileExists{footnotehyper.sty}{\usepackage{footnotehyper}}{\usepackage{footnote}}
\makesavenoteenv{longtable}
\usepackage{graphicx}
\makeatletter
\newsavebox\pandoc@box
\newcommand*\pandocbounded[1]{% scales image to fit in text height/width
  \sbox\pandoc@box{#1}%
  \Gscale@div\@tempa{\textheight}{\dimexpr\ht\pandoc@box+\dp\pandoc@box\relax}%
  \Gscale@div\@tempb{\linewidth}{\wd\pandoc@box}%
  \ifdim\@tempb\p@<\@tempa\p@\let\@tempa\@tempb\fi% select the smaller of both
  \ifdim\@tempa\p@<\p@\scalebox{\@tempa}{\usebox\pandoc@box}%
  \else\usebox{\pandoc@box}%
  \fi%
}
% Set default figure placement to htbp
\def\fps@figure{htbp}
\makeatother


% definitions for citeproc citations
\NewDocumentCommand\citeproctext{}{}
\NewDocumentCommand\citeproc{mm}{%
  \begingroup\def\citeproctext{#2}\cite{#1}\endgroup}
\makeatletter
 % allow citations to break across lines
 \let\@cite@ofmt\@firstofone
 % avoid brackets around text for \cite:
 \def\@biblabel#1{}
 \def\@cite#1#2{{#1\if@tempswa , #2\fi}}
\makeatother
\newlength{\cslhangindent}
\setlength{\cslhangindent}{1.5em}
\newlength{\csllabelwidth}
\setlength{\csllabelwidth}{3em}
\newenvironment{CSLReferences}[2] % #1 hanging-indent, #2 entry-spacing
 {\begin{list}{}{%
  \setlength{\itemindent}{0pt}
  \setlength{\leftmargin}{0pt}
  \setlength{\parsep}{0pt}
  % turn on hanging indent if param 1 is 1
  \ifodd #1
   \setlength{\leftmargin}{\cslhangindent}
   \setlength{\itemindent}{-1\cslhangindent}
  \fi
  % set entry spacing
  \setlength{\itemsep}{#2\baselineskip}}}
 {\end{list}}
\usepackage{calc}
\newcommand{\CSLBlock}[1]{\hfill\break\parbox[t]{\linewidth}{\strut\ignorespaces#1\strut}}
\newcommand{\CSLLeftMargin}[1]{\parbox[t]{\csllabelwidth}{\strut#1\strut}}
\newcommand{\CSLRightInline}[1]{\parbox[t]{\linewidth - \csllabelwidth}{\strut#1\strut}}
\newcommand{\CSLIndent}[1]{\hspace{\cslhangindent}#1}

\ifLuaTeX
\usepackage[bidi=basic]{babel}
\else
\usepackage[bidi=default]{babel}
\fi
% get rid of language-specific shorthands (see #6817):
\let\LanguageShortHands\languageshorthands
\def\languageshorthands#1{}


\setlength{\emergencystretch}{3em} % prevent overfull lines

\providecommand{\tightlist}{%
  \setlength{\itemsep}{0pt}\setlength{\parskip}{0pt}}



 


\usepackage{booktabs}
\usepackage{caption}
\usepackage{longtable}
\usepackage{colortbl}
\usepackage{array}
\usepackage{anyfontsize}
\usepackage{multirow}
\usepackage{wrapfig}
\usepackage{float}
\usepackage{pdflscape}
\usepackage{tabu}
\usepackage{threeparttable}
\usepackage{threeparttablex}
\usepackage[normalem]{ulem}
\usepackage{makecell}
\usepackage{xcolor}
\usepackage{siunitx}

    \newcolumntype{d}{S[
      table-align-text-before=false,
      table-align-text-after=false,
      input-symbols={-,\*+()}
    ]}
  
\KOMAoption{captions}{tableheading}
\makeatletter
\@ifpackageloaded{tcolorbox}{}{\usepackage[skins,breakable]{tcolorbox}}
\@ifpackageloaded{fontawesome5}{}{\usepackage{fontawesome5}}
\definecolor{quarto-callout-color}{HTML}{909090}
\definecolor{quarto-callout-note-color}{HTML}{0758E5}
\definecolor{quarto-callout-important-color}{HTML}{CC1914}
\definecolor{quarto-callout-warning-color}{HTML}{EB9113}
\definecolor{quarto-callout-tip-color}{HTML}{00A047}
\definecolor{quarto-callout-caution-color}{HTML}{FC5300}
\definecolor{quarto-callout-color-frame}{HTML}{acacac}
\definecolor{quarto-callout-note-color-frame}{HTML}{4582ec}
\definecolor{quarto-callout-important-color-frame}{HTML}{d9534f}
\definecolor{quarto-callout-warning-color-frame}{HTML}{f0ad4e}
\definecolor{quarto-callout-tip-color-frame}{HTML}{02b875}
\definecolor{quarto-callout-caution-color-frame}{HTML}{fd7e14}
\makeatother
\makeatletter
\@ifpackageloaded{bookmark}{}{\usepackage{bookmark}}
\makeatother
\makeatletter
\@ifpackageloaded{caption}{}{\usepackage{caption}}
\AtBeginDocument{%
\ifdefined\contentsname
  \renewcommand*\contentsname{Índice}
\else
  \newcommand\contentsname{Índice}
\fi
\ifdefined\listfigurename
  \renewcommand*\listfigurename{Lista de Figuras}
\else
  \newcommand\listfigurename{Lista de Figuras}
\fi
\ifdefined\listtablename
  \renewcommand*\listtablename{Lista de Tabelas}
\else
  \newcommand\listtablename{Lista de Tabelas}
\fi
\ifdefined\figurename
  \renewcommand*\figurename{Figura}
\else
  \newcommand\figurename{Figura}
\fi
\ifdefined\tablename
  \renewcommand*\tablename{Tabela}
\else
  \newcommand\tablename{Tabela}
\fi
}
\@ifpackageloaded{float}{}{\usepackage{float}}
\floatstyle{ruled}
\@ifundefined{c@chapter}{\newfloat{codelisting}{h}{lop}}{\newfloat{codelisting}{h}{lop}[chapter]}
\floatname{codelisting}{Listagem}
\newcommand*\listoflistings{\listof{codelisting}{Lista de Listagens}}
\makeatother
\makeatletter
\makeatother
\makeatletter
\@ifpackageloaded{caption}{}{\usepackage{caption}}
\@ifpackageloaded{subcaption}{}{\usepackage{subcaption}}
\makeatother
\usepackage{bookmark}
\IfFileExists{xurl.sty}{\usepackage{xurl}}{} % add URL line breaks if available
\urlstyle{same}
\hypersetup{
  pdflang={pt-BR},
  colorlinks=true,
  linkcolor={blue},
  filecolor={Maroon},
  citecolor={Blue},
  urlcolor={Blue},
  pdfcreator={LaTeX via pandoc}}


\author{}
\date{}
\begin{document}

\renewcommand*\contentsname{Índice}
{
\hypersetup{linkcolor=}
\setcounter{tocdepth}{2}
\tableofcontents
}

\bookmarksetup{startatroot}

\chapter*{Prefácio}\label{prefuxe1cio}
\addcontentsline{toc}{chapter}{Prefácio}

\markboth{Prefácio}{Prefácio}

O presente texto constitui as notas de aula do curso de verão sobre
Introdução à Estatística Espacial, a ser ministrado no Instituto de
Matemática e Ciência da Computação da Universidade de São Paulo
(IME-USP)\footnote{O nome do Instituto foi atualizado em 2025 para incorporar explicitamente a Ciência da Computação.}.
O conteúdo programático abrange os três pilares fundamentais da área:
geoestatística, dados de área e processos pontuais, tratados aqui sob
uma perspectiva introdutória.

Dada a imprescindibilidade da computação para a aplicação prática dos
conceitos, precede-se a apresentação da teoria espacial com uma
introdução ao ambiente \texttt{R\ e\ RStudio}. Ressalta-se, contudo, que
este material possui caráter de notas de aula e não pretende exaurir o
tema.

Para aqueles interessados em um aprofundamento teórico e prático,
recomenda-se enfaticamente a obra \textbf{ANÁLISE DE DADOS ESPACIAIS COM
APLICAÇÃOES EM R}, de autoria do Professor \textbf{João Domingos Scalon}
(Universidade Federal de Lavras - UFLA). Trata-se, até o momento (pelo
que sei), da única obra em língua portuguesa que contempla as três
grandes subáreas da estatística espacial supracitadas. \textbf{Aos
alunos inscritos no curso de verão de 2026, a
\href{https://www.editoraufla.com.br/}{Livraria UFLA} oferecerá um
DESCONTO de 20\% na aquisição deste livro, cujo custo já é bastante
acessível.}

Aos estudantes que buscam incorporar esta disciplina em seus programas
de graduação ou pós-graduação, a
\href{https://sigaa.ufla.br/sigaa/public/programa/portal.jsf?lc=pt_BR&id=1702}{UFLA}
oferta regularmente a disciplina de Geoestatística. Adicionalmente, para
quem seja econometrista e queira a visão econométrica da estatística
espacial o professor
\href{https://scholar.google.com/citations?user=BvpRIv4AAAAJ&hl=en}{André
Luis Squarize Chagas} da Faculdade de Economia, Administração,
Contabilidade e Atuária da Universidade de São Paulo, Butantã (FEA-USP),
ministra as disciplinas de
\href{https://www.fea.usp.br/economia/pos-graduacao/estrutura-curricular-e-disciplinas/disciplinas-do-programa/disciplinas/disciplinas?area=12138&disc=EAE6050}{econometria
espacial I} e
\href{https://www.fea.usp.br/economia/pos-graduacao/estrutura-curricular-e-disciplinas/disciplinas-do-programa/disciplinas/disciplinas?area=12138&disc=EAE6051}{econometria
espacial II}. Tais cursos possuem enfoque em dados de área e apresentam
rigoroso desenvolvimento matemático e computacional (R/Rstudio).

No cenário da literatura internacional, recomenda-se a leitura de
Model-based Geostatistics dos professores
\href{http://www.leg.ufpr.br/~paulojus/}{Paulo Justiniano Ribeiro} e
\href{https://ysph.yale.edu/profile/peter-diggle/}{Peter Diggle}; e
Spatial Statistics for Data Science: Theory and Practice with R, da
professora \href{https://www.paulamoraga.com/book-spatial/}{Paula
Moraga}. Outras referências essenciais incluem obras de dos professores
\href{https://mdsr-book.github.io/mdsr2e/}{Benjamin S. Baumer, Daniel T.
Kaplan, and Nicholas J. Horton},
\href{https://scholar.google.com/citations?user=xM69PQ0AAAAJ&hl=en}{Adrian
Baddeley},
\href{https://scholar.google.com/citations?user=BTVSL8cAAAAJ&hl=en}{Noel
Cressie}. Para o estudo de Regressão Ponderada Geograficamente
(\href{https://en.wikipedia.org/wiki/GWR}{GWR}), sugere-se o livro dos
professores Brunsdon, Charlton e Fotheringham, complementada pelos
artigos do professor
\href{https://scholar.google.com/citations?user=so7sG5kAAAAJ&hl=pt-PT}{Alan
Ricardo da Silva} da UnB.

Sob a ótica da inferência Bayesiana e não só, destacam-se as
contribuições dos professores
\href{https://scholar.google.com/citations?user=zdJt7UYAAAAJ&hl=en}{Dani
Gamerman} (UFRJ),
\href{https://scholar.google.com/citations?user=fYTIHUYAAAAJ&hl=en}{Alexandra
M. Schmidt} (McGill University),
\href{https://scholar.google.com.br/citations?user=Ujun714AAAAJ&hl=pt-BR}{Marcos
Oliveira Prates} (UFMG),
\href{https://scholar.google.com/citations?user=lUecPc4AAAAJ&hl=en}{Guilherme
Ludwig} (UNICAMP), bem como os livros e artigos do professor
\href{https://scholar.google.es/citations?user=Teoz1w0AAAAJ&hl=es}{Jorge
Mateu} (Universitat Jaume I). Outro livro que merece menção é a obra dos
professores
\href{https://buscatextual.cnpq.br/buscatextual/visualizacv.do;jsessionid=9B72E3F1EB9B42B57E6C5F66E806115D.buscatextual_0}{Jorge
Kazuo Yamamoto} e
\href{https://buscatextual.cnpq.br/buscatextual/visualizacv.do}{Paulo M.
Barbosa Landim}, intitulada \emph{Geoestatística: conceitos e
aplicações}.

A comunidade científica de estatística espacial é vasta; os nomes aqui
citados representam apenas uma fração de excelentes pesquisadores, tanto
no Brasil quanto no exterior, cujas obras merecem ser exploradas.

\begin{tcolorbox}[enhanced jigsaw, left=2mm, toptitle=1mm, colback=white, colframe=quarto-callout-important-color-frame, colbacktitle=quarto-callout-important-color!10!white, opacityback=0, rightrule=.15mm, bottomtitle=1mm, arc=.35mm, title=\textcolor{quarto-callout-important-color}{\faExclamation}\hspace{0.5em}{Nota sobre Direitos Autorais e Citação}, titlerule=0mm, bottomrule=.15mm, leftrule=.75mm, coltitle=black, toprule=.15mm, breakable, opacitybacktitle=0.6]

O conteúdo da Capítulo~\ref{sec-proc_pont} é baseado na dissertação de
mestrado do Alex Monito Nhancololo, desenvolvida sob orientação do
Prof.~Dr.~João Domingos Scalon na Universidade Federal de Lavras (UFLA).
Caso utilize informações ou ilustrações deste capítulo, por favor, cite
a fonte original: Nhancololo (2024a).

\end{tcolorbox}

\part{Introdução ao R/Rstudio}

\chapter{Introdução ao R/RStudio}\label{sec-capitulo1}

A leitura desta seção é fundamental para compreender a base
computacional deste material. Utilizamos o ambiente \emph{R/RStudio} em
todas as análises aqui apresentadas.

Ainda que existam alternativas no mercado como
\href{https://pt.wikipedia.org/wiki/Python}{Python},
\href{https://pt.wikipedia.org/wiki/SPSS}{SPSS},
\href{https://en.wikipedia.org/wiki/Julia_(programming_language)}{julia},
\href{https://en.wikipedia.org/wiki/Jamovi}{Jamovi},
\href{https://en.wikipedia.org/wiki/Microsoft_Excel}{Excel}, etc. A
escolha pelo \texttt{R} para o ensino de Estatística (especialmente a
Espacial) é deliberada. Além de ser um software gratuito, de código
livre (\href{https://en.wikipedia.org/wiki/Open_source}{open source}) e
criado para computação estatística e gráfica, o \texttt{R} destaca-se
pela vasta biblioteca de pacotes dedicados à fronteira do conhecimento
científico.

\begin{tcolorbox}[enhanced jigsaw, left=2mm, toptitle=1mm, colback=white, colframe=quarto-callout-note-color-frame, colbacktitle=quarto-callout-note-color!10!white, opacityback=0, rightrule=.15mm, bottomtitle=1mm, arc=.35mm, title=\textcolor{quarto-callout-note-color}{\faInfo}\hspace{0.5em}{Fontes e Referências}, titlerule=0mm, bottomrule=.15mm, leftrule=.75mm, coltitle=black, toprule=.15mm, breakable, opacitybacktitle=0.6]

A elaboração deste conteúdo é fruto da prática do autor, alicerçada na
literatura fundamental da comunidade R. Recomendamos a leitura
complementar das seguintes obras base: Paradis (2005), Venables, Smith,
e R Core Team (2015), Wickham, Çetinkaya-Rundel, e Grolemund (2023),
Wickham (2016) e Wickham (2019), entre outras.

\end{tcolorbox}

\section{O Ecossistema R vs RStudio}\label{o-ecossistema-r-vs-rstudio}

Uma distinção fundamental para quem está começando:

\begin{itemize}
\item
  \textbf{R:} é uma linguagem de programação e um ambiente de software
  livre usado para analisar dados, criar gráficos e aplicar métodos
  estatísticos. Foi criado por
  \href{https://en.wikipedia.org/wiki/Ross_Ihaka}{Ross Ihaka} e
  \href{https://en.wikipedia.org/wiki/Robert_Gentleman_(statistician)}{Robert
  Gentleman} na Universidade de Auckland, Nova Zelândia, em 1993,
  inspirado na
  \href{https://pt.wikipedia.org/wiki/S_(linguagem_de_programa\%C3\%A7\%C3\%A3o)}{linguagem
  S}. Hoje, é muito utilizado por estatísticos, cientistas de dados e
  pesquisadores em várias áreas, contando com uma grande comunidade que
  desenvolve pacotes e amplia suas funcionalidades.
\item
  \textbf{RStudio:} é um programa que facilita o uso do \texttt{R}. Ele
  oferece uma interface gráfica amigável, onde o usuário pode escrever e
  executar códigos, visualizar gráficos, organizar arquivos e acompanhar
  os objetos criados durante a análise. É um Ambiente de Desenvolvimento
  Integrado (IDE), ou seja, um espaço que reúne várias ferramentas para
  programar de forma mais prática e organizada. O RStudio foi
  desenvolvido pela empresa RStudio, PBC (atualmente
  \href{https://en.wikipedia.org/wiki/Posit_PBC}{Posit Software, PBC}),
  fundada por
  \href{https://en.wikipedia.org/wiki/Joseph_J._Allaire}{Joseph J.
  Allaire}, e teve sua primeira versão lançada em 2011.
\end{itemize}

\begin{tcolorbox}[enhanced jigsaw, left=2mm, toptitle=1mm, colback=white, colframe=quarto-callout-tip-color-frame, colbacktitle=quarto-callout-tip-color!10!white, opacityback=0, rightrule=.15mm, bottomtitle=1mm, arc=.35mm, title=\textcolor{quarto-callout-tip-color}{\faLightbulb}\hspace{0.5em}{Saiba mais}, titlerule=0mm, bottomrule=.15mm, leftrule=.75mm, coltitle=black, toprule=.15mm, breakable, opacitybacktitle=0.6]

Clique \href{https://www.youtube.com/watch?v=4uo4r_qeui0}{aqui} para ver
o vídeo produzido pela \textbf{\(C^2\) Conexão Ciência}. Para mais
informações sobre \texttt{R}, Clique
\href{https://pt.wikipedia.org/wiki/R_(linguagem_de_programa\%C3\%A7\%C3\%A3o)}{aqui}.

\end{tcolorbox}

O RStudio divide-se em quatro painéis principais que você deve conhecer
(Figura~\ref{fig-parts_rstudio}):

\begin{enumerate}
\def\labelenumi{\arabic{enumi}.}
\item
  \textbf{Source Editor (Editor de Fonte):} Onde você escreve seus
  scripts (.R) e relatórios em Rmarkdown (.Rmd) ou Quarto (.qmd). É aqui
  que o código é salvo.
\item
  \textbf{Console:} Onde o código é executado interativamente e os
  resultados de texto aparecem imediatamente.
\item
  \textbf{Environment (Ambiente):} Mostra todos os objetos (dados,
  variáveis, funções) carregados na memória RAM (Workspace).
\item
  \textbf{Files/Plots/Help:} Área multifuncional para gerenciamento de
  arquivos, visualização de gráficos e consulta de documentação.
\end{enumerate}

\begin{figure}

\centering{

\includegraphics[width=1\linewidth,height=\textheight,keepaspectratio]{Figures/parts_rstudio.png}

}

\caption{\label{fig-parts_rstudio}Página de download do R no CRAN.}

\end{figure}%

\section{Instalação do R/Rstudio}\label{instalauxe7uxe3o-do-rrstudio}

Para usar o \texttt{RStudio}, você deve primeiro instalar o \texttt{R}.

\textbf{Passo 1: Instalar o R}

\begin{enumerate}
\def\labelenumi{\arabic{enumi}.}
\tightlist
\item
  Acesse o site oficial do R: Clique
  \href{https://cran.r-project.org/}{aqui} e assista ao vídeo colocado
  no final destas instruções.
\item
  Escolha o link de download para o seu sistema operacional (Linux,
  MacOS ou Windows) (veja a Figura~\ref{fig-R}, onde seleciono Windows
  porque é meu sistema operacional. Se não conhece o seu sistema
  operacional, veja o vídeo do \textbf{Ativando o Saber}: Clique
  \href{https://www.youtube.com/watch?v=uy7FWVd_h6M}{aqui}. Se em algum
  lugar aparecer \emph{Windows}, então você usa \emph{Windows}).
\item
  Clique em base (ou na versão mais recente recomendada) para baixar o
  instalador principal.
\item
  Execute o arquivo baixado e siga as instruções, mantendo as
  configurações padrão.
\end{enumerate}

Assista ao vídeo 1 da
\href{https://curso.fernandafperes.com.br/}{Professora Fernanda Peres}:
Clique
\href{https://www.youtube.com/watch?v=WVogdSlk7gY&list=PLOw62cBQ5j9VE9X4cCCfFMjW_hhEAJUhU}{aqui}
para ver o vídeo.

\begin{figure}

\centering{

\includegraphics[width=1\linewidth,height=\textheight,keepaspectratio]{Figures/R.png}

}

\caption{\label{fig-R}Página de download do R no CRAN.}

\end{figure}%

\textbf{Passo 2: Instalar o RStudio}

\begin{enumerate}
\def\labelenumi{\arabic{enumi}.}
\tightlist
\item
  Visite a página de download do RStudio: Clique
  \href{https://www.rstudio.com/products/rstudio/download/}{aqui}.
\item
  A versão gratuita ``RStudio Desktop'' é a recomendada para iniciantes.
  Clique no botão de download (a Figura~\ref{fig-Rstudio} mostra isso e
  o vídeo contém o tutorial).
\item
  O site geralmente detecta seu sistema operacional e sugere o
  instalador correto. Baixe-o.
\item
  Execute o instalador e siga as instruções, mantendo as opções padrão.
\end{enumerate}

\begin{figure}

\centering{

\includegraphics[width=1\linewidth,height=\textheight,keepaspectratio]{Figures/Link R.png}

}

\caption{\label{fig-Rstudio}Página de download do RStudio.}

\end{figure}%

\begin{tcolorbox}[enhanced jigsaw, left=2mm, toptitle=1mm, colback=white, colframe=quarto-callout-note-color-frame, colbacktitle=quarto-callout-note-color!10!white, opacityback=0, rightrule=.15mm, bottomtitle=1mm, arc=.35mm, title=\textcolor{quarto-callout-note-color}{\faInfo}\hspace{0.5em}{Nota}, titlerule=0mm, bottomrule=.15mm, leftrule=.75mm, coltitle=black, toprule=.15mm, breakable, opacitybacktitle=0.6]

Assim que instalar o \texttt{R} e o \texttt{RStudio}, sempre que quiser
fazer análises, use o \texttt{RStudio} clicando nele duas vezes. Se não
encontrar o \texttt{RStudio} na área de trabalho, pesquise por
\textbf{Rstu} em uma das opções abaixo (1 ou 2) mostradas na
Figura~\ref{fig-app_sumiu}.

\end{tcolorbox}

\begin{figure}

\centering{

\includegraphics[width=1\linewidth,height=\textheight,keepaspectratio]{Figures/sumiu aplicativo.jpg}

}

\caption{\label{fig-app_sumiu}Encontrar aplicativo que sumiu.}

\end{figure}%

\section{Atalhos de teclado
essenciais}\label{atalhos-de-teclado-essenciais}

\begin{itemize}
\tightlist
\item
  \texttt{Ctrl+Enter} (no sistema operacional Windows) /
  \texttt{Cmd+Enter} (no Mac): Executa a linha de código atual.
\item
  \texttt{Alt+-}: Insere o operador de atribuição
  (\texttt{\textless{}-}).
\item
  \texttt{Ctrl+Shift+M}: Insere o operador Pipe
  (\texttt{\%\textgreater{}\%} ou \texttt{\textbar{}\textgreater{}}).
\end{itemize}

\begin{tcolorbox}[enhanced jigsaw, left=2mm, toptitle=1mm, colback=white, colframe=quarto-callout-note-color-frame, colbacktitle=quarto-callout-note-color!10!white, opacityback=0, rightrule=.15mm, bottomtitle=1mm, arc=.35mm, title=\textcolor{quarto-callout-note-color}{\faInfo}\hspace{0.5em}{Pipe (\texttt{\%\textgreater{}\%} ou \texttt{\textbar{}\textgreater{}})}, titlerule=0mm, bottomrule=.15mm, leftrule=.75mm, coltitle=black, toprule=.15mm, breakable, opacitybacktitle=0.6]

O Pipe transfere o resultado da expressão à esquerda para ser o primeiro
argumento da função à direita, transformando a difícil leitura
matemática aninhada \(f(g(x))\) em uma sequência linear lógica
\(x \to g \to f\). Ele é fundamental para a higiene do código,
permitindo ler o script como uma receita (``pegue os dados, \emph{e
então} filtre, \emph{e então} calcule''), eliminando o excesso de
parênteses e variáveis temporárias.

\end{tcolorbox}

\begin{itemize}
\tightlist
\item
  \textbf{Sem Pipe} (Leitura: de dentro para fora, propenso a erro):
  localize a função central (\texttt{subset}) para filtrar os dados,
  depois expanda o olhar para calcular a média (\texttt{mean}) desse
  resultado, e finalmente olhe para as extremidades para aplicar o
  arredondamento (\texttt{round}) com o argumento 2 que está longe do
  início.
\end{itemize}

\begin{Shaded}
\begin{Highlighting}[]
\FunctionTok{round}\NormalTok{(}\FunctionTok{mean}\NormalTok{(}\FunctionTok{subset}\NormalTok{(dados, valor }\SpecialCharTok{\textgreater{}} \DecValTok{10}\NormalTok{)}\SpecialCharTok{$}\NormalTok{valor, }\AttributeTok{na.rm =} \ConstantTok{TRUE}\NormalTok{), }\DecValTok{2}\NormalTok{)}
\end{Highlighting}
\end{Shaded}

\begin{itemize}
\tightlist
\item
  \textbf{Com Pipe} (Leitura Lógica: da esquerda para a direita): Pegue
  os \texttt{dados}, \emph{e então} filtre onde
  \texttt{valor\ \textgreater{}\ 10}, \emph{e então} calcule a média,
  \emph{e então} arredonde o resultado final para 2 casas, seguindo a
  ordem lógica e cronológica da execução.
\end{itemize}

\begin{Shaded}
\begin{Highlighting}[]
\NormalTok{dados }\SpecialCharTok{|\textgreater{}} 
  \FunctionTok{subset}\NormalTok{(valor }\SpecialCharTok{\textgreater{}} \DecValTok{10}\NormalTok{) }\SpecialCharTok{|\textgreater{}} 
  \FunctionTok{mean}\NormalTok{(}\AttributeTok{na.rm =} \ConstantTok{TRUE}\NormalTok{) }\SpecialCharTok{|\textgreater{}} 
  \FunctionTok{round}\NormalTok{(}\DecValTok{2}\NormalTok{)}
\end{Highlighting}
\end{Shaded}

\begin{itemize}
\tightlist
\item
  \texttt{Esc}: Interrompe um comando que está demorando muito ou
  travou.
\item
  \texttt{Ctrl+Shift+F10}: Reinicia a sessão do R (limpa a memória RAM).
\item
  \texttt{Ctrl\ +\ Shift\ +\ C}: Comentar/ignorar algo no Windows/Linux
\item
  \texttt{Cmd\ +\ Shift\ +\ C}: Comentar/ignorar algo no Mac
\end{itemize}

\begin{tcolorbox}[enhanced jigsaw, left=2mm, toptitle=1mm, colback=white, colframe=quarto-callout-tip-color-frame, colbacktitle=quarto-callout-tip-color!10!white, opacityback=0, rightrule=.15mm, bottomtitle=1mm, arc=.35mm, title=\textcolor{quarto-callout-tip-color}{\faLightbulb}\hspace{0.5em}{Dica}, titlerule=0mm, bottomrule=.15mm, leftrule=.75mm, coltitle=black, toprule=.15mm, breakable, opacitybacktitle=0.6]

Além de \texttt{Ctrl\ +\ Shift\ +\ C} ou \texttt{Cmd\ +\ Shift\ +\ C},
você também pode usar o caractere \texttt{\#} para instruir o R a
ignorar completamente tudo o que estiver escrito à direita dele na mesma
linha. Isso serve para deixar anotações para você mesmo (ou outros
programadores) sem interferir na execução do código.

\end{tcolorbox}

\begin{Shaded}
\begin{Highlighting}[]
\NormalTok{x }\OtherTok{\textless{}{-}} \DecValTok{10}
\CommentTok{\# x \textless{}{-} 20  \textless{}{-}{-} Esta linha não vai rodar, o valor continua 10.}
\NormalTok{y }\OtherTok{\textless{}{-}}\NormalTok{ x }\SpecialCharTok{+} \DecValTok{5}
\end{Highlighting}
\end{Shaded}

\textbf{A Filosofia do objeto}

No R, você raramente verá uma tela cheia de resultados imediatos após
uma análise complexa. O R opera salvando resultados em \emph{objetos}.
Se você roda uma regressão linear, o resultado não é apenas um texto
impresso, mas um objeto complexo que pode ser consultado, manipulado e
usado como entrada para outras funções posteriormente.

\section{Fundamentos da linguagem R}\label{fundamentos-da-linguagem-r}

Antes de algoritmos complexos, você precisa entender como o R processa
informações básicas.

\subsection{Ajuda e Documentação}\label{ajuda-e-documentauxe7uxe3o}

O R possui um sistema de documentação robusto integrado. Aprenda a pedir
ajuda:

\begin{itemize}
\tightlist
\item
  \texttt{help(topic)}: Abre a documentação completa e técnica sobre o
  tópico especificado. Exemplo, se você quiser saber o que a função
  \texttt{str()} faz, faça \texttt{help(str)}
\end{itemize}

\begin{Shaded}
\begin{Highlighting}[]
\FunctionTok{help}\NormalTok{(str) }
\end{Highlighting}
\end{Shaded}

\begin{itemize}
\tightlist
\item
  \texttt{?topic}: Um atalho rápido para o comando acima (ex:
  \texttt{?mean}).
\end{itemize}

\begin{Shaded}
\begin{Highlighting}[]
\NormalTok{?str }
\end{Highlighting}
\end{Shaded}

\begin{itemize}
\tightlist
\item
  \texttt{help.search("termo")} ou \texttt{??termo}: Pesquisa o termo em
  todo o sistema de ajuda (títulos e descrições), útil quando você não
  sabe o nome exato da função (ex: \texttt{??regression}).
\end{itemize}

\begin{Shaded}
\begin{Highlighting}[]
\FunctionTok{help.search}\NormalTok{(}\StringTok{"str"}\NormalTok{)}
\end{Highlighting}
\end{Shaded}

\begin{itemize}
\tightlist
\item
  \texttt{apropos("termo")}: Retorna uma lista com os nomes de todos os
  objetos e funções disponíveis no ambiente atual que contêm o termo
  (ex: \texttt{apropos("test")}).
\end{itemize}

\begin{Shaded}
\begin{Highlighting}[]
\FunctionTok{apropos}\NormalTok{(}\StringTok{"str"}\NormalTok{) }\CommentTok{\#descomente}
\end{Highlighting}
\end{Shaded}

\begin{itemize}
\tightlist
\item
  \texttt{help.start()}: Inicia a versão HTML completa da ajuda no seu
  navegador padrão.
\end{itemize}

\begin{Shaded}
\begin{Highlighting}[]
\FunctionTok{help.start}\NormalTok{() }\CommentTok{\#descomente}
\end{Highlighting}
\end{Shaded}

\subsection{Operadores e
Atribuição}\label{operadores-e-atribuiuxe7uxe3o}

\textbf{Atribuição}

\begin{tcolorbox}[enhanced jigsaw, left=2mm, toptitle=1mm, colback=white, colframe=quarto-callout-important-color-frame, colbacktitle=quarto-callout-important-color!10!white, opacityback=0, rightrule=.15mm, bottomtitle=1mm, arc=.35mm, title=\textcolor{quarto-callout-important-color}{\faExclamation}\hspace{0.5em}{Atenção: O R é case-sensitive}, titlerule=0mm, bottomrule=.15mm, leftrule=.75mm, coltitle=black, toprule=.15mm, breakable, opacitybacktitle=0.6]

O R diferencia letras maiúsculas de minúsculas. Isso vale para nomes de
funções, variáveis e também para caminhos de arquivos e pastas no seu
computador.

\end{tcolorbox}

\begin{itemize}
\tightlist
\item
  \texttt{\textless{}-}: Atribuição à esquerda. É a convenção padrão e
  histórica do \texttt{R}. Cria uma variável. Se eu quisesse criar um
  objeto de nome ``Alex'', recebendo número 4, temos:
\end{itemize}

\begin{Shaded}
\begin{Highlighting}[]
\NormalTok{Alex }\OtherTok{\textless{}{-}} \DecValTok{4}
\end{Highlighting}
\end{Shaded}

\begin{itemize}
\tightlist
\item
  \texttt{=}: Atribuição à esquerda. Funciona, mas é preferível usar
  apenas para definir argumentos dentro de funções.
\end{itemize}

\begin{Shaded}
\begin{Highlighting}[]
\NormalTok{Alex }\OtherTok{=} \DecValTok{4}
\end{Highlighting}
\end{Shaded}

\begin{itemize}
\tightlist
\item
  \texttt{\textless{}\textless{}-}: Superatribuição. Atribui um valor a
  uma variável no ambiente pai (global), geralmente usado dentro de
  funções avançadas para alterar variáveis externas.
\end{itemize}

\begin{Shaded}
\begin{Highlighting}[]
\NormalTok{total }\OtherTok{\textless{}{-}} \DecValTok{100} \CommentTok{\# Variável definida no ambiente global}

\CommentTok{\# Função com atribuição normal (\textless{}{-})}
\NormalTok{teste\_local }\OtherTok{\textless{}{-}} \ControlFlowTok{function}\NormalTok{() \{}
\NormalTok{  total }\OtherTok{\textless{}{-}} \DecValTok{50}  \CommentTok{\# Cria uma nova variável LOCAL chamada \textquotesingle{}total\textquotesingle{}. A global continua 100.}
\NormalTok{\}}

\CommentTok{\# Função com superatribuição (\textless{}\textless{}{-})}
\NormalTok{teste\_global }\OtherTok{\textless{}{-}} \ControlFlowTok{function}\NormalTok{() \{}
\NormalTok{  total }\OtherTok{\textless{}\textless{}{-}} \DecValTok{50} \CommentTok{\# Altera a variável \textquotesingle{}total\textquotesingle{} que já existe lá fora.}
\NormalTok{\}}

\CommentTok{\# Executando:}
\FunctionTok{teste\_local}\NormalTok{()}
\FunctionTok{print}\NormalTok{(total) }\CommentTok{\# Resultado: 100 (não mudou)}
\end{Highlighting}
\end{Shaded}

\begin{verbatim}
[1] 100
\end{verbatim}

\begin{Shaded}
\begin{Highlighting}[]
\FunctionTok{teste\_global}\NormalTok{()}
\FunctionTok{print}\NormalTok{(total) }\CommentTok{\# Resultado: 50 (foi alterada pela superatribuição)}
\end{Highlighting}
\end{Shaded}

\begin{verbatim}
[1] 50
\end{verbatim}

\textbf{Operações Aritméticas} * \texttt{+}, \texttt{-}, \texttt{*},
\texttt{/}: Soma, Subtração, Multiplicação e Divisão.

\begin{Shaded}
\begin{Highlighting}[]
\DecValTok{2} \SpecialCharTok{+} \DecValTok{4}  \CommentTok{\# Soma}
\end{Highlighting}
\end{Shaded}

\begin{verbatim}
[1] 6
\end{verbatim}

\begin{Shaded}
\begin{Highlighting}[]
\DecValTok{2} \SpecialCharTok{{-}} \DecValTok{4}  \CommentTok{\# Subtração}
\end{Highlighting}
\end{Shaded}

\begin{verbatim}
[1] -2
\end{verbatim}

\begin{Shaded}
\begin{Highlighting}[]
\DecValTok{2} \SpecialCharTok{*} \DecValTok{4}  \CommentTok{\# Multiplicação}
\end{Highlighting}
\end{Shaded}

\begin{verbatim}
[1] 8
\end{verbatim}

\begin{Shaded}
\begin{Highlighting}[]
\DecValTok{2} \SpecialCharTok{/} \DecValTok{4}  \CommentTok{\# Divisão}
\end{Highlighting}
\end{Shaded}

\begin{verbatim}
[1] 0.5
\end{verbatim}

\begin{itemize}
\tightlist
\item
  \texttt{\^{}} ou \texttt{**}: Exponenciação (potência).
\end{itemize}

\begin{tcolorbox}[enhanced jigsaw, left=2mm, toptitle=1mm, colback=white, colframe=quarto-callout-note-color-frame, colbacktitle=quarto-callout-note-color!10!white, opacityback=0, rightrule=.15mm, bottomtitle=1mm, arc=.35mm, title=\textcolor{quarto-callout-note-color}{\faInfo}\hspace{0.5em}{Nota}, titlerule=0mm, bottomrule=.15mm, leftrule=.75mm, coltitle=black, toprule=.15mm, breakable, opacitybacktitle=0.6]

\texttt{\^{}} é o mais comum no \texttt{R}, mas \texttt{**} também
funciona).

\end{tcolorbox}

\begin{Shaded}
\begin{Highlighting}[]
\DecValTok{2} \SpecialCharTok{\^{}} \DecValTok{3}  \CommentTok{\# 2 elevado a 3 = 2x2x2= 8}
\end{Highlighting}
\end{Shaded}

\begin{verbatim}
[1] 8
\end{verbatim}

\begin{itemize}
\tightlist
\item
  \texttt{\%\%}: Módulo (Resto da divisão). Útil para verificar paridade
  ou ciclos.
\end{itemize}

\begin{Shaded}
\begin{Highlighting}[]
\DecValTok{5} \SpecialCharTok{\%\%} \DecValTok{2}  \CommentTok{\# Retorna 1 (pois 5 dividido por 2 dá 2 e sobra 1)}
\end{Highlighting}
\end{Shaded}

\begin{verbatim}
[1] 1
\end{verbatim}

\begin{itemize}
\tightlist
\item
  \texttt{\%/\%}: Divisão inteira (Quociente). Descarta a parte decimal.
\end{itemize}

\begin{Shaded}
\begin{Highlighting}[]
\DecValTok{5} \SpecialCharTok{\%/\%} \DecValTok{2} \CommentTok{\# Retorna 2 (a parte inteira da divisão)}
\end{Highlighting}
\end{Shaded}

\begin{verbatim}
[1] 2
\end{verbatim}

\begin{tcolorbox}[enhanced jigsaw, left=2mm, toptitle=1mm, colback=white, colframe=quarto-callout-important-color-frame, colbacktitle=quarto-callout-important-color!10!white, opacityback=0, rightrule=.15mm, bottomtitle=1mm, arc=.35mm, title=\textcolor{quarto-callout-important-color}{\faExclamation}\hspace{0.5em}{Execução de uma operação}, titlerule=0mm, bottomrule=.15mm, leftrule=.75mm, coltitle=black, toprule=.15mm, breakable, opacitybacktitle=0.6]

Se você executar uma operação sem atribuí-la a um objeto (usando
\texttt{\textless{}-} ou \texttt{=}), o \texttt{R} apenas imprime o
resultado no console e o descarta imediatamente. Ele não fica salvo na
memória.

\end{tcolorbox}

\begin{itemize}
\item
  \emph{Exemplo:} Ao rodar apenas \texttt{2\ +\ 4}, você vê \texttt{6}
  na tela, mas não consegue usar esse \texttt{6} em contas futuras.
  \textbf{Solução:} use sempre
  \texttt{nome\_do\_objeto\_da\_sua\_preferência\ \textless{}-\ 2\ +\ 4}
\item
  \textbf{Sobrescrita de Variáveis:} O R não guarda o histórico de uma
  variável. Se você atribuir um novo valor a um objeto que já existe, o
  valor antigo é \textbf{apagado} e substituído pelo novo (a última
  operação prevalece).
\end{itemize}

\begin{Shaded}
\begin{Highlighting}[]
\NormalTok{b }\OtherTok{\textless{}{-}} \DecValTok{10}      \CommentTok{\# \textquotesingle{}b\textquotesingle{} vale 10}
\NormalTok{b }\OtherTok{\textless{}{-}} \DecValTok{2} \SpecialCharTok{+} \DecValTok{4}   \CommentTok{\# Agora \textquotesingle{}b\textquotesingle{} vale 6. O 10 foi perdido.}
\end{Highlighting}
\end{Shaded}

\textbf{Lógica e Comparação}

Resultam sempre em \texttt{TRUE} ou \texttt{FALSE}. * \texttt{==},
\texttt{!=}: Igualdade exata e Diferença.

\begin{Shaded}
\begin{Highlighting}[]
\DecValTok{10} \SpecialCharTok{==} \DecValTok{10}  \CommentTok{\# TRUE (10 é igual a 10)}
\end{Highlighting}
\end{Shaded}

\begin{verbatim}
[1] TRUE
\end{verbatim}

\begin{Shaded}
\begin{Highlighting}[]
\DecValTok{10} \SpecialCharTok{!=} \DecValTok{10}  \CommentTok{\# FALSE (10 não é diferente de 10)}
\end{Highlighting}
\end{Shaded}

\begin{verbatim}
[1] FALSE
\end{verbatim}

\begin{Shaded}
\begin{Highlighting}[]
\StringTok{"R"} \SpecialCharTok{==} \StringTok{"Python"} \CommentTok{\# FALSE}
\end{Highlighting}
\end{Shaded}

\begin{verbatim}
[1] FALSE
\end{verbatim}

\begin{itemize}
\tightlist
\item
  \texttt{\textless{}}, \texttt{\textgreater{}}, \texttt{\textless{}=},
  \texttt{\textgreater{}=}: Menor, Maior, Menor ou igual, Maior ou
  igual.
\end{itemize}

\begin{Shaded}
\begin{Highlighting}[]
\DecValTok{5} \SpecialCharTok{\textgreater{}} \DecValTok{3}    \CommentTok{\# TRUE}
\end{Highlighting}
\end{Shaded}

\begin{verbatim}
[1] TRUE
\end{verbatim}

\begin{Shaded}
\begin{Highlighting}[]
\DecValTok{5} \SpecialCharTok{\textless{}=} \DecValTok{2}   \CommentTok{\# FALSE}
\end{Highlighting}
\end{Shaded}

\begin{verbatim}
[1] FALSE
\end{verbatim}

\begin{Shaded}
\begin{Highlighting}[]
\NormalTok{x }\OtherTok{\textless{}{-}} \DecValTok{10}
\NormalTok{x }\SpecialCharTok{\textgreater{}=} \DecValTok{10}  \CommentTok{\# TRUE}
\end{Highlighting}
\end{Shaded}

\begin{verbatim}
[1] TRUE
\end{verbatim}

\begin{itemize}
\tightlist
\item
  \texttt{!}: Negação (NÃO). Inverte o valor lógico (\texttt{!TRUE} é
  \texttt{FALSE}).
\end{itemize}

\begin{Shaded}
\begin{Highlighting}[]
\SpecialCharTok{!}\ConstantTok{TRUE}         \CommentTok{\# Retorna FALSE}
\end{Highlighting}
\end{Shaded}

\begin{verbatim}
[1] FALSE
\end{verbatim}

\begin{Shaded}
\begin{Highlighting}[]
\SpecialCharTok{!}\NormalTok{(}\DecValTok{5} \SpecialCharTok{\textgreater{}} \DecValTok{3}\NormalTok{)      }\CommentTok{\# 5 \textgreater{} 3 é TRUE, logo a negação retorna FALSE}
\end{Highlighting}
\end{Shaded}

\begin{verbatim}
[1] FALSE
\end{verbatim}

\begin{Shaded}
\begin{Highlighting}[]
\FunctionTok{is.na}\NormalTok{(}\DecValTok{4}\NormalTok{)      }\CommentTok{\# Verifica se é Nulo (FALSE)}
\end{Highlighting}
\end{Shaded}

\begin{verbatim}
[1] FALSE
\end{verbatim}

\begin{Shaded}
\begin{Highlighting}[]
\SpecialCharTok{!}\FunctionTok{is.na}\NormalTok{(}\DecValTok{4}\NormalTok{)     }\CommentTok{\# "Não é nulo" (TRUE)}
\end{Highlighting}
\end{Shaded}

\begin{verbatim}
[1] TRUE
\end{verbatim}

\begin{itemize}
\tightlist
\item
  \texttt{\&}, \texttt{\textbar{}}: ``E'' / ``OU'' vetorizados. Comparam
  elemento por elemento de dois vetores.
\end{itemize}

\begin{Shaded}
\begin{Highlighting}[]
\NormalTok{v1 }\OtherTok{\textless{}{-}} \FunctionTok{c}\NormalTok{(}\ConstantTok{TRUE}\NormalTok{, }\ConstantTok{TRUE}\NormalTok{, }\ConstantTok{FALSE}\NormalTok{)}
\NormalTok{v2 }\OtherTok{\textless{}{-}} \FunctionTok{c}\NormalTok{(}\ConstantTok{FALSE}\NormalTok{, }\ConstantTok{TRUE}\NormalTok{, }\ConstantTok{FALSE}\NormalTok{)}

\NormalTok{v1 }\SpecialCharTok{\&}\NormalTok{ v2  }\CommentTok{\# Retorna: FALSE TRUE FALSE (Apenas a 2ª posição tem TRUE nos dois)}
\end{Highlighting}
\end{Shaded}

\begin{verbatim}
[1] FALSE  TRUE FALSE
\end{verbatim}

\begin{Shaded}
\begin{Highlighting}[]
\NormalTok{v1 }\SpecialCharTok{|}\NormalTok{ v2  }\CommentTok{\# Retorna: TRUE TRUE FALSE (Basta um ser TRUE)}
\end{Highlighting}
\end{Shaded}

\begin{verbatim}
[1]  TRUE  TRUE FALSE
\end{verbatim}

\begin{itemize}
\tightlist
\item
  \texttt{\&\&}, \texttt{\textbar{}\textbar{}}: ``E'' / ``OU'' de
  controle. Avaliam apenas o primeiro elemento.
\end{itemize}

\begin{Shaded}
\begin{Highlighting}[]
\NormalTok{x }\OtherTok{\textless{}{-}} \DecValTok{10}
\NormalTok{y }\OtherTok{\textless{}{-}} \DecValTok{3}
\NormalTok{divisao\_inteira }\OtherTok{\textless{}{-}}\NormalTok{ x }\SpecialCharTok{\%/\%}\NormalTok{ y}
\NormalTok{resto }\OtherTok{\textless{}{-}}\NormalTok{ x }\SpecialCharTok{\%\%}\NormalTok{ y}
\NormalTok{logica }\OtherTok{\textless{}{-}}\NormalTok{ (x }\SpecialCharTok{\textgreater{}} \DecValTok{5}\NormalTok{) }\SpecialCharTok{\&}\NormalTok{ (y }\SpecialCharTok{\textless{}} \DecValTok{5}\NormalTok{) }\CommentTok{\# TRUE E TRUE = TRUE}
\end{Highlighting}
\end{Shaded}

Usados exclusivamente em estruturas de controle como \texttt{if}.

\begin{Shaded}
\begin{Highlighting}[]
\NormalTok{tem\_saldo }\OtherTok{\textless{}{-}} \ConstantTok{TRUE}
\NormalTok{conta\_ativa }\OtherTok{\textless{}{-}} \ConstantTok{TRUE}

\ControlFlowTok{if}\NormalTok{ (tem\_saldo }\SpecialCharTok{\&\&}\NormalTok{ conta\_ativa) \{}
  \FunctionTok{print}\NormalTok{(}\StringTok{"Compra aprovada"}\NormalTok{)}
\NormalTok{\}}
\end{Highlighting}
\end{Shaded}

\begin{verbatim}
[1] "Compra aprovada"
\end{verbatim}

\subsection{Vetores}\label{vetores}

Não existem escalares (números soltos) no R. Um número sozinho é, na
verdade, um vetor de comprimento 1. Saiba mais sobre vetores clicando
\href{https://pt.wikipedia.org/wiki/Vetor_(matem\%C3\%A1tica)}{aqui}

\textbf{Criação}

\begin{itemize}
\tightlist
\item
  \texttt{c(...)}: Função ``combine'' ou ``concatenate''. Junta
  elementos num mesmo vetor.
\end{itemize}

\begin{Shaded}
\begin{Highlighting}[]
\NormalTok{números }\OtherTok{\textless{}{-}} \FunctionTok{c}\NormalTok{(}\DecValTok{10}\NormalTok{, }\DecValTok{20}\NormalTok{, }\DecValTok{30}\NormalTok{) }\CommentTok{\# Vetor numérico}

\NormalTok{nomes }\OtherTok{\textless{}{-}} \FunctionTok{c}\NormalTok{(}\StringTok{"Ana"}\NormalTok{, }\StringTok{"Beto"}\NormalTok{, }\StringTok{"Carla"}\NormalTok{) }\CommentTok{\# Vetor de texto}

\NormalTok{misturado }\OtherTok{\textless{}{-}} \FunctionTok{c}\NormalTok{(}\DecValTok{10}\NormalTok{, }\StringTok{"Ana"}\NormalTok{) }\CommentTok{\# Resultado: "10", "Ana" (Tudo vira texto)}
\end{Highlighting}
\end{Shaded}

\begin{tcolorbox}[enhanced jigsaw, left=2mm, toptitle=1mm, colback=white, colframe=quarto-callout-note-color-frame, colbacktitle=quarto-callout-note-color!10!white, opacityback=0, rightrule=.15mm, bottomtitle=1mm, arc=.35mm, title=\textcolor{quarto-callout-note-color}{\faInfo}\hspace{0.5em}{Nota}, titlerule=0mm, bottomrule=.15mm, leftrule=.75mm, coltitle=black, toprule=.15mm, breakable, opacitybacktitle=0.6]

O \texttt{c} é Minúsculo

\end{tcolorbox}

\begin{itemize}
\tightlist
\item
  \texttt{from:to}: Gera uma sequência de inteiros (ex: \texttt{1:10}).
\end{itemize}

\begin{Shaded}
\begin{Highlighting}[]
\DecValTok{1}\SpecialCharTok{:}\DecValTok{5}   \CommentTok{\# Retorna: 1 2 3 4 5}
\end{Highlighting}
\end{Shaded}

\begin{verbatim}
[1] 1 2 3 4 5
\end{verbatim}

\begin{Shaded}
\begin{Highlighting}[]
\DecValTok{5}\SpecialCharTok{:}\DecValTok{1}   \CommentTok{\# Retorna: 5 4 3 2 1}
\end{Highlighting}
\end{Shaded}

\begin{verbatim}
[1] 5 4 3 2 1
\end{verbatim}

\begin{itemize}
\tightlist
\item
  \texttt{seq(from,\ to,\ by,\ length)}: Gera sequências complexas.
  Permite definir o passo (\texttt{by}) ou o tamanho final desejado
  (\texttt{length}).
\end{itemize}

\begin{Shaded}
\begin{Highlighting}[]
\CommentTok{\# Sequência de 0 a 10, pulando de 2 em 2}
\FunctionTok{seq}\NormalTok{(}\AttributeTok{from =} \DecValTok{0}\NormalTok{, }\AttributeTok{to =} \DecValTok{10}\NormalTok{, }\AttributeTok{by =} \DecValTok{2}\NormalTok{) }\CommentTok{\# Resultado: 0 2 4 6 8 10}
\end{Highlighting}
\end{Shaded}

\begin{verbatim}
[1]  0  2  4  6  8 10
\end{verbatim}

\begin{Shaded}
\begin{Highlighting}[]
\CommentTok{\# Quero exatos 5 números entre 0 e 10}
\FunctionTok{seq}\NormalTok{(}\AttributeTok{from =} \DecValTok{0}\NormalTok{, }\AttributeTok{to =} \DecValTok{10}\NormalTok{, }\AttributeTok{length.out =} \DecValTok{5}\NormalTok{)}\CommentTok{\# Resultado: 0.0  2.5  5.0  7.5  10.0}
\end{Highlighting}
\end{Shaded}

\begin{verbatim}
[1]  0.0  2.5  5.0  7.5 10.0
\end{verbatim}

\begin{itemize}
\tightlist
\item
  \texttt{rep(x,\ times,\ each)}: Replica elementos. \texttt{times}
  repete o vetor todo; \texttt{each} repete cada elemento
  individualmente.
\end{itemize}

\begin{Shaded}
\begin{Highlighting}[]
\NormalTok{vetor }\OtherTok{\textless{}{-}} \FunctionTok{c}\NormalTok{(}\DecValTok{1}\NormalTok{, }\DecValTok{2}\NormalTok{)}

\FunctionTok{rep}\NormalTok{(vetor, }\AttributeTok{times =} \DecValTok{3}\NormalTok{) }\CommentTok{\# Resulta em: 1, 2, 1, 2, 1, 2}
\end{Highlighting}
\end{Shaded}

\begin{verbatim}
[1] 1 2 1 2 1 2
\end{verbatim}

\begin{Shaded}
\begin{Highlighting}[]
\FunctionTok{rep}\NormalTok{(vetor, }\AttributeTok{each =} \DecValTok{3}\NormalTok{) }\CommentTok{\# Resulta em:  1, 1, 1, 2, 2, 2}
\end{Highlighting}
\end{Shaded}

\begin{verbatim}
[1] 1 1 1 2 2 2
\end{verbatim}

\begin{itemize}
\tightlist
\item
  \texttt{\%in\%:\ Verifica\ se\ um\ valor\ está\ dentro\ de\ um\ vetor\ e\ retorna}TRUE\texttt{ou}FALSE`
\end{itemize}

\begin{Shaded}
\begin{Highlighting}[]
\DecValTok{1} \SpecialCharTok{\%in\%}\NormalTok{ vetor  }\CommentTok{\# 1 está dentro do objeto vetor?}
\end{Highlighting}
\end{Shaded}

\begin{verbatim}
[1] TRUE
\end{verbatim}

\begin{Shaded}
\begin{Highlighting}[]
\FunctionTok{c}\NormalTok{()}
\end{Highlighting}
\end{Shaded}

\begin{verbatim}
NULL
\end{verbatim}

\textbf{A Regra da reciclagem}

O superpoder do \texttt{R} é a vetorização. Se você tentar operar dois
vetores de tamanhos diferentes, o \texttt{R} ``recicla'' (repete) o
vetor menor até que ele tenha o tamanho do maior.

\begin{Shaded}
\begin{Highlighting}[]
\NormalTok{valores }\OtherTok{\textless{}{-}} \FunctionTok{c}\NormalTok{(}\DecValTok{10}\NormalTok{, }\DecValTok{20}\NormalTok{, }\DecValTok{30}\NormalTok{, }\DecValTok{40}\NormalTok{)}
\NormalTok{pesos }\OtherTok{\textless{}{-}} \FunctionTok{c}\NormalTok{(}\DecValTok{1}\NormalTok{, }\DecValTok{2}\NormalTok{) }

\NormalTok{resultado }\OtherTok{\textless{}{-}}\NormalTok{ valores }\SpecialCharTok{*}\NormalTok{ pesos }\CommentTok{\# O que acontece internamente: 10*1, 20*2, 30*1, 40*2}
\FunctionTok{print}\NormalTok{(resultado)}
\end{Highlighting}
\end{Shaded}

\begin{verbatim}
[1] 10 40 30 80
\end{verbatim}

\subsection{Tipos e/ou valores especiais e
coerção}\label{tipos-eou-valores-especiais-e-coeruxe7uxe3o}

O \texttt{R} possui termos específicos para lidar com exceções
matemáticas e dados faltantes. É crucial saber diferenciá-los.

\begin{itemize}
\tightlist
\item
  \texttt{NA}: \emph{Not Available}. Representa um dado ausente ou
  desconhecido. O \texttt{NA} é ``contagioso''. A maioria das operações
  matemáticas envolvendo um \texttt{NA} resultará em \texttt{NA} (pois
  não se pode somar algo a um valor desconhecido).
\end{itemize}

\begin{tcolorbox}[enhanced jigsaw, left=2mm, toptitle=1mm, colback=white, colframe=quarto-callout-important-color-frame, colbacktitle=quarto-callout-important-color!10!white, opacityback=0, rightrule=.15mm, bottomtitle=1mm, arc=.35mm, title=\textcolor{quarto-callout-important-color}{\faExclamation}\hspace{0.5em}{Importante}, titlerule=0mm, bottomrule=.15mm, leftrule=.75mm, coltitle=black, toprule=.15mm, breakable, opacitybacktitle=0.6]

Nunca use \texttt{==} para testar se algo é NA (ex: \texttt{x\ ==\ NA}
retorna \texttt{NA}, e não \texttt{TRUE} ou \texttt{FALSE}). Use sempre
a função \texttt{is.na()}.

\end{tcolorbox}

\begin{Shaded}
\begin{Highlighting}[]
\NormalTok{vetor }\OtherTok{\textless{}{-}} \FunctionTok{c}\NormalTok{(}\DecValTok{1}\NormalTok{, }\DecValTok{2}\NormalTok{, }\ConstantTok{NA}\NormalTok{, }\DecValTok{4}\NormalTok{)}
\FunctionTok{mean}\NormalTok{(vetor)      }\CommentTok{\# Retorna NA (a média de dados desconhecidos é desconhecida)}
\end{Highlighting}
\end{Shaded}

\begin{verbatim}
[1] NA
\end{verbatim}

\begin{Shaded}
\begin{Highlighting}[]
\FunctionTok{is.na}\NormalTok{(vetor)     }\CommentTok{\# Retorna: FALSE FALSE TRUE FALSE}
\end{Highlighting}
\end{Shaded}

\begin{verbatim}
[1] FALSE FALSE  TRUE FALSE
\end{verbatim}

\begin{itemize}
\tightlist
\item
  \texttt{NaN}: \emph{Not a Number}. Resultado de uma indeterminação
  matemática. Ocorre quando o cálculo é impossível de ser definido
  numericamente. Resultado de indeterminações matemáticas (ex:
  \texttt{0/0}).
\end{itemize}

\begin{Shaded}
\begin{Highlighting}[]
\DecValTok{0} \SpecialCharTok{/} \DecValTok{0}  \CommentTok{\# Retorna NaN}
\end{Highlighting}
\end{Shaded}

\begin{verbatim}
[1] NaN
\end{verbatim}

\begin{itemize}
\tightlist
\item
  \texttt{Inf}: Infinito (ex: \texttt{1/0}).
\end{itemize}

\begin{Shaded}
\begin{Highlighting}[]
\DecValTok{10} \SpecialCharTok{/} \DecValTok{0}   \CommentTok{\# Retorna Inf}
\end{Highlighting}
\end{Shaded}

\begin{verbatim}
[1] Inf
\end{verbatim}

\begin{Shaded}
\begin{Highlighting}[]
\SpecialCharTok{{-}}\DecValTok{5} \SpecialCharTok{/} \DecValTok{0}   \CommentTok{\# Retorna {-}Inf}
\end{Highlighting}
\end{Shaded}

\begin{verbatim}
[1] -Inf
\end{verbatim}

\begin{itemize}
\tightlist
\item
  \texttt{NULL}: Vazio. Representa a ausência total de conteúdo.
  Diferente do \texttt{NA} (que é um ``espaço reservado para um dado que
  falta''), o \texttt{NULL} significa que a estrutura ou vetor nem
  sequer existe naquele ponto. É muito usado para apagar elementos de
  listas.
\end{itemize}

\begin{Shaded}
\begin{Highlighting}[]
\NormalTok{x }\OtherTok{\textless{}{-}} \ConstantTok{NULL}  \CommentTok{\# O objeto x existe, mas é vazio (tamanho 0).}
\FunctionTok{c}\NormalTok{(}\DecValTok{1}\NormalTok{, }\DecValTok{2}\NormalTok{, }\ConstantTok{NULL}\NormalTok{, }\DecValTok{3}\NormalTok{) }\CommentTok{\# Resultado: 1, 2, 3 (O NULL é ignorado na concatenação)}
\end{Highlighting}
\end{Shaded}

\begin{verbatim}
[1] 1 2 3
\end{verbatim}

\subsection{Funções de conversão e
verificação}\label{funuxe7uxf5es-de-conversuxe3o-e-verificauxe7uxe3o}

No R, é fundamental saber qual é o tipo do objeto (classe) com o qual
você está lidando. Estas famílias de funções permitem testar ou alterar
esses tipos.

\textbf{Funções \texttt{as...} (Conversão / Coerção Explícita)}

Estas funções tentam forçar a transformação de um objeto de um tipo para
outro.

\begin{itemize}
\tightlist
\item
  \texttt{as.numeric()}, \texttt{as.character()}, \texttt{as.logical()},
  \texttt{as.factor()}: Forçam a conversão de um tipo para outro.
\end{itemize}

\begin{Shaded}
\begin{Highlighting}[]
\CommentTok{\# Convertendo Número para Texto}
\FunctionTok{as.character}\NormalTok{(}\DecValTok{123}\NormalTok{) }\CommentTok{\# Resultado: "123" (Note as aspas)}
\end{Highlighting}
\end{Shaded}

\begin{verbatim}
[1] "123"
\end{verbatim}

\begin{Shaded}
\begin{Highlighting}[]
\CommentTok{\# Convertendo Texto para Número}
\FunctionTok{as.numeric}\NormalTok{(}\StringTok{"10.5"}\NormalTok{) }\CommentTok{\# Resultado: 10.5}
\end{Highlighting}
\end{Shaded}

\begin{verbatim}
[1] 10.5
\end{verbatim}

\begin{Shaded}
\begin{Highlighting}[]
\FunctionTok{as.numeric}\NormalTok{(}\StringTok{"Bola"}\NormalTok{) }\CommentTok{\# Resultado: NA ("NAs introduzidos por coerção")}
\end{Highlighting}
\end{Shaded}

\begin{verbatim}
[1] NA
\end{verbatim}

\begin{tcolorbox}[enhanced jigsaw, left=2mm, toptitle=1mm, colback=white, colframe=quarto-callout-tip-color-frame, colbacktitle=quarto-callout-tip-color!10!white, opacityback=0, rightrule=.15mm, bottomtitle=1mm, arc=.35mm, title=\textcolor{quarto-callout-tip-color}{\faLightbulb}\hspace{0.5em}{Dica}, titlerule=0mm, bottomrule=.15mm, leftrule=.75mm, coltitle=black, toprule=.15mm, breakable, opacitybacktitle=0.6]

Uma técnica muito comum é converter \texttt{TRUE} e \texttt{FALSE} em
números para fazer contagens. \texttt{TRUE} vira 1 e \texttt{FALSE} vira
0.

\end{tcolorbox}

\begin{Shaded}
\begin{Highlighting}[]
\FunctionTok{as.numeric}\NormalTok{(}\ConstantTok{TRUE}\NormalTok{)  }\CommentTok{\# Retorna 1}
\end{Highlighting}
\end{Shaded}

\begin{verbatim}
[1] 1
\end{verbatim}

\begin{Shaded}
\begin{Highlighting}[]
\FunctionTok{as.numeric}\NormalTok{(}\ConstantTok{FALSE}\NormalTok{) }\CommentTok{\# Retorna 0}
\end{Highlighting}
\end{Shaded}

\begin{verbatim}
[1] 0
\end{verbatim}

\textbf{Funções \texttt{is...} (Verificação / Teste lógico)}

Estas funções fazem uma pergunta ao objeto e retornam sempre
\texttt{TRUE} ou \texttt{FALSE}. São ideais para usar dentro de
condicionais (\texttt{if}).

\begin{itemize}
\tightlist
\item
  \texttt{is.numeric()}, \texttt{is.character()}, etc.: Verificam se o
  objeto é do tipo especificado.
\end{itemize}

\begin{Shaded}
\begin{Highlighting}[]
\NormalTok{x }\OtherTok{\textless{}{-}} \DecValTok{10}
\NormalTok{y }\OtherTok{\textless{}{-}} \StringTok{"10"}

\FunctionTok{is.numeric}\NormalTok{(x)    }\CommentTok{\# TRUE (É um número)}
\end{Highlighting}
\end{Shaded}

\begin{verbatim}
[1] TRUE
\end{verbatim}

\begin{Shaded}
\begin{Highlighting}[]
\FunctionTok{is.numeric}\NormalTok{(y)    }\CommentTok{\# FALSE (É um texto, apesar de conter dígitos)}
\end{Highlighting}
\end{Shaded}

\begin{verbatim}
[1] FALSE
\end{verbatim}

\begin{Shaded}
\begin{Highlighting}[]
\FunctionTok{is.character}\NormalTok{(y)  }\CommentTok{\# TRUE}
\end{Highlighting}
\end{Shaded}

\begin{verbatim}
[1] TRUE
\end{verbatim}

\section{Gerenciamento de diretórios e
objetos}\label{gerenciamento-de-diretuxf3rios-e-objetos}

Saber onde o \texttt{R} está salvando seus arquivos (diretório de
trabalho) e como limpar a memória é essencial para a organização.

\textbf{Diretório de trabalho}

É a pasta no seu computador onde o \texttt{R} irá ler e salvar arquivos
(como \texttt{.csv} ou \texttt{.xlsx}) por padrão.

\begin{itemize}
\tightlist
\item
  \textbf{\texttt{getwd()} (Get Working Directory):} Descobre em qual
  pasta você está trabalhando agora.
\end{itemize}

\begin{Shaded}
\begin{Highlighting}[]
\FunctionTok{getwd}\NormalTok{() }\CommentTok{\# Exemplo de saída: /home/almonha/curso{-}de{-}verao{-}notas{-}de{-}aulas"}
\end{Highlighting}
\end{Shaded}

\begin{itemize}
\tightlist
\item
  \textbf{\texttt{setwd()} (Set Working Directory):} Muda o diretório de
  trabalho para outra pasta.
\end{itemize}

\begin{Shaded}
\begin{Highlighting}[]
\FunctionTok{setwd}\NormalTok{(}\StringTok{"C:/Users/Alex/Curso\_verao"}\NormalTok{)}
\end{Highlighting}
\end{Shaded}

\begin{tcolorbox}[enhanced jigsaw, left=2mm, toptitle=1mm, colback=white, colframe=quarto-callout-important-color-frame, colbacktitle=quarto-callout-important-color!10!white, opacityback=0, rightrule=.15mm, bottomtitle=1mm, arc=.35mm, title=\textcolor{quarto-callout-important-color}{\faExclamation}\hspace{0.5em}{Use setwd() com cautela}, titlerule=0mm, bottomrule=.15mm, leftrule=.75mm, coltitle=black, toprule=.15mm, breakable, opacitybacktitle=0.6]

Se usar\texttt{setwd("C:/Users/SeuNome/...")}, seu código não será
reprodutível. Se você enviar esse script para outra pessoa, o código
quebrará, pois o computador dele não tem a pasta ``SeuNome'', isto é,
usam diretórios diferentes.

\end{tcolorbox}

A solução profissional é trabalhar com \textbf{R Projects
(\texttt{.Rproj})}.

\textbf{Criação de projeto}

\begin{enumerate}
\def\labelenumi{\arabic{enumi}.}
\tightlist
\item
  No \texttt{RStudio}:
  \texttt{File\ \textgreater{}\ New\ Project\ \textgreater{}\ New\ Directory}.
\item
  Isso cria uma raiz autocontida.
\item
  Utilize o pacote \texttt{here} para referenciar arquivos. Ele encontra
  automaticamente a raiz do projeto, independente de qual subpasta você
  esteja. Isto é, se rodar o código:
\end{enumerate}

\begin{Shaded}
\begin{Highlighting}[]
\NormalTok{dados }\OtherTok{\textless{}{-}} \FunctionTok{read.csv}\NormalTok{(}\FunctionTok{here}\NormalTok{(}\StringTok{"dados"}\NormalTok{, }\StringTok{"tabela.csv"}\NormalTok{))}
\end{Highlighting}
\end{Shaded}

, o computador entende: \emph{Na raiz do projeto, entre na pasta
\texttt{dados}, leia \texttt{tabela.csv}}, ou seja, esta é a forma ideal
e universal de ler arquivos.

\begin{tcolorbox}[enhanced jigsaw, left=2mm, toptitle=1mm, colback=white, colframe=quarto-callout-warning-color-frame, colbacktitle=quarto-callout-warning-color!10!white, opacityback=0, rightrule=.15mm, bottomtitle=1mm, arc=.35mm, title=\textcolor{quarto-callout-warning-color}{\faExclamationTriangle}\hspace{0.5em}{Cuidado com barras}, titlerule=0mm, bottomrule=.15mm, leftrule=.75mm, coltitle=black, toprule=.15mm, breakable, opacitybacktitle=0.6]

No Windows, o caminho copiado usa contrabarra
(\texttt{\textbackslash{}}), mas no R você deve usar barra normal
(\texttt{/}) ou barra dupla (\texttt{\textbackslash{}\textbackslash{}}).

\end{tcolorbox}

\begin{itemize}
\tightlist
\item
  \textbf{Forma correta} (Barra normal ou dupla):
\end{itemize}

\begin{Shaded}
\begin{Highlighting}[]
\FunctionTok{setwd}\NormalTok{(}\StringTok{"C:/Users/SeuNome/Projetos/Analise\_Dados"}\NormalTok{)}
\end{Highlighting}
\end{Shaded}

\begin{itemize}
\tightlist
\item
  \textbf{Forma incorreta} (Vai dar erro no Windows)
\end{itemize}

\begin{Shaded}
\begin{Highlighting}[]
\FunctionTok{setwd}\NormalTok{(}\StringTok{"C:\textbackslash{}Users\textbackslash{}SeuNome\textbackslash{}Projetos\textbackslash{}Analise\_Dados"}\NormalTok{)}
\end{Highlighting}
\end{Shaded}

\textbf{Gerenciando objetos na memória}

Conforme você cria variáveis (\texttt{x}, \texttt{y}, \texttt{dados}),
elas ocupam a memória RAM (o \emph{Environment}).

\begin{itemize}
\tightlist
\item
  \textbf{\texttt{ls()} (Listar):} Mostra os nomes de todos os objetos
  criados no ambiente atual.
\end{itemize}

\begin{Shaded}
\begin{Highlighting}[]
\FunctionTok{ls}\NormalTok{()}
\end{Highlighting}
\end{Shaded}

\begin{itemize}
\tightlist
\item
  \textbf{\texttt{rm()} (Remover):} Apaga objetos da memória para
  liberar espaço ou evitar confusão.
\end{itemize}

\begin{Shaded}
\begin{Highlighting}[]
\FunctionTok{rm}\NormalTok{(nome\_do\_objeto\_a\_remover)}
\end{Highlighting}
\end{Shaded}

\begin{itemize}
\tightlist
\item
  \textbf{Limpeza total (O comando ``Vassoura''):} Para apagar tudo o
  que está na memória e começar do zero (muito comum no início de
  scripts) use \texttt{rm(list\ =\ ls())}
\end{itemize}

\begin{Shaded}
\begin{Highlighting}[]
\FunctionTok{rm}\NormalTok{(}\AttributeTok{list =} \FunctionTok{ls}\NormalTok{())}
\end{Highlighting}
\end{Shaded}

\begin{itemize}
\tightlist
\item
  \texttt{gc()}: (\emph{Garbage Collection}) Força o sistema a liberar
  memória RAM que não está mais sendo usada. Essencial ao trabalhar com
  Big Data.
\end{itemize}

\begin{Shaded}
\begin{Highlighting}[]
\FunctionTok{gc}\NormalTok{()}
\end{Highlighting}
\end{Shaded}

\begin{itemize}
\tightlist
\item
  \textbf{Limpeza de de nomes das colunas de objetos:} Nomes de colunas
  com espaços, acentos ou caracteres especiais (Média (kg), Ano/Mês)
  exigem o uso de crases irritantes no código. Limpe-os imediatamente.
\end{itemize}

\begin{Shaded}
\begin{Highlighting}[]
\ControlFlowTok{if}\NormalTok{ (}\SpecialCharTok{!}\FunctionTok{require}\NormalTok{(}\StringTok{"pacman"}\NormalTok{)) }\FunctionTok{install.packages}\NormalTok{(}\StringTok{"pacman"}\NormalTok{)}
\NormalTok{pacman}\SpecialCharTok{::}\FunctionTok{p\_load}\NormalTok{(tidyverse, janitor,flextable)}

\NormalTok{df\_sujo }\OtherTok{\textless{}{-}} \FunctionTok{data.frame}\NormalTok{(}
  \StringTok{\textquotesingle{}Nome do Aluno\textquotesingle{}} \OtherTok{=} \FunctionTok{c}\NormalTok{(}\StringTok{"Ana"}\NormalTok{, }\StringTok{"Beto"}\NormalTok{),}
  \StringTok{\textquotesingle{}NOTA FINAL\textquotesingle{}} \OtherTok{=} \FunctionTok{c}\NormalTok{(}\DecValTok{9}\NormalTok{, }\DecValTok{8}\NormalTok{),}
  \AttributeTok{check.names =} \ConstantTok{FALSE}
\NormalTok{)}

\NormalTok{df\_limpo }\OtherTok{\textless{}{-}}\NormalTok{ df\_sujo }\SpecialCharTok{\%\textgreater{}\%}
\NormalTok{  janitor}\SpecialCharTok{::}\FunctionTok{clean\_names}\NormalTok{() }\CommentTok{\# Transforma tudo em: nome\_do\_aluno, nota\_final}
\FunctionTok{names}\NormalTok{(df\_limpo)}
\end{Highlighting}
\end{Shaded}

\begin{verbatim}
[1] "nome_do_aluno" "nota_final"   
\end{verbatim}

\section{Pacotes: Leitura e instalação}\label{sec-install_pacotes}

O \texttt{R} base vem apenas com as funcionalidades essenciais. Para
fazer análises avançadas, precisamos instalar \emph{pacotes} (conjuntos
de funções extras criadas pela comunidade). Pense no \texttt{R} como um
celular novo: ele vem com funções de fábrica (ligar, agenda), mas para
usar o Instagram ou WhatsApp, você precisa instalar os \emph{Apps}
(Pacotes).

\textbf{Instalação (\texttt{install.packages})}

Você faz isso apenas uma vez por computador (como baixar o App da loja).

\begin{Shaded}
\begin{Highlighting}[]
\FunctionTok{install.packages}\NormalTok{(}\StringTok{"tidyverse"}\NormalTok{)}
\FunctionTok{install.packages}\NormalTok{(}\StringTok{"ggplot2"}\NormalTok{)}
\end{Highlighting}
\end{Shaded}

\begin{tcolorbox}[enhanced jigsaw, left=2mm, toptitle=1mm, colback=white, colframe=quarto-callout-tip-color-frame, colbacktitle=quarto-callout-tip-color!10!white, opacityback=0, rightrule=.15mm, bottomtitle=1mm, arc=.35mm, title=\textcolor{quarto-callout-tip-color}{\faLightbulb}\hspace{0.5em}{Dica}, titlerule=0mm, bottomrule=.15mm, leftrule=.75mm, coltitle=black, toprule=.15mm, breakable, opacitybacktitle=0.6]

Note que o nome do pacote deve estar entre aspas

\end{tcolorbox}

\textbf{Carregamento (\texttt{library(.)} ou \texttt{require(.)})}

Você deve fazer isso toda vez que abrir o \texttt{RStudio} ou iniciar
uma nova sessão (como abrir o App para usar). Note que aqui não precisa
de aspas.

\begin{Shaded}
\begin{Highlighting}[]
\FunctionTok{library}\NormalTok{(tidyverse)}
\end{Highlighting}
\end{Shaded}

\begin{tcolorbox}[enhanced jigsaw, left=2mm, toptitle=1mm, colback=white, colframe=quarto-callout-note-color-frame, colbacktitle=quarto-callout-note-color!10!white, opacityback=0, rightrule=.15mm, bottomtitle=1mm, arc=.35mm, title=\textcolor{quarto-callout-note-color}{\faInfo}\hspace{0.5em}{Nota}, titlerule=0mm, bottomrule=.15mm, leftrule=.75mm, coltitle=black, toprule=.15mm, breakable, opacitybacktitle=0.6]

Coloque todos os \texttt{library()\ ou\ require()} necessários nas
primeiras linhas do seu script.

\end{tcolorbox}

\begin{tcolorbox}[enhanced jigsaw, left=2mm, toptitle=1mm, colback=white, colframe=quarto-callout-tip-color-frame, colbacktitle=quarto-callout-tip-color!10!white, opacityback=0, rightrule=.15mm, bottomtitle=1mm, arc=.35mm, title=\textcolor{quarto-callout-tip-color}{\faLightbulb}\hspace{0.5em}{Use pacman e p\_load}, titlerule=0mm, bottomrule=.15mm, leftrule=.75mm, coltitle=black, toprule=.15mm, breakable, opacitybacktitle=0.6]

Para evitar deixar seu script cheio de \texttt{install.packages(".")} e
\texttt{library(.)} , onde \texttt{.} é pacote, use o código abaixo
comando que permite instalar e carregar os pacotes simultaneamente.

\end{tcolorbox}

\begin{Shaded}
\begin{Highlighting}[]
\CommentTok{\# O comando que segue diz: se não carregar o pacote pacman instale{-}o}
\ControlFlowTok{if}\NormalTok{ (}\SpecialCharTok{!}\FunctionTok{require}\NormalTok{(}\StringTok{"pacman"}\NormalTok{)) }\FunctionTok{install.packages}\NormalTok{(}\StringTok{"pacman"}\NormalTok{)}
\CommentTok{\#O comando abaixo instala os pacotes se não estão instalados e carrega{-}os}
\NormalTok{pacman}\SpecialCharTok{::}\FunctionTok{p\_load}\NormalTok{(tidyverse, flextable)}
\end{Highlighting}
\end{Shaded}

\section{Estruturas de dados}\label{estruturas-de-dados}

Além de vetores, o \texttt{R} organiza dados em estruturas mais
complexas.

\textbf{Indexação e Seleção (Subsetting)}

No R, utilizamos colchetes \texttt{{[}\ {]}} para acessar, extrair ou
modificar pedaços específicos dos dados.

\textbf{Vetores:}

\begin{Shaded}
\begin{Highlighting}[]
\NormalTok{x }\OtherTok{\textless{}{-}} \FunctionTok{c}\NormalTok{(}\DecValTok{10}\NormalTok{, }\DecValTok{20}\NormalTok{, }\DecValTok{30}\NormalTok{, }\DecValTok{40}\NormalTok{, }\DecValTok{50}\NormalTok{)}
\FunctionTok{names}\NormalTok{(x) }\OtherTok{\textless{}{-}} \FunctionTok{c}\NormalTok{(}\StringTok{"Ana"}\NormalTok{, }\StringTok{"Beto"}\NormalTok{, }\StringTok{"Carla"}\NormalTok{, }\StringTok{"Davi"}\NormalTok{, }\StringTok{"Eva"}\NormalTok{)}
\end{Highlighting}
\end{Shaded}

\begin{itemize}
\tightlist
\item
  \texttt{x{[}n{]}}: Seleciona o elemento na posição \texttt{n}.
\end{itemize}

\begin{Shaded}
\begin{Highlighting}[]
\NormalTok{x[}\DecValTok{2}\NormalTok{]  }\CommentTok{\# Retorna 20 (o segundo elemento)}
\end{Highlighting}
\end{Shaded}

\begin{verbatim}
Beto 
  20 
\end{verbatim}

\begin{itemize}
\tightlist
\item
  \texttt{x{[}-n{]}}: Seleciona tudo \emph{exceto} o elemento na posição
  \texttt{n}.
\end{itemize}

\begin{Shaded}
\begin{Highlighting}[]
\NormalTok{x[}\SpecialCharTok{{-}}\DecValTok{2}\NormalTok{] }\CommentTok{\# Retorna: 10, 30, 40, 50 (O 20 foi removido)}
\end{Highlighting}
\end{Shaded}

\begin{verbatim}
  Ana Carla  Davi   Eva 
   10    30    40    50 
\end{verbatim}

\begin{itemize}
\tightlist
\item
  \texttt{x{[}1:n{]}}: Seleciona uma sequência de posições.
\end{itemize}

\begin{Shaded}
\begin{Highlighting}[]
\NormalTok{x[}\DecValTok{2}\SpecialCharTok{:}\DecValTok{4}\NormalTok{] }\CommentTok{\# Retorna: 20, 30, 40 (Da 2ª à 4ª posição)}
\end{Highlighting}
\end{Shaded}

\begin{verbatim}
 Beto Carla  Davi 
   20    30    40 
\end{verbatim}

\begin{itemize}
\tightlist
\item
  \texttt{x{[}c(1,4){]}}: Seleciona posições específicas não
  sequenciais.
\end{itemize}

\begin{Shaded}
\begin{Highlighting}[]
\NormalTok{x[}\FunctionTok{c}\NormalTok{(}\DecValTok{1}\NormalTok{, }\DecValTok{5}\NormalTok{)] }\CommentTok{\# Retorna: 10, 50 (O primeiro e o último)}
\end{Highlighting}
\end{Shaded}

\begin{verbatim}
Ana Eva 
 10  50 
\end{verbatim}

\begin{itemize}
\tightlist
\item
  \texttt{x{[}x\ \textgreater{}\ 3{]}}: Seleção lógica (filtro). Retorna
  elementos que satisfazem a condição (\texttt{TRUE}). Isto é, O
  \texttt{R} testa cada elemento. Onde for \texttt{TRUE}, ele mantém;
  onde for \texttt{FALSE}, ele descarta.
\end{itemize}

\begin{Shaded}
\begin{Highlighting}[]
\NormalTok{x[x }\SpecialCharTok{\textgreater{}} \DecValTok{30}\NormalTok{] }\CommentTok{\# Resultado: 40, 50}
\end{Highlighting}
\end{Shaded}

\begin{verbatim}
Davi  Eva 
  40   50 
\end{verbatim}

\begin{itemize}
\tightlist
\item
  \texttt{x{[}"nome"{]}}: Seleciona pelo nome, se o vetor tiver o
  atributo \texttt{names}.
\end{itemize}

\begin{Shaded}
\begin{Highlighting}[]
\NormalTok{x[}\StringTok{"Carla"}\NormalTok{]      }\CommentTok{\# Retorna 30}
\end{Highlighting}
\end{Shaded}

\begin{verbatim}
Carla 
   30 
\end{verbatim}

\begin{Shaded}
\begin{Highlighting}[]
\NormalTok{x[}\FunctionTok{c}\NormalTok{(}\StringTok{"Ana"}\NormalTok{, }\StringTok{"Eva"}\NormalTok{)] }\CommentTok{\# Retorna 10 e 50}
\end{Highlighting}
\end{Shaded}

\begin{verbatim}
Ana Eva 
 10  50 
\end{verbatim}

\textbf{Listas:} Diferente dos vetores (que só aceitam um tipo de dado),
as listas são a estrutura mais flexível do \texttt{R}. Elas podem conter
qualquer coisa: números, textos, vetores, data frames e até outras
listas.

\begin{Shaded}
\begin{Highlighting}[]
\NormalTok{aluno }\OtherTok{\textless{}{-}} \FunctionTok{list}\NormalTok{(}
  \AttributeTok{nome =} \StringTok{"Mariana"}\NormalTok{,}
  \AttributeTok{notas =} \FunctionTok{c}\NormalTok{(}\FloatTok{9.5}\NormalTok{, }\FloatTok{8.0}\NormalTok{, }\FloatTok{7.5}\NormalTok{),}
  \AttributeTok{ativo =} \ConstantTok{TRUE}
\NormalTok{)}
\end{Highlighting}
\end{Shaded}

\begin{itemize}
\tightlist
\item
  \texttt{lista{[}n{]}}: Retorna uma nova lista contendo apenas o
  elemento \texttt{n}. (Pense: pega a gaveta inteira).
\end{itemize}

\begin{Shaded}
\begin{Highlighting}[]
\NormalTok{x }\OtherTok{\textless{}{-}}\NormalTok{ aluno[}\DecValTok{2}\NormalTok{]}
\CommentTok{\# O que é \textquotesingle{}x\textquotesingle{}? É uma LISTA contendo as notas.}
\FunctionTok{class}\NormalTok{(x) }\CommentTok{\# Resultado: "list"}
\end{Highlighting}
\end{Shaded}

\begin{verbatim}
[1] "list"
\end{verbatim}

\begin{itemize}
\tightlist
\item
  \texttt{lista{[}{[}n{]}{]}}: Extrai o conteúdo do objeto na posição
  \texttt{n}. (\textbf{Pense:} tira o objeto da gaveta).
\end{itemize}

\begin{Shaded}
\begin{Highlighting}[]
\NormalTok{y }\OtherTok{\textless{}{-}}\NormalTok{ aluno[[}\DecValTok{2}\NormalTok{]]}
\CommentTok{\# O que é \textquotesingle{}y\textquotesingle{}? É um VETOR numérico (as notas em si).}
\FunctionTok{class}\NormalTok{(y) }\CommentTok{\# Resultado: "numeric"}
\end{Highlighting}
\end{Shaded}

\begin{verbatim}
[1] "numeric"
\end{verbatim}

\begin{Shaded}
\begin{Highlighting}[]
\FunctionTok{mean}\NormalTok{(y)  }\CommentTok{\# Funciona (média das notas).}
\end{Highlighting}
\end{Shaded}

\begin{verbatim}
[1] 8.333333
\end{verbatim}

\begin{Shaded}
\begin{Highlighting}[]
\CommentTok{\# mean(x) daria erro, pois não se calcula média de uma "lista".}
\end{Highlighting}
\end{Shaded}

\begin{itemize}
\tightlist
\item
  \texttt{lista\$nome}: É a forma mais comum e legível de usar o
  \texttt{{[}{[}...{]}{]}}. Extrai o conteúdo diretamente usando o nome
  atribuído ao elemento.
\end{itemize}

\begin{Shaded}
\begin{Highlighting}[]
\NormalTok{aluno}\SpecialCharTok{$}\NormalTok{nome   }\CommentTok{\# Retorna: "Mariana"}
\end{Highlighting}
\end{Shaded}

\begin{verbatim}
[1] "Mariana"
\end{verbatim}

\begin{Shaded}
\begin{Highlighting}[]
\NormalTok{aluno}\SpecialCharTok{$}\NormalTok{notas  }\CommentTok{\# Retorna: 9.5 8.0 7.5}
\end{Highlighting}
\end{Shaded}

\begin{verbatim}
[1] 9.5 8.0 7.5
\end{verbatim}

\textbf{Matrizes e Data Frames}

Diferente dos vetores, aqui lidamos com duas dimensões. A regra de ouro
no R é sempre: \texttt{{[}\ LINHA\ ,\ COLUNA\ {]}}.

\begin{Shaded}
\begin{Highlighting}[]
\NormalTok{dados }\OtherTok{\textless{}{-}} \FunctionTok{data.frame}\NormalTok{(}
  \AttributeTok{Nome =} \FunctionTok{c}\NormalTok{(}\StringTok{"Ana"}\NormalTok{, }\StringTok{"Beto"}\NormalTok{, }\StringTok{"Carla"}\NormalTok{),}
  \AttributeTok{Idade =} \FunctionTok{c}\NormalTok{(}\DecValTok{22}\NormalTok{, }\DecValTok{30}\NormalTok{, }\DecValTok{25}\NormalTok{),}
  \AttributeTok{Nota =} \FunctionTok{c}\NormalTok{(}\DecValTok{8}\NormalTok{, }\DecValTok{7}\NormalTok{, }\DecValTok{9}\NormalTok{)}
\NormalTok{)}
\end{Highlighting}
\end{Shaded}

\begin{itemize}
\tightlist
\item
  \texttt{df{[}i,\ j{]}}: Elemento na linha \texttt{i}, coluna
  \texttt{j}.
\end{itemize}

\begin{Shaded}
\begin{Highlighting}[]
\NormalTok{dados[}\DecValTok{1}\NormalTok{, }\DecValTok{2}\NormalTok{] }\CommentTok{\# Linha 1 ("Ana"), Coluna 2 ("Idade") {-}\textgreater{} Retorna 22}
\end{Highlighting}
\end{Shaded}

\begin{verbatim}
[1] 22
\end{verbatim}

\begin{itemize}
\tightlist
\item
  \texttt{df{[}i,\ {]}}: Seleciona toda a linha \texttt{i}.
\end{itemize}

\begin{Shaded}
\begin{Highlighting}[]
\NormalTok{dados[}\DecValTok{2}\NormalTok{, ]}\CommentTok{\# Retorna todos os dados do "Beto"}
\end{Highlighting}
\end{Shaded}

\begin{verbatim}
  Nome Idade Nota
2 Beto    30    7
\end{verbatim}

\begin{itemize}
\tightlist
\item
  \texttt{df{[},\ j{]}}: Seleciona toda a coluna \texttt{j}.
\end{itemize}

\begin{Shaded}
\begin{Highlighting}[]
\NormalTok{dados[, }\DecValTok{2}\NormalTok{]}\CommentTok{\# Retorna o vetor: 22 30 25}
\end{Highlighting}
\end{Shaded}

\begin{verbatim}
[1] 22 30 25
\end{verbatim}

\begin{itemize}
\tightlist
\item
  \texttt{df\$coluna}: Seleciona a coluna pelo nome (específico para
  Data Frames e Listas). Note que este não funciona em matrizes.
\end{itemize}

\begin{Shaded}
\begin{Highlighting}[]
\NormalTok{dados}\SpecialCharTok{$}\NormalTok{Nome  }\CommentTok{\#seleciona a coluna "Nome"}
\end{Highlighting}
\end{Shaded}

\textbf{Matrizes e Arrays}

\begin{enumerate}
\def\labelenumi{\arabic{enumi}.}
\tightlist
\item
  \textbf{Matrizes}
\end{enumerate}

Matrizes são vetores com duas dimensões (linhas e colunas). Todos os
dados devem ser do mesmo tipo (ex: tudo numérico).

\begin{itemize}
\tightlist
\item
  \textbf{\texttt{matrix(data,\ nrow,\ ncol)}}: Cria a matriz de
  \texttt{nrow} linhas e \texttt{ncol} colunas. O preenchimento padrão é
  por coluna. Use \texttt{byrow=TRUE} para preencher por linha.
\end{itemize}

\begin{Shaded}
\begin{Highlighting}[]
\NormalTok{M }\OtherTok{\textless{}{-}} \FunctionTok{matrix}\NormalTok{(}\DecValTok{1}\SpecialCharTok{:}\DecValTok{6}\NormalTok{, }\AttributeTok{nrow =} \DecValTok{2}\NormalTok{, }\AttributeTok{ncol =} \DecValTok{3}\NormalTok{, }\AttributeTok{byrow =} \ConstantTok{TRUE}\NormalTok{);M}
\end{Highlighting}
\end{Shaded}

\begin{verbatim}
     [,1] [,2] [,3]
[1,]    1    2    3
[2,]    4    5    6
\end{verbatim}

\begin{itemize}
\tightlist
\item
  \textbf{\texttt{t(x)}}: Transposta (inverte linhas por colunas).
\end{itemize}

\begin{Shaded}
\begin{Highlighting}[]
\NormalTok{M\_t }\OtherTok{\textless{}{-}} \FunctionTok{t}\NormalTok{(M);M\_t}
\end{Highlighting}
\end{Shaded}

\begin{verbatim}
     [,1] [,2]
[1,]    1    4
[2,]    2    5
[3,]    3    6
\end{verbatim}

\begin{itemize}
\tightlist
\item
  \textbf{\texttt{\%*\%}}: O operador para multiplicação matricial real
  (diferente de \texttt{*} que multiplica elemento por elemento).
\end{itemize}

\begin{Shaded}
\begin{Highlighting}[]
\NormalTok{M\_mult }\OtherTok{\textless{}{-}}\NormalTok{ M }\SpecialCharTok{\%*\%}\NormalTok{ M\_t; M\_mult}
\end{Highlighting}
\end{Shaded}

\begin{verbatim}
     [,1] [,2]
[1,]   14   32
[2,]   32   77
\end{verbatim}

\begin{itemize}
\tightlist
\item
  \textbf{\texttt{solve(A,\ b)}}: Resolve sistemas lineares \(Ax = b\).
  Se \texttt{b} for omitido, inverte a matriz \texttt{A}.
\end{itemize}

\begin{Shaded}
\begin{Highlighting}[]
\NormalTok{A }\OtherTok{\textless{}{-}} \FunctionTok{matrix}\NormalTok{(}\FunctionTok{c}\NormalTok{(}\DecValTok{4}\NormalTok{, }\DecValTok{2}\NormalTok{, }\DecValTok{7}\NormalTok{, }\DecValTok{6}\NormalTok{), }\AttributeTok{nrow =} \DecValTok{2}\NormalTok{)}
\NormalTok{A\_inv }\OtherTok{\textless{}{-}} \FunctionTok{solve}\NormalTok{(A) }\CommentTok{\# Inversa de A}
\end{Highlighting}
\end{Shaded}

\begin{Shaded}
\begin{Highlighting}[]
\CommentTok{\#SISTEMA DE EQUAÇÕES: Ax=b}
\CommentTok{\# 3x + 2y = 5}
\CommentTok{\# 1x + 4y = 10}
\NormalTok{A }\OtherTok{\textless{}{-}} \FunctionTok{matrix}\NormalTok{(}\FunctionTok{c}\NormalTok{(}\DecValTok{3}\NormalTok{, }\DecValTok{1}\NormalTok{, }\DecValTok{2}\NormalTok{, }\DecValTok{4}\NormalTok{), }\AttributeTok{nrow =} \DecValTok{2}\NormalTok{)}
\NormalTok{b }\OtherTok{\textless{}{-}} \FunctionTok{c}\NormalTok{(}\DecValTok{5}\NormalTok{, }\DecValTok{10}\NormalTok{)}

\CommentTok{\#RESOLVENDO{-}O}
\NormalTok{resultado\_x }\OtherTok{\textless{}{-}} \FunctionTok{solve}\NormalTok{(A, b)}
\FunctionTok{print}\NormalTok{(resultado\_x) }
\end{Highlighting}
\end{Shaded}

\begin{verbatim}
[1] 0.0 2.5
\end{verbatim}

\begin{enumerate}
\def\labelenumi{\arabic{enumi}.}
\setcounter{enumi}{1}
\tightlist
\item
  \textbf{Arrays}
\end{enumerate}

Enquanto matrizes são estritamente bidimensionais (linhas e colunas),
\texttt{arrays} são estruturas de dados \texttt{n-dimensionais}. Pense
em um \texttt{array\ 3D} como um cubo de dados ou uma pilha de matrizes.
Eles são fundamentais em estatística espacial (x, y, tempo) ou em
imagens (x, y, canais de cor).

\begin{itemize}
\tightlist
\item
  \texttt{array(data,\ dim)}: O argumento \texttt{dim} é um vetor que
  define o tamanho de cada dimensão
  \texttt{c(linhas,\ colunas,\ profundidade/camadas,\ ...)}.
\end{itemize}

\begin{Shaded}
\begin{Highlighting}[]
\CommentTok{\# Imagine que são dados de temperatura de 2 cidades, em 3 meses, por 2 anos.}
\NormalTok{meu\_array }\OtherTok{\textless{}{-}} \FunctionTok{array}\NormalTok{(}\AttributeTok{data =} \DecValTok{1}\SpecialCharTok{:}\DecValTok{12}\NormalTok{, }\AttributeTok{dim =} \FunctionTok{c}\NormalTok{(}\DecValTok{2}\NormalTok{, }\DecValTok{3}\NormalTok{, }\DecValTok{2}\NormalTok{))}

\FunctionTok{print}\NormalTok{(meu\_array)}
\end{Highlighting}
\end{Shaded}

\begin{verbatim}
, , 1

     [,1] [,2] [,3]
[1,]    1    3    5
[2,]    2    4    6

, , 2

     [,1] [,2] [,3]
[1,]    7    9   11
[2,]    8   10   12
\end{verbatim}

\begin{Shaded}
\begin{Highlighting}[]
\FunctionTok{dim}\NormalTok{(meu\_array) }\CommentTok{\# Retorna 2 3 2}
\end{Highlighting}
\end{Shaded}

\begin{verbatim}
[1] 2 3 2
\end{verbatim}

\textbf{Data Frames}

É a estrutura mais importante para Ciência de Dados. Pense nele como uma
planilha de Excel: colunas podem ter tipos diferentes (texto, números,
datas), mas todas devem ter o mesmo comprimento (número de linhas) (
Tabela~\ref{tbl-df}).

\begin{itemize}
\tightlist
\item
  \texttt{data.frame(...)}: Cria um data frame manualmente.
\end{itemize}

\begin{Shaded}
\begin{Highlighting}[]
\NormalTok{pacman}\SpecialCharTok{::}\FunctionTok{p\_load}\NormalTok{(gt)}

\NormalTok{df }\OtherTok{\textless{}{-}} \FunctionTok{data.frame}\NormalTok{(}
  \AttributeTok{id =} \DecValTok{1}\SpecialCharTok{:}\DecValTok{3}\NormalTok{,}
  \AttributeTok{nome =} \FunctionTok{c}\NormalTok{(}\StringTok{"Ana"}\NormalTok{, }\StringTok{"Beto"}\NormalTok{, }\StringTok{"Carla"}\NormalTok{),}
  \AttributeTok{nota =} \FunctionTok{c}\NormalTok{(}\FloatTok{8.5}\NormalTok{, }\FloatTok{9.0}\NormalTok{, }\FloatTok{7.5}\NormalTok{)}
\NormalTok{)}

\NormalTok{df }\SpecialCharTok{|\textgreater{}}
  \FunctionTok{gt}\NormalTok{()}
\end{Highlighting}
\end{Shaded}

\begin{table}

\caption{\label{tbl-df}Notas dos alunos}

\centering{

\fontsize{12.0pt}{14.0pt}\selectfont
\begin{tabular*}{\linewidth}{@{\extracolsep{\fill}}rlr}
\toprule
id & nome & nota \\ 
\midrule\addlinespace[2.5pt]
1 & Ana & 8.5 \\ 
2 & Beto & 9.0 \\ 
3 & Carla & 7.5 \\ 
\bottomrule
\end{tabular*}

}

\end{table}%

\begin{itemize}
\tightlist
\item
  \texttt{head(df)}, \texttt{tail(df)}: Mostra as primeiras/últimas
  linhas ( Tabela~\ref{tbl-headTail} ).
\end{itemize}

\begin{Shaded}
\begin{Highlighting}[]
\CommentTok{\# Tabela 1 (Esquerda)}
\FunctionTok{head}\NormalTok{(df) }\SpecialCharTok{|\textgreater{}} 
  \FunctionTok{gt}\NormalTok{() }
\CommentTok{\# Tabela 2 (Direita)}
\FunctionTok{tail}\NormalTok{(df) }\SpecialCharTok{|\textgreater{}} 
  \FunctionTok{gt}\NormalTok{()}
\end{Highlighting}
\end{Shaded}

\begin{table}

\caption{\label{tbl-headTail}Inspeção dos Dados: Primeiras e Últimas
linhas}

\begin{minipage}{0.50\linewidth}

\subcaption{\label{tbl-headTail-1}Visualização do Topo (Head)}

\centering{

\fontsize{12.0pt}{14.0pt}\selectfont
\begin{tabular*}{\linewidth}{@{\extracolsep{\fill}}rlr}
\toprule
id & nome & nota \\ 
\midrule\addlinespace[2.5pt]
1 & Ana & 8.5 \\ 
2 & Beto & 9.0 \\ 
3 & Carla & 7.5 \\ 
\bottomrule
\end{tabular*}

}

\end{minipage}%
%
\begin{minipage}{0.50\linewidth}

\subcaption{\label{tbl-headTail-2}Visualização do Final (Tail)}

\centering{

\fontsize{12.0pt}{14.0pt}\selectfont
\begin{tabular*}{\linewidth}{@{\extracolsep{\fill}}rlr}
\toprule
id & nome & nota \\ 
\midrule\addlinespace[2.5pt]
1 & Ana & 8.5 \\ 
2 & Beto & 9.0 \\ 
3 & Carla & 7.5 \\ 
\bottomrule
\end{tabular*}

}

\end{minipage}%

\end{table}%

\begin{itemize}
\tightlist
\item
  \texttt{nrow(df)}, \texttt{ncol(df)}: Número de linhas e colunas.
\end{itemize}

\begin{Shaded}
\begin{Highlighting}[]
\FunctionTok{nrow}\NormalTok{(df) }\CommentTok{\#número de linhas}
\end{Highlighting}
\end{Shaded}

\begin{verbatim}
[1] 3
\end{verbatim}

\begin{Shaded}
\begin{Highlighting}[]
\FunctionTok{ncol}\NormalTok{(df) }\CommentTok{\#número de colunas}
\end{Highlighting}
\end{Shaded}

\begin{verbatim}
[1] 3
\end{verbatim}

\begin{itemize}
\tightlist
\item
  \texttt{dim(df)}: Retorna dimensões (linhas, colunas).
\end{itemize}

\begin{Shaded}
\begin{Highlighting}[]
\FunctionTok{dim}\NormalTok{(df) }\CommentTok{\#dimensão}
\end{Highlighting}
\end{Shaded}

\begin{verbatim}
[1] 3 3
\end{verbatim}

\begin{itemize}
\tightlist
\item
  \texttt{str(df),\ glimpse(df)}: Mostra a estrutura interna (tipos de
  dados de cada coluna).
\end{itemize}

\begin{Shaded}
\begin{Highlighting}[]
\FunctionTok{str}\NormalTok{(df)}
\end{Highlighting}
\end{Shaded}

\begin{verbatim}
'data.frame':   3 obs. of  3 variables:
 $ id  : int  1 2 3
 $ nome: chr  "Ana" "Beto" "Carla"
 $ nota: num  8.5 9 7.5
\end{verbatim}

Essencial para depuração.

\begin{itemize}
\tightlist
\item
  \texttt{names(df)}: Retorna ou define os nomes das colunas.
\end{itemize}

\begin{Shaded}
\begin{Highlighting}[]
\FunctionTok{names}\NormalTok{(df)}
\end{Highlighting}
\end{Shaded}

\begin{verbatim}
[1] "id"   "nome" "nota"
\end{verbatim}

\begin{itemize}
\tightlist
\item
  \texttt{rbind(df1,\ df2)}: É usada quando você tem dados novos com a
  mesma estrutura (mesmas colunas) e quer adicioná-los aos antigos (
  Tabela~\ref{tbl-rbind}).
\end{itemize}

\begin{Shaded}
\begin{Highlighting}[]
\NormalTok{pacman}\SpecialCharTok{::}\FunctionTok{p\_load}\NormalTok{(gt,dplyr)}

\NormalTok{grupo\_jan }\OtherTok{\textless{}{-}} \FunctionTok{data.frame}\NormalTok{(}
  \AttributeTok{id =} \DecValTok{1}\SpecialCharTok{:}\DecValTok{2}\NormalTok{,}
  \AttributeTok{vendas =} \FunctionTok{c}\NormalTok{(}\DecValTok{100}\NormalTok{, }\DecValTok{150}\NormalTok{)}
\NormalTok{)}

\NormalTok{grupo\_jan }\SpecialCharTok{\%\textgreater{}\%}
  \FunctionTok{gt}\NormalTok{()}
\NormalTok{grupo\_fev }\OtherTok{\textless{}{-}} \FunctionTok{data.frame}\NormalTok{(}
  \AttributeTok{id =} \DecValTok{3}\SpecialCharTok{:}\DecValTok{4}\NormalTok{,}
  \AttributeTok{vendas =} \FunctionTok{c}\NormalTok{(}\DecValTok{200}\NormalTok{, }\DecValTok{120}\NormalTok{)}
\NormalTok{) }

\NormalTok{grupo\_fev }\SpecialCharTok{\%\textgreater{}\%}
  \FunctionTok{gt}\NormalTok{()}
\CommentTok{\# Juntando (Bind de Linhas {-} Rows)}
\NormalTok{todos\_dados }\OtherTok{\textless{}{-}} \FunctionTok{rbind}\NormalTok{(grupo\_jan, grupo\_fev)}
\NormalTok{todos\_dados }\SpecialCharTok{\%\textgreater{}\%}
  \FunctionTok{gt}\NormalTok{()}
\end{Highlighting}
\end{Shaded}

\begin{table}

\caption{\label{tbl-rbind}Demonstração da função rbind}

\begin{minipage}{0.33\linewidth}

\subcaption{\label{tbl-rbind-1}Grupo Janeiro}

\centering{

\fontsize{12.0pt}{14.0pt}\selectfont
\begin{tabular*}{\linewidth}{@{\extracolsep{\fill}}rr}
\toprule
id & vendas \\ 
\midrule\addlinespace[2.5pt]
1 & 100 \\ 
2 & 150 \\ 
\bottomrule
\end{tabular*}

}

\end{minipage}%
%
\begin{minipage}{0.33\linewidth}

\subcaption{\label{tbl-rbind-2}Grupo Fevereiro}

\centering{

\fontsize{12.0pt}{14.0pt}\selectfont
\begin{tabular*}{\linewidth}{@{\extracolsep{\fill}}rr}
\toprule
id & vendas \\ 
\midrule\addlinespace[2.5pt]
3 & 200 \\ 
4 & 120 \\ 
\bottomrule
\end{tabular*}

}

\end{minipage}%
%
\begin{minipage}{0.33\linewidth}

\subcaption{\label{tbl-rbind-3}Resultado da União}

\centering{

\fontsize{12.0pt}{14.0pt}\selectfont
\begin{tabular*}{\linewidth}{@{\extracolsep{\fill}}rr}
\toprule
id & vendas \\ 
\midrule\addlinespace[2.5pt]
1 & 100 \\ 
2 & 150 \\ 
3 & 200 \\ 
4 & 120 \\ 
\bottomrule
\end{tabular*}

}

\end{minipage}%

\end{table}%

\begin{tcolorbox}[enhanced jigsaw, left=2mm, toptitle=1mm, colback=white, colframe=quarto-callout-important-color-frame, colbacktitle=quarto-callout-important-color!10!white, opacityback=0, rightrule=.15mm, bottomtitle=1mm, arc=.35mm, title=\textcolor{quarto-callout-important-color}{\faExclamation}\hspace{0.5em}{Importante}, titlerule=0mm, bottomrule=.15mm, leftrule=.75mm, coltitle=black, toprule=.15mm, breakable, opacitybacktitle=0.6]

Os nomes das colunas devem ser exatamente iguais e estar na mesma ordem
(embora data frames modernos tentem alinhar por nome, é boa prática
garantir a ordem).

\end{tcolorbox}

\begin{itemize}
\tightlist
\item
  \texttt{cbind(df1,\ df2)}: Esta função cola/junta dois data frames
  lado a lado. É uma colagem cega baseada na posição. Ela cola a linha 1
  do \texttt{df1} com a linha 1 do \texttt{df2} (
  Tabela~\ref{tbl-cbind}).
\end{itemize}

\begin{Shaded}
\begin{Highlighting}[]
\FunctionTok{library}\NormalTok{(gt)}
\FunctionTok{library}\NormalTok{(dplyr) }\CommentTok{\# Necessário se usar o pipe \%\textgreater{}\%}

\NormalTok{df\_nomes }\OtherTok{\textless{}{-}} \FunctionTok{data.frame}\NormalTok{(}\AttributeTok{nome =} \FunctionTok{c}\NormalTok{(}\StringTok{"Ana"}\NormalTok{, }\StringTok{"Beto"}\NormalTok{, }\StringTok{"Carla"}\NormalTok{))}
\NormalTok{df\_nomes }\SpecialCharTok{\%\textgreater{}\%}
  \FunctionTok{gt}\NormalTok{()}
\NormalTok{df\_idades }\OtherTok{\textless{}{-}} \FunctionTok{data.frame}\NormalTok{(}\AttributeTok{idade =} \FunctionTok{c}\NormalTok{(}\DecValTok{25}\NormalTok{, }\DecValTok{30}\NormalTok{, }\DecValTok{22}\NormalTok{))}
\NormalTok{df\_idades }\SpecialCharTok{\%\textgreater{}\%}
  \FunctionTok{gt}\NormalTok{()}
\CommentTok{\# Colando lado a lado e exibindo}
\NormalTok{df\_completo }\OtherTok{\textless{}{-}} \FunctionTok{cbind}\NormalTok{(df\_nomes, df\_idades)}
\NormalTok{df\_completo }\SpecialCharTok{\%\textgreater{}\%}
  \FunctionTok{gt}\NormalTok{()}
\end{Highlighting}
\end{Shaded}

\begin{table}

\caption{\label{tbl-cbind}Demonstração da função cbind}

\begin{minipage}{0.33\linewidth}

\subcaption{\label{tbl-cbind-1}Tabela de nomes}

\centering{

\fontsize{12.0pt}{14.0pt}\selectfont
\begin{tabular*}{\linewidth}{@{\extracolsep{\fill}}l}
\toprule
nome \\ 
\midrule\addlinespace[2.5pt]
Ana \\ 
Beto \\ 
Carla \\ 
\bottomrule
\end{tabular*}

}

\end{minipage}%
%
\begin{minipage}{0.33\linewidth}

\subcaption{\label{tbl-cbind-2}Tabela de idades}

\centering{

\fontsize{12.0pt}{14.0pt}\selectfont
\begin{tabular*}{\linewidth}{@{\extracolsep{\fill}}r}
\toprule
idade \\ 
\midrule\addlinespace[2.5pt]
25 \\ 
30 \\ 
22 \\ 
\bottomrule
\end{tabular*}

}

\end{minipage}%
%
\begin{minipage}{0.33\linewidth}

\subcaption{\label{tbl-cbind-3}Tabela final}

\centering{

\fontsize{12.0pt}{14.0pt}\selectfont
\begin{tabular*}{\linewidth}{@{\extracolsep{\fill}}lr}
\toprule
nome & idade \\ 
\midrule\addlinespace[2.5pt]
Ana & 25 \\ 
Beto & 30 \\ 
Carla & 22 \\ 
\bottomrule
\end{tabular*}

}

\end{minipage}%

\end{table}%

\begin{tcolorbox}[enhanced jigsaw, left=2mm, toptitle=1mm, colback=white, colframe=quarto-callout-important-color-frame, colbacktitle=quarto-callout-important-color!10!white, opacityback=0, rightrule=.15mm, bottomtitle=1mm, arc=.35mm, title=\textcolor{quarto-callout-important-color}{\faExclamation}\hspace{0.5em}{Importante}, titlerule=0mm, bottomrule=.15mm, leftrule=.75mm, coltitle=black, toprule=.15mm, breakable, opacitybacktitle=0.6]

Se a ordem das linhas estiver diferente (ex: o usuário 1 está na
primeira linha da tabela A, mas o usuário 5 está na primeira linha da
tabela B), seus dados ficarão corrompidos. Use apenas quando tiver
certeza absoluta que a ordem das linhas é idêntica.

\end{tcolorbox}

\begin{itemize}
\tightlist
\item
  \texttt{merge(x,\ y)}: Diferente do \texttt{cbind}, o \texttt{merge} (
  Tabela~\ref{tbl-merge}) não depende da ordem das linhas. Ele procura
  uma coluna chave \texttt{(ID,\ CPF,\ Código)} comum entre as duas
  tabelas e alinha as informações corretamente. Os parâmetros
  importantes são, \texttt{by\ =\ coluna\_chave}, a coluna usada para
  fazer o cruzamento; \texttt{all\ =\ TRUE}, que mantém todas as linhas
  (\texttt{Full\ Outer\ Join}) e \texttt{all.x\ =\ TRUE} que mantém
  todas as linhas da tabela da esquerda (\texttt{Left\ Join}).
\end{itemize}

\begin{Shaded}
\begin{Highlighting}[]
\NormalTok{funcionarios }\OtherTok{\textless{}{-}} \FunctionTok{data.frame}\NormalTok{(}
  \AttributeTok{id =} \FunctionTok{c}\NormalTok{(}\DecValTok{3}\NormalTok{, }\DecValTok{1}\NormalTok{, }\DecValTok{2}\NormalTok{),}
  \AttributeTok{nome =} \FunctionTok{c}\NormalTok{(}\StringTok{"Carlos"}\NormalTok{, }\StringTok{"Ana"}\NormalTok{, }\StringTok{"Bia"}\NormalTok{)}
\NormalTok{)}

\NormalTok{funcionarios }\SpecialCharTok{\%\textgreater{}\%}
  \FunctionTok{gt}\NormalTok{()}
\NormalTok{salarios }\OtherTok{\textless{}{-}} \FunctionTok{data.frame}\NormalTok{(}
  \AttributeTok{id =} \FunctionTok{c}\NormalTok{(}\DecValTok{1}\NormalTok{, }\DecValTok{2}\NormalTok{, }\DecValTok{3}\NormalTok{),}
  \AttributeTok{salario =} \FunctionTok{c}\NormalTok{(}\DecValTok{5000}\NormalTok{, }\DecValTok{6000}\NormalTok{, }\DecValTok{5500}\NormalTok{)}
\NormalTok{)}
\NormalTok{salarios }\SpecialCharTok{\%\textgreater{}\%}
  \FunctionTok{gt}\NormalTok{()}
\CommentTok{\# O merge procura o \textquotesingle{}id\textquotesingle{} igual e alinha as linhas corretamente}
\NormalTok{fusao }\OtherTok{\textless{}{-}} \FunctionTok{merge}\NormalTok{(}\AttributeTok{x =}\NormalTok{ funcionarios, }\AttributeTok{y =}\NormalTok{ salarios, }\AttributeTok{by =} \StringTok{"id"}\NormalTok{)}

\NormalTok{fusao }\SpecialCharTok{\%\textgreater{}\%}
  \FunctionTok{gt}\NormalTok{()}
\end{Highlighting}
\end{Shaded}

\begin{table}

\caption{\label{tbl-merge}Demonstração da função merge}

\begin{minipage}{0.33\linewidth}

\subcaption{\label{tbl-merge-1}Tabela Funcionários}

\centering{

\fontsize{12.0pt}{14.0pt}\selectfont
\begin{tabular*}{\linewidth}{@{\extracolsep{\fill}}rl}
\toprule
id & nome \\ 
\midrule\addlinespace[2.5pt]
3 & Carlos \\ 
1 & Ana \\ 
2 & Bia \\ 
\bottomrule
\end{tabular*}

}

\end{minipage}%
%
\begin{minipage}{0.33\linewidth}

\subcaption{\label{tbl-merge-2}Tabela Salários}

\centering{

\fontsize{12.0pt}{14.0pt}\selectfont
\begin{tabular*}{\linewidth}{@{\extracolsep{\fill}}rr}
\toprule
id & salario \\ 
\midrule\addlinespace[2.5pt]
1 & 5000 \\ 
2 & 6000 \\ 
3 & 5500 \\ 
\bottomrule
\end{tabular*}

}

\end{minipage}%
%
\begin{minipage}{0.33\linewidth}

\subcaption{\label{tbl-merge-3}Resultado do Merge}

\centering{

\fontsize{12.0pt}{14.0pt}\selectfont
\begin{tabular*}{\linewidth}{@{\extracolsep{\fill}}rlr}
\toprule
id & nome & salario \\ 
\midrule\addlinespace[2.5pt]
1 & Ana & 5000 \\ 
2 & Bia & 6000 \\ 
3 & Carlos & 5500 \\ 
\bottomrule
\end{tabular*}

}

\end{minipage}%

\end{table}%

\begin{itemize}
\tightlist
\item
  \texttt{summary(df)}: Resumo estatístico básico.
\end{itemize}

\begin{Shaded}
\begin{Highlighting}[]
\FunctionTok{summary}\NormalTok{(fusao)}
\end{Highlighting}
\end{Shaded}

\begin{verbatim}
       id          nome              salario    
 Min.   :1.0   Length:3           Min.   :5000  
 1st Qu.:1.5   Class :character   1st Qu.:5250  
 Median :2.0   Mode  :character   Median :5500  
 Mean   :2.0                      Mean   :5500  
 3rd Qu.:2.5                      3rd Qu.:5750  
 Max.   :3.0                      Max.   :6000  
\end{verbatim}

\textbf{Fatores}

Fatores são usados para variáveis categóricas (qualitativas). O
\texttt{R} armazena internamente como números inteiros (1,2,3\ldots),
mas exibe rótulos (labels). A ordem dos níveis (levels) é crucial para a
ordem em gráficos.

\begin{itemize}
\tightlist
\item
  \texttt{factor(x,\ levels,\ ordered)}: Cria um fator. Definir
  \texttt{levels} é crucial para fixar a ordem das categorias (ex: em
  gráficos ou modelos).
\end{itemize}

\begin{Shaded}
\begin{Highlighting}[]
\NormalTok{sexo }\OtherTok{\textless{}{-}} \FunctionTok{c}\NormalTok{(}\StringTok{"M"}\NormalTok{, }\StringTok{"F"}\NormalTok{, }\StringTok{"F"}\NormalTok{, }\StringTok{"M"}\NormalTok{) }\CommentTok{\# Cria um vetor de texto}

\NormalTok{fator\_sexo }\OtherTok{\textless{}{-}} \FunctionTok{factor}\NormalTok{(sexo, }\AttributeTok{levels =} \FunctionTok{c}\NormalTok{(}\StringTok{"F"}\NormalTok{, }\StringTok{"M"}\NormalTok{)) }\CommentTok{\# Converte para fator}
\end{Highlighting}
\end{Shaded}

\section{Entrada e Manipulação de Dados}\label{sec-Entr_Man}

\textbf{Importação e Exportação}

Para analisar os dados primeiro deve ler (importar) eles para o
R/Rstudio. O pacote \texttt{readr} (do \texttt{Tidyverse}) é preferível
ao \texttt{R} base por ser mais rápido e não converter texto em fator
automaticamente.

\textbf{Pacote \texttt{readr} (\texttt{Tidyverse} - Moderno e Rápido):}

\begin{itemize}
\tightlist
\item
  \texttt{read\_csv("arquivo.csv")}: Lê arquivos separados por vírgula.
\item
  \texttt{read\_csv2("arquivo.csv")}: Lê arquivos separados por ponto e
  vírgula (comum no Brasil/Europa onde a vírgula é decimal).
\item
  \texttt{read\_delim("arquivo.txt",\ delim\ =\ "\textbar{}")}: Lê
  arquivos com delimitadores personalizados.
\item
  \texttt{write\_csv(x,\ "arquivo.csv")}: Salva um data frame em disco.
\end{itemize}

\begin{Shaded}
\begin{Highlighting}[]
\NormalTok{pacman}\SpecialCharTok{::}\FunctionTok{p\_load}\NormalTok{(readxl)}
\NormalTok{dados }\OtherTok{\textless{}{-}} \FunctionTok{read\_csv2}\NormalTok{(}\StringTok{"dados\_brasil.csv"}\NormalTok{)}

\FunctionTok{write\_csv}\NormalTok{(dados, }\StringTok{"dados\_limpos.csv"}\NormalTok{)}

\CommentTok{\# Para Excel (requer pacote extra)}
\NormalTok{dados\_excel }\OtherTok{\textless{}{-}} \FunctionTok{read\_excel}\NormalTok{(}\StringTok{"planilha.xlsx"}\NormalTok{, }\AttributeTok{sheet =} \StringTok{"Aba1"}\NormalTok{)}
\end{Highlighting}
\end{Shaded}

\textbf{Base R (Clássico):}

\begin{itemize}
\tightlist
\item
  \texttt{read.table("arquivo.txt")}: A função base mais flexível e
  genérica para importar dados tabulares de arquivos de texto,
  permitindo controle total sobre todos os parâmetros (separadores,
  decimais, cabeçalhos). Utilize quando o arquivo de dados não segue
  padrões comuns (como CSV padrão) ou quando você precisa especificar
  manualmente como o \texttt{R} deve interpretar o arquivo.
\end{itemize}

\begin{Shaded}
\begin{Highlighting}[]
\CommentTok{\# Lê arquivo separado por tabulação (\textquotesingle{}\textbackslash{}t\textquotesingle{}) com cabeçalho}
\NormalTok{dados }\OtherTok{\textless{}{-}} \FunctionTok{read.table}\NormalTok{(}\StringTok{"dados.txt"}\NormalTok{, }\AttributeTok{header =} \ConstantTok{TRUE}\NormalTok{, }\AttributeTok{sep =} \StringTok{"}\SpecialCharTok{\textbackslash{}t}\StringTok{"}\NormalTok{)}
\end{Highlighting}
\end{Shaded}

\begin{itemize}
\tightlist
\item
  \texttt{read.csv()}: Um ``wrapper'' (atalho) do \texttt{read.table}
  pré-configurado especificamente para arquivos separados por vírgula
  (padrão internacional). Utilize para a leitura rápida de arquivos
  \texttt{.csv} padrão sem precisar configurar parâmetros extras.
\end{itemize}

\begin{Shaded}
\begin{Highlighting}[]
\NormalTok{vendas }\OtherTok{\textless{}{-}} \FunctionTok{read.csv}\NormalTok{(}\StringTok{"vendas\_2024.csv"}\NormalTok{)}
\NormalTok{dados }\OtherTok{\textless{}{-}}\NormalTok{ readr}\SpecialCharTok{::}\FunctionTok{read\_csv}\NormalTok{(}\StringTok{"https://raw.githubusercontent.com/mwaskom/seaborn{-}data/master/iris.csv"}\NormalTok{) }\CommentTok{\#usando link}
\end{Highlighting}
\end{Shaded}

\begin{tcolorbox}[enhanced jigsaw, left=2mm, toptitle=1mm, colback=white, colframe=quarto-callout-important-color-frame, colbacktitle=quarto-callout-important-color!10!white, opacityback=0, rightrule=.15mm, bottomtitle=1mm, arc=.35mm, title=\textcolor{quarto-callout-important-color}{\faExclamation}\hspace{0.5em}{Problemas no encoding}, titlerule=0mm, bottomrule=.15mm, leftrule=.75mm, coltitle=black, toprule=.15mm, breakable, opacitybacktitle=0.6]

No Português se usa vírgula \texttt{(,)} para decimal e acentos
(\href{https://pt.wikipedia.org/wiki/ISO/IEC_8859-1}{\texttt{Latin1/ISO-8859-1}}).
O padrão mundial é ponto \texttt{(.)} para decimal e
\href{https://pt.wikipedia.org/wiki/UTF-8}{\texttt{UTF-8}}.

\begin{itemize}
\item
  use \texttt{read\_csv()}: Espera separador vírgula (padrão US).
\item
  Use \texttt{read\_csv2()}: Espera separador ponto-e-vírgula (padrão
  BR).
\end{itemize}

Se seus textos (nomes de cidades) aparecerem com símbolos estranhos
(São Paulo), force o encoding

\end{tcolorbox}

\begin{Shaded}
\begin{Highlighting}[]
\NormalTok{pacman}\SpecialCharTok{::}\FunctionTok{p\_load}\NormalTok{(readr)}
\NormalTok{dados }\OtherTok{\textless{}{-}} \FunctionTok{read\_csv2}\NormalTok{(}\StringTok{"dados\_br.csv"}\NormalTok{, }\AttributeTok{locale =} \FunctionTok{locale}\NormalTok{(}\AttributeTok{encoding =} \StringTok{"Latin1"}\NormalTok{))}
\end{Highlighting}
\end{Shaded}

\begin{itemize}
\tightlist
\item
  \texttt{write\_csv()}: Salva data frames em arquivo \texttt{CSV} de
  forma mais rápida e moderna que o \texttt{write.csv} base. Utilize
  sempre que precisar exportar dados processados para CSV, pois ele não
  escreve nomes de linhas (row names) por padrão e lida melhor com
  caracteres especiais.
\end{itemize}

\begin{Shaded}
\begin{Highlighting}[]
\CommentTok{\# Salva o arquivo sem criar aquela coluna de índice numérico (1, 2, 3...)}
\NormalTok{readr}\SpecialCharTok{::}\FunctionTok{write\_csv}\NormalTok{(iris, }\StringTok{"iris\_limpo.csv"}\NormalTok{)}
\end{Highlighting}
\end{Shaded}

\begin{itemize}
\tightlist
\item
  \texttt{scan()}: Leitura primitiva e/ou função de baixo nível que lê
  dados sequencialmente e os converte em vetores ou listas, em vez de
  data frames. Utilize para ler arquivos com estrutura irregular, para
  colar dados copiados diretamente no \texttt{console} ou quando
  \texttt{read.table} falha devido a inconsistências no arquivo.
\end{itemize}

\begin{Shaded}
\begin{Highlighting}[]
\CommentTok{\# Lê números de um arquivo diretamente para um vetor numérico}
\NormalTok{vetor\_números }\OtherTok{\textless{}{-}} \FunctionTok{scan}\NormalTok{(}\StringTok{"números.txt"}\NormalTok{, }\AttributeTok{what =} \FunctionTok{numeric}\NormalTok{())}
\end{Highlighting}
\end{Shaded}

\begin{itemize}
\tightlist
\item
  \texttt{save(obj,\ file="dados.RData")}: Salva um ou mais objetos
  \texttt{R} específicos (variáveis, modelos, data frames) que estão no
  \texttt{Environment} em um arquivo binário compactado
  \texttt{(.RData)}, preservando tipos e classes. Utilize para salvar
  resultados intermediários importantes (ex: ajustou um modelo que levou
  muito tempo e não quer voltar ajustar novamente) para uso futuro, sem
  salvar o ``lixo'' do ambiente de trabalho inteiro.
\end{itemize}

\begin{Shaded}
\begin{Highlighting}[]
\CommentTok{\# Você ajustou um modelo e chamou{-}o de fit}
\CommentTok{\# O codigo abaixo salva apenas o dataframe \textquotesingle{}dados\textquotesingle{} e o modelo \textquotesingle{}fit\textquotesingle{} que estão no Environment}
\FunctionTok{save}\NormalTok{(dados, fit, }\AttributeTok{file =} \StringTok{"resultados\_parciais.RData"}\NormalTok{)}
\end{Highlighting}
\end{Shaded}

\begin{itemize}
\tightlist
\item
  \texttt{load("dados.RData")}: Carrega objetos salvos em arquivos
  \texttt{.RData} diretamente para a memória do \texttt{R}, mantendo os
  nomes originais dos objetos. Utilize para retomar análises carregando
  dados pré-processados ou modelos salvos, economizando o tempo de rodar
  scripts novamente.
\end{itemize}

\begin{Shaded}
\begin{Highlighting}[]
\FunctionTok{load}\NormalTok{(}\StringTok{"resultados\_parciais.RData"}\NormalTok{)}
\end{Highlighting}
\end{Shaded}

\begin{itemize}
\tightlist
\item
  \texttt{save.image()}: Um atalho que salva todos os objetos presentes
  no seu ambiente de trabalho (Workspace) atual em um único arquivo.
  Utilize ao encerrar uma sessão de trabalho complexa para garantir que
  você possa continuar exatamente de onde parou (geralmente salva como
  \texttt{.RData}).
\end{itemize}

\begin{Shaded}
\begin{Highlighting}[]
\FunctionTok{save.image}\NormalTok{(}\AttributeTok{file =} \StringTok{"backup\_projeto\_tarde.RData"}\NormalTok{)}
\end{Highlighting}
\end{Shaded}

\textbf{Excel e Google Sheets:}

\begin{itemize}
\tightlist
\item
  \texttt{readxl::read\_excel("arq.xlsx",\ sheet\ =\ 1)}: A função mais
  robusta e eficiente para importar dados de arquivos Microsoft Excel
  (\texttt{.xls} e \texttt{.xlsx}) sem depender de instalações externas
  complexas (como \texttt{Java}). Utilize para carregar dados
  armazenados localmente em arquivos \texttt{Excel}, permitindo
  especificar qual aba (\texttt{sheet}) deve ser lida pelo nome ou
  índice.
\end{itemize}

\begin{Shaded}
\begin{Highlighting}[]
\NormalTok{tabela }\OtherTok{\textless{}{-}}\NormalTok{ readxl}\SpecialCharTok{::}\FunctionTok{read\_excel}\NormalTok{(}\StringTok{"relatorio\_anual.xlsx"}\NormalTok{, }\AttributeTok{sheet =} \StringTok{"Dados\_Brutos"}\NormalTok{)}
\end{Highlighting}
\end{Shaded}

\begin{itemize}
\tightlist
\item
  \texttt{googlesheets4::read\_sheet("URL")}: Função que conecta
  diretamente à
  \href{https://pt.wikipedia.org/wiki/Interface_de_programa\%C3\%A7\%C3\%A3o_de_aplica\%C3\%A7\%C3\%B5es}{API}
  do Google para baixar e importar dados de planilhas hospedadas na
  nuvem (\texttt{Google\ Drive}), gerenciando a autenticação do usuário.
  Utilize para acessar dados diretamente pela
  \href{https://pt.wikipedia.org/wiki/URL}{\texttt{URL}} ou \texttt{ID}
  da planilha, eliminando a necessidade de baixar o arquivo manualmente
  antes de ler.
\end{itemize}

\begin{Shaded}
\begin{Highlighting}[]
\NormalTok{dados\_online }\OtherTok{\textless{}{-}}\NormalTok{ googlesheets4}\SpecialCharTok{::}\FunctionTok{read\_sheet}\NormalTok{(}\StringTok{"https://docs.google.com/spreadsheets/d/..."}\NormalTok{)}
\end{Highlighting}
\end{Shaded}

\textbf{Ler dados de alta dimensão (Big Data)}

\begin{itemize}
\tightlist
\item
  \texttt{read\_parquet()}: Lê arquivos no formato \texttt{Parquet} do
  pacote \texttt{arrow}, um formato de armazenamento colunar altamente
  comprimido e eficiente, amplamente usado em Big Data. Utilize para
  importar grandes volumes de dados (milhões de linhas) com extrema
  rapidez e baixo uso de memória.
\end{itemize}

\begin{Shaded}
\begin{Highlighting}[]
\NormalTok{dados\_gigantes }\OtherTok{\textless{}{-}} \FunctionTok{read\_parquet}\NormalTok{(}\StringTok{"dados\_grandes.parquet"}\NormalTok{)}
\end{Highlighting}
\end{Shaded}

\begin{itemize}
\tightlist
\item
  \texttt{write\_parquet()}: Salva um data frame ou tibble no formato
  Parquet. Utilize para armazenar dados processados ocupando muito menos
  espaço em disco que um \texttt{CSV} e permitindo leituras futuras
  muito mais rápidas.
\end{itemize}

\begin{Shaded}
\begin{Highlighting}[]
\FunctionTok{write\_parquet}\NormalTok{(iris, }\StringTok{"iris\_otimizado.parquet"}\NormalTok{)}
\end{Highlighting}
\end{Shaded}

\begin{itemize}
\tightlist
\item
  \texttt{saveRDS()} salva um único objeto do \texttt{R} em um arquivo
  binário, sem salvar o nome original da variável. A melhor opção para
  salvar objetos individuais para ser carregado posteriormente com
  qualquer nome.
\end{itemize}

\begin{Shaded}
\begin{Highlighting}[]
\CommentTok{\# Vc ajustou modelo e pode salvar apenas o resultado final em um arquivo .rds}
\FunctionTok{saveRDS}\NormalTok{(modelo\_final, }\StringTok{"meu\_modelo.rds"}\NormalTok{)}
\end{Highlighting}
\end{Shaded}

\begin{itemize}
\tightlist
\item
  \texttt{readRDS()}: Lê um arquivo \texttt{.rds} e retorna o objeto
  salvo, exigindo que você o atribua a uma nova variável. Utilize para
  carregar objetos salvos com \texttt{saveRDS}. Diferente de
  \texttt{load()}, ele não suja seu ambiente com nomes variáveis
  desconhecidos; você escolhe o nome.
\end{itemize}

\begin{Shaded}
\begin{Highlighting}[]
\CommentTok{\# Carrega o modelo salvo atribuindo a uma nova variável}
\NormalTok{modelo\_carregado }\OtherTok{\textless{}{-}} \FunctionTok{readRDS}\NormalTok{(}\StringTok{"meu\_modelo.rds"}\NormalTok{)}
\end{Highlighting}
\end{Shaded}

\section{\texorpdfstring{Manipulação com \texttt{dplyr}
(Tidyverse)}{Manipulação com dplyr (Tidyverse)}}\label{sec-dplyr}

O padrão ouro atual para manipulação de dados. Utiliza o operador pipe
\texttt{\%\textgreater{}\%} para encadear ações de forma legível.

\textbf{Algumas funções:}

\begin{itemize}
\tightlist
\item
  \texttt{select(df,\ col1,\ col2)}: Seleciona colunas e mantém apenas
  as colunas selecionadas em um data frame. Utilize para reduzir o
  conjunto de dados, descartando variáveis irrelevantes para a análise
  atual ou para reordenar colunas.
\end{itemize}

\begin{Shaded}
\begin{Highlighting}[]
\CommentTok{\# Seleciona apenas as colunas Sepal.Length e Species}
\NormalTok{iris }\SpecialCharTok{\%\textgreater{}\%} \FunctionTok{select}\NormalTok{(Sepal.Length, Species)}\SpecialCharTok{\%\textgreater{}\%}
  \FunctionTok{reactable}\NormalTok{(  }\AttributeTok{searchable =} \ConstantTok{TRUE}\NormalTok{,}
\NormalTok{)}
\end{Highlighting}
\end{Shaded}

\pandocbounded{\includegraphics[keepaspectratio]{intro_files/figure-pdf/unnamed-chunk-90-1.pdf}}

\begin{itemize}
\tightlist
\item
  \texttt{filter(df,\ condicao)}: Filtra linhas baseado em condições
  lógicas. Utilize para extrair observações específicas, como ``apenas
  vendas acima de 100'' ou ``apenas dados de 2024''.
\end{itemize}

\begin{Shaded}
\begin{Highlighting}[]
\CommentTok{\# Filtra apenas linhas da espécie setosa}
\NormalTok{iris }\SpecialCharTok{\%\textgreater{}\%} \FunctionTok{filter}\NormalTok{(Species }\SpecialCharTok{==} \StringTok{"setosa"}\NormalTok{)}\SpecialCharTok{\%\textgreater{}\%}
  \FunctionTok{reactable}\NormalTok{(  }\AttributeTok{sortable =} \ConstantTok{TRUE}\NormalTok{,}
  \AttributeTok{resizable =} \ConstantTok{TRUE}\NormalTok{,}
  \AttributeTok{filterable =} \ConstantTok{TRUE}\NormalTok{,}
  \AttributeTok{searchable =} \ConstantTok{TRUE}\NormalTok{)}
\end{Highlighting}
\end{Shaded}

\pandocbounded{\includegraphics[keepaspectratio]{intro_files/figure-pdf/unnamed-chunk-91-1.pdf}}

\begin{itemize}
\tightlist
\item
  \texttt{mutate(df,\ nova\_col\ =\ x\ +\ y)}: Cria ou modifica colunas
  existentes preservando as demais. Utilize para criar cálculos (ex:
  conversão de unidades), transformar dados ou criar variáveis
  derivadas.
\end{itemize}

\begin{Shaded}
\begin{Highlighting}[]
\CommentTok{\# Cria uma nova coluna com a razão (divisão) entre sépala e pétala}
\NormalTok{iris }\SpecialCharTok{\%\textgreater{}\%} \FunctionTok{mutate}\NormalTok{(}\AttributeTok{razao =}\NormalTok{ Sepal.Length }\SpecialCharTok{/}\NormalTok{ Petal.Length)}\SpecialCharTok{\%\textgreater{}\%}
  \FunctionTok{reactable}\NormalTok{(  }\AttributeTok{sortable =} \ConstantTok{TRUE}\NormalTok{,}
  \AttributeTok{resizable =} \ConstantTok{TRUE}\NormalTok{,}
  \AttributeTok{filterable =} \ConstantTok{TRUE}\NormalTok{,}
  \AttributeTok{searchable =} \ConstantTok{TRUE}\NormalTok{)}
\end{Highlighting}
\end{Shaded}

\pandocbounded{\includegraphics[keepaspectratio]{intro_files/figure-pdf/unnamed-chunk-92-1.pdf}}

\begin{itemize}
\tightlist
\item
  \texttt{arrange(df,\ col)}: Ordena as linhas. Use \texttt{desc(col)}
  para ordem decrescente. Utilize para classificar dados, como colocar
  os maiores valores no topo (\texttt{desc()}) ou ordenar
  alfabeticamente.
\end{itemize}

\begin{Shaded}
\begin{Highlighting}[]
\CommentTok{\# Ordena por Sepal.Length de forma decrescente}
\NormalTok{iris }\SpecialCharTok{\%\textgreater{}\%} \FunctionTok{arrange}\NormalTok{(}\FunctionTok{desc}\NormalTok{(Sepal.Length))}\SpecialCharTok{\%\textgreater{}\%}
  \FunctionTok{reactable}\NormalTok{()}
\end{Highlighting}
\end{Shaded}

\pandocbounded{\includegraphics[keepaspectratio]{intro_files/figure-pdf/unnamed-chunk-93-1.pdf}}

\begin{itemize}
\tightlist
\item
  \texttt{summarise(df,\ media\ =\ mean(x))}: Reduz múltiplos valores a
  um único resumo estatístico (soma, média, contagem). Utilize,
  geralmente após um \texttt{group\_by}, para obter métricas agregadas
  dos seus dados.
\end{itemize}

\begin{Shaded}
\begin{Highlighting}[]
\CommentTok{\# Calcula a média do comprimento das sépalas}
\NormalTok{iris }\SpecialCharTok{\%\textgreater{}\%} \FunctionTok{summarise}\NormalTok{(}\AttributeTok{media\_sepala =} \FunctionTok{mean}\NormalTok{(Sepal.Length))}\SpecialCharTok{\%\textgreater{}\%}
  \FunctionTok{gt}\NormalTok{()}
\end{Highlighting}
\end{Shaded}

\begin{table}
\fontsize{12.0pt}{14.0pt}\selectfont
\begin{tabular*}{\linewidth}{@{\extracolsep{\fill}}r}
\toprule
media\_sepala \\ 
\midrule\addlinespace[2.5pt]
5.843333 \\ 
\bottomrule
\end{tabular*}
\end{table}

\begin{itemize}
\tightlist
\item
  \texttt{group\_by(df,\ categoria)}: Agrupa os dados. Utilize
  imediatamente antes de \texttt{summarise} ou mutate para aplicar a
  cada grupo separadamente.
\end{itemize}

\begin{Shaded}
\begin{Highlighting}[]
\CommentTok{\# Agrupa por espécie (preparação para cálculo subsequente)}
\NormalTok{iris }\SpecialCharTok{\%\textgreater{}\%} \FunctionTok{group\_by}\NormalTok{(Species)}\SpecialCharTok{\%\textgreater{}\%}
  \FunctionTok{reactable}\NormalTok{(}\AttributeTok{filterable =} \ConstantTok{TRUE}\NormalTok{,}
  \AttributeTok{searchable =} \ConstantTok{TRUE}\NormalTok{,)}
\end{Highlighting}
\end{Shaded}

\pandocbounded{\includegraphics[keepaspectratio]{intro_files/figure-pdf/unnamed-chunk-95-1.pdf}}

\begin{itemize}
\tightlist
\item
  \texttt{rename(df,\ novo\ =\ velho)}: Renomeia colunas. Utilize para
  tornar nomes de variáveis mais legíveis ou compatíveis com padrões de
  código (\texttt{novo\_nome\ =\ velho\_nome}).
\end{itemize}

\begin{Shaded}
\begin{Highlighting}[]
\CommentTok{\# Renomeia Sepal.Length para comprimento\_sepala}
\NormalTok{iris }\SpecialCharTok{\%\textgreater{}\%} \FunctionTok{rename}\NormalTok{(}\AttributeTok{comprimento\_sepala =}\NormalTok{ Sepal.Length)}\SpecialCharTok{\%\textgreater{}\%}
  \FunctionTok{reactable}\NormalTok{(}\AttributeTok{filterable =} \ConstantTok{TRUE}\NormalTok{,}
  \AttributeTok{searchable =} \ConstantTok{TRUE}\NormalTok{,)}
\end{Highlighting}
\end{Shaded}

\pandocbounded{\includegraphics[keepaspectratio]{intro_files/figure-pdf/unnamed-chunk-96-1.pdf}}

\begin{itemize}
\tightlist
\item
  \texttt{relocate(df,\ col,\ .before\ =\ col2)}: Reordena a posição das
  colunas. Isto é, move colunas para novas posições dentro do data
  frame. Utilize para organizar a visualização, trazendo colunas
  importantes para o início (\texttt{.before} ou \texttt{.after}).
\end{itemize}

\begin{Shaded}
\begin{Highlighting}[]
\CommentTok{\# Move a coluna Species para antes de todas as outras}

\NormalTok{iris }\SpecialCharTok{\%\textgreater{}\%} \FunctionTok{relocate}\NormalTok{(Species, }\AttributeTok{.before =} \FunctionTok{everything}\NormalTok{())}\SpecialCharTok{\%\textgreater{}\%}
  \FunctionTok{reactable}\NormalTok{()}
\end{Highlighting}
\end{Shaded}

\pandocbounded{\includegraphics[keepaspectratio]{intro_files/figure-pdf/unnamed-chunk-97-1.pdf}}

\begin{itemize}
\tightlist
\item
  \texttt{slice(df,\ n:m)}: Seleciona linhas baseando-se em suas
  posições (índices) inteiras. Utilize quando precisar de linhas
  específicas pela posição, como \emph{as 5 primeiras} ou \emph{a última
  linha}, independente dos valores ( Tabela~\ref{tbl-slice}).
\end{itemize}

\begin{Shaded}
\begin{Highlighting}[]
\CommentTok{\# Seleciona da linha 10 até a linha 15}
\NormalTok{iris }\SpecialCharTok{\%\textgreater{}\%} \FunctionTok{slice}\NormalTok{(}\DecValTok{10}\SpecialCharTok{:}\DecValTok{15}\NormalTok{)}\SpecialCharTok{\%\textgreater{}\%}
  \FunctionTok{gt}\NormalTok{()}
\end{Highlighting}
\end{Shaded}

\begin{table}

\caption{\label{tbl-slice}Linhas 10 a 15 do conjunto de dados flor Iris}

\centering{

\fontsize{12.0pt}{14.0pt}\selectfont
\begin{tabular*}{\linewidth}{@{\extracolsep{\fill}}rrrrc}
\toprule
Sepal.Length & Sepal.Width & Petal.Length & Petal.Width & Species \\ 
\midrule\addlinespace[2.5pt]
4.9 & 3.1 & 1.5 & 0.1 & setosa \\ 
5.4 & 3.7 & 1.5 & 0.2 & setosa \\ 
4.8 & 3.4 & 1.6 & 0.2 & setosa \\ 
4.8 & 3.0 & 1.4 & 0.1 & setosa \\ 
4.3 & 3.0 & 1.1 & 0.1 & setosa \\ 
5.8 & 4.0 & 1.2 & 0.2 & setosa \\ 
\bottomrule
\end{tabular*}

}

\end{table}%

\begin{itemize}
\tightlist
\item
  \texttt{recode():} Substitui valores específicos num vetor numérico ou
  de caracteres ( Tabela~\ref{tbl-recode}). Utilize dentro de um
  \texttt{mutate} para corrigir erros de digitação ou traduzir
  categorias específicas rapidamente (Substituído modernamente por
  \texttt{case\_match}).
\end{itemize}

\begin{Shaded}
\begin{Highlighting}[]
\CommentTok{\# Renomeia "setosa" para "Setosa\_Pura" na coluna Species}
\NormalTok{iris }\SpecialCharTok{\%\textgreater{}\%} \FunctionTok{mutate}\NormalTok{(}\AttributeTok{Species =} \FunctionTok{recode}\NormalTok{(Species, }\StringTok{"setosa"} \OtherTok{=} \StringTok{"Setosa\_Pura"}\NormalTok{))}\SpecialCharTok{\%\textgreater{}\%}
  \FunctionTok{reactable}\NormalTok{()}
\end{Highlighting}
\end{Shaded}

\begin{table}

\caption{\label{tbl-recode}Coluna Species renomeada de Setosa para
Setosa\_Pura}

\centering{

\pandocbounded{\includegraphics[keepaspectratio]{intro_files/figure-pdf/tbl-recode-1.pdf}}

}

\end{table}%

\begin{itemize}
\tightlist
\item
  \texttt{across()}: Função auxiliar que permite aplicar uma mesma
  transformação ou função de resumo a múltiplas colunas selecionadas
  simultaneamente. Utilize dentro de \texttt{mutate()} ou
  \texttt{summarise()} quando precisar repetir a mesma operação (ex:
  calcular média, converter tipo) em várias variáveis sem duplicar
  código ( Tabela~\ref{tbl-sep_media}).
\end{itemize}

\begin{Shaded}
\begin{Highlighting}[]
\CommentTok{\# Exemplo: Calcular a média apenas das colunas que começam com "Sepal"}
\NormalTok{iris }\SpecialCharTok{\%\textgreater{}\%}
  \FunctionTok{summarise}\NormalTok{(}\FunctionTok{across}\NormalTok{(}\FunctionTok{starts\_with}\NormalTok{(}\StringTok{"Sepal"}\NormalTok{), mean))}\SpecialCharTok{\%\textgreater{}\%}
  \FunctionTok{gt}\NormalTok{()}
\end{Highlighting}
\end{Shaded}

\begin{table}

\caption{\label{tbl-sep_media}Média das Sepalas}

\centering{

\fontsize{12.0pt}{14.0pt}\selectfont
\begin{tabular*}{\linewidth}{@{\extracolsep{\fill}}rr}
\toprule
Sepal.Length & Sepal.Width \\ 
\midrule\addlinespace[2.5pt]
5.843333 & 3.057333 \\ 
\bottomrule
\end{tabular*}

}

\end{table}%

\begin{itemize}
\tightlist
\item
  \texttt{case\_when():} Uma estrutura condicional vetorizada que
  permite criar ou modificar valores baseando-se em uma sequência de
  múltiplas condições lógicas (como vários \texttt{if-else} encadeados).
  Utilize para categorizar variáveis ou criar novas colunas baseadas em
  regras complexas, evitando o uso confuso de múltiplos
  \texttt{ifelse()} aninhados (Table Tabela~\ref{tbl-cat_petalas}).
\end{itemize}

\begin{Shaded}
\begin{Highlighting}[]
\CommentTok{\# Exemplo: Categorizar o tamanho da pétala em Pequena, Média, Grande}
\NormalTok{iris }\SpecialCharTok{\%\textgreater{}\%}
  \FunctionTok{mutate}\NormalTok{(}\AttributeTok{Categoria =} \FunctionTok{case\_when}\NormalTok{(}
\NormalTok{    Petal.Length }\SpecialCharTok{\textless{}} \DecValTok{2} \SpecialCharTok{\textasciitilde{}} \StringTok{"Pequena"}\NormalTok{,}
\NormalTok{    Petal.Length }\SpecialCharTok{\textless{}} \DecValTok{5} \SpecialCharTok{\textasciitilde{}} \StringTok{"Média"}\NormalTok{,}
    \ConstantTok{TRUE} \SpecialCharTok{\textasciitilde{}} \StringTok{"Grande"} \CommentTok{\# \textquotesingle{}TRUE\textquotesingle{} age como o \textquotesingle{}else\textquotesingle{} (caso contrário) final}
\NormalTok{  )) }\SpecialCharTok{\%\textgreater{}\%}
  \FunctionTok{head}\NormalTok{()}\SpecialCharTok{\%\textgreater{}\%}
  \FunctionTok{gt}\NormalTok{()}
\end{Highlighting}
\end{Shaded}

\begin{table}

\caption{\label{tbl-cat_petalas}Pétala categorizadas em pequena, média,
grande}

\centering{

\fontsize{12.0pt}{14.0pt}\selectfont
\begin{tabular*}{\linewidth}{@{\extracolsep{\fill}}rrrrcl}
\toprule
Sepal.Length & Sepal.Width & Petal.Length & Petal.Width & Species & Categoria \\ 
\midrule\addlinespace[2.5pt]
5.1 & 3.5 & 1.4 & 0.2 & setosa & Pequena \\ 
4.9 & 3.0 & 1.4 & 0.2 & setosa & Pequena \\ 
4.7 & 3.2 & 1.3 & 0.2 & setosa & Pequena \\ 
4.6 & 3.1 & 1.5 & 0.2 & setosa & Pequena \\ 
5.0 & 3.6 & 1.4 & 0.2 & setosa & Pequena \\ 
5.4 & 3.9 & 1.7 & 0.4 & setosa & Pequena \\ 
\bottomrule
\end{tabular*}

}

\end{table}%

\begin{Shaded}
\begin{Highlighting}[]
\CommentTok{\# "Pegue o df, ENTÃO filtre notas altas, ENTÃO crie uma coluna de status"}
\NormalTok{df\_novo }\OtherTok{\textless{}{-}}\NormalTok{ df }\SpecialCharTok{\%\textgreater{}\%}
  \FunctionTok{filter}\NormalTok{(nota }\SpecialCharTok{\textgreater{}} \DecValTok{8}\NormalTok{) }\SpecialCharTok{\%\textgreater{}\%}
  \FunctionTok{mutate}\NormalTok{(}\AttributeTok{status =} \StringTok{"Aprovado com Louvor"}\NormalTok{) }\SpecialCharTok{\%\textgreater{}\%}
  \FunctionTok{select}\NormalTok{(nome, status)}

\CommentTok{\# Agrupamento}
\NormalTok{resumo }\OtherTok{\textless{}{-}}\NormalTok{ df }\SpecialCharTok{\%\textgreater{}\%}
  \FunctionTok{group\_by}\NormalTok{(nome) }\SpecialCharTok{\%\textgreater{}\%}   \CommentTok{\# Supondo que \textquotesingle{}nome\textquotesingle{} seja uma categoria}
  \FunctionTok{summarise}\NormalTok{(}
    \AttributeTok{media\_nota =} \FunctionTok{mean}\NormalTok{(nota, }\AttributeTok{na.rm =} \ConstantTok{TRUE}\NormalTok{),}
    \AttributeTok{total =} \FunctionTok{n}\NormalTok{() }
\NormalTok{  )}
\end{Highlighting}
\end{Shaded}

\textbf{Joins (Combinação de Tabelas):}

\begin{itemize}
\tightlist
\item
  \texttt{left\_join(x,\ y,\ by\ =\ "key")}: Mantém todas as linhas de
  \texttt{x}, traz correspondências de \texttt{y}, baseando-se na coluna
  \texttt{key} em comum entre \texttt{x} e \texttt{y}. Isto é, combina
  duas tabelas mantendo todas as linhas da tabela da esquerda (x) e
  adicionando as colunas da direita (y) onde houver correspondência na
  chave (\texttt{key}). Utilize para enriquecer uma tabela principal com
  dados auxiliares sem perder observações originais.
\end{itemize}

\begin{Shaded}
\begin{Highlighting}[]
\CommentTok{\# Adiciona dados dos produtos à tabela de vendas}
\NormalTok{vendas\_detalhadas }\OtherTok{\textless{}{-}} \FunctionTok{left\_join}\NormalTok{(vendas, produtos, }\AttributeTok{by =} \StringTok{"id\_produto"}\NormalTok{)}
\end{Highlighting}
\end{Shaded}

\begin{itemize}
\tightlist
\item
  \texttt{inner\_join(x,\ y)}: Mantém apenas linhas que existem em ambas
  as tabelas. Isto é, retorna apenas as linhas onde a chave de ligação
  existe simultaneamente em ambas as tabelas (interseção), descartando o
  resto. Utilize quando você precisa analisar apenas casos completos que
  tenham dados em ambas as fontes.
\end{itemize}

\begin{Shaded}
\begin{Highlighting}[]
\CommentTok{\# Mantém apenas alunos que têm notas registradas}
\NormalTok{alunos\_com\_notas }\OtherTok{\textless{}{-}} \FunctionTok{inner\_join}\NormalTok{(alunos, notas, }\AttributeTok{by =} \StringTok{"matricula"}\NormalTok{)}
\end{Highlighting}
\end{Shaded}

\begin{itemize}
\tightlist
\item
  \texttt{full\_join(x,\ y)}: Mantém todas as linhas de ambas as
  tabelas. Onde não houver correspondência, o \texttt{R} preenche os
  valores faltantes com \texttt{NA}. Utilize para garantir que nenhum
  dado seja perdido de nenhum dos lados, ideal para comparar cadastros
  discrepantes.
\end{itemize}

\begin{Shaded}
\begin{Highlighting}[]
\NormalTok{presenca\_total }\OtherTok{\textless{}{-}} \FunctionTok{full\_join}\NormalTok{(dia\_1, dia\_2, }\AttributeTok{by =} \StringTok{"nome\_aluno"}\NormalTok{)}
\end{Highlighting}
\end{Shaded}

\begin{itemize}
\tightlist
\item
  \texttt{anti\_join(x,\ y)}: Retorna linhas de \texttt{x} que NÃO têm
  correspondência em \texttt{y}. Essencial para identificar
  inconsistências entre duas bases de dados.
\end{itemize}

\begin{Shaded}
\begin{Highlighting}[]
\CommentTok{\# Encontra produtos cadastrados que nunca foram vendidos}
\NormalTok{produtos\_encalhados }\OtherTok{\textless{}{-}} \FunctionTok{anti\_join}\NormalTok{(produtos, vendas, }\AttributeTok{by =} \StringTok{"id\_produto"}\NormalTok{)}
\end{Highlighting}
\end{Shaded}

\begin{itemize}
\tightlist
\item
  \texttt{pivot\_longer(df,\ cols,\ ...)}: Transforma colunas em linhas
  (formato longo). Útil quando variáveis estão espalhadas no cabeçalho
  (ex: anos 2000, 2001, 2002). Ist é, converte tabela ``larga'' para
  ``longa'', empilhando cabeçalhos de colunas em uma única variável
  categórica e seus valores em outra ( Tabela~\ref{tbl-pivot}) .
\end{itemize}

\begin{Shaded}
\begin{Highlighting}[]
\CommentTok{\# Transforma colunas de anos (2000 a 2010) em: coluna \textquotesingle{}ano\textquotesingle{} e coluna \textquotesingle{}pib\textquotesingle{}}
\NormalTok{df\_longo }\OtherTok{\textless{}{-}} \FunctionTok{pivot\_longer}\NormalTok{(pib\_paises, }\AttributeTok{cols =} \StringTok{\textasciigrave{}}\AttributeTok{2000}\StringTok{\textasciigrave{}}\SpecialCharTok{:}\StringTok{\textasciigrave{}}\AttributeTok{2010}\StringTok{\textasciigrave{}}\NormalTok{, }\AttributeTok{names\_to =} \StringTok{"ano"}\NormalTok{, }\AttributeTok{values\_to =} \StringTok{"pib"}\NormalTok{)}
\end{Highlighting}
\end{Shaded}

\begin{itemize}
\tightlist
\item
  \texttt{pivot\_wider(df,\ names\_from,\ values\_from)}: Transforma
  linhas em colunas (formato largo). Isso é, é o inverso do anterior
  (\texttt{pivot\_longer(df,\ cols,\ ...)}); expande categorias de uma
  coluna em múltiplas colunas novas, preenchendo com valores associados.
  Utilize para criar tabelas de resumo final (pivot tables) legíveis
  para humanos ou relatórios em Excel ( Tabela~\ref{tbl-pivot}) .
\end{itemize}

\begin{Shaded}
\begin{Highlighting}[]
\CommentTok{\# Transforma a coluna \textquotesingle{}tipo\_despesa\textquotesingle{} em várias colunas (Aluguel, Comida, etc)}
\NormalTok{df\_largo }\OtherTok{\textless{}{-}} \FunctionTok{pivot\_wider}\NormalTok{(financas, }\AttributeTok{names\_from =}\NormalTok{ tipo\_despesa, }\AttributeTok{values\_from =}\NormalTok{ valor)}
\end{Highlighting}
\end{Shaded}

\begin{Shaded}
\begin{Highlighting}[]
\NormalTok{pacman}\SpecialCharTok{::}\FunctionTok{p\_load}\NormalTok{(gt,dplyr,tidyr)}

\NormalTok{dados\_largo }\OtherTok{\textless{}{-}} \FunctionTok{data.frame}\NormalTok{(}\AttributeTok{pais =} \StringTok{"Brasil"}\NormalTok{, }\AttributeTok{ano2020 =} \DecValTok{10}\NormalTok{, }\AttributeTok{ano2021 =} \DecValTok{12}\NormalTok{)}

\CommentTok{\# Tabela 1 (Esquerda)}
\NormalTok{dados\_largo }\SpecialCharTok{\%\textgreater{}\%}
  \FunctionTok{gt}\NormalTok{()}
\CommentTok{\# Transformação}
\NormalTok{dados\_longo }\OtherTok{\textless{}{-}}\NormalTok{ dados\_largo }\SpecialCharTok{\%\textgreater{}\%}
  \FunctionTok{pivot\_longer}\NormalTok{(}
    \AttributeTok{cols =} \FunctionTok{c}\NormalTok{(ano2020, ano2021),}
    \AttributeTok{names\_to =} \StringTok{"ano"}\NormalTok{,}
    \AttributeTok{values\_to =} \StringTok{"pib"}
\NormalTok{  )}

\CommentTok{\# Tabela 2 (Direita)}
\NormalTok{dados\_longo }\SpecialCharTok{\%\textgreater{}\%}
  \FunctionTok{gt}\NormalTok{()}
\end{Highlighting}
\end{Shaded}

\begin{table}

\caption{\label{tbl-pivot}Transformação de dados com pivot\_longer}

\begin{minipage}{0.50\linewidth}

\subcaption{\label{tbl-pivot-1}Dados Largos (Original)}

\centering{

\fontsize{12.0pt}{14.0pt}\selectfont
\begin{tabular*}{\linewidth}{@{\extracolsep{\fill}}lrr}
\toprule
pais & ano2020 & ano2021 \\ 
\midrule\addlinespace[2.5pt]
Brasil & 10 & 12 \\ 
\bottomrule
\end{tabular*}

}

\end{minipage}%
%
\begin{minipage}{0.50\linewidth}

\subcaption{\label{tbl-pivot-2}Dados Longos (Resultado)}

\centering{

\fontsize{12.0pt}{14.0pt}\selectfont
\begin{tabular*}{\linewidth}{@{\extracolsep{\fill}}llr}
\toprule
pais & ano & pib \\ 
\midrule\addlinespace[2.5pt]
Brasil & ano2020 & 10 \\ 
Brasil & ano2021 & 12 \\ 
\bottomrule
\end{tabular*}

}

\end{minipage}%

\end{table}%

\begin{itemize}
\tightlist
\item
  \texttt{separate(df,\ col,\ ...)}: Divide uma coluna de texto em
  múltiplas novas colunas usando um caractere separador (ponto, traço,
  barra). Útil para quebrar datas (2024-12-01), nomes completos ou
  códigos compostos (BR-SP-01) em componentes individuais.
\end{itemize}

\begin{Shaded}
\begin{Highlighting}[]
\CommentTok{\# Divide "2024{-}12{-}01" em três colunas: ano, mes, dia}
\NormalTok{df\_limpo }\OtherTok{\textless{}{-}} \FunctionTok{separate}\NormalTok{(df, data\_string, }\AttributeTok{into =} \FunctionTok{c}\NormalTok{(}\StringTok{"ano"}\NormalTok{, }\StringTok{"mes"}\NormalTok{, }\StringTok{"dia"}\NormalTok{), }\AttributeTok{sep =} \StringTok{"{-}"}\NormalTok{)}
\end{Highlighting}
\end{Shaded}

\begin{itemize}
\tightlist
\item
  \texttt{unite(df,\ ...)}: Junta várias colunas em uma string única. É
  operação inversa ao separate; concatena valores de múltiplas colunas
  em uma única string, inserindo um separador. Utilize para criar chaves
  únicas combinando ID e Data, ou juntar Nome e Sobrenome.
\end{itemize}

\begin{Shaded}
\begin{Highlighting}[]
\CommentTok{\# Junta \textquotesingle{}ddd\textquotesingle{} e \textquotesingle{}número\textquotesingle{} para formar \textquotesingle{}telefone\_completo\textquotesingle{}}
\NormalTok{df\_contato }\OtherTok{\textless{}{-}} \FunctionTok{unite}\NormalTok{(df, }\StringTok{"telefone\_completo"}\NormalTok{, ddd, número, }\AttributeTok{sep =} \StringTok{" "}\NormalTok{)}
\end{Highlighting}
\end{Shaded}

\begin{itemize}
\tightlist
\item
  \texttt{any(is.na(.))\ e\ sum(is.na(.))}: Funções lógicas que varrem
  os dados para detectar a existência de algum (any) ou
  quantificar/somar (sum) valores ausentes. Obrigatório na Análise
  Exploratória (EDA) para decidir se você deve remover as linhas
  (\texttt{drop\_na}) ou realizar imputação.
\end{itemize}

\begin{Shaded}
\begin{Highlighting}[]
\CommentTok{\# Verifica se existe algum NA em cada coluna do dataframe}
\NormalTok{iris }\SpecialCharTok{\%\textgreater{}\%} \FunctionTok{summarise}\NormalTok{(}\FunctionTok{across}\NormalTok{(}\FunctionTok{everything}\NormalTok{(), }\SpecialCharTok{\textasciitilde{}}\FunctionTok{any}\NormalTok{(}\FunctionTok{is.na}\NormalTok{(.))))}
\end{Highlighting}
\end{Shaded}

\begin{verbatim}
  Sepal.Length Sepal.Width Petal.Length Petal.Width Species
1        FALSE       FALSE        FALSE       FALSE   FALSE
\end{verbatim}

\begin{Shaded}
\begin{Highlighting}[]
\CommentTok{\# Conta quantos NAs existem em cada coluna}
\NormalTok{iris }\SpecialCharTok{\%\textgreater{}\%} \FunctionTok{summarise}\NormalTok{(}\FunctionTok{across}\NormalTok{(}\FunctionTok{everything}\NormalTok{(), }\SpecialCharTok{\textasciitilde{}}\FunctionTok{sum}\NormalTok{(}\FunctionTok{is.na}\NormalTok{(.))))}
\end{Highlighting}
\end{Shaded}

\begin{verbatim}
  Sepal.Length Sepal.Width Petal.Length Petal.Width Species
1            0           0            0           0       0
\end{verbatim}

\begin{itemize}
\tightlist
\item
  \texttt{drop\_na(df)}: Remove linhas inteiras se houver qualquer valor
  ausente (\texttt{NA}) nas colunas especificadas (ou em todas, se
  nenhuma for citada). Para limpeza rápida de dados onde observações
  incompletas não são úteis para a modelagem estatística.
\end{itemize}

\begin{Shaded}
\begin{Highlighting}[]
\NormalTok{df\_limpo }\OtherTok{\textless{}{-}} \FunctionTok{drop\_na}\NormalTok{(cliente\_df, idade, renda)}
\end{Highlighting}
\end{Shaded}

\begin{itemize}
\tightlist
\item
  \texttt{fill(df)}: Substitui valores \texttt{NA} propagando o último
  valor válido observado anterior/posterior.
\end{itemize}

\begin{Shaded}
\begin{Highlighting}[]
\CommentTok{\# Preenche os NAs da cotação com o valor do dia anterior (down)}
\NormalTok{df\_preenchido }\OtherTok{\textless{}{-}} \FunctionTok{fill}\NormalTok{(acoes, cotacao, }\AttributeTok{.direction =} \StringTok{"down"}\NormalTok{)}
\end{Highlighting}
\end{Shaded}

\begin{itemize}
\tightlist
\item
  \texttt{tidyr::replace\_na()}: Substitui valores ausentes por um valor
  fixo específico (como zero ou ``Desconhecido''). Utilize quando o
  valor ausente tem um significado real (ex: falta de registro de dívida
  significa dívida zero) ou para variáveis categóricas.
\end{itemize}

\begin{Shaded}
\begin{Highlighting}[]
\CommentTok{\# Substitui NA na coluna \textquotesingle{}Species\textquotesingle{} por "Não Identificada"}
\NormalTok{df\_limpo }\OtherTok{\textless{}{-}}\NormalTok{ df }\SpecialCharTok{\%\textgreater{}\%} 
  \FunctionTok{mutate}\NormalTok{(}\AttributeTok{Species =} \FunctionTok{replace\_na}\NormalTok{(Species, }\StringTok{"Não Identificada"}\NormalTok{))}
\end{Highlighting}
\end{Shaded}

\begin{Shaded}
\begin{Highlighting}[]
\CommentTok{\# Substitui NAs da coluna \textquotesingle{}Sepal.Length\textquotesingle{} pela sua média}
\NormalTok{df\_imputado }\OtherTok{\textless{}{-}}\NormalTok{ iris }\SpecialCharTok{\%\textgreater{}\%} 
  \FunctionTok{mutate}\NormalTok{(}\AttributeTok{Sepal.Length =} \FunctionTok{if\_else}\NormalTok{(}\FunctionTok{is.na}\NormalTok{(Sepal.Length), }
                                \FunctionTok{mean}\NormalTok{(Sepal.Length, }\AttributeTok{na.rm =} \ConstantTok{TRUE}\NormalTok{), }
\NormalTok{                                Sepal.Length))}
\end{Highlighting}
\end{Shaded}

\begin{itemize}
\tightlist
\item
  \texttt{mice::mice()}: \emph{Multivariate Imputation by Chained
  Equations.} Cria múltiplos datasets completos estimando os valores
  faltantes com base nas correlações com outras variáveis (regressão,
  florestas aleatórias, etc.). Utilize quando a imputação pela média
  introduziria viés nos seus modelos e a eliminação dos valores ausentes
  faria-lhe perder muita informação.
\end{itemize}

\begin{Shaded}
\begin{Highlighting}[]
\CommentTok{\# Cria 5 datasets com dados imputados usando método padrão (pmm)}
\NormalTok{dados\_imputados }\OtherTok{\textless{}{-}} \FunctionTok{mice}\NormalTok{(iris, }\AttributeTok{m =} \DecValTok{5}\NormalTok{, }\AttributeTok{method =} \StringTok{\textquotesingle{}pmm\textquotesingle{}}\NormalTok{, }\AttributeTok{printFlag =} \ConstantTok{FALSE}\NormalTok{)}

\CommentTok{\# Completa o dataset final (pega o primeiro dos 5 gerados)}
\NormalTok{df\_final }\OtherTok{\textless{}{-}} \FunctionTok{complete}\NormalTok{(dados\_imputados, }\DecValTok{1}\NormalTok{)}
\end{Highlighting}
\end{Shaded}

\section{\texorpdfstring{Manipulação de Strings
(\texttt{stringr})}{Manipulação de Strings (stringr)}}\label{manipulauxe7uxe3o-de-strings-stringr}

\begin{itemize}
\tightlist
\item
  \texttt{paste(...,\ sep),\ paste0()}: \texttt{paste} concatena/junta
  vetores de \texttt{strings} usando um separador especificado;
  \texttt{paste0} é um atalho que concatena sem separador. Utilize para
  criar chaves compostas, frases dinâmicas ou combinar colunas (ex: Nome
  + Sobrenome).
\end{itemize}

\begin{Shaded}
\begin{Highlighting}[]
\CommentTok{\# paste junta com espaço padrão; paste0 cola tudo junto}
\NormalTok{nome\_comp }\OtherTok{\textless{}{-}} \FunctionTok{paste}\NormalTok{(}\StringTok{"João"}\NormalTok{, }\StringTok{"Silva"}\NormalTok{, }\AttributeTok{sep =} \StringTok{"\_"}\NormalTok{) }\CommentTok{\# resultado "João\_Silva"}
\NormalTok{cod\_id }\OtherTok{\textless{}{-}} \FunctionTok{paste0}\NormalTok{(}\StringTok{"ID"}\NormalTok{, }\DecValTok{123}\NormalTok{)                    }\CommentTok{\# \# resultado "ID123"}
\end{Highlighting}
\end{Shaded}

\begin{itemize}
\tightlist
\item
  \texttt{str\_detect(string,\ pattern)}: Retorna um vetor lógico
  (\texttt{TRUE/FALSE}) indicando se um padrão (texto fixo ou Regex)
  existe na string. Essencial dentro de um \texttt{filter()} para
  selecionar linhas que contenham termos específicos (ex: emails que
  contêm \texttt{@gmail}).
\end{itemize}

\begin{Shaded}
\begin{Highlighting}[]
\CommentTok{\# Filtra apenas frutas que terminam com "a" (regex $)}
\NormalTok{frutas\_a }\OtherTok{\textless{}{-}}\NormalTok{ frutas }\SpecialCharTok{\%\textgreater{}\%} \FunctionTok{filter}\NormalTok{(}\FunctionTok{str\_detect}\NormalTok{(nome, }\StringTok{"a$"}\NormalTok{))}
\end{Highlighting}
\end{Shaded}

\begin{itemize}
\tightlist
\item
  \texttt{str\_replace(string,\ pattern,\ replacement),\ str\_replace\_all()}:
  Substitui ocorrências de um padrão (\texttt{pattern}) por um novo
  texto. \texttt{str\_replace} altera apenas a primeira ocorrência
  encontrada; \texttt{str\_replace\_all} altera todas. Utilize para
  limpeza de dados, como remover símbolos de moeda, corrigir erros de
  digitação recorrentes ou padronizar nomes.
\end{itemize}

\begin{Shaded}
\begin{Highlighting}[]
\CommentTok{\# Remove o cifrão e vírgulas para converter em número depois}
\NormalTok{valor\_limpo }\OtherTok{\textless{}{-}} \FunctionTok{str\_replace\_all}\NormalTok{(}\StringTok{"R$ 1.200,00"}\NormalTok{, }\StringTok{"[R$.]"}\NormalTok{, }\StringTok{""}\NormalTok{)}
\end{Highlighting}
\end{Shaded}

\begin{itemize}
\tightlist
\item
  \texttt{str\_sub(string,\ start,\ end)}: Extrai ou substitui partes de
  uma \texttt{string} baseando-se em posições de índices (início e fim).
  Utilize quando os dados têm posição fixa, como extrair o DDD de um
  telefone (caracteres 1 e 2) ou o ano de uma data sem separadores.
\end{itemize}

\begin{Shaded}
\begin{Highlighting}[]
\CommentTok{\# Pega os 3 primeiros caracteres da string}
\NormalTok{prefixo }\OtherTok{\textless{}{-}} \FunctionTok{str\_sub}\NormalTok{(}\StringTok{"São Paulo"}\NormalTok{, }\AttributeTok{start =} \DecValTok{1}\NormalTok{, }\AttributeTok{end =} \DecValTok{3}\NormalTok{) }\CommentTok{\# resultado"São"}
\end{Highlighting}
\end{Shaded}

\begin{itemize}
\tightlist
\item
  \texttt{str\_extract()\ /\ str\_extract\_all()}: Extrai o texto real
  que corresponde a um padrão (Regex), ignorando o resto da
  \texttt{string}. Utilize para ``pescar'' informações específicas
  dentro de um texto sujo, como extrair apenas números de um endereço ou
  apenas o domínio de um email.
\end{itemize}

\begin{Shaded}
\begin{Highlighting}[]
\CommentTok{\# Extrai apenas a sequência de dígitos da string}
\NormalTok{número }\OtherTok{\textless{}{-}} \FunctionTok{str\_extract}\NormalTok{(}\StringTok{"Pedido número 4502 enviado"}\NormalTok{, }\StringTok{"}\SpecialCharTok{\textbackslash{}\textbackslash{}}\StringTok{d+"}\NormalTok{) }\CommentTok{\#resultado "4502"}
\end{Highlighting}
\end{Shaded}

\begin{itemize}
\tightlist
\item
  \texttt{str\_to\_lower()\ /\ str\_to\_upper()}: Converte todo o texto
  para minúsculas (\emph{lower}) ou maiúsculas (\emph{upper}). Passo
  obrigatório antes de fazer junção ou comparações de texto para evitar
  que ``Brasil'' seja diferente de ``brasil''.
\end{itemize}

\begin{Shaded}
\begin{Highlighting}[]
\CommentTok{\# Normaliza os nomes para evitar duplicidade de caixa}
\NormalTok{nomes\_norm }\OtherTok{\textless{}{-}} \FunctionTok{str\_to\_lower}\NormalTok{(}\FunctionTok{c}\NormalTok{(}\StringTok{"Ana"}\NormalTok{, }\StringTok{"ANA"}\NormalTok{, }\StringTok{"ana"}\NormalTok{))}
\end{Highlighting}
\end{Shaded}

\begin{itemize}
\tightlist
\item
  \texttt{str\_squish()}: Remove espaços em branco no início e no fim da
  \texttt{string}, e também reduz múltiplos espaços internos
  consecutivos a um único espaço. Muito superior ao \texttt{str\_trim}
  para limpar dados digitados por humanos, removendo acidentes como
  ``Nome Sobrenome''.
\end{itemize}

\begin{Shaded}
\begin{Highlighting}[]
\CommentTok{\# Transforma "  Data   Science  " em "Data Science"}
\NormalTok{texto\_limpo }\OtherTok{\textless{}{-}} \FunctionTok{str\_squish}\NormalTok{(}\StringTok{"  Data   Science  "}\NormalTok{)}
\end{Highlighting}
\end{Shaded}

\begin{itemize}
\tightlist
\item
  \texttt{str\_glue()}: Uma evolução moderna do \texttt{paste}, permite
  inserir variáveis diretamente dentro da \texttt{string} usando chaves
  \texttt{\{\}}. Utilize para tornar o código mais legível ao criar
  mensagens de log, títulos de gráficos ou URLs dinâmicas.
\end{itemize}

\begin{Shaded}
\begin{Highlighting}[]
\NormalTok{nome }\OtherTok{\textless{}{-}} \StringTok{"Maria"}\NormalTok{; idade }\OtherTok{\textless{}{-}} \DecValTok{30}
\NormalTok{msg }\OtherTok{\textless{}{-}} \FunctionTok{str\_glue}\NormalTok{(}\StringTok{"A aluna \{nome\} tem \{idade\} anos."}\NormalTok{) }\CommentTok{\# Retorna "A aluna Maria tem 30 anos}
\end{Highlighting}
\end{Shaded}

\section{\texorpdfstring{Datas
(\texttt{lubridate})}{Datas (lubridate)}}\label{datas-lubridate}

\begin{itemize}
\tightlist
\item
  \texttt{ymd()}, \texttt{dmy()}, \texttt{mdy()}: Funções que convertem
  texto em objetos Date. O nome da função dita a ordem esperada dos
  componentes \texttt{(y=ano,\ m=mês,\ d=dia)}. Utilize para transformar
  colunas de texto (importadas de CSVs/Excel) em datas reais,
  independentemente do separador (barra, traço, ponto) usado no texto
  original.
\end{itemize}

\begin{Shaded}
\begin{Highlighting}[]
\NormalTok{pacman}\SpecialCharTok{::}\FunctionTok{p\_load}\NormalTok{(lubridate)}
\CommentTok{\# Converte texto "2023/12/25" ou "2023{-}12{-}25" para data}
\NormalTok{natal }\OtherTok{\textless{}{-}} \FunctionTok{ymd}\NormalTok{(}\StringTok{"20231225"}\NormalTok{) }
\NormalTok{data\_br }\OtherTok{\textless{}{-}} \FunctionTok{dmy}\NormalTok{(}\StringTok{"31/01/2024"}\NormalTok{) }\CommentTok{\# Dia, Mês, Ano}
\end{Highlighting}
\end{Shaded}

\begin{itemize}
\tightlist
\item
  \texttt{years(),\ months(),\ days()}: Criam objetos de ``Período'' que
  permitem aritmética intuitiva com datas, lidando automaticamente com
  nuances como anos bissextos ou meses com 30/31 dias. Utilize para
  calcular datas de vencimento, projetar cenários futuros ou filtrar
  dados de um período específico (``últimos 3 meses'').
\end{itemize}

\begin{Shaded}
\begin{Highlighting}[]
\CommentTok{\# Adiciona 1 ano e 6 meses a uma data}
\NormalTok{vencimento }\OtherTok{\textless{}{-}} \FunctionTok{ymd}\NormalTok{(}\StringTok{"2024{-}01{-}01"}\NormalTok{) }\SpecialCharTok{+} \FunctionTok{years}\NormalTok{(}\DecValTok{1}\NormalTok{) }\SpecialCharTok{+} \FunctionTok{months}\NormalTok{(}\DecValTok{6}\NormalTok{)}
\end{Highlighting}
\end{Shaded}

\begin{itemize}
\tightlist
\item
  \texttt{ymd\_hms()\ (e\ suas\ variantes)}: Similar ao \texttt{ymd()},
  mas para dados que incluem horário
  (\texttt{timestamp:\ Hora,\ Minuto,\ Segundo}). Cria objetos
  \texttt{POSIXct}. Essencial para analisar logs de servidor, transações
  financeiras \texttt{intraday} ou qualquer dado onde a hora exata
  importa.
\end{itemize}

\begin{Shaded}
\begin{Highlighting}[]
\CommentTok{\# Lê data e hora com fuso horário UTC}
\NormalTok{momento\_exato }\OtherTok{\textless{}{-}} \FunctionTok{ymd\_hms}\NormalTok{(}\StringTok{"2024{-}05{-}10 14:30:59"}\NormalTok{, }\AttributeTok{tz =} \StringTok{"UTC"}\NormalTok{)}
\end{Highlighting}
\end{Shaded}

\begin{itemize}
\tightlist
\item
  \texttt{year(),\ month(),\ wday()}: Funções que retornam apenas uma
  parte específica da data (o ano, o mês ou o dia da semana).
  Fundamental para análise de sazonalidade (ex: ``vendas por dia da
  semana'' ou ``evolução anual''). O argumento label = TRUE em
  \texttt{wday} retorna o nome (Dom, Seg\ldots) em vez do número.
\end{itemize}

\begin{Shaded}
\begin{Highlighting}[]
\NormalTok{data }\OtherTok{\textless{}{-}} \FunctionTok{ymd}\NormalTok{(}\StringTok{"2024{-}12{-}25"}\NormalTok{)}
\NormalTok{mes }\OtherTok{\textless{}{-}} \FunctionTok{month}\NormalTok{(data)         }\CommentTok{\# Retorna 12}
\NormalTok{dia\_sem }\OtherTok{\textless{}{-}} \FunctionTok{wday}\NormalTok{(data, }\AttributeTok{label =} \ConstantTok{TRUE}\NormalTok{) }\CommentTok{\# Retorna "Wed" (ou "qua" dependendo do locale)}
\end{Highlighting}
\end{Shaded}

\begin{itemize}
\tightlist
\item
  \texttt{floor\_date()}: Arredonda uma data para baixo até a unidade de
  tempo especificada (semana, mês, hora). A função mais importante para
  agrupar séries temporais. Use para transformar dados diários em
  mensais/semanais antes de um \texttt{group\_by}.
\end{itemize}

\begin{Shaded}
\begin{Highlighting}[]
\CommentTok{\# Agrupa todas as datas para o primeiro dia do respectivo mês}
\NormalTok{vendas }\SpecialCharTok{\%\textgreater{}\%} 
  \FunctionTok{mutate}\NormalTok{(}\AttributeTok{mes\_referencia =} \FunctionTok{floor\_date}\NormalTok{(data\_venda, }\AttributeTok{unit =} \StringTok{"month"}\NormalTok{))}
\end{Highlighting}
\end{Shaded}

\begin{itemize}
\tightlist
\item
  \texttt{today()\ /\ now()}: Retornam, respectivamente, a data atual
  (\texttt{Date}) e o instante atual exato (\texttt{POSIXct}) do
  sistema. Utilize para calcular a idade de registros (``dias desde o
  cadastro até hoje'') ou para carimbar data de execução em relatórios.
\end{itemize}

\begin{Shaded}
\begin{Highlighting}[]
\CommentTok{\# Calcula a diferença em dias entre hoje e uma data passada}
\NormalTok{dias\_passados }\OtherTok{\textless{}{-}} \FunctionTok{today}\NormalTok{() }\SpecialCharTok{{-}} \FunctionTok{ymd}\NormalTok{(}\StringTok{"2000{-}01{-}01"}\NormalTok{)}
\end{Highlighting}
\end{Shaded}

\begin{itemize}
\tightlist
\item
  \texttt{interval()\ e\ \%-\/-\%}: Cria um objeto de intervalo entre
  duas datas, permitindo cálculos precisos de duração. Use para
  verificar se uma data cai dentro de um período específico ou para
  calcular a duração exata em segundos/anos entre dois pontos.
\end{itemize}

\begin{Shaded}
\begin{Highlighting}[]
\NormalTok{inicio }\OtherTok{\textless{}{-}} \FunctionTok{ymd}\NormalTok{(}\StringTok{"2023{-}01{-}01"}\NormalTok{)}
\NormalTok{fim }\OtherTok{\textless{}{-}} \FunctionTok{today}\NormalTok{()}
\NormalTok{meu\_intervalo }\OtherTok{\textless{}{-}} \FunctionTok{interval}\NormalTok{(inicio, fim) }\CommentTok{\# ou inicio \%{-}{-}\% fim}

\CommentTok{\# Verifica se a data X está dentro do intervalo}
\FunctionTok{ymd}\NormalTok{(}\StringTok{"2023{-}06{-}01"}\NormalTok{) }\SpecialCharTok{\%within\%}\NormalTok{ meu\_intervalo}
\end{Highlighting}
\end{Shaded}

\section{Funções Matemáticas
Básicas}\label{funuxe7uxf5es-matemuxe1ticas-buxe1sicas}

\begin{itemize}
\tightlist
\item
  \texttt{sum()\ /\ prod()}: Calculam, respectivamente, o somatório
  \((\sum)\) e o produtório \((\prod)\) de todos os valores de um vetor
  numérico. Utilize para agregações básicas, verificações de totais (ex:
  receita total) ou cálculos de probabilidade conjunta (produtório).
\end{itemize}

\begin{Shaded}
\begin{Highlighting}[]
\NormalTok{total }\OtherTok{\textless{}{-}} \FunctionTok{sum}\NormalTok{(}\FunctionTok{c}\NormalTok{(}\DecValTok{10}\NormalTok{, }\DecValTok{20}\NormalTok{, }\DecValTok{30}\NormalTok{))  }\CommentTok{\# 60}
\NormalTok{fatorial\_simples }\OtherTok{\textless{}{-}} \FunctionTok{prod}\NormalTok{(}\DecValTok{1}\SpecialCharTok{:}\DecValTok{5}\NormalTok{) }\CommentTok{\# 1 * 2 * 3 * 4 * 5 = 120}
\end{Highlighting}
\end{Shaded}

\begin{itemize}
\tightlist
\item
  \texttt{min()\ /\ max()}: Retornam o valor mais baixo e o mais alto de
  um conjunto de dados. Utilize para identificar limites (inferiores e
  superiores), detectar \emph{outliers} óbvios ou definir escalas de
  gráficos.
\end{itemize}

\begin{Shaded}
\begin{Highlighting}[]
\CommentTok{\# Encontra o valor máximo ignorando NAs}
\NormalTok{maior\_nota }\OtherTok{\textless{}{-}} \FunctionTok{max}\NormalTok{(}\FunctionTok{c}\NormalTok{(}\DecValTok{5}\NormalTok{, }\DecValTok{8}\NormalTok{, }\DecValTok{9}\NormalTok{, }\ConstantTok{NA}\NormalTok{), }\AttributeTok{na.rm =} \ConstantTok{TRUE}\NormalTok{)}
\end{Highlighting}
\end{Shaded}

\begin{itemize}
\tightlist
\item
  \texttt{range()}: Retorna um vetor de dois elementos contendo o mínimo
  e o máximo: \texttt{c(min,\ max)}. \textbf{Não} retorna a amplitude
  (diferença) diretamente. Utilize para verificar rapidamente a extensão
  dos dados ou definir os limites (\texttt{limits}) de eixos em
  gráficos.
\end{itemize}

\begin{Shaded}
\begin{Highlighting}[]
\NormalTok{limites }\OtherTok{\textless{}{-}} \FunctionTok{range}\NormalTok{(iris}\SpecialCharTok{$}\NormalTok{Sepal.Length) }\CommentTok{\# Retorna c(4.3, 7.9)}
\NormalTok{amplitude }\OtherTok{\textless{}{-}} \FunctionTok{diff}\NormalTok{(limites)          }\CommentTok{\# Calcula a diferença (3.6)}
\end{Highlighting}
\end{Shaded}

\begin{itemize}
\tightlist
\item
  \texttt{round()}: Arredonda números para um número especificado de
  casas decimais seguindo o padrão internacional (arredonda para o par
  mais próximo). Utilize para formatar saídas para relatórios ou
  simplificar a visualização de números com muitas casas decimais.
\end{itemize}

\begin{Shaded}
\begin{Highlighting}[]
\CommentTok{\# Arredonda para 2 casas}
\NormalTok{pi\_curto }\OtherTok{\textless{}{-}} \FunctionTok{round}\NormalTok{(}\FloatTok{3.14159}\NormalTok{, }\AttributeTok{digits =} \DecValTok{2}\NormalTok{) }\CommentTok{\# 3.14}
\end{Highlighting}
\end{Shaded}

\begin{itemize}
\tightlist
\item
  \texttt{abs()}: Retorna o valor absoluto (módulo) de um número,
  ignorando o sinal negativo (\(|x|\)). Utilize para calcular
  distâncias, erros absolutos (diferença entre previsto e real) ou
  magnitudes.
\end{itemize}

\begin{Shaded}
\begin{Highlighting}[]
\CommentTok{\# Transforma diferenças negativas em positivas}
\NormalTok{erro\_absoluto }\OtherTok{\textless{}{-}} \FunctionTok{abs}\NormalTok{(}\SpecialCharTok{{-}}\DecValTok{50}\NormalTok{) }\CommentTok{\# 50}
\end{Highlighting}
\end{Shaded}

\begin{itemize}
\tightlist
\item
  \texttt{log()\ /\ exp()}: \texttt{log} calcula o logaritmo natural
  (\(\ln\), base \(e\)) e \texttt{exp} calcula a exponencial (\(e^x\)).
  Para base 10, use \texttt{log10()}. Utilize para transformar dados
  assimétricos (normalizar distribuição), calcular retornos financeiros
  ou reverter transformações logarítmicas.
\end{itemize}

\begin{Shaded}
\begin{Highlighting}[]
\CommentTok{\# Transformação Log para reduzir assimetria de salários}
\NormalTok{log\_salario }\OtherTok{\textless{}{-}} \FunctionTok{log}\NormalTok{(}\FunctionTok{c}\NormalTok{(}\DecValTok{1000}\NormalTok{, }\DecValTok{10000}\NormalTok{, }\DecValTok{100000}\NormalTok{)) }
\end{Highlighting}
\end{Shaded}

\begin{itemize}
\tightlist
\item
  \texttt{cumsum()}: Retorna a soma acumulada dos elementos. O resultado
  tem o mesmo tamanho do vetor original. Essencial para criar Gráficos
  de Pareto, calcular saldos bancários dia a dia ou frequências
  acumuladas.
\end{itemize}

\begin{Shaded}
\begin{Highlighting}[]
\NormalTok{vendas\_diarias }\OtherTok{\textless{}{-}} \FunctionTok{c}\NormalTok{(}\DecValTok{10}\NormalTok{, }\DecValTok{20}\NormalTok{, }\DecValTok{5}\NormalTok{)}
\NormalTok{acumulado }\OtherTok{\textless{}{-}} \FunctionTok{cumsum}\NormalTok{(vendas\_diarias) }\CommentTok{\# Retorna 10, 30, 35}
\end{Highlighting}
\end{Shaded}

\section{Estatística Descritiva}\label{estatuxedstica-descritiva}

\begin{itemize}
\tightlist
\item
  \texttt{mean()\ /\ median()}: Calculam a média aritmética (centro de
  gravidade) e a mediana (valor central que divide a amostra em 50/50).
  Use a média para distribuições normais e a mediana quando houver
  \emph{outliers} (valores extremos) que distorcem a média.
\end{itemize}

\begin{Shaded}
\begin{Highlighting}[]
\NormalTok{salarios }\OtherTok{\textless{}{-}} \FunctionTok{c}\NormalTok{(}\DecValTok{1000}\NormalTok{, }\DecValTok{1200}\NormalTok{, }\DecValTok{50000}\NormalTok{) }\CommentTok{\# O 50000 distorce a média}
\NormalTok{media }\OtherTok{\textless{}{-}} \FunctionTok{mean}\NormalTok{(salarios)   }\CommentTok{\# 17400 (Distorcida)}
\NormalTok{mediana }\OtherTok{\textless{}{-}} \FunctionTok{median}\NormalTok{(salarios) }\CommentTok{\# 1200 (Representativa)}
\end{Highlighting}
\end{Shaded}

\begin{itemize}
\tightlist
\item
  \texttt{sd()\ /\ var()}: Calculam o desvio padrão e a variância
  amostral (\(n-1\)), medidas de dispersão que indicam o quanto os dados
  variam em torno da média. Utilize para quantificar o risco,
  volatilidade ou a consistência de um processo.
\end{itemize}

\begin{Shaded}
\begin{Highlighting}[]
\CommentTok{\# Calcula o desvio padrão ignorando falhas}
\NormalTok{volatilidade }\OtherTok{\textless{}{-}} \FunctionTok{sd}\NormalTok{(}\FunctionTok{c}\NormalTok{(}\DecValTok{10}\NormalTok{, }\DecValTok{12}\NormalTok{, }\DecValTok{9}\NormalTok{, }\ConstantTok{NA}\NormalTok{), }\AttributeTok{na.rm =} \ConstantTok{TRUE}\NormalTok{)}
\end{Highlighting}
\end{Shaded}

\begin{itemize}
\tightlist
\item
  \texttt{quantile()}: Divide os dados ordenados em probabilidades
  específicas. Por padrão retorna mín, 25\%, 50\%, 75\%, máx. Utilize
  para entender a distribuição detalhada, criar Boxplots manuais ou
  identificar faixas de valores (ex: ``os 10\% mais ricos'').
\end{itemize}

\begin{Shaded}
\begin{Highlighting}[]
\CommentTok{\# Calcula os decis (10\%, 20\%... 90\%)}
\NormalTok{decis }\OtherTok{\textless{}{-}} \FunctionTok{quantile}\NormalTok{(iris}\SpecialCharTok{$}\NormalTok{Sepal.Length, }\AttributeTok{probs =} \FunctionTok{seq}\NormalTok{(}\DecValTok{0}\NormalTok{, }\DecValTok{1}\NormalTok{, }\FloatTok{0.1}\NormalTok{))}
\end{Highlighting}
\end{Shaded}

\begin{itemize}
\tightlist
\item
  \texttt{cor()}: Calcula a força e direção da relação linear entre duas
  variáveis (vai de -1 a 1). Utilize na análise exploratória para checar
  multicolinearidade ou se uma variável influencia a outra.
\end{itemize}

\begin{Shaded}
\begin{Highlighting}[]
\NormalTok{correlacao }\OtherTok{\textless{}{-}} \FunctionTok{cor}\NormalTok{(iris}\SpecialCharTok{$}\NormalTok{Sepal.Length, iris}\SpecialCharTok{$}\NormalTok{Petal.Length, }\AttributeTok{use =} \StringTok{"complete.obs"}\NormalTok{)}
\end{Highlighting}
\end{Shaded}

\begin{itemize}
\tightlist
\item
  \texttt{summary()}: Uma função polimórfica que retorna um ``raio-x''
  do objeto. Para vetores numéricos, dá as medidas descritivas; para
  fatores, a contagem. O primeiro comando a rodar ao receber dados novos
  para ter uma visão geral rápida da distribuição e identificar
  \texttt{NAs}.
\end{itemize}

\begin{Shaded}
\begin{Highlighting}[]
\FunctionTok{summary}\NormalTok{(iris) }\CommentTok{\# Resumo de todas as colunas do dataset}
\end{Highlighting}
\end{Shaded}

\begin{itemize}
\tightlist
\item
  \texttt{table()\ /\ prop.table()}: \texttt{table} cria tabelas de
  frequência (contagem) para dados categóricos; \texttt{prop.table}
  converte essas contagens em porcentagens/proporções. Fundamental para
  analisar variáveis qualitativas (ex: ``Quantos clientes são de SP vs
  RJ?'').
\end{itemize}

\begin{Shaded}
\begin{Highlighting}[]
\NormalTok{contagem }\OtherTok{\textless{}{-}} \FunctionTok{table}\NormalTok{(iris}\SpecialCharTok{$}\NormalTok{Species)}
\NormalTok{porcentagem }\OtherTok{\textless{}{-}} \FunctionTok{prop.table}\NormalTok{(contagem) }\SpecialCharTok{*} \DecValTok{100} \CommentTok{\# Em \%}
\end{Highlighting}
\end{Shaded}

\begin{itemize}
\tightlist
\item
  \texttt{scale()}: Padroniza (normaliza) os dados subtraindo a média e
  dividindo pelo desvio padrão (Z-score),
  \(z=\frac{x-\bar{x}}{\sigma}\). Obrigatório antes de algoritmos de
  Machine Learning baseados em distância (como K-means ou KNN) para que
  variáveis grandes não dominem as pequenas.
\end{itemize}

\begin{Shaded}
\begin{Highlighting}[]
\CommentTok{\# Coloca os dados na mesma escala (Média 0, SD 1)}
\NormalTok{dados\_normalizados }\OtherTok{\textless{}{-}} \FunctionTok{scale}\NormalTok{(iris[, }\DecValTok{1}\SpecialCharTok{:}\DecValTok{4}\NormalTok{])}
\end{Highlighting}
\end{Shaded}

\begin{itemize}
\tightlist
\item
  \texttt{unique()\ /\ length()}: \texttt{unique} retorna os valores
  únicos (sem repetição); \texttt{length} conta o tamanho total do
  vetor. Use a combinação \texttt{length(unique(x))} para saber a
  cardinalidade (quantos itens distintos existem).
\end{itemize}

\begin{Shaded}
\begin{Highlighting}[]
\NormalTok{qtd\_especies }\OtherTok{\textless{}{-}} \FunctionTok{length}\NormalTok{(}\FunctionTok{unique}\NormalTok{(iris}\SpecialCharTok{$}\NormalTok{Species)) }\CommentTok{\# 3}
\end{Highlighting}
\end{Shaded}

Aqui estão as definições aprimoradas para distribuições e modelagem
estatística, organizadas para clareza e aplicação prática.

\section{Distribuições de Probabilidade (Prefixos d, p, q,
r)}\label{distribuiuxe7uxf5es-de-probabilidade-prefixos-d-p-q-r}

O R utiliza um sistema consistente:
\texttt{{[}prefixo{]}{[}distribuição{]}}. Ex: \texttt{norm},
\texttt{binom}, \texttt{pois}, \texttt{exp}, \texttt{unif}, \texttt{t},
\texttt{chisq}, \texttt{f}.

\textbf{d\ldots{} (Densidade/Probabilidade Pontual)}: Calcula a altura
da curva da densidade (PDF) para variáveis contínuas, ou a probabilidade
exata \(P(X=x)\) para variáveis discretas. Utilize para desenhar o
gráfico da distribuição ou calcular verossimilhança (Likelihood).

\begin{Shaded}
\begin{Highlighting}[]
\CommentTok{\# Qual a probabilidade exata de obter 2 caras em 3 lançamentos (Binomial)?}
\NormalTok{prob\_exata }\OtherTok{\textless{}{-}} \FunctionTok{dbinom}\NormalTok{(}\AttributeTok{x =} \DecValTok{2}\NormalTok{, }\AttributeTok{size =} \DecValTok{3}\NormalTok{, }\AttributeTok{prob =} \FloatTok{0.5}\NormalTok{) }
\end{Highlighting}
\end{Shaded}

\begin{itemize}
\tightlist
\item
  \texttt{p...\ (Probabilidade\ Acumulada\ -\ CDF)}: Calcula a área sob
  a curva à esquerda de um ponto \(P(X \le x)\). Utilize para calcular
  valores-p ou a probabilidade de uma variável ser menor que um certo
  valor.
\end{itemize}

\begin{Shaded}
\begin{Highlighting}[]
\CommentTok{\# Qual a probabilidade de um valor numa Normal(0,1) ser menor que {-}1.96?}
\NormalTok{prob\_acumulada }\OtherTok{\textless{}{-}} \FunctionTok{pnorm}\NormalTok{(}\SpecialCharTok{{-}}\FloatTok{1.96}\NormalTok{) }\CommentTok{\# \textasciitilde{}0.025 (2.5\%)}
\end{Highlighting}
\end{Shaded}

\begin{itemize}
\tightlist
\item
  \texttt{q...\ (Quantil\ -\ Inverso\ da\ CDF)}: Dado uma probabilidade
  (área), retorna o valor de \(x\) correspondente. Utilize para
  encontrar \emph{valores críticos} para intervalos de confiança (ex: o
  Z para 95\%).
\end{itemize}

\begin{Shaded}
\begin{Highlighting}[]
\CommentTok{\# Qual valor deixa 2.5\% da cauda superior na distribuição T (gl=29)?}
\NormalTok{valor\_critico }\OtherTok{\textless{}{-}} \FunctionTok{qt}\NormalTok{(}\AttributeTok{p =} \FloatTok{0.975}\NormalTok{, }\AttributeTok{df =} \DecValTok{29}\NormalTok{) }
\end{Highlighting}
\end{Shaded}

\begin{itemize}
\tightlist
\item
  \texttt{r...\ (Random\ -\ Geração\ de\ números\ (pseudo)\ Aleatória)}:
  Gera números (pseudo) aleatórios que seguem a distribuição
  especificada. Utilize para simulações de Monte Carlo, criar dados
  sintéticos ou \texttt{bootstrap}.
\end{itemize}

\begin{Shaded}
\begin{Highlighting}[]
\CommentTok{\# Gera 100 observações de uma Poisson com lambda = 5}
\NormalTok{amostra }\OtherTok{\textless{}{-}} \FunctionTok{rpois}\NormalTok{(}\AttributeTok{n =} \DecValTok{100}\NormalTok{, }\AttributeTok{lambda =} \DecValTok{5}\NormalTok{)}
\end{Highlighting}
\end{Shaded}

\section{Testes de Hipótese}\label{testes-de-hipuxf3tese}

\begin{itemize}
\tightlist
\item
  \texttt{t.test()}: Realiza o teste T de Student para comparar médias
  de uma ou duas amostras (independentes ou pareadas). Utilize para
  verificar se há diferença significativa entre dois grupos numéricos
  (ex: tratamento vs controle).
\end{itemize}

\begin{Shaded}
\begin{Highlighting}[]
\NormalTok{pacman}\SpecialCharTok{::}\FunctionTok{p\_load}\NormalTok{(mtcars)}
\CommentTok{\# Compara se há diferença significativa no consumo (mpg) entre carros automáticos e manuais (am) no dataset mtcars.}
\NormalTok{resultado }\OtherTok{\textless{}{-}} \FunctionTok{t.test}\NormalTok{(mpg }\SpecialCharTok{\textasciitilde{}}\NormalTok{ am, }\AttributeTok{data =}\NormalTok{ mtcars)}
\end{Highlighting}
\end{Shaded}

\begin{itemize}
\tightlist
\item
  \texttt{cor.test()}: Testa se a correlação entre duas variáveis é
  significativamente diferente de zero. Utilize para validar
  estatisticamente a associação linear observada com \texttt{cor()}.
\end{itemize}

\begin{Shaded}
\begin{Highlighting}[]
\CommentTok{\# Testa a correlação entre o peso do carro (wt) e a eficiência (mpg). Espera{-}se uma correlação negativa forte.}
\NormalTok{teste\_cor }\OtherTok{\textless{}{-}} \FunctionTok{cor.test}\NormalTok{(mtcars}\SpecialCharTok{$}\NormalTok{wt, mtcars}\SpecialCharTok{$}\NormalTok{mp, }\AttributeTok{method =} \StringTok{"pearson"}\NormalTok{)}
\end{Highlighting}
\end{Shaded}

\begin{itemize}
\tightlist
\item
  \texttt{chisq.test()}: Teste Qui-Quadrado de independência. Utilize
  para verificar associação entre duas variáveis \emph{categóricas} (ex:
  Cor dos Olhos vs Cor do Cabelo).
\end{itemize}

\begin{tcolorbox}[enhanced jigsaw, left=2mm, toptitle=1mm, colback=white, colframe=quarto-callout-important-color-frame, colbacktitle=quarto-callout-important-color!10!white, opacityback=0, rightrule=.15mm, bottomtitle=1mm, arc=.35mm, title=\textcolor{quarto-callout-important-color}{\faExclamation}\hspace{0.5em}{Importante}, titlerule=0mm, bottomrule=.15mm, leftrule=.75mm, coltitle=black, toprule=.15mm, breakable, opacitybacktitle=0.6]

A entrada ideal é uma tabela de contingência (\texttt{table(.)}).

\end{tcolorbox}

\begin{Shaded}
\begin{Highlighting}[]
\CommentTok{\# Verifica se existe associação entre o tipo de motor (vs: em V ou reto) e o tipo de transmissão (am: auto ou manual).}
\NormalTok{teste\_qui }\OtherTok{\textless{}{-}} \FunctionTok{chisq.test}\NormalTok{(}\FunctionTok{table}\NormalTok{(mtcars}\SpecialCharTok{$}\NormalTok{vs, mtcars}\SpecialCharTok{$}\NormalTok{am))}
\end{Highlighting}
\end{Shaded}

\begin{itemize}
\tightlist
\item
  \texttt{shapiro.test()}: Teste de normalidade de \emph{Shapiro-Wilk}
  (\(H_0\): Os dados seguem uma distribuição Normal). Utilize para
  verificar os pressupostos de normalidade dos resíduos de um modelo ou
  de uma variável antes de aplicar testes paramétricos. (Melhor para
  \(N < 5000\)).
\end{itemize}

\begin{Shaded}
\begin{Highlighting}[]
\CommentTok{\# P{-}valor \textless{} 0.05 indica que os dados NÃO são normais}
\NormalTok{teste\_norm }\OtherTok{\textless{}{-}} \FunctionTok{shapiro.test}\NormalTok{(}\FunctionTok{resid}\NormalTok{(modelo\_linear)) }
\CommentTok{\#Verifica se a largura das sépalas (Sepal.Width) das flores iris segue uma distribuição Normal.}
\NormalTok{Shap }\OtherTok{\textless{}{-}} \FunctionTok{shapiro.test}\NormalTok{(iris}\SpecialCharTok{$}\NormalTok{Sepal.Width) }\CommentTok{\# H0: Os dados são normais (p \textgreater{} 0.05 indica normalidade)}
\end{Highlighting}
\end{Shaded}

\begin{itemize}
\tightlist
\item
  \texttt{wilcox.test()}: Versão não-paramétrica do Teste T (Teste de
  Mann-Whitney ou Wilcoxon). Utilize para comparar grupos quando a
  suposição de normalidade falha. Compara postos (ranks) e não médias.
\end{itemize}

\begin{Shaded}
\begin{Highlighting}[]
\CommentTok{\# Comparação sem assumir normalidade}
\NormalTok{teste\_np }\OtherTok{\textless{}{-}} \FunctionTok{wilcox.test}\NormalTok{(valor }\SpecialCharTok{\textasciitilde{}}\NormalTok{ grupo, }\AttributeTok{data =}\NormalTok{ dados)}
\end{Highlighting}
\end{Shaded}

\section{Modelagem Estatística}\label{modelagem-estatuxedstica}

\begin{itemize}
\tightlist
\item
  \texttt{lm()}: Ajusta modelos de regressão linear (Mínimos Quadrados
  Ordinários - OLS). A base para prever uma variável numérica contínua
  baseada em preditores. Use \texttt{summary()} no objeto para ver os
  coeficientes e \(R^2\).
\end{itemize}

\begin{Shaded}
\begin{Highlighting}[]
\CommentTok{\# Modela o consumo de combustível (mpg) baseando{-}se no peso (wt) e na potência (hp) dos carros.}
\NormalTok{modelo\_linear }\OtherTok{\textless{}{-}} \FunctionTok{lm}\NormalTok{(mpg }\SpecialCharTok{\textasciitilde{}}\NormalTok{ wt }\SpecialCharTok{+}\NormalTok{ hp, }\AttributeTok{data =}\NormalTok{ mtcars)}
\FunctionTok{summary}\NormalTok{(modelo\_linear) }\CommentTok{\#saída do modelo, embora não elegante}
\end{Highlighting}
\end{Shaded}

\begin{verbatim}

Call:
lm(formula = mpg ~ wt + hp, data = mtcars)

Residuals:
   Min     1Q Median     3Q    Max 
-3.941 -1.600 -0.182  1.050  5.854 

Coefficients:
            Estimate Std. Error t value Pr(>|t|)    
(Intercept) 37.22727    1.59879  23.285  < 2e-16 ***
wt          -3.87783    0.63273  -6.129 1.12e-06 ***
hp          -0.03177    0.00903  -3.519  0.00145 ** 
---
Signif. codes:  0 '***' 0.001 '**' 0.01 '*' 0.05 '.' 0.1 ' ' 1

Residual standard error: 2.593 on 29 degrees of freedom
Multiple R-squared:  0.8268,    Adjusted R-squared:  0.8148 
F-statistic: 69.21 on 2 and 29 DF,  p-value: 9.109e-12
\end{verbatim}

\begin{tcolorbox}[enhanced jigsaw, left=2mm, toptitle=1mm, colback=white, colframe=quarto-callout-tip-color-frame, colbacktitle=quarto-callout-tip-color!10!white, opacityback=0, rightrule=.15mm, bottomtitle=1mm, arc=.35mm, title=\textcolor{quarto-callout-tip-color}{\faLightbulb}\hspace{0.5em}{Dica}, titlerule=0mm, bottomrule=.15mm, leftrule=.75mm, coltitle=black, toprule=.15mm, breakable, opacitybacktitle=0.6]

Usar \texttt{summary(modelo\_ajustado)} é forma mais simples mas não
profissional para trabalhos acadêmicos. Existem vários pacotes que
permitem uma extração mais profissional, como segue abaixo.

\end{tcolorbox}

\begin{Shaded}
\begin{Highlighting}[]
\NormalTok{pacman}\SpecialCharTok{::}\FunctionTok{p\_load}\NormalTok{(broom, dplyr)}
\CommentTok{\# Extraido os resultados do modelo ajustado acima de forma mais profissional}
\NormalTok{resultados\_tidy }\OtherTok{\textless{}{-}} \FunctionTok{tidy}\NormalTok{(modelo\_linear, }\AttributeTok{conf.int =} \ConstantTok{TRUE}\NormalTok{)}
\FunctionTok{gt}\NormalTok{(resultados\_tidy)}
\end{Highlighting}
\end{Shaded}

\begin{table}
\fontsize{12.0pt}{14.0pt}\selectfont
\begin{tabular*}{\linewidth}{@{\extracolsep{\fill}}lrrrrrr}
\toprule
term & estimate & std.error & statistic & p.value & conf.low & conf.high \\ 
\midrule\addlinespace[2.5pt]
(Intercept) & 37.22727012 & 1.59878754 & 23.284689 & 2.565459e-20 & 33.95738245 & 40.49715778 \\ 
wt & -3.87783074 & 0.63273349 & -6.128695 & 1.119647e-06 & -5.17191604 & -2.58374544 \\ 
hp & -0.03177295 & 0.00902971 & -3.518712 & 1.451229e-03 & -0.05024078 & -0.01330512 \\ 
\bottomrule
\end{tabular*}
\end{table}

\begin{Shaded}
\begin{Highlighting}[]
\CommentTok{\# Agora você pode acessar o p{-}valor assim: resultados\_tidy$p.value[2]}

\CommentTok{\#Resumo do modelo (R\^{}2, AIC, etc)}
\NormalTok{qualidade\_modelo }\OtherTok{\textless{}{-}} \FunctionTok{glance}\NormalTok{(modelo\_linear)}
\FunctionTok{gt}\NormalTok{(qualidade\_modelo)}
\end{Highlighting}
\end{Shaded}

\begin{table}
\fontsize{12.0pt}{14.0pt}\selectfont
\begin{tabular*}{\linewidth}{@{\extracolsep{\fill}}rrrrrrrrrrrr}
\toprule
r.squared & adj.r.squared & sigma & statistic & p.value & df & logLik & AIC & BIC & deviance & df.residual & nobs \\ 
\midrule\addlinespace[2.5pt]
0.8267855 & 0.8148396 & 2.593412 & 69.21121 & 9.109054e-12 & 2 & -74.32617 & 156.6523 & 162.5153 & 195.0478 & 29 & 32 \\ 
\bottomrule
\end{tabular*}
\end{table}

\begin{Shaded}
\begin{Highlighting}[]
\CommentTok{\# Diagnóstico (Valores preditos e resíduos linha a linha)}
\NormalTok{diagnostico }\OtherTok{\textless{}{-}} \FunctionTok{augment}\NormalTok{(modelo\_linear)}
\FunctionTok{reactable}\NormalTok{(diagnostico,   }\AttributeTok{filterable =} \ConstantTok{TRUE}\NormalTok{,}
  \AttributeTok{searchable =} \ConstantTok{TRUE}\NormalTok{)}
\end{Highlighting}
\end{Shaded}

\pandocbounded{\includegraphics[keepaspectratio]{intro_files/figure-pdf/unnamed-chunk-153-1.pdf}}

\begin{itemize}
\tightlist
\item
  \texttt{glm()}: Modelos Lineares Generalizados. Estende a regressão
  linear para variáveis resposta não-normais através de funções de
  ligação (\emph{link function}). Utilize para Regressão Logística
  (\texttt{family=binomial} - resposta 0/1) ou Contagem
  (\texttt{family=poisson}), etc.
\end{itemize}

\begin{Shaded}
\begin{Highlighting}[]
\NormalTok{pacman}\SpecialCharTok{::}\FunctionTok{p\_load}\NormalTok{(report) }
\CommentTok{\# Prevê a probabilidade de um carro ser automático ou manual (am, binário 0/1) baseado no consumo (mpg).}
\NormalTok{modelo\_log }\OtherTok{\textless{}{-}} \FunctionTok{glm}\NormalTok{(am }\SpecialCharTok{\textasciitilde{}}\NormalTok{ mpg, }\AttributeTok{family =} \StringTok{"binomial"}\NormalTok{, }\AttributeTok{data =}\NormalTok{ mtcars)}
\FunctionTok{report}\NormalTok{(modelo\_log) }\CommentTok{\#Gera interpretação em inglês.}
\end{Highlighting}
\end{Shaded}

\begin{verbatim}
We fitted a logistic model (estimated using ML) to predict am with mpg
(formula: am ~ mpg). The model's explanatory power is substantial (Tjur's R2 =
0.37). The model's intercept, corresponding to mpg = 0, is at -6.60 (95% CI
[-12.33, -2.77], p = 0.005). Within this model:

  - The effect of mpg is statistically significant and positive (beta = 0.31, 95%
CI [0.12, 0.59], p = 0.008; Std. beta = 1.85, 95% CI [0.74, 3.54])

Standardized parameters were obtained by fitting the model on a standardized
version of the dataset. 95% Confidence Intervals (CIs) and p-values were
computed using a Wald z-distribution approximation.
\end{verbatim}

\begin{Shaded}
\begin{Highlighting}[]
\NormalTok{pacman}\SpecialCharTok{::}\FunctionTok{p\_load}\NormalTok{(gtsummary) }\CommentTok{\#mais técnico}

\NormalTok{modelo\_linear }\SpecialCharTok{\%\textgreater{}\%}
  \FunctionTok{tbl\_regression}\NormalTok{(}
    \AttributeTok{intercept =} \ConstantTok{TRUE}\NormalTok{,         }\CommentTok{\# Mostrar o intercepto }
    \AttributeTok{estimate\_fun =} \SpecialCharTok{\textasciitilde{}} \FunctionTok{style\_number}\NormalTok{(.x, }\AttributeTok{digits =} \DecValTok{3}\NormalTok{) }\CommentTok{\# Formatar casas decimais}
\NormalTok{  ) }\SpecialCharTok{\%\textgreater{}\%}
  \FunctionTok{add\_global\_p}\NormalTok{() }\SpecialCharTok{\%\textgreater{}\%}          \CommentTok{\# Adiciona p{-}valor global se tiver variáveis categóricas}
  \FunctionTok{bold\_p}\NormalTok{(}\AttributeTok{t =} \FloatTok{0.05}\NormalTok{) }\SpecialCharTok{\%\textgreater{}\%}        \CommentTok{\# Negrito nos valores{-}p significativos}
  \FunctionTok{add\_glance\_source\_note}\NormalTok{(     }\CommentTok{\# Adiciona R\^{}2 e estatísticas no rodapé}
    \AttributeTok{label =} \FunctionTok{list}\NormalTok{(r.squared }\SpecialCharTok{\textasciitilde{}} \StringTok{"$R\^{}2$"}\NormalTok{, AIC }\SpecialCharTok{\textasciitilde{}} \StringTok{"AIC"}\NormalTok{)}
\NormalTok{  )}
\end{Highlighting}
\end{Shaded}

\begin{table}
\fontsize{12.0pt}{14.0pt}\selectfont
\begin{tabular*}{\linewidth}{@{\extracolsep{\fill}}lccc}
\toprule
\textbf{Characteristic} & \textbf{Beta} & \textbf{95\% CI} & \textbf{p-value} \\ 
\midrule\addlinespace[2.5pt]
(Intercept) & 37.227 & 33.957, 40.497 & {\bfseries <0.001} \\ 
wt & -3.878 & -5.172, -2.584 & {\bfseries <0.001} \\ 
hp & -0.032 & -0.050, -0.013 & {\bfseries 0.001} \\ 
\bottomrule
\end{tabular*}
\begin{minipage}{\linewidth}
\vspace{.05em}
Abbreviation: CI = Confidence Interval\\
\(R^2\) = 0.827; Adjusted R² = 0.815; Sigma = 2.59; Statistic = 69.2; p-value = \textless{}0.001; df = 2; Log-likelihood = -74.3; AIC = 157; BIC = 163; Deviance = 195; Residual df = 29; No. Obs. = 32\\
\end{minipage}
\end{table}

\begin{itemize}
\tightlist
\item
  \texttt{gam()} (do pacote \texttt{mgcv}): Modelos Aditivos
  Generalizados. Estende os GLMs permitindo modelar relações
  não-lineares e formas livres (curvas) através de funções de suavização
  (\emph{smooth functions}, denotadas por \texttt{s()}). Utilize quando
  a relação entre X e Y não for uma linha reta simples.
\end{itemize}

\begin{Shaded}
\begin{Highlighting}[]
\NormalTok{pacman}\SpecialCharTok{::}\FunctionTok{p\_load}\NormalTok{(mgcv, report,ggplot2, gtsummary) }

\CommentTok{\# Prevê o consumo (mpg) baseado na potência (hp) assumindo uma relação não{-}linear (curva suave)}
\CommentTok{\# Note o uso de s() para indicar um termo de suavização (spline)}
\NormalTok{modelo\_gam }\OtherTok{\textless{}{-}} \FunctionTok{gam}\NormalTok{(mpg }\SpecialCharTok{\textasciitilde{}} \FunctionTok{s}\NormalTok{(hp), }\AttributeTok{data =}\NormalTok{ mtcars)}

\FunctionTok{report}\NormalTok{(modelo\_gam) }\CommentTok{\# Gera interpretação textual descrevendo a não{-}linearidade.}
\end{Highlighting}
\end{Shaded}

\begin{verbatim}
We fitted a linear model (estimated using GCV and magic optimizer) to predict
mpg with hp (formula: mpg ~ s(hp)). The model's explanatory power is
substantial (R2 = 0.73). The model's intercept, corresponding to hp = 0, is at
20.09 (95% CI [18.97, 21.21], p < .001). Within this model:

  - The effect of Smooth term (hp) is statistically significant and NA (beta = ,
p < .001; Std. beta = , )

Standardized parameters were obtained by fitting the model on a standardized
version of the dataset.
\end{verbatim}

\begin{Shaded}
\begin{Highlighting}[]
\NormalTok{modelo\_gam }\SpecialCharTok{\%\textgreater{}\%}
  \FunctionTok{tbl\_regression}\NormalTok{(}
    \AttributeTok{intercept =} \ConstantTok{TRUE}\NormalTok{,             }\CommentTok{\# Mostrar o intercepto}
    \AttributeTok{estimate\_fun =} \SpecialCharTok{\textasciitilde{}} \FunctionTok{style\_number}\NormalTok{(.x, }\AttributeTok{digits =} \DecValTok{3}\NormalTok{) }\CommentTok{\# Formatar casas decimais}
\NormalTok{  ) }
\end{Highlighting}
\end{Shaded}

\begin{table}
\fontsize{12.0pt}{14.0pt}\selectfont
\begin{tabular*}{\linewidth}{@{\extracolsep{\fill}}lccc}
\toprule
\textbf{Characteristic} & \textbf{Beta} & \textbf{95\% CI} & \textbf{p-value} \\ 
\midrule\addlinespace[2.5pt]
(Intercept) & 20.091 & 19.015, 21.166 & <0.001 \\ 
s(hp) &  &  & <0.001 \\ 
\bottomrule
\end{tabular*}
\begin{minipage}{\linewidth}
\vspace{.05em}
Abbreviation: CI = Confidence Interval\\
\end{minipage}
\end{table}

\begin{Shaded}
\begin{Highlighting}[]
\CommentTok{\# Criando a previsão para plotar}
\NormalTok{pred\_data }\OtherTok{\textless{}{-}} \FunctionTok{data.frame}\NormalTok{(}\AttributeTok{hp =} \FunctionTok{seq}\NormalTok{(}\FunctionTok{min}\NormalTok{(mtcars}\SpecialCharTok{$}\NormalTok{hp), }\FunctionTok{max}\NormalTok{(mtcars}\SpecialCharTok{$}\NormalTok{hp), }\AttributeTok{length =} \DecValTok{100}\NormalTok{))}
\NormalTok{pred\_data}\SpecialCharTok{$}\NormalTok{fit }\OtherTok{\textless{}{-}} \FunctionTok{predict}\NormalTok{(modelo\_gam, }\AttributeTok{newdata =}\NormalTok{ pred\_data)}

\FunctionTok{ggplot}\NormalTok{() }\SpecialCharTok{+}
  \FunctionTok{geom\_line}\NormalTok{(}\AttributeTok{data =}\NormalTok{ pred\_data, }\FunctionTok{aes}\NormalTok{(}\AttributeTok{x =}\NormalTok{ hp, }\AttributeTok{y =}\NormalTok{ fit), }\AttributeTok{color =} \StringTok{"blue"}\NormalTok{, }\AttributeTok{size =} \DecValTok{1}\NormalTok{) }\SpecialCharTok{+}
  \FunctionTok{geom\_rug}\NormalTok{(}\AttributeTok{data =}\NormalTok{ mtcars, }\FunctionTok{aes}\NormalTok{(}\AttributeTok{x =}\NormalTok{ hp), }\AttributeTok{sides =} \StringTok{"b"}\NormalTok{, }\AttributeTok{alpha =} \FloatTok{0.5}\NormalTok{) }\SpecialCharTok{+}
  \FunctionTok{theme\_classic}\NormalTok{()}
\end{Highlighting}
\end{Shaded}

\pandocbounded{\includegraphics[keepaspectratio]{intro_files/figure-pdf/unnamed-chunk-156-1.pdf}}

\begin{tcolorbox}[enhanced jigsaw, left=2mm, toptitle=1mm, colback=white, colframe=quarto-callout-tip-color-frame, colbacktitle=quarto-callout-tip-color!10!white, opacityback=0, rightrule=.15mm, bottomtitle=1mm, arc=.35mm, title=\textcolor{quarto-callout-tip-color}{\faLightbulb}\hspace{0.5em}{Sugestão, use o pacote gratia}, titlerule=0mm, bottomrule=.15mm, leftrule=.75mm, coltitle=black, toprule=.15mm, breakable, opacitybacktitle=0.6]

Existe um pacote excelente chamado \texttt{gratia} (criado por
\href{https://scholar.google.com/citations?user=BFuK-JEAAAAJ&hl=en}{Gavin
Simpson}), feito especificamente para traduzir modelos do \texttt{mgcv}
para a linguagem do \texttt{ggplot2} ( Figura~\ref{fig-gratia}).

\end{tcolorbox}

\begin{Shaded}
\begin{Highlighting}[]
\NormalTok{pacman}\SpecialCharTok{::}\FunctionTok{p\_load}\NormalTok{(gratia)}

\NormalTok{modelo }\OtherTok{\textless{}{-}} \FunctionTok{gam}\NormalTok{(mpg }\SpecialCharTok{\textasciitilde{}} \FunctionTok{s}\NormalTok{(hp) }\SpecialCharTok{+} \FunctionTok{s}\NormalTok{(wt), }\AttributeTok{data =}\NormalTok{ mtcars)}
\FunctionTok{draw}\NormalTok{(modelo)}
\end{Highlighting}
\end{Shaded}

\begin{figure}[H]

\centering{

\pandocbounded{\includegraphics[keepaspectratio]{intro_files/figure-pdf/fig-gratia-1.pdf}}

}

\caption{\label{fig-gratia}Gráfico do modelo gam usando pacote gratia}

\end{figure}%

\begin{itemize}
\tightlist
\item
  \texttt{aov()}: Análise de Variância (ANOVA). Matematicamente
  equivalente a um modelo linear com preditores categóricos, mas focado
  na variância entre grupos. Utilize para comparar médias de três ou
  mais grupos. Use \texttt{TukeyHSD()} depois para ver onde estão as
  diferenças.
\end{itemize}

\begin{Shaded}
\begin{Highlighting}[]
\CommentTok{\# ANOVA de uma via}
\NormalTok{anova\_res }\OtherTok{\textless{}{-}} \FunctionTok{aov}\NormalTok{(Petal.Length }\SpecialCharTok{\textasciitilde{}}\NormalTok{ Species, }\AttributeTok{data =}\NormalTok{ iris)}
\FunctionTok{summary}\NormalTok{(anova\_res)}
\end{Highlighting}
\end{Shaded}

\begin{verbatim}
             Df Sum Sq Mean Sq F value Pr(>F)    
Species       2  437.1  218.55    1180 <2e-16 ***
Residuals   147   27.2    0.19                   
---
Signif. codes:  0 '***' 0.001 '**' 0.01 '*' 0.05 '.' 0.1 ' ' 1
\end{verbatim}

\begin{tcolorbox}[enhanced jigsaw, left=2mm, toptitle=1mm, colback=white, colframe=quarto-callout-tip-color-frame, colbacktitle=quarto-callout-tip-color!10!white, opacityback=0, rightrule=.15mm, bottomtitle=1mm, arc=.35mm, title=\textcolor{quarto-callout-tip-color}{\faLightbulb}\hspace{0.5em}{Use o pacote \emph{broom}}, titlerule=0mm, bottomrule=.15mm, leftrule=.75mm, coltitle=black, toprule=.15mm, breakable, opacitybacktitle=0.6]

Assim como na regressão, o \texttt{summary(anova\_res)} retorna um
objeto difícil de manipular ou não profissionalmente formatado e o
\texttt{gtsummary} foca mais em coeficientes de regressão, para a Tabela
de ANOVA (aquela com Soma de Quadrados, GL, F), a melhor combinação é
limpar com \texttt{broom} e formatar com \texttt{flextable\ ou\ gt}.

\end{tcolorbox}

\begin{Shaded}
\begin{Highlighting}[]
\NormalTok{anova\_res }\OtherTok{\textless{}{-}} \FunctionTok{aov}\NormalTok{(mpg }\SpecialCharTok{\textasciitilde{}} \FunctionTok{factor}\NormalTok{(cyl) }\SpecialCharTok{+} \FunctionTok{factor}\NormalTok{(gear), }\AttributeTok{data =}\NormalTok{ mtcars)}

\CommentTok{\# Transforma a saída feia do console em um data frame organizado}
\NormalTok{tabela\_limpa }\OtherTok{\textless{}{-}} \FunctionTok{tidy}\NormalTok{(anova\_res)}

\FunctionTok{gt}\NormalTok{(tabela\_limpa)}
\end{Highlighting}
\end{Shaded}

\begin{table}
\fontsize{12.0pt}{14.0pt}\selectfont
\begin{tabular*}{\linewidth}{@{\extracolsep{\fill}}lrrrrr}
\toprule
term & df & sumsq & meansq & statistic & p.value \\ 
\midrule\addlinespace[2.5pt]
factor(cyl) & 2 & 824.784590 & 412.392295 & 38.000627 & 1.412658e-08 \\ 
factor(gear) & 2 & 8.251855 & 4.125927 & 0.380191 & 6.873334e-01 \\ 
Residuals & 27 & 293.010743 & 10.852250 & NA & NA \\ 
\bottomrule
\end{tabular*}
\end{table}

\begin{tcolorbox}[enhanced jigsaw, left=2mm, toptitle=1mm, colback=white, colframe=quarto-callout-tip-color-frame, colbacktitle=quarto-callout-tip-color!10!white, opacityback=0, rightrule=.15mm, bottomtitle=1mm, arc=.35mm, title=\textcolor{quarto-callout-tip-color}{\faLightbulb}\hspace{0.5em}{Pacote effectsize}, titlerule=0mm, bottomrule=.15mm, leftrule=.75mm, coltitle=black, toprule=.15mm, breakable, opacitybacktitle=0.6]

Atualmente, apresentar apenas o valor-p da ANOVA é considerado
insuficiente. Você deve apresentar o tamanho do efeito (\(\eta^2\) ou
Partial Eta Squared). O pacote \texttt{effectsize} (da família
\texttt{easystats}) faz isso e já formata para você.

\end{tcolorbox}

\begin{Shaded}
\begin{Highlighting}[]
\NormalTok{pacman}\SpecialCharTok{::}\FunctionTok{p\_load}\NormalTok{(effectsize)}

\CommentTok{\# Calcula Eta Squared e já formata a tabela}
\FunctionTok{eta\_squared}\NormalTok{(anova\_res, }\AttributeTok{partial =} \ConstantTok{FALSE}\NormalTok{) }\SpecialCharTok{\%\textgreater{}\%}
  \FunctionTok{as.data.frame}\NormalTok{() }\SpecialCharTok{\%\textgreater{}\%}
  \FunctionTok{gt}\NormalTok{() }\CommentTok{\# ?effectsize para mais informações}
\end{Highlighting}
\end{Shaded}

\begin{table}
\fontsize{12.0pt}{14.0pt}\selectfont
\begin{tabular*}{\linewidth}{@{\extracolsep{\fill}}lrrrr}
\toprule
Parameter & Eta2 & CI & CI\_low & CI\_high \\ 
\midrule\addlinespace[2.5pt]
factor(cyl) & 0.732460060 & 0.95 & 0.5667547 & 1 \\ 
factor(gear) & 0.007328161 & 0.95 & 0.0000000 & 1 \\ 
\bottomrule
\end{tabular*}
\end{table}

\begin{itemize}
\tightlist
\item
  \texttt{predict()}: Função genérica para gerar valores previstos
  usando um modelo ajustado (\texttt{lm}, \texttt{glm}, etc.) em novos
  dados. Utilize para aplicar seu modelo treinado em um conjunto de
  teste ou novos dados reais.
\end{itemize}

\begin{Shaded}
\begin{Highlighting}[]
\CommentTok{\# Prevê probabilidades para novos clientes (Regressão Logística)}
\NormalTok{previsoes }\OtherTok{\textless{}{-}} \FunctionTok{predict}\NormalTok{(mod\_log, }\AttributeTok{newdata =}\NormalTok{ novos\_clientes, }\AttributeTok{type =} \StringTok{"response"}\NormalTok{)}
\end{Highlighting}
\end{Shaded}

\begin{tcolorbox}[enhanced jigsaw, left=2mm, toptitle=1mm, colback=white, colframe=quarto-callout-tip-color-frame, colbacktitle=quarto-callout-tip-color!10!white, opacityback=0, rightrule=.15mm, bottomtitle=1mm, arc=.35mm, title=\textcolor{quarto-callout-tip-color}{\faLightbulb}\hspace{0.5em}{Pacote finalfit}, titlerule=0mm, bottomrule=.15mm, leftrule=.75mm, coltitle=black, toprule=.15mm, breakable, opacitybacktitle=0.6]

Rápido e muito técnico também

\end{tcolorbox}

\begin{Shaded}
\begin{Highlighting}[]
\NormalTok{pacman}\SpecialCharTok{::}\FunctionTok{p\_load}\NormalTok{(finalfit, dplyr, ggplot2,Hmisc)}

\NormalTok{explanatory }\OtherTok{=} \FunctionTok{c}\NormalTok{(}\StringTok{"age.factor"}\NormalTok{, }\StringTok{"sex.factor"}\NormalTok{, }
  \StringTok{"obstruct.factor"}\NormalTok{, }\StringTok{"perfor.factor"}\NormalTok{)}
\NormalTok{dependent }\OtherTok{=} \StringTok{\textquotesingle{}mort\_5yr\textquotesingle{}}
\NormalTok{colon\_s }\SpecialCharTok{\%\textgreater{}\%}
  \FunctionTok{or\_plot}\NormalTok{(dependent, explanatory)}
\end{Highlighting}
\end{Shaded}

\pandocbounded{\includegraphics[keepaspectratio]{intro_files/figure-pdf/unnamed-chunk-161-1.pdf}}

\textbf{Extraindo informações dos modelos:}

\begin{itemize}
\tightlist
\item
  \texttt{coef(modelo)}: Extrai os coeficientes estimados.
\item
  \texttt{residuals(modelo)}: Extrai os resíduos.
\item
  \texttt{anova(modelo)}: Tabela de análise de variância.
\item
  \texttt{AIC(modelo)}: Critério de informação de Akaike.
\end{itemize}

\begin{tcolorbox}[enhanced jigsaw, left=2mm, toptitle=1mm, colback=white, colframe=quarto-callout-tip-color-frame, colbacktitle=quarto-callout-tip-color!10!white, opacityback=0, rightrule=.15mm, bottomtitle=1mm, arc=.35mm, title=\textcolor{quarto-callout-tip-color}{\faLightbulb}\hspace{0.5em}{Recomendação Especial: GLM}, titlerule=0mm, bottomrule=.15mm, leftrule=.75mm, coltitle=black, toprule=.15mm, breakable, opacitybacktitle=0.6]

Aos interessados em dominar \textbf{Modelos Lineares Generalizados
(GLM)}, deixo minha forte recomendação para a disciplina
\href{https://uspdigital.usp.br/janus/componente/disciplinasOferecidasInicial.jsf?action=3&sgldis=MAE5763}{MAE5763-Modelos
Lineares Generalizados}. Ela é oferecida no segundo semestre no IME-USP
e ministrada pelo Professor
\href{https://scholar.google.com.br/citations?hl=en&user=lNNFsqMAAAAJ}{Gilberto
Alvarenga Paula}. Pela profundidade e didática, considero esta uma das
disciplinas mais valiosas da minha formação.

\end{tcolorbox}

\section{Visualização de Dados}\label{sec-Visualizauxe7uxe3o}

\textbf{Gráficos do\texttt{R} Base}

Úteis para inspeção rápida de dados.

\begin{itemize}
\tightlist
\item
  \texttt{plot(x,\ y)}: Função genérica e polimórfica que adapta o
  gráfico dependendo do tipo de objeto fornecido. Se receber dois
  \texttt{vetores\ numéricos}, cria um \texttt{gráfico\ de\ dispersão} (
  Figura~\ref{fig-plot_func}); se receber um \texttt{fator} e
  \texttt{um\ numérico}, cria \texttt{boxplots}. A ferramenta universal
  para uma primeira olhada na relação entre duas variáveis
  \((X \times Y)\) ou para visualizar objetos complexos (como resíduos
  de um modelo linear \texttt{plot(modelo)}).
\end{itemize}

\begin{Shaded}
\begin{Highlighting}[]
\FunctionTok{plot}\NormalTok{(}\AttributeTok{x =}\NormalTok{ mtcars}\SpecialCharTok{$}\NormalTok{wt, }\AttributeTok{y =}\NormalTok{ mtcars}\SpecialCharTok{$}\NormalTok{mpg, }
     \AttributeTok{main =} \StringTok{"Peso vs Consumo"}\NormalTok{, }\AttributeTok{xlab =} \StringTok{"Peso"}\NormalTok{, }\AttributeTok{ylab =} \StringTok{"MPG"}\NormalTok{, }\AttributeTok{pch =} \DecValTok{19}\NormalTok{)}
\end{Highlighting}
\end{Shaded}

\begin{figure}[H]

\centering{

\pandocbounded{\includegraphics[keepaspectratio]{intro_files/figure-pdf/fig-plot_func-1.pdf}}

}

\caption{\label{fig-plot_func}Gráfico de dispersão (Scatterplot) entre
Peso (wt) e Consumo (mpg)}

\end{figure}%

\begin{itemize}
\tightlist
\item
  \texttt{hist(x)}: Divide uma variável numérica contínua em intervalos
  (bins) e conta a frequência de observações em cada um (
  Figura~\ref{fig-hist_plot}). Utilize para entender a distribuição dos
  dados: verificar normalidade, assimetria (\emph{skewness}) ou
  identificar se a distribuição é bimodal.
\end{itemize}

\begin{Shaded}
\begin{Highlighting}[]
\FunctionTok{hist}\NormalTok{(iris}\SpecialCharTok{$}\NormalTok{Petal.Length, }\AttributeTok{xlab =} \StringTok{"Petal.Length"}\NormalTok{,}
     \AttributeTok{col =} \StringTok{"lightblue"}\NormalTok{, }\AttributeTok{main =} \StringTok{"Distribuição das Pétalas"}\NormalTok{, }\AttributeTok{breaks =} \DecValTok{10}\NormalTok{)}
\end{Highlighting}
\end{Shaded}

\begin{figure}[H]

\centering{

\pandocbounded{\includegraphics[keepaspectratio]{intro_files/figure-pdf/fig-hist_plot-1.pdf}}

}

\caption{\label{fig-hist_plot}Histograma do comprimento das pétalas}

\end{figure}%

\begin{itemize}
\tightlist
\item
  \texttt{boxplot(x)}: Representação visual do resumo de cinco números:
  mínimo, 1º quartil, mediana, 3º quartil e máximo (
  Figura~\ref{fig-boxplot_plot}). Pontos fora dos ``bigodes'' indicam
  outliers. A melhor ferramenta para comparar distribuições entre
  diferentes grupos e detectar anomalias (\emph{outliers}) rapidamente.
  Aceita sintaxe de \texttt{fórmula\ (y\ \textasciitilde{}\ grupo)}.
\end{itemize}

\begin{Shaded}
\begin{Highlighting}[]
\FunctionTok{boxplot}\NormalTok{(len }\SpecialCharTok{\textasciitilde{}}\NormalTok{ supp, }\AttributeTok{data =}\NormalTok{ ToothGrowth, }
        \AttributeTok{col =} \FunctionTok{c}\NormalTok{(}\StringTok{"orange"}\NormalTok{, }\StringTok{"yellow"}\NormalTok{), }\AttributeTok{main =} \StringTok{""}\NormalTok{)}
\end{Highlighting}
\end{Shaded}

\begin{figure}[H]

\centering{

\pandocbounded{\includegraphics[keepaspectratio]{intro_files/figure-pdf/fig-boxplot_plot-1.pdf}}

}

\caption{\label{fig-boxplot_plot}Tamanho do dente por tipo de suplemento
(ToothGrowth)}

\end{figure}%

\begin{itemize}
\tightlist
\item
  \texttt{barplot(height)}: Cria barras com alturas proporcionais aos
  valores fornecidos ( Figura~\ref{fig-barplot}).
\end{itemize}

\begin{tcolorbox}[enhanced jigsaw, left=2mm, toptitle=1mm, colback=white, colframe=quarto-callout-important-color-frame, colbacktitle=quarto-callout-important-color!10!white, opacityback=0, rightrule=.15mm, bottomtitle=1mm, arc=.35mm, title=\textcolor{quarto-callout-important-color}{\faExclamation}\hspace{0.5em}{Atenção}, titlerule=0mm, bottomrule=.15mm, leftrule=.75mm, coltitle=black, toprule=.15mm, breakable, opacitybacktitle=0.6]

Diferente do histograma, ele exige que você forneça um vetor ou matriz
de contagens/valores já sumarizados (ex: resultado de \texttt{table()}).
Utilize para comparar quantidades entre categorias discretas.

\end{tcolorbox}

\begin{Shaded}
\begin{Highlighting}[]
\NormalTok{contagem }\OtherTok{\textless{}{-}} \FunctionTok{table}\NormalTok{(mtcars}\SpecialCharTok{$}\NormalTok{cyl)}
\FunctionTok{barplot}\NormalTok{(contagem, }\AttributeTok{xlab=}\StringTok{"Número de carros"}\NormalTok{,}
        \AttributeTok{main =} \StringTok{""}\NormalTok{, }\AttributeTok{col =} \StringTok{"steelblue"}\NormalTok{, }\AttributeTok{ylab =} \StringTok{"Número de cilindros"}\NormalTok{)}
\end{Highlighting}
\end{Shaded}

\begin{figure}[H]

\centering{

\pandocbounded{\includegraphics[keepaspectratio]{intro_files/figure-pdf/fig-barplot-1.pdf}}

}

\caption{\label{fig-barplot}Contagem de carros por número de cilindros}

\end{figure}%

\begin{itemize}
\tightlist
\item
  \texttt{pairs(df)}: Cria uma matriz de scatterplots, cruzando todas as
  variáveis numéricas do dataframe umas contra as outras (N x N).
  Essencial na fase inicial de Análise Exploratória (EDA) multivariada
  para detectar correlações lineares, clusters naturais ou padrões entre
  múltiplas variáveis simultaneamente ( Figura~\ref{fig-parplot}).
\end{itemize}

\begin{Shaded}
\begin{Highlighting}[]
\FunctionTok{pairs}\NormalTok{(iris[, }\DecValTok{1}\SpecialCharTok{:}\DecValTok{4}\NormalTok{], }
      \AttributeTok{col =}\NormalTok{ iris}\SpecialCharTok{$}\NormalTok{Species, }\AttributeTok{pch =} \DecValTok{1}\NormalTok{, }\AttributeTok{main =} \StringTok{""}\NormalTok{)}
\end{Highlighting}
\end{Shaded}

\begin{figure}[H]

\centering{

\pandocbounded{\includegraphics[keepaspectratio]{intro_files/figure-pdf/fig-parplot-1.pdf}}

}

\caption{\label{fig-parplot}Matriz de dispersão das 4 variáveis
numéricas do iris}

\end{figure}%

\begin{itemize}
\tightlist
\item
  \texttt{abline():} Adiciona linhas retas (horizontais, verticais ou
  regressão) a um gráfico existente. Útil para adicionar a linha de
  tendência.
\end{itemize}

\begin{Shaded}
\begin{Highlighting}[]
\FunctionTok{plot}\NormalTok{(mtcars}\SpecialCharTok{$}\NormalTok{wt, mtcars}\SpecialCharTok{$}\NormalTok{mpg, }\AttributeTok{xlab =} \StringTok{"wt"}\NormalTok{, }\AttributeTok{ylab=}\StringTok{"mpg"}\NormalTok{)}
\FunctionTok{abline}\NormalTok{(}\FunctionTok{lm}\NormalTok{(mpg }\SpecialCharTok{\textasciitilde{}}\NormalTok{ wt, }\AttributeTok{data =}\NormalTok{ mtcars), }\AttributeTok{col =} \StringTok{"red"}\NormalTok{) }\CommentTok{\# Adiciona linha de regressão}
\end{Highlighting}
\end{Shaded}

\pandocbounded{\includegraphics[keepaspectratio]{intro_files/figure-pdf/unnamed-chunk-162-1.pdf}}

\textbf{Elementos de Baixo Nível}

\begin{itemize}
\tightlist
\item
  \texttt{points(x,\ y)}: Adiciona símbolos (pontos) a um gráfico já
  aberto. Permite sobrepor novas séries de dados ou destacar pontos
  específicos com cores/formatos diferentes. Utilize para destacar
  \emph{outliers}, adicionar uma segunda variável no mesmo eixo Y ou
  marcar médias sobre um boxplot ( Figura~\ref{fig-points}).
\end{itemize}

\begin{Shaded}
\begin{Highlighting}[]
\FunctionTok{plot}\NormalTok{(mtcars}\SpecialCharTok{$}\NormalTok{wt, mtcars}\SpecialCharTok{$}\NormalTok{mpg, }\AttributeTok{xlab =} \StringTok{"wt"}\NormalTok{, }\AttributeTok{ylab =} \StringTok{"mpg"}\NormalTok{)}
\FunctionTok{points}\NormalTok{(mtcars}\SpecialCharTok{$}\NormalTok{wt[mtcars}\SpecialCharTok{$}\NormalTok{cyl }\SpecialCharTok{==} \DecValTok{8}\NormalTok{], mtcars}\SpecialCharTok{$}\NormalTok{mpg[mtcars}\SpecialCharTok{$}\NormalTok{cyl }\SpecialCharTok{==} \DecValTok{8}\NormalTok{], }\AttributeTok{col =} \StringTok{"red"}\NormalTok{, }\AttributeTok{pch =} \DecValTok{19}\NormalTok{)}
\end{Highlighting}
\end{Shaded}

\begin{figure}[H]

\centering{

\pandocbounded{\includegraphics[keepaspectratio]{intro_files/figure-pdf/fig-points-1.pdf}}

}

\caption{\label{fig-points}Adicionando pontos vermelhos para carros com
8 cilindros}

\end{figure}%

\begin{itemize}
\tightlist
\item
  \texttt{lines(x,\ y)}: Conecta coordenadas (x, y) com segmentos de
  linha.
\end{itemize}

\begin{tcolorbox}[enhanced jigsaw, left=2mm, toptitle=1mm, colback=white, colframe=quarto-callout-important-color-frame, colbacktitle=quarto-callout-important-color!10!white, opacityback=0, rightrule=.15mm, bottomtitle=1mm, arc=.35mm, title=\textcolor{quarto-callout-important-color}{\faExclamation}\hspace{0.5em}{Importante}, titlerule=0mm, bottomrule=.15mm, leftrule=.75mm, coltitle=black, toprule=.15mm, breakable, opacitybacktitle=0.6]

Atenção: Os dados devem estar ordenados pelo eixo X, senão a linha
ficará ``rabiscada''. Utilize para desenhar séries temporais, curvas de
modelos ajustados (suavizações) ou polígonos de contorno (
Figura~\ref{fig-lines}).

\end{tcolorbox}

\begin{Shaded}
\begin{Highlighting}[]
\FunctionTok{plot}\NormalTok{(pressure}\SpecialCharTok{$}\NormalTok{temperature, pressure}\SpecialCharTok{$}\NormalTok{pressure, }\AttributeTok{xlab =} \StringTok{"temperature"}\NormalTok{, }\AttributeTok{ylab=}\StringTok{"pressure"}\NormalTok{)}
\FunctionTok{lines}\NormalTok{(pressure}\SpecialCharTok{$}\NormalTok{temperature, pressure}\SpecialCharTok{$}\NormalTok{pressure, }\AttributeTok{col =} \StringTok{"blue"}\NormalTok{, }\AttributeTok{lwd =} \DecValTok{2}\NormalTok{)}
\end{Highlighting}
\end{Shaded}

\begin{figure}[H]

\centering{

\pandocbounded{\includegraphics[keepaspectratio]{intro_files/figure-pdf/fig-lines-1.pdf}}

}

\caption{\label{fig-lines}Conectando os pontos com uma linha azul}

\end{figure}%

\begin{itemize}
\tightlist
\item
  \texttt{abline(a,\ b)}: Adiciona linhas retas de referência que
  atravessam todo o gráfico. Pode ser horizontal \texttt{(h)},
  \texttt{vertical\ (v)} ou baseada em intercepto e inclinação (a, b).
  Ideal para adicionar a linha de regressão \texttt{(abline(modelo))},
  linhas de média (horizontal) ou limites de especificação/datas
  importantes (vertical) ( Figura~\ref{fig-hline_vline}).
\end{itemize}

\begin{Shaded}
\begin{Highlighting}[]
\FunctionTok{plot}\NormalTok{(mtcars}\SpecialCharTok{$}\NormalTok{wt, mtcars}\SpecialCharTok{$}\NormalTok{mpg, }\AttributeTok{xlab =} \StringTok{"wt"}\NormalTok{, }\AttributeTok{ylab =} \StringTok{"mpg"}\NormalTok{)}
\FunctionTok{abline}\NormalTok{(}\AttributeTok{h =} \FunctionTok{mean}\NormalTok{(mtcars}\SpecialCharTok{$}\NormalTok{mpg), }\AttributeTok{col =} \StringTok{"red"}\NormalTok{); }\FunctionTok{abline}\NormalTok{(}\AttributeTok{v =} \DecValTok{3}\NormalTok{, }\AttributeTok{col =} \StringTok{"green"}\NormalTok{)}
\end{Highlighting}
\end{Shaded}

\begin{figure}[H]

\centering{

\pandocbounded{\includegraphics[keepaspectratio]{intro_files/figure-pdf/fig-hline_vline-1.pdf}}

}

\caption{\label{fig-hline_vline}Adiciona linha horizontal na média
(vermelha) e vertical em x=3 (verde)}

\end{figure}%

\begin{itemize}
\tightlist
\item
  \texttt{text(x,\ y,\ labels)}: Plota \texttt{strings} de texto dentro
  da área do gráfico nas coordenadas (x, y) especificadas. Utilize para
  rotular pontos de dados individuais (ex: nome do país no scatterplot)
  ou anotar equações e valores-p ( Figura~\ref{fig-ad_text}).
\end{itemize}

\begin{Shaded}
\begin{Highlighting}[]
\FunctionTok{plot}\NormalTok{(mtcars}\SpecialCharTok{$}\NormalTok{wt, mtcars}\SpecialCharTok{$}\NormalTok{mpg, }\AttributeTok{xlab =} \StringTok{"wt"}\NormalTok{, }\AttributeTok{ylab =} \StringTok{"mpg"}\NormalTok{)}
\FunctionTok{text}\NormalTok{(}\AttributeTok{x =} \FunctionTok{max}\NormalTok{(mtcars}\SpecialCharTok{$}\NormalTok{wt)}\SpecialCharTok{{-}}\DecValTok{2}\NormalTok{, }\AttributeTok{y =} \FunctionTok{min}\NormalTok{(mtcars}\SpecialCharTok{$}\NormalTok{mpg)}\SpecialCharTok{+}\DecValTok{3}\NormalTok{, }\AttributeTok{labels =} \StringTok{"Carro Mais Pesado"}\NormalTok{, }\AttributeTok{pos =} \DecValTok{2}\NormalTok{)}
\end{Highlighting}
\end{Shaded}

\begin{figure}[H]

\centering{

\pandocbounded{\includegraphics[keepaspectratio]{intro_files/figure-pdf/fig-ad_text-1.pdf}}

}

\caption{\label{fig-ad_text}Escreve o nome do carro no ponto mais
pesado}

\end{figure}%

\begin{itemize}
\tightlist
\item
  \texttt{legend(x,\ y,\ legend)}: Adiciona uma caixa de legenda
  explicando o significado das cores, símbolos ou tipos de linha (
  Figura~\ref{fig-plot_legend_base}). Essencial sempre que você usar
  cores ou tipos de linha diferentes para representar grupos, pois o
  \texttt{R} Base não cria legendas automáticas (ao contrário do
  \texttt{ggplot2}).
\end{itemize}

\begin{Shaded}
\begin{Highlighting}[]
\CommentTok{\# O dataset iris tem 3 espécies. Vamos criar um vetor de 3 cores.}
\NormalTok{minhas\_cores }\OtherTok{\textless{}{-}} \FunctionTok{c}\NormalTok{(}\StringTok{"red"}\NormalTok{, }\StringTok{"blue"}\NormalTok{, }\StringTok{"forestgreen"}\NormalTok{)}

\CommentTok{\#Criar o gráfico base}
\FunctionTok{plot}\NormalTok{(}\AttributeTok{x =}\NormalTok{ iris}\SpecialCharTok{$}\NormalTok{Sepal.Length, }
     \AttributeTok{y =}\NormalTok{ iris}\SpecialCharTok{$}\NormalTok{Sepal.Width,}
     \AttributeTok{col =}\NormalTok{ minhas\_cores[iris}\SpecialCharTok{$}\NormalTok{Species], }
     \AttributeTok{pch =} \DecValTok{19}\NormalTok{, }\CommentTok{\# Bolinha sólida}
     \AttributeTok{main =} \StringTok{""}\NormalTok{,}
     \AttributeTok{xlab =} \StringTok{"Comprimento"}\NormalTok{,}
     \AttributeTok{ylab =} \StringTok{"Largura"}\NormalTok{)}

\CommentTok{\#Adicionar a legenda}
\FunctionTok{legend}\NormalTok{(}\StringTok{"topright"}\NormalTok{, }
       \AttributeTok{legend =} \FunctionTok{levels}\NormalTok{(iris}\SpecialCharTok{$}\NormalTok{Species), }\CommentTok{\# Pega os nomes reais: setosa, versicolor...}
       \AttributeTok{col =}\NormalTok{ minhas\_cores,            }\CommentTok{\# Usa AS MESMAS cores definidas acima}
       \AttributeTok{pch =} \DecValTok{19}\NormalTok{,                      }\CommentTok{\# Usa O MESMO símbolo do plot}
       \AttributeTok{title =} \StringTok{"Espécies"}\NormalTok{,}
       \AttributeTok{bty =} \StringTok{"n"}\NormalTok{)                     }\CommentTok{\#Remove a caixa em volta da legenda}
\end{Highlighting}
\end{Shaded}

\begin{figure}[H]

\centering{

\pandocbounded{\includegraphics[keepaspectratio]{intro_files/figure-pdf/fig-plot_legend_base-1.pdf}}

}

\caption{\label{fig-plot_legend_base}Sépala: Comprimento vs Largura}

\end{figure}%

\textbf{Parâmetros Gráficos (\texttt{par}):} Configuram o dispositivo
gráfico globalmente.

\begin{itemize}
\tightlist
\item
  \texttt{par(mar\ =\ c(bottom,\ left,\ top,\ right))}: Define as
  margens da figura na ordem: Baixo, Esquerda, Cima, Direita. A unidade
  é ``linhas de texto''. O padrão é \texttt{c(5,\ 4,\ 4,\ 2)\ +\ 0.1}.
  Utilize quando os rótulos dos eixos são cortados ou quando você
  precisa de espaço extra fora do gráfico para uma legenda externa (
  Figura~\ref{fig-margens}).
\end{itemize}

\begin{Shaded}
\begin{Highlighting}[]
\CommentTok{\# Gráfico 1: Margens padrão}
\CommentTok{\# O texto longo no eixo X será cortado}
\FunctionTok{barplot}\NormalTok{(}\DecValTok{1}\SpecialCharTok{:}\DecValTok{5}\NormalTok{, }
        \AttributeTok{names.arg =} \FunctionTok{c}\NormalTok{(}\StringTok{"Nome Muito Longo 1"}\NormalTok{, }\StringTok{"Nome Muito Longo 2"}\NormalTok{, }\StringTok{"..."}\NormalTok{, }\StringTok{"..."}\NormalTok{, }\StringTok{"..."}\NormalTok{), }
        \AttributeTok{las =} \DecValTok{2}\NormalTok{,}
        \AttributeTok{main =} \StringTok{"Margem Padrão"}\NormalTok{)}

\CommentTok{\# Gráfico 2: Margens ajustadas}
\CommentTok{\# par(mar = c(baixo, esquerda, cima, direita))}
\CommentTok{\# Aumentamos o primeiro valor (baixo) de \textasciitilde{}5 para 10}
\FunctionTok{par}\NormalTok{(}\AttributeTok{mar =} \FunctionTok{c}\NormalTok{(}\DecValTok{10}\NormalTok{, }\DecValTok{4}\NormalTok{, }\DecValTok{4}\NormalTok{, }\DecValTok{2}\NormalTok{))}

\FunctionTok{barplot}\NormalTok{(}\DecValTok{1}\SpecialCharTok{:}\DecValTok{5}\NormalTok{, }
        \AttributeTok{names.arg =} \FunctionTok{c}\NormalTok{(}\StringTok{"Nome Muito Longo 1"}\NormalTok{, }\StringTok{"Nome Muito Longo 2"}\NormalTok{, }\StringTok{"..."}\NormalTok{, }\StringTok{"..."}\NormalTok{, }\StringTok{"..."}\NormalTok{), }
        \AttributeTok{las =} \DecValTok{2}\NormalTok{,}
        \AttributeTok{main =} \StringTok{"Margem Ajustada"}\NormalTok{)}
\end{Highlighting}
\end{Shaded}

\begin{figure}

\begin{minipage}{0.50\linewidth}

\centering{

\pandocbounded{\includegraphics[keepaspectratio]{intro_files/figure-pdf/fig-margens-1.pdf}}

}

\subcaption{\label{fig-margens-1}Sem aumentar as margens (Corta o
texto)}

\end{minipage}%
%
\begin{minipage}{0.50\linewidth}

\centering{

\pandocbounded{\includegraphics[keepaspectratio]{intro_files/figure-pdf/fig-margens-2.pdf}}

}

\subcaption{\label{fig-margens-2}Aumentando as margens (Texto visível)}

\end{minipage}%

\caption{\label{fig-margens}Ajuste de margens em barplots}

\end{figure}%

\begin{itemize}
\tightlist
\item
  \texttt{par(mfrow\ =\ c(rows,\ cols))}: Cria uma grade (matriz) para
  plotar múltiplos gráficos na mesma janela.
  \texttt{c(linhas,\ colunas)}. Utilize para comparar visualizações lado
  a lado (ex: um histograma ao lado de um boxplot da mesma variável).
\end{itemize}

\begin{Shaded}
\begin{Highlighting}[]
\FunctionTok{par}\NormalTok{(}\AttributeTok{mfrow =} \FunctionTok{c}\NormalTok{(}\DecValTok{1}\NormalTok{, }\DecValTok{2}\NormalTok{)) }\CommentTok{\# Prepara grade 1 linha x 2 colunas}
\FunctionTok{hist}\NormalTok{(mtcars}\SpecialCharTok{$}\NormalTok{mpg, }\AttributeTok{xlab=}\StringTok{""}\NormalTok{, }\AttributeTok{ylab=}\StringTok{""}\NormalTok{, }\AttributeTok{main=}\StringTok{""}\NormalTok{)}
\FunctionTok{boxplot}\NormalTok{(mtcars}\SpecialCharTok{$}\NormalTok{mpg, }\AttributeTok{xlab=}\StringTok{""}\NormalTok{, }\AttributeTok{ylab=}\StringTok{""}\NormalTok{, }\AttributeTok{main=}\StringTok{""}\NormalTok{) }\CommentTok{\# Plota os dois}
\end{Highlighting}
\end{Shaded}

\pandocbounded{\includegraphics[keepaspectratio]{intro_files/figure-pdf/unnamed-chunk-163-1.pdf}}

\begin{Shaded}
\begin{Highlighting}[]
\FunctionTok{par}\NormalTok{(}\AttributeTok{mfrow =} \FunctionTok{c}\NormalTok{(}\DecValTok{1}\NormalTok{, }\DecValTok{1}\NormalTok{)) }\CommentTok{\# Restaura para 1 gráfico por vez}
\end{Highlighting}
\end{Shaded}

\begin{itemize}
\tightlist
\item
  \texttt{par(usr)}: Retorna (ou define) os limites extremos das
  coordenadas do gráfico no formato c(x1, x2, y1, y2). Utilize para
  saber exatamente onde terminam os eixos para posicionar textos nos
  cantos absolutos da figura, independente dos dados.
\end{itemize}

\begin{Shaded}
\begin{Highlighting}[]
\FunctionTok{plot}\NormalTok{(}\DecValTok{1}\SpecialCharTok{:}\DecValTok{10}\NormalTok{)}
\end{Highlighting}
\end{Shaded}

\pandocbounded{\includegraphics[keepaspectratio]{intro_files/figure-pdf/unnamed-chunk-164-1.pdf}}

\begin{Shaded}
\begin{Highlighting}[]
\NormalTok{limites }\OtherTok{\textless{}{-}} \FunctionTok{par}\NormalTok{(}\StringTok{"usr"}\NormalTok{) }\CommentTok{\# Retorna ex: c(0.64, 10.36, 0.64, 10.36)}
\CommentTok{\# Use o limite superior do eixo Y para posicionar algo no topo}
\end{Highlighting}
\end{Shaded}

\begin{itemize}
\tightlist
\item
  \texttt{strheight()}, \texttt{strwidth()}: Calcula quanto espaço (nas
  coordenadas do usuário) uma string de texto vai ocupar. Utilize para
  posicionar legendas personalizadas exatamente ``ao lado'' de um ponto
  sem sobrepor, ou para desenhar caixas ao redor de textos manualmente.
\end{itemize}

\begin{Shaded}
\begin{Highlighting}[]
\FunctionTok{plot}\NormalTok{(}\DecValTok{1}\SpecialCharTok{:}\DecValTok{10}\NormalTok{)}
\NormalTok{largura }\OtherTok{\textless{}{-}} \FunctionTok{strwidth}\NormalTok{(}\StringTok{"Meu Texto"}\NormalTok{)}
\CommentTok{\# Desenha uma linha exatamente do tamanho do texto}
\FunctionTok{segments}\NormalTok{(}\AttributeTok{x0=}\DecValTok{1}\NormalTok{, }\AttributeTok{y0=}\DecValTok{5}\NormalTok{, }\AttributeTok{x1=}\DecValTok{1}\SpecialCharTok{+}\NormalTok{largura, }\AttributeTok{y1=}\DecValTok{5}\NormalTok{)}
\end{Highlighting}
\end{Shaded}

\pandocbounded{\includegraphics[keepaspectratio]{intro_files/figure-pdf/unnamed-chunk-165-1.pdf}}

\begin{tcolorbox}[enhanced jigsaw, left=2mm, toptitle=1mm, colback=white, colframe=quarto-callout-important-color-frame, colbacktitle=quarto-callout-important-color!10!white, opacityback=0, rightrule=.15mm, bottomtitle=1mm, arc=.35mm, title=\textcolor{quarto-callout-important-color}{\faExclamation}\hspace{0.5em}{ggplot2}, titlerule=0mm, bottomrule=.15mm, leftrule=.75mm, coltitle=black, toprule=.15mm, breakable, opacitybacktitle=0.6]

Um pacote muito bom e completo para questões de visualização gráfica

\end{tcolorbox}

\section{\texorpdfstring{Visualização Profissional de Dados com
\texttt{ggplot2}}{Visualização Profissional de Dados com ggplot2}}\label{visualizauxe7uxe3o-profissional-de-dados-com-ggplot2}

O pacote \href{https://en.wikipedia.org/wiki/Ggplot2}{\texttt{ggplot2}},
implementação da Gramática dos Gráficos, que constrói visualizações
complexas sobrepondo camadas lógicas independentes. Foca no mapeamento
de dados em atributos visuais, permitindo flexibilidade total sem
precisar memorizar dezenas de funções isoladas. Clique
\href{https://en.wikipedia.org/wiki/Wilkinson\%27s_Grammar_of_Graphics}{aqui}
para se informar melhor.

\subsection{Estrutura fundamental}\label{estrutura-fundamental}

Todo gráfico necessita de três componentes mínimos:

\begin{enumerate}
\def\labelenumi{\arabic{enumi}.}
\tightlist
\item
  \textbf{Dados} (Data): O data frame contendo as informações brutas e
  variáveis que alimentarão o gráfico. É a base fundamental que
  permanece inalterada, servindo de fonte para todas as camadas visuais
  subsequentes.
\item
  \textbf{Estética} (Aesthetics - \texttt{aes}): O que representa o eixo
  X?, o eixo Y? a cor?, etc. Mapeia colunas de dados para propriedades
  visuais do gráfico (como eixos X/Y, cores ou tamanhos). Define ``o
  quê'' será mostrado e como as variáveis controlarão a aparência
  dinâmica dos elementos.
\item
  \textbf{Geometria} (Geoms): Qual a forma visual (barra, ponto,
  linha)?. É o objeto visual escolhido para representar os dados na
  tela, definindo a forma do gráfico (ex: barras, pontos, linhas).
\end{enumerate}

\begin{itemize}
\tightlist
\item
  \texttt{Template\ Universal\ /"A\ Anatomia\ do\ ggplot"}
\end{itemize}

O \texttt{ggplot2} funciona através da sobreposição de camadas
(\emph{layers}) conectadas pelo sinal de \texttt{+}.

\begin{tcolorbox}[enhanced jigsaw, left=2mm, toptitle=1mm, colback=white, colframe=quarto-callout-important-color-frame, colbacktitle=quarto-callout-important-color!10!white, opacityback=0, rightrule=.15mm, bottomtitle=1mm, arc=.35mm, title=\textcolor{quarto-callout-important-color}{\faExclamation}\hspace{0.5em}{Importante}, titlerule=0mm, bottomrule=.15mm, leftrule=.75mm, coltitle=black, toprule=.15mm, breakable, opacitybacktitle=0.6]

A ordem importa, o que vem depois é desenhado por cima do anterior.

\end{tcolorbox}

\begin{Shaded}
\begin{Highlighting}[]
\FunctionTok{ggplot}\NormalTok{(}
  \AttributeTok{data =} \SpecialCharTok{\textless{}}\NormalTok{DADOS}\SpecialCharTok{\textgreater{}}\NormalTok{, }
  \AttributeTok{mapping =} \FunctionTok{aes}\NormalTok{(}\AttributeTok{x =} \SpecialCharTok{\textless{}}\NormalTok{VAR\_X}\SpecialCharTok{\textgreater{}}\NormalTok{, }\AttributeTok{y =} \SpecialCharTok{\textless{}}\NormalTok{VAR\_Y}\SpecialCharTok{\textgreater{}}\NormalTok{, }\AttributeTok{color =} \SpecialCharTok{\textless{}}\NormalTok{VAR\_GRUPO}\SpecialCharTok{\textgreater{}}\NormalTok{, }\AttributeTok{fill =} \SpecialCharTok{\textless{}}\NormalTok{VAR\_PREENCHIMENTO}\SpecialCharTok{\textgreater{}}\NormalTok{)}
\NormalTok{) }\SpecialCharTok{+}
  
\CommentTok{\#1. GEOMETRIAS (A forma visual) {-}{-}{-}}
\CommentTok{\# Podem herdar os aes() globais ou ter os seus próprios}
\ErrorTok{\textless{}}\NormalTok{GEOM\_FUNÇÃO}\SpecialCharTok{\textgreater{}}\NormalTok{(}
  \AttributeTok{mapping =} \FunctionTok{aes}\NormalTok{(...),           }\CommentTok{\# Mapeamento específico desta camada}
  \AttributeTok{data =} \SpecialCharTok{\textless{}}\NormalTok{DADOS\_FILTRADOS}\SpecialCharTok{\textgreater{}}\NormalTok{,     }\CommentTok{\# Usar dados diferentes nesta camada}
  \AttributeTok{stat =} \StringTok{"\textless{}ESTATÍSTICA\textgreater{}"}\NormalTok{,       }\CommentTok{\# Transformação estatística (ex: "identity", "count")}
  \AttributeTok{position =} \StringTok{"\textless{}POSIÇÃO\textgreater{}"}\NormalTok{,       }\CommentTok{\# Ajuste de sobreposição (ex: "dodge", "jitter", "stack")}
  \AttributeTok{alpha =} \FloatTok{0.5}\NormalTok{,                  }\CommentTok{\# Transparência fixa (não mapeada)}
  \AttributeTok{size =} \DecValTok{2}                      \CommentTok{\# Tamanho fixo}
\NormalTok{) }\SpecialCharTok{+}

\CommentTok{\#2. ESTATÍSTICAS}
\CommentTok{\# Útil quando você quer resumir dados (ex: média e erro) sem pré{-}calcular}
\FunctionTok{stat\_summary}\NormalTok{(}\AttributeTok{fun =}\NormalTok{ mean, }\AttributeTok{geom =} \StringTok{"point"}\NormalTok{) }\SpecialCharTok{+}
\FunctionTok{stat\_smooth}\NormalTok{(}\AttributeTok{method =} \StringTok{"lm"}\NormalTok{) }\SpecialCharTok{+}

\CommentTok{\#FACETAS (Small Multiples)}
\CommentTok{\# Dividir em sub{-}gráficos}
\FunctionTok{facet\_wrap}\NormalTok{(}\SpecialCharTok{\textasciitilde{}} \ErrorTok{\textless{}}\NormalTok{VAR\_CATEGORIA}\SpecialCharTok{\textgreater{}}\NormalTok{, }\AttributeTok{scales =} \StringTok{"free\_y"}\NormalTok{) }\SpecialCharTok{+}  \CommentTok{\# Ou facet\_grid(l \textasciitilde{} c)}

\CommentTok{\#ESCALAS (O controle fino dos mapeamentos)}
\CommentTok{\# Controlam eixos (limites, log), cores, tamanhos e formatos}
\FunctionTok{scale\_x\_continuous}\NormalTok{(}\AttributeTok{limits =} \FunctionTok{c}\NormalTok{(}\DecValTok{0}\NormalTok{, }\DecValTok{100}\NormalTok{), }\AttributeTok{breaks =} \FunctionTok{seq}\NormalTok{(}\DecValTok{0}\NormalTok{, }\DecValTok{100}\NormalTok{, }\DecValTok{10}\NormalTok{)) }\SpecialCharTok{+}
\FunctionTok{scale\_y\_log10}\NormalTok{() }\SpecialCharTok{+}
\FunctionTok{scale\_color\_manual}\NormalTok{(}\AttributeTok{values =} \FunctionTok{c}\NormalTok{(}\StringTok{"red"}\NormalTok{, }\StringTok{"blue"}\NormalTok{, }\StringTok{"green"}\NormalTok{)) }\SpecialCharTok{+} \CommentTok{\# Cores personalizadas}
\FunctionTok{scale\_fill\_viridis\_d}\NormalTok{() }\SpecialCharTok{+}                                 \CommentTok{\# Paletas prontas}

\CommentTok{\#COORDENADAS (O espaço do gráfico)}
\CommentTok{\# Inverter eixos, fixar proporção, mapas ou polar}
\FunctionTok{coord\_flip}\NormalTok{() }\SpecialCharTok{+}           \CommentTok{\# Deita o gráfico (x vira y)}
\CommentTok{\# coord\_cartesian(ylim = c(0, 50)) + \# Zoom sem cortar dados}
\CommentTok{\# coord\_fixed(ratio = 1) +           \# Garante proporção 1:1}

\CommentTok{\#RÓTULOS E ANOTAÇÕES}
\FunctionTok{labs}\NormalTok{(}
  \AttributeTok{title =} \StringTok{"Título Principal"}\NormalTok{,}
  \AttributeTok{subtitle =} \StringTok{"Subtítulo explicativo"}\NormalTok{,}
  \AttributeTok{caption =} \StringTok{"Fonte: Base de dados X"}\NormalTok{,}
  \AttributeTok{x =} \StringTok{"Rótulo Eixo X"}\NormalTok{,}
  \AttributeTok{y =} \StringTok{"Rótulo Eixo Y"}\NormalTok{,}
  \AttributeTok{color =} \StringTok{"Título da Legenda de Cor"}\NormalTok{,}
  \AttributeTok{tag =} \StringTok{"Fig. A"} \CommentTok{\# Atribui letra de identificação a figura, }
                 \CommentTok{\#casos em em vai colocar dois ou mais graficos e quer identificar{-}os pode letra}
\NormalTok{) }\SpecialCharTok{+}

\CommentTok{\# Adicionar texto/setas manuais soltos no gráfico}
\FunctionTok{annotate}\NormalTok{(}\StringTok{"text"}\NormalTok{, }\AttributeTok{x =} \DecValTok{10}\NormalTok{, }\AttributeTok{y =} \DecValTok{50}\NormalTok{, }\AttributeTok{label =} \StringTok{"Ponto Crítico"}\NormalTok{, }\AttributeTok{color =} \StringTok{"red"}\NormalTok{) }\SpecialCharTok{+}

\CommentTok{\# GUIAS (Customização das Legendas)}
\CommentTok{\# Ajustes finos na aparência das legendas (remover, mudar linhas, etc)}
\FunctionTok{guides}\NormalTok{(}
  \AttributeTok{color =} \FunctionTok{guide\_legend}\NormalTok{(}\AttributeTok{override.aes =} \FunctionTok{list}\NormalTok{(}\AttributeTok{size =} \DecValTok{5}\NormalTok{)), }\CommentTok{\# Aumenta bolinha na legenda}
  \AttributeTok{fill =} \StringTok{"none"} \CommentTok{\# Remove a legenda de preenchimento}
\NormalTok{) }\SpecialCharTok{+}

\CommentTok{\# TEMA (Aparência Geral)}
\FunctionTok{theme\_minimal}\NormalTok{(}\AttributeTok{base\_size =} \DecValTok{14}\NormalTok{) }\SpecialCharTok{+}  \CommentTok{\# Comece com um tema pronto}

\CommentTok{\# AJUSTES DE TEMA (Personalização final)}
\CommentTok{\# Sobrescreve detalhes do tema escolhido acima}
\FunctionTok{theme}\NormalTok{(}
  \AttributeTok{legend.position =} \StringTok{"top"}\NormalTok{,             }\CommentTok{\# Posição da legenda (top, bottom, left, right, none)}
  \AttributeTok{plot.title =} \FunctionTok{element\_text}\NormalTok{(}\AttributeTok{face =} \StringTok{"bold"}\NormalTok{, }\AttributeTok{hjust =} \FloatTok{0.5}\NormalTok{), }\CommentTok{\# Título centralizado e negrito}
  \AttributeTok{axis.text.x =} \FunctionTok{element\_text}\NormalTok{(}\AttributeTok{angle =} \DecValTok{45}\NormalTok{, }\AttributeTok{hjust =} \DecValTok{1}\NormalTok{),     }\CommentTok{\# Rotação do texto do eixo X}
  \AttributeTok{panel.grid.minor =} \FunctionTok{element\_blank}\NormalTok{()   }\CommentTok{\# Remove grades menores}
\NormalTok{)}
\end{Highlighting}
\end{Shaded}

\textbf{Mapeamento vs.~Configuração}

A maior fonte de erros para iniciantes e intermediários é a distinção
entre estar \textbf{dentro} ou \textbf{fora} do \texttt{aes()}.

\begin{tcolorbox}[enhanced jigsaw, left=2mm, toptitle=1mm, colback=white, colframe=quarto-callout-important-color-frame, colbacktitle=quarto-callout-important-color!10!white, opacityback=0, rightrule=.15mm, bottomtitle=1mm, arc=.35mm, title=\textcolor{quarto-callout-important-color}{\faExclamation}\hspace{0.5em}{aes()}, titlerule=0mm, bottomrule=.15mm, leftrule=.75mm, coltitle=black, toprule=.15mm, breakable, opacitybacktitle=0.6]

\begin{itemize}
\tightlist
\item
  \textbf{Dentro do \texttt{aes()} (Mapeamento):} Conecta uma variável
  dos dados a uma propriedade visual, fazendo a aparência mudar
  dinamicamente conforme o valor (ex: cores diferentes para espécies
  diferentes). O \texttt{ggplot2} cria automaticamente uma legenda para
  explicar essa relação entre dado e visual.

  \begin{itemize}
  \tightlist
  \item
    Ex: \texttt{aes(color\ =\ Species)} - ``A cor muda conforme a
    Espécie''.
  \end{itemize}
\item
  \textbf{Fora do \texttt{aes()} (Configuração):} Aplica um estilo fixo
  e uniforme a todos os elementos da geometria, ignorando os valores dos
  dados (ex: pintar todos os pontos de azul). Não gera legenda, pois
  define apenas uma constante estética sem vínculo estatístico.

  \begin{itemize}
  \tightlist
  \item
    Ex: \texttt{geom\_point(color\ =\ "blue")} - ``Todos os pontos são
    azuis''.
  \end{itemize}
\end{itemize}

\end{tcolorbox}

\subsection{Geometrias (Geoms)}\label{geometrias-geoms}

Definem a forma do gráfico.

\textbf{Caso Univariado (Uma Variável)}

A. \textbf{Para Variáveis Contínuas (Números Reais)}

\emph{Ex: Salário, Idade, Temperatura, Altura.}

\begin{enumerate}
\def\labelenumi{\arabic{enumi}.}
\item
  \texttt{geom\_histogram()} Figura~\ref{fig-histograma}

  \begin{itemize}
  \tightlist
  \item
    Divide os dados em compartimentos (\emph{bins}) e conta a
    frequência. Útil para visualização da distribuição dos dados.
  \item
    \textbf{Parâmetros Chave:} \texttt{binwidth} (largura do intervalo)
    ou \texttt{bins} (quantidade de barras). Sempre teste larguras de
    \texttt{bins} diferentes; a escolha padrão (30) raramente é a ideal.
  \end{itemize}
\end{enumerate}

\begin{Shaded}
\begin{Highlighting}[]
\NormalTok{pacman}\SpecialCharTok{::}\FunctionTok{p\_load}\NormalTok{(ggplot2)}
\FunctionTok{ggplot}\NormalTok{(diamonds, }\FunctionTok{aes}\NormalTok{(}\AttributeTok{x =}\NormalTok{ price)) }\SpecialCharTok{+}
  \FunctionTok{geom\_histogram}\NormalTok{(}\AttributeTok{binwidth =} \DecValTok{1000}\NormalTok{, }\AttributeTok{fill =} \StringTok{"dodgerblue"}\NormalTok{, }\AttributeTok{color =} \StringTok{"white"}\NormalTok{) }\SpecialCharTok{+}
  \FunctionTok{theme\_minimal}\NormalTok{()}
\end{Highlighting}
\end{Shaded}

\begin{figure}[H]

\centering{

\pandocbounded{\includegraphics[keepaspectratio]{intro_files/figure-pdf/fig-histograma-1.pdf}}

}

\caption{\label{fig-histograma}Histograma do Preço de Diamantes}

\end{figure}%

\begin{enumerate}
\def\labelenumi{\arabic{enumi}.}
\setcounter{enumi}{1}
\item
  \texttt{geom\_density()} Figura~\ref{fig-density}

  \begin{itemize}
  \tightlist
  \item
    Estima a Função Densidade de Probabilidade (KDE). É um histograma
    ``suavizado''. Útil quando você quer ver a forma da distribuição sem
    o ruído das barras.
  \item
    \textbf{Parâmetros Chave:} \texttt{adjust} (suavização: \textless1 é
    detalhado, \textgreater1 é liso), \texttt{alpha} (transparência).
  \end{itemize}
\end{enumerate}

\begin{Shaded}
\begin{Highlighting}[]
\FunctionTok{ggplot}\NormalTok{(diamonds, }\FunctionTok{aes}\NormalTok{(}\AttributeTok{x =}\NormalTok{ price, }\AttributeTok{fill =}\NormalTok{ cut)) }\SpecialCharTok{+}
  \FunctionTok{geom\_density}\NormalTok{(}\AttributeTok{alpha =} \FloatTok{0.5}\NormalTok{) }\SpecialCharTok{+}
  \FunctionTok{theme\_minimal}\NormalTok{()}
\end{Highlighting}
\end{Shaded}

\begin{figure}[H]

\centering{

\pandocbounded{\includegraphics[keepaspectratio]{intro_files/figure-pdf/fig-density-1.pdf}}

}

\caption{\label{fig-density}Densidade suavizada}

\end{figure}%

\begin{enumerate}
\def\labelenumi{\arabic{enumi}.}
\setcounter{enumi}{2}
\item
  \textbf{\texttt{geom\_freqpoly()}} Figura~\ref{fig-freqpoly}

  \begin{itemize}
  \tightlist
  \item
    Calcula o mesmo que o histograma, mas desenha linhas conectando os
    topos das barras em vez das barras em si. Ideal para comparar
    distribuições de vários grupos sobrepostos (onde histogramas
    ficariam bagunçados).
  \end{itemize}
\end{enumerate}

\begin{Shaded}
\begin{Highlighting}[]
\FunctionTok{ggplot}\NormalTok{(diamonds, }\FunctionTok{aes}\NormalTok{(}\AttributeTok{x =}\NormalTok{ price, }\AttributeTok{color =}\NormalTok{ cut)) }\SpecialCharTok{+}
  \FunctionTok{geom\_histogram}\NormalTok{(}\AttributeTok{binwidth =} \DecValTok{1000}\NormalTok{) }\SpecialCharTok{+} \CommentTok{\#barras histograma}
  \FunctionTok{geom\_freqpoly}\NormalTok{(}\AttributeTok{binwidth =} \DecValTok{1000}\NormalTok{) }\SpecialCharTok{+} \CommentTok{\#linhas }
  \FunctionTok{theme\_minimal}\NormalTok{()}
\end{Highlighting}
\end{Shaded}

\begin{figure}[H]

\centering{

\pandocbounded{\includegraphics[keepaspectratio]{intro_files/figure-pdf/fig-freqpoly-1.pdf}}

}

\caption{\label{fig-freqpoly}Polígono de frequência comparando cortes}

\end{figure}%

\begin{enumerate}
\def\labelenumi{\arabic{enumi}.}
\setcounter{enumi}{3}
\item
  \texttt{geom\_dotplot()} Figura~\ref{fig-dotplot}

  \begin{itemize}
  \tightlist
  \item
    Empilha um ponto para cada observação. É ótimo para conjuntos de
    dados pequenos (\(N < 100\)) onde você quer mostrar cada indivíduo.
  \end{itemize}
\end{enumerate}

\begin{tcolorbox}[enhanced jigsaw, left=2mm, toptitle=1mm, colback=white, colframe=quarto-callout-important-color-frame, colbacktitle=quarto-callout-important-color!10!white, opacityback=0, rightrule=.15mm, bottomtitle=1mm, arc=.35mm, title=\textcolor{quarto-callout-important-color}{\faExclamation}\hspace{0.5em}{Atenção}, titlerule=0mm, bottomrule=.15mm, leftrule=.75mm, coltitle=black, toprule=.15mm, breakable, opacitybacktitle=0.6]

Requer \texttt{binaxis\ =\ "x"} e \texttt{stackdir\ =\ "center"} ou
\texttt{"up"} para funcionar bem como univariado.

\end{tcolorbox}

\begin{Shaded}
\begin{Highlighting}[]
\FunctionTok{ggplot}\NormalTok{(mtcars, }\FunctionTok{aes}\NormalTok{(}\AttributeTok{x =}\NormalTok{ mpg)) }\SpecialCharTok{+}
  \FunctionTok{geom\_dotplot}\NormalTok{(}\AttributeTok{binaxis =} \StringTok{"x"}\NormalTok{, }\AttributeTok{stackdir =} \StringTok{"center"}\NormalTok{, }\AttributeTok{dotsize =} \FloatTok{0.7}\NormalTok{) }\SpecialCharTok{+}
  \FunctionTok{theme\_minimal}\NormalTok{() }\CommentTok{\# Remove eixos para focar nos pontos}
\end{Highlighting}
\end{Shaded}

\begin{figure}[H]

\centering{

\pandocbounded{\includegraphics[keepaspectratio]{intro_files/figure-pdf/fig-dotplot-1.pdf}}

}

\caption{\label{fig-dotplot}Dotplot do consumo de combustivel}

\end{figure}%

\begin{enumerate}
\def\labelenumi{\arabic{enumi}.}
\setcounter{enumi}{4}
\item
  \texttt{stat\_ecdf()} (ou \texttt{geom\_step}) Figura~\ref{fig-ecdf}

  \begin{itemize}
  \tightlist
  \item
    Função de Distribuição Acumulada Empírica. Útil para pesponder
    perguntas como ``Qual porcentagem dos meus dados está abaixo do
    valor X?''. É a visualização mais estatisticamente íntegra, pois não
    depende de \emph{bins} ou suavização.
  \end{itemize}
\end{enumerate}

\begin{Shaded}
\begin{Highlighting}[]
\FunctionTok{ggplot}\NormalTok{(diamonds, }\FunctionTok{aes}\NormalTok{(}\AttributeTok{x =}\NormalTok{ price)) }\SpecialCharTok{+}
  \FunctionTok{stat\_ecdf}\NormalTok{(}\AttributeTok{geom =} \StringTok{"step"}\NormalTok{, }\AttributeTok{color =} \StringTok{"darkred"}\NormalTok{) }\SpecialCharTok{+}
  \FunctionTok{labs}\NormalTok{(}\AttributeTok{y =} \StringTok{"Probabilidade Acumulada"}\NormalTok{) }\SpecialCharTok{+}
  \FunctionTok{theme\_minimal}\NormalTok{()}
\end{Highlighting}
\end{Shaded}

\begin{figure}[H]

\centering{

\pandocbounded{\includegraphics[keepaspectratio]{intro_files/figure-pdf/fig-ecdf-1.pdf}}

}

\caption{\label{fig-ecdf}Distribuição Acumulada}

\end{figure}%

\begin{enumerate}
\def\labelenumi{\arabic{enumi}.}
\setcounter{enumi}{5}
\item
  \textbf{\texttt{geom\_rug()}} Figura~\ref{fig-rug}

  \begin{itemize}
  \tightlist
  \item
    Desenha pequenos riscos (tiques) nas margens dos eixos para cada
    dado existente (Veja eixo X da Figura~\ref{fig-rug} e
    Figura~\ref{fig-gratia}). Usado como complemento em histogramas ou
    densidades para mostrar onde estão os dados reais, especialmente
    para identificar \emph{outliers} ou clusters.
  \end{itemize}
\end{enumerate}

\begin{Shaded}
\begin{Highlighting}[]
\FunctionTok{ggplot}\NormalTok{(mtcars, }\FunctionTok{aes}\NormalTok{(}\AttributeTok{x =}\NormalTok{ wt)) }\SpecialCharTok{+}
  \FunctionTok{geom\_density}\NormalTok{(}\AttributeTok{fill =} \StringTok{"gray90"}\NormalTok{) }\SpecialCharTok{+}
  \FunctionTok{geom\_rug}\NormalTok{(}\AttributeTok{sides =} \StringTok{"b"}\NormalTok{, }\AttributeTok{length =} \FunctionTok{unit}\NormalTok{(}\FloatTok{0.2}\NormalTok{, }\StringTok{"cm"}\NormalTok{)) }\SpecialCharTok{+} \CommentTok{\# \textquotesingle{}b\textquotesingle{} = bottom (embaixo)}
  \FunctionTok{theme\_minimal}\NormalTok{()}
\end{Highlighting}
\end{Shaded}

\begin{figure}[H]

\centering{

\pandocbounded{\includegraphics[keepaspectratio]{intro_files/figure-pdf/fig-rug-1.pdf}}

}

\caption{\label{fig-rug}Densidade com Rug Plot na base}

\end{figure}%

B. \textbf{Para Variáveis Discretas (Categorias)}

\emph{Ex: Espécie, País, Mês, Sim/Não.}

\begin{enumerate}
\def\labelenumi{\arabic{enumi}.}
\setcounter{enumi}{6}
\item
  \texttt{geom\_bar()} Figura~\ref{fig-bar}

  \begin{itemize}
  \tightlist
  \item
    Conta quantas linhas existem para cada categoria. O \texttt{R} faz o
    trabalho de contar por você. Útil quando você tem a lista bruta de
    dados (ex: uma planilha com 1000 linhas onde a coluna ``Estado''
    repete SP, MG, RJ várias vezes).
  \item
    \textbf{Stat Padrão:} \texttt{count} (o R calcula a contagem).
  \item
    \textbf{Parâmetros Chave:} \texttt{fill} (cor interna das barras),
    \texttt{width} (largura das barras).
  \end{itemize}
\end{enumerate}

\begin{Shaded}
\begin{Highlighting}[]
\FunctionTok{ggplot}\NormalTok{(mtcars, }\FunctionTok{aes}\NormalTok{(}\AttributeTok{x =} \FunctionTok{factor}\NormalTok{(cyl))) }\SpecialCharTok{+}
  \FunctionTok{geom\_bar}\NormalTok{(}\AttributeTok{fill =} \StringTok{"steelblue"}\NormalTok{) }\SpecialCharTok{+}
  \FunctionTok{theme\_minimal}\NormalTok{()}
\end{Highlighting}
\end{Shaded}

\begin{figure}[H]

\centering{

\pandocbounded{\includegraphics[keepaspectratio]{intro_files/figure-pdf/fig-bar-1.pdf}}

}

\caption{\label{fig-bar}Contagem automática de carros por cilindro}

\end{figure}%

\begin{enumerate}
\def\labelenumi{\arabic{enumi}.}
\setcounter{enumi}{7}
\item
  \texttt{geom\_col()} Figura~\ref{fig-col}

  \begin{itemize}
  \tightlist
  \item
    Desenha barras com alturas baseadas em valores numéricos que já
    existem na sua tabela. Útil quando você já tem o resumo pronto (ex:
    uma tabela pequena com apenas 3 linhas: SP=500, MG=300, RJ=200).
  \item
    \textbf{Stat Padrão:} \texttt{identity} (o \texttt{R} usa o valor
    exatamente como ele é).
  \end{itemize}
\end{enumerate}

\begin{Shaded}
\begin{Highlighting}[]
\CommentTok{\# Criando uma tabela resumo fictícia}
\NormalTok{resumo }\OtherTok{\textless{}{-}} \FunctionTok{data.frame}\NormalTok{(}\AttributeTok{fruta =} \FunctionTok{c}\NormalTok{(}\StringTok{"Maçã"}\NormalTok{, }\StringTok{"Banana"}\NormalTok{), }\AttributeTok{valor =} \FunctionTok{c}\NormalTok{(}\DecValTok{20}\NormalTok{, }\DecValTok{30}\NormalTok{))}

\FunctionTok{ggplot}\NormalTok{(resumo, }\FunctionTok{aes}\NormalTok{(}\AttributeTok{x =}\NormalTok{ fruta, }\AttributeTok{y =}\NormalTok{ valor)) }\SpecialCharTok{+}
  \FunctionTok{geom\_col}\NormalTok{(}\AttributeTok{fill =} \StringTok{"orange"}\NormalTok{) }\SpecialCharTok{+}
  \FunctionTok{theme\_minimal}\NormalTok{()}
\end{Highlighting}
\end{Shaded}

\begin{figure}[H]

\centering{

\pandocbounded{\includegraphics[keepaspectratio]{intro_files/figure-pdf/fig-col-1.pdf}}

}

\caption{\label{fig-col}Barras com valores pré-definidos}

\end{figure}%

\begin{tcolorbox}[enhanced jigsaw, left=2mm, toptitle=1mm, colback=white, colframe=quarto-callout-tip-color-frame, colbacktitle=quarto-callout-tip-color!10!white, opacityback=0, rightrule=.15mm, bottomtitle=1mm, arc=.35mm, title=\textcolor{quarto-callout-tip-color}{\faLightbulb}\hspace{0.5em}{Bar vs Col}, titlerule=0mm, bottomrule=.15mm, leftrule=.75mm, coltitle=black, toprule=.15mm, breakable, opacitybacktitle=0.6]

\begin{itemize}
\tightlist
\item
  Use \texttt{geom\_bar()} se você tiver dados brutos e quiser que o
  \texttt{R} conte.
\item
  Use \texttt{geom\_col()} se você já tiver uma tabela resumo com os
  totais.
\end{itemize}

\end{tcolorbox}

C. \textbf{Gráficos de Resumo e Distribuição (Pseudo-Univariados)}

Embora geralmente usados para comparar grupos (Bivariados), eles são
excelentes para analisar uma única variável numérica se você mapear
\texttt{x\ =\ ""} ou \texttt{x\ =\ 1}.

\begin{enumerate}
\def\labelenumi{\arabic{enumi}.}
\item
  \texttt{geom\_boxplot()} Figura~\ref{fig-boxplot}

  \begin{itemize}
  \tightlist
  \item
    O diagrama de caixa. Mostra a Mediana (linha central), o Intervalo
    Interquartil (a caixa, de 25\% a 75\%) e os \emph{outliers} (pontos
    além dos bigodes). Útil para para detectar valores extremos,
    distribuição, assimetria, etc.
  \item
    \textbf{Parâmetros Chave:} \texttt{outlier.color} (destacar os
    outliers), \texttt{notch\ =\ TRUE} (adiciona um entalhe na mediana
    para comparação visual de significância).
  \end{itemize}
\end{enumerate}

\begin{Shaded}
\begin{Highlighting}[]
\FunctionTok{ggplot}\NormalTok{(mtcars, }\FunctionTok{aes}\NormalTok{(}\AttributeTok{x =} \FunctionTok{factor}\NormalTok{(cyl), }\AttributeTok{y =}\NormalTok{ hp)) }\SpecialCharTok{+}
  \FunctionTok{geom\_boxplot}\NormalTok{(}\AttributeTok{outlier.color =} \StringTok{"red"}\NormalTok{, }\AttributeTok{outlier.shape =} \DecValTok{8}\NormalTok{) }\SpecialCharTok{+}
  \FunctionTok{theme\_bw}\NormalTok{()}
\end{Highlighting}
\end{Shaded}

\begin{figure}[H]

\centering{

\pandocbounded{\includegraphics[keepaspectratio]{intro_files/figure-pdf/fig-boxplot-1.pdf}}

}

\caption{\label{fig-boxplot}Boxplot da potência por cilindro}

\end{figure}%

\begin{enumerate}
\def\labelenumi{\arabic{enumi}.}
\setcounter{enumi}{1}
\item
  \texttt{geom\_violin()} Figura~\ref{fig-violin}

  \begin{itemize}
  \tightlist
  \item
    Um espelhamento do \texttt{geom\_density} (densidade) rotacionado em
    90 graus. Mostra a forma completa da distribuição de maneira
    compacta. É mais rico que o boxplot porque revela se a distribuição
    é bimodal (tem ``duas corcovas''), algo que o boxplot esconde.
  \item
    \textbf{Parâmetros Chave:} \texttt{draw\_quantiles} (desenha linhas
    nos quartis dentro do violino).
  \end{itemize}
\end{enumerate}

\begin{Shaded}
\begin{Highlighting}[]
\FunctionTok{ggplot}\NormalTok{(diamonds[}\DecValTok{1}\SpecialCharTok{:}\DecValTok{500}\NormalTok{,], }\FunctionTok{aes}\NormalTok{(}\AttributeTok{x =}\NormalTok{ cut, }\AttributeTok{y =}\NormalTok{ price)) }\SpecialCharTok{+}
  \FunctionTok{geom\_violin}\NormalTok{(}\AttributeTok{draw\_quantiles =} \FunctionTok{c}\NormalTok{(}\FloatTok{0.25}\NormalTok{, }\FloatTok{0.5}\NormalTok{, }\FloatTok{0.75}\NormalTok{), }\AttributeTok{fill =} \StringTok{"lightblue"}\NormalTok{) }\SpecialCharTok{+}
  \FunctionTok{theme\_classic}\NormalTok{()}
\end{Highlighting}
\end{Shaded}

\begin{figure}[H]

\centering{

\pandocbounded{\includegraphics[keepaspectratio]{intro_files/figure-pdf/fig-violin-1.pdf}}

}

\caption{\label{fig-violin}Violin plot comparando distribuições}

\end{figure}%

\begin{enumerate}
\def\labelenumi{\arabic{enumi}.}
\setcounter{enumi}{1}
\item
  \texttt{geom\_jitter()} Figura~\ref{fig-jitter}

  \begin{itemize}
  \tightlist
  \item
    Adiciona um pequeno ruído aleatório aos pontos para evitar
    sobreposição (\emph{overplotting}). Frequentemente usado por cima do
    \texttt{geom\_boxplot()} para mostrar os dados brutos reais junto
    com o resumo estatístico.
  \item
    \textbf{Parâmetros Chave:} \texttt{width} e \texttt{height}
    (controlam o quanto os pontos podem ``tremer'').
  \end{itemize}
\end{enumerate}

\begin{Shaded}
\begin{Highlighting}[]
\FunctionTok{ggplot}\NormalTok{(mtcars, }\FunctionTok{aes}\NormalTok{(}\AttributeTok{x =} \FunctionTok{factor}\NormalTok{(cyl), }\AttributeTok{y =}\NormalTok{ hp)) }\SpecialCharTok{+}
  \FunctionTok{geom\_boxplot}\NormalTok{(}\AttributeTok{alpha =} \FloatTok{0.3}\NormalTok{) }\SpecialCharTok{+}
  \FunctionTok{geom\_jitter}\NormalTok{(}\AttributeTok{width =} \FloatTok{0.2}\NormalTok{, }\AttributeTok{color =} \StringTok{"darkblue"}\NormalTok{) }\SpecialCharTok{+} \CommentTok{\# width controla a \textquotesingle{}tremida\textquotesingle{} horizontal}
  \FunctionTok{theme\_minimal}\NormalTok{()}
\end{Highlighting}
\end{Shaded}

\begin{figure}[H]

\centering{

\pandocbounded{\includegraphics[keepaspectratio]{intro_files/figure-pdf/fig-jitter-1.pdf}}

}

\caption{\label{fig-jitter}Boxplot + Jitter (Dados brutos)}

\end{figure}%

\textbf{Bivariada: Contínua X Contínua} \emph{Ex: Peso vs Altura, Preço
vs Quilometragem.}

\begin{enumerate}
\def\labelenumi{\arabic{enumi}.}
\tightlist
\item
  \texttt{geom\_point()} Figura~\ref{fig-point}

  \begin{itemize}
  \tightlist
  \item
    O clássico gráfico de dispersão (Scatterplot). Mostra a relação
    exata entre duas variáveis.
  \end{itemize}
\end{enumerate}

\begin{Shaded}
\begin{Highlighting}[]
\FunctionTok{ggplot}\NormalTok{(mtcars, }\FunctionTok{aes}\NormalTok{(}\AttributeTok{x =}\NormalTok{ wt, }\AttributeTok{y =}\NormalTok{ mpg)) }\SpecialCharTok{+}
  \FunctionTok{geom\_point}\NormalTok{(}\AttributeTok{size =} \DecValTok{3}\NormalTok{, }\AttributeTok{color =} \StringTok{"darkgreen"}\NormalTok{) }\SpecialCharTok{+}
  \FunctionTok{theme\_minimal}\NormalTok{()}
\end{Highlighting}
\end{Shaded}

\begin{figure}[H]

\centering{

\pandocbounded{\includegraphics[keepaspectratio]{intro_files/figure-pdf/fig-point-1.pdf}}

}

\caption{\label{fig-point}Scatterplot clássico}

\end{figure}%

\begin{enumerate}
\def\labelenumi{\arabic{enumi}.}
\setcounter{enumi}{1}
\item
  \texttt{geom\_jitter()} Figura~\ref{fig-smooth}
\item
  \texttt{geom\_smooth()}

  \begin{itemize}
  \tightlist
  \item
    Adiciona uma linha de tendência (regressão linear, LOESS, GAM).
    Ajuda o olho a ver o padrão no meio dos pontos.
  \end{itemize}
\end{enumerate}

\begin{Shaded}
\begin{Highlighting}[]
\FunctionTok{ggplot}\NormalTok{(mtcars, }\FunctionTok{aes}\NormalTok{(}\AttributeTok{x =}\NormalTok{ wt, }\AttributeTok{y =}\NormalTok{ mpg)) }\SpecialCharTok{+}
  \FunctionTok{geom\_point}\NormalTok{() }\SpecialCharTok{+}
  \FunctionTok{geom\_smooth}\NormalTok{(}\AttributeTok{method =} \StringTok{"lm"}\NormalTok{, }\AttributeTok{se =} \ConstantTok{TRUE}\NormalTok{, }\AttributeTok{color =} \StringTok{"red"}\NormalTok{) }\SpecialCharTok{+} \CommentTok{\# se=TRUE mostra intervalo de confiança}
  \FunctionTok{theme\_minimal}\NormalTok{()}
\end{Highlighting}
\end{Shaded}

\begin{figure}[H]

\centering{

\pandocbounded{\includegraphics[keepaspectratio]{intro_files/figure-pdf/fig-smooth-1.pdf}}

}

\caption{\label{fig-smooth}Scatterplot com linha de tendência linear}

\end{figure}%

\begin{enumerate}
\def\labelenumi{\arabic{enumi}.}
\setcounter{enumi}{3}
\tightlist
\item
  \texttt{geom\_quantile()} Figura~\ref{fig-quantile}

  \begin{itemize}
  \tightlist
  \item
    Regressão quantílica. Em vez da média (como o \texttt{smooth}),
    mostra tendências para os 25\%, 50\% e 75\% mais extremos.
  \end{itemize}
\end{enumerate}

\begin{Shaded}
\begin{Highlighting}[]
\FunctionTok{ggplot}\NormalTok{(mpg, }\FunctionTok{aes}\NormalTok{(}\AttributeTok{x =}\NormalTok{ displ, }\AttributeTok{y =}\NormalTok{ hwy)) }\SpecialCharTok{+}
  \FunctionTok{geom\_point}\NormalTok{(}\AttributeTok{alpha =} \FloatTok{0.3}\NormalTok{) }\SpecialCharTok{+}
  
  \CommentTok{\# Regressão Linear (Média) em Vermelho para comparação}
  \FunctionTok{geom\_smooth}\NormalTok{(}\AttributeTok{method =} \StringTok{"lm"}\NormalTok{, }\AttributeTok{se =} \ConstantTok{FALSE}\NormalTok{, }\AttributeTok{color =} \StringTok{"red"}\NormalTok{) }\SpecialCharTok{+}
  
  \CommentTok{\# Regressão Quantílica (25\%, 50\%, 75\%) em Azul}
  \FunctionTok{geom\_quantile}\NormalTok{(}\AttributeTok{quantiles =} \FunctionTok{c}\NormalTok{(}\FloatTok{0.25}\NormalTok{, }\FloatTok{0.5}\NormalTok{, }\FloatTok{0.75}\NormalTok{), }\AttributeTok{color =} \StringTok{"blue"}\NormalTok{, }\AttributeTok{size =} \DecValTok{1}\NormalTok{) }\SpecialCharTok{+}
  
  \FunctionTok{theme\_minimal}\NormalTok{() }\SpecialCharTok{+}
  \FunctionTok{labs}\NormalTok{(}\AttributeTok{title =} \StringTok{"Tendências dos Extremos vs Média"}\NormalTok{,}
       \AttributeTok{subtitle =} \StringTok{"Azul: Quantis (0.25, 0.50, 0.75) | Vermelho: Média (LM)"}\NormalTok{)}
\end{Highlighting}
\end{Shaded}

\begin{figure}[H]

\centering{

\pandocbounded{\includegraphics[keepaspectratio]{intro_files/figure-pdf/fig-quantile-1.pdf}}

}

\caption{\label{fig-quantile}Regressão Quantílica (25\%, 50\%, 75\%) vs
Regressão Linear (Vermelho)}

\end{figure}%

\begin{enumerate}
\def\labelenumi{\arabic{enumi}.}
\setcounter{enumi}{4}
\tightlist
\item
  \texttt{geom\_text()} e \texttt{geom\_label()}
  Figura~\ref{fig-text_label}

  \begin{itemize}
  \tightlist
  \item
    Escrevem texto no gráfico nas coordenadas X e Y.
  \item
    \textbf{Diferença:} \texttt{geom\_label} desenha uma caixinha
    colorida atrás do texto para facilitar a leitura;
    \texttt{geom\_text} desenha apenas as letras.
  \end{itemize}
\end{enumerate}

\begin{Shaded}
\begin{Highlighting}[]
\NormalTok{df }\OtherTok{\textless{}{-}} \FunctionTok{data.frame}\NormalTok{(}\AttributeTok{x =} \FunctionTok{c}\NormalTok{(}\DecValTok{1}\NormalTok{, }\DecValTok{3}\NormalTok{), }\AttributeTok{y =} \FunctionTok{c}\NormalTok{(}\DecValTok{1}\NormalTok{, }\DecValTok{1}\NormalTok{), }\AttributeTok{texto =} \FunctionTok{c}\NormalTok{(}\StringTok{"Texto Puro"}\NormalTok{, }\StringTok{"Etiqueta"}\NormalTok{))}

\FunctionTok{ggplot}\NormalTok{(df, }\FunctionTok{aes}\NormalTok{(}\AttributeTok{x =}\NormalTok{ x, }\AttributeTok{y =}\NormalTok{ y, }\AttributeTok{label =}\NormalTok{ texto)) }\SpecialCharTok{+}
  \FunctionTok{xlim}\NormalTok{(}\DecValTok{0}\NormalTok{, }\DecValTok{4}\NormalTok{) }\SpecialCharTok{+} \FunctionTok{ylim}\NormalTok{(}\DecValTok{0}\NormalTok{, }\DecValTok{2}\NormalTok{) }\SpecialCharTok{+}
  \CommentTok{\# Lado Esquerdo: Texto solto}
  \FunctionTok{geom\_text}\NormalTok{(}\AttributeTok{data =} \FunctionTok{subset}\NormalTok{(df, x }\SpecialCharTok{==} \DecValTok{1}\NormalTok{), }\AttributeTok{size =} \DecValTok{6}\NormalTok{) }\SpecialCharTok{+}
  \CommentTok{\# Lado Direito: Texto dentro da caixinha}
  \FunctionTok{geom\_label}\NormalTok{(}\AttributeTok{data =} \FunctionTok{subset}\NormalTok{(df, x }\SpecialCharTok{==} \DecValTok{3}\NormalTok{), }\AttributeTok{size =} \DecValTok{6}\NormalTok{, }\AttributeTok{fill =} \StringTok{"white"}\NormalTok{) }\SpecialCharTok{+}
  \FunctionTok{theme\_bw}\NormalTok{()}
\end{Highlighting}
\end{Shaded}

\begin{figure}[H]

\centering{

\pandocbounded{\includegraphics[keepaspectratio]{intro_files/figure-pdf/fig-text_label-1.pdf}}

}

\caption{\label{fig-text_label}Diferença entre geom\_text e geom\_label}

\end{figure}%

\begin{enumerate}
\def\labelenumi{\arabic{enumi}.}
\setcounter{enumi}{5}
\tightlist
\item
  \texttt{geom\_bin2d()} e \texttt{geom\_hex()} Figura~\ref{fig-hex}

  \begin{itemize}
  \tightlist
  \item
    Quando você tem milhões de pontos, o scatterplot vira uma mancha
    preta. Estes \texttt{geoms} dividem o plano em quadrados
    (\texttt{bin2d}) ou hexágonos (\texttt{hex}) e pintam a cor baseada
    na contagem de pontos naquela região (mapa de calor 2D).
  \end{itemize}
\end{enumerate}

\begin{Shaded}
\begin{Highlighting}[]
\NormalTok{pacman}\SpecialCharTok{::}\FunctionTok{p\_load}\NormalTok{(hexbin)}

\FunctionTok{ggplot}\NormalTok{(diamonds, }\FunctionTok{aes}\NormalTok{(}\AttributeTok{x =}\NormalTok{ carat, }\AttributeTok{y =}\NormalTok{ price)) }\SpecialCharTok{+}
  \FunctionTok{geom\_hex}\NormalTok{(}\AttributeTok{bins =} \DecValTok{30}\NormalTok{) }\SpecialCharTok{+}
  \FunctionTok{scale\_fill\_viridis\_c}\NormalTok{() }\SpecialCharTok{+}
  \FunctionTok{theme\_minimal}\NormalTok{()}
\end{Highlighting}
\end{Shaded}

\begin{figure}[H]

\centering{

\pandocbounded{\includegraphics[keepaspectratio]{intro_files/figure-pdf/fig-hex-1.pdf}}

}

\caption{\label{fig-hex}Mapa de calor hexagonal (Hexbin)}

\end{figure}%

\textbf{Bivariada: Discreta X Contínua}

\emph{Ex: Grupo (A, B) vs Nota.}

Além dos já citados \texttt{geom\_boxplot}, \texttt{geom\_violin} e
\texttt{geom\_col}:

\begin{enumerate}
\def\labelenumi{\arabic{enumi}.}
\setcounter{enumi}{6}
\tightlist
\item
  \texttt{geom\_dotplot(binaxis\ =\ "y")} Figura~\ref{fig-dotplot_bi}

  \begin{itemize}
  \tightlist
  \item
    Semelhante ao violino, mas feito de bolinhas empilhadas.
  \end{itemize}
\end{enumerate}

\begin{Shaded}
\begin{Highlighting}[]
\FunctionTok{ggplot}\NormalTok{(mtcars, }\FunctionTok{aes}\NormalTok{(}\AttributeTok{x =} \FunctionTok{factor}\NormalTok{(cyl), }\AttributeTok{y =}\NormalTok{ mpg, }\AttributeTok{fill =} \FunctionTok{factor}\NormalTok{(cyl))) }\SpecialCharTok{+}
  \CommentTok{\# binaxis = "y" faz as bolinhas subirem no eixo Y}
  \CommentTok{\# stackdir = "center" empilha elas para os dois lados (parece um violino)}
  \FunctionTok{geom\_dotplot}\NormalTok{(}\AttributeTok{binaxis =} \StringTok{"y"}\NormalTok{, }\AttributeTok{stackdir =} \StringTok{"center"}\NormalTok{, }\AttributeTok{dotsize =} \FloatTok{0.8}\NormalTok{) }\SpecialCharTok{+}
  \FunctionTok{theme\_minimal}\NormalTok{() }\SpecialCharTok{+}
  \FunctionTok{theme}\NormalTok{(}\AttributeTok{legend.position =} \StringTok{"none"}\NormalTok{) }\CommentTok{\# Remove a legenda pois a cor já indica os cilindros}
\end{Highlighting}
\end{Shaded}

\begin{figure}[H]

\centering{

\pandocbounded{\includegraphics[keepaspectratio]{intro_files/figure-pdf/fig-dotplot_bi-1.pdf}}

}

\caption{\label{fig-dotplot_bi}Dotplot Bivariado: Distribuição do
consumo por cilindros}

\end{figure}%

\subsection{Séries Temporais e Funções
(Evolução)}\label{suxe9ries-temporais-e-funuxe7uxf5es-evoluuxe7uxe3o}

\emph{Ex: Data no Eixo X.}

\begin{enumerate}
\def\labelenumi{\arabic{enumi}.}
\tightlist
\item
  \texttt{geom\_line()} Figura~\ref{fig-line}

  \begin{itemize}
  \tightlist
  \item
    Conecta os pontos na ordem da variável X. Padrão para séries
    temporais.
  \end{itemize}
\end{enumerate}

\begin{Shaded}
\begin{Highlighting}[]
\NormalTok{economics }\SpecialCharTok{|\textgreater{}} 
  \FunctionTok{ggplot}\NormalTok{(}\FunctionTok{aes}\NormalTok{(}\AttributeTok{x =}\NormalTok{ date, }\AttributeTok{y =}\NormalTok{ unemploy)) }\SpecialCharTok{+}
  \FunctionTok{geom\_line}\NormalTok{(}\AttributeTok{color =} \StringTok{"blue"}\NormalTok{) }\SpecialCharTok{+}
  \FunctionTok{theme\_minimal}\NormalTok{()}
\end{Highlighting}
\end{Shaded}

\begin{figure}[H]

\centering{

\pandocbounded{\includegraphics[keepaspectratio]{intro_files/figure-pdf/fig-line-1.pdf}}

}

\caption{\label{fig-line}Série temporal simples}

\end{figure}%

\begin{enumerate}
\def\labelenumi{\arabic{enumi}.}
\setcounter{enumi}{1}
\tightlist
\item
  \texttt{geom\_path()} Figura~\ref{fig-path}

  \begin{itemize}
  \tightlist
  \item
    Conecta os pontos na ordem em que aparecem na tabela (mesmo que o X
    volte para trás). Usado para desenhar trajetórias em mapas ou formas
    complexas.
  \end{itemize}
\end{enumerate}

\begin{Shaded}
\begin{Highlighting}[]
\CommentTok{\# Note que a linha faz voltas e loops, o que seria impossível com geom\_line()}
\CommentTok{\# geom\_path conecta os dados cronologicamente}
\FunctionTok{ggplot}\NormalTok{(economics, }\FunctionTok{aes}\NormalTok{(}\AttributeTok{x =}\NormalTok{ uempmed, }\AttributeTok{y =}\NormalTok{ psavert)) }\SpecialCharTok{+}
  \FunctionTok{geom\_path}\NormalTok{(}\AttributeTok{color =} \StringTok{"purple"}\NormalTok{) }\SpecialCharTok{+}
  \FunctionTok{labs}\NormalTok{(}
    \AttributeTok{title =} \StringTok{"Trajetória Econômica (1967{-}2015)"}\NormalTok{,}
    \AttributeTok{subtitle =} \StringTok{"A linha conecta os pontos na ordem cronológica"}\NormalTok{,}
    \AttributeTok{x =} \StringTok{"Duração do Desemprego (Mediana)"}\NormalTok{,}
    \AttributeTok{y =} \StringTok{"Taxa de Poupança Pessoal"}
\NormalTok{  ) }\SpecialCharTok{+}
  \FunctionTok{theme\_minimal}\NormalTok{()}
\end{Highlighting}
\end{Shaded}

\begin{figure}[H]

\centering{

\pandocbounded{\includegraphics[keepaspectratio]{intro_files/figure-pdf/fig-path-1.pdf}}

}

\caption{\label{fig-path}Geom Path: Ciclo Econômico (Desemprego vs
Poupança)}

\end{figure}%

\begin{enumerate}
\def\labelenumi{\arabic{enumi}.}
\setcounter{enumi}{2}
\tightlist
\item
  \texttt{geom\_step()} Figuar Figura~\ref{fig-step}

  \begin{itemize}
  \tightlist
  \item
    Gráfico de ``escada''. A linha só muda de nível abruptamente. Bom
    para visualizar taxas de juros ou estoques que mudam em saltos.
  \end{itemize}
\end{enumerate}

\begin{Shaded}
\begin{Highlighting}[]
\CommentTok{\# Criando dados fictícios de uma taxa que muda pouco}
\NormalTok{dados\_step }\OtherTok{\textless{}{-}} \FunctionTok{data.frame}\NormalTok{(}
  \AttributeTok{mes =} \DecValTok{1}\SpecialCharTok{:}\DecValTok{12}\NormalTok{,}
  \AttributeTok{taxa =} \FunctionTok{c}\NormalTok{(}\DecValTok{2}\NormalTok{, }\DecValTok{2}\NormalTok{, }\DecValTok{2}\NormalTok{, }\DecValTok{5}\NormalTok{, }\DecValTok{5}\NormalTok{, }\DecValTok{5}\NormalTok{, }\DecValTok{5}\NormalTok{, }\DecValTok{3}\NormalTok{, }\DecValTok{3}\NormalTok{, }\DecValTok{6}\NormalTok{, }\DecValTok{6}\NormalTok{, }\DecValTok{6}\NormalTok{)}
\NormalTok{)}

\FunctionTok{ggplot}\NormalTok{(dados\_step, }\FunctionTok{aes}\NormalTok{(}\AttributeTok{x =}\NormalTok{ mes, }\AttributeTok{y =}\NormalTok{ taxa)) }\SpecialCharTok{+}
  \FunctionTok{geom\_step}\NormalTok{(}\AttributeTok{color =} \StringTok{"darkblue"}\NormalTok{, }\AttributeTok{size =} \DecValTok{1}\NormalTok{) }\SpecialCharTok{+}
  \FunctionTok{geom\_point}\NormalTok{(}\AttributeTok{color =} \StringTok{"red"}\NormalTok{) }\SpecialCharTok{+} \CommentTok{\# Adicionando pontos para mostrar quando a medição ocorreu}
  \FunctionTok{labs}\NormalTok{(}\AttributeTok{title =} \StringTok{"Evolução da Taxa (Gráfico de Escada)"}\NormalTok{) }\SpecialCharTok{+}
  \FunctionTok{scale\_x\_continuous}\NormalTok{(}\AttributeTok{breaks =} \DecValTok{1}\SpecialCharTok{:}\DecValTok{12}\NormalTok{) }\SpecialCharTok{+}
  \FunctionTok{theme\_bw}\NormalTok{()}
\end{Highlighting}
\end{Shaded}

\begin{figure}[H]

\centering{

\pandocbounded{\includegraphics[keepaspectratio]{intro_files/figure-pdf/fig-step-1.pdf}}

}

\caption{\label{fig-step}Geom Step: Mudanças na Taxa de Juros
(Simulado)}

\end{figure}%

\begin{enumerate}
\def\labelenumi{\arabic{enumi}.}
\setcounter{enumi}{3}
\tightlist
\item
  \texttt{geom\_area()} Figura~\ref{fig-area}

  \begin{itemize}
  \tightlist
  \item
    Um gráfico de linha onde a área abaixo dela (até o 0) é preenchida.
  \end{itemize}
\end{enumerate}

\begin{Shaded}
\begin{Highlighting}[]
\NormalTok{economics }\SpecialCharTok{|\textgreater{}} 
  \FunctionTok{ggplot}\NormalTok{(}\FunctionTok{aes}\NormalTok{(}\AttributeTok{x =}\NormalTok{ date, }\AttributeTok{y =}\NormalTok{ unemploy)) }\SpecialCharTok{+}
  \FunctionTok{geom\_area}\NormalTok{(}\AttributeTok{fill =} \StringTok{"lightblue"}\NormalTok{, }\AttributeTok{alpha =} \FloatTok{0.6}\NormalTok{) }\SpecialCharTok{+}
  \FunctionTok{theme\_minimal}\NormalTok{()}
\end{Highlighting}
\end{Shaded}

\begin{figure}[H]

\centering{

\pandocbounded{\includegraphics[keepaspectratio]{intro_files/figure-pdf/fig-area-1.pdf}}

}

\caption{\label{fig-area}Gráfico de área}

\end{figure}%

\begin{enumerate}
\def\labelenumi{\arabic{enumi}.}
\setcounter{enumi}{4}
\tightlist
\item
  \texttt{geom\_ribbon()} Figura~\ref{fig-ribbon}

  \begin{itemize}
  \tightlist
  \item
    Desenha uma faixa colorida entre um valor mínimo (\texttt{ymin}) e
    máximo (\texttt{ymax}). Fundamental para desenhar Intervalos de
    Confiança ao redor de uma linha.
  \end{itemize}
\end{enumerate}

\begin{Shaded}
\begin{Highlighting}[]
\CommentTok{\# Criando dados fictícios de previsão}
\NormalTok{df\_ribbon }\OtherTok{\textless{}{-}} \FunctionTok{data.frame}\NormalTok{(}
  \AttributeTok{ano =} \DecValTok{2000}\SpecialCharTok{:}\DecValTok{2010}\NormalTok{,}
  \AttributeTok{valor =} \DecValTok{10}\SpecialCharTok{:}\DecValTok{20}\NormalTok{,}
  \AttributeTok{min =}\NormalTok{ (}\DecValTok{10}\SpecialCharTok{:}\DecValTok{20}\NormalTok{) }\SpecialCharTok{{-}} \DecValTok{2}\NormalTok{,}
  \AttributeTok{max =}\NormalTok{ (}\DecValTok{10}\SpecialCharTok{:}\DecValTok{20}\NormalTok{) }\SpecialCharTok{+} \DecValTok{2}
\NormalTok{)}

\FunctionTok{ggplot}\NormalTok{(df\_ribbon, }\FunctionTok{aes}\NormalTok{(}\AttributeTok{x =}\NormalTok{ ano, }\AttributeTok{y =}\NormalTok{ valor)) }\SpecialCharTok{+}
  \FunctionTok{geom\_ribbon}\NormalTok{(}\FunctionTok{aes}\NormalTok{(}\AttributeTok{ymin =}\NormalTok{ min, }\AttributeTok{ymax =}\NormalTok{ max), }\AttributeTok{fill =} \StringTok{"grey80"}\NormalTok{) }\SpecialCharTok{+}
  \FunctionTok{geom\_line}\NormalTok{() }\SpecialCharTok{+}
  \FunctionTok{theme\_minimal}\NormalTok{()}
\end{Highlighting}
\end{Shaded}

\begin{figure}[H]

\centering{

\pandocbounded{\includegraphics[keepaspectratio]{intro_files/figure-pdf/fig-ribbon-1.pdf}}

}

\caption{\label{fig-ribbon}Ribbon (faixa) ao redor de uma linha}

\end{figure}%

\textbf{Visualização de Incerteza e Erro} \emph{Ex: Médias com desvio
padrão.}

\begin{enumerate}
\def\labelenumi{\arabic{enumi}.}
\tightlist
\item
  \texttt{geom\_errorbar()} Figura~\ref{fig-errorbar}

  \begin{itemize}
  \tightlist
  \item
    A clássica barra de erro (formato ``I'').
  \end{itemize}
\end{enumerate}

\begin{Shaded}
\begin{Highlighting}[]
\NormalTok{df\_erro }\OtherTok{\textless{}{-}} \FunctionTok{data.frame}\NormalTok{(}\AttributeTok{grp =} \FunctionTok{c}\NormalTok{(}\StringTok{"A"}\NormalTok{, }\StringTok{"B"}\NormalTok{), }\AttributeTok{media =} \FunctionTok{c}\NormalTok{(}\DecValTok{10}\NormalTok{, }\DecValTok{15}\NormalTok{), }\AttributeTok{sd =} \FunctionTok{c}\NormalTok{(}\DecValTok{1}\NormalTok{, }\DecValTok{2}\NormalTok{))}

\NormalTok{p1 }\OtherTok{\textless{}{-}} \FunctionTok{ggplot}\NormalTok{(df\_erro, }\FunctionTok{aes}\NormalTok{(}\AttributeTok{x =}\NormalTok{ grp, }\AttributeTok{y =}\NormalTok{ media)) }\SpecialCharTok{+}
  \FunctionTok{geom\_point}\NormalTok{() }\SpecialCharTok{+}
  \FunctionTok{geom\_errorbar}\NormalTok{(}\FunctionTok{aes}\NormalTok{(}\AttributeTok{ymin =}\NormalTok{ media}\SpecialCharTok{{-}}\NormalTok{sd, }\AttributeTok{ymax =}\NormalTok{ media}\SpecialCharTok{+}\NormalTok{sd), }\AttributeTok{width =} \FloatTok{0.2}\NormalTok{) }\SpecialCharTok{+}
  \FunctionTok{ggtitle}\NormalTok{(}\StringTok{"Errorbar"}\NormalTok{)}\SpecialCharTok{+}
  \FunctionTok{theme\_minimal}\NormalTok{()}

\NormalTok{p2 }\OtherTok{\textless{}{-}} \FunctionTok{ggplot}\NormalTok{(df\_erro, }\FunctionTok{aes}\NormalTok{(}\AttributeTok{x =}\NormalTok{ grp, }\AttributeTok{y =}\NormalTok{ media)) }\SpecialCharTok{+}
  \FunctionTok{geom\_pointrange}\NormalTok{(}\FunctionTok{aes}\NormalTok{(}\AttributeTok{ymin =}\NormalTok{ media}\SpecialCharTok{{-}}\NormalTok{sd, }\AttributeTok{ymax =}\NormalTok{ media}\SpecialCharTok{+}\NormalTok{sd)) }\SpecialCharTok{+}
  \FunctionTok{ggtitle}\NormalTok{(}\StringTok{"Pointrange"}\NormalTok{)}\SpecialCharTok{+}
  \FunctionTok{theme\_minimal}\NormalTok{()}

\NormalTok{pacman}\SpecialCharTok{::}\FunctionTok{p\_load}\NormalTok{(patchwork)}
\NormalTok{p1 }\SpecialCharTok{+}\NormalTok{ p2}
\end{Highlighting}
\end{Shaded}

\begin{figure}[H]

\centering{

\pandocbounded{\includegraphics[keepaspectratio]{intro_files/figure-pdf/fig-errorbar-1.pdf}}

}

\caption{\label{fig-errorbar}Barras de erro e Pointrange}

\end{figure}%

\begin{enumerate}
\def\labelenumi{\arabic{enumi}.}
\setcounter{enumi}{1}
\tightlist
\item
  \texttt{geom\_linerange()} Figura~\ref{fig-linerange}

  \begin{itemize}
  \tightlist
  \item
    Uma linha vertical simples indicando o intervalo (sem os traços
    horizontais nas pontas). Visual mais limpo.
  \end{itemize}
\end{enumerate}

\begin{Shaded}
\begin{Highlighting}[]
\NormalTok{dados\_erro }\OtherTok{\textless{}{-}} \FunctionTok{data.frame}\NormalTok{(}
  \AttributeTok{grupo =} \FunctionTok{c}\NormalTok{(}\StringTok{"Controle"}\NormalTok{, }\StringTok{"Tratamento A"}\NormalTok{, }\StringTok{"Tratamento B"}\NormalTok{),}
  \AttributeTok{media =} \FunctionTok{c}\NormalTok{(}\DecValTok{20}\NormalTok{, }\DecValTok{25}\NormalTok{, }\DecValTok{18}\NormalTok{),}
  \AttributeTok{min =} \FunctionTok{c}\NormalTok{(}\DecValTok{18}\NormalTok{, }\DecValTok{23}\NormalTok{, }\DecValTok{15}\NormalTok{),}
  \AttributeTok{max =} \FunctionTok{c}\NormalTok{(}\DecValTok{22}\NormalTok{, }\DecValTok{27}\NormalTok{, }\DecValTok{21}\NormalTok{)}
\NormalTok{)}

\FunctionTok{ggplot}\NormalTok{(dados\_erro, }\FunctionTok{aes}\NormalTok{(}\AttributeTok{x =}\NormalTok{ grupo, }\AttributeTok{y =}\NormalTok{ media)) }\SpecialCharTok{+}
  \CommentTok{\# O linerange desenha apenas a linha vertical}
  \FunctionTok{geom\_linerange}\NormalTok{(}\FunctionTok{aes}\NormalTok{(}\AttributeTok{ymin =}\NormalTok{ min, }\AttributeTok{ymax =}\NormalTok{ max), }\AttributeTok{size =} \FloatTok{1.2}\NormalTok{, }\AttributeTok{color =} \StringTok{"gray50"}\NormalTok{) }\SpecialCharTok{+}
  \CommentTok{\# Adicionamos o ponto separadamente para marcar a média}
  \FunctionTok{geom\_point}\NormalTok{(}\AttributeTok{size =} \DecValTok{4}\NormalTok{, }\AttributeTok{color =} \StringTok{"blue"}\NormalTok{) }\SpecialCharTok{+}
  \FunctionTok{theme\_minimal}\NormalTok{() }\SpecialCharTok{+}
  \FunctionTok{labs}\NormalTok{(}\AttributeTok{title =} \StringTok{"Geom Linerange + Point"}\NormalTok{)}
\end{Highlighting}
\end{Shaded}

\begin{figure}[H]

\centering{

\pandocbounded{\includegraphics[keepaspectratio]{intro_files/figure-pdf/fig-linerange-1.pdf}}

}

\caption{\label{fig-linerange}Linerange: Intervalo limpo (Min-Max)}

\end{figure}%

\begin{enumerate}
\def\labelenumi{\arabic{enumi}.}
\setcounter{enumi}{2}
\tightlist
\item
  \texttt{geom\_pointrange()} Figura~\ref{fig-pointrange}

  \begin{itemize}
  \tightlist
  \item
    Combina um ponto (média) e uma linha vertical (intervalo) em um
    único geom geométrico. Muito usado em \emph{Forest Plots}.
  \end{itemize}
\end{enumerate}

\begin{Shaded}
\begin{Highlighting}[]
\FunctionTok{ggplot}\NormalTok{(dados\_erro, }\FunctionTok{aes}\NormalTok{(}\AttributeTok{x =}\NormalTok{ grupo, }\AttributeTok{y =}\NormalTok{ media)) }\SpecialCharTok{+}
  \CommentTok{\# ymin e ymax são obrigatórios aqui}
  \FunctionTok{geom\_pointrange}\NormalTok{(}\FunctionTok{aes}\NormalTok{(}\AttributeTok{ymin =}\NormalTok{ min, }\AttributeTok{ymax =}\NormalTok{ max), }\AttributeTok{color =} \StringTok{"darkred"}\NormalTok{) }\SpecialCharTok{+}
  \FunctionTok{coord\_flip}\NormalTok{() }\SpecialCharTok{+} \CommentTok{\# Forest plots geralmente são deitados}
  \FunctionTok{theme\_bw}\NormalTok{() }\SpecialCharTok{+}
  \FunctionTok{labs}\NormalTok{(}\AttributeTok{title =} \StringTok{"Geom Pointrange (Invertido)"}\NormalTok{)}
\end{Highlighting}
\end{Shaded}

\begin{figure}[H]

\centering{

\pandocbounded{\includegraphics[keepaspectratio]{intro_files/figure-pdf/fig-pointrange-1.pdf}}

}

\caption{\label{fig-pointrange}Pointrange: O padrão para Forest Plots}

\end{figure}%

\begin{enumerate}
\def\labelenumi{\arabic{enumi}.}
\setcounter{enumi}{3}
\tightlist
\item
  \texttt{geom\_crossbar()} Figura~\ref{fig-crossbar}

  \begin{itemize}
  \tightlist
  \item
    Uma caixa vazia ou preenchida representando o intervalo, com uma
    linha na média.
  \end{itemize}
\end{enumerate}

\begin{Shaded}
\begin{Highlighting}[]
\FunctionTok{ggplot}\NormalTok{(dados\_erro, }\FunctionTok{aes}\NormalTok{(}\AttributeTok{x =}\NormalTok{ grupo, }\AttributeTok{y =}\NormalTok{ media)) }\SpecialCharTok{+}
  \FunctionTok{geom\_crossbar}\NormalTok{(}
    \FunctionTok{aes}\NormalTok{(}\AttributeTok{ymin =}\NormalTok{ min, }\AttributeTok{ymax =}\NormalTok{ max), }
    \AttributeTok{width =} \FloatTok{0.5}\NormalTok{, }
    \AttributeTok{fill =} \StringTok{"lightblue"}\NormalTok{, }
    \AttributeTok{alpha =} \FloatTok{0.5}
\NormalTok{  ) }\SpecialCharTok{+}
  \FunctionTok{theme\_minimal}\NormalTok{() }\SpecialCharTok{+}
  \FunctionTok{labs}\NormalTok{(}\AttributeTok{title =} \StringTok{"Geom Crossbar"}\NormalTok{)}
\end{Highlighting}
\end{Shaded}

\begin{figure}[H]

\centering{

\pandocbounded{\includegraphics[keepaspectratio]{intro_files/figure-pdf/fig-crossbar-1.pdf}}

}

\caption{\label{fig-crossbar}Crossbar: Intervalo como caixa sólida}

\end{figure}%

\textbf{Três Variáveis (Mapas de Calor e Contornos)}

\emph{Ex: Z = Altitude, variando em X (Lat) e Y (Long).}

\begin{enumerate}
\def\labelenumi{\arabic{enumi}.}
\tightlist
\item
  \texttt{geom\_tile()} e \texttt{geom\_raster()} Figura~\ref{fig-tile}

  \begin{itemize}
  \tightlist
  \item
    Criam superfícies ou mapas de calor (\emph{heatmaps}).
    \texttt{geom\_raster} é uma versão otimizada de \texttt{geom\_tile}
    para quando todos os quadrados têm o mesmo tamanho (renderiza mais
    rápido).
  \end{itemize}
\end{enumerate}

\begin{Shaded}
\begin{Highlighting}[]
\CommentTok{\# Matriz de correlação transformada em formato longo}
\NormalTok{cormat }\OtherTok{\textless{}{-}} \FunctionTok{as.data.frame}\NormalTok{(}\FunctionTok{as.table}\NormalTok{(}\FunctionTok{cor}\NormalTok{(mtcars)))}

\FunctionTok{ggplot}\NormalTok{(cormat, }\FunctionTok{aes}\NormalTok{(Var1, Var2, }\AttributeTok{fill =}\NormalTok{ Freq)) }\SpecialCharTok{+}
  \FunctionTok{geom\_tile}\NormalTok{() }\SpecialCharTok{+}
  \FunctionTok{scale\_fill\_gradient2}\NormalTok{(}\AttributeTok{low =} \StringTok{"blue"}\NormalTok{, }\AttributeTok{high =} \StringTok{"red"}\NormalTok{, }\AttributeTok{mid =} \StringTok{"white"}\NormalTok{) }\SpecialCharTok{+}
  \FunctionTok{theme\_minimal}\NormalTok{()}
\end{Highlighting}
\end{Shaded}

\begin{figure}[H]

\centering{

\pandocbounded{\includegraphics[keepaspectratio]{intro_files/figure-pdf/fig-tile-1.pdf}}

}

\caption{\label{fig-tile}Heatmap (Mapa de Calor)}

\end{figure}%

\begin{enumerate}
\def\labelenumi{\arabic{enumi}.}
\setcounter{enumi}{1}
\tightlist
\item
  \texttt{geom\_contour()} e \texttt{geom\_contour\_filled()}
  Figura~\ref{fig-contour}

  \begin{itemize}
  \tightlist
  \item
    Desenha linhas de contorno (como em mapas topográficos) baseadas na
    variável Z.
  \end{itemize}
\end{enumerate}

\begin{Shaded}
\begin{Highlighting}[]
\NormalTok{pacman}\SpecialCharTok{::}\FunctionTok{p\_load}\NormalTok{(reshape2)}
\NormalTok{volcano\_df }\OtherTok{\textless{}{-}} \FunctionTok{melt}\NormalTok{(volcano)}
\FunctionTok{names}\NormalTok{(volcano\_df) }\OtherTok{\textless{}{-}} \FunctionTok{c}\NormalTok{(}\StringTok{"x"}\NormalTok{, }\StringTok{"y"}\NormalTok{, }\StringTok{"z"}\NormalTok{)}

\FunctionTok{ggplot}\NormalTok{(volcano\_df, }\FunctionTok{aes}\NormalTok{(x, y, }\AttributeTok{z =}\NormalTok{ z)) }\SpecialCharTok{+}
  \CommentTok{\# Preenchimento das regiões (o visual colorido)}
  \FunctionTok{geom\_contour\_filled}\NormalTok{() }\SpecialCharTok{+}
  \CommentTok{\# Linhas pretas finas para demarcar }
  \FunctionTok{geom\_contour}\NormalTok{(}\AttributeTok{color =} \StringTok{"black"}\NormalTok{, }\AttributeTok{size =} \FloatTok{0.1}\NormalTok{) }\SpecialCharTok{+}
  \FunctionTok{theme\_void}\NormalTok{() }\SpecialCharTok{+} \CommentTok{\# Remove eixos para parecer um mapa}
  \FunctionTok{coord\_fixed}\NormalTok{()}
\end{Highlighting}
\end{Shaded}

\begin{figure}[H]

\centering{

\pandocbounded{\includegraphics[keepaspectratio]{intro_files/figure-pdf/fig-contour-1.pdf}}

}

\caption{\label{fig-contour}Topografia do Vulcão Maunga Whau (Contour
Filled)}

\end{figure}%

\textbf{Mapas e Espacial}\{\#sec-espacial\_visualizacao\}

\begin{enumerate}
\def\labelenumi{\arabic{enumi}.}
\tightlist
\item
  \texttt{geom\_sf()} Figura~\ref{fig-sf} e Figura~\ref{fig-mapa-moz}

  \begin{itemize}
  \tightlist
  \item
    Plota objetos \emph{Simple Features} (do pacote \texttt{sf}). Lida
    automaticamente com projeções geográficas, fronteiras de países e
    coordenadas.
  \end{itemize}
\end{enumerate}

\begin{Shaded}
\begin{Highlighting}[]
\NormalTok{pacman}\SpecialCharTok{::}\FunctionTok{p\_load}\NormalTok{(sf, ggplot2)}

\CommentTok{\# Lendo um shapefile de exemplo que vem dentro do pacote sf}
\NormalTok{nc }\OtherTok{\textless{}{-}} \FunctionTok{st\_read}\NormalTok{(}\FunctionTok{system.file}\NormalTok{(}\StringTok{"shape/nc.shp"}\NormalTok{, }\AttributeTok{package=}\StringTok{"sf"}\NormalTok{), }\AttributeTok{quiet =} \ConstantTok{TRUE}\NormalTok{)}

\FunctionTok{ggplot}\NormalTok{(nc) }\SpecialCharTok{+}
  \FunctionTok{geom\_sf}\NormalTok{(}\FunctionTok{aes}\NormalTok{(}\AttributeTok{fill =}\NormalTok{ AREA), }\AttributeTok{color =} \StringTok{"white"}\NormalTok{) }\SpecialCharTok{+}
  \FunctionTok{scale\_fill\_viridis\_c}\NormalTok{(}\AttributeTok{name =} \StringTok{"Área"}\NormalTok{) }\SpecialCharTok{+}
  \FunctionTok{theme\_minimal}\NormalTok{() }\SpecialCharTok{+}
  \FunctionTok{labs}\NormalTok{(}\AttributeTok{title =} \StringTok{""}\NormalTok{)}
\end{Highlighting}
\end{Shaded}

\begin{figure}[H]

\centering{

\pandocbounded{\includegraphics[keepaspectratio]{intro_files/figure-pdf/fig-sf-1.pdf}}

}

\caption{\label{fig-sf}Mapa da Carolina do Norte (Shapefile nativo)}

\end{figure}%

\begin{Shaded}
\begin{Highlighting}[]
\NormalTok{pacman}\SpecialCharTok{::}\FunctionTok{p\_load}\NormalTok{(sf, ggplot2, geodata, terra, viridis)}

\CommentTok{\#Baixar dados do GADM}
\CommentTok{\# country = "MOZ" (Código ISO de Moçambique)}
\CommentTok{\# level = 2 (Distritos)}
\CommentTok{\# path = tempdir() salva numa pasta temporária. Para salvar no projeto, use path = "."}

\FunctionTok{tryCatch}\NormalTok{(\{}
\NormalTok{    moz\_data\_spat }\OtherTok{\textless{}{-}}\NormalTok{ geodata}\SpecialCharTok{::}\FunctionTok{gadm}\NormalTok{(}\AttributeTok{country =} \StringTok{"MOZ"}\NormalTok{, }\AttributeTok{level =} \DecValTok{2}\NormalTok{, }\AttributeTok{path =} \FunctionTok{tempdir}\NormalTok{(), }\AttributeTok{version=}\StringTok{"latest"}\NormalTok{)}
\NormalTok{  \}, }\AttributeTok{error =} \ControlFlowTok{function}\NormalTok{(e) \{}
    \FunctionTok{message}\NormalTok{(}\StringTok{"Erro ao baixar do GADM. Verifique a conexão."}\NormalTok{)}
    \CommentTok{\# Cria um objeto sf vazio em caso de erro para não quebrar o render}
\NormalTok{    moz\_data\_spat }\OtherTok{\textless{}{-}} \ConstantTok{NULL}
\NormalTok{\})}


\ControlFlowTok{if}\NormalTok{ (}\SpecialCharTok{!}\FunctionTok{is.null}\NormalTok{(moz\_data\_spat)) \{}
  \CommentTok{\#Converter de SpatVector (terra) para objeto sf}
\NormalTok{  moz\_sf }\OtherTok{\textless{}{-}}\NormalTok{ sf}\SpecialCharTok{::}\FunctionTok{st\_as\_sf}\NormalTok{(moz\_data\_spat)}

  \CommentTok{\#Plotar}
\NormalTok{  ggplot2}\SpecialCharTok{::}\FunctionTok{ggplot}\NormalTok{(moz\_sf) }\SpecialCharTok{+}
    \FunctionTok{geom\_sf}\NormalTok{(}\FunctionTok{aes}\NormalTok{(}\AttributeTok{fill =}\NormalTok{ NAME\_1), }\AttributeTok{color =} \StringTok{"white"}\NormalTok{, }\AttributeTok{size =} \FloatTok{0.1}\NormalTok{) }\SpecialCharTok{+}
    \FunctionTok{scale\_fill\_viridis\_d}\NormalTok{(}\AttributeTok{name =} \StringTok{"Província"}\NormalTok{, }\AttributeTok{option =} \StringTok{"D"}\NormalTok{) }\SpecialCharTok{+} \CommentTok{\# Paleta discreta bonita}
    \FunctionTok{theme\_minimal}\NormalTok{() }\SpecialCharTok{+}
    \FunctionTok{theme}\NormalTok{(}\AttributeTok{legend.position =} \StringTok{"right"}\NormalTok{) }\SpecialCharTok{+} \CommentTok{\# Legenda à direita}
    \FunctionTok{labs}\NormalTok{(}\AttributeTok{title =} \StringTok{""}\NormalTok{,}
         \AttributeTok{caption =} \StringTok{"Fonte: Dados GADM via pacote \{geodata\}"}\NormalTok{)}

\NormalTok{\} }\ControlFlowTok{else}\NormalTok{ \{}
  \CommentTok{\# Mensagem alternativa no plot caso o download falhe}
\NormalTok{  ggplot2}\SpecialCharTok{::}\FunctionTok{ggplot}\NormalTok{() }\SpecialCharTok{+}
    \FunctionTok{annotate}\NormalTok{(}\StringTok{"text"}\NormalTok{, }\AttributeTok{x=}\DecValTok{0}\NormalTok{, }\AttributeTok{y=}\DecValTok{0}\NormalTok{, }\AttributeTok{label=}\StringTok{"Não foi possível baixar os dados do mapa."}\NormalTok{) }\SpecialCharTok{+}
    \FunctionTok{theme\_void}\NormalTok{()}
\NormalTok{\}}
\end{Highlighting}
\end{Shaded}

\begin{itemize}
\tightlist
\item
  O GADM as vezes tem problemas, se não funcionou o codigo acima baixa
  pelo link abaixo ou vá ao site do GADM
\end{itemize}

\begin{Shaded}
\begin{Highlighting}[]
\NormalTok{pacman}\SpecialCharTok{::}\FunctionTok{p\_load}\NormalTok{(sf, ggplot2, ggrepel) }

\CommentTok{\#Se geodata não funcionar baixe os dados no link: https://drive.google.com/drive/folders/1pL\_MLujCv\_6h5VJqK20B8pUHSNlZ5Olt?usp=sharing}

\CommentTok{\# Abra o arquivo que terá baixado no link acima caso gadm não tenha funcionado}
\CommentTok{\#troque meu "/home/almonha/Downloads/Curso de Verão/moz\_adm/moz\_admbnda\_adm1\_ine\_20190607.shp" pelo seu.}

\FunctionTok{tryCatch}\NormalTok{(\{  }
\NormalTok{  moz\_data\_spat }\OtherTok{\textless{}{-}} \FunctionTok{read\_sf}\NormalTok{(}\StringTok{"/home/almonha/Downloads/Curso de Verão/moz\_adm/moz\_admbnda\_adm1\_ine\_20190607.shp"}\NormalTok{) }
\NormalTok{  \}, }\AttributeTok{error =} \ControlFlowTok{function}\NormalTok{(e) \{}
    \FunctionTok{message}\NormalTok{(}\StringTok{"O link não funcionou vou baixar pelo GADM."}\NormalTok{)}
\NormalTok{moz\_data\_spat }\OtherTok{\textless{}{-}}\NormalTok{ geodata}\SpecialCharTok{::}\FunctionTok{gadm}\NormalTok{(}\AttributeTok{country =} \StringTok{"MOZ"}\NormalTok{, }\AttributeTok{level =} \DecValTok{2}\NormalTok{, }\AttributeTok{path =} \FunctionTok{tempdir}\NormalTok{(), }\AttributeTok{version=}\StringTok{"latest"}\NormalTok{)}
\NormalTok{\}) }



\ControlFlowTok{if}\NormalTok{ (}\SpecialCharTok{!}\FunctionTok{is.null}\NormalTok{(moz\_data\_spat)) \{}
  
\NormalTok{  moz\_sf }\OtherTok{\textless{}{-}}\NormalTok{ sf}\SpecialCharTok{::}\FunctionTok{st\_as\_sf}\NormalTok{(moz\_data\_spat)}
  
\NormalTok{  ggplot2}\SpecialCharTok{::}\FunctionTok{ggplot}\NormalTok{(}\AttributeTok{data =}\NormalTok{ moz\_sf) }\SpecialCharTok{+}
    \FunctionTok{geom\_sf}\NormalTok{(}\FunctionTok{aes}\NormalTok{(}\AttributeTok{fill =}\NormalTok{ ADM1\_PT), }\AttributeTok{color =} \StringTok{"white"}\NormalTok{, }\AttributeTok{size =} \FloatTok{0.1}\NormalTok{) }\SpecialCharTok{+}
  \FunctionTok{scale\_fill\_viridis\_d}\NormalTok{(}\AttributeTok{name =} \StringTok{"Província"}\NormalTok{, }\AttributeTok{option =} \StringTok{"D"}\NormalTok{) }\SpecialCharTok{+} 
  \FunctionTok{theme\_minimal}\NormalTok{()}\SpecialCharTok{+}
    \FunctionTok{theme}\NormalTok{(}\AttributeTok{legend.position =} \StringTok{"right"}\NormalTok{,}
    \AttributeTok{axis.title =} \FunctionTok{element\_blank}\NormalTok{()}
\NormalTok{    )}\SpecialCharTok{+} \CommentTok{\# Remove X e Y de forma mais limpa) + }
    \FunctionTok{labs}\NormalTok{(}
      \AttributeTok{title =} \StringTok{"Mapa de Moçambique"}\NormalTok{,}
      \AttributeTok{caption =} \StringTok{"Fonte: Dados GADM via pacote \{geodata\}"}\NormalTok{,}
\NormalTok{    )}

\NormalTok{\} }\ControlFlowTok{else}\NormalTok{ \{}
\NormalTok{  ggplot2}\SpecialCharTok{::}\FunctionTok{ggplot}\NormalTok{() }\SpecialCharTok{+}
    \FunctionTok{annotate}\NormalTok{(}\StringTok{"text"}\NormalTok{, }\AttributeTok{x=}\DecValTok{0}\NormalTok{, }\AttributeTok{y=}\DecValTok{0}\NormalTok{, }\AttributeTok{label=}\StringTok{"Não foi possível baixar os dados do mapa."}\NormalTok{) }\SpecialCharTok{+}
    \FunctionTok{theme\_void}\NormalTok{()}
\NormalTok{\}}
\end{Highlighting}
\end{Shaded}

\begin{figure}[H]

\centering{

\pandocbounded{\includegraphics[keepaspectratio]{intro_files/figure-pdf/fig-mapa-moz-1.pdf}}

}

\caption{\label{fig-mapa-moz}Mapa de Moçambique - Nível Administrativo 2
(Distritos)}

\end{figure}%

\begin{enumerate}
\def\labelenumi{\arabic{enumi}.}
\setcounter{enumi}{1}
\tightlist
\item
  \texttt{geom\_map()}

  \begin{itemize}
  \tightlist
  \item
    Versão antiga para desenhar polígonos de mapas. Prefira
    \texttt{geom\_sf}.
  \end{itemize}
\end{enumerate}

\textbf{Primitivas e Referências}: Usados para anotações manuais.

\begin{enumerate}
\def\labelenumi{\arabic{enumi}.}
\item
  \texttt{geom\_abline()} , \texttt{geom\_hline()},
  \texttt{geom\_vline()}, segue o descrito na seção
  Seção~\ref{sec-Visualização} .
\item
  \texttt{geom\_segment()} e \texttt{geom\_curve()}
  Figura~\ref{fig-segment}

  \begin{itemize}
  \tightlist
  \item
    Desenha linhas ou curvas de um ponto A (x, y) até um ponto B (xend,
    yend). Útil para setas e anotações.
  \end{itemize}
\end{enumerate}

\begin{Shaded}
\begin{Highlighting}[]
\NormalTok{df\_pontos }\OtherTok{\textless{}{-}} \FunctionTok{data.frame}\NormalTok{(}\AttributeTok{x =} \FunctionTok{c}\NormalTok{(}\DecValTok{2}\NormalTok{, }\DecValTok{8}\NormalTok{), }\AttributeTok{y =} \FunctionTok{c}\NormalTok{(}\DecValTok{2}\NormalTok{, }\DecValTok{8}\NormalTok{), }\AttributeTok{label =} \FunctionTok{c}\NormalTok{(}\StringTok{"Início"}\NormalTok{, }\StringTok{"Fim"}\NormalTok{))}

\FunctionTok{ggplot}\NormalTok{(df\_pontos, }\FunctionTok{aes}\NormalTok{(x, y)) }\SpecialCharTok{+}
  \FunctionTok{geom\_point}\NormalTok{(}\AttributeTok{size =} \DecValTok{3}\NormalTok{) }\SpecialCharTok{+}
  \FunctionTok{xlim}\NormalTok{(}\DecValTok{0}\NormalTok{, }\DecValTok{10}\NormalTok{) }\SpecialCharTok{+} \FunctionTok{ylim}\NormalTok{(}\DecValTok{0}\NormalTok{, }\DecValTok{10}\NormalTok{) }\SpecialCharTok{+}
  \CommentTok{\# Segmento Reto com Seta}
  \FunctionTok{geom\_segment}\NormalTok{(}\FunctionTok{aes}\NormalTok{(}\AttributeTok{x =} \DecValTok{2}\NormalTok{, }\AttributeTok{y =} \DecValTok{2}\NormalTok{, }\AttributeTok{xend =} \DecValTok{5}\NormalTok{, }\AttributeTok{yend =} \DecValTok{5}\NormalTok{), }
               \AttributeTok{arrow =} \FunctionTok{arrow}\NormalTok{(}\AttributeTok{length =} \FunctionTok{unit}\NormalTok{(}\FloatTok{0.3}\NormalTok{, }\StringTok{"cm"}\NormalTok{)), }\AttributeTok{color =} \StringTok{"blue"}\NormalTok{) }\SpecialCharTok{+}
  \CommentTok{\# Curva com Seta}
  \FunctionTok{geom\_curve}\NormalTok{(}\FunctionTok{aes}\NormalTok{(}\AttributeTok{x =} \DecValTok{5}\NormalTok{, }\AttributeTok{y =} \DecValTok{5}\NormalTok{, }\AttributeTok{xend =} \DecValTok{8}\NormalTok{, }\AttributeTok{yend =} \DecValTok{8}\NormalTok{), }
             \AttributeTok{curvature =} \SpecialCharTok{{-}}\FloatTok{0.3}\NormalTok{, }\AttributeTok{arrow =} \FunctionTok{arrow}\NormalTok{(), }\AttributeTok{color =} \StringTok{"red"}\NormalTok{) }\SpecialCharTok{+}
  \FunctionTok{theme\_bw}\NormalTok{()}
\end{Highlighting}
\end{Shaded}

\begin{figure}[H]

\centering{

\pandocbounded{\includegraphics[keepaspectratio]{intro_files/figure-pdf/fig-segment-1.pdf}}

}

\caption{\label{fig-segment}Anotações com Segmentos e Curvas}

\end{figure}%

\begin{enumerate}
\def\labelenumi{\arabic{enumi}.}
\setcounter{enumi}{2}
\tightlist
\item
  \texttt{geom\_rect()} e \texttt{geom\_polygon()} Figura~\ref{fig-rect}

  \begin{itemize}
  \tightlist
  \item
    Desenha retângulos ou formas arbitrárias baseadas em coordenadas.
  \end{itemize}
\end{enumerate}

\begin{Shaded}
\begin{Highlighting}[]
\FunctionTok{ggplot}\NormalTok{(mtcars, }\FunctionTok{aes}\NormalTok{(wt, mpg)) }\SpecialCharTok{+}
  \CommentTok{\# O retângulo deve vir ANTES dos pontos para ficar no fundo}
  \FunctionTok{geom\_rect}\NormalTok{(}\FunctionTok{aes}\NormalTok{(}\AttributeTok{xmin =} \DecValTok{3}\NormalTok{, }\AttributeTok{xmax =} \DecValTok{4}\NormalTok{, }\AttributeTok{ymin =} \DecValTok{10}\NormalTok{, }\AttributeTok{ymax =} \DecValTok{25}\NormalTok{), }
            \AttributeTok{fill =} \StringTok{"yellow"}\NormalTok{, }\AttributeTok{alpha =} \FloatTok{0.05}\NormalTok{, }\AttributeTok{color =} \ConstantTok{NA}\NormalTok{) }\SpecialCharTok{+}
  \FunctionTok{geom\_point}\NormalTok{() }\SpecialCharTok{+}
  \FunctionTok{annotate}\NormalTok{(}\StringTok{"text"}\NormalTok{, }\AttributeTok{x =} \FloatTok{3.5}\NormalTok{, }\AttributeTok{y =} \DecValTok{26}\NormalTok{, }\AttributeTok{label =} \StringTok{"Zona de Atenção"}\NormalTok{) }\SpecialCharTok{+}
  \FunctionTok{theme\_minimal}\NormalTok{()}
\end{Highlighting}
\end{Shaded}

\begin{figure}[H]

\centering{

\pandocbounded{\includegraphics[keepaspectratio]{intro_files/figure-pdf/fig-rect-1.pdf}}

}

\caption{\label{fig-rect}Destacando uma região de interesse com
geom\_rect}

\end{figure}%

\begin{enumerate}
\def\labelenumi{\arabic{enumi}.}
\setcounter{enumi}{3}
\tightlist
\item
  \texttt{geom\_spoke()} Figura~\ref{fig-spoke}

  \begin{itemize}
  \tightlist
  \item
    Desenha segmentos definidos por ângulo e raio (útil para campos
    vetoriais, como direção do vento).
  \end{itemize}
\end{enumerate}

\begin{Shaded}
\begin{Highlighting}[]
\NormalTok{df\_vento }\OtherTok{\textless{}{-}} \FunctionTok{expand.grid}\NormalTok{(}\AttributeTok{x =} \DecValTok{1}\SpecialCharTok{:}\DecValTok{10}\NormalTok{, }\AttributeTok{y =} \DecValTok{1}\SpecialCharTok{:}\DecValTok{10}\NormalTok{)}
\NormalTok{df\_vento}\SpecialCharTok{$}\NormalTok{angle }\OtherTok{\textless{}{-}} \FunctionTok{runif}\NormalTok{(}\DecValTok{100}\NormalTok{, }\DecValTok{0}\NormalTok{, }\DecValTok{2}\SpecialCharTok{*}\NormalTok{pi) }\CommentTok{\# Ângulo em radianos}
\NormalTok{df\_vento}\SpecialCharTok{$}\NormalTok{speed }\OtherTok{\textless{}{-}} \FunctionTok{runif}\NormalTok{(}\DecValTok{100}\NormalTok{, }\FloatTok{0.4}\NormalTok{, }\FloatTok{0.8}\NormalTok{) }\CommentTok{\# Comprimento do vetor}

\FunctionTok{ggplot}\NormalTok{(df\_vento, }\FunctionTok{aes}\NormalTok{(x, y)) }\SpecialCharTok{+}
  \FunctionTok{geom\_spoke}\NormalTok{(}\FunctionTok{aes}\NormalTok{(}\AttributeTok{angle =}\NormalTok{ angle, }\AttributeTok{radius =}\NormalTok{ speed), }\AttributeTok{arrow =} \FunctionTok{arrow}\NormalTok{(}\AttributeTok{length =} \FunctionTok{unit}\NormalTok{(}\FloatTok{0.1}\NormalTok{, }\StringTok{"cm"}\NormalTok{))) }\SpecialCharTok{+}
  \FunctionTok{theme\_void}\NormalTok{() }\SpecialCharTok{+}
  \FunctionTok{labs}\NormalTok{(}\AttributeTok{title =} \StringTok{""}\NormalTok{)}
\end{Highlighting}
\end{Shaded}

\begin{figure}[H]

\centering{

\pandocbounded{\includegraphics[keepaspectratio]{intro_files/figure-pdf/fig-spoke-1.pdf}}

}

\caption{\label{fig-spoke}Campo Vetorial (Direção e Intensidade
Aleatórias)}

\end{figure}%

\begin{enumerate}
\def\labelenumi{\arabic{enumi}.}
\setcounter{enumi}{4}
\tightlist
\item
  \texttt{geom\_qq()} e \texttt{geom\_qq\_line()} Figura~\ref{fig-qq}

  \begin{itemize}
  \tightlist
  \item
    Gráfico Quantil-Quantil (QQ Plot) para checar normalidade dos
    resíduos.
  \end{itemize}
\end{enumerate}

\begin{Shaded}
\begin{Highlighting}[]
\FunctionTok{ggplot}\NormalTok{(mtcars, }\FunctionTok{aes}\NormalTok{(}\AttributeTok{sample =}\NormalTok{ mpg)) }\SpecialCharTok{+}
  \FunctionTok{geom\_qq}\NormalTok{() }\SpecialCharTok{+}        \CommentTok{\# Os pontos}
  \FunctionTok{geom\_qq\_line}\NormalTok{(}\AttributeTok{color =} \StringTok{"red"}\NormalTok{) }\SpecialCharTok{+} \CommentTok{\# A linha de referência normal}
  \FunctionTok{labs}\NormalTok{(}\AttributeTok{title =} \StringTok{"Os dados de MPG são normais?"}\NormalTok{, }
       \AttributeTok{subtitle =} \StringTok{"Pontos fora da linha indicam desvio da normalidade"}\NormalTok{) }\SpecialCharTok{+}
  \FunctionTok{theme\_minimal}\NormalTok{()}
\end{Highlighting}
\end{Shaded}

\begin{figure}[H]

\centering{

\pandocbounded{\includegraphics[keepaspectratio]{intro_files/figure-pdf/fig-qq-1.pdf}}

}

\caption{\label{fig-qq}QQ Plot: Verificação de Normalidade}

\end{figure}%

\textbf{Faceting (Small Multiples)}

A melhor técnica para evitar gráficos ``espaguete'' (muitas linhas
misturadas).

\begin{itemize}
\tightlist
\item
  \texttt{facet\_wrap(\textasciitilde{}variavel)}: Fluxo contínuo
  (quebra linha quando acaba espaço), Figura~\ref{fig-facet_wrap} .
\end{itemize}

\begin{Shaded}
\begin{Highlighting}[]
\FunctionTok{ggplot}\NormalTok{(mpg, }\FunctionTok{aes}\NormalTok{(}\AttributeTok{x =}\NormalTok{ displ, }\AttributeTok{y =}\NormalTok{ hwy)) }\SpecialCharTok{+}
  \FunctionTok{geom\_point}\NormalTok{(}\AttributeTok{alpha =} \FloatTok{0.5}\NormalTok{) }\SpecialCharTok{+}
  \CommentTok{\# Divide o gráfico pelo tipo de carro (\textquotesingle{}class\textquotesingle{})}
  \CommentTok{\# ncol = 3 força 3 colunas}
  \FunctionTok{facet\_wrap}\NormalTok{(}\SpecialCharTok{\textasciitilde{}}\NormalTok{class, }\AttributeTok{ncol =} \DecValTok{3}\NormalTok{) }\SpecialCharTok{+} 
  \FunctionTok{theme\_bw}\NormalTok{() }\SpecialCharTok{+}
  \FunctionTok{labs}\NormalTok{(}\AttributeTok{title =} \StringTok{"Consumo vs Motor"}\NormalTok{, }\AttributeTok{subtitle =} \StringTok{"Separado por categoria (class)"}\NormalTok{)}
\end{Highlighting}
\end{Shaded}

\begin{figure}[H]

\centering{

\pandocbounded{\includegraphics[keepaspectratio]{intro_files/figure-pdf/fig-facet_wrap-1.pdf}}

}

\caption{\label{fig-facet_wrap}Facet Wrap: Consumo por Tipo de Carro}

\end{figure}%

\begin{itemize}
\tightlist
\item
  \texttt{facet\_grid(lin\ \textasciitilde{}\ col)}: Matriz estrita. Use
  \texttt{.} se quiser apenas uma dimensão (ex:
  \texttt{.\ \textasciitilde{}\ col}), Figura~\ref{fig-facet_grid} .
\end{itemize}

\begin{Shaded}
\begin{Highlighting}[]
\NormalTok{mtcars\_facets }\OtherTok{\textless{}{-}}\NormalTok{ mtcars}
\NormalTok{mtcars\_facets}\SpecialCharTok{$}\NormalTok{am }\OtherTok{\textless{}{-}} \FunctionTok{factor}\NormalTok{(mtcars}\SpecialCharTok{$}\NormalTok{am, }\AttributeTok{labels =} \FunctionTok{c}\NormalTok{(}\StringTok{"Automático"}\NormalTok{, }\StringTok{"Manual"}\NormalTok{))}
\NormalTok{mtcars\_facets}\SpecialCharTok{$}\NormalTok{cyl }\OtherTok{\textless{}{-}} \FunctionTok{factor}\NormalTok{(mtcars}\SpecialCharTok{$}\NormalTok{cyl, }\AttributeTok{labels =} \FunctionTok{c}\NormalTok{(}\StringTok{"4 Cilindros"}\NormalTok{, }\StringTok{"6 Cilindros"}\NormalTok{, }\StringTok{"8 Cilindros"}\NormalTok{))}

\FunctionTok{ggplot}\NormalTok{(mtcars\_facets, }\FunctionTok{aes}\NormalTok{(}\AttributeTok{x =}\NormalTok{ wt, }\AttributeTok{y =}\NormalTok{ mpg)) }\SpecialCharTok{+}
  \FunctionTok{geom\_point}\NormalTok{(}\AttributeTok{color =} \StringTok{"darkred"}\NormalTok{) }\SpecialCharTok{+}
  \CommentTok{\# Linhas definidas pelo Câmbio (am) \textasciitilde{} Colunas definidas pelos Cilindros (cyl)}
  \FunctionTok{facet\_grid}\NormalTok{(am }\SpecialCharTok{\textasciitilde{}}\NormalTok{ cyl) }\SpecialCharTok{+} 
  \FunctionTok{theme\_bw}\NormalTok{() }\SpecialCharTok{+}
  \FunctionTok{labs}\NormalTok{(}\AttributeTok{title =} \StringTok{"Comparação Matricial"}\NormalTok{)}
\end{Highlighting}
\end{Shaded}

\begin{figure}[H]

\centering{

\pandocbounded{\includegraphics[keepaspectratio]{intro_files/figure-pdf/fig-facet_grid-1.pdf}}

}

\caption{\label{fig-facet_grid}Facet Grid: Cruzamento de Câmbio (Linhas)
vs Cilindros (Colunas)}

\end{figure}%

\begin{tcolorbox}[enhanced jigsaw, left=2mm, toptitle=1mm, colback=white, colframe=quarto-callout-tip-color-frame, colbacktitle=quarto-callout-tip-color!10!white, opacityback=0, rightrule=.15mm, bottomtitle=1mm, arc=.35mm, title=\textcolor{quarto-callout-tip-color}{\faLightbulb}\hspace{0.5em}{Escalas Livres}, titlerule=0mm, bottomrule=.15mm, leftrule=.75mm, coltitle=black, toprule=.15mm, breakable, opacitybacktitle=0.6]

Por padrão, o \texttt{ggplot2} força todos os painéis a terem a mesma
escala de eixo X e Y para facilitar a comparação. Se os seus grupos têm
magnitudes muito diferentes (ex: um vai até 10, outro até 1000), use o
argumento scales:

\begin{itemize}
\item
  scales = ``free'': Eixos X e Y independentes.
\item
  scales = ``free\_y'': Só o Y varia.
\item
  scales = ``free\_x'': Só o X varia.
\end{itemize}

\end{tcolorbox}

\textbf{Escalas e Temas}

\begin{enumerate}
\def\labelenumi{\arabic{enumi}.}
\tightlist
\item
  Scales (\texttt{scale\_*})
\end{enumerate}

Controlam como os dados são mapeados visualmente.

\begin{itemize}
\tightlist
\item
  \textbf{Eixos:} \texttt{scale\_x\_log10()},
  \texttt{scale\_x\_continuous(limits\ =\ c(0,\ 100),\ breaks\ =\ seq(0,100,10))}.
\item
  \textbf{Cores:} \texttt{scale\_color\_manual()} (controle total),
  \texttt{scale\_color\_viridis\_d()} (acessibilidade).
\end{itemize}

\begin{enumerate}
\def\labelenumi{\arabic{enumi}.}
\setcounter{enumi}{1}
\tightlist
\item
  Themes (\texttt{theme})
\end{enumerate}

Controlam elementos que não são dados.

\begin{itemize}
\tightlist
\item
  Temas prontos: \texttt{theme\_bw()}, \texttt{theme\_minimal()},
  \texttt{theme\_classic()}, \texttt{theme\_avoid()}, etc.
\item
  Ajuste fino:
  \texttt{theme(legend.position\ =\ "bottom",\ axis.text\ =\ element\_text(...))}.
\end{itemize}

\begin{tcolorbox}[enhanced jigsaw, left=2mm, toptitle=1mm, colback=white, colframe=quarto-callout-note-color-frame, colbacktitle=quarto-callout-note-color!10!white, opacityback=0, rightrule=.15mm, bottomtitle=1mm, arc=.35mm, title=\textcolor{quarto-callout-note-color}{\faInfo}\hspace{0.5em}{Nota sobre Extensões}, titlerule=0mm, bottomrule=.15mm, leftrule=.75mm, coltitle=black, toprule=.15mm, breakable, opacitybacktitle=0.6]

O poder do \texttt{ggplot2} é infinito graças aos pacotes da comunidade.
Se não achou o \texttt{geom\ aqui}, provavelmente ele existe em um
pacote extra:

\begin{itemize}
\item
  \texttt{ggforce}: Geoms avançados (círculos, elipses, zoom).
\item
  \texttt{ggridges}: Gráficos de ``Joyplot'' (distribuições empilhadas).
\item
  \texttt{ggraph}: Para redes e grafos.
\end{itemize}

\end{tcolorbox}

\section{\texorpdfstring{Extensões Essenciais do
\texttt{ggplot2}}{Extensões Essenciais do ggplot2}}\label{extensuxf5es-essenciais-do-ggplot2}

O poder do \texttt{ggplot2} reside em suas extensões.

\textbf{Rótulos Inteligentes (\texttt{ggrepel})}

Evita que rótulos de texto se sobreponham uns aos outros ou aos pontos.
Essencial para scatterplots com nomes.

\begin{Shaded}
\begin{Highlighting}[]
\NormalTok{pacman}\SpecialCharTok{::}\FunctionTok{p\_load}\NormalTok{(ggrepel)}

\CommentTok{\# Seleciona alguns carros para rotular}
\NormalTok{subset\_cars }\OtherTok{\textless{}{-}}\NormalTok{ mtcars[}\DecValTok{1}\SpecialCharTok{:}\DecValTok{10}\NormalTok{, ]}
\NormalTok{subset\_cars}\SpecialCharTok{$}\NormalTok{car\_name }\OtherTok{\textless{}{-}} \FunctionTok{rownames}\NormalTok{(subset\_cars)}

\FunctionTok{ggplot}\NormalTok{(subset\_cars, }\FunctionTok{aes}\NormalTok{(wt, mpg, }\AttributeTok{label =}\NormalTok{ car\_name)) }\SpecialCharTok{+}
  \FunctionTok{geom\_point}\NormalTok{(}\AttributeTok{color =} \StringTok{"red"}\NormalTok{) }\SpecialCharTok{+} \CommentTok{\#adiciona os pontos}
  \FunctionTok{geom\_text\_repel}\NormalTok{() }\SpecialCharTok{+} \CommentTok{\# Afasta os textos automaticamente}
  \FunctionTok{theme\_classic}\NormalTok{()}
\end{Highlighting}
\end{Shaded}

\pandocbounded{\includegraphics[keepaspectratio]{intro_files/figure-pdf/unnamed-chunk-166-1.pdf}}

\textbf{Composição (\texttt{patchwork})}

Substitui \texttt{par(mfrow)} e \texttt{gridExtra}. Permite somar
gráficos matematicamente.

\begin{Shaded}
\begin{Highlighting}[]
\NormalTok{pacman}\SpecialCharTok{::}\FunctionTok{p\_load}\NormalTok{(patchwork)}
\NormalTok{g1 }\OtherTok{\textless{}{-}} \FunctionTok{ggplot}\NormalTok{(mtcars, }\FunctionTok{aes}\NormalTok{(mpg)) }\SpecialCharTok{+} \FunctionTok{geom\_histogram}\NormalTok{()}\SpecialCharTok{+}\FunctionTok{theme\_classic}\NormalTok{()}
\NormalTok{g2 }\OtherTok{\textless{}{-}} \FunctionTok{ggplot}\NormalTok{(mtcars, }\FunctionTok{aes}\NormalTok{(wt, mpg)) }\SpecialCharTok{+} \FunctionTok{geom\_point}\NormalTok{()}\SpecialCharTok{+}\FunctionTok{theme\_classic}\NormalTok{()}


\NormalTok{(g1 }\SpecialCharTok{|}\NormalTok{ g2) }\SpecialCharTok{/}\NormalTok{ g2 }\SpecialCharTok{+}\FunctionTok{plot\_annotation}\NormalTok{(}\AttributeTok{title =} \StringTok{""}\NormalTok{,}\AttributeTok{tag\_levels =} \StringTok{"A"}\NormalTok{) }\CommentTok{\#tag\_levels adiciona letras em cada grafico}
\end{Highlighting}
\end{Shaded}

\pandocbounded{\includegraphics[keepaspectratio]{intro_files/figure-pdf/unnamed-chunk-167-1.pdf}}

\textbf{Interatividade (\texttt{plotly})}

Transforma ggplots estáticos em HTML interativo com zoom e tooltips.

\begin{Shaded}
\begin{Highlighting}[]
\NormalTok{pacman}\SpecialCharTok{::}\FunctionTok{p\_load}\NormalTok{(plotly)}
\NormalTok{g }\OtherTok{\textless{}{-}} \FunctionTok{ggplot}\NormalTok{(iris, }\FunctionTok{aes}\NormalTok{(Sepal.Length, Sepal.Width, }\AttributeTok{color =}\NormalTok{ Species)) }\SpecialCharTok{+} \FunctionTok{geom\_point}\NormalTok{()}
\NormalTok{plotly}\SpecialCharTok{::}\FunctionTok{ggplotly}\NormalTok{(g)}
\end{Highlighting}
\end{Shaded}

\pandocbounded{\includegraphics[keepaspectratio]{intro_files/figure-pdf/unnamed-chunk-168-1.pdf}}

\textbf{Guardar/salvar o gráfico (\texttt{ggsave})}

Evite usar o botão \texttt{Export} da janela de plots para artigos. Ele
não garante resolução (DPI) nem tamanho fixo.

\begin{Shaded}
\begin{Highlighting}[]
\CommentTok{\# Salva o último gráfico plotado}
\FunctionTok{ggsave}\NormalTok{(}
  \AttributeTok{filename =} \StringTok{"figura\_final.pdf"}\NormalTok{, }\CommentTok{\# PDF é vetorial (melhor para artigos)}
  \AttributeTok{width =} \DecValTok{10}\NormalTok{, }
  \AttributeTok{height =} \DecValTok{6}\NormalTok{, }
  \AttributeTok{units =} \StringTok{"in"}\NormalTok{, }
  \AttributeTok{dpi =} \DecValTok{300} \CommentTok{\# 300dpi é o padrão para impressão}
\NormalTok{)}
\end{Highlighting}
\end{Shaded}

\chapter{Programação e Controle de
Fluxo}\label{programauxe7uxe3o-e-controle-de-fluxo}

A transição de um usuário de \texttt{R} ``operacional'' para um
cientista de dados ou estatístico avançado acontece quando se deixa de
rodar scripts linha a linha e passa-se a construir ferramentas robustas,
automatizadas e reprodutíveis.

\textbf{Estruturas Condicionais}

As condicionais permitem que o código tome decisões baseadas em testes
lógicos. No \texttt{R}, é crucial distinguir entre o controle de fluxo
(escalar) e a manipulação de dados (vetorizada).

A. O \texttt{if} (se) e \texttt{else} (caso contrário) são para Controle
de Fluxo

Esta estrutura controla se um bloco de código deve ser executado ou
ignorado, agindo como uma ``comporta'' no seu script.

\begin{itemize}
\item
  \texttt{if} (\ldots): A palavra-chave que inicia o teste.
\item
  \texttt{(\ )} Parênteses: Envolvem a condição lógica. O \texttt{R}
  avalia o que está aqui dentro. Deve resultar em um único \texttt{TRUE}
  ou \texttt{FALSE}.
\item
  \texttt{\{\ \}} Chaves: Delimitam o bloco de código. Elas dizem ao R:
  ``Tudo o que estiver aqui dentro pertence a este if''.
\end{itemize}

\begin{tcolorbox}[enhanced jigsaw, left=2mm, toptitle=1mm, colback=white, colframe=quarto-callout-tip-color-frame, colbacktitle=quarto-callout-tip-color!10!white, opacityback=0, rightrule=.15mm, bottomtitle=1mm, arc=.35mm, title=\textcolor{quarto-callout-tip-color}{\faLightbulb}\hspace{0.5em}{Por que usar?}, titlerule=0mm, bottomrule=.15mm, leftrule=.75mm, coltitle=black, toprule=.15mm, breakable, opacitybacktitle=0.6]

Sem as chaves \texttt{\{\ \}}, o \texttt{if} executaria apenas a
primeira linha imediatamente abaixo dele. As chaves \texttt{\{\ \}}
permitem agrupar dezenas de comandos (cálculos, gráficos, impressões)
para serem executados juntos caso a condição seja aceita.

\end{tcolorbox}

\begin{tcolorbox}[enhanced jigsaw, left=2mm, toptitle=1mm, colback=white, colframe=quarto-callout-important-color-frame, colbacktitle=quarto-callout-important-color!10!white, opacityback=0, rightrule=.15mm, bottomtitle=1mm, arc=.35mm, title=\textcolor{quarto-callout-important-color}{\faExclamation}\hspace{0.5em}{Regra do Escalar}, titlerule=0mm, bottomrule=.15mm, leftrule=.75mm, coltitle=black, toprule=.15mm, breakable, opacitybacktitle=0.6]

O \texttt{if} aceita apenas um valor lógico único. Não tente passar um
vetor inteiro (ex: \texttt{if\ (coluna\ \textgreater{}\ 0)}), pois isso
gerará um erro ou aviso. Para vetores, use \texttt{ifelse()}.

\end{tcolorbox}

\begin{Shaded}
\begin{Highlighting}[]
\NormalTok{x }\OtherTok{\textless{}{-}} \DecValTok{5}
\CommentTok{\#  A condição (x \textgreater{} 0) é testada dentro dos parênteses.}
\ControlFlowTok{if}\NormalTok{ (x }\SpecialCharTok{\textgreater{}} \DecValTok{0}\NormalTok{) \{                                }\CommentTok{\#se x for maior que zero}
  \CommentTok{\#Como é VERDADEIRO, o R entra neste bloco (entre chaves).}
  \CommentTok{\#Aqui podemos ter múltiplas linhas de comando.}
  \FunctionTok{print}\NormalTok{(}\StringTok{"O número é positivo."}\NormalTok{)             }\CommentTok{\#imprima "O número é positivo."}
\NormalTok{\} }\ControlFlowTok{else}\NormalTok{ \{                                    }\CommentTok{\#caso contrário}
\CommentTok{\# Este bloco só rodaria se x fosse negativo ou zero.}
  \FunctionTok{print}\NormalTok{(}\StringTok{"O número não é positivo."}\NormalTok{)         }\CommentTok{\#imprima "O número é negativo ou zero."}
\NormalTok{\}}
\end{Highlighting}
\end{Shaded}

\begin{verbatim}
[1] "O número é positivo."
\end{verbatim}

\begin{Shaded}
\begin{Highlighting}[]
\NormalTok{x }\OtherTok{\textless{}{-}} \DecValTok{5}

\ControlFlowTok{if}\NormalTok{ (x }\SpecialCharTok{\textgreater{}} \DecValTok{0}\NormalTok{) \{}
\NormalTok{  resultado }\OtherTok{\textless{}{-}}\NormalTok{ x }\SpecialCharTok{*} \DecValTok{2}
  \FunctionTok{print}\NormalTok{(}\FunctionTok{paste}\NormalTok{(}\StringTok{"O número é positivo. O dobro é:"}\NormalTok{, resultado))}
\NormalTok{\} }\ControlFlowTok{else}\NormalTok{ \{ }
  \FunctionTok{print}\NormalTok{(}\StringTok{"O número não é positivo."}\NormalTok{)}
\NormalTok{\}}
\end{Highlighting}
\end{Shaded}

\begin{verbatim}
[1] "O número é positivo. O dobro é: 10"
\end{verbatim}

\begin{tcolorbox}[enhanced jigsaw, left=2mm, toptitle=1mm, colback=white, colframe=quarto-callout-important-color-frame, colbacktitle=quarto-callout-important-color!10!white, opacityback=0, rightrule=.15mm, bottomtitle=1mm, arc=.35mm, title=\textcolor{quarto-callout-important-color}{\faExclamation}\hspace{0.5em}{Atenção à posição do else}, titlerule=0mm, bottomrule=.15mm, leftrule=.75mm, coltitle=black, toprule=.15mm, breakable, opacitybacktitle=0.6]

No \texttt{R}, o \texttt{else} deve estar na mesma linha da chave de
fechamento \texttt{\}} do \texttt{if}. Se você colocá-lo na linha de
baixo, o \texttt{R} achará que o código acabou antes e dará erro.

\end{tcolorbox}

B. O \texttt{ifelse()}- A Condicional Vetorizada

Enquanto o \texttt{if} tradicional falha ao receber múltiplos valores, o
\texttt{ifelse()} foi desenhado especificamente para lidar com vetores e
colunas de dados inteiras de uma só vez. Ele funciona de forma análoga à
função \texttt{=SE()\ do\ Excel}.

\begin{tcolorbox}[enhanced jigsaw, left=2mm, toptitle=1mm, colback=white, colframe=quarto-callout-tip-color-frame, colbacktitle=quarto-callout-tip-color!10!white, opacityback=0, rightrule=.15mm, bottomtitle=1mm, arc=.35mm, title=\textcolor{quarto-callout-tip-color}{\faLightbulb}\hspace{0.5em}{Como funciona}, titlerule=0mm, bottomrule=.15mm, leftrule=.75mm, coltitle=black, toprule=.15mm, breakable, opacitybacktitle=0.6]

\begin{itemize}
\tightlist
\item
  O \texttt{R} verifica a condição lógica para cada elemento do vetor,
  criando internamente uma lista de Verdadeiro (\texttt{TRUE}) /Falso.
\item
  Para cada posição onde o teste for \texttt{TRUE}, ele escolhe o valor
  do argumento \texttt{yes}. Onde for \texttt{FALSE}, escolhe o valor do
  argumento \texttt{no}.
\item
  Ele retorna um novo vetor com o mesmo comprimento do original,
  contendo as substituições feitas.
\end{itemize}

\end{tcolorbox}

\begin{tcolorbox}[enhanced jigsaw, left=2mm, toptitle=1mm, colback=white, colframe=quarto-callout-tip-color-frame, colbacktitle=quarto-callout-tip-color!10!white, opacityback=0, rightrule=.15mm, bottomtitle=1mm, arc=.35mm, title=\textcolor{quarto-callout-tip-color}{\faLightbulb}\hspace{0.5em}{Analogia com Excel}, titlerule=0mm, bottomrule=.15mm, leftrule=.75mm, coltitle=black, toprule=.15mm, breakable, opacitybacktitle=0.6]

Pense no \texttt{ifelse(coluna\ \textgreater{}\ 0,\ A,\ B)} exatamente
como a fórmula \texttt{=SE(A1\ \textgreater{}\ 0;\ "A";\ "B")} arrastada
para todas as linhas da planilha.

\end{tcolorbox}

\begin{Shaded}
\begin{Highlighting}[]
\NormalTok{notas }\OtherTok{\textless{}{-}} \FunctionTok{c}\NormalTok{(}\DecValTok{8}\NormalTok{, }\DecValTok{4}\NormalTok{, }\DecValTok{9}\NormalTok{, }\DecValTok{5}\NormalTok{)}
\NormalTok{status }\OtherTok{\textless{}{-}} \FunctionTok{ifelse}\NormalTok{(}\AttributeTok{test =}\NormalTok{ notas }\SpecialCharTok{\textgreater{}=} \DecValTok{7}\NormalTok{, }
                 \AttributeTok{yes =} \StringTok{"Aprovado"}\NormalTok{, }
                 \AttributeTok{no =} \StringTok{"Reprovado"}\NormalTok{)}
\CommentTok{\# fluxo:}
\CommentTok{\# 8 \textgreater{}= 7? SIM {-}\textgreater{} "Aprovado"}
\CommentTok{\# 4 \textgreater{}= 7? NÃO {-}\textgreater{} "Reprovado"}
\CommentTok{\# ... e assim por diante.}
\FunctionTok{print}\NormalTok{(status)}
\end{Highlighting}
\end{Shaded}

\begin{verbatim}
[1] "Aprovado"  "Reprovado" "Aprovado"  "Reprovado"
\end{verbatim}

\begin{Shaded}
\begin{Highlighting}[]
\CommentTok{\# Resultado: "Aprovado" "Reprovado" "Aprovado" "Reprovado"}
\end{Highlighting}
\end{Shaded}

\textbf{Loops (Laços de Repetição)}

Loops são usados para repetir uma tarefa múltiplas vezes.

A. \texttt{for} (Para cada\ldots) --- Iteração Definida

O laço \texttt{for} é a estrutura fundamental para repetir uma tarefa
quando você sabe antecipadamente sobre o que deseja iterar. Ele percorre
um vetor ou lista, elemento por elemento, executando o bloco de código
para cada um deles.

\begin{itemize}
\tightlist
\item
  \texttt{for\ (...):} A palavra-chave que inicia a estrutura de
  repetição.
\item
  \texttt{(\ var\ in\ seq\ )} Parênteses: Definem a regra da iteração.
\item
  \texttt{var\ (ex:\ i):} É a variável iteradora. Ela é um
  ``placeholder'' (um cursor) que assume um valor diferente a cada volta
  do \texttt{loop}.
\item
  \texttt{in:} O operador que conecta a variável à sequência. Lê-se:
  \emph{Para cada i DENTRO de seq}.
\item
  \texttt{seq:} O vetor ou lista que será percorrido (ex: 1:10 ou
  c(``A'', ``B'')).
\item
  \texttt{\{\ \}} Chaves: Delimitam o bloco de execução. Todo o código
  aqui dentro será repetido N vezes, onde N é o tamanho da sequência.
\end{itemize}

A.1 \textbf{Lógica de Execução:}

\begin{enumerate}
\def\labelenumi{\arabic{enumi}.}
\item
  Na 1ª volta, i assume o primeiro valor da sequência. O código entre
  chaves roda.
\item
  Na 2ª volta, i assume o segundo valor. O código roda novamente.
\item
  O processo repete até que acabem os itens da sequência.
\end{enumerate}

\begin{Shaded}
\begin{Highlighting}[]
\NormalTok{frutas }\OtherTok{\textless{}{-}} \FunctionTok{c}\NormalTok{(}\StringTok{"Maçã"}\NormalTok{, }\StringTok{"Banana"}\NormalTok{, }\StringTok{"Uva"}\NormalTok{)}

\ControlFlowTok{for}\NormalTok{ (item }\ControlFlowTok{in}\NormalTok{ frutas) \{                    }\CommentTok{\# Para cada \textquotesingle{}fruta\textquotesingle{} dentro do vetor \textquotesingle{}frutas\textquotesingle{}...}
  \CommentTok{\# ... execute este bloco:}
\NormalTok{  mensagem }\OtherTok{\textless{}{-}} \FunctionTok{paste}\NormalTok{(}\StringTok{"Eu gosto de"}\NormalTok{, item)}
  \FunctionTok{print}\NormalTok{(mensagem)}
\NormalTok{\}}
\end{Highlighting}
\end{Shaded}

\begin{verbatim}
[1] "Eu gosto de Maçã"
[1] "Eu gosto de Banana"
[1] "Eu gosto de Uva"
\end{verbatim}

\begin{Shaded}
\begin{Highlighting}[]
\CommentTok{\# O loop roda 3 vezes, pois há 3 frutas.}
\end{Highlighting}
\end{Shaded}

\begin{tcolorbox}[enhanced jigsaw, left=2mm, toptitle=1mm, colback=white, colframe=quarto-callout-tip-color-frame, colbacktitle=quarto-callout-tip-color!10!white, opacityback=0, rightrule=.15mm, bottomtitle=1mm, arc=.35mm, title=\textcolor{quarto-callout-tip-color}{\faLightbulb}\hspace{0.5em}{seq\_along()}, titlerule=0mm, bottomrule=.15mm, leftrule=.75mm, coltitle=black, toprule=.15mm, breakable, opacitybacktitle=0.6]

Ao iterar sobre um vetor, evite usar \texttt{1:length(vetor)}, pois se o
vetor estiver vazio, isso gera um erro (gera a sequência 1:0). Prefira
\texttt{seq\_along(vetor)}, que lida corretamente com vetores vazios.

\end{tcolorbox}

\begin{Shaded}
\begin{Highlighting}[]
\NormalTok{vetor\_vazio }\OtherTok{\textless{}{-}} \FunctionTok{c}\NormalTok{() }\CommentTok{\# Criando um vetor vazio}

\CommentTok{\#FORMA PERIGOSA}
\ControlFlowTok{for}\NormalTok{ (i }\ControlFlowTok{in} \DecValTok{1}\SpecialCharTok{:}\FunctionTok{length}\NormalTok{(vetor\_vazio)) \{}
  \FunctionTok{print}\NormalTok{(}\FunctionTok{paste}\NormalTok{(}\StringTok{"Tentando acessar o índice:"}\NormalTok{, i))}
\NormalTok{\}}
\DocumentationTok{\#\# [1] "Tentando acessar o índice: 1"}
\DocumentationTok{\#\# [1] "Tentando acessar o índice: 0"}
\CommentTok{\#Resultado: "Tentando acessar o índice: 1".  O loop roda duas vezes tentando acessar dados que não existem!}

\CommentTok{\#FORMA SEGURA}
\ControlFlowTok{for}\NormalTok{ (i }\ControlFlowTok{in} \FunctionTok{seq\_along}\NormalTok{(vetor\_vazio)) \{}
  \FunctionTok{print}\NormalTok{(}\FunctionTok{paste}\NormalTok{(}\StringTok{"Tentando acessar o índice:"}\NormalTok{, i))}
\NormalTok{\}}
\CommentTok{\# Resultado: "Tentando acessar o índice: 0". O loop é ignorado corretamente (silêncio total).}
\end{Highlighting}
\end{Shaded}

B. O \texttt{while} (Enquanto\ldots) --- Iteração Indefinida

Diferente do \texttt{for}, que percorre uma sequência fixa, o
\texttt{while} é usado quando não sabemos antecipadamente quantas vezes
o código precisa rodar. Ele repete o bloco de instruções indefinidamente
enquanto a condição especificada for verdadeira.

\begin{itemize}
\tightlist
\item
  \texttt{while\ (...):} A palavra-chave de controle.
\item
  \texttt{(\ condição\ )} Parênteses: Envolvem o critério de parada. O R
  avalia essa expressão lógica antes de cada volta. Se for
  \texttt{TRUE}, entra no loop. Se for \texttt{FALSE}, o loop é
  encerrado imediatamente.
\item
  \texttt{\{\ \}} Chaves: Delimitam o bloco de execução.
\end{itemize}

\begin{tcolorbox}[enhanced jigsaw, left=2mm, toptitle=1mm, colback=white, colframe=quarto-callout-note-color-frame, colbacktitle=quarto-callout-note-color!10!white, opacityback=0, rightrule=.15mm, bottomtitle=1mm, arc=.35mm, title=\textcolor{quarto-callout-note-color}{\faInfo}\hspace{0.5em}{Nota}, titlerule=0mm, bottomrule=.15mm, leftrule=.75mm, coltitle=black, toprule=.15mm, breakable, opacitybacktitle=0.6]

Dentro destas chaves, deve haver obrigatoriamente um comando que altere
a variável testada na condição (ex: um incremento). Se a condição nunca
se tornar \texttt{FALSE}, cria-se um loop infinito, travando a sessão do
\texttt{R}.

\end{tcolorbox}

\textbf{Lógica de Execução}

\begin{enumerate}
\def\labelenumi{\arabic{enumi}.}
\item
  Testa a condição.
\item
  Se TRUE: Executa o bloco \{\}.
\item
  Ao final do bloco, volta para o passo 1.
\item
  Se FALSE: Pula o bloco e segue o script.
\end{enumerate}

\begin{Shaded}
\begin{Highlighting}[]
\NormalTok{contador }\OtherTok{\textless{}{-}} \DecValTok{1}
\ControlFlowTok{while}\NormalTok{ (contador }\SpecialCharTok{\textless{}=} \DecValTok{3}\NormalTok{) \{               }\CommentTok{\# Enquanto o contador for menor ou igual a 3...}
  \FunctionTok{print}\NormalTok{(}\FunctionTok{paste}\NormalTok{(}\StringTok{"Contagem:"}\NormalTok{, contador))}
\NormalTok{  contador }\OtherTok{\textless{}{-}}\NormalTok{ contador }\SpecialCharTok{+} \DecValTok{1}            \CommentTok{\# Passo Obrigatório: Atualizar a variável de controle}
\NormalTok{\}}
\end{Highlighting}
\end{Shaded}

\begin{verbatim}
[1] "Contagem: 1"
[1] "Contagem: 2"
[1] "Contagem: 3"
\end{verbatim}

\begin{Shaded}
\begin{Highlighting}[]
\CommentTok{\# Quando contador vira 4, a condição (4 \textless{}= 3) é FALSE e o loop para.}
\end{Highlighting}
\end{Shaded}

\begin{tcolorbox}[enhanced jigsaw, left=2mm, toptitle=1mm, colback=white, colframe=quarto-callout-warning-color-frame, colbacktitle=quarto-callout-warning-color!10!white, opacityback=0, rightrule=.15mm, bottomtitle=1mm, arc=.35mm, title=\textcolor{quarto-callout-warning-color}{\faExclamationTriangle}\hspace{0.5em}{Gestão de Memória: Não faça o vetor crescer}, titlerule=0mm, bottomrule=.15mm, leftrule=.75mm, coltitle=black, toprule=.15mm, breakable, opacitybacktitle=0.6]

Um erro clássico é começar com um vetor vazio
\texttt{(x\ \textless{}-\ c())} e adicionar elementos dentro do loop
\texttt{(x{[}i{]}\ \textless{}-\ valor)}.

\textbf{Por que é ruim?} No \texttt{R}, vetores precisam de memória
contígua. A cada novo elemento adicionado, o \texttt{R} precisa: (1)
achar um novo espaço maior na memória, (2) copiar todos os dados antigos
para lá, (3) adicionar o novo dado e (4) apagar o antigo. Fazer isso
milhares de vezes torna o script extremamente lento (\(O(n^2)\)).

A Solução (Pré-alocação) é criar o vetor com o tamanho final desejado
preenchido com zeros ou NAs.

\end{tcolorbox}

\begin{Shaded}
\begin{Highlighting}[]
\NormalTok{n }\OtherTok{\textless{}{-}} \DecValTok{1000} \CommentTok{\# Forma Correta: Aloca memória apenas uma vez}
\NormalTok{vetor\_resultado }\OtherTok{\textless{}{-}} \FunctionTok{numeric}\NormalTok{(}\AttributeTok{length =}\NormalTok{ n)  }\CommentTok{\#vetor vazio a ser preenchido}

\ControlFlowTok{for}\NormalTok{ (i }\ControlFlowTok{in} \DecValTok{1}\SpecialCharTok{:}\NormalTok{n) \{}
\NormalTok{  vetor\_resultado[i] }\OtherTok{\textless{}{-}}\NormalTok{ i }\SpecialCharTok{*} \DecValTok{2} \CommentTok{\# Apenas preenche a gaveta já existente}
\NormalTok{\}}
\end{Highlighting}
\end{Shaded}

\section{Funções personalizadas}\label{funuxe7uxf5es-personalizadas}

Crie suas Próprias Ferramentas

O princípio
\href{https://en.wikipedia.org/wiki/Don\%27t_repeat_yourself}{DRY}
(\emph{Don't Repeat Yourself} --- Não se Repita) defende que se você
precisou copiar e colar o mesmo bloco de código mais de duas vezes, é
hora de transformá-lo em uma função.

\begin{itemize}
\item
  Uma função funciona como uma ``caixa preta'' ou uma pequena máquina:
  você insere ingredientes (argumentos), a máquina processa (corpo) e
  entrega um produto final (retorno).

  \begin{itemize}
  \tightlist
  \item
    nome \textless- \texttt{function(...)}: Você ``salva'' a lógica
    dentro de um objeto. A partir de agora, o \texttt{R} reconhece esse
    nome como um comando executável.
  \item
    ( arg1, arg2 = padrão ) Argumentos: São as entradas. Funcionam como
    variáveis temporárias que só existem dentro da função. Você pode
    definir valores padrão (=), que serão usados caso o usuário não
    informe nada naquele argumento.
  \end{itemize}
\item
  \texttt{\{\ \}} Corpo: O bloco de processamento. É um ``ambiente
  isolado'' (escopo local). Variáveis criadas aqui dentro nascem e
  morrem aqui; elas não poluem o seu ambiente de trabalho global.
\item
  return(\ldots) Retorno: A saída. Define explicitamente o que a função
  devolve para o usuário. Se omitido, o \texttt{R} retorna o resultado
  da última linha executada, mas usar return() torna o código mais
  legível.
\end{itemize}

\begin{Shaded}
\begin{Highlighting}[]
\NormalTok{nome\_da\_funcao }\OtherTok{\textless{}{-}} \ControlFlowTok{function}\NormalTok{(arg1, }\AttributeTok{arg2 =}\NormalTok{ valor\_padrao) \{}
\NormalTok{  soma }\OtherTok{\textless{}{-}}\NormalTok{ arg1 }\SpecialCharTok{+}\NormalTok{ arg2}
  \FunctionTok{return}\NormalTok{(soma)          }\CommentTok{\# Retorno: O que sai da função}
\NormalTok{\}}

\CommentTok{\# Uso}
\NormalTok{resultado }\OtherTok{\textless{}{-}} \FunctionTok{nome\_da\_funcao}\NormalTok{(}\AttributeTok{arg1 =} \DecValTok{10}\NormalTok{, }\AttributeTok{arg2 =} \DecValTok{5}\NormalTok{)}
\end{Highlighting}
\end{Shaded}

\begin{Shaded}
\begin{Highlighting}[]
\NormalTok{calcular\_imc }\OtherTok{\textless{}{-}} \ControlFlowTok{function}\NormalTok{(peso, altura) \{}
\NormalTok{  imc\_valor }\OtherTok{\textless{}{-}}\NormalTok{ peso }\SpecialCharTok{/}\NormalTok{ (altura }\SpecialCharTok{\^{}} \DecValTok{2}\NormalTok{)              }\CommentTok{\# O R recebe os valores, calcula e guarda em \textquotesingle{}imc\_valor\textquotesingle{}}
  \FunctionTok{return}\NormalTok{(imc\_valor)                             }\CommentTok{\# Retorno: O que é enviado de volta para quem chamou}
\NormalTok{\}}

\NormalTok{meu\_indice }\OtherTok{\textless{}{-}} \FunctionTok{calcular\_imc}\NormalTok{(}\AttributeTok{peso =} \DecValTok{80}\NormalTok{, }\AttributeTok{altura =} \FloatTok{1.80}\NormalTok{)}
\FunctionTok{print}\NormalTok{(meu\_indice) }\CommentTok{\# Resultado: \textasciitilde{}24.69}
\end{Highlighting}
\end{Shaded}

\begin{verbatim}
[1] 24.69136
\end{verbatim}

\begin{Shaded}
\begin{Highlighting}[]
\CommentTok{\# Nota: A variável \textquotesingle{}imc\_valor\textquotesingle{} não existe aqui fora. Ela morreu ao fim da função.}
\end{Highlighting}
\end{Shaded}

\begin{tcolorbox}[enhanced jigsaw, left=2mm, toptitle=1mm, colback=white, colframe=quarto-callout-important-color-frame, colbacktitle=quarto-callout-important-color!10!white, opacityback=0, rightrule=.15mm, bottomtitle=1mm, arc=.35mm, title=\textcolor{quarto-callout-important-color}{\faExclamation}\hspace{0.5em}{Escopo Local vs.~Global}, titlerule=0mm, bottomrule=.15mm, leftrule=.75mm, coltitle=black, toprule=.15mm, breakable, opacitybacktitle=0.6]

Tudo o que acontece dentro das chaves \texttt{\{\}} da função fica
isolado. Se você criar uma variável \texttt{x\ \textless{}-\ 10} dentro
da função, ela não altera uma variável x que já exista no seu ambiente
global. Isso garante segurança e evita bugs.

\end{tcolorbox}

\section{Programação Funcional (A alternativa aos
Loops)}\label{programauxe7uxe3o-funcional-a-alternativa-aos-loops}

O \texttt{R} é, em sua essência, uma linguagem funcional. Enquanto o
loop \texttt{for} é imperativo (você dá ordens passo-a-passo:
\emph{pegue o item 1, faça isso, salve ali}), a programação funcional é
declarativa (você diz o que quer: \emph{aplique esta função a todos os
itens desta lista}). Essa abordagem abstrai a complexidade da iteração,
resultando em códigos mais limpos, legíveis e menos propensos a erros de
indexação.

A. \textbf{Família \texttt{apply} (R Base)}

Este é o conjunto de funções nativas do \texttt{R} para iteração. São
robustas e onipresentes em códigos legados e pacotes fundamentais.

\begin{itemize}
\tightlist
\item
  \texttt{apply(X,\ MARGIN,\ FUN)}: Projetada especificamente para
  estruturas bidimensionais (Matrizes e Data Frames). O argumento
  \texttt{MARGIN} define a direção: se for 1-aplica a função nas linhas,
  se for 2-aplica nas colunas.
\end{itemize}

\begin{Shaded}
\begin{Highlighting}[]
\NormalTok{matriz }\OtherTok{\textless{}{-}} \FunctionTok{matrix}\NormalTok{(}\DecValTok{1}\SpecialCharTok{:}\DecValTok{9}\NormalTok{, }\AttributeTok{nrow =} \DecValTok{3}\NormalTok{)  }\CommentTok{\# Criando uma matriz simples 3x3}

\CommentTok{\# Aplicando apply nas LINHAS (Margin = 1) {-}\textgreater{} Soma}
\FunctionTok{apply}\NormalTok{(matriz, }\DecValTok{1}\NormalTok{, sum)  }\CommentTok{\# 1+4+7, 2+5+8...}
\end{Highlighting}
\end{Shaded}

\begin{verbatim}
[1] 12 15 18
\end{verbatim}

\begin{Shaded}
\begin{Highlighting}[]
\CommentTok{\# Aplicando apply nas COLUNAS (Margin = 2) {-}\textgreater{} Média}
\FunctionTok{apply}\NormalTok{(matriz, }\DecValTok{2}\NormalTok{, mean)}
\end{Highlighting}
\end{Shaded}

\begin{verbatim}
[1] 2 5 8
\end{verbatim}

\begin{itemize}
\tightlist
\item
  \texttt{lapply(X,\ FUN)\ (List\ Apply):} A função mais importante da
  família. Recebe um vetor ou lista, aplica a função a cada elemento e
  retorna sempre uma Lista. É a mais segura programaticamente, pois o
  formato de saída é previsível.
\end{itemize}

\begin{Shaded}
\begin{Highlighting}[]
\NormalTok{números }\OtherTok{\textless{}{-}} \FunctionTok{list}\NormalTok{(}\AttributeTok{a =} \DecValTok{4}\NormalTok{, }\AttributeTok{b =} \DecValTok{16}\NormalTok{, }\AttributeTok{c =} \DecValTok{25}\NormalTok{)}

\CommentTok{\# Aplica a raiz quadrada em cada item}
\NormalTok{resultado\_lista }\OtherTok{\textless{}{-}} \FunctionTok{lapply}\NormalTok{(números, sqrt);resultado\_lista}
\end{Highlighting}
\end{Shaded}

\begin{verbatim}
$a
[1] 2

$b
[1] 4

$c
[1] 5
\end{verbatim}

\begin{Shaded}
\begin{Highlighting}[]
\FunctionTok{class}\NormalTok{(resultado\_lista) }\CommentTok{\# Confirma que é uma lista}
\end{Highlighting}
\end{Shaded}

\begin{verbatim}
[1] "list"
\end{verbatim}

\begin{itemize}
\tightlist
\item
  \texttt{sapply(X,\ FUN)\ (Simplify\ Apply):} Uma versão de
  conveniência do \texttt{lapply.} Ela tenta ``adivinhar'' o formato de
  saída mais simples (um vetor ou matriz) para o resultado.
\end{itemize}

\begin{Shaded}
\begin{Highlighting}[]
\CommentTok{\# A mesma operação acima, mas retornando um VETOR limpo}
\NormalTok{resultado\_vetor }\OtherTok{\textless{}{-}} \FunctionTok{sapply}\NormalTok{(números, sqrt);resultado\_vetor}
\end{Highlighting}
\end{Shaded}

\begin{verbatim}
a b c 
2 4 5 
\end{verbatim}

\begin{Shaded}
\begin{Highlighting}[]
\FunctionTok{class}\NormalTok{(resultado\_vetor) }\CommentTok{\# Confirma que simplificou para numérico}
\end{Highlighting}
\end{Shaded}

\begin{verbatim}
[1] "numeric"
\end{verbatim}

\begin{tcolorbox}[enhanced jigsaw, left=2mm, toptitle=1mm, colback=white, colframe=quarto-callout-important-color-frame, colbacktitle=quarto-callout-important-color!10!white, opacityback=0, rightrule=.15mm, bottomtitle=1mm, arc=.35mm, title=\textcolor{quarto-callout-important-color}{\faExclamation}\hspace{0.5em}{Cuidado}, titlerule=0mm, bottomrule=.15mm, leftrule=.75mm, coltitle=black, toprule=.15mm, breakable, opacitybacktitle=0.6]

Se a função falhar ou retornar tipos diferentes, o \texttt{sapply} pode
devolver uma lista inesperadamente, quebrando scripts automatizados.

\end{tcolorbox}

B. \textbf{Pacote purrr (Tidyverse)}

A evolução moderna da família \texttt{apply.} O \texttt{purrr} resolve o
problema da imprevisibilidade do \texttt{sapply} introduzindo a
Segurança de Tipos
(\href{https://en.wikipedia.org/wiki/Type_safety}{Type Safety}). O
sufixo da função determina obrigatoriamente o tipo de dado que será
retornado.

\begin{Shaded}
\begin{Highlighting}[]
\NormalTok{pacman}\SpecialCharTok{::}\FunctionTok{p\_load}\NormalTok{(purrr)}

\NormalTok{notas }\OtherTok{\textless{}{-}} \FunctionTok{list}\NormalTok{(}
  \AttributeTok{joao =} \FunctionTok{c}\NormalTok{(}\DecValTok{10}\NormalTok{, }\DecValTok{9}\NormalTok{, }\DecValTok{8}\NormalTok{),}
  \AttributeTok{maria =} \FunctionTok{c}\NormalTok{(}\DecValTok{5}\NormalTok{, }\DecValTok{6}\NormalTok{, }\DecValTok{7}\NormalTok{),}
  \AttributeTok{pedro =} \FunctionTok{c}\NormalTok{(}\DecValTok{2}\NormalTok{, }\DecValTok{4}\NormalTok{, }\DecValTok{3}\NormalTok{)}
\NormalTok{)}
\end{Highlighting}
\end{Shaded}

\begin{itemize}
\tightlist
\item
  \texttt{map(x,\ fun):} Equivalente ao \texttt{lapply.} Retorna sempre
  uma Lista.
\end{itemize}

\begin{Shaded}
\begin{Highlighting}[]
\CommentTok{\# Retorna uma lista com a média de cada aluno}
\FunctionTok{map}\NormalTok{(notas, mean)}
\end{Highlighting}
\end{Shaded}

\begin{verbatim}
$joao
[1] 9

$maria
[1] 6

$pedro
[1] 3
\end{verbatim}

\begin{itemize}
\tightlist
\item
  \texttt{map\_dbl(x,\ fun):} Aplica a função e retorna um vetor
  Numérico (\texttt{double}).
\end{itemize}

\begin{Shaded}
\begin{Highlighting}[]
\CommentTok{\# Retorna um vetor numérico limpo com as médias}
\FunctionTok{map\_dbl}\NormalTok{(notas, mean)}
\end{Highlighting}
\end{Shaded}

\begin{verbatim}
 joao maria pedro 
    9     6     3 
\end{verbatim}

\begin{tcolorbox}[enhanced jigsaw, left=2mm, toptitle=1mm, colback=white, colframe=quarto-callout-tip-color-frame, colbacktitle=quarto-callout-tip-color!10!white, opacityback=0, rightrule=.15mm, bottomtitle=1mm, arc=.35mm, title=\textcolor{quarto-callout-tip-color}{\faLightbulb}\hspace{0.5em}{Dica}, titlerule=0mm, bottomrule=.15mm, leftrule=.75mm, coltitle=black, toprule=.15mm, breakable, opacitybacktitle=0.6]

Se a função tentar retornar um texto ou booleano, o \texttt{map\_dbl}
gera um erro imediato, alertando sobre a inconsistência nos dados.

\end{tcolorbox}

\begin{itemize}
\tightlist
\item
  \texttt{map\_dfr(x,\ fun):} Aplica a função e tenta empilhar os
  resultados, retornando um Data Frame unido pelas linhas
  (\texttt{Data\ Frame\ Row-bind}).
\end{itemize}

\begin{Shaded}
\begin{Highlighting}[]
\CommentTok{\# A função anônima (\textasciitilde{}) cria uma tabela para cada aluno}
\FunctionTok{map\_dfr}\NormalTok{(notas, }\SpecialCharTok{\textasciitilde{}}\FunctionTok{data.frame}\NormalTok{(}
  \AttributeTok{media =} \FunctionTok{mean}\NormalTok{(.x), }
  \AttributeTok{status =} \ControlFlowTok{if}\NormalTok{(}\FunctionTok{mean}\NormalTok{(.x) }\SpecialCharTok{\textgreater{}=} \DecValTok{7}\NormalTok{) }\StringTok{"Aprovado"} \ControlFlowTok{else} \StringTok{"Reprovado"}
\NormalTok{), }\AttributeTok{.id =} \StringTok{"aluno"}\NormalTok{)}\SpecialCharTok{\%\textgreater{}\%} \CommentTok{\# .id cria a coluna com o nome da lista original}
\FunctionTok{gt}\NormalTok{()}
\end{Highlighting}
\end{Shaded}

\begin{table}
\fontsize{12.0pt}{14.0pt}\selectfont
\begin{tabular*}{\linewidth}{@{\extracolsep{\fill}}lrl}
\toprule
aluno & media & status \\ 
\midrule\addlinespace[2.5pt]
joao & 9 & Aprovado \\ 
maria & 6 & Reprovado \\ 
pedro & 3 & Reprovado \\ 
\bottomrule
\end{tabular*}
\end{table}

\begin{tcolorbox}[enhanced jigsaw, left=2mm, toptitle=1mm, colback=white, colframe=quarto-callout-note-color-frame, colbacktitle=quarto-callout-note-color!10!white, opacityback=0, rightrule=.15mm, bottomtitle=1mm, arc=.35mm, title=\textcolor{quarto-callout-note-color}{\faInfo}\hspace{0.5em}{Nota}, titlerule=0mm, bottomrule=.15mm, leftrule=.75mm, coltitle=black, toprule=.15mm, breakable, opacitybacktitle=0.6]

Se tentássemos usar map\_dbl em uma coluna de texto, o \texttt{R} daria
erro, protegendo nossa análise de cálculos inválidos.

\end{tcolorbox}

\section{Relatórios Dinâmicos (R Markdown e
Quarto)}\label{relatuxf3rios-dinuxe2micos-r-markdown-e-quarto}

A reprodutibilidade é o pilar da ciência moderna. Ferramentas como
\href{https://rmarkdown.rstudio.com/}{R Markdown} (\texttt{.Rmd}) e seu
sucessor \href{https://quarto.org/}{Quarto} (\texttt{.qmd}) permitem
integrar código, narrativa e resultados em um único documento. Leia os
materiais colocados nas referências bibliográficas.

\section{Tabelas Profissionais}\label{tabelas-profissionais}

A apresentação de tabelas deve ser adequada ao formato final do
relatório (HTML, Word ou PDF).

A. \textbf{Para Word e PowerPoint (\texttt{flextable})}

Se o seu destino é um documento Office editável, esta é a melhor opção.
Converte data frames em objetos nativos do Word.

\begin{Shaded}
\begin{Highlighting}[]
\NormalTok{pacman}\SpecialCharTok{::}\FunctionTok{p\_load}\NormalTok{(gdtools,flextable)}
\NormalTok{Tab }\OtherTok{\textless{}{-}} \FunctionTok{head}\NormalTok{(mtcars) }\SpecialCharTok{|\textgreater{}} 
  \FunctionTok{flextable}\NormalTok{() }\SpecialCharTok{|\textgreater{}} 
  \FunctionTok{autofit}\NormalTok{() }\SpecialCharTok{|\textgreater{}} \CommentTok{\# Ajusta largura das colunas}
  \FunctionTok{color}\NormalTok{(}\AttributeTok{i =} \DecValTok{1}\NormalTok{, }\AttributeTok{color =} \StringTok{"black"}\NormalTok{, }\AttributeTok{part =} \StringTok{"header"}\NormalTok{) }\SpecialCharTok{|\textgreater{}} \CommentTok{\# Cabeçalho vermelho}
  \FunctionTok{save\_as\_docx}\NormalTok{(}\AttributeTok{path =} \StringTok{"tabela.docx"}\NormalTok{) }\CommentTok{\#salva em word}
\CommentTok{\#}
\FunctionTok{head}\NormalTok{(mtcars) }\SpecialCharTok{|\textgreater{}} 
  \FunctionTok{flextable}\NormalTok{() }\SpecialCharTok{|\textgreater{}} 
  \FunctionTok{autofit}\NormalTok{() }\SpecialCharTok{|\textgreater{}} \CommentTok{\# Ajusta largura das colunas}
  \FunctionTok{color}\NormalTok{(}\AttributeTok{i =} \DecValTok{1}\NormalTok{, }\AttributeTok{color =} \StringTok{"black"}\NormalTok{, }\AttributeTok{part =} \StringTok{"header"}\NormalTok{)}
\end{Highlighting}
\end{Shaded}

B. \textbf{Para Publicação Acadêmica - HTML/PDF (\texttt{gt})}

Conhecido como o ``ggplot das tabelas''. Permite estruturações complexas
com notas de rodapé, títulos e agrupamentos.

\begin{Shaded}
\begin{Highlighting}[]
\NormalTok{pacman}\SpecialCharTok{::}\FunctionTok{p\_load}\NormalTok{(gt)}
\FunctionTok{head}\NormalTok{(mtcars) }\SpecialCharTok{|\textgreater{}} 
  \FunctionTok{gt}\NormalTok{() }\SpecialCharTok{|\textgreater{}} 
  \FunctionTok{tab\_header}\NormalTok{(}\AttributeTok{title =} \StringTok{""}\NormalTok{) }\SpecialCharTok{|\textgreater{}} 
  \FunctionTok{fmt\_number}\NormalTok{(}\AttributeTok{columns =} \FunctionTok{c}\NormalTok{(mpg, wt), }\AttributeTok{decimals =} \DecValTok{2}\NormalTok{) }\SpecialCharTok{|\textgreater{}} \CommentTok{\# Formatação precisa}
  \FunctionTok{tab\_source\_note}\NormalTok{(}\AttributeTok{source\_note =} \StringTok{"Fonte: Motor Trend"}\NormalTok{)}
\end{Highlighting}
\end{Shaded}

\begin{table}
\caption{Análise de Veículos}\tabularnewline

\caption*{
{\fontsize{20}{25}\selectfont  \fontsize{12}{15}\selectfont }
} 
\fontsize{12.0pt}{14.0pt}\selectfont
\begin{tabular*}{\linewidth}{@{\extracolsep{\fill}}rrrrrrrrrrr}
\toprule
mpg & cyl & disp & hp & drat & wt & qsec & vs & am & gear & carb \\ 
\midrule\addlinespace[2.5pt]
21.00 & 6 & 160 & 110 & 3.90 & 2.62 & 16.46 & 0 & 1 & 4 & 4 \\ 
21.00 & 6 & 160 & 110 & 3.90 & 2.88 & 17.02 & 0 & 1 & 4 & 4 \\ 
22.80 & 4 & 108 & 93 & 3.85 & 2.32 & 18.61 & 1 & 1 & 4 & 1 \\ 
21.40 & 6 & 258 & 110 & 3.08 & 3.21 & 19.44 & 1 & 0 & 3 & 1 \\ 
18.70 & 8 & 360 & 175 & 3.15 & 3.44 & 17.02 & 0 & 0 & 3 & 2 \\ 
18.10 & 6 & 225 & 105 & 2.76 & 3.46 & 20.22 & 1 & 0 & 3 & 1 \\ 
\bottomrule
\end{tabular*}
\begin{minipage}{\linewidth}
\vspace{.05em}
Fonte: Motor Trend\\
\end{minipage}
\end{table}

C. \textbf{Para Resumo Estatístico (\texttt{gtsummary})}

Automatiza a criação da Tabela de artigos científicos, calculando
descritivas e testes estatísticos automaticamente.

\begin{Shaded}
\begin{Highlighting}[]
\NormalTok{pacman}\SpecialCharTok{::}\FunctionTok{p\_load}\NormalTok{(gtsummary)}
\NormalTok{mtcars }\SpecialCharTok{|\textgreater{}} 
  \FunctionTok{select}\NormalTok{(mpg, cyl, hp) }\SpecialCharTok{|\textgreater{}} 
  \FunctionTok{tbl\_summary}\NormalTok{(}
    \AttributeTok{by =}\NormalTok{ cyl, }\CommentTok{\# Agrupa por cilindros}
    \AttributeTok{statistic =} \FunctionTok{list}\NormalTok{(}\FunctionTok{all\_continuous}\NormalTok{() }\SpecialCharTok{\textasciitilde{}} \StringTok{"\{mean\} (\{sd\})"}\NormalTok{)}
\NormalTok{  ) }\SpecialCharTok{|\textgreater{}} 
  \FunctionTok{add\_p}\NormalTok{() }\CommentTok{\# Adiciona p{-}valor (ANOVA/Kruskal)}
\end{Highlighting}
\end{Shaded}

\begin{table}
\fontsize{12.0pt}{14.0pt}\selectfont
\begin{tabular*}{\linewidth}{@{\extracolsep{\fill}}lcccc}
\toprule
\textbf{Characteristic} & \textbf{4}  N = 11\textsuperscript{\textit{1}} & \textbf{6}  N = 7\textsuperscript{\textit{1}} & \textbf{8}  N = 14\textsuperscript{\textit{1}} & \textbf{p-value}\textsuperscript{\textit{2}} \\ 
\midrule\addlinespace[2.5pt]
mpg & 26.7 (4.5) & 19.7 (1.5) & 15.1 (2.6) & <0.001 \\ 
hp & 83 (21) & 122 (24) & 209 (51) & <0.001 \\ 
\bottomrule
\end{tabular*}
\begin{minipage}{\linewidth}
\vspace{.05em}
\textsuperscript{\textit{1}} Mean (SD)\\
\textsuperscript{\textit{2}} Kruskal-Wallis rank sum test\\
\end{minipage}
\end{table}

D. \textbf{Para Dashboards Interativos (\texttt{reactable})}

Ideal para relatórios HTML onde o usuário precisa filtrar, buscar ou
ordenar os dados.

\begin{Shaded}
\begin{Highlighting}[]
\NormalTok{pacman}\SpecialCharTok{::}\FunctionTok{p\_load}\NormalTok{(reactable)}
\FunctionTok{reactable}\NormalTok{(}
\NormalTok{  mtcars,}
  \AttributeTok{searchable =} \ConstantTok{TRUE}\NormalTok{,  }\CommentTok{\# Adiciona barra de busca}
  \AttributeTok{striped =} \ConstantTok{TRUE}\NormalTok{,     }\CommentTok{\# Linhas zebradas}
  \AttributeTok{highlight =} \ConstantTok{TRUE}    \CommentTok{\# Destaque ao passar o mouse}
\NormalTok{)}
\end{Highlighting}
\end{Shaded}

\pandocbounded{\includegraphics[keepaspectratio]{intro_files/figure-pdf/unnamed-chunk-182-1.pdf}}

\section{Equações Matemáticas
(LaTeX)}\label{equauxe7uxf5es-matemuxe1ticas-latex}

A escrita matemática em relatórios dinâmicos utiliza a sintaxe
\href{https://en.wikipedia.org/wiki/LaTeX}{LaTeX}, o padrão global para
tipografia científica. O \texttt{R\ Markdown/Quarto} renderiza esses
códigos em fórmulas visuais de alta qualidade tanto em HTML quanto em
PDF.

A. \textbf{Modos de Exibição (Inline vs.~Display)}

O modo \emph{inline} (\texttt{\$}) insere a fórmula na fluidez do texto,
enquanto o modo \emph{display} (\texttt{\$\$}) quebra a linha e
centraliza a equação para destaque.

\begin{Shaded}
\begin{Highlighting}[]
\NormalTok{O modelo de regressão linear simples é dado por }\SpecialCharTok{$}\NormalTok{y }\OtherTok{=}\NormalTok{ \textbackslash{}alpha }\SpecialCharTok{+}\NormalTok{ \textbackslash{}beta x }\SpecialCharTok{+}\NormalTok{ \textbackslash{}epsilon}\SpecialCharTok{$}\NormalTok{.}

\NormalTok{A estimativa dos parâmetros minimiza a Soma dos Quadrados dos Resíduos}\SpecialCharTok{:}
\ErrorTok{$$}\FunctionTok{S}\NormalTok{(\textbackslash{}alpha, \textbackslash{}beta) }\OtherTok{=}\NormalTok{ \textbackslash{}sum\_\{i}\OtherTok{=}\DecValTok{1}\NormalTok{\}}\SpecialCharTok{\^{}}\NormalTok{\{n\} (y\_i }\SpecialCharTok{{-}}\NormalTok{ \textbackslash{}alpha }\SpecialCharTok{{-}}\NormalTok{ \textbackslash{}beta x\_i)}\SpecialCharTok{\^{}}\DecValTok{2}\SpecialCharTok{$}\ErrorTok{$}
\end{Highlighting}
\end{Shaded}

O modelo de regressão linear simples é dado por
\(y = \alpha + \beta x + \epsilon\).

A estimativa dos parâmetros minimiza a Soma dos Quadrados dos Resíduos:
\[S(\alpha, \beta) = \sum_{i=1}^{n} (y_i - \alpha - \beta x_i)^2\]

B. \textbf{Sintaxe Essencial (Símbolos e Estruturas)}

Comandos especiais iniciam com barra invertida
(\texttt{\textbackslash{}}). Use \emph{underscore} (\texttt{\_}) para
subscritos, acento circunflexo (\texttt{\^{}}) para expoentes e chaves
\texttt{\{\}} para agrupar termos.

\begin{Shaded}
\begin{Highlighting}[]
\SpecialCharTok{$}\ErrorTok{$}\NormalTok{\textbackslash{}bar\{x\} }\OtherTok{=}\NormalTok{ \textbackslash{}frac\{}\DecValTok{1}\NormalTok{\}\{n\} \textbackslash{}sum\_\{i}\OtherTok{=}\DecValTok{1}\NormalTok{\}}\SpecialCharTok{\^{}}\NormalTok{\{n\} x\_i}\SpecialCharTok{$}\ErrorTok{$}

\ErrorTok{$$}\NormalTok{\textbackslash{}sigma }\OtherTok{=}\NormalTok{ \textbackslash{}sqrt\{\textbackslash{}frac\{\textbackslash{}}\FunctionTok{sum}\NormalTok{(x }\SpecialCharTok{{-}}\NormalTok{ \textbackslash{}mu)}\SpecialCharTok{\^{}}\DecValTok{2}\NormalTok{\}\{N\}\}}\SpecialCharTok{$}\ErrorTok{$}
\end{Highlighting}
\end{Shaded}

\[\bar{x} = \frac{1}{n} \sum_{i=1}^{n} x_i\]

\[\sigma = \sqrt{\frac{\sum(x - \mu)^2}{N}}\]

C. \textbf{Extração Automática de Modelos (\texttt{equatiomatic})}

Assim como o \texttt{gtsummary} automatiza tabelas, o pacote
\texttt{equatiomatic} extrai a equação matemática formatada diretamente
de um modelo estatístico ajustado no \texttt{R}.

\begin{Shaded}
\begin{Highlighting}[]
\NormalTok{pacman}\SpecialCharTok{::}\FunctionTok{p\_load}\NormalTok{(equatiomatic)}

\NormalTok{modelo }\OtherTok{\textless{}{-}} \FunctionTok{lm}\NormalTok{(mpg }\SpecialCharTok{\textasciitilde{}}\NormalTok{ cyl }\SpecialCharTok{+}\NormalTok{ hp, }\AttributeTok{data =}\NormalTok{ mtcars)}

\FunctionTok{extract\_eq}\NormalTok{(modelo, }\AttributeTok{use\_coefs =} \ConstantTok{TRUE}\NormalTok{, }\AttributeTok{wrap =} \ConstantTok{TRUE}\NormalTok{)}
\end{Highlighting}
\end{Shaded}

\begin{equation}
\begin{aligned}
\operatorname{\widehat{mpg}} &= 36.91 - 2.26(\operatorname{cyl}) - 0.02(\operatorname{hp})
\end{aligned}
\end{equation}

D. \textbf{Equações com Valores Dinâmicos (Inline R)}

Combine a tipografia \texttt{LaTeX} com o poder do \texttt{R} inserindo
códigos executáveis diretamente na fórmula via \emph{r código}. Garante
que os números na equação sejam atualizados se os dados mudarem.

\begin{Shaded}
\begin{Highlighting}[]
\NormalTok{media }\OtherTok{\textless{}{-}} \FunctionTok{mean}\NormalTok{(mtcars}\SpecialCharTok{$}\NormalTok{mpg)}
\NormalTok{desvio }\OtherTok{\textless{}{-}} \FunctionTok{sd}\NormalTok{(mtcars}\SpecialCharTok{$}\NormalTok{mpg)}
\end{Highlighting}
\end{Shaded}

\begin{Shaded}
\begin{Highlighting}[]
\NormalTok{A distribuição segue uma Normal}\SpecialCharTok{:} \ErrorTok{$}\NormalTok{X \textbackslash{}sim }\FunctionTok{N}\NormalTok{(\textbackslash{}}\AttributeTok{mu =} \FloatTok{20.1}\NormalTok{, \textbackslash{}sigma}\SpecialCharTok{\^{}}\DecValTok{2} \OtherTok{=} \DecValTok{6}\NormalTok{)}\SpecialCharTok{$}\NormalTok{.}
\end{Highlighting}
\end{Shaded}

A distribuição segue uma Normal: \(X \sim N(\mu = 20.1, \sigma^2 = 6)\).

\begin{tcolorbox}[enhanced jigsaw, left=2mm, toptitle=1mm, colback=white, colframe=quarto-callout-tip-color-frame, colbacktitle=quarto-callout-tip-color!10!white, opacityback=0, rightrule=.15mm, bottomtitle=1mm, arc=.35mm, title=\textcolor{quarto-callout-tip-color}{\faLightbulb}\hspace{0.5em}{Cheatsheets (Guias de Consulta Rápida)}, titlerule=0mm, bottomrule=.15mm, leftrule=.75mm, coltitle=black, toprule=.15mm, breakable, opacitybacktitle=0.6]

Decorar todos os comandos e funcionalidades dos pacotes no ecossistema
\texttt{R/RStudio} é uma tarefa desafiadora. Por isso, a comunidade
desenvolve os chamados \emph{cheatsheets} (ou folhas de consulta):
resumos concisos que reúnem informações essenciais, fórmulas e comandos
sobre temas específicos. Eles são projetados para facilitar a consulta
rápida, economizar tempo e auxiliar na memorização durante o fluxo de
trabalho.

Um dos repositórios mais completos, que inclui guias para pacotes de
análise espacial (como o \texttt{sf}), pode ser acessado neste
\href{https://github.com/rstudio/cheatsheets/}{link}.

Veja também a formatação de tabelas no
\href{https://modelsummary.com/vignettes/appearance.html}{link}

\end{tcolorbox}

\part{Fundamentos da Estatística Espacial}

\chapter{Fundamentos da Estatística Espacial}\label{sec-fundamentos}

A estatística é definida pela Associação Americana de Estatística (ASA)
como a ciência de aprender com dados, e de medir, controlar e comunicar
a incerteza (Wild, Utts, e Horton 2017). Ela atua como uma
metadisciplina focada em como pensar sobre a transformação de dados em
conhecimento do mundo real, sendo também descrita como o estudo da
variabilidade e da tomada de decisão diante da incerteza (Bartholomew
1995; Fienberg 2014).

A incerteza mencionada está intrinsecamente associada a uma
característica da população, a qual é representada por um parâmetro,
denotado por \(\theta\in \Theta\). Para quantificar a incerteza
associada a \(\theta\), destacam-se duas grandes abordagens:

\begin{enumerate}
\def\labelenumi{\arabic{enumi}.}
\item
  \textbf{Frequentista:} Onde \(\theta\) é um valor fixo e desconhecido
  da natureza. A probabilidade é interpretada como a frequência relativa
  de um evento em um longo prazo (repetição do experimento).
\item
  \textbf{Bayesiana:} Onde \(\theta\) é tratado como uma variável
  aleatória. Atribui-se a ele uma crença inicial, formalizada por uma
  distribuição de probabilidade a priori, que é atualizada pelos dados
  observados para gerar uma distribuição a posteriori.
\end{enumerate}

Independentemente da abordagem (Frequentista ou Bayesiana), se o
processo de modelagem não considera explicitamente a estrutura espacial
ou as coordenadas geográficas dos eventos em estudo, estamos diante da
estatística clássica. Por outro lado, quando a localização geográfica é
incorporada ao processo de modelagem, entramos no domínio espacial:
trata-se de estatística espacial quando há quantificação da incerteza, e
de análise espacial na ausência dessa quantificação.

A estatística espacial representa uma mudança paradigmática. Enquanto a
estatística clássica investiga o quê?, quanto?, quando? e como?, a
estatística espacial impõe a pergunta fundamental: onde? e,
crucialmente, a localização geográfica condiciona o valor observado?

Nesta seção, rompemos com a suposição de independência entre observações
que é estabelecida pela estatística clássica e estabelecemos a estrutura
teórica para modelar fenômenos onde o espaço geográfico influencia nos
valores que observamos.

\begin{tcolorbox}[enhanced jigsaw, left=2mm, toptitle=1mm, colback=white, colframe=quarto-callout-note-color-frame, colbacktitle=quarto-callout-note-color!10!white, opacityback=0, rightrule=.15mm, bottomtitle=1mm, arc=.35mm, title=\textcolor{quarto-callout-note-color}{\faInfo}\hspace{0.5em}{O Conceito de Espaço}, titlerule=0mm, bottomrule=.15mm, leftrule=.75mm, coltitle=black, toprule=.15mm, breakable, opacitybacktitle=0.6]

A palavra espaço é polissêmica
(\href{https://pt.wikipedia.org/wiki/Espa\%C3\%A7o_amostral}{espaço
amostral} \(\Omega\), espaço paramétrico \(\Theta\),
\href{https://pt.wikipedia.org/wiki/Espa\%C3\%A7o_sideral}{espaço
sideral}). No contexto deste material, referimo-nos ao \textbf{Espaço
Geográfico}
(\href{https://pt.wikipedia.org/wiki/Espa\%C3\%A7o_euclidiano}{Euclidiano}
ou \href{https://pt.wikipedia.org/wiki/Geod\%C3\%A9sica}{Geodésico}),
onde objetos possuem coordenadas de posição, orientação e mantêm
relações de distância e topologia entre si.

\end{tcolorbox}

\section{Estatística Clássica vs.~Estatística
Espacial}\label{sec-class_spac}

Para compreender a distinção entre a estatística clássica e a espacial,
devemos primeiro definir formalmente o que é um modelo estatístico e, em
seguida, observar como a estrutura de dependência altera esse modelo.

Um modelo estatístico é definido como um par ordenado
\((\Omega, \mathcal{P})\), em que \(\Omega\) é o espaço amostral,
representando o conjunto de todos os possíveis valores observáveis dos
dados \(\{Y_i\}_{i=1}^{n}\), e
\(\mathcal{P} = \{P_\theta : \theta \in \Theta\}\) é uma família de
distribuições de probabilidade indexada por um parâmetro \(\theta\),
sendo \(\Theta\) o espaço paramétrico que contém todos os valores
possíveis para o parâmetro desconhecido \(\theta\). Seja
\(\mathbf{Y} = (Y_1, Y_2, \dots, Y_n)^\top\) um vetor aleatório de
dimensão \(n \times 1\) representando nossos dados observados; a
distinção central entre a abordagem clássica e a espacial reside na
estrutura da matriz de covariância deste vetor.

\begin{enumerate}
\def\labelenumi{\arabic{enumi}.}
\tightlist
\item
  \textbf{A Abordagem Clássica}
\end{enumerate}

Na estatística clássica, assume-se frequentemente que as observações
\(\{Y_i\}_{i=1}^{n}\) são independentes e identicamente distribuídas
(i.i.d.) Getis (1999). Isso implica que a informação contida em uma
observação \(Y_i\) não altera a probabilidade de ocorrência de outra
observação \(Y_j\) (para \(i \neq j\)). Assim, considerando inicialmente
a equação para uma única observação \(i\) (onde \(i = 1, \dots, n\)):

\[
Y_i = \beta_0 + \beta_1 x_{i1} + \dots + \beta_p x_{ip} + \varepsilon_i, \quad \varepsilon_i \overset{iid}{\sim} N(0, \sigma^2)
\]

Pela propriedade de independência, temos que
\(Cov(\varepsilon_i, \varepsilon_j) = E[\varepsilon_i \varepsilon_j] = 0\)
para todo \(i \neq j\), o que implica que a distribuição condicional de
\(Y_i\) é dada por
\(Y_i | \mathbf{x}_i \sim N(\mathbf{x}_i^\top \boldsymbol{\beta}, \sigma^2)\).
Ou seja, a informação contida em \(Y_i\) não altera a probabilidade de
ocorrência de \(Y_j\). Matricialmente, temos que:

\[
\mathbf{Y} = \mathbf{X}\boldsymbol{\beta} + \boldsymbol{\varepsilon}
\]

Onde \(\mathbf{Y}\) é o vetor de variáveis resposta (observações) de
dimensão \(n \times 1\), \(\mathbf{X}\) é a matriz de planejamento (ou
matriz de \emph{design}) de dimensão \(n \times p\) contendo as
variáveis explicativas, \(\boldsymbol{\beta}\) é o vetor de parâmetros
desconhecidos (coeficientes de regressão) de dimensão \(p \times 1\), e
\(\boldsymbol{\varepsilon}\) é o vetor de erros aleatórios de dimensão
\(n \times 1\). Assume-se que os erros seguem uma distribuição Normal
multivariada, denotada por
\(\boldsymbol{\varepsilon} \sim N_n(\mathbf{0}, \Sigma)\), o que implica
que a distribuição de \(\mathbf{Y}\) condicionada a \(\mathbf{X}\) é
\(\mathbf{Y}| \mathbf{X} \sim N_n(\mathbf{X}\boldsymbol{\beta}, \Sigma)\).

Neste contexto clássico, a matriz de variância-covariância \(\Sigma\)
assume uma forma específica:

\begin{equation}\phantomsection\label{eq-cov_linear}{
Cov(\mathbf{Y}) = \Sigma = \sigma^2 \mathbf{I}_n = 
\begin{bmatrix} 
\sigma^2 & 0 & \dots & 0 \\
0 & \sigma^2 & \dots & 0 \\
\vdots & \vdots & \ddots & \vdots \\
0 & 0 & \dots & \sigma^2 
\end{bmatrix}
}\end{equation}

Em que \(\sigma^2\) representa a variância constante do erro
(homocedasticidade) e \(\mathbf{I}_n\) é a matriz identidade de ordem
\(n\), com valores 1 na diagonal principal e 0 fora dela.
Consequentemente, temos que a covariância entre quaisquer duas
observações distintas é nula, ou seja, \(Cov(Y_i, Y_j) = 0\) para todo
\(i \neq j\) (Figura~\ref{fig-espacial-vs-classica}). Isso é análogo ao
processo de retirar bolas de uma urna com reposição: a probabilidade de
retirar uma bola vermelha na segunda tentativa independe completamente
do resultado da primeira.

\begin{enumerate}
\def\labelenumi{\arabic{enumi}.}
\setcounter{enumi}{1}
\tightlist
\item
  \textbf{Abordagem Espacial}
\end{enumerate}

Na estatística espacial, a independência é considerada a exceção, não a
regra. Assume-se que a covariância entre observações em locais distintos
é não nula, ou seja, \(Cov(Y_i, Y_j) \neq 0\), e que essa covariância é
uma função explícita da estrutura espacial \(S\) (Getis 1999). Para
formalizar essa relação, denotamos o processo estocástico por
\(\{Y(\mathbf{s}) : \mathbf{s} \in D\}\) conforme definido por Noel
Cressie e Moores (2022), onde \(\mathbf{s} \in D \subset \mathbb{R}^d\)
representa o vetor de coordenadas de localização no domínio espacial de
interesse. Consequentemente, a matriz de covariância \(\Sigma\) defnida
na Eq.~\ref{eq-cov_linear} deixa de ser diagonal e torna-se uma matriz
densa, capturando as interações entre todas as localidades:

\begin{equation}\phantomsection\label{eq-cov_espacial}{Cov(\mathbf{Y}) = \Sigma = 
\begin{bmatrix} 
C(\mathbf{s}_1, \mathbf{s}_1) & C(\mathbf{s}_1, \mathbf{s}_2) & \dots & C(\mathbf{s}_1, \mathbf{s}_n) \\
C(\mathbf{s}_2, \mathbf{s}_1) & C(\mathbf{s}_2, \mathbf{s}_2) & \dots & C(\mathbf{s}_2, \mathbf{s}_n) \\
\vdots & \vdots & \ddots & \vdots \\
C(\mathbf{s}_n, \mathbf{s}_1) & C(\mathbf{s}_n, \mathbf{s}_2) & \dots & C(\mathbf{s}_n, \mathbf{s}_n) 
\end{bmatrix}}\end{equation}

A função \(C(\mathbf{s}_i, \mathbf{s}_j)\) na Eq.~\ref{eq-cov_espacial}
define a estrutura de dependência e pode ser modelada de diferentes
formas, dependendo da natureza dos dados. No contexto da geoestatística
(Capítulo~\ref{sec-geoest}), assume-se frequentemente a
\href{https://pt.wikipedia.org/wiki/Fun\%C3\%A7\%C3\%A3o_de_covari\%C3\%A2ncia}{estacionariedade}
(Seção~\ref{sec-estacionaridade}), onde a covariância depende apenas da
distância euclidiana \(h = ||\mathbf{s}_i - \mathbf{s}_j||\) entre os
pontos, tal que \(Cov(Y(\mathbf{s}_i), Y(\mathbf{s}_j)) = C(h)\). Aqui,
\(C(h)\) é uma função de distância que decai com a distância, indicando
que a correlação entre observações diminui à medida que a distância
\(h\) entre elas aumenta Figura~\ref{fig-espacial-vs-classica}.
Alternativamente, para dados de área/lattice
(Capítulo~\ref{sec-dados_area}), a dependência é modelada através da
estrutura de vizinhança ou contiguidade. Introduz-se uma matriz de pesos
espaciais \(\mathbf{W}\), onde o elemento \(w_{ij}\) é definido
binariamente (ou por distâncias inversas) para indicar a conexão entre
unidades espaciais: \(w_{ij} = 1\) se as áreas \(\mathbf{s}_i\) e
\(\mathbf{s}_j\) compartilham fronteira (são vizinhos) e \(w_{ij} = 0\)
caso contrário. Neste cenário, modelos autorregressivos (como o CAR -
Conditional Autoregressive) assumem que o valor esperado ou a variância
de \(Y_i\) é condicionado aos seus vizinhos, expresso formalmente como
\(Y_i | \mathbf{Y}_{-i} \sim N(\sum_{j \in \text{vizinhos}} w_{ij} Y_j, \tau^2)\),
onde \(\mathbf{Y}_{-i}\) representa todas as observações exceto a
\(i\)-ésima e \(\tau^2\) é a variância condicional.

\begin{tcolorbox}[enhanced jigsaw, left=2mm, toptitle=1mm, colback=white, colframe=quarto-callout-important-color-frame, colbacktitle=quarto-callout-important-color!10!white, opacityback=0, rightrule=.15mm, bottomtitle=1mm, arc=.35mm, title=\textcolor{quarto-callout-important-color}{\faExclamation}\hspace{0.5em}{Nem tudo é estatística espacial}, titlerule=0mm, bottomrule=.15mm, leftrule=.75mm, coltitle=black, toprule=.15mm, breakable, opacitybacktitle=0.6]

É fundamental distinguir dois termos frequentemente confundidos,
conforme elucidado por Noel Cressie e Moores (2022):

\begin{itemize}
\item
  \textbf{Análise Espacial:} Refere-se, de modo amplo, ao estudo da
  informação de localização associada a atributos. É um termo comum na
  Geografia e em Sistemas de Informação Geográfica (SIG). Caracteriza-se
  pelo uso de algoritmos determinísticos para manipulação e consulta de
  dados espaciais, sem a quantificação explícita da incerteza via
  modelos probabilísticos.

  \begin{itemize}
  \tightlist
  \item
    \emph{Exemplos:} Mapas de áreas, zonas de influência, caminho
    mínimo, Interpolação IDW, sobreposição de camadas.
  \end{itemize}
\item
  \textbf{Estatística Espacial:} Ocorre quando a análise incorpora
  formalmente a quantificação da incerteza. Baseia-se na premissa de que
  valores próximos são estatisticamente mais dependentes do que os
  distantes. Ela utiliza as localizações espaciais (\(s\)) para modelar
  essa dependência através de efeitos fixos (tendências) e aleatórios
  (covariância) dentro de um modelo de probabilidade estocástico.

  \begin{itemize}
  \tightlist
  \item
    \emph{Exemplos:} Krigagem (geoestatística), Testes de Autocorrelação
    (Moran/Geary), Regressão Espacial (SAR/CAR), Modelagem de Processos
    Pontuais (Ripley's K).
  \end{itemize}
\end{itemize}

\end{tcolorbox}

\textbf{A Primeira Lei da Geografia}

\begin{quote}
\emph{``Everything is related to everything else, but near things are
more related than distant things.''} (Todas as coisas estão relacionadas
entre si, mas coisas próximas estão mais relacionadas do que coisas
distantes.) --- Waldo Tobler (1970)
\end{quote}

Esta lei justifica a existência de métodos de interpolação (como
Krigagem) e modelos de regressão espacial. Se o mundo fosse puramente
aleatório (como na Figura~\ref{fig-espacial-vs-classica} lado esquerdo),
a geografia seria irrelevante e a melhor previsão para um local não
amostrado seria a média global, e não a média dos vizinhos.

\begin{tcolorbox}[enhanced jigsaw, left=2mm, toptitle=1mm, colback=white, colframe=quarto-callout-tip-color-frame, colbacktitle=quarto-callout-tip-color!10!white, opacityback=0, rightrule=.15mm, bottomtitle=1mm, arc=.35mm, title=\textcolor{quarto-callout-tip-color}{\faLightbulb}\hspace{0.5em}{Decaimento da Distância}, titlerule=0mm, bottomrule=.15mm, leftrule=.75mm, coltitle=black, toprule=.15mm, breakable, opacitybacktitle=0.6]

A influência de A sobre B tende a diminuir conforme a distância entre
eles aumenta. A forma como essa influência cai (linearmente,
exponencialmente, etc.) é o que modelamos estatísticamente.

\end{tcolorbox}

\section{Autocorrelação, dependência e vizinhança
espacial}\label{sec-dependencia}

\textbf{Dependência espacial}

A Dependência Espacial é a propriedade estatística fundamental que
descreve a tendência de que os valores de uma variável em uma
determinada localização geográfica estejam funcionalmente relacionados
aos valores dessa mesma variável em localizações vizinhas
(Figura~\ref{fig-espacial-vs-classica} direita) (Crawford 2009) . Ela
representa a manifestação estatística da Primeira Lei da Geografia de
Tobler.

A existência de dependência espacial constitui uma violação direta da
suposição de independência estatística (i.i.d.)
(Figura~\ref{fig-espacial-vs-classica} esquerda), uma premissa basilar
de muitos métodos convencionais (como a regressão linear clássica/OLS).
Segundo Chun e Griffith (2017), a dependência refere-se à existência de
uma covariância não nula entre valores de uma única variável quando
inspecionados em conjunto com suas localizações espaciais, indicando que
o evento observado em um ponto \(s_i\) é condicionado pelo seu
entorno/vizinhos mais próximos \(s_j\).

Para ilustrar a ruptura com a estatística clássica, Unwin e Hepple
(1974) oferecem uma analogia: enquanto a estatística tradicional trata
as observações como ``bolas em uma urna'' (onde a posição física da bola
na urna é irrelevante para sua probabilidade de ser sorteada), a
estatística espacial trata os dados como ``cachos de uvas''. Em um
cacho, a posição de uma uva fornece informações cruciais sobre as uvas
adjacentes (maturação, tamanho, exposição ao sol). Portanto, na análise
espacial, a ordem e a proximidade carregam informação estatística que
não pode ser descartada.

\begin{Shaded}
\begin{Highlighting}[]
\ControlFlowTok{if}\NormalTok{ (}\SpecialCharTok{!}\FunctionTok{require}\NormalTok{(}\StringTok{"pacman"}\NormalTok{)) }\FunctionTok{install.packages}\NormalTok{(}\StringTok{"pacman"}\NormalTok{)}
\NormalTok{pacman}\SpecialCharTok{::}\FunctionTok{p\_load}\NormalTok{(ggplot2, dplyr, viridis, gstat, sf, tibble)}

\CommentTok{\#}
\NormalTok{grid\_df }\OtherTok{\textless{}{-}} \FunctionTok{expand.grid}\NormalTok{(}\AttributeTok{x =} \DecValTok{1}\SpecialCharTok{:}\DecValTok{50}\NormalTok{, }\AttributeTok{y =} \DecValTok{1}\SpecialCharTok{:}\DecValTok{50}\NormalTok{)}
\FunctionTok{set.seed}\NormalTok{(}\DecValTok{42}\NormalTok{)}
\NormalTok{grid\_df}\SpecialCharTok{$}\NormalTok{valor\_classico }\OtherTok{\textless{}{-}} \FunctionTok{rnorm}\NormalTok{(}\DecValTok{2500}\NormalTok{)}
\NormalTok{grid\_sf }\OtherTok{\textless{}{-}} \FunctionTok{st\_as\_sf}\NormalTok{(grid\_df, }\AttributeTok{coords =} \FunctionTok{c}\NormalTok{(}\StringTok{"x"}\NormalTok{, }\StringTok{"y"}\NormalTok{))}
\NormalTok{modelo\_vgm }\OtherTok{\textless{}{-}}\NormalTok{ gstat}\SpecialCharTok{::}\FunctionTok{vgm}\NormalTok{(}\AttributeTok{psill =} \DecValTok{1}\NormalTok{, }\AttributeTok{model =} \StringTok{"Sph"}\NormalTok{, }\AttributeTok{range =} \DecValTok{15}\NormalTok{, }\AttributeTok{nugget =} \FloatTok{0.1}\NormalTok{)}
\NormalTok{g\_dummy }\OtherTok{\textless{}{-}}\NormalTok{ gstat}\SpecialCharTok{::}\FunctionTok{gstat}\NormalTok{(}\AttributeTok{formula =}\NormalTok{ z}\SpecialCharTok{\textasciitilde{}}\DecValTok{1}\NormalTok{, }\AttributeTok{locations =}\NormalTok{ grid\_sf, }\AttributeTok{dummy =}\NormalTok{ T, }\AttributeTok{beta =} \DecValTok{0}\NormalTok{, }\AttributeTok{model =}\NormalTok{ modelo\_vgm, }\AttributeTok{nmax =} \DecValTok{20}\NormalTok{)}

\CommentTok{\# Predição}
\FunctionTok{invisible}\NormalTok{(}\FunctionTok{capture.output}\NormalTok{(yy }\OtherTok{\textless{}{-}} \FunctionTok{predict}\NormalTok{(g\_dummy, }\AttributeTok{newdata =}\NormalTok{ grid\_sf, }\AttributeTok{nsim =} \DecValTok{1}\NormalTok{)))}
\NormalTok{grid\_df}\SpecialCharTok{$}\NormalTok{valor\_espacial }\OtherTok{\textless{}{-}}\NormalTok{ yy}\SpecialCharTok{$}\NormalTok{sim1}

\CommentTok{\#}
\NormalTok{minha\_legenda }\OtherTok{\textless{}{-}} \FunctionTok{guide\_colorbar}\NormalTok{(}
  \AttributeTok{title =} \ConstantTok{NULL}\NormalTok{,}
  \AttributeTok{barwidth =} \FunctionTok{unit}\NormalTok{(.}\DecValTok{5}\NormalTok{, }\StringTok{"npc"}\NormalTok{),}
  \AttributeTok{barheight =} \FunctionTok{unit}\NormalTok{(}\FloatTok{0.5}\NormalTok{, }\StringTok{"cm"}\NormalTok{),}
  \AttributeTok{label.position =} \StringTok{"bottom"}
\NormalTok{)}

\CommentTok{\# Clássico}
\FunctionTok{ggplot}\NormalTok{(grid\_df, }\FunctionTok{aes}\NormalTok{(x, y, }\AttributeTok{fill =}\NormalTok{ valor\_classico)) }\SpecialCharTok{+}
  \FunctionTok{geom\_tile}\NormalTok{() }\SpecialCharTok{+}
  \FunctionTok{scale\_fill\_viridis\_c}\NormalTok{(}\AttributeTok{option =} \StringTok{"B"}\NormalTok{) }\SpecialCharTok{+}
  \FunctionTok{coord\_fixed}\NormalTok{() }\SpecialCharTok{+}
  \FunctionTok{theme\_void}\NormalTok{() }\SpecialCharTok{+}
  \FunctionTok{theme}\NormalTok{(}\AttributeTok{legend.position =} \StringTok{"bottom"}\NormalTok{,}
        \AttributeTok{legend.margin =} \FunctionTok{margin}\NormalTok{(}\AttributeTok{t =} \DecValTok{5}\NormalTok{, }\AttributeTok{r =} \DecValTok{0}\NormalTok{, }\AttributeTok{b =} \DecValTok{0}\NormalTok{, }\AttributeTok{l =} \DecValTok{0}\NormalTok{)) }\SpecialCharTok{+}
  \FunctionTok{guides}\NormalTok{(}\AttributeTok{fill =}\NormalTok{ minha\_legenda)}

\CommentTok{\# Espacial}
\FunctionTok{ggplot}\NormalTok{(grid\_df, }\FunctionTok{aes}\NormalTok{(x, y, }\AttributeTok{fill =}\NormalTok{ valor\_espacial)) }\SpecialCharTok{+}
  \FunctionTok{geom\_tile}\NormalTok{() }\SpecialCharTok{+}
  \FunctionTok{scale\_fill\_viridis\_c}\NormalTok{(}\AttributeTok{option =} \StringTok{"B"}\NormalTok{) }\SpecialCharTok{+}
  \FunctionTok{coord\_fixed}\NormalTok{() }\SpecialCharTok{+}
  \FunctionTok{theme\_void}\NormalTok{() }\SpecialCharTok{+}
  \FunctionTok{theme}\NormalTok{(}\AttributeTok{legend.position =} \StringTok{"bottom"}\NormalTok{,}
        \AttributeTok{legend.margin =} \FunctionTok{margin}\NormalTok{(}\AttributeTok{t =} \DecValTok{5}\NormalTok{, }\AttributeTok{r =} \DecValTok{0}\NormalTok{, }\AttributeTok{b =} \DecValTok{0}\NormalTok{, }\AttributeTok{l =} \DecValTok{0}\NormalTok{)) }\SpecialCharTok{+}
  \FunctionTok{guides}\NormalTok{(}\AttributeTok{fill =}\NormalTok{ minha\_legenda)}
\end{Highlighting}
\end{Shaded}

\begin{figure}

\begin{minipage}{0.50\linewidth}

\centering{

\pandocbounded{\includegraphics[keepaspectratio]{fundEstspatial_files/figure-pdf/fig-espacial-vs-classica-1.pdf}}

}

\subcaption{\label{fig-espacial-vs-classica-1}Estatística Clássica: Sem
dependência espacial (i.i.d.)}

\end{minipage}%
%
\begin{minipage}{0.50\linewidth}

\centering{

\pandocbounded{\includegraphics[keepaspectratio]{fundEstspatial_files/figure-pdf/fig-espacial-vs-classica-2.pdf}}

}

\subcaption{\label{fig-espacial-vs-classica-2}Estatística Espacial:
Existe dependência espacial}

\end{minipage}%

\caption{\label{fig-espacial-vs-classica}Comparação Visual entre
Processos Estocásticos}

\end{figure}%

\textbf{Autocorrelação espacial}

Enquanto a dependência espacial constitui a propriedade teórica
intrínseca ao processo gerador dos dados, a autocorrelação espacial é a
medida estatística utilizada para quantificá-la. O termo distingue-se
fundamentalmente da correlação convencional, como a de Pearson, que
avalia a associação linear entre duas variáveis distintas (\(X\) e
\(Y\)). A autocorrelação espacial, por sua vez, examina a correlação de
uma única variável consigo mesma, porém deslocada no espaço geográfico,
confrontando o valor da variável no local \(s_i\) com os valores
observados na sua vizinhança \(s_j\). Segundo Chun e Griffith (2017),
essa métrica quantifica simultaneamente a força e a direção da relação
espacial, servindo como um diagnóstico crucial para a validade das
análises subsequentes. A sua identificação não é apenas descritiva, mas
um pré-requisito metodológico, uma vez que Getis (1999) alerta que
ignorar essa estrutura e aplicar métodos clássicos, como os Mínimos
Quadrados Ordinários (MQO), viola o pressuposto de independência dos
erros, resultando em estimativas de variância enviesadas e testes de
hipótese inválidos.

A autocorrelação espacial manifesta-se em três padrões estruturais
distintos, visualizados na Figura~\ref{fig-moran-types}. A configuração
mais frequente em fenômenos naturais e sociais é a autocorrelação
espacial positiva, que ocorre quando valores similares tendem a se
agrupar no espaço, formando \emph{clusters}. Neste cenário, observa-se
que locais com valores altos são vizinhos de outros valores altos, e
locais com valores baixos são vizinhos de outros valores baixos,
indicando processos de continuidade ou contágio, comuns em variáveis
como temperatura, altitude ou preços imobiliários. Em contraste, a
autocorrelação espacial negativa caracteriza-se pela adjacência de
valores dissimilares, onde um local com valor alto tende a ser cercado
por vizinhos com valores baixos, e vice-versa. Este padrão, visualmente
semelhante a um tabuleiro de xadrez, é menos frequente na natureza e
geralmente sinaliza processos de competição ou inibição espacial, como a
localização de estabelecimentos comerciais concorrentes. Por fim, a
ausência de autocorrelação denota uma distribuição puramente estocástica
(completa aleatoriedade espacial) dos valores no espaço, onde o valor
observado em um ponto não fornece informação estatística sobre seus
vizinhos, representando a independência espacial e constituindo a
hipótese nula (\(H_0\)) na maioria dos testes estatísticos espaciais.

\begin{Shaded}
\begin{Highlighting}[]
\ControlFlowTok{if}\NormalTok{ (}\SpecialCharTok{!}\FunctionTok{require}\NormalTok{(}\StringTok{"pacman"}\NormalTok{)) }\FunctionTok{install.packages}\NormalTok{(}\StringTok{"pacman"}\NormalTok{)}
\NormalTok{pacman}\SpecialCharTok{::}\FunctionTok{p\_load}\NormalTok{(ggplot2, patchwork, viridis)}

\NormalTok{df\_grid }\OtherTok{\textless{}{-}} \FunctionTok{expand.grid}\NormalTok{(}\AttributeTok{x =} \DecValTok{1}\SpecialCharTok{:}\DecValTok{8}\NormalTok{, }\AttributeTok{y =} \DecValTok{1}\SpecialCharTok{:}\DecValTok{8}\NormalTok{)}

\CommentTok{\#Simular Padrão Positivo}
\NormalTok{df\_grid}\SpecialCharTok{$}\NormalTok{z\_pos }\OtherTok{\textless{}{-}}\NormalTok{ (df\_grid}\SpecialCharTok{$}\NormalTok{x }\SpecialCharTok{+}\NormalTok{ df\_grid}\SpecialCharTok{$}\NormalTok{y) }\SpecialCharTok{+} \FunctionTok{rnorm}\NormalTok{(}\DecValTok{64}\NormalTok{, }\DecValTok{0}\NormalTok{, }\FloatTok{0.5}\NormalTok{)}

\CommentTok{\# Simular Padrão Negativo }
\NormalTok{df\_grid}\SpecialCharTok{$}\NormalTok{z\_neg }\OtherTok{\textless{}{-}} \FunctionTok{ifelse}\NormalTok{((df\_grid}\SpecialCharTok{$}\NormalTok{x }\SpecialCharTok{+}\NormalTok{ df\_grid}\SpecialCharTok{$}\NormalTok{y) }\SpecialCharTok{\%\%} \DecValTok{2} \SpecialCharTok{==} \DecValTok{0}\NormalTok{, }\DecValTok{1}\NormalTok{, }\DecValTok{0}\NormalTok{)}

\CommentTok{\#Simular Padrão Aleatório}
\FunctionTok{set.seed}\NormalTok{(}\DecValTok{123}\NormalTok{)}
\NormalTok{df\_grid}\SpecialCharTok{$}\NormalTok{z\_rand }\OtherTok{\textless{}{-}} \FunctionTok{rnorm}\NormalTok{(}\DecValTok{64}\NormalTok{)}

\NormalTok{plot\_pattern }\OtherTok{\textless{}{-}} \ControlFlowTok{function}\NormalTok{(data, var, title) \{}
  \FunctionTok{ggplot}\NormalTok{(data, }\FunctionTok{aes}\NormalTok{(x, y, }\AttributeTok{fill =}\NormalTok{ \{\{var\}\})) }\SpecialCharTok{+}
    \FunctionTok{geom\_tile}\NormalTok{(}\AttributeTok{color =} \StringTok{"white"}\NormalTok{, }\AttributeTok{lwd =} \FloatTok{0.5}\NormalTok{) }\SpecialCharTok{+}
    \FunctionTok{scale\_fill\_viridis\_c}\NormalTok{(}\AttributeTok{option =} \StringTok{"mako"}\NormalTok{, }\AttributeTok{guide =} \StringTok{"none"}\NormalTok{) }\SpecialCharTok{+}
    \FunctionTok{theme\_void}\NormalTok{() }\SpecialCharTok{+}
    \FunctionTok{coord\_fixed}\NormalTok{() }\SpecialCharTok{+}
    \FunctionTok{labs}\NormalTok{(}\AttributeTok{title =}\NormalTok{ title) }\SpecialCharTok{+}
    \FunctionTok{theme}\NormalTok{(}\AttributeTok{plot.title =} \FunctionTok{element\_text}\NormalTok{(}\AttributeTok{hjust =} \FloatTok{0.5}\NormalTok{, }\AttributeTok{size =} \DecValTok{10}\NormalTok{))}
\NormalTok{\}}

\NormalTok{p1 }\OtherTok{\textless{}{-}} \FunctionTok{plot\_pattern}\NormalTok{(df\_grid, z\_pos, }\StringTok{"Positiva (Agrupamento)"}\NormalTok{)}
\NormalTok{p2 }\OtherTok{\textless{}{-}} \FunctionTok{plot\_pattern}\NormalTok{(df\_grid, z\_rand, }\StringTok{"Aleatória (Independência)"}\NormalTok{)}
\NormalTok{p3 }\OtherTok{\textless{}{-}} \FunctionTok{plot\_pattern}\NormalTok{(df\_grid, z\_neg, }\StringTok{"Negativa (Competição)"}\NormalTok{)}

\CommentTok{\# Combinar com patchwork}
\NormalTok{p1 }\SpecialCharTok{+}\NormalTok{ p2 }\SpecialCharTok{+}\NormalTok{ p3}
\end{Highlighting}
\end{Shaded}

\begin{figure}[H]

\centering{

\pandocbounded{\includegraphics[keepaspectratio]{fundEstspatial_files/figure-pdf/fig-moran-types-1.pdf}}

}

\caption{\label{fig-moran-types}Tipologia da Autocorrelação Espacial.}

\end{figure}%

\textbf{Vizinhança Espacial}

Para operacionalizar a mensuração da dependência e o cálculo da
autocorrelação, torna-se imperativo definir formalmente a estrutura de
conectividade entre as unidades espaciais, estabelecendo o conceito de
vizinhança. Conforme definem Chun e Griffith (2017), a quantificação da
dependência exige a identificação precisa de um conjunto de valores
vizinhos que covariam com a observação de interesse, sendo essa relação
estruturada algebricamente através de uma Matriz de Pesos Espaciais
(\(\mathbf{W}\)). Tradicionalmente, essa matriz é construída sob
critérios de contiguidade física, utilizando analogias do xadrez para
dados de área: o critério Torre (\emph{Rook}) estipula que as unidades
são vizinhas apenas se compartilharem uma fronteira física
(\href{https://pt.wikipedia.org/wiki/Aresta}{arestas}), enquanto o
critério Rainha (\emph{Queen}) considera vizinhas as unidades que
compartilham qualquer ponto (arestas ou
\href{https://pt.wikipedia.org/wiki/V\%C3\%A9rtice}{vértices}).
Alternativamente, especialmente em geoestatística, a vizinhança é
definida por funções de distância, onde todas as unidades dentro de um
raio \(d\) são consideradas conectadas, ou onde a magnitude da
influência decai inversamente à distância euclidiana entre os centróides
Figura~\ref{fig-vizinhanca}.

Entretanto, a definição de ``espaço'' na modelagem contemporânea
expandiu-se para além da geografia física. Econometristas e teóricos
regionais, como H. Kelejian e Piras (2017), argumentam que a distância
não deve se limitar à métrica euclidiana, mas sim representar o
enfraquecimento das conexões entre unidades observacionais em múltiplas
dimensões. Nesta perspectiva, a matriz \(\mathbf{W}\) deve ser capaz de
capturar a ``distância econômica'' ou institucional, permitindo que a
vizinhança seja definida por semelhanças em estruturas de mercado,
alinhamento político ou hierarquia urbana. Sob essa ótica, metrópoles
fisicamente distantes (como São Paulo e Nova York) podem ser
consideradas vizinhas devido aos fluxos financeiros e competição
econômica direta, enquanto municípios contíguos de menor porte podem
apresentar uma interação estatística negligenciável.

A crítica à primazia exclusiva da proximidade física é aprofundada pela
geografia econômica evolucionária. Boschma (2005) defende que a
proximidade geográfica não é condição necessária nem suficiente para a
interação e o aprendizado, atuando, no máximo, como uma facilitadora
para outras formas de conexão. Para que ocorra o efetivo transbordamento
de conhecimento
(\href{https://en.wikipedia.org/wiki/Spillover_(experiment)}{spillovers}),
a ``vizinhança'' real deve ser compreendida através de outras quatro
dimensões de proximidade: a cognitiva, que envolve uma base de
conhecimento compartilhada; a organizacional, referente à capacidade de
controle e coordenação; a social, baseada em relações de confiança e
parentesco; e a institucional, que diz respeito a normas e legislações
comuns. Portanto, uma definição robusta de vizinhança em modelos
espaciais modernos deve reconhecer que a interação entre agentes é
frequentemente moldada pela afinidade socioeconômica e institucional,
transcendendo a simples adjacência física.

\begin{Shaded}
\begin{Highlighting}[]
\ControlFlowTok{if}\NormalTok{ (}\SpecialCharTok{!}\FunctionTok{require}\NormalTok{(}\StringTok{"pacman"}\NormalTok{)) }\FunctionTok{install.packages}\NormalTok{(}\StringTok{"pacman"}\NormalTok{)}
\NormalTok{pacman}\SpecialCharTok{::}\FunctionTok{p\_load}\NormalTok{(ggplot2, patchwork)}

\NormalTok{plot\_neighbors }\OtherTok{\textless{}{-}} \ControlFlowTok{function}\NormalTok{(type) \{}
  \CommentTok{\# Grid 3x3}
\NormalTok{  df }\OtherTok{\textless{}{-}} \FunctionTok{expand.grid}\NormalTok{(}\AttributeTok{x =} \DecValTok{1}\SpecialCharTok{:}\DecValTok{3}\NormalTok{, }\AttributeTok{y =} \DecValTok{1}\SpecialCharTok{:}\DecValTok{3}\NormalTok{)}
  \CommentTok{\# Definir o centro}
\NormalTok{  center }\OtherTok{\textless{}{-}}\NormalTok{ df}\SpecialCharTok{$}\NormalTok{x }\SpecialCharTok{==} \DecValTok{2} \SpecialCharTok{\&}\NormalTok{ df}\SpecialCharTok{$}\NormalTok{y }\SpecialCharTok{==} \DecValTok{2}
  \CommentTok{\# Definir vizinhos}
  \ControlFlowTok{if}\NormalTok{ (type }\SpecialCharTok{==} \StringTok{"Torre (Rook)"}\NormalTok{) \{}
\NormalTok{    neighbors }\OtherTok{\textless{}{-}}\NormalTok{ (}\FunctionTok{abs}\NormalTok{(df}\SpecialCharTok{$}\NormalTok{x }\SpecialCharTok{{-}} \DecValTok{2}\NormalTok{) }\SpecialCharTok{+} \FunctionTok{abs}\NormalTok{(df}\SpecialCharTok{$}\NormalTok{y }\SpecialCharTok{{-}} \DecValTok{2}\NormalTok{)) }\SpecialCharTok{==} \DecValTok{1}
\NormalTok{  \} }\ControlFlowTok{else} \ControlFlowTok{if}\NormalTok{ (type }\SpecialCharTok{==} \StringTok{"Rainha (Queen)"}\NormalTok{) \{}
\NormalTok{    neighbors }\OtherTok{\textless{}{-}}\NormalTok{ (}\FunctionTok{abs}\NormalTok{(df}\SpecialCharTok{$}\NormalTok{x }\SpecialCharTok{{-}} \DecValTok{2}\NormalTok{) }\SpecialCharTok{\textless{}=} \DecValTok{1} \SpecialCharTok{\&} \FunctionTok{abs}\NormalTok{(df}\SpecialCharTok{$}\NormalTok{y }\SpecialCharTok{{-}} \DecValTok{2}\NormalTok{) }\SpecialCharTok{\textless{}=} \DecValTok{1}\NormalTok{) }\SpecialCharTok{\&} \SpecialCharTok{!}\NormalTok{center}
\NormalTok{  \}}
  
\NormalTok{  df}\SpecialCharTok{$}\NormalTok{legenda }\OtherTok{\textless{}{-}} \StringTok{"Outros"}
\NormalTok{  df}\SpecialCharTok{$}\NormalTok{legenda[neighbors] }\OtherTok{\textless{}{-}} \StringTok{"Vizinho"}
  
\NormalTok{  df}\SpecialCharTok{$}\NormalTok{legenda[center] }\OtherTok{\textless{}{-}} \StringTok{"Local de interesse"}
\NormalTok{  df}\SpecialCharTok{$}\NormalTok{legenda }\OtherTok{\textless{}{-}} \FunctionTok{factor}\NormalTok{(df}\SpecialCharTok{$}\NormalTok{legenda, }\AttributeTok{levels =} \FunctionTok{c}\NormalTok{(}\StringTok{"Local de interesse"}\NormalTok{, }\StringTok{"Vizinho"}\NormalTok{, }\StringTok{"Outros"}\NormalTok{))}
  
  \CommentTok{\# Plot}
  \FunctionTok{ggplot}\NormalTok{(df, }\FunctionTok{aes}\NormalTok{(x, y, }\AttributeTok{fill =}\NormalTok{ legenda)) }\SpecialCharTok{+} 
    \FunctionTok{geom\_tile}\NormalTok{(}\AttributeTok{color =} \StringTok{"black"}\NormalTok{, }\AttributeTok{lwd =} \DecValTok{1}\NormalTok{) }\SpecialCharTok{+}
    \FunctionTok{scale\_fill\_manual}\NormalTok{(}\AttributeTok{values =} \FunctionTok{c}\NormalTok{(}\StringTok{"Local de interesse"} \OtherTok{=} \StringTok{"darkred"}\NormalTok{, }
                                 \StringTok{"Vizinho"} \OtherTok{=} \StringTok{"steelblue"}\NormalTok{, }
                                 \StringTok{"Outros"} \OtherTok{=} \StringTok{"white"}\NormalTok{)) }\SpecialCharTok{+}
    \FunctionTok{coord\_fixed}\NormalTok{() }\SpecialCharTok{+}
    \FunctionTok{theme\_void}\NormalTok{() }\SpecialCharTok{+}
    \FunctionTok{labs}\NormalTok{(}\AttributeTok{title =}\NormalTok{ type, }\AttributeTok{fill =} \StringTok{""}\NormalTok{) }\SpecialCharTok{+}
    \FunctionTok{theme}\NormalTok{(}\AttributeTok{legend.position =} \StringTok{"bottom"}\NormalTok{, }
          \AttributeTok{plot.title =} \FunctionTok{element\_text}\NormalTok{(}\AttributeTok{hjust =} \FloatTok{0.5}\NormalTok{))}
\NormalTok{\}}

\CommentTok{\#}
\NormalTok{p1 }\OtherTok{\textless{}{-}} \FunctionTok{plot\_neighbors}\NormalTok{(}\StringTok{"Torre (Rook)"}\NormalTok{)}
\NormalTok{p2 }\OtherTok{\textless{}{-}} \FunctionTok{plot\_neighbors}\NormalTok{(}\StringTok{"Rainha (Queen)"}\NormalTok{)}

\CommentTok{\#}
\NormalTok{p1 }\SpecialCharTok{+}\NormalTok{ p2}
\end{Highlighting}
\end{Shaded}

\begin{figure}[H]

\centering{

\pandocbounded{\includegraphics[keepaspectratio]{fundEstspatial_files/figure-pdf/fig-vizinhanca-1.pdf}}

}

\caption{\label{fig-vizinhanca}Critérios de Vizinhança por
Contiguidade.}

\end{figure}%

\textbf{Heterogeneidade espacial}

A Heterogeneidade Espacial é definida fundamentalmente como a
complexidade e a variabilidade de uma propriedade de um sistema no
espaço e no tempo Li e Reynolds (1994). No âmbito da estatística
espacial, ela transcende a simples variação dos dados brutos e
representa uma característica intrínseca do Processo Gerador de Dados
(\href{https://en.wikipedia.org/wiki/Data_generating_process}{Data
Generation Process}), o qual se mostra inconsistente ou instável ao
longo do domínio geográfico Peng e Inoue (2024). É crucial distinguir
este conceito da dependência espacial: enquanto a dependência foca na
força da conexão/interação ou autocorrelação entre vizinhos (o quanto se
influenciam), a heterogeneidade foca na variação da estrutura do
fenômeno (o como se relacionam), implicando que os parâmetros
estatísticos (como médias, variância, e coeficientes dos modelos, etc.)
não são constantes em toda a área de estudo
Figura~\ref{fig-heterogeneidade}.

Sob uma ótica estatística, a presença de heterogeneidade implica
frequentemente na violação da suposição de estacionariedade. Segundo
Wagner e Fortin (2005), ela deve ser compreendida como a variabilidade
espacialmente estruturada de uma propriedade de interesse, significando
que parâmetros estatísticos fundamentais como a média, a variância ou a
estrutura de covariância não são constantes em toda a área de estudo.
Isso distingue a heterogeneidade, frequentemente associada a
\href{https://pt.wikipedia.org/wiki/Exogeneidade}{fatores exógenos} e
não-estacionários, da autocorrelação espacial pura, que é comumente
associada a
\href{https://pt.wikipedia.org/wiki/Processo_end\%C3\%B3geno}{processos
endógenos estacionários}.

Essa variabilidade estrutural manifesta-se simultaneamente de duas
formas: como heterogeneidade contínua, onde as relações mudam
gradualmente através do espaço (em gradientes globais ou locais), ou
como heterogeneidade discreta, caracterizada por mudanças abruptas em
fronteiras administrativas ou zonas geográficas específicas Peng e Inoue
(2024). Para dados categóricos, Li e Reynolds (1994) operacionalizam
essa heterogeneidade como a complexidade na composição (número e
proporção de tipos de manchas) e na configuração (arranjo espacial e
forma). Na ecologia de paisagens, o conceito expande-se para descrever
variações vitais para a dinâmica populacional, influenciando diretamente
a persistência, extinção e coexistência de espécies, uma vez que a
localização espacial determina a abundância e distribuição dos
organismos.

A consequência prática da heterogeneidade é que o contexto local altera
as ``regras do jogo''. Na estatística clássica, assume-se um modelo
global \(y = \alpha + \beta x+\varepsilon, \: y|x \sim FE(.)\), onde
\(FE(.)\) é
\href{https://en.wikipedia.org/wiki/Exponential_family}{família
exponencial}, o coeficiente \(\beta\) (o efeito de \(x\) em \(y\)) é
fixo e universal para todo o banco de dados. Na presença de
heterogeneidade espacial, reconhecemos que este efeito é local.
Considere um modelo hedônico (sugestão de leitura: (Fávero 2003))
prevendo o preço de imóveis (\(y\)) com base na área construída (\(x\)):
em um bairro nobre, um metro quadrado adicional pode valorizar o imóvel
em R\$ 10.000 (um \(\beta\) alto), enquanto em uma área periférica sem
infraestrutura, o mesmo metro quadrado adicional pode agregar apenas R\$
1.000 (um \(\beta\) baixo). Devido a essa inconsistência
(não-estacionariedade do parâmetro), modelos globais tendem a produzir
resultados enviesados (Peng e Inoue 2024).

\begin{Shaded}
\begin{Highlighting}[]
\ControlFlowTok{if}\NormalTok{ (}\SpecialCharTok{!}\FunctionTok{require}\NormalTok{(}\StringTok{"pacman"}\NormalTok{)) }\FunctionTok{install.packages}\NormalTok{(}\StringTok{"pacman"}\NormalTok{)}
\NormalTok{pacman}\SpecialCharTok{::}\FunctionTok{p\_load}\NormalTok{(ggplot2, viridis, sf)}

\CommentTok{\#}
\NormalTok{df\_het }\OtherTok{\textless{}{-}} \FunctionTok{expand.grid}\NormalTok{(}\AttributeTok{x =} \DecValTok{1}\SpecialCharTok{:}\DecValTok{20}\NormalTok{, }\AttributeTok{y =} \DecValTok{1}\SpecialCharTok{:}\DecValTok{20}\NormalTok{)}

\CommentTok{\# Simular Heterogeneidade Contínua}
\NormalTok{df\_het}\SpecialCharTok{$}\NormalTok{beta\_real }\OtherTok{\textless{}{-}}\NormalTok{ (df\_het}\SpecialCharTok{$}\NormalTok{y }\SpecialCharTok{/} \DecValTok{20}\NormalTok{) }\SpecialCharTok{*} \DecValTok{2} \SpecialCharTok{+}\NormalTok{ (df\_het}\SpecialCharTok{$}\NormalTok{x }\SpecialCharTok{/} \DecValTok{20}\NormalTok{) }

\CommentTok{\# Simular Heterogeneidade Discreta (uma "zona" diferente no centro)}
\NormalTok{centro }\OtherTok{\textless{}{-}}\NormalTok{ (df\_het}\SpecialCharTok{$}\NormalTok{x }\SpecialCharTok{{-}} \DecValTok{10}\NormalTok{)}\SpecialCharTok{\^{}}\DecValTok{2} \SpecialCharTok{+}\NormalTok{ (df\_het}\SpecialCharTok{$}\NormalTok{y }\SpecialCharTok{{-}} \DecValTok{10}\NormalTok{)}\SpecialCharTok{\^{}}\DecValTok{2} \SpecialCharTok{\textless{}} \DecValTok{16}
\NormalTok{df\_het}\SpecialCharTok{$}\NormalTok{beta\_real[centro] }\OtherTok{\textless{}{-}}\NormalTok{ df\_het}\SpecialCharTok{$}\NormalTok{beta\_real[centro] }\SpecialCharTok{+} \DecValTok{3}

\FunctionTok{ggplot}\NormalTok{(df\_het, }\FunctionTok{aes}\NormalTok{(x, y, }\AttributeTok{fill =}\NormalTok{ beta\_real)) }\SpecialCharTok{+}
  \FunctionTok{geom\_tile}\NormalTok{() }\SpecialCharTok{+}
  \FunctionTok{scale\_fill\_viridis\_c}\NormalTok{(}\AttributeTok{option =} \StringTok{"turbo"}\NormalTok{, }\AttributeTok{name =} \FunctionTok{expression}\NormalTok{(}\FunctionTok{beta}\NormalTok{(s))) }\SpecialCharTok{+}
  \FunctionTok{coord\_fixed}\NormalTok{() }\SpecialCharTok{+}
  \FunctionTok{theme\_void}\NormalTok{() }\SpecialCharTok{+}
  \FunctionTok{labs}\NormalTok{(}\AttributeTok{title =} \StringTok{""}\NormalTok{,}
       \AttributeTok{subtitle =} \StringTok{""}\NormalTok{) }\SpecialCharTok{+}
  \FunctionTok{theme}\NormalTok{(}\AttributeTok{plot.title =} \FunctionTok{element\_text}\NormalTok{(}\AttributeTok{hjust =} \FloatTok{0.5}\NormalTok{))}
\end{Highlighting}
\end{Shaded}

\begin{figure}[H]

\centering{

\pandocbounded{\includegraphics[keepaspectratio]{fundEstspatial_files/figure-pdf/fig-heterogeneidade-1.pdf}}

}

\caption{\label{fig-heterogeneidade}Heterogeneidade Espacial: O efeito
de X em Y não é constante (\(\beta(s)\) muda gradualmente e abruptamente
no centro)}

\end{figure}%

\section{Estacionariedade e Não Estacionariedade
Espacial}\label{sec-estacionaridade}

\textbf{Estacionariedade}

A Estacionariedade é um conceito fundamental que sustenta a inferência
estatística em processos estocásticos espaciais. Na maioria das
investigações geocientíficas, deparamo-nos com um desafio intrínseco:
possuímos apenas uma única realização (uma única ``foto'') do processo
estocástico sob investigação. Não podemos replicar o processo gerador e
observar como o padrão de chuva ou a distribuição de minérios se
formaria novamente sob as mesmas condições probabilísticas. Portanto,
para calcularmos estatísticas vitais como a média e a variância, e
realizarmos previsões (inferência), precisamos assumir algum grau de
estabilidade ou repetição nas propriedades do fenômeno através do
espaço.

Intuitivamente, a estacionariedade espacial sugere que as propriedades
estatísticas (momentos da distribuição) do fenômeno são uniformes em
toda a região de estudo, permanecendo inalteradas sob translação da
origem do sistema de coordenadas (Schmidt e O'Hagan 2003). Em termos
práticos, isso significa que as ``leis'' que governam a variabilidade
dos dados não mudam de um local para outro, permitindo que utilizemos
dados de uma parte da região para estimar parâmetros válidos para outra
parte.

Para formalizar este conceito, conforme detalhado por Sahu (2022),
devemos decompor a estacionariedade em níveis hierárquicos baseados nos
momentos da distribuição (média, variância e covariância):

\begin{enumerate}
\def\labelenumi{\arabic{enumi}.}
\tightlist
\item
  \textbf{Estacionariedade de Primeira Ordem}
\end{enumerate}

Um processo estocástico espacial \(Y(\mathbf{s})\) é classificado como
estacionário de primeira ordem se o seu valor esperado (média) for
constante em todo o domínio de estudo \(D\):

\[E[Y(\mathbf{s})] = \mu, \quad \forall \mathbf{s} \in D\] Isso implica
que não existe uma tendência (\emph{trend}) global ou deriva sistemática
nos dados. Um mapa da média teórica desse processo seria
\href{https://pt.wikipedia.org/wiki/Monocromia}{``monocromático''} ou
plano, sem gradientes direcionais. É crucial distinguir a média do
processo (o parâmetro populacional \(\mu\), que é constante) da
realização observada (os valores \(y_i\), que variam). A
estacionariedade de primeira ordem garante que as flutuações observadas
ocorrem ao redor de um patamar fixo. Se a média do processo altera-se em
função da localização (por exemplo, a temperatura média diminuindo
sistematicamente conforme a latitude aumenta), dizemos que o processo é
não estacionário de primeira ordem. Matematicamente, isso é expresso
como \(E[Y(\mathbf{s})] = \mu(\mathbf{s})\), indicando que a média é uma
função determinística da posição \(\mathbf{s}\).

\begin{enumerate}
\def\labelenumi{\arabic{enumi}.}
\setcounter{enumi}{1}
\tightlist
\item
  \textbf{Estacionariedade de Segunda Ordem (Fraca)}
\end{enumerate}

Para a maioria das aplicações em geoestatística, como a Krigagem, a
estabilidade apenas da média é insuficiente; necessitamos também que a
estrutura de variabilidade e correlação seja estável. Um processo é dito
estacionário de segunda ordem (ou fracamente estacionário) se satisfaz
simultaneamente duas condições:

\begin{itemize}
\item
  Possui média constante (atende à primeira ordem):
  \(E[Y(\mathbf{s})] = \mu\).
\item
  A covariância entre dois pontos (\(\{s_i,\: s_j\}_{j\neq i}\))
  quaisquer depende exclusivamente do vetor de separação ou distância
  entre eles (\(\mathbf{h} = \mathbf{s_i} - \mathbf{s_j}\)), e não de
  suas localizações absolutas geográficas:
\end{itemize}

\[Cov[Y(\mathbf{s}), Y(\mathbf{s}+\mathbf{h})] = C(\mathbf{h})\]

Esta propriedade é crítica pois permite estimar uma função de
covariância global \(C(\mathbf{h})\) ou um variograma utilizando todos
os pares de pontos disponíveis na amostra, simplificando drasticamente a
modelagem ao reduzir o número de parâmetros necessários (Bandyopadhyay e
Rao 2017). Além disso, ela implica na estacionariedade da variância
(\(C(0) = \sigma^2\)), ou seja, a dispersão dos dados é constante em
todo o domínio (homocedasticidade espacial).

\begin{tcolorbox}[enhanced jigsaw, left=2mm, toptitle=1mm, colback=white, colframe=quarto-callout-note-color-frame, colbacktitle=quarto-callout-note-color!10!white, opacityback=0, rightrule=.15mm, bottomtitle=1mm, arc=.35mm, title=\textcolor{quarto-callout-note-color}{\faInfo}\hspace{0.5em}{Outros Graus de Estacionariedade}, titlerule=0mm, bottomrule=.15mm, leftrule=.75mm, coltitle=black, toprule=.15mm, breakable, opacitybacktitle=0.6]

\begin{itemize}
\tightlist
\item
  \textbf{Estacionariedade Estrita (Forte):} Uma condição mais
  restritiva onde toda a distribuição conjunta de probabilidade
  permanece inalterada sob qualquer deslocamento espacial. Formalmente,
  para qualquer conjunto finito de \(n\) localizações
  \(\{s_1, s_2, \dots, s_n\}\) e qualquer vetor de deslocamento \(h\), a
  distribuição conjunta de probabilidade deve satisfazer:
\end{itemize}

\[P(Y(s_1) \le y_1, \dots, Y(s_n) \le y_n) = P(Y(s_1+h) \le y_1, \dots, Y(s_n+h) \le y_n)\]

Ou, de forma simplificada, a igualdade em distribuição:
\[(Y(s_1), \dots, Y(s_n)) \stackrel{d}{=} (Y(s_1+h), \dots, Y(s_n+h))\]
Em Processos Gaussianos
(GP)\footnote{Um processo estocástico $\{Y(s), s \in D\}$ é definido como um Processo Gaussiano se, para qualquer conjunto finito de localizações $s_1, \dots, s_n$, o vetor aleatório conjunto segue uma distribuição Normal Multivariada:

$$(Y(s_1), \dots, Y(s_n))^\top \sim \mathcal{N}_n(\boldsymbol{\mu}, \boldsymbol{\Sigma})$$

onde $\boldsymbol{\mu}$ é o vetor de médias com $\mu_i = E[Y(s_i)]$ e $\boldsymbol{\Sigma}$ é a matriz de covariância com $\Sigma_{ij} = Cov(Y(s_i), Y(s_j))$.},
como a distribuição é totalmente caracterizada pela média e covariância,
a estacionariedade de segunda ordem implica automaticamente a estrita
(Schmidt e O'Hagan 2003). isto é, como a distribuição Normal
Multivariada (\(\mathcal{N}\)) depende exclusivamente dos dois primeiros
momentos (\(\mu\) e \(\Sigma\)), se esses momentos forem invariantes por
translação (estacionariedade de segunda ordem), a distribuição inteira
também será (estacionariedade estrita). Não há parâmetros de ordem
superior (como assimetria ou curtose) que possam variar.

\begin{itemize}
\tightlist
\item
  \textbf{Estacionariedade Intrínseca:} Frequentemente usada no cálculo
  do Variograma, é menos restritiva que a de segunda ordem. Exige apenas
  que a média das diferenças seja zero e que a variância das diferenças
  entre observações dependa apenas da distância:
  \(Var(Y(s +h) - Y(\mathbf{s})) = 2\gamma(\mathbf{h})\).
\end{itemize}

\end{tcolorbox}

\textbf{Não Estacionariedade Espacial}

Em contrapartida, a não estacionariedade espacial descreve a condição
onde um modelo global único é incapaz de capturar a complexidade das
relações, pois a própria natureza do processo se altera sobre o espaço
Brunsdon, Fotheringham, e Charlton (1996). Isso implica a violação das
suposições de média ou covariância constantes descritas acima.

A não estacionariedade pode manifestar-se como uma tendência espacial na
média (heterogeneidade de primeira ordem) ou como uma mudança na
estrutura de dependência (heterogeneidade de segunda ordem), onde, por
exemplo, o alcance da correlação espacial é curto em áreas urbanas, mas
longo em áreas rurais Dreesman e Tutz (2001). Ignorar essas
heterogeneidades e forçar um modelo estacionário em dados não
estacionários pode resultar em erros de estimativa sistemáticos,
enviesamento de previsões e, em aplicações práticas, levar a decisões
equivocadas (Bandyopadhyay e Rao 2017).

\begin{Shaded}
\begin{Highlighting}[]
\ControlFlowTok{if}\NormalTok{ (}\SpecialCharTok{!}\FunctionTok{require}\NormalTok{(}\StringTok{"pacman"}\NormalTok{)) }\FunctionTok{install.packages}\NormalTok{(}\StringTok{"pacman"}\NormalTok{)}

\NormalTok{pacman}\SpecialCharTok{::}\FunctionTok{p\_load}\NormalTok{(ggplot2, gstat, sf, viridis, gridExtra, patchwork)}

\CommentTok{\#}
\NormalTok{grid\_df }\OtherTok{\textless{}{-}} \FunctionTok{expand.grid}\NormalTok{(}\AttributeTok{x =} \DecValTok{1}\SpecialCharTok{:}\DecValTok{40}\NormalTok{, }\AttributeTok{y =} \DecValTok{1}\SpecialCharTok{:}\DecValTok{40}\NormalTok{)}
\NormalTok{grid\_sf }\OtherTok{\textless{}{-}} \FunctionTok{st\_as\_sf}\NormalTok{(grid\_df, }\AttributeTok{coords =} \FunctionTok{c}\NormalTok{(}\StringTok{"x"}\NormalTok{, }\StringTok{"y"}\NormalTok{))}

\NormalTok{modelo\_vgm }\OtherTok{\textless{}{-}}\NormalTok{ gstat}\SpecialCharTok{::}\FunctionTok{vgm}\NormalTok{(}\AttributeTok{psill =} \DecValTok{10}\NormalTok{, }\AttributeTok{model =} \StringTok{"Sph"}\NormalTok{, }\AttributeTok{range =} \DecValTok{15}\NormalTok{, }\AttributeTok{nugget =} \DecValTok{1}\NormalTok{)}

\FunctionTok{set.seed}\NormalTok{(}\DecValTok{123}\NormalTok{)}
\NormalTok{objeto\_gstat }\OtherTok{\textless{}{-}}\NormalTok{ gstat}\SpecialCharTok{::}\FunctionTok{gstat}\NormalTok{(}\AttributeTok{formula =}\NormalTok{ z}\SpecialCharTok{\textasciitilde{}}\DecValTok{1}\NormalTok{, }\AttributeTok{locations =}\NormalTok{ grid\_sf, }\AttributeTok{dummy =} \ConstantTok{TRUE}\NormalTok{, }\AttributeTok{beta =} \DecValTok{0}\NormalTok{, }\AttributeTok{model =}\NormalTok{ modelo\_vgm, }\AttributeTok{nmax =} \DecValTok{20}\NormalTok{)}
\FunctionTok{invisible}\NormalTok{(}\FunctionTok{capture.output}\NormalTok{(sim\_estacionaria }\OtherTok{\textless{}{-}} \FunctionTok{predict}\NormalTok{(objeto\_gstat, }\AttributeTok{newdata =}\NormalTok{ grid\_sf, }\AttributeTok{nsim =} \DecValTok{1}\NormalTok{)))}

\NormalTok{df\_sim }\OtherTok{\textless{}{-}} \FunctionTok{data.frame}\NormalTok{(grid\_df, }\AttributeTok{z\_estacionario =}\NormalTok{ sim\_estacionaria}\SpecialCharTok{$}\NormalTok{sim1)}

\CommentTok{\# z(s) = mu(s) + erro(s)}
\NormalTok{df\_sim}\SpecialCharTok{$}\NormalTok{tendencia }\OtherTok{\textless{}{-}} \FloatTok{0.5} \SpecialCharTok{*}\NormalTok{ df\_sim}\SpecialCharTok{$}\NormalTok{x }\SpecialCharTok{+} \FloatTok{0.5} \SpecialCharTok{*}\NormalTok{ df\_sim}\SpecialCharTok{$}\NormalTok{y}
\NormalTok{df\_sim}\SpecialCharTok{$}\NormalTok{z\_nao\_estacionario }\OtherTok{\textless{}{-}}\NormalTok{ df\_sim}\SpecialCharTok{$}\NormalTok{z\_estacionario }\SpecialCharTok{+}\NormalTok{ df\_sim}\SpecialCharTok{$}\NormalTok{tendencia}

\NormalTok{minha\_legenda }\OtherTok{\textless{}{-}} \FunctionTok{guide\_colorbar}\NormalTok{(}
  \AttributeTok{title =} \StringTok{""}\NormalTok{,}
  \AttributeTok{barwidth =} \FunctionTok{unit}\NormalTok{(}\FloatTok{0.4}\NormalTok{, }\StringTok{"npc"}\NormalTok{),}
  \AttributeTok{barheight =} \FunctionTok{unit}\NormalTok{(}\FloatTok{0.3}\NormalTok{, }\StringTok{"cm"}\NormalTok{),}
  \AttributeTok{label.position =} \StringTok{"bottom"}
\NormalTok{)}

\CommentTok{\#}
\NormalTok{p1 }\OtherTok{\textless{}{-}} \FunctionTok{ggplot}\NormalTok{(df\_sim, }\FunctionTok{aes}\NormalTok{(x, y, }\AttributeTok{fill =}\NormalTok{ z\_estacionario)) }\SpecialCharTok{+}
  \FunctionTok{geom\_tile}\NormalTok{() }\SpecialCharTok{+} \FunctionTok{scale\_fill\_viridis\_c}\NormalTok{(}\AttributeTok{option =} \StringTok{"B"}\NormalTok{) }\SpecialCharTok{+}
  \FunctionTok{coord\_fixed}\NormalTok{() }\SpecialCharTok{+} \FunctionTok{theme\_void}\NormalTok{() }\SpecialCharTok{+}
  \FunctionTok{labs}\NormalTok{(}\AttributeTok{title =} \StringTok{"Estacionário (1ª Ordem)"}\NormalTok{, }\AttributeTok{fill=}\StringTok{""}\NormalTok{) }\SpecialCharTok{+}
  \FunctionTok{theme}\NormalTok{(}\AttributeTok{plot.title =} \FunctionTok{element\_text}\NormalTok{(}\AttributeTok{hjust =} \FloatTok{0.5}\NormalTok{), }\AttributeTok{legend.position =} \StringTok{"bottom"}\NormalTok{) }\SpecialCharTok{+}
  \FunctionTok{guides}\NormalTok{(}\AttributeTok{fill =}\NormalTok{ minha\_legenda)}

\NormalTok{p2 }\OtherTok{\textless{}{-}} \FunctionTok{ggplot}\NormalTok{(df\_sim, }\FunctionTok{aes}\NormalTok{(x, y, }\AttributeTok{fill =}\NormalTok{ z\_nao\_estacionario)) }\SpecialCharTok{+}
  \FunctionTok{geom\_tile}\NormalTok{() }\SpecialCharTok{+} \FunctionTok{scale\_fill\_viridis\_c}\NormalTok{(}\AttributeTok{option =} \StringTok{"B"}\NormalTok{) }\SpecialCharTok{+}
  \FunctionTok{coord\_fixed}\NormalTok{() }\SpecialCharTok{+} \FunctionTok{theme\_void}\NormalTok{() }\SpecialCharTok{+}
  \FunctionTok{labs}\NormalTok{(}\AttributeTok{title =} \StringTok{"Não Estacionário (Tendência)"}\NormalTok{, }\AttributeTok{fill=}\StringTok{""}\NormalTok{) }\SpecialCharTok{+}
  \FunctionTok{theme}\NormalTok{(}\AttributeTok{plot.title =} \FunctionTok{element\_text}\NormalTok{(}\AttributeTok{hjust =} \FloatTok{0.5}\NormalTok{), }\AttributeTok{legend.position =} \StringTok{"bottom"}\NormalTok{) }\SpecialCharTok{+}
  \FunctionTok{guides}\NormalTok{(}\AttributeTok{fill =}\NormalTok{ minha\_legenda)}

\NormalTok{p1 }\SpecialCharTok{+}\NormalTok{ p2}
\end{Highlighting}
\end{Shaded}

\begin{figure}[H]

\centering{

\pandocbounded{\includegraphics[keepaspectratio]{fundEstspatial_files/figure-pdf/fig-stationarity-1.pdf}}

}

\caption{\label{fig-stationarity}Processo Estacionário vs.~Processo com
Tendência (Não Estacionário na Média).}

\end{figure}%

\begin{tcolorbox}[enhanced jigsaw, left=2mm, toptitle=1mm, colback=white, colframe=quarto-callout-important-color-frame, colbacktitle=quarto-callout-important-color!10!white, opacityback=0, rightrule=.15mm, bottomtitle=1mm, arc=.35mm, title=\textcolor{quarto-callout-important-color}{\faExclamation}\hspace{0.5em}{Consequência Prática na Modelagem}, titlerule=0mm, bottomrule=.15mm, leftrule=.75mm, coltitle=black, toprule=.15mm, breakable, opacitybacktitle=0.6]

A maioria dos métodos clássicos de geoestatística, como a Krigagem
Simples ou Ordinária, assume implicitamente a estacionariedade dos
dados. Se a análise exploratória revelar um padrão semelhante a
Figura~\ref{fig-stationarity} da direita (Não Estacionário), a aplicação
direta desses métodos será inválida. Nestes casos, o analista deve optar
por:

\begin{enumerate}
\def\labelenumi{\arabic{enumi}.}
\item
  Remover a tendência (\emph{detrending}) modelando-a com uma superfície
  polinomial e analisando apenas os resíduos; ou
\item
  Utilizar métodos que incorporem a tendência explicitamente, como a
  Krigagem Universal ou modelos de regressão geograficamente ponderada.
\end{enumerate}

\end{tcolorbox}

\textbf{O mapa como ferramenta analítica (e não decorativa)}

A transição do mapa de um artefato meramente ilustrativo para um
instrumento analítico robusto demanda uma reorientação metodológica
fundamental na ciência de dados espaciais. Frequentemente, o mapa é
tratado como o estágio final da pesquisa --- uma imagem estática
destinada apenas a validar resultados já obtidos ou indicar a
localização de um evento. No entanto, uma abordagem estatística rigorosa
reposiciona o mapa não como um fim, mas como um meio dinâmico de
inquirição. Segundo Waller (2024), mapas servem para localizar números
que, por sua vez, necessitam de mapas para transformar a simples
incidência de dados em ideias sobre causalidade. Essa perspectiva insere
a cartografia no que o autor descreve como o ``vórtice rodopiante da
análise'' (\href{https://www.jstor.org/stable/pdf/74133.pdf}{whirling
vortex of analysis},
\href{https://en.wikipedia.org/wiki/Vortex}{vortex}), um ciclo contínuo
onde a visualização espacial motiva as perguntas iniciais, define a
coleta de dados necessária e expõe as limitações dos métodos
estatísticos disponíveis.

Para exercer essa função diagnóstica, o analista deve interrogar o mapa
em busca de padrões estruturais específicos, nomeadamente: tendências,
verificando a existência de gradientes direcionais (ex: o Leste é
sistematicamente mais rico que o Oeste?); \emph{clusters}/agrupamentos,
identificando a presença de ilhas de valores altos ou baixos (pontos
quentes/frios); e \emph{outliers}/valores atípicos espaciais, detectando
observações que desafiam a lógica do seu entorno, como uma ilha de
riqueza cercada por um mar de pobreza. Contudo, para que essa leitura
seja válida, a construção do mapa deve obedecer a regras estritas de
\href{https://pt.wikipedia.org/wiki/Linguagem_visual}{semiologia
gráfica}. Loonis e Bellefon (2018) alerta que a escolha incorreta da
variável visual pode enviesar completamente a interpretação. Um erro
clássico e grave é a representação de dados absolutos (volumes, como
população total ou PIB) através de mapas coropléticos (áreas coloridas).
Isso gera uma distorção perceptiva onde unidades geográficas fisicamente
extensas, mas pouco povoadas, dominam visualmente o mapa, sugerindo uma
importância que não possuem. A prática analítica correta exige o uso de
símbolos proporcionais (círculos ou quadrados) para volumes, reservando
o uso de cores (mapas coropléticos) exclusivamente para variáveis
normalizadas, como taxas, densidades ou proporções.

Além da escolha do tipo de mapa, a eficácia analítica reside na
capacidade do pesquisador em manipular conscientemente a distorção da
realidade. Monmonier (2005) argumenta que a generalização cartográfica
não é uma falha, mas uma necessidade; um mapa que tentasse contar ``toda
a verdade'' na escala 1:1 resultaria em uma exibição confusa e inútil. O
ponto crítico dessa manipulação ocorre na discretização dos dados
(definição dos intervalos de classe). A aceitação ingênua das
classificações automáticas de software (\emph{default settings}) pode
mascarar tendências vitais ou criar padrões espúrios. A escolha entre
métodos como quantis (que enfatizam a ordem relativa), intervalos iguais
(que facilitam a leitura da legenda mas falham em dados assimétricos) ou
\href{https://pt.wikipedia.org/wiki/Otimiza\%C3\%A7\%C3\%A3o_de_Intervalos_Naturais_de_Jenks}{quebras
naturais de Jenks} (que buscam minimizar a variância interna dos grupos)
deve ser precedida por uma análise da distribuição dos dados (ex: usando
histograma). Como demonstrado por Monmonier (2005), alterar o método de
classificação é uma forma de análise exploratória que pode modificar
drasticamente a percepção de correlações espaciais.

A utilidade do mapa como ferramenta científica depende de sua coerência
interna e da confiança que ele inspira. Mocnik (2023) propõe que a
legibilidade de um mapa deriva da coerência entre as afirmações que ele
faz sobre o espaço, permitindo que diferentes observadores convirjam
para uma interpretação comum. Em um ambiente saturado de informações,
Prestby (2025) destaca que mapas são frequentemente usados como
dispositivos retóricos de autoridade. Portanto, para transcender a
simples ``credibilidade'' superficial e fomentar uma confiança
duradoura, o mapa analítico deve ser transparente: ele não deve
apresentar o território como uma ilha flutuando no vazio, mas incluir
contexto vizinho, metadados detalhados e, crucialmente, a visualização
explícita das incertezas inerentes ao processo de modelagem.

\begin{Shaded}
\begin{Highlighting}[]
\ControlFlowTok{if}\NormalTok{ (}\SpecialCharTok{!}\FunctionTok{require}\NormalTok{(}\StringTok{"pacman"}\NormalTok{)) }\FunctionTok{install.packages}\NormalTok{(}\StringTok{"pacman"}\NormalTok{)}
\NormalTok{pacman}\SpecialCharTok{::}\FunctionTok{p\_load}\NormalTok{(sf, ggplot2, dplyr, viridis, patchwork)}

\NormalTok{nc }\OtherTok{\textless{}{-}} \FunctionTok{st\_read}\NormalTok{(}\FunctionTok{system.file}\NormalTok{(}\StringTok{"shape/nc.shp"}\NormalTok{, }\AttributeTok{package=}\StringTok{"sf"}\NormalTok{), }\AttributeTok{quiet =} \ConstantTok{TRUE}\NormalTok{)}

\NormalTok{minha\_legenda }\OtherTok{\textless{}{-}} \FunctionTok{guide\_colorbar}\NormalTok{(}
  \AttributeTok{title =} \StringTok{""}\NormalTok{, }
  \AttributeTok{barwidth =} \FunctionTok{unit}\NormalTok{(}\FloatTok{0.4}\NormalTok{, }\StringTok{"npc"}\NormalTok{), }
  \AttributeTok{barheight =} \FunctionTok{unit}\NormalTok{(}\FloatTok{0.3}\NormalTok{, }\StringTok{"cm"}\NormalTok{),}
  \AttributeTok{label.position =} \StringTok{"bottom"}
\NormalTok{)}
\CommentTok{\#}
\NormalTok{p\_erro }\OtherTok{\textless{}{-}} \FunctionTok{ggplot}\NormalTok{(nc) }\SpecialCharTok{+}
  \FunctionTok{geom\_sf}\NormalTok{(}\FunctionTok{aes}\NormalTok{(}\AttributeTok{fill =}\NormalTok{ SID74)) }\SpecialCharTok{+}
  \FunctionTok{scale\_fill\_viridis\_c}\NormalTok{(}\AttributeTok{option =} \StringTok{"B"}\NormalTok{, }\AttributeTok{name =} \StringTok{""}\NormalTok{) }\SpecialCharTok{+}
  \FunctionTok{theme\_void}\NormalTok{() }\SpecialCharTok{+}
  \FunctionTok{labs}\NormalTok{(}\AttributeTok{title =} \StringTok{"Total de mortalidade súbita infantil (Incorreto) }\SpecialCharTok{\textbackslash{}n}\StringTok{ áreas com maior nr de nascimentos dominam"}\NormalTok{,}
       \AttributeTok{subtitle =} \StringTok{""}\NormalTok{, }\AttributeTok{fill=}\StringTok{""}\NormalTok{) }\SpecialCharTok{+}
  \FunctionTok{theme}\NormalTok{(}\AttributeTok{legend.position =} \StringTok{"bottom"}\NormalTok{, }
        \AttributeTok{plot.title =} \FunctionTok{element\_text}\NormalTok{(}\AttributeTok{size =} \DecValTok{14}\NormalTok{,}\AttributeTok{hjust =} \FloatTok{0.5}\NormalTok{))}\SpecialCharTok{+}
  \FunctionTok{guides}\NormalTok{(}\AttributeTok{fill =}\NormalTok{ minha\_legenda)}

\CommentTok{\#.}
\NormalTok{nc}\SpecialCharTok{$}\NormalTok{taxa }\OtherTok{\textless{}{-}}\NormalTok{ (nc}\SpecialCharTok{$}\NormalTok{SID74 }\SpecialCharTok{/}\NormalTok{ nc}\SpecialCharTok{$}\NormalTok{BIR74) }\SpecialCharTok{*} \DecValTok{1000}

\CommentTok{\#}
\NormalTok{nc}\SpecialCharTok{$}\NormalTok{taxa\_cat }\OtherTok{\textless{}{-}} \FunctionTok{cut}\NormalTok{(nc}\SpecialCharTok{$}\NormalTok{taxa, }
                   \AttributeTok{breaks =} \FunctionTok{quantile}\NormalTok{(nc}\SpecialCharTok{$}\NormalTok{taxa, }\AttributeTok{probs =} \FunctionTok{seq}\NormalTok{(}\DecValTok{0}\NormalTok{, }\DecValTok{1}\NormalTok{, }\FloatTok{0.25}\NormalTok{)), }
                   \AttributeTok{include.lowest =} \ConstantTok{TRUE}\NormalTok{, }
                   \AttributeTok{labels =} \FunctionTok{c}\NormalTok{(}\StringTok{"Q1 (Baixa)"}\NormalTok{, }\StringTok{"Q2"}\NormalTok{, }\StringTok{"Q3"}\NormalTok{, }\StringTok{"Q4 (Alta)"}\NormalTok{))}

\NormalTok{p\_taxa }\OtherTok{\textless{}{-}} \FunctionTok{ggplot}\NormalTok{(nc) }\SpecialCharTok{+}
  \FunctionTok{geom\_sf}\NormalTok{(}\FunctionTok{aes}\NormalTok{(}\AttributeTok{fill =}\NormalTok{ taxa\_cat)) }\SpecialCharTok{+}
  \FunctionTok{scale\_fill\_viridis\_d}\NormalTok{(}\AttributeTok{option =} \StringTok{"B"}\NormalTok{, }\AttributeTok{name =} \StringTok{""}\NormalTok{) }\SpecialCharTok{+}
  \FunctionTok{theme\_void}\NormalTok{() }\SpecialCharTok{+}
  \FunctionTok{labs}\NormalTok{(}\AttributeTok{title =} \StringTok{"Taxa de mortalidade súbita infantil}\SpecialCharTok{\textbackslash{}n}\StringTok{ por 1.000 nascimentos (Correto)"}\NormalTok{,}
       \AttributeTok{subtitle =} \StringTok{""}\NormalTok{, }\AttributeTok{fill=}\StringTok{""}\NormalTok{) }\SpecialCharTok{+}
  \FunctionTok{theme}\NormalTok{(}\AttributeTok{legend.position =} \StringTok{"bottom"}\NormalTok{,}
        \AttributeTok{plot.title =} \FunctionTok{element\_text}\NormalTok{(}\AttributeTok{size =} \DecValTok{14}\NormalTok{, }\AttributeTok{hjust =} \FloatTok{0.5}\NormalTok{))}

\CommentTok{\#}
\NormalTok{p\_intervals }\OtherTok{\textless{}{-}} \FunctionTok{ggplot}\NormalTok{(nc) }\SpecialCharTok{+}
  \FunctionTok{geom\_sf}\NormalTok{(}\FunctionTok{aes}\NormalTok{(}\AttributeTok{fill =}\NormalTok{ taxa)) }\SpecialCharTok{+}
  \FunctionTok{scale\_fill\_stepsn}\NormalTok{(}\AttributeTok{colors =}\NormalTok{ viridis}\SpecialCharTok{::}\FunctionTok{viridis}\NormalTok{(}\DecValTok{5}\NormalTok{), }
                    \AttributeTok{n.breaks =} \DecValTok{4}\NormalTok{, }\CommentTok{\# Tenta forçar intervalos iguais numéricos}
                    \AttributeTok{name =} \StringTok{""}\NormalTok{) }\SpecialCharTok{+}
  \FunctionTok{theme\_void}\NormalTok{() }\SpecialCharTok{+}
  \FunctionTok{labs}\NormalTok{(}\AttributeTok{title =} \StringTok{"Taxa intervalar de mortalidade}\SpecialCharTok{\textbackslash{}n}\StringTok{ súbita infantil por 1.000 nascimentos."}\NormalTok{,}
       \AttributeTok{subtitle =} \StringTok{""}\NormalTok{, }\AttributeTok{fill=}\StringTok{""}\NormalTok{) }\SpecialCharTok{+}
  \FunctionTok{theme}\NormalTok{(}\AttributeTok{legend.position =} \StringTok{"bottom"}\NormalTok{,}
        \AttributeTok{plot.title =} \FunctionTok{element\_text}\NormalTok{(}\AttributeTok{size =} \DecValTok{14}\NormalTok{, }\AttributeTok{hjust =} \FloatTok{0.5}\NormalTok{))}\SpecialCharTok{+}
  \FunctionTok{guides}\NormalTok{(}\AttributeTok{fill =}\NormalTok{ minha\_legenda)}

\NormalTok{p\_erro}

\NormalTok{p\_taxa}

\NormalTok{p\_intervals}
\end{Highlighting}
\end{Shaded}

\begin{figure}

\begin{minipage}{0.33\linewidth}

\begin{figure}[H]

\centering{

\pandocbounded{\includegraphics[keepaspectratio]{fundEstspatial_files/figure-pdf/fig-mapa-analitico-1.pdf}}

}

\caption{\label{fig-mapa-analitico-1}O Impacto da escolha visual do
síndrome da morte súbita infantil (SIDS) na Carolina do Norte: 1974-78}

\end{figure}%

\end{minipage}%
%
\begin{minipage}{0.33\linewidth}

\begin{figure}[H]

\centering{

\pandocbounded{\includegraphics[keepaspectratio]{fundEstspatial_files/figure-pdf/fig-mapa-analitico-2.pdf}}

}

\caption{\label{fig-mapa-analitico-2}O Impacto da escolha visual do
síndrome da morte súbita infantil (SIDS) na Carolina do Norte: 1974-78}

\end{figure}%

\end{minipage}%
%
\begin{minipage}{0.33\linewidth}

\begin{figure}[H]

\centering{

\pandocbounded{\includegraphics[keepaspectratio]{fundEstspatial_files/figure-pdf/fig-mapa-analitico-3.pdf}}

}

\caption{\label{fig-mapa-analitico-3}O Impacto da escolha visual do
síndrome da morte súbita infantil (SIDS) na Carolina do Norte: 1974-78}

\end{figure}%

\end{minipage}%

\end{figure}%

\begin{tcolorbox}[enhanced jigsaw, left=2mm, toptitle=1mm, colback=white, colframe=quarto-callout-important-color-frame, colbacktitle=quarto-callout-important-color!10!white, opacityback=0, rightrule=.15mm, bottomtitle=1mm, arc=.35mm, title=\textcolor{quarto-callout-important-color}{\faExclamation}\hspace{0.5em}{Importante}, titlerule=0mm, bottomrule=.15mm, leftrule=.75mm, coltitle=black, toprule=.15mm, breakable, opacitybacktitle=0.6]

A diferença entre um mapa decorativo e um mapa analítico reside na
intenção e na transparência do processo de construção:

\begin{itemize}
\item
  \textbf{Decorativo:} Busca a estética e a persuasão imediata,
  frequentemente ocultando a incerteza e aceitando classificações
  automáticas de software sem crítica.
\item
  \textbf{Analítico:} Busca a descoberta e a compreensão de processos.
  Ele utiliza a generalização como recurso metodológico consciente e
  estabelece a confiança através da explicitação das fontes, dos métodos
  de classificação escolhidos e das limitações dos dados.
\end{itemize}

\end{tcolorbox}

\section{O Espaço Geográfico}\label{o-espauxe7o-geogruxe1fico}

Na estatística espacial, a representação rigorosa do espaço é a base
fundamental sobre a qual todas as análises subsequentes são construídas.
Diferente da estatística clássica, onde as observações existem em um
espaço abstrato, na estatística espacial a localização física
\((x,y,z)\) e as relações métricas entre as observações são
determinantes. Saber onde algo acontece é tão importante quanto saber o
que aconteceu. Para determinar a localização de eventos, calcular
distâncias entre vizinhos ou mensurar áreas com validade física e
estatística, é imprescindível estabelecer superfícies de referência. A
seguir, detalhamos os fundamentos geodésicos essenciais para evitar os
erros mais comuns em análise espacial.

\begin{tcolorbox}[enhanced jigsaw, left=2mm, toptitle=1mm, colback=white, colframe=quarto-callout-note-color-frame, colbacktitle=quarto-callout-note-color!10!white, opacityback=0, rightrule=.15mm, bottomtitle=1mm, arc=.35mm, title=\textcolor{quarto-callout-note-color}{\faInfo}\hspace{0.5em}{Dica de Estudo}, titlerule=0mm, bottomrule=.15mm, leftrule=.75mm, coltitle=black, toprule=.15mm, breakable, opacitybacktitle=0.6]

Esta seção aborda conceitos fundamentais, porém densos. Se sentir
dificuldade na primeira leitura, não desanime: a percepção espacial
exige tempo e abstração. Recomendamos que, em caso de dúvida, você
complemente o estudo com as referências sugeridas e explore vídeos
explicativos no YouTube ou artigos no Google Scholar para ver esses
conceitos aplicados na prática.

\end{tcolorbox}

\textbf{A forma da Terra: esfera, elipsoide e geoide}

\begin{itemize}
\tightlist
\item
  Como representamos a Terra matematicamente?
\end{itemize}

A primeira etapa para localizar um ponto no espaço é definir a
superfície sobre a qual estamos trabalhando. Historicamente e
computacionalmente, trabalhamos com três aproximações da forma da terra,
cada uma com um propósito distinto na modelagem espacial.

\begin{enumerate}
\def\labelenumi{\arabic{enumi}.}
\tightlist
\item
  \textbf{Superfície topográfica} (A terra real):
  \footnote{Topografia provém do grego, onde topos significa lugar, região, e grapho significa descrever, portanto topografia é descrição de um lugar}
  É o chão onde pisamos e onde realizamos as medições. É irregular e
  rugosa. Matematicamente, é uma superfície complexa demais para
  realizar cálculos geométricos globais diretos, servindo apenas como
  objeto de medição, não de referência matemática
  Figura~\ref{fig-supTop}.
\end{enumerate}

\begin{figure}

\centering{

\includegraphics[width=0.5\linewidth,height=\textheight,keepaspectratio]{Figures/SupTop.png}

}

\caption{\label{fig-supTop}Superfície topográfica, Fonte:
\href{https://adenilsongiovanini.com.br/blog/superficie-topografica-o-que-e-e-como-mapear/}{Prof.~Adenilson
Giovanini}}

\end{figure}%

\begin{enumerate}
\def\labelenumi{\arabic{enumi}.}
\setcounter{enumi}{1}
\tightlist
\item
  \textbf{Geoide} (A Terra da Física): Imagine que a Terra fosse coberta
  inteiramente por água, sem ventos ou marés, influenciada apenas pela
  gravidade. A forma que essa água tomaria é o Geoide. O Geoide é a
  superfície equipotencial do campo gravitacional que coincide com o
  Nível Médio do Mar em repouso e se estende continuamente sob os
  continentes (Iliffe 2000). Se os oceanos pudessem fluir livremente sob
  a terra através de canais, a superfície que a água formaria seria o
  Geoide. Fisicamente, ele define a ``vertical'' (direção da gravidade).
  Porém, como é ondulado (devido à distribuição desigual de massa da
  Terra), não serve como superfície de cálculo de coordenadas
  (latitude/longitude), mas é fundamental para definir a
  \href{https://pt.wikipedia.org/wiki/Altitude_ortom\%C3\%A9trica}{altitude
  ortométrica} (onde a água flui) Figura~\ref{fig-Geioide}.
\end{enumerate}

\begin{figure}

\centering{

\includegraphics[width=0.3\linewidth,height=\textheight,keepaspectratio]{Figures/Geioide.png}

}

\caption{\label{fig-Geioide}Geoide, Fonte:
\href{https://adenilsongiovanini.com.br/blog/geoide/}{Prof.~Adenilson
Giovanini}}

\end{figure}%

\begin{enumerate}
\def\labelenumi{\arabic{enumi}.}
\setcounter{enumi}{2}
\tightlist
\item
  \textbf{Elipsoide de Revolução} ((A Terra da Matemática): Devido à
  complexidade do Geoide, utiliza-se o Elipsoide de revolução, que é uma
  figura matemática suave gerada pela rotação de uma elipse ao redor do
  seu eixo menor. Ele não tem realidade física (você não sente o
  elipsoide), mas é a superfície de referência onde projetamos as
  coordenadas para fazer contas. A relação fundamental é
  \(h \approx H + N\) (Altitude Geométrica = Ortométrica + Ondulação
  Geoidal). O elipsoide é definido por dois parâmetros principais: o
  semi-eixo maior \(a\) e o semi-eixo menor \(b\)
  Figura~\ref{fig-elipsoide}. A partir destes, define-se o achatamento
  \(f\), que descreve o quanto a Terra se desvia de uma esfera perfeita,
  calculado pela fórmula \(f = (a - b)/a\). Ignorar este achatamento em
  escalas locais ou regionais pode introduzir erros significativos de
  posição (Iliffe 2000).
\end{enumerate}

\begin{figure}

\centering{

\includegraphics[width=0.5\linewidth,height=\textheight,keepaspectratio]{Figures/elipsoide1.png}

}

\caption{\label{fig-elipsoide}Elipsoide de Revolução, Fonte: Janssen
(2009)}

\end{figure}%

\begin{tcolorbox}[enhanced jigsaw, left=2mm, toptitle=1mm, colback=white, colframe=quarto-callout-important-color-frame, colbacktitle=quarto-callout-important-color!10!white, opacityback=0, rightrule=.15mm, bottomtitle=1mm, arc=.35mm, title=\textcolor{quarto-callout-important-color}{\faExclamation}\hspace{0.5em}{Saiba mais}, titlerule=0mm, bottomrule=.15mm, leftrule=.75mm, coltitle=black, toprule=.15mm, breakable, opacitybacktitle=0.6]

Recomenda-se aos interessados aprofundar o estudo deste tema através da
leitura do livro \emph{Datums and Map Projections for remote sensing,
GIS, and surveying} {[}iliffe2000datums{]}, bem como consultando os
materiais práticos do blog do
\href{https://adenilsongiovanini.com.br/}{Professor Adenilson
Giovanini}.

\end{tcolorbox}

\section{Fundamentos de geodesia}\label{sec-grid}

A Geodesia é a ciência responsável por medir e representar a forma da
Terra, sua orientação no espaço e seu campo gravitacional. Sua tarefa
primordial é criar o arcabouço matemático que permite transformar a
superfície física irregular da Terra em dados de coordenadas \(X, Y\) e
\(Z\) processáveis por computadores Vanicek e Krakiwsky (2015). Enquanto
a topografia clássica operava separando o posicionamento horizontal do
vertical devido às limitações dos instrumentos óticos, a geodesia
moderna, impulsionada pela era espacial, opera num modelo tridimensional
integrado. Isso significa que a posição de um ponto é definida por um
vetor tridimensional com origem no centro de massa da Terra, unificando
a geometria e a física do planeta para fornecer localizações precisas em
qualquer lugar do globo.

\textbf{Sistemas geodésicos e datum (horizontal e vertical)}

Um elipsoide (Figura~\ref{fig-elipsoide}) é apenas uma forma geométrica.
Para que ele sirva como sistema de coordenadas, precisamos ``ancorá-lo''
à Terra, e essa ancoragem é definida pelo \texttt{Datum}. O Datum define
a posição do centro do elipsoide, sua orientação e escala em relação ao
planeta. Existem dois tipos principais de Datum Horizontal Janssen
(2009).

\begin{itemize}
\item
  \textbf{Datum Topocêntrico (Local)}: O elipsoide é encaixado para
  servir bem a uma região (ex: SAD69 para o Brasil). Seu centro não
  coincide com o centro de massa da Terra, sendo posicionado e orientado
  para se ajustar perfeitamente a uma região específica, como um país ou
  continente.
\item
  \textbf{Datum Geocêntrico (Global):} O centro do elipsoide coincide
  com o centro de massa da Terra (ex: SIRGAS2000, WGS84). Este é o
  padrão obrigatório para uso com GNSS/GPS.
\end{itemize}

A distinção entre datums é crucial porque uma coordenada composta por
latitude e longitude não é um local único e absoluto. Sem a
especificação do Datum, esses valores numéricos são ambíguos. A mudança
de um datum para outro, conhecida como \emph{Datum Shift}, implica que
as coordenadas numéricas de um mesmo ponto físico no chão se alteram. No
Brasil
(\href{https://www.embratop.com.br/noticias/o-que-e-datum/}{Link}), a
transição do \texttt{SAD69} para o \texttt{SIRGAS2000} implicou um
deslocamento de aproximadamente 65 metros para as mesmas coordenadas
geográficas. Além do horizontal, existe o \texttt{Datum\ Vertical}, que
define a superfície de referência para a altitude zero, geralmente
associada a um marégrafo específico que monitora o nível médio do mar
localmente.

\textbf{Sistema Geodésico Mundial (WGS84)}

O Sistema Geodésico Mundial de 1984, conhecido como \texttt{WGS84}, é o
sistema de referência padrão para o sistema GPS. Ele fornece um
referencial globalmente consistente que permite que receptores em
qualquer lugar do planeta calculem suas posições de forma compatível
Leick, Rapoport, e Tatarnikov (2015). O WGS84 não é apenas um elipsoide,
mas um sistema geodésico completo que inclui um modelo gravitacional da
Terra e parâmetros angulares de rotação. Seus parâmetros definem o
semi-eixo maior da Terra como exatamente \(6.378.137,0\) metros e um
achatamento de aproximadamente \(1/298.257\) Janssen (2009). Para a
maioria das aplicações práticas em estatística espacial e
geoprocessamento no Brasil, o WGS84 é considerado praticamente idêntico
ao SIRGAS2000, o padrão oficial brasileiro, diferindo apenas na ordem de
milímetros devido a atualizações temporais nas placas tectônicas.

\textbf{Posicionamento global (GNSS: GPS, GLONASS, Galileo)}

O termo GNSS (Global Navigation Satellite Systems) refere-se ao conjunto
de
\href{https://pt.wikipedia.org/wiki/Constela\%C3\%A7\%C3\%A3o_de_sat\%C3\%A9lites}{constelações
de satélites} (um grupo de satélites similares que orbitam a Terra de
forma sincronizada e otimizada) que permitem o posicionamento
geoespacial autônomo, englobando o GPS americano, o GLONASS russo, o
Galileo europeu e o BeiDou chinês. O funcionamento desses sistemas
baseia-se no princípio da trilateração espacial. Um receptor GNSS mede o
tempo \(t\) que um sinal de rádio leva para ``viajar'' do satélite até
ele. Conhecendo a velocidade da luz \(c\), calcula-se a distância \(d\)
através da equação \(d = c \cdot t\). Como o relógio do receptor não é
perfeitamente sincronizado com os relógios atômicos dos satélites, o
sistema precisa resolver um sistema de equações lineares com quatro
incógnitas: as três coordenadas de posição \(X, Y, Z\) e o erro do
relógio do receptor \(\delta t\). Portanto, são necessários sinais de no
mínimo quatro satélites simultaneamente para que um receptor possa
calcular uma posição tridimensional válida (Kaplan e Hegarty 2017).

\textbf{Precisão, acurácia e fontes de erro}

Na coleta de dados espaciais, é vital distinguir precisão de acurácia. A
acurácia ou exatidão refere-se a quão próximo o valor medido está do
valor verdadeiro no terreno, enquanto a precisão refere-se ao grau de
repetibilidade da medida, ou seja, quão próximos os valores medidos
estão uns dos outros em repetidas observações. Um conjunto de dados pode
ser preciso (todos os pontos agrupados) mas não acurado (todos
deslocados do local real e/ou que se deseja).

Várias fontes de erro afetam o posicionamento GNSS. Os efeitos
atmosféricos são significativos, pois a ionosfera e a troposfera
refratam o sinal de rádio, alterando sua velocidade e causando atrasos
que o receptor interpreta erroneamente como distâncias maiores. O erro
de multicaminho (\emph{multipath}) ocorre quando o sinal reflete em
superfícies como prédios, árvores ou o próprio solo antes de chegar à
antena, aumentando o tempo de viagem e gerando ``fantasmas'' de
localização, o que é crítico em ambientes urbanos. Além disso, a
geometria dos satélites, medida pelo índice DOP
(\href{https://en.wikipedia.org/wiki/Dilution_of_precision}{Dilution of
Precision}), influencia a qualidade da posição; se os satélites visíveis
estiverem agrupados em uma pequena região do céu, a incerteza na
trilateração aumenta drasticamente Langley et al. (1999).

\textbf{Coordenadas geográficas: latitude e longitude}

Para descrever a posição de um ponto sobre a superfície curva do
elipsoide, utilizamos
\href{https://pt.wikipedia.org/wiki/Coordenadas_curvil\%C3\%ADneas}{coordenadas
curvilíneas}. A Latitude (\(\phi\)) é o ângulo medido no centro do
elipsoide entre o plano equatorial e a linha normal (perpendicular) à
superfície do elipsoide que passa pelo ponto de interesse, variando de
\(-90^\circ\) no Polo Sul a \(+90^\circ\) no Polo Norte
Figura~\ref{fig-coord1} e Figura~\ref{fig-coord2}. A Longitude
(\(\lambda\)) é o ângulo medido no plano equatorial entre o
\href{https://pt.wikipedia.org/wiki/Meridiano_de_Greenwich}{Meridiano de
Greenwich}, que serve como origem, e o meridiano que passa pelo ponto,
variando de \(-180^\circ\) a Oeste a \(+180^\circ\) a Leste (Iliffe
2000).

A relação entre estas coordenadas angulares e a posição cartesiana
tridimensional \((X, Y, Z)\) é dada por: \[
    \begin{bmatrix} X \\ Y \\ Z \end{bmatrix} = 
    \begin{bmatrix} 
    (N + h) \cos \phi \cos \lambda \\ 
    (N + h) \cos \phi \sin \lambda \\ 
    [N(1-e^2) + h] \sin \phi 
    \end{bmatrix}
\] Onde \(N\) é o raio de curvatura da primeira vertical (Grande Normal)
e \(e\) é a excentricidade do elipsoide (Iliffe 2000).

É fundamental notar a implicação prática disto nas distâncias: enquanto
um grau de latitude tem comprimento linear quase constante
(\(\approx 111\) km), a distância linear correspondente a um grau de
longitude (\(d_{long}\)) varia com o cosseno da latitude:
\(d_{long} \approx 111 \cdot \cos(\phi) \text{ km}\) Isso explica o
estreitamento das distâncias, que vão de \(\approx 111\) km no Equador
até zero nos polos.

\begin{figure}

\begin{minipage}{0.50\linewidth}

\begin{figure}[H]

\centering{

\pandocbounded{\includegraphics[keepaspectratio]{Figures/coordenadas.png}}

}

\caption{\label{fig-coord1}Coordenadas geográficas, Fonte:
\href{https://www.suportegeografico.com/2023/07/coordenadas-geograficas-texto-atividades_16.html}{Suporte
Geográfico}}

\end{figure}%

\end{minipage}%
%
\begin{minipage}{0.50\linewidth}

\begin{figure}[H]

\centering{

\pandocbounded{\includegraphics[keepaspectratio]{Figures/coordnadas.png}}

}

\caption{\label{fig-coord2}Sinal de cada polo nas coordenadas
geográficas, Fonte:
\href{http://santa_isabel.tripod.com/tecnica/orientacao/latitude_longitude.html}{Santa
Isabel Tripod}}

\end{figure}%

\end{minipage}%

\end{figure}%

\textbf{Graus, minutos e segundos vs.~coordenadas decimais}

Tradicionalmente, as coordenadas geográficas são expressas no
\href{https://pt.wikipedia.org/wiki/Sistema_de_numera\%C3\%A7\%C3\%A3o_sexagesimal}{sistema
sexagesimal} de Graus, Minutos e Segundos (DMS). No entanto, para
análise computacional e estatística em ambientes como
\texttt{R\ ou\ Python}, é imperativo converter esses valores para Graus
Decimais (DD). A fórmula de conversão é dada por

\begin{equation}\phantomsection\label{eq-conversao}{
DD = (\text{sinal}) \times \left(D_{graus} + M_{minutos}/60 + S_{segundos}/3600\right).
}\end{equation}

É crucial observar o sinal: se a direção for Sul (S) ou Oeste (W), o
resultado final deve ser negativo, isto é, o sinal descrito na
Eq.~\ref{eq-conversao} deve ser negativo Figura~\ref{fig-coord2}.

Tomando como exemplo a latitude \(23^\circ 33' 15'' S\), termina por
\(S\), logo a direção é Sul . Portanto, sabemos de antemão que o
resultado numérico final deve ser negativo Figura~\ref{fig-coord2}.
Assim, realizamos a soma das partes e multiplicamos o total por \(-1\)
para posicionar corretamente a coordenada no globo:

\begin{equation}\phantomsection\label{eq-DMS}{
\begin{aligned}
DD &= (-1) \times \left( D_{graus} + \frac{M_{minutos}}{60} + \frac{S_{segundos}}{3600} \right) \\
DD &= (-1) \times \left( 23 + \frac{33}{60} + \frac{15}{3600} \right) \\
DD &= (-1) \times \left( 23 + 0,55 + 0,00416\dots \right) \\
DD &= -23,554167^\circ
\end{aligned}
}\end{equation}

A não realização dessa conversão ou o tratamento incorreto dos sinais é
uma fonte comum de erros grosseiros em análises espaciais.

A conversão inversa, de Graus Decimais (\(DD\)) de volta para o sistema
Sexagesimal (\(DMS\)), é igualmente importante para a comunicação de
resultados. O procedimento matemático isola a parte inteira e
fracionária sucessivamente. Primeiramente, o valor absoluto da
coordenada decimal fornece os Graus inteiros
(\(D = \lfloor |DD| \rfloor\)). A parte fracionária restante é
multiplicada por 60 para obter os minutos decimais; a parte inteira
deste resultado torna-se os Minutos (\(M\)). Por fim, a nova parte
fracionária restante (dos minutos) é multiplicada novamente por 60 para
obter os Segundos (\(S\)). O sinal original do valor decimal (\(+\) ou
\(-\)) determina o hemisfério (Norte/Sul para latitude, Leste/Oeste para
longitude).

Revertendo o valor \(-23,554167^\circ\) calculado anteriormente,
obtemos:

\[
\begin{aligned}
\text{Graus} &= \lfloor | -23,554167 | \rfloor = 23^\circ \\
\text{Resto}_1 &= 0,554167 \\
\text{Minutos} &= \lfloor 0,554167 \times 60 \rfloor = \lfloor 33,25002 \rfloor = 33' \\
\text{Resto}_2 &= 0,25002 \\
\text{Segundos} &= 0,25002 \times 60 \approx 15,00''
\end{aligned}
\]

Como o valor original (\(-23,554167^\circ\)) era negativo, combinamos o
resultado calculado com a direção identificada no início, resultando em:
\(23^\circ 33' 15'' S\).

\begin{tcolorbox}[enhanced jigsaw, left=2mm, toptitle=1mm, colback=white, colframe=quarto-callout-warning-color-frame, colbacktitle=quarto-callout-warning-color!10!white, opacityback=0, rightrule=.15mm, bottomtitle=1mm, arc=.35mm, title=\textcolor{quarto-callout-warning-color}{\faExclamationTriangle}\hspace{0.5em}{Atenção na Manipulação de Coordenadas}, titlerule=0mm, bottomrule=.15mm, leftrule=.75mm, coltitle=black, toprule=.15mm, breakable, opacitybacktitle=0.6]

\begin{enumerate}
\def\labelenumi{\arabic{enumi}.}
\item
  \textbf{Identificação dos Eixos:} É fundamental identificar
  corretamente quais valores correspondem à Latitude (Y) e quais à
  Longitude (X) consultando a documentação dos dados. A inversão
  acidental dessas coordenadas altera drasticamente a localização
  geográfica no mapa, podendo posicionar o objeto em outro hemisfério ou
  invalidar a geometria (ex: Latitude \(> 90^\circ\)).
\item
  \textbf{Precisão Numérica:} Ao converter de \emph{Graus Decimais} de
  volta para \emph{DMS}, é comum encontrar resíduos numéricos (ex:
  \(14.9999''\) em vez de \(15''\)). Isso ocorre devido à aritmética de
  ponto flutuante
  (\href{https://en.wikipedia.org/wiki/Floating-point_arithmetic}{floating
  point arithmetic}) dos computadores. Para visualização em mapas,
  recomenda-se arredondar os segundos para duas casas decimais, exceto
  em casos de geodésia de alta precisão.
\end{enumerate}

\end{tcolorbox}

\textbf{Grid geográfico}

O cruzamento de paralelos (linhas de latitude constante) e meridianos
(linhas de longitude constante) forma o Grid Geográfico. Este sistema
fornece uma localização absoluta e única para cada ponto na superfície
terrestre, permitindo a referência universal. Contudo, para a
estatística espacial, o grid geográfico curvo apresenta desafios
significativos. Métodos analíticos como a estimativa de densidade de
Kernel ou a função \texttt{K\ de\ Ripley} assumem distâncias euclidianas
em um plano \href{https://pt.wikipedia.org/wiki/Isotropia}{cartesiano
isotrópico}. Calcular essas distâncias diretamente sobre coordenadas
angulares (graus) introduz distorções, pois a geometria do grid não é
quadrada. Portanto, frequentemente é necessário projetar esse grid
geográfico em um plano cartesiano através de um Sistema de Coordenadas
Projetadas, processo que introduz distorções de área, forma ou
distância, mas permite a aplicação correta da geometria euclidiana
localmente.

\textbf{Sistemas de Referência de Coordenadas (CRS)}

Para manipular dados espaciais em um ambiente computacional e realizar
cálculos de distância euclidiana, precisamos traduzir a Terra curva para
uma superfície plana. O mecanismo que gerencia essa tradução e assegura
a integridade espacial dos dados é o Sistema de Referência de
Coordenadas, ou CRS
(\href{https://en.wikipedia.org/wiki/Spatial_reference_system}{Coordinate
Reference System}).

Um CRS define matematicamente como as coordenadas bidimensionais
projetadas (\(x, y\)) de um mapa se relacionam com localizações reais na
superfície da Terra. Ele contém todas as informações necessárias para
entender os números que compõem a geometria dos dados:
\texttt{o\ datum,\ o\ elipsoide\ de\ referência} e, se aplicável,
\texttt{a\ projeção\ cartográfica\ utilizada} (Snyder 1987).

\begin{tcolorbox}[enhanced jigsaw, left=2mm, toptitle=1mm, colback=white, colframe=quarto-callout-important-color-frame, colbacktitle=quarto-callout-important-color!10!white, opacityback=0, rightrule=.15mm, bottomtitle=1mm, arc=.35mm, title=\textcolor{quarto-callout-important-color}{\faExclamation}\hspace{0.5em}{Importante}, titlerule=0mm, bottomrule=.15mm, leftrule=.75mm, coltitle=black, toprule=.15mm, breakable, opacitybacktitle=0.6]

A definição de um Sistema de Referência de Coordenadas (CRS) é
indispensável para a análise espacial. Todo objeto espacial, seja um
ponto, linha ou polígono, deve ter um CRS associado. Sem isso, essas
formas são tratadas apenas como figuras geométricas em um plano
abstrato, sem correspondência real com a superfície terrestre, o que
impede a correta sobreposição de camadas (\emph{layers}) e a integração
de dados de fontes distintas.

Frequentemente, bases de dados apresentam unidades diferentes, como
graus (sistemas geográficos) e metros (sistemas projetados, ex: UTM).
Nessa situação, é imprescindível padronizar todas as camadas para um
único sistema. Cabe ao analista escolher o CRS de referência e realizar
a conversão (reprojeção) dos demais dados, decisão que deve considerar o
objetivo da análise (ex: cálculos de distância ou área requerem sistemas
projetados).

\end{tcolorbox}

\begin{tcolorbox}[enhanced jigsaw, left=2mm, toptitle=1mm, colback=white, colframe=quarto-callout-note-color-frame, colbacktitle=quarto-callout-note-color!10!white, opacityback=0, rightrule=.15mm, bottomtitle=1mm, arc=.35mm, title=\textcolor{quarto-callout-note-color}{\faInfo}\hspace{0.5em}{A Relação entre Coordenadas Geográficas e Eixos Cartesianos}, titlerule=0mm, bottomrule=.15mm, leftrule=.75mm, coltitle=black, toprule=.15mm, breakable, opacitybacktitle=0.6]

A associação correta entre coordenadas geográficas e os eixos do plano
cartesiano frequentemente gera confusão devido a uma distinção sutil
entre a representação visual das linhas e a direção de sua variação
numérica. É fundamental compreender que, embora os meridianos
(longitude) sejam desenhados como linhas verticais que conectam os
polos, o valor da longitude varia deslocando-se no sentido Leste-Oeste;
portanto, ela corresponde ao eixo horizontal (\(X\)). Inversamente,
embora os paralelos (latitude) sejam visualizados como anéis
horizontais, o valor da latitude altera-se ao mover-se no sentido
Norte-Sul, definindo o posicionamento no eixo vertical (\(Y\)).

Essa lógica impõe uma atenção redobrada na estruturação dos dados, pois
existe um conflito direto entre a convenção de fala e a formalização
algorítmica. Enquanto na linguagem coloquial e na navegação
convencionou-se dizer ``Latitude e Longitude'', a matemática e a
computação operam rigorosamente com pares ordenados na forma \((X, Y)\).
Consequentemente, ao introduzir coordenadas em softwares estatísticos ou
linguagens de programação, é imperativo inverter a ordem falada e adotar
a ordem matemática: \((\text{Longitude}, \text{Latitude})\).

\end{tcolorbox}

\textbf{Sistemas geográficos vs.~sistemas projetados}

Existem dois tipos fundamentais de CRS. Os Sistemas de Coordenadas
Geográficas (GCS) utilizam uma superfície tridimensional esférica ou
elipsoidal para definir localizações. As unidades são angulares,
geralmente graus decimais, e os exemplos mais comuns incluem o
\texttt{WGS84\ e\ o\ SIRGAS2000}. Estes sistemas são ideais para o
armazenamento global de dados, mas não são adequados para cálculos
diretos de distâncias ou áreas em duas dimensões. Já os Sistemas de
Coordenadas Projetadas (PCS) utilizam uma superfície plana
bidimensional. Eles são baseados em um GCS, mas aplicam uma
transformação matemática (projeção) para ``aplanar'' a Terra. As
unidades nestes sistemas são lineares, como metros ou pés, tornando-os a
escolha correta para cálculos de área, distância e para a maioria das
análises estatísticas espaciais. Exemplos incluem o
\href{https://pt.wikipedia.org/wiki/Universal_Transversa_de_Mercator}{sistema
UTM} e a
\href{https://pt.wikipedia.org/wiki/Proje\%C3\%A7\%C3\%A3o_de_Albers}{projeção
de Albers} (Iliffe 2000).

\begin{tcolorbox}[enhanced jigsaw, left=2mm, toptitle=1mm, colback=white, colframe=quarto-callout-tip-color-frame, colbacktitle=quarto-callout-tip-color!10!white, opacityback=0, rightrule=.15mm, bottomtitle=1mm, arc=.35mm, title=\textcolor{quarto-callout-tip-color}{\faLightbulb}\hspace{0.5em}{Saiba mais}, titlerule=0mm, bottomrule=.15mm, leftrule=.75mm, coltitle=black, toprule=.15mm, breakable, opacitybacktitle=0.6]

Para um aprofundamento teórico, recomenda-se a leitura das seções 1 a 6
do livro \emph{Spatial Data Science With Applications in R}, dos
professores
\href{https://scholar.google.com/citations?user=d6jdqdQAAAAJ&hl=en}{Edzer
Pebesma} e
\href{https://scholar.google.com/citations?user=AWeghB0AAAAJ&hl=en}{Roger
Bivand}. O material completo está disponível neste
\href{https://r-spatial.org/book/02-Spaces.html}{link}.

\end{tcolorbox}

\textbf{Por que projetar a superfície da Terra}

Embora vivamos em um globo, as telas de computador, os mapas impressos e
a matemática da geometria euclidiana são planos. A projeção é necessária
não apenas para a visualização, mas fundamentalmente para a análise.
Muitos algoritmos de estatística espacial assumem um espaço isotrópico
onde o Teorema de Pitágoras (\(h=\sqrt{\Delta x^2 + \Delta y^2}\)) é
válido. Aplicar essa fórmula diretamente a coordenadas de latitude e
longitude gera erros grosseiros e variáveis, uma vez que a distância
representada por um grau de longitude diminui à medida que nos afastamos
do Equador. Projetar os dados transforma a superfície curva em um plano
métrico (\(X, Y\)) onde a geometria euclidiana funciona corretamente
dentro de limites locais específicos (Iliffe 2000).

\textbf{Tipos de distorção: área, forma, distância e direção}

É matematicamente impossível aplanar uma superfície esférica sem
distorcer alguma de suas propriedades geométricas. As projeções
cartográficas são classificadas com base na propriedade que elas
preservam, aceitando a distorção nas outras.

\begin{enumerate}
\def\labelenumi{\arabic{enumi}.}
\item
  \textbf{Projeções conformes} preservam formas locais e ângulos, sendo
  úteis para navegação e topografia, mas distorcem drasticamente as
  áreas, fazendo regiões polares parecerem maiores do que são.
\item
  \textbf{Projeções equivalentes} ou \emph{Equal-Area} preservam as
  proporções das áreas relativas, sendo essenciais para mapas
  coropléticos estatísticos e análises de densidade, embora distorçam as
  formas.
\item
  \textbf{Projeções equidistantes} preservam as distâncias a partir de
  um ou dois pontos específicos, mas distorcem formas e áreas em outros
  locais.
\end{enumerate}

\begin{tcolorbox}[enhanced jigsaw, left=2mm, toptitle=1mm, colback=white, colframe=quarto-callout-note-color-frame, colbacktitle=quarto-callout-note-color!10!white, opacityback=0, rightrule=.15mm, bottomtitle=1mm, arc=.35mm, title=\textcolor{quarto-callout-note-color}{\faInfo}\hspace{0.5em}{Projeção}, titlerule=0mm, bottomrule=.15mm, leftrule=.75mm, coltitle=black, toprule=.15mm, breakable, opacitybacktitle=0.6]

O analista deve escolher a projeção que minimiza a distorção na
propriedade mais importante para sua análise específica.

\end{tcolorbox}

\textbf{Projeções cartográficas mais usadas em estatística espacial}

Dentre as inúmeras projeções existentes, algumas assumem protagonismo na
prática da análise espacial devido às suas propriedades geométricas
específicas:

\begin{itemize}
\tightlist
\item
  \textbf{Sistema UTM (Universal Transversa de Mercator):} Segmenta a
  Terra em 60 zonas (fusos) de 6 graus de longitude. Dentro de cada
  zona, a projeção é conforme (preserva ângulos) e as distorções
  lineares são mínimas, tornando-o o padrão para mapeamentos e análises
  em escalas local e regional.
\end{itemize}

\begin{figure}

\centering{

\includegraphics[width=0.5\linewidth,height=\textheight,keepaspectratio]{Figures/Utm-zones.jpg}

}

\caption{\label{fig-UTM}Sistema UTM, Fonte:
\href{https://pt.wikipedia.org/wiki/Universal_Transversa_de_Mercator}{Wikipedia}}

\end{figure}%

\begin{itemize}
\tightlist
\item
  \textbf{Projeção Cônica de Albers (\emph{Albers Equal-Area}):}
  Frequentemente adotada para análises que cobrem vastas extensões
  territoriais, como países de dimensões continentais (ex: Brasil ou
  EUA). Por ser uma projeção equivalente, ela preserva a área real das
  feições, assegurando que cálculos de densidade e comparações visuais
  sejam rigorosamente justos.
\end{itemize}

\begin{figure}

\centering{

\includegraphics[width=0.5\linewidth,height=\textheight,keepaspectratio]{Figures/Albers_projection_SW.jpg}

}

\caption{\label{fig-Albers}Projeção Cônica de Albers, Fonte:
\href{https://pt.wikipedia.org/wiki/Proje\%C3\%A7\%C3\%A3o_de_Albers}{Wikipedia}}

\end{figure}%

\begin{itemize}
\tightlist
\item
  \textbf{Mercator e \emph{Web Mercator}:} Embora onipresentes em mapas
  digitais (Google Maps, OpenStreetMap) devido à preservação de ângulos
  (útil para navegação), estas projeções distorcem severamente as áreas
  em direção aos polos. Portanto, são inadequadas para análises
  estatísticas que envolvam comparações de área, densidade global ou
  mapas coropléticos de grandes extensões.
\end{itemize}

\begin{figure}

\centering{

\includegraphics[width=0.5\linewidth,height=\textheight,keepaspectratio]{Figures/Web_maps_Mercator_projection_SW.jpg}

}

\caption{\label{fig-Mercator}Projeção Mercator, Fonte:
\href{https://en.wikipedia.org/wiki/Web_Mercator_projection}{Wikipedia}}

\end{figure}%

\textbf{Códigos EPSG, PROJ e WKT}

Para gerenciar a complexidade de centenas de datums e projeções,
padronizou-se internacionalmente o uso dos códigos do registro
\textbf{EPSG} (\href{https://epsg.org/home.html}{European Petroleum
Survey Group}). Estes códigos numéricos curtos funcionam como
identificadores para um CRS. Por exemplo:

\begin{itemize}
\item
  \textbf{EPSG:4326}: Refere-se ao WGS84 geográfico (Lat/Lon), padrão do
  GPS.
\item
  \textbf{EPSG:31983}: Refere-se ao SIRGAS 2000 projetado na zona UTM
  23S.
\end{itemize}

Nos bastidores, softwares de SIG e pacotes de estatística espacial
utilizam pacotes como o
\href{https://hypertidy.github.io/PROJ/index.html}{PROJ} para realizar
as transformações matemáticas entre esses sistemas. Já as definições
completas descrevendo detalhadamente o datum, o elipsoide, a projeção e
os parâmetros de transformação são armazenadas em cadeias de texto
padronizadas chamadas \textbf{WKT}
(\href{https://en.wikipedia.org/wiki/Well-known_text_representation_of_geometry}{\emph{Well-Known
Text}}).

\begin{tcolorbox}[enhanced jigsaw, left=2mm, toptitle=1mm, colback=white, colframe=quarto-callout-tip-color-frame, colbacktitle=quarto-callout-tip-color!10!white, opacityback=0, rightrule=.15mm, bottomtitle=1mm, arc=.35mm, title=\textcolor{quarto-callout-tip-color}{\faLightbulb}\hspace{0.5em}{Como descobrir o EPSG}, titlerule=0mm, bottomrule=.15mm, leftrule=.75mm, coltitle=black, toprule=.15mm, breakable, opacitybacktitle=0.6]

Para identificar o código correto de uma região, recomenda-se o uso do
repositório \href{https://epsg.io/\#google_vignette}{epsg.io}.

Ao acessar o site, você pode inserir a latitude e longitude ou utilizar
a opção \emph{``Get position on a map''}. Ao clicar na área de interesse
no mapa, o site lista os zonas UTM e os códigos EPSG vigentes para
aquela localização.

\begin{itemize}
\tightlist
\item
  Uma distinção técnica crítica que frequentemente causa erros é a
  diferença entre atribuir e transformar um CRS. Atribuir um CRS
  significa apenas dizer ao software qual é o sistema de coordenadas dos
  dados, alterando o rótulo sem modificar os valores numéricos das
  coordenadas. Isso deve ser feito apenas quando os dados não possuem
  uma definição de CRS associada mas se sabe qual deveria ser.
  Transformar ou reprojetar um CRS envolve a aplicação de fórmulas
  matemáticas para converter as coordenadas de um sistema para outro,
  alterando os valores numéricos de (\(x, y\)) para corresponder à nova
  referência. Atribuir um CRS incorreto e depois tentar transformar
  resulta em dados posicionados erroneamente no espaço.
\end{itemize}

\end{tcolorbox}

\begin{tcolorbox}[enhanced jigsaw, left=2mm, toptitle=1mm, colback=white, colframe=quarto-callout-important-color-frame, colbacktitle=quarto-callout-important-color!10!white, opacityback=0, rightrule=.15mm, bottomtitle=1mm, arc=.35mm, title=\textcolor{quarto-callout-important-color}{\faExclamation}\hspace{0.5em}{Todos mapas estão errados}, titlerule=0mm, bottomrule=.15mm, leftrule=.75mm, coltitle=black, toprule=.15mm, breakable, opacitybacktitle=0.6]

Assista ao vídeo do canal \textbf{Ciência Todo Dia}, clicando neste
\href{https://www.youtube.com/watch?v=NHMy19Gv-DA}{Link}. Recomendo
também o material produzido pela \textbf{Vox}, que é mais detalhado e
possui visualização dinâmica, acessível por este
\href{https://www.youtube.com/watch?v=kIID5FDi2JQ}{Link}

Confira a aula detalhada sobre projeções cartográficas do
\href{https://app.quemterepresenta.com.br/perfil-candidatos/professor-ricardo-marcilio-sp}{Prof.~Ricardo
Marcílio}, clicando neste
\href{https://www.youtube.com/watch?v=bxH4TwvcdMY}{Link}

\end{tcolorbox}

\section{Formas de representação espacial dos
dados}\label{formas-de-representauxe7uxe3o-espacial-dos-dados}

Para transformar a infinita complexidade do mundo real em um ambiente
computacional finito e passível de análise estatística, precisamos de
modelos de abstração. Na Ciência da Informação Geográfica, existem duas
visões fundamentais e dicotômicas sobre como modelar a realidade: a
visão baseada em Objetos (Modelo Vetorial) e a visão baseada em Campos
(Modelo Matricial/Raster) (E. Pebesma 2018).

A escolha entre um e outro não é meramente técnica, mas ontológica: ela
depende de como percebemos o fenômeno estudado. Uma casa é um objeto
discreto (tem borda definida), enquanto a temperatura do ar é um campo
contínuo (existe em toda parte e varia suavemente).

\textbf{Geometria e atributos}

Na
\href{https://pt.wikipedia.org/wiki/Geoinform\%C3\%A1tica}{geoinformática}
moderna, a unidade fundamental de análise vetorial é a Feição Simples
Feição Simples
(\href{https://en.wikipedia.org/wiki/Simple_Features}{Simple Feature}).
Este conceito, padronizado internacionalmente pela
\href{https://en.wikipedia.org/wiki/International_Organization_for_Standardization}{ISO
19125}, e descrito em E. Pebesma (2018) define que um objeto espacial é
composto pela indissociabilidade entre sua forma e seus dados:

\begin{enumerate}
\def\labelenumi{\arabic{enumi}.}
\item
  \textbf{Geometria} (Onde): É a descrição matemática da forma espacial
  e da localização absoluta do objeto no sistema de coordenadas.
\item
  \textbf{Atributos} (O quê/Quanto): São as variáveis estatísticas (como
  população, temperatura ou nome) associadas a essa geometria,
  organizadas em estruturas tabulares onde cada linha corresponde a uma
  geometria.
\end{enumerate}

A arquitetura definida pela \href{https://www.ogc.org/who-we-are/}{Open
Geospatial Consortium}
(\href{https://en.wikipedia.org/wiki/Open_Geospatial_Consortium}{OGC})
estipula que uma geometria é simples quando sua representação é
bidimensional e a interpolação entre seus vértices é estritamente linear
(linhas retas), excluindo curvas matemáticas complexas como
\href{https://en.wikipedia.org/wiki/Spline_(mathematics)}{splines} para
maximizar a eficiência computacional (Cox 2011).

\textbf{Modelo vetorial}

O modelo vetorial representa a geografia através de coordenadas
explícitas que definem vértices e arestas. A Cox (2011) estabelece uma
hierarquia de classes geométricas que derivam de uma classe raiz
abstrata \texttt{Geometry} Figura~\ref{fig-vetores}.

\begin{enumerate}
\def\labelenumi{\arabic{enumi}.}
\tightlist
\item
  \textbf{Pontos} (\texttt{POINT}): é a entidade geométrica elementar de
  dimensão zero (0-D). Definidos por uma única coordenada (x,y) ou, em
  sistemas geográficos, (Longitude, Latitude). Representam eventos onde
  a localização exata importa, mas a extensão física é irrelevante na
  escala do mapa (ex: localização de um crime, uma árvore, um poço).
\end{enumerate}

\begin{Shaded}
\begin{Highlighting}[]
\ControlFlowTok{if}\NormalTok{ (}\SpecialCharTok{!}\FunctionTok{require}\NormalTok{(}\StringTok{"pacman"}\NormalTok{)) }\FunctionTok{install.packages}\NormalTok{(}\StringTok{"pacman"}\NormalTok{)}
\NormalTok{pacman}\SpecialCharTok{::}\FunctionTok{p\_load}\NormalTok{(sf, ggplot2, patchwork)}

\CommentTok{\#Ponto}
\NormalTok{ponto }\OtherTok{\textless{}{-}} \FunctionTok{st\_point}\NormalTok{(}\FunctionTok{c}\NormalTok{(}\DecValTok{2}\NormalTok{, }\DecValTok{2}\NormalTok{))}
\NormalTok{df\_ponto }\OtherTok{\textless{}{-}} \FunctionTok{st\_sf}\NormalTok{(}\AttributeTok{geometry =} \FunctionTok{st\_sfc}\NormalTok{(ponto), }\AttributeTok{id =} \DecValTok{1}\NormalTok{, }\AttributeTok{tipo =} \StringTok{"Ponto"}\NormalTok{)}
\FunctionTok{print}\NormalTok{(df\_ponto)}
\end{Highlighting}
\end{Shaded}

\begin{verbatim}
Simple feature collection with 1 feature and 2 fields
Geometry type: POINT
Dimension:     XY
Bounding box:  xmin: 2 ymin: 2 xmax: 2 ymax: 2
CRS:           NA
  id  tipo    geometry
1  1 Ponto POINT (2 2)
\end{verbatim}

\begin{enumerate}
\def\labelenumi{\arabic{enumi}.}
\setcounter{enumi}{1}
\tightlist
\item
  \textbf{Linhas} (\texttt{LINESTRING}): é uma geometria unidimensional
  (1-D) definida por uma sequência ordenada de dois ou mais pontos
  conectados por segmentos retos. Representam fluxos ou redes (ex: rios,
  estradas, trajetórias). Uma linha é considerada simples se ela não
  cruza a si mesma (não possui auto-interseção), exceto se o ponto final
  coincidir com o inicial, formando um anel fechado
  (\texttt{LinearRing}).
\end{enumerate}

\begin{Shaded}
\begin{Highlighting}[]
\CommentTok{\#Linha}
\NormalTok{linha }\OtherTok{\textless{}{-}} \FunctionTok{st\_linestring}\NormalTok{(}\FunctionTok{rbind}\NormalTok{(}\FunctionTok{c}\NormalTok{(}\DecValTok{1}\NormalTok{, }\DecValTok{1}\NormalTok{), }\FunctionTok{c}\NormalTok{(}\DecValTok{3}\NormalTok{, }\DecValTok{3}\NormalTok{), }\FunctionTok{c}\NormalTok{(}\DecValTok{4}\NormalTok{, }\DecValTok{1}\NormalTok{)))}
\NormalTok{df\_linha }\OtherTok{\textless{}{-}} \FunctionTok{st\_sf}\NormalTok{(}\AttributeTok{geometry =} \FunctionTok{st\_sfc}\NormalTok{(linha), }\AttributeTok{id =} \DecValTok{2}\NormalTok{, }\AttributeTok{tipo =} \StringTok{"Linha"}\NormalTok{)}
\FunctionTok{print}\NormalTok{(df\_linha)}
\end{Highlighting}
\end{Shaded}

\begin{verbatim}
Simple feature collection with 1 feature and 2 fields
Geometry type: LINESTRING
Dimension:     XY
Bounding box:  xmin: 1 ymin: 1 xmax: 4 ymax: 3
CRS:           NA
  id  tipo                   geometry
1  2 Linha LINESTRING (1 1, 3 3, 4 1)
\end{verbatim}

\begin{enumerate}
\def\labelenumi{\arabic{enumi}.}
\setcounter{enumi}{2}
\tightlist
\item
  \textbf{Polígonos} (\texttt{POLYGON})
\end{enumerate}

O Polígono é uma superfície plana bidimensional (2-D) definida por um
anel externo fechado (e opcionalmente anéis internos representando
``buracos''). Representam áreas com limites definidos (ex: limites
municipais, lagos, edifícios).

\begin{Shaded}
\begin{Highlighting}[]
\CommentTok{\#Polígono}
\NormalTok{poly\_coords }\OtherTok{\textless{}{-}} \FunctionTok{rbind}\NormalTok{(}\FunctionTok{c}\NormalTok{(}\DecValTok{1}\NormalTok{, }\DecValTok{1}\NormalTok{), }\FunctionTok{c}\NormalTok{(}\DecValTok{1}\NormalTok{, }\DecValTok{4}\NormalTok{), }\FunctionTok{c}\NormalTok{(}\DecValTok{4}\NormalTok{, }\DecValTok{4}\NormalTok{), }\FunctionTok{c}\NormalTok{(}\DecValTok{4}\NormalTok{, }\DecValTok{1}\NormalTok{), }\FunctionTok{c}\NormalTok{(}\DecValTok{1}\NormalTok{, }\DecValTok{1}\NormalTok{))}
\NormalTok{poligono }\OtherTok{\textless{}{-}} \FunctionTok{st\_polygon}\NormalTok{(}\FunctionTok{list}\NormalTok{(poly\_coords))}
\NormalTok{df\_poly }\OtherTok{\textless{}{-}} \FunctionTok{st\_sf}\NormalTok{(}\AttributeTok{geometry =} \FunctionTok{st\_sfc}\NormalTok{(poligono), }\AttributeTok{id =} \DecValTok{3}\NormalTok{, }\AttributeTok{tipo =} \StringTok{"Polígono"}\NormalTok{)}
\FunctionTok{print}\NormalTok{(df\_poly)}
\end{Highlighting}
\end{Shaded}

\begin{verbatim}
Simple feature collection with 1 feature and 2 fields
Geometry type: POLYGON
Dimension:     XY
Bounding box:  xmin: 1 ymin: 1 xmax: 4 ymax: 4
CRS:           NA
  id     tipo                       geometry
1  3 Polígono POLYGON ((1 1, 1 4, 4 4, 4 ...
\end{verbatim}

\begin{tcolorbox}[enhanced jigsaw, left=2mm, toptitle=1mm, colback=white, colframe=quarto-callout-note-color-frame, colbacktitle=quarto-callout-note-color!10!white, opacityback=0, rightrule=.15mm, bottomtitle=1mm, arc=.35mm, title=\textcolor{quarto-callout-note-color}{\faInfo}\hspace{0.5em}{Regra da Mão Direita}, titlerule=0mm, bottomrule=.15mm, leftrule=.75mm, coltitle=black, toprule=.15mm, breakable, opacitybacktitle=0.6]

Para garantir que cálculos de área em superfícies esféricas sejam
inequívocos, normas modernas como o
\href{https://www.ibm.com/docs/pt-br/db2/11.5.x?topic=formats-geojson-format}{RFC
7946} (GeoJSON) impõem uma regra de orientação: o anel exterior deve ser
desenhado no sentido anti-horário, enquanto os anéis interiores
(buracos) devem seguir o sentido horário (Cox 2011).

\end{tcolorbox}

\begin{Shaded}
\begin{Highlighting}[]
\CommentTok{\#}
\NormalTok{g1 }\OtherTok{\textless{}{-}} \FunctionTok{ggplot}\NormalTok{(df\_ponto) }\SpecialCharTok{+} \FunctionTok{geom\_sf}\NormalTok{(}\AttributeTok{size =} \DecValTok{4}\NormalTok{, }\AttributeTok{color =} \StringTok{"red"}\NormalTok{) }\SpecialCharTok{+} 
  \FunctionTok{ggtitle}\NormalTok{(}\StringTok{"Ponto (POINT)"}\NormalTok{) }\SpecialCharTok{+} \FunctionTok{theme\_void}\NormalTok{() }\SpecialCharTok{+} 
  \FunctionTok{theme}\NormalTok{(}\AttributeTok{plot.title =} \FunctionTok{element\_text}\NormalTok{(}\AttributeTok{hjust =} \FloatTok{0.5}\NormalTok{))}

\NormalTok{g2 }\OtherTok{\textless{}{-}} \FunctionTok{ggplot}\NormalTok{(df\_linha) }\SpecialCharTok{+} \FunctionTok{geom\_sf}\NormalTok{(}\AttributeTok{size =} \FloatTok{1.5}\NormalTok{, }\AttributeTok{color =} \StringTok{"blue"}\NormalTok{) }\SpecialCharTok{+} 
  \FunctionTok{ggtitle}\NormalTok{(}\StringTok{"Linha (LINESTRING)"}\NormalTok{) }\SpecialCharTok{+} \FunctionTok{theme\_void}\NormalTok{() }\SpecialCharTok{+} 
  \FunctionTok{theme}\NormalTok{(}\AttributeTok{plot.title =} \FunctionTok{element\_text}\NormalTok{(}\AttributeTok{hjust =} \FloatTok{0.5}\NormalTok{))}

\NormalTok{g3 }\OtherTok{\textless{}{-}} \FunctionTok{ggplot}\NormalTok{(df\_poly) }\SpecialCharTok{+} \FunctionTok{geom\_sf}\NormalTok{(}\AttributeTok{fill =} \StringTok{"lightgreen"}\NormalTok{, }\AttributeTok{alpha =} \FloatTok{0.5}\NormalTok{) }\SpecialCharTok{+} 
  \FunctionTok{ggtitle}\NormalTok{(}\StringTok{"Polígono (POLYGON)"}\NormalTok{) }\SpecialCharTok{+} \FunctionTok{theme\_void}\NormalTok{() }\SpecialCharTok{+} 
  \FunctionTok{theme}\NormalTok{(}\AttributeTok{plot.title =} \FunctionTok{element\_text}\NormalTok{(}\AttributeTok{hjust =} \FloatTok{0.5}\NormalTok{))}

\NormalTok{g1 }\SpecialCharTok{+}\NormalTok{ g2 }\SpecialCharTok{+}\NormalTok{ g3}
\end{Highlighting}
\end{Shaded}

\begin{figure}[H]

\centering{

\pandocbounded{\includegraphics[keepaspectratio]{fundEstspatial_files/figure-pdf/fig-vetores-1.pdf}}

}

\caption{\label{fig-vetores}Primitivas Geométricas Vetoriais}

\end{figure}%

\begin{enumerate}
\def\labelenumi{\arabic{enumi}.}
\setcounter{enumi}{3}
\tightlist
\item
  \textbf{Multi-geometrias}
  (\texttt{MULTIPOINT,\ MULTILINESTRING,\ MULTIPOLYGON)}
\end{enumerate}

Nem todo fenômeno geográfico é contínuo ou contíguo. Pense no Japão, na
Indonésia ou, em menor escala, em um município que possui ilhas. Embora
existam múltiplos polígonos desconexos fisicamente (as ilhas), eles
constituem um único objeto lógico no banco de dados. Isso significa que,
na tabela de atributos, haverá apenas uma linha (um registro)
representando o ``Japão'', mas a coluna de geometria conterá um
MULTIPOLYGON com centenas de partes. Isso é fundamental para manter a
consistência estatística (ex: o PIB é do país inteiro, não de cada ilha
separadamente).

\begin{enumerate}
\def\labelenumi{\arabic{enumi}.}
\setcounter{enumi}{4}
\tightlist
\item
  \textbf{Geometrias com buracos}
\end{enumerate}

A topologia correta exige rigor na definição de áreas vazias. Um lago
dentro de uma ilha, por exemplo, não deve ser modelado como um polígono
de água desenhado sobre o polígono de terra. Topologicamente, o lago é
uma ausência de área (um buraco) dentro da ilha.

Matematicamente, um Polígono é definido por:

\begin{enumerate}
\def\labelenumi{\arabic{enumi}.}
\item
  Anel Exterior (Exterior Ring): Define a fronteira externa.
\item
  0 ou mais Anéis Interiores (Interior Rings): Definem os buracos. O
  cálculo da área geométrica é feito automaticamente subtraindo-se o
  interior do exterior.
\end{enumerate}

\begin{Shaded}
\begin{Highlighting}[]
\ControlFlowTok{if}\NormalTok{ (}\SpecialCharTok{!}\FunctionTok{require}\NormalTok{(}\StringTok{"pacman"}\NormalTok{)) }\FunctionTok{install.packages}\NormalTok{(}\StringTok{"pacman"}\NormalTok{)}
\NormalTok{pacman}\SpecialCharTok{::}\FunctionTok{p\_load}\NormalTok{(sf, ggplot2, patchwork)}

\CommentTok{\#}
\NormalTok{p1 }\OtherTok{\textless{}{-}} \FunctionTok{rbind}\NormalTok{(}\FunctionTok{c}\NormalTok{(}\DecValTok{0}\NormalTok{,}\DecValTok{0}\NormalTok{), }\FunctionTok{c}\NormalTok{(}\DecValTok{2}\NormalTok{,}\DecValTok{0}\NormalTok{), }\FunctionTok{c}\NormalTok{(}\DecValTok{2}\NormalTok{,}\DecValTok{2}\NormalTok{), }\FunctionTok{c}\NormalTok{(}\DecValTok{0}\NormalTok{,}\DecValTok{2}\NormalTok{), }\FunctionTok{c}\NormalTok{(}\DecValTok{0}\NormalTok{,}\DecValTok{0}\NormalTok{))}
\NormalTok{p2 }\OtherTok{\textless{}{-}} \FunctionTok{rbind}\NormalTok{(}\FunctionTok{c}\NormalTok{(}\DecValTok{3}\NormalTok{,}\DecValTok{3}\NormalTok{), }\FunctionTok{c}\NormalTok{(}\DecValTok{4}\NormalTok{,}\DecValTok{3}\NormalTok{), }\FunctionTok{c}\NormalTok{(}\DecValTok{4}\NormalTok{,}\DecValTok{4}\NormalTok{), }\FunctionTok{c}\NormalTok{(}\DecValTok{3}\NormalTok{,}\DecValTok{4}\NormalTok{), }\FunctionTok{c}\NormalTok{(}\DecValTok{3}\NormalTok{,}\DecValTok{3}\NormalTok{))}

\NormalTok{multi\_poly }\OtherTok{\textless{}{-}} \FunctionTok{st\_multipolygon}\NormalTok{(}\FunctionTok{list}\NormalTok{(}\FunctionTok{list}\NormalTok{(p1), }\FunctionTok{list}\NormalTok{(p2)))}

\NormalTok{df\_multi }\OtherTok{\textless{}{-}} \FunctionTok{st\_sf}\NormalTok{(}\AttributeTok{geometry =} \FunctionTok{st\_sfc}\NormalTok{(multi\_poly), }
                  \AttributeTok{id =} \DecValTok{1}\NormalTok{, }
                  \AttributeTok{nome =} \StringTok{"País Arquipélago"}\NormalTok{)}

\FunctionTok{print}\NormalTok{(df\_multi)}
\end{Highlighting}
\end{Shaded}

\begin{verbatim}
Simple feature collection with 1 feature and 2 fields
Geometry type: MULTIPOLYGON
Dimension:     XY
Bounding box:  xmin: 0 ymin: 0 xmax: 4 ymax: 4
CRS:           NA
  id             nome                       geometry
1  1 País Arquipélago MULTIPOLYGON (((0 0, 2 0, 2...
\end{verbatim}

\begin{Shaded}
\begin{Highlighting}[]
\CommentTok{\#}
\NormalTok{outer }\OtherTok{\textless{}{-}} \FunctionTok{rbind}\NormalTok{(}\FunctionTok{c}\NormalTok{(}\DecValTok{0}\NormalTok{,}\DecValTok{0}\NormalTok{), }\FunctionTok{c}\NormalTok{(}\DecValTok{5}\NormalTok{,}\DecValTok{0}\NormalTok{), }\FunctionTok{c}\NormalTok{(}\DecValTok{5}\NormalTok{,}\DecValTok{5}\NormalTok{), }\FunctionTok{c}\NormalTok{(}\DecValTok{0}\NormalTok{,}\DecValTok{5}\NormalTok{), }\FunctionTok{c}\NormalTok{(}\DecValTok{0}\NormalTok{,}\DecValTok{0}\NormalTok{))}
\CommentTok{\# }
\NormalTok{hole  }\OtherTok{\textless{}{-}} \FunctionTok{rbind}\NormalTok{(}\FunctionTok{c}\NormalTok{(}\DecValTok{1}\NormalTok{,}\DecValTok{1}\NormalTok{), }\FunctionTok{c}\NormalTok{(}\DecValTok{1}\NormalTok{,}\DecValTok{4}\NormalTok{), }\FunctionTok{c}\NormalTok{(}\DecValTok{4}\NormalTok{,}\DecValTok{4}\NormalTok{), }\FunctionTok{c}\NormalTok{(}\DecValTok{4}\NormalTok{,}\DecValTok{1}\NormalTok{), }\FunctionTok{c}\NormalTok{(}\DecValTok{1}\NormalTok{,}\DecValTok{1}\NormalTok{))}

\NormalTok{poly\_hole }\OtherTok{\textless{}{-}} \FunctionTok{st\_polygon}\NormalTok{(}\FunctionTok{list}\NormalTok{(outer, hole))}
\NormalTok{df\_hole }\OtherTok{\textless{}{-}} \FunctionTok{st\_sf}\NormalTok{(}\AttributeTok{geometry =} \FunctionTok{st\_sfc}\NormalTok{(poly\_hole), }
                 \AttributeTok{id =} \DecValTok{1}\NormalTok{, }
                 \AttributeTok{nome =} \StringTok{"Ilha com Lago"}\NormalTok{)}


\CommentTok{\#}
\NormalTok{g1 }\OtherTok{\textless{}{-}} \FunctionTok{ggplot}\NormalTok{(df\_multi) }\SpecialCharTok{+} 
  \FunctionTok{geom\_sf}\NormalTok{(}\AttributeTok{fill =} \StringTok{"orange"}\NormalTok{, }\AttributeTok{color =} \StringTok{"black"}\NormalTok{) }\SpecialCharTok{+}
  \FunctionTok{geom\_sf\_text}\NormalTok{(}\FunctionTok{aes}\NormalTok{(}\AttributeTok{label =}\NormalTok{ id), }\AttributeTok{nudge\_y =} \FloatTok{0.5}\NormalTok{, }\AttributeTok{color=}\StringTok{"white"}\NormalTok{) }\SpecialCharTok{+}
  \FunctionTok{ggtitle}\NormalTok{(}\StringTok{"MULTIPOLYGON}\SpecialCharTok{\textbackslash{}n}\StringTok{(1 Linha de dados, 2 Formas)"}\NormalTok{) }\SpecialCharTok{+} 
  \FunctionTok{theme\_void}\NormalTok{() }\SpecialCharTok{+} 
  \FunctionTok{theme}\NormalTok{(}\AttributeTok{plot.title =} \FunctionTok{element\_text}\NormalTok{(}\AttributeTok{hjust =} \FloatTok{0.5}\NormalTok{, }\AttributeTok{size=}\DecValTok{11}\NormalTok{, }\AttributeTok{face=}\StringTok{"bold"}\NormalTok{))}

\NormalTok{g1}
\end{Highlighting}
\end{Shaded}

\begin{figure}[H]

\centering{

\pandocbounded{\includegraphics[keepaspectratio]{fundEstspatial_files/figure-pdf/fig-geometrias-complexas1-1.pdf}}

}

\caption{\label{fig-geometrias-complexas1}MULTIPOLYGON}

\end{figure}%

\begin{Shaded}
\begin{Highlighting}[]
\NormalTok{g2 }\OtherTok{\textless{}{-}} \FunctionTok{ggplot}\NormalTok{(df\_hole) }\SpecialCharTok{+} 
  \FunctionTok{geom\_sf}\NormalTok{(}\AttributeTok{fill =} \StringTok{"skyblue"}\NormalTok{, }\AttributeTok{color =} \StringTok{"blue"}\NormalTok{) }\SpecialCharTok{+}
  \FunctionTok{ggtitle}\NormalTok{(}\StringTok{"POLYGON com Buraco}\SpecialCharTok{\textbackslash{}n}\StringTok{(O branco é \textquotesingle{}vazio\textquotesingle{})"}\NormalTok{) }\SpecialCharTok{+} 
  \FunctionTok{theme\_void}\NormalTok{() }\SpecialCharTok{+} 
  \FunctionTok{theme}\NormalTok{(}\AttributeTok{plot.title =} \FunctionTok{element\_text}\NormalTok{(}\AttributeTok{hjust =} \FloatTok{0.5}\NormalTok{, }\AttributeTok{size=}\DecValTok{11}\NormalTok{, }\AttributeTok{face=}\StringTok{"bold"}\NormalTok{))}

\FunctionTok{print}\NormalTok{(df\_hole);}
\end{Highlighting}
\end{Shaded}

\begin{verbatim}
Simple feature collection with 1 feature and 2 fields
Geometry type: POLYGON
Dimension:     XY
Bounding box:  xmin: 0 ymin: 0 xmax: 5 ymax: 5
CRS:           NA
  id          nome                       geometry
1  1 Ilha com Lago POLYGON ((0 0, 5 0, 5 5, 0 ...
\end{verbatim}

\begin{Shaded}
\begin{Highlighting}[]
\NormalTok{g2}
\end{Highlighting}
\end{Shaded}

\begin{figure}[H]

\centering{

\pandocbounded{\includegraphics[keepaspectratio]{fundEstspatial_files/figure-pdf/fig-geometrias-complexas-1.pdf}}

}

\caption{\label{fig-geometrias-complexas}POLYGON}

\end{figure}%

\textbf{Modelo raster}

O modelo raster abandona a noção de objetos discretos em favor de uma
representação baseada em campo (\emph{field-based}). O espaço é
particionado em uma grade regular (matriz) de células, conhecidas como
pixels Figura~\ref{fig-raster-types}.

\begin{enumerate}
\def\labelenumi{\arabic{enumi}.}
\tightlist
\item
  \textbf{Pixels e Resolução}
\end{enumerate}

\begin{itemize}
\item
  \textbf{Pixels:} Cada célula da grade armazena um valor numérico
  único. Ao contrário do vetor, onde o espaço vazio não consome memória,
  no raster o vazio deve ser preenchido (com valores \emph{NoData} ou
  zero), cobrindo toda a extensão.
\item
  \textbf{Resolução:} É determinada pelo tamanho da célula no terreno
  (ex: 30m x 30m). Há um \emph{trade-off} constante: resoluções mais
  finas capturam mais detalhes, mas aumentam quadraticamente o tamanho
  do arquivo e o custo de processamento.
\item
  \textbf{Matrizes Espaciais:} A geolocalização é implícita; baseia-se
  na posição da célula (linha/coluna) em relação a uma origem de
  coordenadas e ao tamanho do pixel, dispensando o armazenamento de
  coordenadas para cada ponto individualmente.
\end{itemize}

\textbf{Superfícies contínuas e categóricas}

Conte (2023) classifica a aplicação dos rasters em dois domínios:

\begin{enumerate}
\def\labelenumi{\arabic{enumi}.}
\item
  \textbf{Dados Contínuos:} Representam variáveis que variam suavemente
  no espaço, como elevação (DEM), temperatura ou densidade populacional
  estimada. Os valores dos pixels são números reais (ponto flutuante).
\item
  \textbf{Dados Categóricos:} Representam classes discretas. Os pixels
  contêm números inteiros que funcionam como rótulos para uma tabela de
  atributos (ex: 1 = Floresta, 2 = Água, 3 = Urbano).
\end{enumerate}

\begin{Shaded}
\begin{Highlighting}[]
\ControlFlowTok{if}\NormalTok{ (}\SpecialCharTok{!}\FunctionTok{require}\NormalTok{(}\StringTok{"pacman"}\NormalTok{)) }\FunctionTok{install.packages}\NormalTok{(}\StringTok{"pacman"}\NormalTok{)}
\NormalTok{pacman}\SpecialCharTok{::}\FunctionTok{p\_load}\NormalTok{(terra, ggplot2, tidyterra, patchwork)}

\CommentTok{\#}
\NormalTok{r\_base }\OtherTok{\textless{}{-}} \FunctionTok{rast}\NormalTok{(}\AttributeTok{nrows =} \DecValTok{10}\NormalTok{, }\AttributeTok{ncols =} \DecValTok{10}\NormalTok{, }
               \AttributeTok{xmin =} \DecValTok{0}\NormalTok{, }\AttributeTok{xmax =} \DecValTok{10}\NormalTok{, }\AttributeTok{ymin =} \DecValTok{0}\NormalTok{, }\AttributeTok{ymax =} \DecValTok{10}\NormalTok{)}

\CommentTok{\#}
\FunctionTok{values}\NormalTok{(r\_base) }\OtherTok{\textless{}{-}} \DecValTok{1}\SpecialCharTok{:}\DecValTok{100} 
\NormalTok{r\_continuo }\OtherTok{\textless{}{-}}\NormalTok{ r\_base}
\FunctionTok{names}\NormalTok{(r\_continuo) }\OtherTok{\textless{}{-}} \StringTok{"Elevacao"}

\CommentTok{\#}
\FunctionTok{set.seed}\NormalTok{(}\DecValTok{123}\NormalTok{)}
\FunctionTok{values}\NormalTok{(r\_base) }\OtherTok{\textless{}{-}} \FunctionTok{sample}\NormalTok{(}\FunctionTok{c}\NormalTok{(}\DecValTok{1}\NormalTok{, }\DecValTok{2}\NormalTok{, }\DecValTok{3}\NormalTok{), }\DecValTok{100}\NormalTok{, }\AttributeTok{replace =} \ConstantTok{TRUE}\NormalTok{)}
\NormalTok{r\_categorico }\OtherTok{\textless{}{-}} \FunctionTok{as.factor}\NormalTok{(r\_base) }
\FunctionTok{names}\NormalTok{(r\_categorico) }\OtherTok{\textless{}{-}} \StringTok{"Uso\_Solo"}

\CommentTok{\#}
\NormalTok{g1 }\OtherTok{\textless{}{-}} \FunctionTok{ggplot}\NormalTok{() }\SpecialCharTok{+}
  \FunctionTok{geom\_spatraster}\NormalTok{(}\AttributeTok{data =}\NormalTok{ r\_continuo) }\SpecialCharTok{+}
  \FunctionTok{scale\_fill\_viridis\_c}\NormalTok{(}\AttributeTok{option =} \StringTok{"B"}\NormalTok{, }\AttributeTok{name =} \StringTok{"Altitude (m)"}\NormalTok{) }\SpecialCharTok{+}
  \FunctionTok{ggtitle}\NormalTok{(}\StringTok{"Contínuo"}\NormalTok{) }\SpecialCharTok{+}
  \FunctionTok{theme\_void}\NormalTok{() }\SpecialCharTok{+}
  \FunctionTok{theme}\NormalTok{(}\AttributeTok{plot.title =} \FunctionTok{element\_text}\NormalTok{(}\AttributeTok{hjust =} \FloatTok{0.5}\NormalTok{)) }\SpecialCharTok{+}
  \FunctionTok{coord\_sf}\NormalTok{(}\AttributeTok{expand =} \ConstantTok{FALSE}\NormalTok{) }

\CommentTok{\#}
\NormalTok{g2 }\OtherTok{\textless{}{-}} \FunctionTok{ggplot}\NormalTok{() }\SpecialCharTok{+}
  \FunctionTok{geom\_spatraster}\NormalTok{(}\AttributeTok{data =}\NormalTok{ r\_categorico) }\SpecialCharTok{+}
  \FunctionTok{scale\_fill\_manual}\NormalTok{(}\AttributeTok{values =} \FunctionTok{c}\NormalTok{(}\StringTok{"1"} \OtherTok{=} \StringTok{"blue"}\NormalTok{, }\StringTok{"2"} \OtherTok{=} \StringTok{"forestgreen"}\NormalTok{, }\StringTok{"3"} \OtherTok{=} \StringTok{"sandybrown"}\NormalTok{),}
                    \AttributeTok{labels =} \FunctionTok{c}\NormalTok{(}\StringTok{"Água"}\NormalTok{, }\StringTok{"Floresta"}\NormalTok{, }\StringTok{"Solo"}\NormalTok{),}
                    \AttributeTok{name =} \StringTok{"Classe"}\NormalTok{, }
                    \AttributeTok{na.value =} \StringTok{"transparent"}\NormalTok{) }\SpecialCharTok{+}
  \FunctionTok{ggtitle}\NormalTok{(}\StringTok{"Categórico"}\NormalTok{) }\SpecialCharTok{+}
  \FunctionTok{theme\_void}\NormalTok{() }\SpecialCharTok{+}
  \FunctionTok{theme}\NormalTok{(}\AttributeTok{plot.title =} \FunctionTok{element\_text}\NormalTok{(}\AttributeTok{hjust =} \FloatTok{0.5}\NormalTok{)) }\SpecialCharTok{+}
  \FunctionTok{coord\_sf}\NormalTok{(}\AttributeTok{expand =} \ConstantTok{FALSE}\NormalTok{)}

\NormalTok{g1 }\SpecialCharTok{+}\NormalTok{ g2}
\end{Highlighting}
\end{Shaded}

\begin{figure}[H]

\centering{

\centering{

\pandocbounded{\includegraphics[keepaspectratio]{fundEstspatial_files/figure-pdf/fig-raster-types-1.pdf}}

}

\subcaption{\label{fig-raster-types}Raster Contínuo (Gradiente de
Elevação)}

}

\caption{\label{fig-raster-types}Tipos de Dados Raster}

\end{figure}%

\textbf{Formatos de dados espaciais}

Para garantir a troca de informações entre diferentes softwares e
usuários, utilizam-se padrões de arquivo específicos para cada modelo de
dados.

\begin{enumerate}
\def\labelenumi{\arabic{enumi}.}
\tightlist
\item
  \textbf{Formatos vetoriais}
\end{enumerate}

\begin{itemize}
\tightlist
\item
  \textbf{Shapefile (.shp):} Desenvolvido pela
  \href{https://www.esri.com/pt-br/home}{ESRI}, é historicamente o
  formato mais ubíquo em SIG. No entanto, é tecnicamente obsoleto e
  possui limitações severas (tamanho máximo de 2GB, nomes de colunas
  limitados a 10 caracteres).
\end{itemize}

\begin{tcolorbox}[enhanced jigsaw, left=2mm, toptitle=1mm, colback=white, colframe=quarto-callout-important-color-frame, colbacktitle=quarto-callout-important-color!10!white, opacityback=0, rightrule=.15mm, bottomtitle=1mm, arc=.35mm, title=\textcolor{quarto-callout-important-color}{\faExclamation}\hspace{0.5em}{Alerta sobre Shapefiles}, titlerule=0mm, bottomrule=.15mm, leftrule=.75mm, coltitle=black, toprule=.15mm, breakable, opacitybacktitle=0.6]

Um erro extremamente comum é tratar o
\href{https://en.wikipedia.org/wiki/Shapefile}{Shapefile} como um
arquivo único. Ele não é um arquivo único. O Shapefile é, na verdade, um
pacote de arquivos que funcionam obrigatoriamente em conjunto. Para que
o dado espacial funcione, você precisa ter, no mínimo, três arquivos na
mesma pasta e com o mesmo nome:

\begin{enumerate}
\def\labelenumi{\arabic{enumi}.}
\item
  \textbf{\texttt{.shp}}: Contém a geometria (o desenho do mapa).
\item
  \textbf{\texttt{.shx}}: Contém o índice posicional (para o software
  ler o desenho rápido).
\item
  \textbf{\texttt{.dbf}}: Contém a tabela de atributos (os dados
  estatísticos).
\end{enumerate}

Ao enviar um shapefile por e-mail ou mover de pasta, você deve mover
todos esses arquivos juntos (geralmente zipando-os). Se faltar um deles
(especialmente o .shx ou .dbf), o arquivo corrompe e não abre.

\end{tcolorbox}

\begin{itemize}
\tightlist
\item
  \textbf{GeoJSON (.json):} Um formato baseado em texto (JSON) leve e
  legível por humanos. É o padrão da web moderna. A norma RFC 7946 impõe
  restrições estritas para garantir interoperabilidade: utiliza sempre o
  datum \textbf{WGS 84} (coordenadas geográficas) e codificação de
  caracteres \textbf{UTF-8}.
\item
  \textbf{GeoPackage (.gpkg):} A alternativa moderna e aberta ao
  Shapefile. É um arquivo único (baseado em banco de dados SQLite) que
  não sofre das limitações de tamanho ou truncamento de nomes de
  colunas, suportando tanto vetores quanto rasters.
\item
  \textbf{KML/KMZ:} Formatos baseados em XML focados em visualização
  tridimensional no Google Earth, contendo informações de estilo e
  simbologia além dos dados.
\end{itemize}

\textbf{Formatos raster}

\begin{itemize}
\item
  \textbf{GeoTIFF (.tiff):} O padrão da indústria para imagens de
  satélite e modelos de elevação. É um arquivo de imagem TIFF
  convencional que possui metadados geográficos (tags) embutidos no
  cabeçalho, permitindo que o software saiba exatamente onde a imagem se
  encaixa na Terra.
\item
  \textbf{NetCDF (.nc):} O formato \emph{Network Common Data Form} é o
  padrão em oceanografia e climatologia. Sua estrutura multidimensional
  permite armazenar ``cubos de dados'' (latitude, longitude, tempo,
  altitude), sendo ideal para séries temporais de dados climáticos.
\end{itemize}

\section{Tipos de dados espaciais}\label{tipos-de-dados-espaciais}

Conforme discutido anteriormente na seção Seção~\ref{sec-class_spac}, a
estatística espacial ocupa-se da análise de dados indexados
espacialmente. Esta disciplina diverge da estatística clássica ao
incorporar explicitamente a dependência espacial e distancia-se da
análise espacial
\href{https://pt.wikipedia.org/wiki/Stricto_sensu}{sensu stricto} pelo
tratamento formal da incerteza (Noel Cressie e Moores 2022). Enquanto a
análise espacial pode restringir-se a operações geométricas ou
algorítmicas sobre informações geográficas, a estatística espacial
fundamenta-se em um formalismo probabilístico, assumindo que a
proximidade espacial implica maior dependência estatística entre
observações (Tobler 1970).

A modelagem dessa dependência é sistematizada pela natureza do domínio
espacial \(D\) onde o processo estocástico
\{\(Y(s) : s \in D \subset \mathbb{R}^d\)\} ocorre. Noel Cressie e
Moores (2022) formalizam essa estrutura através de um modelo hierárquico
de probabilidade conjunta, utilizando a notação de colchetes \([\cdot]\)
para densidades:

\begin{equation}\phantomsection\label{eq-staModel}{[Y, D] = [Y | D] [D]}\end{equation}

O componente \([D]\) modela a incerteza sobre onde as observações
ocorrem (se as localizações são fixas ou aleatórias), enquanto \([Y∣D]\)
descreve a variabilidade do atributo condicionada a essas posições. É a
natureza desse conjunto \(D\), especificamente se ele é contínuo,
discreto ou reticulado, que fundamenta a divisão clássica da estatística
espacial em três categorias:

\begin{enumerate}
\def\labelenumi{\arabic{enumi}.}
\item
  Geoestatística (\(Y(\mathbf{s}) : \mathbf {s} \in D^G \subset D\)): O
  domínio \(D^G\) é fixo e contínuo, permitindo que o atributo seja,
  teoricamente, observado em qualquer ponto. O objetivo principal é a
  predição em locais não amostrados (Chen, Genton, e Sun 2021;
  Nhancololo et al. 2024).
\item
  Dados de Área (\(Y(\mathbf{s}) : \mathbf{s} \in D^L \subset D\)): O
  domínio \(D^L\) é fixo, mas discreto e contável, consistindo em
  unidades geográficas agregadas (como municípios ou pixels) onde a
  dependência é definida por estruturas de vizinhança (Noel Cressie e
  Moores 2022).
\item
  Processos Pontuais (\(Y(\mathbf{s}): \mathbf{s} \in D^P \subset D\)):
  O domínio \(D^P\) é aleatório, sendo a própria localização dos eventos
  a variável de interesse (Nhancololo 2024b; Møller e Waagepetersen
  2007).
\end{enumerate}

Estes temas são aprofundados detalhadamente nos capítulos
Capítulo~\ref{sec-geoest} (Geoestatística),
Capítulo~\ref{sec-dados_area} (Dados de Área) e
Capítulo~\ref{sec-proc_pont} (Processos Pontuais), respectivamente.

Quando estas estruturas incorporam a dimensão temporal, o processo é
expandido para
\(\{Y(\mathbf{s}, t) : \mathbf{s} \in D, t \in \mathcal{T}\}\),
configurando os dados espaço-temporais (Chen, Genton, e Sun 2021).
Ademais, em uma fronteira metodológica mais recente, os atributos podem
ser tratados não como escalares, mas como funções completas análogo a
séries temporais indexadas pelo espaço. Estes são os dados funcionais
espaciais; caso incluam uma evolução dinâmica no tempo, denominam-se
dados funcionais espaço-temporais. Embora estas extensões transcendam o
escopo destas notas de aula, leitores interessados podem consultar
contribuições fundamentais em Delicado et al. (2010), Moreno et al.
(2023), Burbano-Moreno e Mayrink (2024), Mateu e Giraldo (2022), etc.
Para os fundamentos teóricos da análise funcional
\href{https://en.wikipedia.org/wiki/Per_se}{per se}, as obras de Ramsay
e Silverman (2005) e J.-L. Wang, Chiou, e Müller (2016) permanecem como
referências.

\begin{tcolorbox}[enhanced jigsaw, left=2mm, toptitle=1mm, colback=white, colframe=quarto-callout-warning-color-frame, colbacktitle=quarto-callout-warning-color!10!white, opacityback=0, rightrule=.15mm, bottomtitle=1mm, arc=.35mm, title=\textcolor{quarto-callout-warning-color}{\faExclamationTriangle}\hspace{0.5em}{O mesmo conjunto de dados}, titlerule=0mm, bottomrule=.15mm, leftrule=.75mm, coltitle=black, toprule=.15mm, breakable, opacitybacktitle=0.6]

É fundamental compreender que a distinção entre Geoestatística, Dados de
Área e Processos Pontuais refere-se à abordagem de modelagem escolhida
para responder a uma pergunta científica, e não estritamente ao formato
do arquivo de dados.

Um mesmo conjunto de dados original, como locais de ocorrência de crimes
(pontos com coordenadas \(x,y\)), pode transitar entre as três
categorias:

\begin{enumerate}
\def\labelenumi{\arabic{enumi}.}
\item
  Como Processo Pontual se o objetivo é entender se a localização dos
  crimes é aleatória ou se agrupam (o foco está na coordenada
  \(\mathbf{s}\)).
\item
  Como Dados de Área se contarmos quantos crimes ocorreram dentro de
  cada bairro e relacionarmos isso com a renda média do bairro (o foco
  está na agregação em \(D^L\)).
\item
  Como Geoestatística se tratarmos a densidade de crimes como uma
  superfície contínua de risco e tentarmos interpolar esse risco para
  locais onde não houve medição (o foco está na predição em \(D^G\)).
\end{enumerate}

Portanto, olhe para o seu problema e pergunte qual estrutura estocástica
melhor representa o fenômeno que você deseja investigar.

\end{tcolorbox}

\begin{Shaded}
\begin{Highlighting}[]
\ControlFlowTok{if}\NormalTok{ (}\SpecialCharTok{!}\FunctionTok{require}\NormalTok{(}\StringTok{"pacman"}\NormalTok{)) }\FunctionTok{install.packages}\NormalTok{(}\StringTok{"pacman"}\NormalTok{)}
\NormalTok{pacman}\SpecialCharTok{::}\FunctionTok{p\_load}\NormalTok{(sf, ggplot2, patchwork, dplyr, viridis, gstat, stars)}

\CommentTok{\#}
\FunctionTok{set.seed}\NormalTok{(}\DecValTok{42}\NormalTok{)}
\NormalTok{cidade }\OtherTok{\textless{}{-}} \FunctionTok{st\_polygon}\NormalTok{(}\FunctionTok{list}\NormalTok{(}\FunctionTok{rbind}\NormalTok{(}\FunctionTok{c}\NormalTok{(}\DecValTok{0}\NormalTok{,}\DecValTok{0}\NormalTok{), }\FunctionTok{c}\NormalTok{(}\DecValTok{10}\NormalTok{,}\DecValTok{0}\NormalTok{), }\FunctionTok{c}\NormalTok{(}\DecValTok{10}\NormalTok{,}\DecValTok{10}\NormalTok{), }\FunctionTok{c}\NormalTok{(}\DecValTok{0}\NormalTok{,}\DecValTok{10}\NormalTok{), }\FunctionTok{c}\NormalTok{(}\DecValTok{0}\NormalTok{,}\DecValTok{0}\NormalTok{)))) }\SpecialCharTok{\%\textgreater{}\%} 
  \FunctionTok{st\_sfc}\NormalTok{() }\SpecialCharTok{\%\textgreater{}\%} \FunctionTok{st\_sf}\NormalTok{()}

\NormalTok{n\_pontos }\OtherTok{\textless{}{-}} \DecValTok{60} 
\NormalTok{pontos }\OtherTok{\textless{}{-}} \FunctionTok{st\_sample}\NormalTok{(cidade, }\AttributeTok{size =}\NormalTok{ n\_pontos, }\AttributeTok{type =} \StringTok{"random"}\NormalTok{) }

\CommentTok{\#}
\NormalTok{pontos\_sf }\OtherTok{\textless{}{-}} \FunctionTok{st\_sf}\NormalTok{(}\AttributeTok{geometry =}\NormalTok{ pontos) }\SpecialCharTok{\%\textgreater{}\%} 
  \FunctionTok{mutate}\NormalTok{(}\AttributeTok{x =} \FunctionTok{st\_coordinates}\NormalTok{(.)[,}\DecValTok{1}\NormalTok{], }\AttributeTok{y =} \FunctionTok{st\_coordinates}\NormalTok{(.)[,}\DecValTok{2}\NormalTok{]) }\SpecialCharTok{\%\textgreater{}\%} 
  \FunctionTok{filter}\NormalTok{(x }\SpecialCharTok{\textless{}} \DecValTok{6} \SpecialCharTok{|}\NormalTok{ y }\SpecialCharTok{\textgreater{}} \DecValTok{6}\NormalTok{) }\SpecialCharTok{\%\textgreater{}\%} 
  \FunctionTok{mutate}\NormalTok{(}\AttributeTok{z\_valor =}\NormalTok{ (x }\SpecialCharTok{*} \DecValTok{2} \SpecialCharTok{+}\NormalTok{ y }\SpecialCharTok{*} \DecValTok{3}\NormalTok{) }\SpecialCharTok{+} \FunctionTok{rnorm}\NormalTok{(}\FunctionTok{n}\NormalTok{(), }\DecValTok{0}\NormalTok{, }\DecValTok{5}\NormalTok{))}

\CommentTok{\#}
\NormalTok{legenda\_continua }\OtherTok{\textless{}{-}} \FunctionTok{guides}\NormalTok{(}\AttributeTok{fill =} \FunctionTok{guide\_colorbar}\NormalTok{(}
  \AttributeTok{title =} \ConstantTok{NULL}\NormalTok{,}
  \AttributeTok{barwidth =} \FunctionTok{unit}\NormalTok{(}\FloatTok{0.29}\NormalTok{, }\StringTok{"npc"}\NormalTok{), }
  \AttributeTok{barheight =} \FunctionTok{unit}\NormalTok{(}\FloatTok{0.3}\NormalTok{, }\StringTok{"cm"}\NormalTok{),}
  \AttributeTok{label.position =} \StringTok{"bottom"}
\NormalTok{))}

\CommentTok{\#}
\NormalTok{legenda\_discreta }\OtherTok{\textless{}{-}} \FunctionTok{guides}\NormalTok{(}\AttributeTok{fill =} \FunctionTok{guide\_legend}\NormalTok{(}
  \AttributeTok{title =} \ConstantTok{NULL}\NormalTok{,}
  \AttributeTok{label.position =} \StringTok{"bottom"}\NormalTok{,}
  \AttributeTok{direction =} \StringTok{"horizontal"}\NormalTok{,}
  \AttributeTok{keywidth =} \FunctionTok{unit}\NormalTok{(.}\DecValTok{63}\NormalTok{, }\StringTok{"cm"}\NormalTok{),}
  \AttributeTok{keyheight =} \FunctionTok{unit}\NormalTok{(.}\DecValTok{3}\NormalTok{, }\StringTok{"cm"}\NormalTok{),}
  \AttributeTok{nrow =} \DecValTok{1}
\NormalTok{))}

\CommentTok{\#}
\NormalTok{tema\_base }\OtherTok{\textless{}{-}} \FunctionTok{theme\_void}\NormalTok{() }\SpecialCharTok{+} 
  \FunctionTok{theme}\NormalTok{(}
    \AttributeTok{legend.position =} \StringTok{"bottom"}\NormalTok{,}
    \AttributeTok{plot.title =} \FunctionTok{element\_text}\NormalTok{(}\AttributeTok{size=}\DecValTok{9}\NormalTok{, }\AttributeTok{hjust =} \FloatTok{0.5}\NormalTok{),}
    \AttributeTok{legend.margin =} \FunctionTok{margin}\NormalTok{(}\AttributeTok{t =} \DecValTok{0}\NormalTok{, }\AttributeTok{r =} \DecValTok{0}\NormalTok{, }\AttributeTok{b =} \DecValTok{0}\NormalTok{, }\AttributeTok{l =} \DecValTok{0}\NormalTok{)}
\NormalTok{  )}

\CommentTok{\#}
\NormalTok{g\_pp }\OtherTok{\textless{}{-}} \FunctionTok{ggplot}\NormalTok{(pontos\_sf) }\SpecialCharTok{+}
  \FunctionTok{geom\_sf}\NormalTok{(}\AttributeTok{col =} \StringTok{"red"}\NormalTok{, }\AttributeTok{size =} \DecValTok{2}\NormalTok{, }\AttributeTok{alpha=}\FloatTok{0.7}\NormalTok{) }\SpecialCharTok{+}
  \FunctionTok{geom\_sf}\NormalTok{(}\AttributeTok{data =}\NormalTok{ cidade, }\AttributeTok{fill =} \ConstantTok{NA}\NormalTok{, }\AttributeTok{col =} \StringTok{"black"}\NormalTok{) }\SpecialCharTok{+}
\NormalTok{  tema\_base }\SpecialCharTok{+}
  \FunctionTok{ggtitle}\NormalTok{(}\StringTok{"Processo Pontual}\SpecialCharTok{\textbackslash{}n}\StringTok{(O foco é a localização exata do evento)"}\NormalTok{)}

\CommentTok{\# Dados de Área (Discreto)}
\NormalTok{grid\_area }\OtherTok{\textless{}{-}} \FunctionTok{st\_make\_grid}\NormalTok{(cidade, }\AttributeTok{n =} \FunctionTok{c}\NormalTok{(}\DecValTok{4}\NormalTok{,}\DecValTok{4}\NormalTok{)) }\SpecialCharTok{\%\textgreater{}\%} \FunctionTok{st\_sf}\NormalTok{()}
\NormalTok{interseccao }\OtherTok{\textless{}{-}} \FunctionTok{st\_intersects}\NormalTok{(grid\_area, pontos\_sf)}
\NormalTok{grid\_area}\SpecialCharTok{$}\NormalTok{contagem }\OtherTok{\textless{}{-}} \FunctionTok{lengths}\NormalTok{(interseccao)}

\NormalTok{g\_area }\OtherTok{\textless{}{-}} \FunctionTok{ggplot}\NormalTok{(grid\_area) }\SpecialCharTok{+}
  \FunctionTok{geom\_sf}\NormalTok{(}\FunctionTok{aes}\NormalTok{(}\AttributeTok{fill =} \FunctionTok{factor}\NormalTok{(contagem)), }\AttributeTok{col =} \StringTok{"white"}\NormalTok{) }\SpecialCharTok{+}
  \FunctionTok{scale\_fill\_viridis\_d}\NormalTok{(}\AttributeTok{option =} \StringTok{"mako"}\NormalTok{) }\SpecialCharTok{+} 
  \FunctionTok{geom\_sf}\NormalTok{(}\AttributeTok{data=}\NormalTok{pontos\_sf, }\AttributeTok{size=}\FloatTok{0.5}\NormalTok{, }\AttributeTok{col=}\StringTok{"red"}\NormalTok{, }\AttributeTok{alpha=}\FloatTok{0.3}\NormalTok{) }\SpecialCharTok{+} 
\NormalTok{  tema\_base }\SpecialCharTok{+}
\NormalTok{  legenda\_discreta }\SpecialCharTok{+} 
  \FunctionTok{ggtitle}\NormalTok{(}\StringTok{"Dados de Área}\SpecialCharTok{\textbackslash{}n}\StringTok{(O foco é o nr de eventos por pixel)"}\NormalTok{)}

\CommentTok{\# C. Geoestatística (Contínuo)}
\NormalTok{grid\_pred }\OtherTok{\textless{}{-}} \FunctionTok{st\_as\_stars}\NormalTok{(}\FunctionTok{st\_bbox}\NormalTok{(cidade), }\AttributeTok{dx =} \FloatTok{0.1}\NormalTok{, }\AttributeTok{dy =} \FloatTok{0.1}\NormalTok{)}
\FunctionTok{invisible}\NormalTok{(}\FunctionTok{capture.output}\NormalTok{(interpola }\OtherTok{\textless{}{-}}\NormalTok{ gstat}\SpecialCharTok{::}\FunctionTok{idw}\NormalTok{(z\_valor }\SpecialCharTok{\textasciitilde{}} \DecValTok{1}\NormalTok{, }\AttributeTok{locations =}\NormalTok{ pontos\_sf, }\AttributeTok{newdata =}\NormalTok{ grid\_pred, }\AttributeTok{idp =} \FloatTok{2.0}\NormalTok{)))}

\NormalTok{g\_geo }\OtherTok{\textless{}{-}} \FunctionTok{ggplot}\NormalTok{() }\SpecialCharTok{+}
  \FunctionTok{geom\_stars}\NormalTok{(}\AttributeTok{data =}\NormalTok{ interpola, }\FunctionTok{aes}\NormalTok{(}\AttributeTok{fill =}\NormalTok{ var1.pred)) }\SpecialCharTok{+}
  \FunctionTok{scale\_fill\_viridis\_c}\NormalTok{(}\AttributeTok{option =} \StringTok{"inferno"}\NormalTok{, }\AttributeTok{na.value =} \StringTok{"white"}\NormalTok{) }\SpecialCharTok{+} 
  \FunctionTok{geom\_sf}\NormalTok{(}\AttributeTok{data =}\NormalTok{ cidade, }\AttributeTok{fill =} \ConstantTok{NA}\NormalTok{, }\AttributeTok{col =} \StringTok{"black"}\NormalTok{) }\SpecialCharTok{+} 
\NormalTok{  tema\_base }\SpecialCharTok{+}
\NormalTok{  legenda\_continua }\SpecialCharTok{+}
  \FunctionTok{ggtitle}\NormalTok{(}\StringTok{"Geoestatística}\SpecialCharTok{\textbackslash{}n}\StringTok{(O foco é a predição dos}\SpecialCharTok{\textbackslash{}n}\StringTok{ eventos onde não existem)"}\NormalTok{)}

\NormalTok{g\_pp }\SpecialCharTok{+}\NormalTok{ g\_area }\SpecialCharTok{+}\NormalTok{ g\_geo}
\end{Highlighting}
\end{Shaded}

\begin{figure}[H]

\centering{

\pandocbounded{\includegraphics[keepaspectratio]{fundEstspatial_files/figure-pdf/fig-tres-visoes-final-1.pdf}}

}

\caption{\label{fig-tres-visoes-final}O mesmo dado (pontos), três
abordagens de modelagem.}

\end{figure}%

\section{Manuseamento de dados
espaciais:}\label{manuseamento-de-dados-espaciais}

\subsection{O pacote sf}\label{o-pacote-sf}

O pacote \texttt{sf} (\emph{Simple Features for R}) E. Pebesma (2018)
representa o padrão moderno para a manipulação de dados espaciais no R.
Desenvolvido pelo professor
\href{https://scholar.google.com/citations?user=d6jdqdQAAAAJ&hl=en}{Edzer
Pebesma}, ele substituiu os antigos pacotes \texttt{sp}, \texttt{rgdal}
e \texttt{rgeos}, unificando as funcionalidades de leitura, projeção e
operações geométricas numa estrutura compatível com o
\texttt{Tidyverse}.

O \texttt{sf} depende de bibliotecas externas de sistema (C++)
poderosas: \emph{GDAL} (leitura/escrita de arquivos), \emph{GEOS}
(geometria plana) e \emph{PROJ} (projeções cartográficas). Ao instalar
no R, ele compila ou baixa estas dependências.

A grande revolução do \texttt{sf} é tratar os dados espaciais como data
frames (tabelas) comuns. Enquanto no antigo \texttt{sp} os dados eram
objetos complexos e opacos, no \texttt{sf} a geometria é apenas uma
coluna extra (geralmente chamada \texttt{geometry} ou \texttt{geom})
numa tabela de dados. Isso permite que utilizemos funções do pacote
\texttt{dplyr} (\texttt{select}, \texttt{filter}, \texttt{mutate})
diretamente no mapa.

\begin{enumerate}
\def\labelenumi{\arabic{enumi}.}
\tightlist
\item
  \textbf{Instalação e Carregamento}, ver
  Seção~\ref{sec-install_pacotes}
\end{enumerate}

\begin{Shaded}
\begin{Highlighting}[]
\ControlFlowTok{if}\NormalTok{ (}\SpecialCharTok{!}\FunctionTok{require}\NormalTok{ (}\StringTok{"pacman"}\NormalTok{)) }\FunctionTok{install.packages}\NormalTok{(}\StringTok{"pacman"}\NormalTok{)}
\FunctionTok{p\_load}\NormalTok{(sf, tidyverse)}
\end{Highlighting}
\end{Shaded}

A estrutura é composta por três níveis hierárquicos:

\begin{enumerate}
\def\labelenumi{\arabic{enumi}.}
\item
  \textbf{\texttt{sfg} (Simple Feature Geometry):} É a geometria de uma
  única feição (ex: um único polígono representando um lago).
\item
  \textbf{\texttt{sfc} (Simple Feature Column):} É uma lista que contém
  todas as geometrias (\texttt{sfg}) de todas as linhas da tabela, além
  de metadados cruciais como o sistema de coordenadas (CRS) e a caixa
  delimitadora (Bbox).
\item
  \textbf{\texttt{sf} (Simple Feature):} É a tabela completa, combinando
  a coluna \texttt{sfc} com os atributos (dados estatísticos como nome,
  população, etc.).
\end{enumerate}

\textbf{Criação de Geometrias (\texttt{st\_})}

Podemos criar geometrias do zero usando coordenadas numéricas.

\begin{enumerate}
\def\labelenumi{\arabic{enumi}.}
\tightlist
\item
  \textbf{Criar pontos (\texttt{st\_point})}
\end{enumerate}

\begin{Shaded}
\begin{Highlighting}[]
\NormalTok{ponto }\OtherTok{\textless{}{-}} \FunctionTok{st\_point}\NormalTok{(}\FunctionTok{c}\NormalTok{(}\DecValTok{2}\NormalTok{, }\DecValTok{2}\NormalTok{)) }\CommentTok{\# X=2, Y=2}
\FunctionTok{class}\NormalTok{(ponto) }\CommentTok{\# Retorna "XY", "POINT", "sfg"}
\end{Highlighting}
\end{Shaded}

\begin{verbatim}
[1] "XY"    "POINT" "sfg"  
\end{verbatim}

\begin{enumerate}
\def\labelenumi{\arabic{enumi}.}
\setcounter{enumi}{1}
\tightlist
\item
  \textbf{Criar Linhas (\texttt{st\_linestring})}
\end{enumerate}

\begin{Shaded}
\begin{Highlighting}[]
\CommentTok{\# Matriz de coordenadas}
\NormalTok{matriz\_linha }\OtherTok{\textless{}{-}} \FunctionTok{rbind}\NormalTok{(}\FunctionTok{c}\NormalTok{(}\DecValTok{1}\NormalTok{, }\DecValTok{1}\NormalTok{), }\FunctionTok{c}\NormalTok{(}\DecValTok{3}\NormalTok{, }\DecValTok{3}\NormalTok{), }\FunctionTok{c}\NormalTok{(}\DecValTok{4}\NormalTok{, }\DecValTok{1}\NormalTok{))}
\NormalTok{linha }\OtherTok{\textless{}{-}} \FunctionTok{st\_linestring}\NormalTok{(matriz\_linha)}
\FunctionTok{class}\NormalTok{(linha)}
\end{Highlighting}
\end{Shaded}

\begin{verbatim}
[1] "XY"         "LINESTRING" "sfg"       
\end{verbatim}

\begin{enumerate}
\def\labelenumi{\arabic{enumi}.}
\setcounter{enumi}{2}
\tightlist
\item
  \textbf{Criar polígonos (\texttt{st\_polygon})}
\end{enumerate}

\begin{Shaded}
\begin{Highlighting}[]
\CommentTok{\# Anel externo (Triângulo)}
\NormalTok{matriz\_poly }\OtherTok{\textless{}{-}} \FunctionTok{rbind}\NormalTok{(}\FunctionTok{c}\NormalTok{(}\DecValTok{1}\NormalTok{, }\DecValTok{1}\NormalTok{), }\FunctionTok{c}\NormalTok{(}\DecValTok{4}\NormalTok{, }\DecValTok{1}\NormalTok{), }\FunctionTok{c}\NormalTok{(}\DecValTok{4}\NormalTok{, }\DecValTok{4}\NormalTok{), }\FunctionTok{c}\NormalTok{(}\DecValTok{1}\NormalTok{, }\DecValTok{1}\NormalTok{))}
\NormalTok{poligono }\OtherTok{\textless{}{-}} \FunctionTok{st\_polygon}\NormalTok{(}\FunctionTok{list}\NormalTok{(matriz\_poly))}
\FunctionTok{class}\NormalTok{(poligono)}
\end{Highlighting}
\end{Shaded}

\begin{verbatim}
[1] "XY"      "POLYGON" "sfg"    
\end{verbatim}

\begin{enumerate}
\def\labelenumi{\arabic{enumi}.}
\setcounter{enumi}{3}
\tightlist
\item
  \textbf{Criar o objeto final (\texttt{st\_sfc} e \texttt{st\_sf})}
\end{enumerate}

Para transformar essas geometrias soltas num objeto espacial analisável,
agrupamo-las.

\begin{tcolorbox}[enhanced jigsaw, left=2mm, toptitle=1mm, colback=white, colframe=quarto-callout-tip-color-frame, colbacktitle=quarto-callout-tip-color!10!white, opacityback=0, rightrule=.15mm, bottomtitle=1mm, arc=.35mm, title=\textcolor{quarto-callout-tip-color}{\faLightbulb}\hspace{0.5em}{Quando usar st\_sfc vs.~st\_sf?}, titlerule=0mm, bottomrule=.15mm, leftrule=.75mm, coltitle=black, toprule=.15mm, breakable, opacitybacktitle=0.6]

A distinção entre estas duas funções é fundamental para entender a
estrutura do pacote:

\begin{itemize}
\item
  Use \texttt{st\_sfc()} (Simple Feature Column): Quando você tem a
  lista de geometrias e precisa definir o Sistema de Coordenadas (CRS).
  Pense nela como a função que cria a \emph{coluna} geométrica. Ela não
  aceita dados estatísticos (nomes, valores), apenas formas e
  coordenadas.
\item
  Use \texttt{st\_sf()} (Simple Feature): Quando você quer criar o
  objeto final para análise. Ela age como um \texttt{data.frame} ou
  \texttt{tibble}, colando a coluna geométrica (\texttt{sfc}) aos seus
  dados de atributos (IDs, Variáveis, etc.). É com este objeto que você
  fará gráficos e estatísticas.
\end{itemize}

\end{tcolorbox}

\begin{Shaded}
\begin{Highlighting}[]
\CommentTok{\# Cria a COLUNA de geometria (sfc) }
\CommentTok{\# É aqui que definimos o CRS (ex: 4326 para GPS/WGS84)}
\NormalTok{coluna\_geo }\OtherTok{\textless{}{-}} \FunctionTok{st\_sfc}\NormalTok{(ponto, linha, poligono, }\AttributeTok{crs =} \DecValTok{4326}\NormalTok{)}

\CommentTok{\#Cria o DATA FRAME espacial (sf) }
\CommentTok{\# Aqui unimos os dados (atributos) com a coluna criada acima}
\NormalTok{meu\_mapa }\OtherTok{\textless{}{-}} \FunctionTok{st\_sf}\NormalTok{(}
  \AttributeTok{id =} \DecValTok{1}\SpecialCharTok{:}\DecValTok{3}\NormalTok{,}
  \AttributeTok{nome =} \FunctionTok{c}\NormalTok{(}\StringTok{"Ponto A"}\NormalTok{, }\StringTok{"Estrada B"}\NormalTok{, }\StringTok{"Terreno C"}\NormalTok{),}
  \AttributeTok{geometry =}\NormalTok{ coluna\_geo}
\NormalTok{)}

\FunctionTok{print}\NormalTok{(meu\_mapa)}
\end{Highlighting}
\end{Shaded}

\begin{verbatim}
Simple feature collection with 3 features and 2 fields
Geometry type: GEOMETRY
Dimension:     XY
Bounding box:  xmin: 1 ymin: 1 xmax: 4 ymax: 4
Geodetic CRS:  WGS 84
  id      nome                       geometry
1  1   Ponto A                    POINT (2 2)
2  2 Estrada B     LINESTRING (1 1, 3 3, 4 1)
3  3 Terreno C POLYGON ((1 1, 4 1, 4 4, 1 1))
\end{verbatim}

Ao imprimir o objeto \texttt{meu\_mapa}, o cabeçalho (\texttt{header})
fornece metadados vitais:

\begin{itemize}
\item
  \textbf{Geometry type:} O tipo geométrico predominante.
\item
  \textbf{Dimension:} XY (2D).
\item
  \textbf{Bbox:} A caixa delimitadora (limites geográficos mínimos e
  máximos).
\item
  \textbf{CRS:} O Sistema de Referência de Coordenadas (EPSG: 4326).
\end{itemize}

\textbf{Entrada e Saída (I/O)}

O \texttt{sf} utiliza a biblioteca
\href{https://gdal.org/en/stable/}{GDAL}, permitindo ler e escrever
dezenas de formatos de arquivos
(\texttt{Shapefile,\ GeoJSON,\ KML,\ GPKG,\ PostGIS,\ etc.}).

\begin{enumerate}
\def\labelenumi{\arabic{enumi}.}
\tightlist
\item
  \textbf{Leitura (\texttt{st\_read} e \texttt{read\_sf})}
\end{enumerate}

\begin{itemize}
\item
  \textbf{\texttt{st\_read(dsn,\ layer,\ ...)}:} Função base R. Imprime
  um relatório detalhado sobre o ficheiro ao carregar.
\item
  \textbf{\texttt{read\_sf(dsn,\ ...)}:} Versão ``tidy''. É silenciosa e
  retorna um objeto do tipo \texttt{tibble} (tabela moderna),
  facilitando a visualização no console.
\end{itemize}

Vamos carregar o mapa da Carolina do Norte (\texttt{nc}), incluído no
pacote:

\begin{Shaded}
\begin{Highlighting}[]
\FunctionTok{p\_load}\NormalTok{(tidyverse)}
\CommentTok{\# system.file encontra o caminho do ficheiro no seu computador}
\NormalTok{arquivo }\OtherTok{\textless{}{-}} \FunctionTok{system.file}\NormalTok{(}\StringTok{"shape/nc.shp"}\NormalTok{, }\AttributeTok{package=}\StringTok{"sf"}\NormalTok{)}

\FunctionTok{paste}\NormalTok{(}\StringTok{"O caminho do meu arquivo é:"}\NormalTok{, arquivo)}
\end{Highlighting}
\end{Shaded}

\begin{verbatim}
[1] "O caminho do meu arquivo é: /home/almonha/R/x86_64-pc-linux-gnu-library/4.5/sf/shape/nc.shp"
\end{verbatim}

\begin{Shaded}
\begin{Highlighting}[]
\NormalTok{nc }\OtherTok{\textless{}{-}} \FunctionTok{read\_sf}\NormalTok{(arquivo) }\CommentTok{\# Aqui vc poderia usar o caminho onde está seu arquivo, exemplo:}
                       \CommentTok{\# nc \textless{}{-} read\_sf("/home/almonha/R/x86\_64{-}pc{-}linux{-}gnu{-}library/4.5/sf/shape/nc.shp")}
                       \CommentTok{\# onde nc.shp é o ficheiro de interesse}
\FunctionTok{glimpse}\NormalTok{(nc) }\CommentTok{\# Visualiza a estrutura}
\end{Highlighting}
\end{Shaded}

\begin{verbatim}
Rows: 100
Columns: 15
$ AREA      <dbl> 0.114, 0.061, 0.143, 0.070, 0.153, 0.097, 0.062, 0.091, 0.11~
$ PERIMETER <dbl> 1.442, 1.231, 1.630, 2.968, 2.206, 1.670, 1.547, 1.284, 1.42~
$ CNTY_     <dbl> 1825, 1827, 1828, 1831, 1832, 1833, 1834, 1835, 1836, 1837, ~
$ CNTY_ID   <dbl> 1825, 1827, 1828, 1831, 1832, 1833, 1834, 1835, 1836, 1837, ~
$ NAME      <chr> "Ashe", "Alleghany", "Surry", "Currituck", "Northampton", "H~
$ FIPS      <chr> "37009", "37005", "37171", "37053", "37131", "37091", "37029~
$ FIPSNO    <dbl> 37009, 37005, 37171, 37053, 37131, 37091, 37029, 37073, 3718~
$ CRESS_ID  <int> 5, 3, 86, 27, 66, 46, 15, 37, 93, 85, 17, 79, 39, 73, 91, 42~
$ BIR74     <dbl> 1091, 487, 3188, 508, 1421, 1452, 286, 420, 968, 1612, 1035,~
$ SID74     <dbl> 1, 0, 5, 1, 9, 7, 0, 0, 4, 1, 2, 16, 4, 4, 4, 18, 3, 4, 1, 1~
$ NWBIR74   <dbl> 10, 10, 208, 123, 1066, 954, 115, 254, 748, 160, 550, 1243, ~
$ BIR79     <dbl> 1364, 542, 3616, 830, 1606, 1838, 350, 594, 1190, 2038, 1253~
$ SID79     <dbl> 0, 3, 6, 2, 3, 5, 2, 2, 2, 5, 2, 5, 4, 4, 6, 17, 4, 7, 1, 0,~
$ NWBIR79   <dbl> 19, 12, 260, 145, 1197, 1237, 139, 371, 844, 176, 597, 1369,~
$ geometry  <MULTIPOLYGON [°]> MULTIPOLYGON (((-81.47276 3..., MULTIPOLYGON ((~
\end{verbatim}

\textbf{Salvar arquivo sf (\texttt{st\_write} e \texttt{write\_sf})}

Para exportar o resultado das suas análises:

\begin{itemize}
\item
  \textbf{\texttt{st\_write(obj,\ "arquivo.ext",\ delete\_layer\ =\ TRUE)}:}
  Exporta o objeto. O argumento \texttt{delete\_layer} permite
  sobrescrever arquivos existentes.
\item
  O formato é determinado automaticamente pela extensão do arquivo
  (\texttt{.shp}, \texttt{.json}, \texttt{.gpkg}).
\end{itemize}

\begin{Shaded}
\begin{Highlighting}[]
\CommentTok{\# Guardar como GeoPackage}
\FunctionTok{write\_sf}\NormalTok{(nc, }\StringTok{"mapa\_final.gpkg"}\NormalTok{)}

\CommentTok{\# Guardar como Shapefile}
\FunctionTok{write\_sf}\NormalTok{(nc, }\StringTok{"mapa\_final.shp"}\NormalTok{)}
\end{Highlighting}
\end{Shaded}

\begin{tcolorbox}[enhanced jigsaw, left=2mm, toptitle=1mm, colback=white, colframe=quarto-callout-note-color-frame, colbacktitle=quarto-callout-note-color!10!white, opacityback=0, rightrule=.15mm, bottomtitle=1mm, arc=.35mm, title=\textcolor{quarto-callout-note-color}{\faInfo}\hspace{0.5em}{Outas formas de salvar}, titlerule=0mm, bottomrule=.15mm, leftrule=.75mm, coltitle=black, toprule=.15mm, breakable, opacitybacktitle=0.6]

Consulte Seção~\ref{sec-Entr_Man} para ver as outras formas/extensões de
salvar

\end{tcolorbox}

\textbf{Sistemas de Coordenadas (CRS)}

A gestão do CRS é a causa número um de erros em análise espacial. O
\texttt{sf} oferece duas funções principais para lidar com isto.

\begin{itemize}
\tightlist
\item
  \textbf{Identificação (\texttt{st\_crs})}
\end{itemize}

Consulta o sistema de coordenadas associado ao objeto. Retorna o código
EPSG e a definição WKT.

\begin{Shaded}
\begin{Highlighting}[]
\FunctionTok{st\_crs}\NormalTok{(nc) }\CommentTok{\# Retorna "NAD27" (EPSG 4267)}
\end{Highlighting}
\end{Shaded}

\begin{verbatim}
Coordinate Reference System:
  User input: NAD27 
  wkt:
GEOGCRS["NAD27",
    DATUM["North American Datum 1927",
        ELLIPSOID["Clarke 1866",6378206.4,294.978698213898,
            LENGTHUNIT["metre",1]]],
    PRIMEM["Greenwich",0,
        ANGLEUNIT["degree",0.0174532925199433]],
    CS[ellipsoidal,2],
        AXIS["latitude",north,
            ORDER[1],
            ANGLEUNIT["degree",0.0174532925199433]],
        AXIS["longitude",east,
            ORDER[2],
            ANGLEUNIT["degree",0.0174532925199433]],
    ID["EPSG",4267]]
\end{verbatim}

\begin{itemize}
\tightlist
\item
  \textbf{Transformação (\texttt{st\_transform})}
\end{itemize}

Reprojeta (converte matematicamente) as coordenadas de um sistema para
outro caso necessário.

Exemplo: O mapa \texttt{nc} está em NAD27 (unidade: graus/pés). Queremos
calcular distâncias em metros. Devemos transformar para um sistema
projetado, como UTM (Zona 17N - EPSG 32617).

\begin{Shaded}
\begin{Highlighting}[]
\CommentTok{\# Transformar para UTM (Metros)}
\NormalTok{nc\_utm }\OtherTok{\textless{}{-}} \FunctionTok{st\_transform}\NormalTok{(nc, }\AttributeTok{crs =} \DecValTok{32617}\NormalTok{)}
\FunctionTok{st\_crs}\NormalTok{(nc\_utm)}
\end{Highlighting}
\end{Shaded}

\begin{verbatim}
Coordinate Reference System:
  User input: EPSG:32617 
  wkt:
PROJCRS["WGS 84 / UTM zone 17N",
    BASEGEOGCRS["WGS 84",
        ENSEMBLE["World Geodetic System 1984 ensemble",
            MEMBER["World Geodetic System 1984 (Transit)"],
            MEMBER["World Geodetic System 1984 (G730)"],
            MEMBER["World Geodetic System 1984 (G873)"],
            MEMBER["World Geodetic System 1984 (G1150)"],
            MEMBER["World Geodetic System 1984 (G1674)"],
            MEMBER["World Geodetic System 1984 (G1762)"],
            MEMBER["World Geodetic System 1984 (G2139)"],
            ELLIPSOID["WGS 84",6378137,298.257223563,
                LENGTHUNIT["metre",1]],
            ENSEMBLEACCURACY[2.0]],
        PRIMEM["Greenwich",0,
            ANGLEUNIT["degree",0.0174532925199433]],
        ID["EPSG",4326]],
    CONVERSION["UTM zone 17N",
        METHOD["Transverse Mercator",
            ID["EPSG",9807]],
        PARAMETER["Latitude of natural origin",0,
            ANGLEUNIT["degree",0.0174532925199433],
            ID["EPSG",8801]],
        PARAMETER["Longitude of natural origin",-81,
            ANGLEUNIT["degree",0.0174532925199433],
            ID["EPSG",8802]],
        PARAMETER["Scale factor at natural origin",0.9996,
            SCALEUNIT["unity",1],
            ID["EPSG",8805]],
        PARAMETER["False easting",500000,
            LENGTHUNIT["metre",1],
            ID["EPSG",8806]],
        PARAMETER["False northing",0,
            LENGTHUNIT["metre",1],
            ID["EPSG",8807]]],
    CS[Cartesian,2],
        AXIS["(E)",east,
            ORDER[1],
            LENGTHUNIT["metre",1]],
        AXIS["(N)",north,
            ORDER[2],
            LENGTHUNIT["metre",1]],
    USAGE[
        SCOPE["Navigation and medium accuracy spatial referencing."],
        AREA["Between 84°W and 78°W, northern hemisphere between equator and 84°N, onshore and offshore. Bahamas. Ecuador - north of equator. Canada - Nunavut; Ontario; Quebec. Cayman Islands. Colombia. Costa Rica. Cuba. Jamaica. Nicaragua. Panama. United States (USA)."],
        BBOX[0,-84,84,-78]],
    ID["EPSG",32617]]
\end{verbatim}

\begin{Shaded}
\begin{Highlighting}[]
\CommentTok{\# Transformar para WGS84 (GPS {-} Lat/Long)}
\NormalTok{nc\_gps }\OtherTok{\textless{}{-}} \FunctionTok{st\_transform}\NormalTok{(nc, }\AttributeTok{crs =} \DecValTok{4326}\NormalTok{)}
\FunctionTok{st\_crs}\NormalTok{(nc\_gps)}
\end{Highlighting}
\end{Shaded}

\begin{verbatim}
Coordinate Reference System:
  User input: EPSG:4326 
  wkt:
GEOGCRS["WGS 84",
    ENSEMBLE["World Geodetic System 1984 ensemble",
        MEMBER["World Geodetic System 1984 (Transit)"],
        MEMBER["World Geodetic System 1984 (G730)"],
        MEMBER["World Geodetic System 1984 (G873)"],
        MEMBER["World Geodetic System 1984 (G1150)"],
        MEMBER["World Geodetic System 1984 (G1674)"],
        MEMBER["World Geodetic System 1984 (G1762)"],
        MEMBER["World Geodetic System 1984 (G2139)"],
        ELLIPSOID["WGS 84",6378137,298.257223563,
            LENGTHUNIT["metre",1]],
        ENSEMBLEACCURACY[2.0]],
    PRIMEM["Greenwich",0,
        ANGLEUNIT["degree",0.0174532925199433]],
    CS[ellipsoidal,2],
        AXIS["geodetic latitude (Lat)",north,
            ORDER[1],
            ANGLEUNIT["degree",0.0174532925199433]],
        AXIS["geodetic longitude (Lon)",east,
            ORDER[2],
            ANGLEUNIT["degree",0.0174532925199433]],
    USAGE[
        SCOPE["Horizontal component of 3D system."],
        AREA["World."],
        BBOX[-90,-180,90,180]],
    ID["EPSG",4326]]
\end{verbatim}

\begin{tcolorbox}[enhanced jigsaw, left=2mm, toptitle=1mm, colback=white, colframe=quarto-callout-important-color-frame, colbacktitle=quarto-callout-important-color!10!white, opacityback=0, rightrule=.15mm, bottomtitle=1mm, arc=.35mm, title=\textcolor{quarto-callout-important-color}{\faExclamation}\hspace{0.5em}{st\_crs vs st\_transform}, titlerule=0mm, bottomrule=.15mm, leftrule=.75mm, coltitle=black, toprule=.15mm, breakable, opacitybacktitle=0.6]

\begin{itemize}
\item
  Use \texttt{st\_crs(x)\ \textless{}-\ valor} apenas se os dados não
  tiverem CRS definido ou estiverem errados (isso apenas coloca um
  rótulo).
\item
  Use \texttt{st\_transform(x,\ valor)} se os dados tiverem CRS e
  preterar-o (isso recalcula as coordenadas X e Y).
\end{itemize}

\end{tcolorbox}

\textbf{Confirmação Geométrica}

Estas funções respondem a perguntas de Verdadeiro/Falso sobre a relação
espacial entre duas geometrias (X e Y). Elas retornam, por padrão, uma
matriz esparsa (lista de índices) indicando quais elementos de X se
relacionam com quais de Y.

\begin{itemize}
\item
  \textbf{\texttt{st\_intersects(x,\ y)}:} O mais comum. Retorna
  \texttt{TRUE} se X e Y partilham qualquer ponto no espaço (se tocam,
  cruzam ou sobrepõem).
\item
  \textbf{\texttt{st\_disjoint(x,\ y)}:} O oposto de intersects. Retorna
  \texttt{TRUE} se X e Y estão completamente separados.
\item
  \textbf{\texttt{st\_contains(x,\ y)}:} Retorna \texttt{TRUE} se nenhum
  ponto de Y está fora de X e pelo menos um ponto do interior de Y está
  dentro de X (Y está dentro de X).
\item
  \textbf{\texttt{st\_within(x,\ y)}:} O inverso de contains. X está
  dentro de Y.
\item
  \textbf{\texttt{st\_touches(x,\ y)}:} Retorna \texttt{TRUE} se as
  geometrias se tocam apenas na borda/fronteira, mas os seus interiores
  não se sobrepõem (ex: países vizinhos).
\item
  \textbf{\texttt{st\_crosses(x,\ y)}:} Retorna \texttt{TRUE} se as
  geometrias se cruzam (comum entre linhas ou linha/polígono),
  resultando numa geometria de dimensão inferior à máxima das duas.
\item
  \textbf{\texttt{st\_overlaps(x,\ y)}:} Retorna \texttt{TRUE} se duas
  geometrias da mesma dimensão partilham espaço, mas uma não contém a
  outra completamente.
\item
  \textbf{\texttt{st\_equals(x,\ y)}:} Retorna \texttt{TRUE} se as
  geometrias são topologicamente idênticas (mesma forma e localização),
  independente da ordem dos vértices.
\end{itemize}

\begin{Shaded}
\begin{Highlighting}[]
\NormalTok{ponto\_teste }\OtherTok{\textless{}{-}} \FunctionTok{st\_point}\NormalTok{(}\FunctionTok{c}\NormalTok{(}\SpecialCharTok{{-}}\FloatTok{81.5}\NormalTok{, }\FloatTok{36.4}\NormalTok{)) }\SpecialCharTok{|\textgreater{}} \FunctionTok{st\_sfc}\NormalTok{(}\AttributeTok{crs =} \DecValTok{4267}\NormalTok{)}

\CommentTok{\#}
\NormalTok{resultado }\OtherTok{\textless{}{-}} \FunctionTok{st\_intersects}\NormalTok{(nc, ponto\_teste, }\AttributeTok{sparse =} \ConstantTok{FALSE}\NormalTok{)}

\CommentTok{\# Qual o nome do condado onde o ponto caiu?}
\NormalTok{nc}\SpecialCharTok{$}\NormalTok{NAME[resultado[,}\DecValTok{1}\NormalTok{]]}
\end{Highlighting}
\end{Shaded}

\begin{verbatim}
[1] "Ashe"
\end{verbatim}

\textbf{Operações Geométricas}

Estas funções criam novas geometrias a partir de transformações nas
existentes.

\begin{itemize}
\tightlist
\item
  \textbf{Unárias} (Transformam uma geometria nela mesma)
\end{itemize}

\begin{enumerate}
\def\labelenumi{\arabic{enumi}.}
\item
  \texttt{st\_buffer(x,\ dist)}: Cria um polígono que representa a zona
  de influência de raio \texttt{dist} ao redor da geometria. Fundamental
  para análises de proximidade.
\item
  \texttt{st\_centroid(x)}: Reduz a geometria ao seu centro de massa
  geométrico (ponto).
\item
  \texttt{st\_boundary(x)}: Extrai apenas a linha de fronteira
  (contorno) de um polígono.
\item
  \texttt{st\_convex\_hull(x)}: Cria o menor polígono convexo que
  envolve todos os pontos da geometria (como um elástico esticado ao
  redor de pregos).
\item
  \texttt{st\_simplify(x,\ dTolerance)}: Reduz o número de vértices de
  linhas/polígonos complexos, mantendo a forma geral (algoritmo
  \href{https://en.wikipedia.org/wiki/Ramer\%E2\%80\%93Douglas\%E2\%80\%93Peucker_algorithm}{Douglas-Peucker}).
  Útil para tornar mapas leves para web.
\item
  \texttt{st\_segmentize(x,\ dfMaxLength)}: Adiciona pontos a linhas
  longas para garantir que elas sigam a curvatura da terra ao serem
  reprojetadas.
\end{enumerate}

\begin{Shaded}
\begin{Highlighting}[]
\NormalTok{nc\_utm }\OtherTok{\textless{}{-}} \FunctionTok{st\_transform}\NormalTok{(nc, }\DecValTok{32617}\NormalTok{) }\CommentTok{\#Transformar para metros (UTM)}
\NormalTok{centroides }\OtherTok{\textless{}{-}} \FunctionTok{st\_centroid}\NormalTok{(nc\_utm) }\CommentTok{\#Calcular Centroide}
\NormalTok{buffers }\OtherTok{\textless{}{-}} \FunctionTok{st\_buffer}\NormalTok{(centroides, }\AttributeTok{dist =} \DecValTok{10000}\NormalTok{) }\CommentTok{\# Criar Buffer 10000 metros}

\FunctionTok{ggplot}\NormalTok{() }\SpecialCharTok{+}
  \FunctionTok{geom\_sf}\NormalTok{(}\AttributeTok{data =}\NormalTok{ nc\_utm, }\AttributeTok{fill =} \StringTok{"white"}\NormalTok{) }\SpecialCharTok{+}
    \FunctionTok{geom\_sf}\NormalTok{(}\AttributeTok{data =}\NormalTok{ centroides, }\AttributeTok{fill =} \StringTok{"black"}\NormalTok{, }\AttributeTok{alpha =} \FloatTok{0.3}\NormalTok{) }\SpecialCharTok{+} \CommentTok{\#adiciona o centro de cada poligono}
  \FunctionTok{geom\_sf}\NormalTok{(}\AttributeTok{data =}\NormalTok{ buffers, }\AttributeTok{fill =} \StringTok{"red"}\NormalTok{, }\AttributeTok{alpha =} \FloatTok{0.3}\NormalTok{) }\SpecialCharTok{+} \CommentTok{\# Faz um Buffer 10000 metros ao redor de cada centro}
  \FunctionTok{theme\_void}\NormalTok{()}
\end{Highlighting}
\end{Shaded}

\pandocbounded{\includegraphics[keepaspectratio]{fundEstspatial_files/figure-pdf/unnamed-chunk-15-1.pdf}}

\begin{itemize}
\tightlist
\item
  \textbf{Binárias} (Combinam duas geometrias)
\end{itemize}

Estas são operações de conjuntos:

\begin{enumerate}
\def\labelenumi{\arabic{enumi}.}
\item
  \texttt{st\_intersection(x,\ y)}: Retorna a área comum entre X e Y.
  Preserva os atributos de ambos.
\item
  \texttt{st\_difference(x,\ y)}: Retorna a parte de X que \textbf{não}
  está em Y.
\item
  \texttt{st\_union(x,\ y)}: Quando fornecidos dois objetos, a função
  funde as duas camadas numa única, mantendo todas as geometrias (a soma
  de x e y).
\item
  \texttt{st\_union(x)}: Quando fornecido apenas um objeto, a função
  dissolve as fronteiras internas entre as polígonos adjacentes,
  agregando-os numa única geometria (Multi-Polígono).
\item
  \texttt{st\_sym\_difference(x,\ y)}: (XOR) Retorna as áreas que estão
  em X ou em Y, mas não em ambos (o oposto da intersecção).
\end{enumerate}

\begin{Shaded}
\begin{Highlighting}[]
\CommentTok{\# Criar dois círculos sobrepostos}
\NormalTok{c1 }\OtherTok{\textless{}{-}} \FunctionTok{st\_point}\NormalTok{(}\FunctionTok{c}\NormalTok{(}\DecValTok{0}\NormalTok{, }\DecValTok{0}\NormalTok{)) }\SpecialCharTok{|\textgreater{}} \FunctionTok{st\_buffer}\NormalTok{(}\DecValTok{1}\NormalTok{) }\SpecialCharTok{|\textgreater{}} \FunctionTok{st\_sfc}\NormalTok{() }\SpecialCharTok{|\textgreater{}} \FunctionTok{st\_sf}\NormalTok{(}\AttributeTok{id =} \StringTok{"A"}\NormalTok{)}
\NormalTok{c2 }\OtherTok{\textless{}{-}} \FunctionTok{st\_point}\NormalTok{(}\FunctionTok{c}\NormalTok{(}\DecValTok{1}\NormalTok{, }\DecValTok{0}\NormalTok{)) }\SpecialCharTok{|\textgreater{}} \FunctionTok{st\_buffer}\NormalTok{(}\DecValTok{1}\NormalTok{) }\SpecialCharTok{|\textgreater{}} \FunctionTok{st\_sfc}\NormalTok{() }\SpecialCharTok{|\textgreater{}} \FunctionTok{st\_sf}\NormalTok{(}\AttributeTok{id =} \StringTok{"B"}\NormalTok{)}

\NormalTok{base\_plot }\OtherTok{\textless{}{-}} \FunctionTok{ggplot}\NormalTok{() }\SpecialCharTok{+}
  \FunctionTok{geom\_sf}\NormalTok{(}\AttributeTok{data =}\NormalTok{ c1, }\AttributeTok{fill =} \StringTok{"red"}\NormalTok{, }\AttributeTok{alpha =} \FloatTok{0.5}\NormalTok{) }\SpecialCharTok{+}
  \FunctionTok{geom\_sf}\NormalTok{(}\AttributeTok{data =}\NormalTok{ c2, }\AttributeTok{fill =} \StringTok{"blue"}\NormalTok{, }\AttributeTok{alpha =} \FloatTok{0.5}\NormalTok{) }\SpecialCharTok{+}
  \FunctionTok{geom\_sf\_text}\NormalTok{(}\AttributeTok{data =}\NormalTok{ c1, }\FunctionTok{aes}\NormalTok{(}\AttributeTok{label =} \StringTok{"A"}\NormalTok{), }\AttributeTok{nudge\_x =} \SpecialCharTok{{-}}\FloatTok{0.3}\NormalTok{) }\SpecialCharTok{+}
  \FunctionTok{geom\_sf\_text}\NormalTok{(}\AttributeTok{data =}\NormalTok{ c2, }\FunctionTok{aes}\NormalTok{(}\AttributeTok{label =} \StringTok{"B"}\NormalTok{), }\AttributeTok{nudge\_x =} \FloatTok{0.3}\NormalTok{) }\SpecialCharTok{+}
  \FunctionTok{theme\_void}\NormalTok{() }\SpecialCharTok{+}
  \FunctionTok{ggtitle}\NormalTok{(}\StringTok{"Estado Inicial (A e B)"}\NormalTok{) }\SpecialCharTok{+}
  \FunctionTok{theme}\NormalTok{(}\AttributeTok{plot.title =} \FunctionTok{element\_text}\NormalTok{(}\AttributeTok{hjust =} \FloatTok{0.5}\NormalTok{))}

\CommentTok{\# INTERSECTION }
\NormalTok{inter }\OtherTok{\textless{}{-}} \FunctionTok{st\_intersection}\NormalTok{(c1, c2)}
\NormalTok{p1 }\OtherTok{\textless{}{-}} \FunctionTok{ggplot}\NormalTok{(inter) }\SpecialCharTok{+} \FunctionTok{geom\_sf}\NormalTok{(}\AttributeTok{fill =} \StringTok{"purple"}\NormalTok{, }\AttributeTok{alpha=}\FloatTok{0.8}\NormalTok{) }\SpecialCharTok{+} 
      \FunctionTok{ggtitle}\NormalTok{(}\StringTok{"st\_intersection(A, B)"}\NormalTok{) }\SpecialCharTok{+} \FunctionTok{theme\_void}\NormalTok{()}\SpecialCharTok{+}
  \FunctionTok{theme}\NormalTok{(}\AttributeTok{plot.title =} \FunctionTok{element\_text}\NormalTok{(}\AttributeTok{hjust =} \FloatTok{0.5}\NormalTok{))}

\CommentTok{\#DIFFERENCE}
\NormalTok{dif }\OtherTok{\textless{}{-}} \FunctionTok{st\_difference}\NormalTok{(c1, c2)}
\NormalTok{p2 }\OtherTok{\textless{}{-}} \FunctionTok{ggplot}\NormalTok{(dif) }\SpecialCharTok{+} \FunctionTok{geom\_sf}\NormalTok{(}\AttributeTok{fill =} \StringTok{"red"}\NormalTok{, }\AttributeTok{alpha=}\FloatTok{0.8}\NormalTok{) }\SpecialCharTok{+} 
      \FunctionTok{ggtitle}\NormalTok{(}\StringTok{"st\_difference(A, B)"}\NormalTok{) }\SpecialCharTok{+} \FunctionTok{theme\_void}\NormalTok{()}\SpecialCharTok{+}
  \FunctionTok{theme}\NormalTok{(}\AttributeTok{plot.title =} \FunctionTok{element\_text}\NormalTok{(}\AttributeTok{hjust =} \FloatTok{0.5}\NormalTok{))}

\CommentTok{\#UNION }
\NormalTok{uni }\OtherTok{\textless{}{-}} \FunctionTok{st\_union}\NormalTok{(c1, c2)}
\NormalTok{p3 }\OtherTok{\textless{}{-}} \FunctionTok{ggplot}\NormalTok{(uni) }\SpecialCharTok{+} \FunctionTok{geom\_sf}\NormalTok{(}\AttributeTok{fill =} \StringTok{"darkgreen"}\NormalTok{, }\AttributeTok{alpha=}\FloatTok{0.8}\NormalTok{) }\SpecialCharTok{+} 
      \FunctionTok{ggtitle}\NormalTok{(}\StringTok{"st\_union(A, B)"}\NormalTok{) }\SpecialCharTok{+} \FunctionTok{theme\_void}\NormalTok{()}\SpecialCharTok{+}
  \FunctionTok{theme}\NormalTok{(}\AttributeTok{plot.title =} \FunctionTok{element\_text}\NormalTok{(}\AttributeTok{hjust =} \FloatTok{0.5}\NormalTok{))}

\CommentTok{\#SYM\_DIFFERENCE}
\NormalTok{sym }\OtherTok{\textless{}{-}} \FunctionTok{st\_sym\_difference}\NormalTok{(c1, c2)}
\NormalTok{p4 }\OtherTok{\textless{}{-}} \FunctionTok{ggplot}\NormalTok{(sym) }\SpecialCharTok{+} \FunctionTok{geom\_sf}\NormalTok{(}\AttributeTok{fill =} \StringTok{"orange"}\NormalTok{, }\AttributeTok{alpha=}\FloatTok{0.8}\NormalTok{) }\SpecialCharTok{+} 
      \FunctionTok{ggtitle}\NormalTok{(}\StringTok{"st\_sym\_difference(A, B)"}\NormalTok{) }\SpecialCharTok{+} \FunctionTok{theme\_void}\NormalTok{()}\SpecialCharTok{+}
  \FunctionTok{theme}\NormalTok{(}\AttributeTok{plot.title =} \FunctionTok{element\_text}\NormalTok{(}\AttributeTok{hjust =} \FloatTok{0.5}\NormalTok{))}


\NormalTok{(base\_plot}\SpecialCharTok{+}\NormalTok{p1 }\SpecialCharTok{+}\NormalTok{ p3) }\SpecialCharTok{/}\NormalTok{ (p2 }\SpecialCharTok{+}\NormalTok{ p4)}
\end{Highlighting}
\end{Shaded}

\begin{figure}[H]

\centering{

\pandocbounded{\includegraphics[keepaspectratio]{fundEstspatial_files/figure-pdf/fig-set-ops-results-1.pdf}}

}

\caption{\label{fig-set-ops-results}Operações Geométricas Binárias}

\end{figure}%

\textbf{Medidas Geométricas}

Calculam propriedades métricas das feições. O \texttt{sf} integra-se com
o pacote \texttt{units}, retornando valores com unidades físicas
explícitas (m, km, ha).

\begin{itemize}
\item
  \textbf{\texttt{st\_area(x)}:} Calcula a área de polígonos.
\item
  \textbf{\texttt{st\_length(x)}:} Calcula o comprimento de linhas ou
  perímetros de polígonos.
\item
  \textbf{\texttt{st\_distance(x,\ y)}:} Calcula a distância euclidiana
  mais curta entre dois conjuntos de geometrias (retorna uma matriz de
  distâncias).
\end{itemize}

\begin{Shaded}
\begin{Highlighting}[]
\CommentTok{\# Área}
\NormalTok{area\_sqm }\OtherTok{\textless{}{-}} \FunctionTok{st\_area}\NormalTok{(nc\_utm[}\DecValTok{1}\SpecialCharTok{:}\DecValTok{5}\NormalTok{, ])}
\FunctionTok{print}\NormalTok{(area\_sqm) }\CommentTok{\# Em metros quadrados}
\end{Highlighting}
\end{Shaded}

\begin{verbatim}
Units: [m^2]
[1] 1136550961  610599191 1422398043  697400554 1523431642
\end{verbatim}

\begin{Shaded}
\begin{Highlighting}[]
\CommentTok{\# Converter para km\^{}2 usando pacote units}
\FunctionTok{library}\NormalTok{(units)}
\NormalTok{area\_km2 }\OtherTok{\textless{}{-}} \FunctionTok{set\_units}\NormalTok{(area\_sqm, km}\SpecialCharTok{\^{}}\DecValTok{2}\NormalTok{)}
\FunctionTok{print}\NormalTok{(area\_km2)}
\end{Highlighting}
\end{Shaded}

\begin{verbatim}
Units: [km^2]
[1] 1136.5510  610.5992 1422.3980  697.4006 1523.4316
\end{verbatim}

\textbf{Manipulação de Dados}

O \texttt{sf} permite usar a gramática do pacote \texttt{dplyr} para
manipular a tabela de atributos sem perder a geometria (ver
Seção~\ref{sec-dplyr}).

\begin{itemize}
\item
  \texttt{filter()}: Seleciona linhas (regiões) baseadas em critérios
  lógicos.
\item
  \texttt{select()}: Seleciona colunas. \textbf{Nota:} A coluna de
  geometria é ``pegajosa'' e permanece mesmo que não seja selecionada
  explicitamente.
\item
  \texttt{mutate()}: Cria novas colunas (ex: calcular densidade
  demográfica).
\item
  \texttt{st\_drop\_geometry()}: Remove a parte espacial e retorna um
  \texttt{data.frame} puro. Essencial para análises estatísticas
  convencionais ou exportação para Excel.
\end{itemize}

\textbf{Junções Espaciais (\texttt{st\_join})}

Esta é uma das funções mais poderosas do geoprocessamento no R. Enquanto
o \texttt{left\_join} (Seção~\ref{sec-dplyr}) une duas tabelas
baseando-se numa \textbf{chave comum} (como uma coluna de ID ou CPF), o
\texttt{st\_join} une duas tabelas baseando-se na \textbf{localização
espacial} (geometria).

\begin{itemize}
\tightlist
\item
  \textbf{Sintaxe e Argumentos Principais}
\end{itemize}

\begin{Shaded}
\begin{Highlighting}[]
\FunctionTok{st\_join}\NormalTok{(x, y, }\AttributeTok{join =}\NormalTok{ st\_intersects, ..., }\AttributeTok{left =} \ConstantTok{TRUE}\NormalTok{, }\AttributeTok{largest =} \ConstantTok{FALSE}\NormalTok{)}
\end{Highlighting}
\end{Shaded}

\begin{enumerate}
\def\labelenumi{\arabic{enumi}.}
\item
  \textbf{\texttt{x} (Alvo):} O objeto \texttt{sf} que receberá os
  dados. A geometria resultante será sempre a geometria de \texttt{x}.
\item
  \textbf{\texttt{y} (Fonte):} O objeto \texttt{sf} de onde vêm os novos
  atributos.
\item
  \textbf{\texttt{join}:} A regra geométrica a usar.
\end{enumerate}

\begin{itemize}
\item
  \texttt{st\_intersects} (Padrão): Une se tocarem de qualquer forma.
\item
  \texttt{st\_within}: Une apenas se \texttt{x} estiver totalmente
  dentro de \texttt{y}.
\item
  \texttt{st\_contains}: Une apenas se \texttt{x} contiver \texttt{y}.
\item
  \texttt{st\_is\_within\_distance}: Une se estiverem a uma certa
  distância (requer argumento \texttt{dist}).
\item
  \texttt{left} (Tipo de Junção):
\item
  \texttt{TRUE} (Padrão): Realiza um \texttt{Left\ Join}. Mantém todas
  as linhas de \texttt{x}. Se não houver correspondência espacial em
  \texttt{y}, as colunas novas vêm preenchidas com \texttt{NA}.
\item
  \texttt{FALSE}: Realiza um
  \texttt{Inner\ Join\textasciigrave{}\textasciigrave{}\textasciigrave{}.\ Mantém\ apenas\ as\ linhas\ de}x\texttt{que\ realmente\ intersectam\ com}y`,
  descartando o resto.
\end{itemize}

*\texttt{suffix}: Se houver colunas com nomes iguais em ambas as tabelas
(ex: \texttt{nome}), define o sufixo para desambiguar (ex:
\texttt{nome.x}, \texttt{nome.y}).

\begin{tcolorbox}[enhanced jigsaw, left=2mm, toptitle=1mm, colback=white, colframe=quarto-callout-important-color-frame, colbacktitle=quarto-callout-important-color!10!white, opacityback=0, rightrule=.15mm, bottomtitle=1mm, arc=.35mm, title=\textcolor{quarto-callout-important-color}{\faExclamation}\hspace{0.5em}{Atenção à duplicação de registos}, titlerule=0mm, bottomrule=.15mm, leftrule=.75mm, coltitle=black, toprule=.15mm, breakable, opacitybacktitle=0.6]

Diferente de um \texttt{left\_join} tradicional onde esperamos uma
relação 1:1, no espaço é comum haver sobreposições. Se um ponto
\texttt{x} cair exatamente na fronteira entre dois polígonos \texttt{y}
(ou numa área onde dois polígonos se sobrepõem), o \texttt{st\_join}
duplicará o ponto \texttt{x}. O resultado terá duas linhas para aquele
ponto: uma com os dados do polígono A e outra com os dados do polígono
B.

Sempre verifique o número de linhas (\texttt{nrow}) antes e depois do
\texttt{st\_join}. Se aumentou, houve duplicatas espaciais.

\end{tcolorbox}

\begin{itemize}
\tightlist
\item
  \textbf{Junção por Distância} (\texttt{st\_is\_within\_distance})
\end{itemize}

Às vezes, os objetos não se tocam, mas queremos unir pela proximidade
(ex: atribuir a uma casa os dados da estação de metro num raio de 500m).
Para isso, alteramos o argumento \texttt{join} e fornecemos a distância.

\begin{Shaded}
\begin{Highlighting}[]
\FunctionTok{p\_load}\NormalTok{(sf, dplyr)}

\NormalTok{p1 }\OtherTok{\textless{}{-}} \FunctionTok{st\_polygon}\NormalTok{(}\FunctionTok{list}\NormalTok{(}\FunctionTok{rbind}\NormalTok{(}\FunctionTok{c}\NormalTok{(}\DecValTok{0}\NormalTok{,}\DecValTok{0}\NormalTok{), }\FunctionTok{c}\NormalTok{(}\DecValTok{2}\NormalTok{,}\DecValTok{0}\NormalTok{), }\FunctionTok{c}\NormalTok{(}\DecValTok{2}\NormalTok{,}\DecValTok{2}\NormalTok{), }\FunctionTok{c}\NormalTok{(}\DecValTok{0}\NormalTok{,}\DecValTok{2}\NormalTok{), }\FunctionTok{c}\NormalTok{(}\DecValTok{0}\NormalTok{,}\DecValTok{0}\NormalTok{))))}
\NormalTok{p2 }\OtherTok{\textless{}{-}} \FunctionTok{st\_polygon}\NormalTok{(}\FunctionTok{list}\NormalTok{(}\FunctionTok{rbind}\NormalTok{(}\FunctionTok{c}\NormalTok{(}\DecValTok{2}\NormalTok{,}\DecValTok{0}\NormalTok{), }\FunctionTok{c}\NormalTok{(}\DecValTok{4}\NormalTok{,}\DecValTok{0}\NormalTok{), }\FunctionTok{c}\NormalTok{(}\DecValTok{4}\NormalTok{,}\DecValTok{2}\NormalTok{), }\FunctionTok{c}\NormalTok{(}\DecValTok{2}\NormalTok{,}\DecValTok{2}\NormalTok{), }\FunctionTok{c}\NormalTok{(}\DecValTok{2}\NormalTok{,}\DecValTok{0}\NormalTok{))))}
\NormalTok{parques }\OtherTok{\textless{}{-}} \FunctionTok{st\_sf}\NormalTok{(}\AttributeTok{nome\_parque =} \FunctionTok{c}\NormalTok{(}\StringTok{"Parque A"}\NormalTok{, }\StringTok{"Parque B"}\NormalTok{), }
                 \AttributeTok{gestor =} \FunctionTok{c}\NormalTok{(}\StringTok{"Prefeitura"}\NormalTok{, }\StringTok{"Estado"}\NormalTok{),}
                 \AttributeTok{geometry =} \FunctionTok{st\_sfc}\NormalTok{(p1, p2))}

\CommentTok{\#}
\FunctionTok{set.seed}\NormalTok{(}\DecValTok{123}\NormalTok{)}
\NormalTok{arvores }\OtherTok{\textless{}{-}} \FunctionTok{st\_as\_sf}\NormalTok{(}\FunctionTok{data.frame}\NormalTok{(}
  \AttributeTok{id\_arvore =} \DecValTok{1}\SpecialCharTok{:}\DecValTok{5}\NormalTok{,}
  \AttributeTok{x =} \FunctionTok{runif}\NormalTok{(}\DecValTok{5}\NormalTok{, }\DecValTok{0}\NormalTok{, }\DecValTok{5}\NormalTok{), }
  \AttributeTok{y =} \FunctionTok{runif}\NormalTok{(}\DecValTok{5}\NormalTok{, }\DecValTok{0}\NormalTok{, }\DecValTok{3}\NormalTok{)}
\NormalTok{), }\AttributeTok{coords =} \FunctionTok{c}\NormalTok{(}\StringTok{"x"}\NormalTok{, }\StringTok{"y"}\NormalTok{))}

\CommentTok{\#}
\NormalTok{resultado }\OtherTok{\textless{}{-}} \FunctionTok{st\_join}\NormalTok{(arvores, parques) }\CommentTok{\# usou por padrão \textasciigrave{}st\_intersects\textasciigrave{}}

\FunctionTok{print}\NormalTok{(resultado)}
\end{Highlighting}
\end{Shaded}

\begin{verbatim}
Simple feature collection with 5 features and 3 fields
Geometry type: POINT
Dimension:     XY
Bounding box:  xmin: 1.437888 ymin: 0.1366695 xmax: 4.702336 ymax: 2.677257
CRS:           NA
  id_arvore nome_parque     gestor                   geometry
1         1    Parque A Prefeitura POINT (1.437888 0.1366695)
2         2    Parque B     Estado  POINT (3.941526 1.584316)
3         3        <NA>       <NA>  POINT (2.044885 2.677257)
4         4        <NA>       <NA>  POINT (4.415087 1.654305)
5         5        <NA>       <NA>  POINT (4.702336 1.369844)
\end{verbatim}

\textbf{Junção pelo Vizinho Mais Próximo}

O \texttt{st\_join} padrão não resolve a pergunta ``Qual é a farmácia
mais próxima desta casa?'', ele apenas resolve ``Esta casa toca numa
farmácia?''. Para encontrar o vizinho mais próximo
(\texttt{Nearest\ Neighbor}) e trazer seus dados, usamos uma combinação
especial:

\begin{Shaded}
\begin{Highlighting}[]
\CommentTok{\# Une x ao y baseando{-}se em quem está mais perto (mesmo que longe)}
\FunctionTok{st\_join}\NormalTok{(x, y, }\AttributeTok{join =}\NormalTok{ st\_nearest\_feature)}
\end{Highlighting}
\end{Shaded}

\textbf{Visualização (\texttt{ggplot2})}

O \texttt{sf} fornece a geometria \texttt{geom\_sf()}, que simplifica a
cartografia no R.

\begin{Shaded}
\begin{Highlighting}[]
\FunctionTok{ggplot}\NormalTok{() }\SpecialCharTok{+}
  \CommentTok{\# Camada 1: Mapa base}
  \FunctionTok{geom\_sf}\NormalTok{(}\AttributeTok{data =}\NormalTok{ nc, }\FunctionTok{aes}\NormalTok{(}\AttributeTok{fill =}\NormalTok{ BIR74), }\AttributeTok{color =} \StringTok{"white"}\NormalTok{, }\AttributeTok{size=}\FloatTok{0.2}\NormalTok{) }\SpecialCharTok{+}
  \CommentTok{\# Escala de cores}
  \FunctionTok{scale\_fill\_viridis\_c}\NormalTok{(}\AttributeTok{name =} \StringTok{"Nascimentos (1974)"}\NormalTok{) }\SpecialCharTok{+}
  \CommentTok{\# Camada 2: Centroides}
  \FunctionTok{geom\_sf}\NormalTok{(}\AttributeTok{data =} \FunctionTok{st\_centroid}\NormalTok{(nc), }\AttributeTok{size =} \DecValTok{1}\NormalTok{, }\AttributeTok{color =} \StringTok{"black"}\NormalTok{) }\SpecialCharTok{+}
  \CommentTok{\# Controlo de Coordenadas e Zoom}
  \FunctionTok{coord\_sf}\NormalTok{(}\AttributeTok{datum =} \FunctionTok{st\_crs}\NormalTok{(}\DecValTok{4326}\NormalTok{)) }\SpecialCharTok{+} \CommentTok{\# Força grid em Lat/Long}
  \FunctionTok{theme\_minimal}\NormalTok{() }\SpecialCharTok{+}
  \FunctionTok{labs}\NormalTok{(}\AttributeTok{title =} \StringTok{"Nascimentos na Carolina do Norte"}\NormalTok{)}
\end{Highlighting}
\end{Shaded}

\pandocbounded{\includegraphics[keepaspectratio]{fundEstspatial_files/figure-pdf/unnamed-chunk-18-1.pdf}}

\textbf{Outras Funções Úteis}

\begin{itemize}
\item
  \texttt{st\_bbox(x)}: Retorna a caixa delimitadora (\texttt{xmin},
  \texttt{ymin}, \texttt{xmax}, \texttt{ymax}) do seu shapfile. Útil
  para definir limites de gráficos (\texttt{xlim}, \texttt{ylim}).
\item
  \texttt{st\_make\_grid(x,\ cellsize)}: Cria uma grade retangular ou
  hexagonal cobrindo a área de X. Base para análises raster ou
  amostragem sistemática.
\item
  \texttt{st\_cast(x,\ to)}: Converte tipos de geometria.
\end{itemize}

\begin{enumerate}
\def\labelenumi{\arabic{enumi}.}
\item
  Ex: \texttt{st\_cast(poligono,\ "LINESTRING")} converte a borda do
  polígono em linha.
\item
  Ex: \texttt{st\_cast(multipoligono,\ "POLYGON")} converte multiplos
  polígonos em polígonos individuais (aumenta o número de linhas).
\end{enumerate}

\begin{itemize}
\tightlist
\item
  \textbf{\texttt{st\_make\_valid(x)}:} Corrige erros topológicos (ex:
  polígonos com laços, auto-intersecções) que frequentemente causam
  erros em operações como \texttt{st\_intersection}.
\end{itemize}

\begin{Shaded}
\begin{Highlighting}[]
\CommentTok{\# Grid Hexagonal sobre a Carolina do Norte}
\NormalTok{grid\_hex }\OtherTok{\textless{}{-}} \FunctionTok{st\_make\_grid}\NormalTok{(nc, }\AttributeTok{n =} \FunctionTok{c}\NormalTok{(}\DecValTok{20}\NormalTok{, }\DecValTok{20}\NormalTok{), }\AttributeTok{square =} \ConstantTok{FALSE}\NormalTok{) }\CommentTok{\# square=FALSE faz hexágonos}

\FunctionTok{ggplot}\NormalTok{() }\SpecialCharTok{+}
  \FunctionTok{geom\_sf}\NormalTok{(}\AttributeTok{data =}\NormalTok{ nc, }\AttributeTok{fill =} \ConstantTok{NA}\NormalTok{, }\AttributeTok{color =} \StringTok{"blue"}\NormalTok{) }\SpecialCharTok{+}
  \FunctionTok{geom\_sf}\NormalTok{(}\AttributeTok{data =}\NormalTok{ grid\_hex, }\AttributeTok{fill =} \ConstantTok{NA}\NormalTok{, }\AttributeTok{color =} \StringTok{"gray"}\NormalTok{) }\SpecialCharTok{+}
  \FunctionTok{theme\_void}\NormalTok{()}
\end{Highlighting}
\end{Shaded}

\pandocbounded{\includegraphics[keepaspectratio]{fundEstspatial_files/figure-pdf/unnamed-chunk-19-1.pdf}}

\section{Pacote geobr}\label{pacote-geobr}

Enquanto o pacote \texttt{sf} fornece as ferramentas para manipular
geometrias, o pacote \texttt{geobr} Pereira e Goncalves (2024) fornece
os dados. Desenvolvido por uma equipe liderada por pesquisadores do
\href{https://www.ipea.gov.br/portal/}{Ipea}
(\href{https://scholar.google.co.uk/citations?user=dbRivsEAAAAJ&hl=en}{Rafael
H. M. Pereira}), o \texttt{geobr} é a biblioteca mais robusta para
baixar bases de dados espaciais oficiais do Brasil.

O seu grande diferencial é resolver os maiores problemas de quem
trabalha com dados do IBGE:

\begin{itemize}
\item
  Todos os dados são baixados já projetados no CRS oficial (SIRGAS 2000
  - EPSG 4674).
\item
  Os dados vêm diretamente como objetos \texttt{sf} (Simple Features),
  prontos para análise no R.
\item
  Resolve automaticamente problemas de geometrias inválidas que
  frequentemente aparecem em shapefiles brutos baixados manualmente.
\item
  Permite baixar malhas territoriais de anos anteriores (ex: municípios
  como existiam em 1872, 1991, 2010, etc.), essencial para análises
  temporais consistentes.
\end{itemize}

\begin{enumerate}
\def\labelenumi{\arabic{enumi}.}
\tightlist
\item
  \textbf{Instalação e Carregamento}
\end{enumerate}

\begin{Shaded}
\begin{Highlighting}[]
\ControlFlowTok{if}\NormalTok{ (}\SpecialCharTok{!}\FunctionTok{require}\NormalTok{(}\StringTok{"pacman"}\NormalTok{)) }\FunctionTok{install.packages}\NormalTok{(}\StringTok{"pacman"}\NormalTok{)}
\NormalTok{pacman}\SpecialCharTok{::}\FunctionTok{p\_load}\NormalTok{(geobr, sf, ggplot2, dplyr)}
\end{Highlighting}
\end{Shaded}

\begin{enumerate}
\def\labelenumi{\arabic{enumi}.}
\setcounter{enumi}{1}
\tightlist
\item
  \textbf{Argumentos Comuns}
\end{enumerate}

\begin{itemize}
\item
  \texttt{year}: Define o ano de referência da malha. Se omitido,
  geralmente baixa o último disponível. Ex: \texttt{year\ =\ 2010}
  (Censo), \texttt{year\ =\ 2020}.
\item
  \textbf{\texttt{simplified}}: Um argumento lógico (\texttt{TRUE} ou
  \texttt{FALSE}):
\item
  \texttt{TRUE} (Padrão): Retorna uma geometria simplificada (menos
  vértices). É muito mais leve e rápido para carregar e plotar, ideal
  para visualização em mapas nacionais ou regionais. A simplificação usa
  o \texttt{st\_simplify} preservando a topologia.
\item
  \texttt{FALSE}: Retorna a geometria original com resolução máxima.
  Obrigatório se você for fazer cálculos de precisão (áreas, perímetros)
  ou análises de fronteira muito detalhadas.
\item
  \texttt{showProgress}: Mostra a barra de progresso do download
  (\texttt{TRUE}/\texttt{FALSE}).
\item
  \texttt{cache}: Se \texttt{TRUE} (padrão), salva o arquivo numa pasta
  temporária do seu computador. Se você rodar o comando novamente na
  mesma sessão, ele lê do disco em vez de baixar da internet novamente,
  economizando tempo.
\end{itemize}

\begin{enumerate}
\def\labelenumi{\arabic{enumi}.}
\setcounter{enumi}{2}
\tightlist
\item
  \textbf{Divisões Administrativas}
\end{enumerate}

\begin{itemize}
\tightlist
\item
  \textbf{Estados (\texttt{read\_state})}
\end{itemize}

Baixa a geometria das Unidades da Federação.

\begin{itemize}
\tightlist
\item
  \texttt{code\_state}: Pode ser o código numérico (ex: \texttt{33} para
  RJ), a sigla (\texttt{"RJ"}) ou \texttt{"all"} para baixar o Brasil
  inteiro.
\end{itemize}

\begin{Shaded}
\begin{Highlighting}[]
\FunctionTok{p\_load}\NormalTok{(geobr)}
\CommentTok{\# Ler todos os estados em 2010 (Ano de Censo)}
\NormalTok{estados }\OtherTok{\textless{}{-}} \FunctionTok{read\_state}\NormalTok{(}\AttributeTok{code\_state =} \StringTok{"all"}\NormalTok{, }\AttributeTok{year =} \DecValTok{2010}\NormalTok{, }\AttributeTok{showProgress =} \ConstantTok{FALSE}\NormalTok{)}

\CommentTok{\# Ler apenas São Paulo (código "SP")}
\NormalTok{SP }\OtherTok{\textless{}{-}} \FunctionTok{read\_state}\NormalTok{(}\AttributeTok{code\_state =} \StringTok{"SP"}\NormalTok{, }\AttributeTok{year =} \DecValTok{2010}\NormalTok{, }\AttributeTok{showProgress =} \ConstantTok{FALSE}\NormalTok{)}

\FunctionTok{ggplot}\NormalTok{() }\SpecialCharTok{+}
  \FunctionTok{geom\_sf}\NormalTok{(}\AttributeTok{data =}\NormalTok{ estados, }\AttributeTok{fill =} \StringTok{"white"}\NormalTok{, }\AttributeTok{color =} \StringTok{"gray50"}\NormalTok{) }\SpecialCharTok{+}
  \FunctionTok{geom\_sf}\NormalTok{(}\AttributeTok{data =}\NormalTok{ SP, }\AttributeTok{fill =} \StringTok{"orange"}\NormalTok{, }\AttributeTok{color =} \StringTok{"black"}\NormalTok{) }\SpecialCharTok{+}
  \FunctionTok{theme\_minimal}\NormalTok{() }\SpecialCharTok{+} \CommentTok{\# vc pode trocar aqui pelo que deseja}
  \FunctionTok{labs}\NormalTok{(}\AttributeTok{title =} \StringTok{"Mapa do Brasil com destaque para o São Paulo"}\NormalTok{)}
\end{Highlighting}
\end{Shaded}

\pandocbounded{\includegraphics[keepaspectratio]{fundEstspatial_files/figure-pdf/unnamed-chunk-21-1.pdf}}

\textbf{Recortes Regionais Específicos}

O IBGE e outros órgãos definem regiões que não seguem necessariamente
limites estaduais clássicos. O \texttt{geobr} facilita o acesso a esses
recortes oficiais.

\begin{itemize}
\tightlist
\item
  \textbf{Semiárido Brasileiro (\texttt{read\_semiarid})}
\end{itemize}

Esta função baixa a delimitação oficial do Semiárido Brasileiro. Este
conjunto de dados é fundamental para estudos de seca, desenvolvimento
regional e políticas públicas.

\begin{Shaded}
\begin{Highlighting}[]
\NormalTok{semiarido }\OtherTok{\textless{}{-}} \FunctionTok{read\_semiarid}\NormalTok{(}\AttributeTok{year =} \DecValTok{2017}\NormalTok{, }\AttributeTok{showProgress =} \ConstantTok{FALSE}\NormalTok{)}

\FunctionTok{ggplot}\NormalTok{() }\SpecialCharTok{+}
  \FunctionTok{geom\_sf}\NormalTok{(}\AttributeTok{data =}\NormalTok{ estados, }\AttributeTok{fill =} \StringTok{"gray95"}\NormalTok{, }\AttributeTok{color =} \StringTok{"gray"}\NormalTok{) }\SpecialCharTok{+}
  \FunctionTok{geom\_sf}\NormalTok{(}\AttributeTok{data =}\NormalTok{ semiarido, }\AttributeTok{fill =} \StringTok{"red"}\NormalTok{, }\AttributeTok{alpha =} \FloatTok{0.4}\NormalTok{, }\AttributeTok{color =} \ConstantTok{NA}\NormalTok{) }\SpecialCharTok{+}
  \FunctionTok{theme\_void}\NormalTok{() }\SpecialCharTok{+}
  \FunctionTok{labs}\NormalTok{(}\AttributeTok{title =} \StringTok{"Delimitação Oficial do Semiárido (2017)"}\NormalTok{)}
\end{Highlighting}
\end{Shaded}

\pandocbounded{\includegraphics[keepaspectratio]{fundEstspatial_files/figure-pdf/unnamed-chunk-22-1.pdf}}

\textbf{Dinâmica Urbana}

Para estudos de urbanismo e morfologia das cidades, o pacote oferece
dados que vão além do limite municipal administrativo.

\begin{itemize}
\tightlist
\item
  \textbf{\href{https://metadados.snirh.gov.br/geonetwork/srv/api/records/a426906b-077c-43ec-a82c-200d65acc83d}{Mancha
  Urbanizada} (\texttt{read\_urban\_area})}
\end{itemize}

Traz as áreas efetivamente urbanizadas, baseada em imagens de satélite e
classificação do IBGE.

\begin{Shaded}
\begin{Highlighting}[]
\NormalTok{urbano\_rj }\OtherTok{\textless{}{-}} \FunctionTok{read\_urban\_area}\NormalTok{(}\AttributeTok{year =} \DecValTok{2015}\NormalTok{, }\AttributeTok{code\_state =} \StringTok{"RJ"}\NormalTok{, }\AttributeTok{simplified =} \ConstantTok{FALSE}\NormalTok{, }\AttributeTok{showProgress =} \ConstantTok{FALSE}\NormalTok{)}
\NormalTok{estado\_rj }\OtherTok{\textless{}{-}} \FunctionTok{read\_state}\NormalTok{(}\AttributeTok{code\_state =} \StringTok{"RJ"}\NormalTok{, }\AttributeTok{showProgress =} \ConstantTok{FALSE}\NormalTok{)}

\FunctionTok{ggplot}\NormalTok{() }\SpecialCharTok{+}
  \FunctionTok{geom\_sf}\NormalTok{(}\AttributeTok{data =}\NormalTok{ estado\_rj, }\AttributeTok{fill =} \StringTok{"white"}\NormalTok{) }\SpecialCharTok{+}
  \FunctionTok{geom\_sf}\NormalTok{(}\AttributeTok{data =}\NormalTok{ urbano\_rj, }\AttributeTok{fill =} \StringTok{"darkblue"}\NormalTok{, }\AttributeTok{color =} \ConstantTok{NA}\NormalTok{) }\SpecialCharTok{+}
  \FunctionTok{theme\_void}\NormalTok{() }\SpecialCharTok{+}
  \FunctionTok{labs}\NormalTok{(}\AttributeTok{title =} \StringTok{"Manchas Urbanizadas no RJ (2015)"}\NormalTok{)}
\end{Highlighting}
\end{Shaded}

\pandocbounded{\includegraphics[keepaspectratio]{fundEstspatial_files/figure-pdf/unnamed-chunk-23-1.pdf}}

\begin{itemize}
\tightlist
\item
  \textbf{Concentrações Urbanas (\texttt{read\_urban\_concentrations})}
\end{itemize}

Foca em Arranjos Populacionais. Identifica agrupamentos de municípios
com forte integração (deslocamento para trabalho/estudo), definindo
grandes cidades que ultrapassam fronteiras municipais isoladas. É
essencial para entender metropolização.

\begin{Shaded}
\begin{Highlighting}[]
\NormalTok{concentracoes }\OtherTok{\textless{}{-}} \FunctionTok{read\_urban\_concentrations}\NormalTok{(}\AttributeTok{year =} \DecValTok{2015}\NormalTok{)}
\end{Highlighting}
\end{Shaded}

\textbf{Dados Censitários e Estatísticos}

Para quem trabalha com dados do Censo Demográfico em escalas
inframunicipais (dentro da cidade).

\begin{itemize}
\tightlist
\item
  \textbf{Áreas de Ponderação (\texttt{read\_weighting\_area})}
\end{itemize}

As
\href{https://web.centrodametropole.fflch.usp.br/foruns/index.php/pt/component/content/article/16-conceitos/32-area-de-ponderacao}{Áreas
de Ponderação} são a menor unidade geográfica para a qual o IBGE divulga
os dados da amostra do Censo (o questionário completo). São agregados de
setores censitários.

\begin{itemize}
\tightlist
\item
  \textbf{\texttt{code\_weighting}}: Você pode passar o código do
  município (para baixar todas as áreas daquela cidade) ou do estado.
\end{itemize}

\begin{Shaded}
\begin{Highlighting}[]
\NormalTok{ap\_sobral }\OtherTok{\textless{}{-}} \FunctionTok{read\_weighting\_area}\NormalTok{(}\AttributeTok{code\_weighting =} \DecValTok{2312908}\NormalTok{, }\AttributeTok{year =} \DecValTok{2010}\NormalTok{, }\AttributeTok{showProgress =} \ConstantTok{FALSE}\NormalTok{)}

\FunctionTok{ggplot}\NormalTok{() }\SpecialCharTok{+}
  \FunctionTok{geom\_sf}\NormalTok{(}\AttributeTok{data =}\NormalTok{ ap\_sobral, }\AttributeTok{fill =} \StringTok{"white"}\NormalTok{, }\AttributeTok{color =} \StringTok{"blue"}\NormalTok{) }\SpecialCharTok{+}
  \FunctionTok{theme\_void}\NormalTok{() }\SpecialCharTok{+}
  \FunctionTok{labs}\NormalTok{(}\AttributeTok{title =} \StringTok{"Áreas de Ponderação: Sobral/CE (2010)"}\NormalTok{)}
\end{Highlighting}
\end{Shaded}

\pandocbounded{\includegraphics[keepaspectratio]{fundEstspatial_files/figure-pdf/unnamed-chunk-25-1.pdf}}

\begin{itemize}
\tightlist
\item
  *Grade Estatística (\texttt{read\_statistical\_grid})**
\end{itemize}

A Grade Estatística é uma solução para comparar dados ao longo do tempo
sem sofrer com a mudança das fronteiras dos setores censitários. O IBGE
divide o Brasil em células de grade (ex: 1km x 1km ou 200m x 200m em
áreas urbanas).

\begin{itemize}
\item
  \textbf{\texttt{code\_grid}}: Pode ser a abreviação do estado (ex:
  ``DF''). Se \texttt{code\_grid="all"}, baixa o Brasil todo (Cuidado:
  arquivo muito pesado).
\item
  \textbf{Nota:} A grade geralmente requer poder computacional maior
  devido ao número de polígonos.
\end{itemize}

\begin{Shaded}
\begin{Highlighting}[]
\NormalTok{grid\_df }\OtherTok{\textless{}{-}} \FunctionTok{read\_statistical\_grid}\NormalTok{(}\AttributeTok{code\_grid =} \DecValTok{53}\NormalTok{, }\AttributeTok{year =} \DecValTok{2010}\NormalTok{, }\AttributeTok{showProgress =} \ConstantTok{FALSE}\NormalTok{)}

\FunctionTok{plot}\NormalTok{(}\FunctionTok{st\_geometry}\NormalTok{(grid\_df))}
\end{Highlighting}
\end{Shaded}

\begin{tcolorbox}[enhanced jigsaw, left=2mm, toptitle=1mm, colback=white, colframe=quarto-callout-important-color-frame, colbacktitle=quarto-callout-important-color!10!white, opacityback=0, rightrule=.15mm, bottomtitle=1mm, arc=.35mm, title=\textcolor{quarto-callout-important-color}{\faExclamation}\hspace{0.5em}{Dependência de Conexão}, titlerule=0mm, bottomrule=.15mm, leftrule=.75mm, coltitle=black, toprule=.15mm, breakable, opacitybacktitle=0.6]

Como o pacote \texttt{geobr} baixa dados diretamente dos servidores do
Ipea/IBGE, você precisa de uma conexão ativa com a internet.

Se você estiver rodando um script pesado que baixa muitas coisas,
ocasionalmente o servidor pode falhar (\texttt{timeout}). O argumento
\texttt{cache\ =\ TRUE} é seu melhor amigo aqui: uma vez baixado com
sucesso, o dado fica salvo na sua máquina temporariamente, evitando
downloads repetidos e falhas de conexão.

\end{tcolorbox}

\subsection{Pacote geodata}\label{pacote-geodata}

O pacote \texttt{geodata} Hijmans (2025) oferece acesso direto a
repositórios científicos globais, como o
\href{https://www.worldclim.org/}{WorldClim} (clima),
\href{https://soilgrids.org/}{SoilGrids} (solos),
\href{https://gadm.org/}{GADM} (fronteiras) e
\href{https://www.gbif.org/}{GBIF} (biodiversidade).

A principal vantagem deste pacote é a padronização: ele baixa,
descompacta e carrega os dados diretamente no R.

\textbf{Instalação e Carregamento}

O pacote está no CRAN. Como ele trabalha com dados raster e vetoriais
otimizados, recomenda-se carregar também o pacote
\texttt{{[}terra{]}(https://rspatial.org/pkg/terraPackage.pdf)}.

\begin{Shaded}
\begin{Highlighting}[]
\ControlFlowTok{if}\NormalTok{ (}\SpecialCharTok{!}\FunctionTok{require}\NormalTok{(}\StringTok{"pacman"}\NormalTok{)) }\FunctionTok{install.packages}\NormalTok{(}\StringTok{"pacman"}\NormalTok{)}
\NormalTok{pacman}\SpecialCharTok{::}\FunctionTok{p\_load}\NormalTok{(geodata, terra, ggplot2, tidyterra)}
\end{Highlighting}
\end{Shaded}

\begin{tcolorbox}[enhanced jigsaw, left=2mm, toptitle=1mm, colback=white, colframe=quarto-callout-note-color-frame, colbacktitle=quarto-callout-note-color!10!white, opacityback=0, rightrule=.15mm, bottomtitle=1mm, arc=.35mm, title=\textcolor{quarto-callout-note-color}{\faInfo}\hspace{0.5em}{Caminho dos Arquivos (Path)}, titlerule=0mm, bottomrule=.15mm, leftrule=.75mm, coltitle=black, toprule=.15mm, breakable, opacitybacktitle=0.6]

Quase todas as funções do \texttt{geodata} exigem o argumento
\texttt{path}. Este é o local no seu computador onde os arquivos (muitas
vezes pesados) serão salvos.

\begin{itemize}
\item
  Para testes rápidos, pode usar \texttt{path\ =\ tempdir()} (pasta
  temporária que apaga ao fechar o R).
\item
  Para projetos reais, defina uma pasta fixa (ex:
  \texttt{path\ =\ "dados/"}) para evitar baixar a mesma coisa várias
  vezes.
\end{itemize}

\end{tcolorbox}

\begin{enumerate}
\def\labelenumi{\arabic{enumi}.}
\tightlist
\item
  \textbf{Fronteiras Administrativas (\texttt{gadm})}
\end{enumerate}

Enquanto a função \texttt{world()} baixa um mapa mundi simplificado, a
função \texttt{gadm()} acessa o banco de dados de alta resolução das
Áreas Administrativas Globais (GADM). Ela permite baixar os limites de
um país específico e suas subdivisões internas.

\begin{itemize}
\item
  \textbf{\texttt{country}}: Código ISO-3 do país (ex: ``MOZ'' para
  Moçambique, ``BRA'' para Brasil, ``AGO'' para Angola, etc.).
\item
  \textbf{\texttt{level}}: Nível de detalhe administrativo:
\item
  \texttt{0}: Fronteira do país.
\item
  \texttt{1}: Províncias/Estados.
\item
  \texttt{2}: Distritos/Municípios.
\item
  \texttt{3}: Postos administrativos (quando disponível).
\end{itemize}

\begin{Shaded}
\begin{Highlighting}[]
\CommentTok{\# Baixar limites de Moçambique (Nível 1 {-} Províncias)}
\NormalTok{moz\_prov }\OtherTok{\textless{}{-}}\NormalTok{ geodata}\SpecialCharTok{::}\FunctionTok{gadm}\NormalTok{(}\AttributeTok{country =} \StringTok{"MOZ"}\NormalTok{, }\AttributeTok{level =} \DecValTok{1}\NormalTok{, }\AttributeTok{path =} \FunctionTok{tempdir}\NormalTok{())}

\CommentTok{\# O objeto é um SpatVector. Vamos converter para sf para usar no ggplot}
\NormalTok{moz\_sf }\OtherTok{\textless{}{-}} \FunctionTok{st\_as\_sf}\NormalTok{(moz\_prov)}

\FunctionTok{ggplot}\NormalTok{() }\SpecialCharTok{+}
  \FunctionTok{geom\_sf}\NormalTok{(}\AttributeTok{data =}\NormalTok{ moz\_sf, }\FunctionTok{aes}\NormalTok{(}\AttributeTok{fill =}\NormalTok{ NAME\_1), }\AttributeTok{color =} \StringTok{"white"}\NormalTok{, }\AttributeTok{show.legend =} \ConstantTok{FALSE}\NormalTok{) }\SpecialCharTok{+}
  \FunctionTok{geom\_sf\_text}\NormalTok{(}\AttributeTok{data =}\NormalTok{ moz\_sf, }\FunctionTok{aes}\NormalTok{(}\AttributeTok{label =}\NormalTok{ NAME\_1), }\AttributeTok{size =} \FloatTok{2.5}\NormalTok{, }\AttributeTok{color =} \StringTok{"black"}\NormalTok{) }\SpecialCharTok{+}
  \FunctionTok{scale\_fill\_viridis\_d}\NormalTok{(}\AttributeTok{option =} \StringTok{"mako"}\NormalTok{, }\AttributeTok{alpha =} \FloatTok{0.8}\NormalTok{) }\SpecialCharTok{+}
  \FunctionTok{theme\_void}\NormalTok{() }\SpecialCharTok{+}
  \FunctionTok{labs}\NormalTok{(}\AttributeTok{title =} \StringTok{"Províncias de Moçambique"}\NormalTok{, }\AttributeTok{caption =} \StringTok{"Fonte: GADM via pacote geodata"}\NormalTok{)}
\end{Highlighting}
\end{Shaded}

\begin{Shaded}
\begin{Highlighting}[]
\FunctionTok{tryCatch}\NormalTok{(\{  }
\NormalTok{  moz\_data\_spat }\OtherTok{\textless{}{-}} \FunctionTok{read\_sf}\NormalTok{(}\StringTok{"/home/almonha/Downloads/Curso de Verão/moz\_adm/moz\_admbnda\_adm1\_ine\_20190607.shp"}\NormalTok{) }
\NormalTok{  \}, }\AttributeTok{error =} \ControlFlowTok{function}\NormalTok{(e) \{}
    \FunctionTok{message}\NormalTok{(}\StringTok{"O link não funcionou vou baixar pelo GADM."}\NormalTok{)}
\NormalTok{moz\_data\_spat }\OtherTok{\textless{}{-}}\NormalTok{ geodata}\SpecialCharTok{::}\FunctionTok{gadm}\NormalTok{(}\AttributeTok{country =} \StringTok{"MOZ"}\NormalTok{, }\AttributeTok{level =} \DecValTok{2}\NormalTok{, }\AttributeTok{path =} \FunctionTok{tempdir}\NormalTok{(), }\AttributeTok{version=}\StringTok{"latest"}\NormalTok{)}
\NormalTok{\}) }


\NormalTok{moz\_sf }\OtherTok{\textless{}{-}}\NormalTok{ sf}\SpecialCharTok{::}\FunctionTok{st\_as\_sf}\NormalTok{(moz\_data\_spat)}
\end{Highlighting}
\end{Shaded}

\begin{enumerate}
\def\labelenumi{\arabic{enumi}.}
\setcounter{enumi}{1}
\tightlist
\item
  \textbf{Topografia e Elevação (\texttt{elevation\_30s})}
\end{enumerate}

A elevação é uma covariável fundamental em modelos espaciais. A função
\texttt{elevation\_30s} baixa o Modelo Digital de Elevação (DEM) da
missão \href{https://www.earthdata.nasa.gov/data/instruments/srtm}{SRTM}
da NASA para um país específico, com resolução de \textasciitilde1km (30
segundos de arco).

\begin{Shaded}
\begin{Highlighting}[]
\NormalTok{elevacao }\OtherTok{\textless{}{-}} \FunctionTok{elevation\_30s}\NormalTok{(}\AttributeTok{long=}\FloatTok{32.583}\NormalTok{, }\AttributeTok{lat =} \SpecialCharTok{{-}}\FloatTok{25.967}\NormalTok{, }\AttributeTok{path =} \FunctionTok{tempdir}\NormalTok{())}

\FunctionTok{ggplot}\NormalTok{() }\SpecialCharTok{+}
  \FunctionTok{geom\_spatraster}\NormalTok{(}\AttributeTok{data =}\NormalTok{ elevacao) }\SpecialCharTok{+}
  \FunctionTok{scale\_fill\_hypso\_tint\_c}\NormalTok{(}\AttributeTok{palette =} \StringTok{"dem\_poster"}\NormalTok{, }\AttributeTok{name =} \StringTok{"Altitude (m)"}\NormalTok{) }\SpecialCharTok{+}
  \FunctionTok{geom\_sf}\NormalTok{(}\AttributeTok{data =}\NormalTok{ moz\_sf, }\AttributeTok{fill =} \ConstantTok{NA}\NormalTok{, }\AttributeTok{color =} \StringTok{"black"}\NormalTok{, }\AttributeTok{size =} \FloatTok{0.2}\NormalTok{) }\SpecialCharTok{+}
  \FunctionTok{theme\_minimal}\NormalTok{() }\SpecialCharTok{+}
  \FunctionTok{labs}\NormalTok{(}\AttributeTok{title =} \StringTok{"Topografia de Moçambique"}\NormalTok{)}
\end{Highlighting}
\end{Shaded}

\begin{enumerate}
\def\labelenumi{\arabic{enumi}.}
\setcounter{enumi}{2}
\tightlist
\item
  \textbf{Dados Climáticos (\texttt{worldclim\_country})}
\end{enumerate}

O \href{https://www.worldclim.org/}{WorldClim} é o padrão para dados
climáticos em ecologia e agricultura. A função
\texttt{worldclim\_country} baixa dados históricos (médias de 1970-2000)
recortados para o país de interesse. Isso é muito mais leve do que
baixar o raster global.

\begin{itemize}
\item
  \texttt{var}: Variável desejada.
\item
  \texttt{tmin}, \texttt{tavg}, \texttt{tmax}: Temperatura (°C).
\item
  \texttt{prec}: Precipitação (mm).
\item
  \texttt{bio}: Variáveis bioclimáticas (derivadas estatísticas
  biologicamente significativas).
\item
  \texttt{res}: Resolução. \texttt{10}, \texttt{5}, \texttt{2.5} ou
  \texttt{0.5} minutos. (10 é grosseiro/leve, 0.5 é detalhado/pesado).
\end{itemize}

\begin{Shaded}
\begin{Highlighting}[]
\CommentTok{\#}
\NormalTok{prec\_moz }\OtherTok{\textless{}{-}} \FunctionTok{worldclim\_country}\NormalTok{(}\AttributeTok{country =} \StringTok{"MOZ"}\NormalTok{, }\AttributeTok{var =} \StringTok{"prec"}\NormalTok{, }\AttributeTok{res =} \DecValTok{10}\NormalTok{, }\AttributeTok{path =} \FunctionTok{tempdir}\NormalTok{())}

\CommentTok{\# Selecionar apenas Janeiro (camada 1) e Julho (camada 7)}
\NormalTok{prec\_sazonal }\OtherTok{\textless{}{-}}\NormalTok{ prec\_moz[[}\FunctionTok{c}\NormalTok{(}\DecValTok{1}\NormalTok{, }\DecValTok{7}\NormalTok{)]]}
\FunctionTok{names}\NormalTok{(prec\_sazonal) }\OtherTok{\textless{}{-}} \FunctionTok{c}\NormalTok{(}\StringTok{"Janeiro (Chuvoso)"}\NormalTok{, }\StringTok{"Julho (Seco)"}\NormalTok{)}

\FunctionTok{ggplot}\NormalTok{() }\SpecialCharTok{+}
  \FunctionTok{geom\_spatraster}\NormalTok{(}\AttributeTok{data =}\NormalTok{ prec\_sazonal) }\SpecialCharTok{+}
  \FunctionTok{facet\_wrap}\NormalTok{(}\SpecialCharTok{\textasciitilde{}}\NormalTok{lyr) }\SpecialCharTok{+} 
  \FunctionTok{scale\_fill\_whitebox\_c}\NormalTok{(}\AttributeTok{palette =} \StringTok{"deep"}\NormalTok{, }\AttributeTok{direction =} \DecValTok{1}\NormalTok{, }\AttributeTok{name =} \StringTok{"Chuva (mm)"}\NormalTok{) }\SpecialCharTok{+}
  \FunctionTok{geom\_sf}\NormalTok{(}\AttributeTok{data =}\NormalTok{ moz\_sf, }\AttributeTok{fill =} \ConstantTok{NA}\NormalTok{, }\AttributeTok{color =} \StringTok{"gray30"}\NormalTok{, }\AttributeTok{size =} \FloatTok{0.1}\NormalTok{) }\SpecialCharTok{+}
  \FunctionTok{theme\_void}\NormalTok{() }\SpecialCharTok{+}
  \FunctionTok{labs}\NormalTok{(}\AttributeTok{title =} \StringTok{"Precipitação Média Mensal (WorldClim v2.1)"}\NormalTok{)}
\end{Highlighting}
\end{Shaded}

\begin{tcolorbox}[enhanced jigsaw, left=2mm, toptitle=1mm, colback=white, colframe=quarto-callout-tip-color-frame, colbacktitle=quarto-callout-tip-color!10!white, opacityback=0, rightrule=.15mm, bottomtitle=1mm, arc=.35mm, title=\textcolor{quarto-callout-tip-color}{\faLightbulb}\hspace{0.5em}{Variáveis Bioclimáticas (Bio)}, titlerule=0mm, bottomrule=.15mm, leftrule=.75mm, coltitle=black, toprule=.15mm, breakable, opacitybacktitle=0.6]

Para modelagem de distribuição de espécies (SDM), utilize
\texttt{var\ =\ "bio"}. Isso retornará 19 camadas contendo índices como
``Temperatura do trimestre mais seco'' ou ``Precipitação do trimestre
mais quente'', que são preditores ecológicos mais fortes do que médias
mensais simples.

\end{tcolorbox}

\begin{enumerate}
\def\labelenumi{\arabic{enumi}.}
\setcounter{enumi}{3}
\tightlist
\item
  \textbf{Dados de Solo (\texttt{soil\_af} e \texttt{soil\_world})}
\end{enumerate}

Os dados de solo provêm do projeto
\href{https://soilgrids.org/}{SoilGrids} (ISRIC). São dados complexos
que incluem propriedades físicas (areia, argila) e químicas (pH,
carbono) em várias profundidades padrão (0-5cm, 5-15cm, etc.).

Para países da África, existe a função otimizada \texttt{soil\_af}
(baseada no projeto
\href{https://www.isda-africa.com/isdasoil/}{iSDAsoil}). Para o resto do
mundo, usa-se \texttt{soil\_world}.

O banco de dados global de solos é imenso (Gigabytes). Se você precisa
apenas de dados para alguns pontos específicos (ex: estações de coleta),
não baixe o raster inteiro. Use a conexão virtual (\texttt{\_vsi}).

\begin{Shaded}
\begin{Highlighting}[]
\CommentTok{\#pontos de interesse (Longitude, Latitude)}
\NormalTok{pontos\_amostra }\OtherTok{\textless{}{-}} \FunctionTok{data.frame}\NormalTok{(}
  \AttributeTok{local =} \FunctionTok{c}\NormalTok{(}\StringTok{"Maputo"}\NormalTok{, }\StringTok{"Beira"}\NormalTok{, }\StringTok{"Nampula"}\NormalTok{),}
  \AttributeTok{lon =} \FunctionTok{c}\NormalTok{(}\FloatTok{32.58}\NormalTok{, }\FloatTok{34.83}\NormalTok{, }\FloatTok{39.26}\NormalTok{),}
  \AttributeTok{lat =} \FunctionTok{c}\NormalTok{(}\SpecialCharTok{{-}}\FloatTok{25.96}\NormalTok{, }\SpecialCharTok{{-}}\FloatTok{19.83}\NormalTok{, }\SpecialCharTok{{-}}\FloatTok{15.11}\NormalTok{)}
\NormalTok{)}
\NormalTok{pontos\_vect }\OtherTok{\textless{}{-}} \FunctionTok{vect}\NormalTok{(pontos\_amostra, }\AttributeTok{geom =} \FunctionTok{c}\NormalTok{(}\StringTok{"lon"}\NormalTok{, }\StringTok{"lat"}\NormalTok{), }\AttributeTok{crs =} \StringTok{"EPSG:4326"}\NormalTok{)}

\CommentTok{\# var = "clay", depth = 15 (significa intervalo 5{-}15cm)}
\NormalTok{raster\_virtual }\OtherTok{\textless{}{-}} \FunctionTok{soil\_world\_vsi}\NormalTok{(}\AttributeTok{var =} \StringTok{"clay"}\NormalTok{, }\AttributeTok{depth =} \DecValTok{15}\NormalTok{)}

\CommentTok{\#Extrair os valores para os pontos}
\NormalTok{valores\_solo }\OtherTok{\textless{}{-}}\NormalTok{ terra}\SpecialCharTok{::}\FunctionTok{extract}\NormalTok{(raster\_virtual, pontos\_vect)}

\NormalTok{pontos\_amostra}\SpecialCharTok{$}\NormalTok{argila\_percent }\OtherTok{\textless{}{-}}\NormalTok{ valores\_solo[,}\DecValTok{2}\NormalTok{] }\SpecialCharTok{/} \DecValTok{10} \CommentTok{\# SoilGrids geralmente vem escalado x10}

\FunctionTok{print}\NormalTok{(pontos\_amostra)}
\end{Highlighting}
\end{Shaded}

\begin{enumerate}
\def\labelenumi{\arabic{enumi}.}
\setcounter{enumi}{4}
\tightlist
\item
  \textbf{Biodiversidade (\texttt{sp\_occurrence})}
\end{enumerate}

Esta função conecta-se ao \href{https://www.gbif.org/}{GBIF} (Global
Biodiversity Information Facility) para baixar coordenadas de observação
de espécies. É vital para estudos de ecologia.

\textbf{Argumentos Principais:}

\begin{itemize}
\item
  \texttt{genus}: Gênero da espécie.
\item
  \texttt{species}: Epíteto específico. Se vazio \texttt{""}, baixa todo
  o gênero.
\item
  \texttt{geo}: Se \texttt{TRUE}, baixa apenas registros com coordenadas
  geográficas.
\item
  \texttt{removeZeros}: Remove registros com coordenadas (0,0) que
  geralmente são erros.
\end{itemize}

\begin{Shaded}
\begin{Highlighting}[]
\NormalTok{moz\_sf }\OtherTok{\textless{}{-}} \FunctionTok{st\_transform}\NormalTok{(moz\_sf, }\AttributeTok{crs =} \DecValTok{4326}\NormalTok{)}
\NormalTok{elefantes }\OtherTok{\textless{}{-}} \FunctionTok{sp\_occurrence}\NormalTok{(}\AttributeTok{genus =} \StringTok{"Loxodonta"}\NormalTok{, }
                           \AttributeTok{species =} \StringTok{"africana"}\NormalTok{, }
                           \AttributeTok{ext =}\NormalTok{ moz\_sf, }\CommentTok{\# Usa o mapa baixado antes como limite}
                           \AttributeTok{geo =} \ConstantTok{TRUE}\NormalTok{,     }\CommentTok{\# Apenas com coordenadas}
                           \AttributeTok{removeZeros =} \ConstantTok{TRUE}\NormalTok{) }\CommentTok{\# Remove erros de (0,0)}

\CommentTok{\# Converter para sf}
\NormalTok{elefantes\_sf }\OtherTok{\textless{}{-}} \FunctionTok{st\_as\_sf}\NormalTok{(elefantes, }\AttributeTok{coords =} \FunctionTok{c}\NormalTok{(}\StringTok{"lon"}\NormalTok{, }\StringTok{"lat"}\NormalTok{), }\AttributeTok{crs =} \DecValTok{4326}\NormalTok{)}

\FunctionTok{ggplot}\NormalTok{() }\SpecialCharTok{+}
  \FunctionTok{geom\_sf}\NormalTok{(}\AttributeTok{data =}\NormalTok{ moz\_sf, }\AttributeTok{fill =} \StringTok{"gray95"}\NormalTok{, }\AttributeTok{color =} \StringTok{"gray50"}\NormalTok{) }\SpecialCharTok{+}
  \CommentTok{\# Adiciona os pontos dos elefantes}
  \FunctionTok{geom\_sf}\NormalTok{(}\AttributeTok{data =}\NormalTok{ elefantes\_sf, }\AttributeTok{color =} \StringTok{"darkred"}\NormalTok{, }\AttributeTok{size =} \DecValTok{1}\NormalTok{, }\AttributeTok{alpha =} \FloatTok{0.6}\NormalTok{) }\SpecialCharTok{+}
  \FunctionTok{theme\_minimal}\NormalTok{() }\SpecialCharTok{+}
  \FunctionTok{labs}\NormalTok{(}\AttributeTok{title =} \StringTok{"Ocorrência de Elefantes"}\NormalTok{)}
\end{Highlighting}
\end{Shaded}

\begin{enumerate}
\def\labelenumi{\arabic{enumi}.}
\setcounter{enumi}{5}
\tightlist
\item
  \textbf{Uso e Cobertura do Solo (\texttt{landcover})}
\end{enumerate}

A função \texttt{landcover()} baixa mapas de cobertura do solo
(floresta, agricultura, urbano, água) baseados no satélite
\href{https://esa-worldcover.org/en}{ESA WorldCover}. É um dado
categórico de alta resolução.

\begin{itemize}
\tightlist
\item
  \texttt{path}: Local para salvar. O download é feito por ``tiles''
  (retângulos de área).
\end{itemize}

\begin{Shaded}
\begin{Highlighting}[]
\CommentTok{\# Baixa o uso do solo para a região central do país}
\NormalTok{uso\_solo }\OtherTok{\textless{}{-}} \FunctionTok{landcover}\NormalTok{(}\AttributeTok{var =} \StringTok{"trees"}\NormalTok{, }\AttributeTok{path =} \FunctionTok{tempdir}\NormalTok{(), }\AttributeTok{country =} \StringTok{"MOZ"}\NormalTok{)}
\FunctionTok{plot}\NormalTok{(uso\_solo, }\AttributeTok{main =} \StringTok{"Cobertura Arbórea"}\NormalTok{)}
\end{Highlighting}
\end{Shaded}

\begin{enumerate}
\def\labelenumi{\arabic{enumi}.}
\setcounter{enumi}{6}
\tightlist
\item
  \textbf{Acessibilidade (\texttt{travel\_time})}
\end{enumerate}

Fornece mapas raster de acessibilidade global, representando o tempo de
viagem até centros urbanos ou portos, considerando infraestrutura de
transporte e topografia.

\textbf{Argumentos Principais:}

\begin{itemize}
\item
  \texttt{to}: Destino (\texttt{"city"} ou \texttt{"port"}).
\item
  \texttt{size}: O tamanho do destino.
\item
  Para cidades: 1 (muito grande) a 9 (pequena).
\item
  Ex: \texttt{size=1} calcula o tempo até cidades com \textgreater{} 5
  milhões de habitantes.
\end{itemize}

\begin{Shaded}
\begin{Highlighting}[]
\NormalTok{acesso }\OtherTok{\textless{}{-}} \FunctionTok{travel\_time}\NormalTok{(}\AttributeTok{to =} \StringTok{"city"}\NormalTok{, }\AttributeTok{size =} \DecValTok{4}\NormalTok{, }\AttributeTok{path =} \FunctionTok{tempdir}\NormalTok{())}

\FunctionTok{plot}\NormalTok{(acesso, }\AttributeTok{main =} \StringTok{"Tempo de Viagem para Cidades Médias"}\NormalTok{, }\AttributeTok{col =} \FunctionTok{map.pal}\NormalTok{(}\StringTok{"inferno"}\NormalTok{))}
\end{Highlighting}
\end{Shaded}

\begin{tcolorbox}[enhanced jigsaw, left=2mm, toptitle=1mm, colback=white, colframe=quarto-callout-tip-color-frame, colbacktitle=quarto-callout-tip-color!10!white, opacityback=0, rightrule=.15mm, bottomtitle=1mm, arc=.35mm, title=\textcolor{quarto-callout-tip-color}{\faLightbulb}\hspace{0.5em}{Integre tudo}, titlerule=0mm, bottomrule=.15mm, leftrule=.75mm, coltitle=black, toprule=.15mm, breakable, opacitybacktitle=0.6]

A verdadeira força da análise espacial surge ao cruzar essas camadas.
Com os dados baixados acima, você poderia facilmente responder a
perguntas complexas, como: \emph{Qual é a precipitação média (WorldClim)
nos locais onde foram avistados elefantes (GBIF) na província de Maputo
(GADM)?}.

\end{tcolorbox}

\subsection{O pacote mapsf}\label{o-pacote-mapsf}

O pacote \texttt{mapsf} Giraud (2025), desenvolvido por
\href{https://orcid.org/0000-0002-1932-3323}{Timothée Giraud}, é o
sucessor do pacote
\href{https://cran.r-project.org/web/packages/cartography/vignettes/cartography.html}{cartography}.
O seu objetivo é permitir a criação de mapas temáticos de alta qualidade
visual com um fluxo de trabalho simplificado, baseado em objetos
\texttt{sf}.

Diferente do \texttt{ggplot2}, que trata o mapa como um gráfico
estatístico num plano cartesiano, o \texttt{mapsf}, facilita a inclusão
de elementos cartográficos clássicos (escala, norte, sombras, inset) e a
gestão de layouts complexos.

\begin{enumerate}
\def\labelenumi{\arabic{enumi}.}
\tightlist
\item
  \textbf{Instalação e Carregamento}
\end{enumerate}

\begin{Shaded}
\begin{Highlighting}[]
\ControlFlowTok{if}\NormalTok{ (}\SpecialCharTok{!}\FunctionTok{require}\NormalTok{(}\StringTok{"pacman"}\NormalTok{)) }\FunctionTok{install.packages}\NormalTok{(}\StringTok{"pacman"}\NormalTok{)}
\NormalTok{pacman}\SpecialCharTok{::}\FunctionTok{p\_load}\NormalTok{(mapsf, sf)}
\end{Highlighting}
\end{Shaded}

\begin{Shaded}
\begin{Highlighting}[]
\ControlFlowTok{if}\NormalTok{ (}\SpecialCharTok{!}\FunctionTok{require}\NormalTok{(}\StringTok{"pacman"}\NormalTok{)) }\FunctionTok{install.packages}\NormalTok{(}\StringTok{"pacman"}\NormalTok{)}
\NormalTok{pacman}\SpecialCharTok{::}\FunctionTok{p\_load}\NormalTok{(mapsf, sf, geobr, dplyr)}

\NormalTok{pr\_mun }\OtherTok{\textless{}{-}} \FunctionTok{read\_municipality}\NormalTok{(}\AttributeTok{code\_muni =} \StringTok{"PR"}\NormalTok{, }\AttributeTok{year =} \DecValTok{2020}\NormalTok{, }\AttributeTok{showProgress =} \ConstantTok{FALSE}\NormalTok{)}

\CommentTok{\#Transformar projeção para SIRGAS 2000 / UTM zone 22S (EPSG: 31982)}
\NormalTok{pr\_mun }\OtherTok{\textless{}{-}} \FunctionTok{st\_transform}\NormalTok{(pr\_mun, }\AttributeTok{crs =} \DecValTok{31982}\NormalTok{)}

\CommentTok{\#Simular dados}
\FunctionTok{set.seed}\NormalTok{(}\DecValTok{123}\NormalTok{)}
\NormalTok{pr\_mun}\SpecialCharTok{$}\NormalTok{populacao }\OtherTok{\textless{}{-}} \FunctionTok{sample}\NormalTok{(}\DecValTok{5000}\SpecialCharTok{:}\DecValTok{150000}\NormalTok{, }\AttributeTok{size =} \FunctionTok{nrow}\NormalTok{(pr\_mun), }\AttributeTok{replace =} \ConstantTok{TRUE}\NormalTok{)}
\CommentTok{\# Inserir um outlier realista para Curitiba e Londrina}
\NormalTok{pr\_mun}\SpecialCharTok{$}\NormalTok{populacao[pr\_mun}\SpecialCharTok{$}\NormalTok{name\_muni }\SpecialCharTok{==} \StringTok{"Curitiba"}\NormalTok{] }\OtherTok{\textless{}{-}} \DecValTok{1963726}
\NormalTok{pr\_mun}\SpecialCharTok{$}\NormalTok{populacao[pr\_mun}\SpecialCharTok{$}\NormalTok{name\_muni }\SpecialCharTok{==} \StringTok{"Londrina"}\NormalTok{] }\OtherTok{\textless{}{-}} \DecValTok{575377}

\CommentTok{\# Calcular Densidade (Hab/km\^{}2)}
\NormalTok{pr\_mun}\SpecialCharTok{$}\NormalTok{area\_km2 }\OtherTok{\textless{}{-}} \FunctionTok{as.numeric}\NormalTok{(}\FunctionTok{st\_area}\NormalTok{(pr\_mun)) }\SpecialCharTok{/} \FloatTok{1e6}
\NormalTok{pr\_mun}\SpecialCharTok{$}\NormalTok{densidade }\OtherTok{\textless{}{-}}\NormalTok{ pr\_mun}\SpecialCharTok{$}\NormalTok{populacao }\SpecialCharTok{/}\NormalTok{ pr\_mun}\SpecialCharTok{$}\NormalTok{area\_km2}
\end{Highlighting}
\end{Shaded}

\begin{enumerate}
\def\labelenumi{\arabic{enumi}.}
\tightlist
\item
  \textbf{Temas e Estética} (\texttt{mf\_theme})
\end{enumerate}

Antes de plotar qualquer dado, definimos o ``humor'' do mapa. A função
\texttt{mf\_theme()} controla margens, cores de fundo, fontes e estilos
de borda. O pacote vem com vários temas pré-definidos (ex: ``default'',
``dark'', ``ink'', ``agolalight'', ``iceberg'').

\begin{Shaded}
\begin{Highlighting}[]
\CommentTok{\# Define margens, cor de fundo e família de fonte}
\FunctionTok{mf\_theme}\NormalTok{(}\StringTok{"iceberg"}\NormalTok{) }

\FunctionTok{mf\_map}\NormalTok{(pr\_mun, }\AttributeTok{col =} \StringTok{"grey80"}\NormalTok{, }\AttributeTok{border =} \StringTok{"black"}\NormalTok{) }

\CommentTok{\# Adicionamos apenas o contorno do estado}
\NormalTok{pr\_estado }\OtherTok{\textless{}{-}} \FunctionTok{st\_union}\NormalTok{(pr\_mun)}
\FunctionTok{mf\_map}\NormalTok{(pr\_estado, }\AttributeTok{col =} \ConstantTok{NA}\NormalTok{, }\AttributeTok{border =} \StringTok{"grey40"}\NormalTok{, }\AttributeTok{lwd =} \FloatTok{1.5}\NormalTok{, }\AttributeTok{add =} \ConstantTok{TRUE}\NormalTok{)}

\FunctionTok{mf\_title}\NormalTok{(}\StringTok{"Estado do Paraná"}\NormalTok{, }\AttributeTok{pos =} \StringTok{"center"}\NormalTok{, }\AttributeTok{tab =} \ConstantTok{FALSE}\NormalTok{)}
\end{Highlighting}
\end{Shaded}

\begin{figure}[H]

\centering{

\pandocbounded{\includegraphics[keepaspectratio]{fundEstspatial_files/figure-pdf/fig-theme-clean-1.pdf}}

}

\caption{\label{fig-theme-clean}Mapa base com tema Iceberg e sem bordas
internas.}

\end{figure}%

\begin{enumerate}
\def\labelenumi{\arabic{enumi}.}
\setcounter{enumi}{1}
\tightlist
\item
  \textbf{Sombras e Profundidade (\texttt{mf\_shadow})}
\end{enumerate}

Um recurso estético simples, mas que adiciona grande valor visual
(``efeito pop-up''), é a adição de sombras sob os polígonos. A função
\texttt{mf\_shadow()} desenha uma silhueta deslocada da geometria.

\begin{Shaded}
\begin{Highlighting}[]
\FunctionTok{mf\_shadow}\NormalTok{(pr\_mun, }\AttributeTok{col =} \StringTok{"gray50"}\NormalTok{, }\AttributeTok{cex =} \FloatTok{1.2}\NormalTok{)}
\CommentTok{\#Adicionar o mapa por cima}
\FunctionTok{mf\_map}\NormalTok{(pr\_mun, }\AttributeTok{add =} \ConstantTok{TRUE}\NormalTok{)}
\end{Highlighting}
\end{Shaded}

\begin{enumerate}
\def\labelenumi{\arabic{enumi}.}
\setcounter{enumi}{2}
\tightlist
\item
  \textbf{O Mapa Temático (\texttt{mf\_map})}
\end{enumerate}

Esta é a função central. O argumento \texttt{type} define como os dados
serão representados cartograficamente.

\begin{itemize}
\item
  \texttt{x}: Objeto \texttt{sf}.
\item
  \texttt{var}: Nome da coluna de atributos a ser mapeada.
\item
  \texttt{type}:
\item
  \texttt{"choro"}: Mapa coroplético (áreas coloridas por faixas de
  valores).
\item
  \texttt{"prop"}: Símbolos proporcionais (círculos).
\item
  \texttt{"typo"}: Tipologia (mapa qualitativo/categórico).
\end{itemize}

\begin{Shaded}
\begin{Highlighting}[]
\FunctionTok{mf\_theme}\NormalTok{(}\StringTok{"iceberg"}\NormalTok{) }\CommentTok{\#vc pode trocar aqui para tirar fundo preto}

\CommentTok{\#Sombra para destacar o estado do fundo (efeito 3D)}
\FunctionTok{mf\_shadow}\NormalTok{(pr\_mun, }\AttributeTok{col =} \StringTok{"grey50"}\NormalTok{, }\AttributeTok{cex =} \FloatTok{1.5}\NormalTok{)}

\CommentTok{\# 2. Mapa Temático}
\FunctionTok{mf\_map}\NormalTok{(}
  \AttributeTok{x =}\NormalTok{ pr\_mun, }
  \AttributeTok{var =} \StringTok{"densidade"}\NormalTok{, }
  \AttributeTok{type =} \StringTok{"choro"}\NormalTok{,}
  \AttributeTok{breaks =} \StringTok{"quantile"}\NormalTok{,  }
  \AttributeTok{nbreaks =} \DecValTok{6}\NormalTok{,          }
  \AttributeTok{pal =} \StringTok{"YlOrRd"}\NormalTok{,       }
  \AttributeTok{border =} \ConstantTok{NA}\NormalTok{,         }
  \AttributeTok{leg\_title =} \StringTok{"Densidade}\SpecialCharTok{\textbackslash{}n}\StringTok{(hab/km²)"}\NormalTok{, }
  \AttributeTok{leg\_val\_rnd =} \DecValTok{0}\NormalTok{,      }\CommentTok{\# Arredondar valores da legenda}
  \AttributeTok{leg\_pos =} \StringTok{"topright"}\NormalTok{,}
  \AttributeTok{add =} \ConstantTok{TRUE}
\NormalTok{)}

\CommentTok{\#Contorno do Estado (acabamento)}
\FunctionTok{mf\_map}\NormalTok{(pr\_estado, }\AttributeTok{col =} \ConstantTok{NA}\NormalTok{, }\AttributeTok{border =} \StringTok{"white"}\NormalTok{, }\AttributeTok{lwd =} \FloatTok{0.5}\NormalTok{, }\AttributeTok{add =} \ConstantTok{TRUE}\NormalTok{)}

\FunctionTok{mf\_title}\NormalTok{(}\StringTok{"Densidade Demográfica no Paraná"}\NormalTok{, }\AttributeTok{pos =} \StringTok{"left"}\NormalTok{)}
\FunctionTok{mf\_credits}\NormalTok{(}\StringTok{"Fonte: IBGE/geobr | Elaboração Própria"}\NormalTok{, }\AttributeTok{cex =} \FloatTok{0.6}\NormalTok{)}
\FunctionTok{mf\_scale}\NormalTok{(}\AttributeTok{size =} \DecValTok{100}\NormalTok{, }\AttributeTok{pos =} \StringTok{"bottomright"}\NormalTok{) }\CommentTok{\# Escala discreta}
\FunctionTok{mf\_arrow}\NormalTok{(}\AttributeTok{pos =} \StringTok{"bottomright"}\NormalTok{,}\AttributeTok{adj =} \FunctionTok{c}\NormalTok{(}\DecValTok{1}\NormalTok{, }\DecValTok{1}\NormalTok{))}
\end{Highlighting}
\end{Shaded}

\begin{figure}[H]

\centering{

\pandocbounded{\includegraphics[keepaspectratio]{fundEstspatial_files/figure-pdf/fig-choro-pro-1.pdf}}

}

\caption{\label{fig-choro-pro}Densidade Demográfica com paleta contínua
e layout limpo.}

\end{figure}%

\begin{enumerate}
\def\labelenumi{\arabic{enumi}.}
\setcounter{enumi}{3}
\tightlist
\item
  \textbf{Elementos de Layout (\texttt{mf\_title}, \texttt{mf\_scale},
  \texttt{mf\_arrow})}
\end{enumerate}

Um mapa técnico exige elementos de referência para leitura correta.

\begin{itemize}
\item
  \texttt{mf\_title()}: Título do mapa.
\item
  \texttt{mf\_scale()}: Barra de escala.
\item
  \texttt{mf\_arrow()}: Rosa dos ventos ou seta norte.
\item
  \texttt{mf\_credits()}: Fonte dos dados e autoria.
\end{itemize}

\begin{Shaded}
\begin{Highlighting}[]
\FunctionTok{p\_load}\NormalTok{(viridis)  }\CommentTok{\#para cores}
\FunctionTok{mf\_theme}\NormalTok{(}\ConstantTok{NULL}\NormalTok{)}


\CommentTok{\#}
\FunctionTok{mf\_map}\NormalTok{(}
  \AttributeTok{x =}\NormalTok{ pr\_mun, }
  \AttributeTok{var =} \StringTok{"area\_km2"}\NormalTok{, }
  \AttributeTok{type =} \StringTok{"choro"}\NormalTok{,}
  \AttributeTok{nbreaks =} \DecValTok{5}\NormalTok{,}
  \AttributeTok{border =} \StringTok{"white"}\NormalTok{,  }
  \AttributeTok{lwd =} \FloatTok{0.5}\NormalTok{,}
  \AttributeTok{leg\_pos =} \StringTok{"bottomright"}\NormalTok{,}
  \AttributeTok{leg\_title =} \FunctionTok{expression}\NormalTok{(}\StringTok{"Área Territorial"}\SpecialCharTok{\textasciitilde{}}\NormalTok{km}\SpecialCharTok{\^{}}\DecValTok{2}\NormalTok{)}
\NormalTok{)}

\CommentTok{\#}
\FunctionTok{mf\_map}\NormalTok{(}
  \AttributeTok{x =}\NormalTok{ pr\_mun,}
  \AttributeTok{var =} \StringTok{"populacao"}\NormalTok{,}
  \AttributeTok{type =} \StringTok{"prop"}\NormalTok{,}
  \AttributeTok{inches =} \FloatTok{0.25}\NormalTok{,        }\CommentTok{\# Tamanho do maior círculo}
  \AttributeTok{col =} \StringTok{"gray50"}\NormalTok{,   }
  \AttributeTok{leg\_pos =} \StringTok{"topright"}\NormalTok{,}
  \AttributeTok{leg\_title =} \StringTok{"População Total"}\NormalTok{,}
  \AttributeTok{val\_max =} \FunctionTok{max}\NormalTok{(pr\_mun}\SpecialCharTok{$}\NormalTok{populacao), }
  \AttributeTok{add =} \ConstantTok{TRUE}
\NormalTok{)}

\CommentTok{\#penas para cidades \textgreater{} 300k hab para não poluir}
\NormalTok{big\_cities }\OtherTok{\textless{}{-}}\NormalTok{ pr\_mun[pr\_mun}\SpecialCharTok{$}\NormalTok{populacao }\SpecialCharTok{\textgreater{}} \DecValTok{300000}\NormalTok{, ]}

\FunctionTok{mf\_label}\NormalTok{(}
  \AttributeTok{x =}\NormalTok{ big\_cities, }
  \AttributeTok{var =} \StringTok{"name\_muni"}\NormalTok{, }
  \AttributeTok{col =} \StringTok{"black"}\NormalTok{, }
  \AttributeTok{cex =} \FloatTok{0.7}\NormalTok{, }
  \AttributeTok{overlap =} \ConstantTok{FALSE}\NormalTok{, }
  \AttributeTok{lines =} \ConstantTok{FALSE}
\NormalTok{)}
\FunctionTok{mf\_scale}\NormalTok{(}\AttributeTok{size =} \DecValTok{100}\NormalTok{, }\AttributeTok{pos =} \StringTok{"bottomleft"}\NormalTok{) }\CommentTok{\# Escala discreta}
\FunctionTok{mf\_arrow}\NormalTok{(}\AttributeTok{pos =} \StringTok{"bottomleft"}\NormalTok{,}\AttributeTok{adj =} \FunctionTok{c}\NormalTok{(}\DecValTok{1}\NormalTok{, }\DecValTok{1}\NormalTok{))}
\end{Highlighting}
\end{Shaded}

\begin{figure}[H]

\centering{

\pandocbounded{\includegraphics[keepaspectratio]{fundEstspatial_files/figure-pdf/fig-prop-choro-1.pdf}}

}

\caption{\label{fig-prop-choro}Mapa de dados simulados: Área (Cor) e
População (Círculos).}

\end{figure}%

\begin{enumerate}
\def\labelenumi{\arabic{enumi}.}
\setcounter{enumi}{4}
\tightlist
\item
  \textbf{Mapas de Localização (\texttt{mf\_inset\_on})}
\end{enumerate}

Em vez de usar um mapa mundi (\texttt{mf\_worldmap}), para dados
estaduais é mais comum mostrar onde o estado fica dentro do país
(Brasil). Usamos a função \texttt{mf\_inset\_on()} para criar um
``mini-mapa'' num canto da figura.

Precisaremos do contorno do Brasil para isso.

\begin{Shaded}
\begin{Highlighting}[]
\CommentTok{\# 1. Carregar bases do Brasil}
\NormalTok{br }\OtherTok{\textless{}{-}} \FunctionTok{read\_country}\NormalTok{(}\AttributeTok{year =} \DecValTok{2020}\NormalTok{, }\AttributeTok{showProgress =} \ConstantTok{FALSE}\NormalTok{)}
\NormalTok{estados }\OtherTok{\textless{}{-}} \FunctionTok{read\_state}\NormalTok{(}\AttributeTok{year =} \DecValTok{2020}\NormalTok{, }\AttributeTok{showProgress =} \ConstantTok{FALSE}\NormalTok{)}

\NormalTok{pr\_destaque }\OtherTok{\textless{}{-}}\NormalTok{ estados[estados}\SpecialCharTok{$}\NormalTok{abbrev\_state }\SpecialCharTok{==} \StringTok{"PR"}\NormalTok{, ]}


\FunctionTok{p\_load}\NormalTok{(viridis)  }\CommentTok{\#para cores}
\FunctionTok{mf\_theme}\NormalTok{(}\ConstantTok{NULL}\NormalTok{)}


\CommentTok{\#}
\FunctionTok{mf\_map}\NormalTok{(}
  \AttributeTok{x =}\NormalTok{ pr\_mun, }
  \AttributeTok{var =} \StringTok{"area\_km2"}\NormalTok{, }
  \AttributeTok{type =} \StringTok{"choro"}\NormalTok{,}
  \AttributeTok{nbreaks =} \DecValTok{5}\NormalTok{,}
  \AttributeTok{border =} \StringTok{"white"}\NormalTok{,  }
  \AttributeTok{lwd =} \FloatTok{0.5}\NormalTok{,}
  \AttributeTok{leg\_pos =} \StringTok{"bottomright"}\NormalTok{,}
  \AttributeTok{leg\_title =} \FunctionTok{expression}\NormalTok{(}\StringTok{"Área Territorial"}\SpecialCharTok{\textasciitilde{}}\NormalTok{km}\SpecialCharTok{\^{}}\DecValTok{2}\NormalTok{)}
\NormalTok{)}

\CommentTok{\#}
\FunctionTok{mf\_map}\NormalTok{(}
  \AttributeTok{x =}\NormalTok{ pr\_mun,}
  \AttributeTok{var =} \StringTok{"populacao"}\NormalTok{,}
  \AttributeTok{type =} \StringTok{"prop"}\NormalTok{,}
  \AttributeTok{inches =} \FloatTok{0.25}\NormalTok{,        }\CommentTok{\# Tamanho do maior círculo}
  \AttributeTok{col =} \StringTok{"gray50"}\NormalTok{,   }
  \AttributeTok{leg\_pos =} \StringTok{"topleft"}\NormalTok{,}
  \AttributeTok{leg\_title =} \StringTok{"População Total"}\NormalTok{,}
  \AttributeTok{val\_max =} \FunctionTok{max}\NormalTok{(pr\_mun}\SpecialCharTok{$}\NormalTok{populacao), }
  \AttributeTok{add =} \ConstantTok{TRUE}
\NormalTok{)}

\CommentTok{\#penas para cidades \textgreater{} 300k hab para não poluir}
\NormalTok{big\_cities }\OtherTok{\textless{}{-}}\NormalTok{ pr\_mun[pr\_mun}\SpecialCharTok{$}\NormalTok{populacao }\SpecialCharTok{\textgreater{}} \DecValTok{300000}\NormalTok{, ]}

\FunctionTok{mf\_label}\NormalTok{(}
  \AttributeTok{x =}\NormalTok{ big\_cities, }
  \AttributeTok{var =} \StringTok{"name\_muni"}\NormalTok{, }
  \AttributeTok{col =} \StringTok{"yellow"}\NormalTok{, }
  \AttributeTok{cex =} \FloatTok{0.7}\NormalTok{, }
  \AttributeTok{overlap =} \ConstantTok{FALSE}\NormalTok{, }
  \AttributeTok{lines =} \ConstantTok{FALSE}
\NormalTok{)}
\FunctionTok{mf\_scale}\NormalTok{(}\AttributeTok{size =} \DecValTok{100}\NormalTok{, }\AttributeTok{pos =} \StringTok{"bottomleft"}\NormalTok{) }\CommentTok{\# Escala discreta}
\FunctionTok{mf\_arrow}\NormalTok{(}\AttributeTok{pos =} \StringTok{"bottomleft"}\NormalTok{,}\AttributeTok{adj =} \FunctionTok{c}\NormalTok{(}\DecValTok{1}\NormalTok{, }\DecValTok{1}\NormalTok{))}


\CommentTok{\#INÍCIO DO INSET}
\FunctionTok{mf\_inset\_on}\NormalTok{(}\AttributeTok{x =}\NormalTok{ br, }\AttributeTok{pos =} \StringTok{"topright"}\NormalTok{, }\AttributeTok{cex =} \FloatTok{0.25}\NormalTok{)}

\CommentTok{\#Desenhar todos os estados (Fundo)}
\FunctionTok{mf\_map}\NormalTok{(estados, }\AttributeTok{col =} \StringTok{"grey90"}\NormalTok{, }\AttributeTok{border =} \StringTok{"white"}\NormalTok{, }\AttributeTok{lwd =} \FloatTok{0.3}\NormalTok{)}

\CommentTok{\#Desenhar o destaque}
\FunctionTok{mf\_map}\NormalTok{(pr\_destaque, }\AttributeTok{col =} \StringTok{"black"}\NormalTok{, }\AttributeTok{border =} \ConstantTok{NA}\NormalTok{, }\AttributeTok{add =} \ConstantTok{TRUE}\NormalTok{)}

\CommentTok{\#Caixa}
\FunctionTok{box}\NormalTok{(}\AttributeTok{col =} \StringTok{"grey50"}\NormalTok{)}

\FunctionTok{mf\_inset\_off}\NormalTok{()}
\end{Highlighting}
\end{Shaded}

\begin{figure}[H]

\centering{

\pandocbounded{\includegraphics[keepaspectratio]{fundEstspatial_files/figure-pdf/fig-inset-fixed-1.pdf}}

}

\caption{\label{fig-inset-fixed}}

\end{figure}%

\begin{enumerate}
\def\labelenumi{\arabic{enumi}.}
\setcounter{enumi}{5}
\tightlist
\item
  \textbf{Exportação Vetorial (\texttt{mf\_svg})}
\end{enumerate}

Para finalização profissional em softwares de design (como Adobe
Illustrator ou Inkscape), é preferível exportar o mapa em formato
vetorial (SVG).

\begin{Shaded}
\begin{Highlighting}[]
\CommentTok{\#}
\FunctionTok{mf\_svg}\NormalTok{(}\AttributeTok{x =}\NormalTok{ pr\_mun, }\AttributeTok{filename =} \StringTok{"mapa\_parana.svg"}\NormalTok{, }\AttributeTok{width =} \DecValTok{10}\NormalTok{)}
\FunctionTok{mf\_theme}\NormalTok{(}\StringTok{"default"}\NormalTok{)}
\FunctionTok{mf\_map}\NormalTok{(pr\_mun, }\AttributeTok{var =} \StringTok{"densidade"}\NormalTok{, }\AttributeTok{type =} \StringTok{"choro"}\NormalTok{)}
\FunctionTok{mf\_title}\NormalTok{(}\StringTok{"Mapa Vetorial do Paraná"}\NormalTok{)}

\FunctionTok{dev.off}\NormalTok{()}
\end{Highlighting}
\end{Shaded}

\begin{tcolorbox}[enhanced jigsaw, left=2mm, toptitle=1mm, colback=white, colframe=quarto-callout-tip-color-frame, colbacktitle=quarto-callout-tip-color!10!white, opacityback=0, rightrule=.15mm, bottomtitle=1mm, arc=.35mm, title=\textcolor{quarto-callout-tip-color}{\faLightbulb}\hspace{0.5em}{Dica de Layout}, titlerule=0mm, bottomrule=.15mm, leftrule=.75mm, coltitle=black, toprule=.15mm, breakable, opacitybacktitle=0.6]

O \texttt{mapsf} funciona desenhando em camadas, como se estivesse
pintando uma tela. A ordem dos comandos importa: primeiro o fundo,
depois o mapa, depois as anotações. Se chamar \texttt{mf\_theme()} no
meio do código, ele não alterará o que já foi desenhado, apenas o que
virá a seguir.

\end{tcolorbox}

\subsection{pacote ggmapinset: Insets e
Zoom}\label{pacote-ggmapinset-insets-e-zoom}

O pacote \texttt{ggmapinset} Suster (2024) estende as funcionalidades do
\texttt{ggplot2} para permitir a criação de painéis de zoom (insets)
dentro da mesma área de plotagem.

Diferente de abordagens como o \texttt{patchwork} (que cola dois
gráficos distintos lado a lado), o \texttt{ggmapinset} transforma a
geometria dos dados. Ele recorta a área de interesse, amplia-a (zoom) e
move-a para um espaço vazio no mapa (geralmente o oceano), mantendo a
coerência dos dados e estilos numa única camada.

\begin{enumerate}
\def\labelenumi{\arabic{enumi}.}
\tightlist
\item
  \textbf{Instalação e Carregamento}
\end{enumerate}

\begin{Shaded}
\begin{Highlighting}[]
\ControlFlowTok{if}\NormalTok{ (}\SpecialCharTok{!}\FunctionTok{require}\NormalTok{(}\StringTok{"pacman"}\NormalTok{)) }\FunctionTok{install.packages}\NormalTok{(}\StringTok{"pacman"}\NormalTok{)}
\CommentTok{\# Nota: O ggmapinset é recente, garanta que tem uma versão atualizada do R}
\NormalTok{pacman}\SpecialCharTok{::}\FunctionTok{p\_load}\NormalTok{(ggmapinset, ggplot2, sf, dplyr, geodata)}
\end{Highlighting}
\end{Shaded}

Para este exemplo, utilizaremos o mapa de Moçambique ao nível das
províncias. Vamos focar na província de Maputo Cidade, que é
geograficamente muito pequena e difícil de visualizar num mapa nacional
sem zoom.

\begin{Shaded}
\begin{Highlighting}[]
\CommentTok{\#moz \textless{}{-} geodata::gadm(country = "MOZ", level = 1, path = tempdir())}

\NormalTok{moz\_sf }\OtherTok{\textless{}{-}} \FunctionTok{st\_as\_sf}\NormalTok{(moz\_sf)}

\CommentTok{\#Identificar o alvo do zoom: Maputo Cidade}
\NormalTok{maputo\_alvo }\OtherTok{\textless{}{-}}\NormalTok{ moz\_sf }\SpecialCharTok{|\textgreater{}} 
  \FunctionTok{filter}\NormalTok{(ADM1\_PT }\SpecialCharTok{==} \StringTok{"Maputo City"}\NormalTok{) }\SpecialCharTok{|\textgreater{}}  \CommentTok{\#se usar gadm troque ADM1\_PT por NAME\_!}
  \FunctionTok{st\_centroid}\NormalTok{()}
\end{Highlighting}
\end{Shaded}

\begin{enumerate}
\def\labelenumi{\arabic{enumi}.}
\tightlist
\item
  \textbf{Configuração do Inset} (\texttt{configure\_inset})
\end{enumerate}

Antes de plotar, precisamos definir o que queremos ampliar e para onde
queremos mover essa ampliação.

A função \texttt{configure\_inset()} cria um objeto de configuração que
será usado pelo sistema de coordenadas.

\begin{itemize}
\item
  \texttt{shape}: Define a forma e o local do recorte
  (\texttt{shape\_circle} ou \texttt{shape\_rectangle}).
\item
  \texttt{scale}: O fator de zoom (ex: \texttt{4} aumenta 4x).
\item
  \texttt{translation}: O deslocamento para onde mover o zoom (X, Y).
\item
  \texttt{units}: Unidade de medida (km, mi).
\end{itemize}

\begin{Shaded}
\begin{Highlighting}[]
\NormalTok{config\_maputo }\OtherTok{\textless{}{-}}\NormalTok{ ggmapinset}\SpecialCharTok{::}\FunctionTok{configure\_inset}\NormalTok{(}
  \AttributeTok{shape =}\NormalTok{ ggmapinset}\SpecialCharTok{::}\FunctionTok{shape\_circle}\NormalTok{(}\AttributeTok{centre =}\NormalTok{ maputo\_alvo, }\AttributeTok{radius =} \DecValTok{50}\NormalTok{),}
  \AttributeTok{scale =} \DecValTok{4}\NormalTok{,              }\CommentTok{\# Zoom forte (4x)}
  \AttributeTok{translation =} \FunctionTok{c}\NormalTok{(}\DecValTok{150}\NormalTok{, }\SpecialCharTok{{-}}\DecValTok{100}\NormalTok{), }\CommentTok{\# Move 150km para Leste e 100km para Sul}
  \AttributeTok{units =} \StringTok{"km"}
\NormalTok{)}
\end{Highlighting}
\end{Shaded}

\begin{enumerate}
\def\labelenumi{\arabic{enumi}.}
\setcounter{enumi}{1}
\tightlist
\item
  \textbf{O Sistema de Coordenadas} (\texttt{coord\_sf\_inset})
\end{enumerate}

Para que o \texttt{ggplot2} entenda que deve aplicar essa transformação,
substituímos o tradicional \texttt{coord\_sf()} por
\texttt{coord\_sf\_inset()}.

\begin{itemize}
\tightlist
\item
  \textbf{\texttt{inset}}: O objeto de configuração criado no passo
  anterior.
\end{itemize}

\begin{Shaded}
\begin{Highlighting}[]
\CommentTok{\# A estrutura básica do plot será:}
\FunctionTok{ggplot}\NormalTok{(moz\_sf) }\SpecialCharTok{+}
\NormalTok{  ggmapinset}\SpecialCharTok{::}\FunctionTok{coord\_sf\_inset}\NormalTok{(}\AttributeTok{inset =}\NormalTok{ config\_maputo)}
\end{Highlighting}
\end{Shaded}

\begin{enumerate}
\def\labelenumi{\arabic{enumi}.}
\setcounter{enumi}{2}
\tightlist
\item
  \textbf{Geometrias Adaptadas} (\texttt{geom\_sf\_inset})
\end{enumerate}

Se usarmos o \texttt{geom\_sf()} padrão, ele desenhará o mapa normal.
Para que as geometrias respeitem o zoom e sejam desenhadas tanto na base
quanto no inset, usamos \texttt{geom\_sf\_inset()}.

\begin{Shaded}
\begin{Highlighting}[]
\FunctionTok{p\_load}\NormalTok{(ggmapinset)}

\NormalTok{config\_maputo\_direita }\OtherTok{\textless{}{-}} \FunctionTok{configure\_inset}\NormalTok{(}
  \AttributeTok{shape =} \FunctionTok{shape\_circle}\NormalTok{(}\AttributeTok{centre =}\NormalTok{ maputo\_alvo, }\AttributeTok{radius =} \DecValTok{50}\NormalTok{),}
  \AttributeTok{scale =} \DecValTok{4}\NormalTok{,}
  \AttributeTok{translation =} \FunctionTok{c}\NormalTok{(}\DecValTok{350}\NormalTok{, }\SpecialCharTok{{-}}\DecValTok{100}\NormalTok{), }\CommentTok{\# pode alterar para mover }
  \AttributeTok{units =} \StringTok{"km"}
\NormalTok{)}

\FunctionTok{ggplot}\NormalTok{(moz\_sf) }\SpecialCharTok{+}
\NormalTok{  ggmapinset}\SpecialCharTok{::}\FunctionTok{geom\_sf\_inset}\NormalTok{(}\FunctionTok{aes}\NormalTok{(}\AttributeTok{fill =}\NormalTok{ ADM1\_PT), }\AttributeTok{color =} \StringTok{"white"}\NormalTok{, }\AttributeTok{show.legend =} \ConstantTok{FALSE}\NormalTok{) }\SpecialCharTok{+}
  \FunctionTok{scale\_fill\_viridis\_d}\NormalTok{(}\AttributeTok{option =} \StringTok{"mako"}\NormalTok{) }\SpecialCharTok{+}
  
\NormalTok{  ggmapinset}\SpecialCharTok{::}\FunctionTok{coord\_sf\_inset}\NormalTok{(}\AttributeTok{inset =}\NormalTok{ config\_maputo\_direita) }\SpecialCharTok{+} 
  \FunctionTok{theme\_void}\NormalTok{()}
\end{Highlighting}
\end{Shaded}

\begin{figure}[H]

\centering{

\pandocbounded{\includegraphics[keepaspectratio]{fundEstspatial_files/figure-pdf/fig-moz-inset-basico1-1.pdf}}

}

\caption{\label{fig-moz-inset-basico1}Mapa de Moçambique com zoom em
Maputo (Básico)}

\end{figure}%

\begin{Shaded}
\begin{Highlighting}[]
\NormalTok{forma\_personalizada }\OtherTok{\textless{}{-}} \FunctionTok{filter}\NormalTok{(moz\_sf, moz\_data\_spat}\SpecialCharTok{$}\NormalTok{ADM1\_PT}\SpecialCharTok{==}\StringTok{"Maputo"}\NormalTok{)}\SpecialCharTok{\%\textgreater{}\%}
  \FunctionTok{select}\NormalTok{(geometry)}

\FunctionTok{ggplot}\NormalTok{(moz\_sf) }\SpecialCharTok{+}
  \FunctionTok{geom\_sf\_inset}\NormalTok{() }\SpecialCharTok{+} 
  \FunctionTok{geom\_inset\_frame}\NormalTok{() }\SpecialCharTok{+} 
  \FunctionTok{coord\_sf\_inset}\NormalTok{(}
    \AttributeTok{inset =} \FunctionTok{configure\_inset}\NormalTok{(}
      \AttributeTok{centre =} \FunctionTok{st\_centroid}\NormalTok{(forma\_personalizada), }\CommentTok{\# Centraliza no condado}
      \AttributeTok{scale =} \DecValTok{2}\NormalTok{,                                 }
      \AttributeTok{shape =} \FunctionTok{shape\_sf}\NormalTok{(forma\_personalizada),}
      \AttributeTok{translation =} \FunctionTok{c}\NormalTok{(}\SpecialCharTok{{-}}\DecValTok{370}\NormalTok{, }\SpecialCharTok{{-}}\DecValTok{140}\NormalTok{)}
\NormalTok{    )}
\NormalTok{  ) }\SpecialCharTok{+}
  \FunctionTok{theme\_void}\NormalTok{()}
\end{Highlighting}
\end{Shaded}

\begin{figure}[H]

\centering{

\pandocbounded{\includegraphics[keepaspectratio]{fundEstspatial_files/figure-pdf/fig-moz-inset-basico-1.pdf}}

}

\caption{\label{fig-moz-inset-basico}Mapa de Moçambique com zoom da
Província de Maputo}

\end{figure}%

\begin{enumerate}
\def\labelenumi{\arabic{enumi}.}
\setcounter{enumi}{3}
\tightlist
\item
  \textbf{Moldura e Linhas de Conexão} (\texttt{geom\_inset\_frame})
\end{enumerate}

Para que o mapa seja legível, precisamos desenhar a borda do inset e as
linhas que o conectam ao local original (``burst lines''). A função
\texttt{geom\_inset\_frame()} faz isso automaticamente.

\begin{itemize}
\item
  \texttt{source.aes}: Estética do círculo original no mapa base (local
  de origem).
\item
  \texttt{target.aes}: Estética da moldura do zoom (local de destino).
\item
  \texttt{lines.aes}: Estética das linhas de conexão.
\end{itemize}

\begin{Shaded}
\begin{Highlighting}[]
\FunctionTok{p\_load}\NormalTok{(ggmapinset)}

\NormalTok{coluna\_nome }\OtherTok{\textless{}{-}} \ControlFlowTok{if}\NormalTok{(}\StringTok{"ADM1\_PT"} \SpecialCharTok{\%in\%} \FunctionTok{names}\NormalTok{(moz\_sf)) }\StringTok{"ADM1\_PT"} \ControlFlowTok{else} \StringTok{"NAME\_1"}

\FunctionTok{ggplot}\NormalTok{(moz\_sf) }\SpecialCharTok{+}
\NormalTok{  ggmapinset}\SpecialCharTok{::}\FunctionTok{geom\_sf\_inset}\NormalTok{(}\FunctionTok{aes}\NormalTok{(}\AttributeTok{fill =}\NormalTok{ .data[[coluna\_nome]]), }\AttributeTok{color =} \StringTok{"white"}\NormalTok{, }\AttributeTok{show.legend =} \ConstantTok{FALSE}\NormalTok{) }\SpecialCharTok{+}
\NormalTok{  ggmapinset}\SpecialCharTok{::}\FunctionTok{geom\_inset\_frame}\NormalTok{(}
    \AttributeTok{data =} \ConstantTok{NULL}\NormalTok{,}
    \AttributeTok{inherit.aes =} \ConstantTok{FALSE} 
\NormalTok{  ) }\SpecialCharTok{+}
\NormalTok{  ggmapinset}\SpecialCharTok{::}\FunctionTok{coord\_sf\_inset}\NormalTok{(}\AttributeTok{inset =}\NormalTok{ config\_maputo\_direita) }\SpecialCharTok{+}
  
  \FunctionTok{scale\_fill\_viridis\_d}\NormalTok{(}\AttributeTok{option =} \StringTok{"B"}\NormalTok{, }\AttributeTok{alpha =} \FloatTok{0.8}\NormalTok{) }\SpecialCharTok{+}
  \FunctionTok{theme\_minimal}\NormalTok{() }\SpecialCharTok{+}
  \FunctionTok{labs}\NormalTok{(}\AttributeTok{title =} \StringTok{"Destaque: Província de Maputo Cidade"}\NormalTok{)}
\end{Highlighting}
\end{Shaded}

\begin{figure}[H]

\centering{

\pandocbounded{\includegraphics[keepaspectratio]{fundEstspatial_files/figure-pdf/fig-moz-inset-frame-1.pdf}}

}

\caption{\label{fig-moz-inset-frame}Mapa com moldura e linhas de
conexão}

\end{figure}%

\begin{enumerate}
\def\labelenumi{\arabic{enumi}.}
\setcounter{enumi}{4}
\tightlist
\item
  \textbf{Rótulos e Texto} (\texttt{geom\_sf\_text\_inset})
\end{enumerate}

Adicionar texto em mapas com inset é um desafio: o texto deve aparecer
no local original ou no zoom?

O pacote fornece \texttt{geom\_sf\_text\_inset()} para lidar com isso.

\begin{itemize}
\item
  \texttt{where}: Define onde o texto aparece.
\item
  \texttt{"inset"}: O texto aparece dentro da área ampliada (no novo
  local).
\item
  \texttt{"base"}: O texto aparece no local original.
\end{itemize}

\begin{Shaded}
\begin{Highlighting}[]
\FunctionTok{ggplot}\NormalTok{(moz\_sf) }\SpecialCharTok{+}
\NormalTok{  ggmapinset}\SpecialCharTok{::}\FunctionTok{geom\_sf\_inset}\NormalTok{(}\FunctionTok{aes}\NormalTok{(}\AttributeTok{fill =}\NormalTok{ .data[[coluna\_nome]]), }\AttributeTok{color =} \StringTok{"white"}\NormalTok{, }\AttributeTok{show.legend =} \ConstantTok{FALSE}\NormalTok{) }\SpecialCharTok{+}
\NormalTok{  ggmapinset}\SpecialCharTok{::}\FunctionTok{geom\_inset\_frame}\NormalTok{(}
    \AttributeTok{data =} \ConstantTok{NULL}\NormalTok{,}
    \AttributeTok{inherit.aes =} \ConstantTok{FALSE} 
\NormalTok{  ) }\SpecialCharTok{+}
\NormalTok{  ggmapinset}\SpecialCharTok{::}\FunctionTok{coord\_sf\_inset}\NormalTok{(}\AttributeTok{inset =}\NormalTok{ config\_maputo\_direita) }\SpecialCharTok{+}
  
  \FunctionTok{scale\_fill\_viridis\_d}\NormalTok{(}\AttributeTok{option =} \StringTok{"B"}\NormalTok{, }\AttributeTok{alpha =} \FloatTok{0.8}\NormalTok{) }\SpecialCharTok{+}
  \FunctionTok{labs}\NormalTok{(}\AttributeTok{title =} \StringTok{""}\NormalTok{)}\SpecialCharTok{+}
  \FunctionTok{geom\_sf\_text\_inset}\NormalTok{(}
    \FunctionTok{aes}\NormalTok{(}\AttributeTok{label =}\NormalTok{ ADM1\_PT), }
    \AttributeTok{size =} \DecValTok{3}\NormalTok{, }
    \AttributeTok{color =} \StringTok{"black"}\NormalTok{,}
    \AttributeTok{check\_overlap =} \ConstantTok{TRUE}
\NormalTok{  ) }\SpecialCharTok{+}
    \FunctionTok{theme\_minimal}\NormalTok{()}\SpecialCharTok{+}
  \FunctionTok{theme}\NormalTok{(}
    \AttributeTok{axis.title =} \FunctionTok{element\_blank}\NormalTok{(), }\CommentTok{\# Remove títulos legenda eixo x e y}
    \CommentTok{\#axis.text = element\_blank(),  \# Remove os números das coordenadas}
    \AttributeTok{axis.ticks =} \FunctionTok{element\_blank}\NormalTok{(), }\CommentTok{\# Remove os tracinhos dos eixos (grade)}
    \CommentTok{\#panel.grid = element\_blank()  \# Remove as grades de fundo}
\NormalTok{  )}
\end{Highlighting}
\end{Shaded}

\begin{figure}[H]

\centering{

\pandocbounded{\includegraphics[keepaspectratio]{fundEstspatial_files/figure-pdf/fig-moz-inset-text-1.pdf}}

}

\caption{\label{fig-moz-inset-text}Rótulos aplicados no mapa}

\end{figure}%

\begin{enumerate}
\def\labelenumi{\arabic{enumi}.}
\setcounter{enumi}{5}
\tightlist
\item
  \textbf{Formas Alternativas} (\texttt{shape\_rectangle})
\end{enumerate}

Além de círculos, podemos usar retângulos. Vamos tentar um exemplo
diferente: dar um zoom na região central (Sofala/Beira).

\begin{Shaded}
\begin{Highlighting}[]
\CommentTok{\#Definir centro na Beira (aproximado)}
\NormalTok{centro\_beira }\OtherTok{\textless{}{-}}\NormalTok{ sf}\SpecialCharTok{::}\FunctionTok{st\_sfc}\NormalTok{(sf}\SpecialCharTok{::}\FunctionTok{st\_point}\NormalTok{(}\FunctionTok{c}\NormalTok{(}\FloatTok{34.8}\NormalTok{, }\SpecialCharTok{{-}}\FloatTok{19.8}\NormalTok{)), }\AttributeTok{crs =} \DecValTok{4326}\NormalTok{)}

\CommentTok{\#Configurar Retângulo}
\NormalTok{config\_retangulo }\OtherTok{\textless{}{-}} \FunctionTok{configure\_inset}\NormalTok{(}
  \AttributeTok{shape =} \FunctionTok{shape\_rectangle}\NormalTok{(}\AttributeTok{centre =}\NormalTok{ centro\_beira, }\AttributeTok{hwidth =} \DecValTok{150}\NormalTok{, }\AttributeTok{hheight =} \DecValTok{100}\NormalTok{), }
  \AttributeTok{scale =} \FloatTok{2.5}\NormalTok{,}
  \AttributeTok{translation =} \FunctionTok{c}\NormalTok{(}\DecValTok{700}\NormalTok{, }\SpecialCharTok{{-}}\DecValTok{140}\NormalTok{), }\CommentTok{\# Move 400km para o oceano}
  \AttributeTok{units =} \StringTok{"km"}
\NormalTok{)}

\FunctionTok{ggplot}\NormalTok{(moz\_sf) }\SpecialCharTok{+}
  \FunctionTok{geom\_sf\_inset}\NormalTok{(}\AttributeTok{fill =} \StringTok{"wheat"}\NormalTok{, }\AttributeTok{color =} \StringTok{"tan"}\NormalTok{) }\SpecialCharTok{+}
  \CommentTok{\# Estilizar a moldura do inset}
\NormalTok{  ggmapinset}\SpecialCharTok{::}\FunctionTok{geom\_inset\_frame}\NormalTok{(}
    \AttributeTok{data =} \ConstantTok{NULL}\NormalTok{,}
    \AttributeTok{inherit.aes =} \ConstantTok{FALSE} 
\NormalTok{  ) }\SpecialCharTok{+}
  \FunctionTok{coord\_sf\_inset}\NormalTok{(}\AttributeTok{inset =}\NormalTok{ config\_retangulo) }\SpecialCharTok{+}
  \FunctionTok{theme\_void}\NormalTok{() }\SpecialCharTok{+}
  \FunctionTok{labs}\NormalTok{(}\AttributeTok{title =} \StringTok{"Zoom na Região Centro (Sofala)"}\NormalTok{)}
\end{Highlighting}
\end{Shaded}

\begin{figure}[H]

\centering{

\pandocbounded{\includegraphics[keepaspectratio]{fundEstspatial_files/figure-pdf/fig-moz-inset-rect-1.pdf}}

}

\caption{\label{fig-moz-inset-rect}Inset retangular na região da Beira}

\end{figure}%

\begin{enumerate}
\def\labelenumi{\arabic{enumi}.}
\setcounter{enumi}{6}
\tightlist
\item
  \textbf{Transformação Manual} (\texttt{transform\_to\_inset})
\end{enumerate}

Em casos avançados, pode querer transformar os dados manualmente para
realizar cálculos na geometria transformada ou usar com outros pacotes
de plotagem.

\begin{itemize}
\tightlist
\item
  \texttt{transform\_to\_inset(x,\ inset)}: Recebe um objeto \texttt{sf}
  e a configuração, e retorna um novo objeto \texttt{sf} com as
  geometrias modificadas (deslocadas e ampliadas).
\end{itemize}

\begin{Shaded}
\begin{Highlighting}[]
\CommentTok{\# Cria um novo objeto sf com a geometria transformada}
\NormalTok{moz\_transformado }\OtherTok{\textless{}{-}} \FunctionTok{transform\_to\_inset}\NormalTok{(moz\_sf, config\_maputo)}

\CommentTok{\# Note que as coordenadas mudaram para acomodar o inset}
\CommentTok{\# Isso é útil para debugging ou análises personalizadas}
\FunctionTok{glimpse}\NormalTok{(moz\_transformado)}
\end{Highlighting}
\end{Shaded}

\begin{verbatim}
Rows: 11
Columns: 14
$ Shape_Leng <dbl> 19.276248, 14.641734, 17.581768, 18.685852, 9.614905, 1.821~
$ Shape_Area <dbl> 6.47773653, 6.65593496, 6.05059198, 5.34907555, 2.09040114,~
$ ADM1_PT    <chr> "Cabo Delgado", "Gaza", "Inhambane", "Manica", "Maputo", "M~
$ ADM1_PCODE <chr> "MZ01", "MZ02", "MZ03", "MZ04", "MZ05", "MZ06", "MZ07", "MZ~
$ ADM1_REF   <chr> NA, NA, NA, NA, NA, NA, NA, NA, NA, NA, NA
$ ADM1ALT1PT <chr> NA, NA, NA, NA, NA, NA, NA, NA, NA, NA, NA
$ ADM1ALT2PT <chr> NA, NA, NA, NA, NA, NA, NA, NA, NA, NA, NA
$ ADM0_EN    <chr> "Mozambique", "Mozambique", "Mozambique", "Mozambique", "Mo~
$ ADM0_PT    <chr> "Moçambique", "Moçambique", "Moçambique", "Moçambique", "Mo~
$ ADM0_PCODE <chr> "MZ", "MZ", "MZ", "MZ", "MZ", "MZ", "MZ", "MZ", "MZ", "MZ",~
$ date       <date> 2019-04-02, 2019-04-02, 2019-04-02, 2019-04-02, 2019-04-02,~
$ validOn    <date> 2019-06-07, 2019-06-07, 2019-06-07, 2019-06-07, 2019-06-07,~
$ validTo    <date> -1-11-30, -1-11-30, -1-11-30, -1-11-30, -1-11-30, -1-11-30,~
$ geometry   <MULTIPOLYGON [°]> MULTIPOLYGON (((65.58717 24..., MULTIPOLYGON (((33.48387 -~
\end{verbatim}

\begin{tcolorbox}[enhanced jigsaw, left=2mm, toptitle=1mm, colback=white, colframe=quarto-callout-tip-color-frame, colbacktitle=quarto-callout-tip-color!10!white, opacityback=0, rightrule=.15mm, bottomtitle=1mm, arc=.35mm, title=\textcolor{quarto-callout-tip-color}{\faLightbulb}\hspace{0.5em}{Controle Fino de Camadas}, titlerule=0mm, bottomrule=.15mm, leftrule=.75mm, coltitle=black, toprule=.15mm, breakable, opacitybacktitle=0.6]

O \texttt{geom\_sf\_inset} possui dois argumentos poderosos:
\texttt{map\_base} e \texttt{map\_inset}.

\begin{itemize}
\item
  \texttt{map\_base\ =\ "none"}: Desenha apenas o inset (zoom),
  escondendo o mapa original.
\item
  \texttt{map\_inset\ =\ "none"}: Desenha apenas o mapa original,
  escondendo o zoom.
\end{itemize}

Isso permite aplicar estilos diferentes (ex: cores diferentes) para a
base e para o zoom, chamando a função duas vezes com configurações
diferentes.

\end{tcolorbox}

\subsection{Pacote ggrepel}\label{pacote-ggrepel}

Uma das maiores frustrações ao criar gráficos no \texttt{ggplot2} é a
sobreposição de rótulos de texto (\emph{labels}), tornando a
visualização ilegível. O pacote \texttt{ggrepel} Slowikowski (2024),
desenvolvido por
\href{https://scholar.google.com/citations?user=kMP4830AAAAJ&hl=en}{Kamil
Slowikowski}, resolve este problema.

\textbf{Instalação e Carregamento}

\begin{Shaded}
\begin{Highlighting}[]
\ControlFlowTok{if}\NormalTok{ (}\SpecialCharTok{!}\FunctionTok{require}\NormalTok{(}\StringTok{"pacman"}\NormalTok{)) }\FunctionTok{install.packages}\NormalTok{(}\StringTok{"pacman"}\NormalTok{)}
\NormalTok{pacman}\SpecialCharTok{::}\FunctionTok{p\_load}\NormalTok{(ggrepel, ggplot2, dplyr)}
\end{Highlighting}
\end{Shaded}

Antes de usar o \texttt{ggrepel}, vamos ver o comportamento padrão do
\texttt{ggplot2} (\texttt{geom\_text}) quando tentamos rotular muitos
pontos. Usaremos o dataset \texttt{mtcars} para rotular os modelos de
carros.

\begin{Shaded}
\begin{Highlighting}[]
\FunctionTok{ggplot}\NormalTok{(mtcars, }\FunctionTok{aes}\NormalTok{(wt, mpg, }\AttributeTok{label =} \FunctionTok{rownames}\NormalTok{(mtcars))) }\SpecialCharTok{+}
  \FunctionTok{geom\_point}\NormalTok{(}\AttributeTok{color =} \StringTok{"red"}\NormalTok{) }\SpecialCharTok{+}
  \CommentTok{\# geom\_text padrão não evita sobreposição}
  \FunctionTok{geom\_text}\NormalTok{(}\AttributeTok{hjust =} \DecValTok{0}\NormalTok{, }\AttributeTok{vjust =} \DecValTok{0}\NormalTok{) }\SpecialCharTok{+}
  \FunctionTok{theme\_classic}\NormalTok{() }\SpecialCharTok{+}
  \FunctionTok{labs}\NormalTok{(}\AttributeTok{title =} \StringTok{"Sem ggrepel (Ilegível)"}\NormalTok{)}
\end{Highlighting}
\end{Shaded}

\begin{figure}[H]

\centering{

\pandocbounded{\includegraphics[keepaspectratio]{fundEstspatial_files/figure-pdf/fig-overlapping-text-1.pdf}}

}

\caption{\label{fig-overlapping-text}Texto sobreposto e ilegível com
geom\_text padrão}

\end{figure}%

\begin{enumerate}
\def\labelenumi{\arabic{enumi}.}
\tightlist
\item
  \textbf{Rótulos de Texto} (\texttt{geom\_text\_repel})
\end{enumerate}

Para corrigir o gráfico acima, substituímos \texttt{geom\_text()} por
\texttt{geom\_text\_repel()}. O pacote calculará automaticamente a
melhor posição para cada rótulo e desenhará linhas de conexão
(\emph{segments}) se necessário.

\begin{Shaded}
\begin{Highlighting}[]
\FunctionTok{ggplot}\NormalTok{(mtcars, }\FunctionTok{aes}\NormalTok{(wt, mpg, }\AttributeTok{label =} \FunctionTok{rownames}\NormalTok{(mtcars))) }\SpecialCharTok{+}
  \FunctionTok{geom\_point}\NormalTok{(}\AttributeTok{color =} \StringTok{"red"}\NormalTok{) }\SpecialCharTok{+}
  \CommentTok{\#aqui}
  \FunctionTok{geom\_text\_repel}\NormalTok{() }\SpecialCharTok{+}
  \FunctionTok{theme\_classic}\NormalTok{() }\SpecialCharTok{+}
  \FunctionTok{labs}\NormalTok{(}\AttributeTok{title =} \StringTok{"Com ggrepel (Legível)"}\NormalTok{)}
\end{Highlighting}
\end{Shaded}

\begin{figure}[H]

\centering{

\pandocbounded{\includegraphics[keepaspectratio]{fundEstspatial_files/figure-pdf/fig-ggrepel-text-1.pdf}}

}

\caption{\label{fig-ggrepel-text}Texto organizado com geom\_text\_repel}

\end{figure}%

\begin{enumerate}
\def\labelenumi{\arabic{enumi}.}
\setcounter{enumi}{1}
\tightlist
\item
  \textbf{Rótulos com Caixa} (\texttt{geom\_label\_repel})
\end{enumerate}

Se preferir que o texto fique dentro de uma caixa (para melhor contraste
com o fundo), use \texttt{geom\_label\_repel()}. Ele funciona de forma
idêntica ao \texttt{geom\_text\_repel}, mas desenha um retângulo atrás
do texto.

\begin{Shaded}
\begin{Highlighting}[]
\NormalTok{carros\_eficientes }\OtherTok{\textless{}{-}}\NormalTok{ mtcars }\SpecialCharTok{|\textgreater{}} 
  \FunctionTok{mutate}\NormalTok{(}\AttributeTok{nome =} \FunctionTok{rownames}\NormalTok{(mtcars)) }\SpecialCharTok{|\textgreater{}} 
  \FunctionTok{mutate}\NormalTok{(}\AttributeTok{label\_plot =} \FunctionTok{ifelse}\NormalTok{(mpg }\SpecialCharTok{\textgreater{}} \DecValTok{25}\NormalTok{, nome, }\StringTok{""}\NormalTok{))}

\FunctionTok{ggplot}\NormalTok{(carros\_eficientes, }\FunctionTok{aes}\NormalTok{(wt, mpg, }\AttributeTok{label =}\NormalTok{ label\_plot)) }\SpecialCharTok{+}
  \FunctionTok{geom\_point}\NormalTok{(}\AttributeTok{color =} \StringTok{"grey50"}\NormalTok{) }\SpecialCharTok{+}
  \CommentTok{\# Destacar os pontos eficientes}
  \FunctionTok{geom\_point}\NormalTok{(}\AttributeTok{data =} \FunctionTok{filter}\NormalTok{(carros\_eficientes, mpg }\SpecialCharTok{\textgreater{}} \DecValTok{25}\NormalTok{), }\AttributeTok{color =} \StringTok{"blue"}\NormalTok{) }\SpecialCharTok{+}
  
  \FunctionTok{geom\_label\_repel}\NormalTok{(}
    \AttributeTok{box.padding =} \FloatTok{0.5}\NormalTok{, }\CommentTok{\# Espaço ao redor da caixa}
    \AttributeTok{point.padding =} \FloatTok{0.5}\NormalTok{, }\CommentTok{\# Espaço entre a ponta da linha e o ponto}
    \AttributeTok{segment.color =} \StringTok{"grey50"} \CommentTok{\# Cor da linha de conexão}
\NormalTok{  ) }\SpecialCharTok{+}
  \FunctionTok{theme\_minimal}\NormalTok{()}
\end{Highlighting}
\end{Shaded}

\begin{figure}[H]

\centering{

\pandocbounded{\includegraphics[keepaspectratio]{fundEstspatial_files/figure-pdf/fig-ggrepel-label-1.pdf}}

}

\caption{\label{fig-ggrepel-label}Uso de geom\_label\_repel para
destaque}

\end{figure}%

\begin{enumerate}
\def\labelenumi{\arabic{enumi}.}
\setcounter{enumi}{2}
\tightlist
\item
  \textbf{Controle Fino} (\texttt{force}, \texttt{max.overlaps})
\end{enumerate}

\begin{itemize}
\item
  \texttt{force}: Controla a força de repulsão. Valores maiores afastam
  mais os rótulos.
\item
  \texttt{max.overlaps}: Por padrão, o \texttt{ggrepel} esconde rótulos
  se houver muita sobreposição. Aumente este valor (padrão é 10) se
  quiser forçar a exibição de todos os rótulos, mesmo que fique um pouco
  bagunçado.
\end{itemize}

\begin{Shaded}
\begin{Highlighting}[]
\FunctionTok{geom\_text\_repel}\NormalTok{(}
  \AttributeTok{force =} \DecValTok{2}\NormalTok{,           }\CommentTok{\# Mais força de repulsão}
  \AttributeTok{max.overlaps =} \ConstantTok{Inf}\NormalTok{,  }\CommentTok{\# Nunca esconder rótulos (Infinito)}
  \AttributeTok{min.segment.length =} \DecValTok{0} \CommentTok{\# Desenhar linha para todos os rótulos, mesmo os próximos}
\NormalTok{)}
\end{Highlighting}
\end{Shaded}

\begin{enumerate}
\def\labelenumi{\arabic{enumi}.}
\setcounter{enumi}{3}
\tightlist
\item
  \textbf{Uso em Mapas}
\end{enumerate}

O \texttt{ggrepel} integra-se perfeitamente com mapas
(\texttt{geom\_sf}). Para isso, usamos a geometria
\texttt{geom\_sf\_text()} ou \texttt{geom\_text\_repel} combinada com
\texttt{stat\_sf\_coordinates()}, que extrai automaticamente as
coordenadas X e Y dos centroides dos polígonos.

\begin{Shaded}
\begin{Highlighting}[]
\CommentTok{\#}
\NormalTok{pr\_destaque }\OtherTok{\textless{}{-}}\NormalTok{ pr\_mun }\SpecialCharTok{|\textgreater{}} \FunctionTok{filter}\NormalTok{(populacao }\SpecialCharTok{\textgreater{}} \DecValTok{200000}\NormalTok{)}

\FunctionTok{ggplot}\NormalTok{(pr\_mun) }\SpecialCharTok{+}
  \FunctionTok{geom\_sf}\NormalTok{(}\AttributeTok{fill =} \StringTok{"white"}\NormalTok{, }\AttributeTok{color =} \StringTok{"grey80"}\NormalTok{) }\SpecialCharTok{+}
  \FunctionTok{geom\_sf}\NormalTok{(}\AttributeTok{data =}\NormalTok{ pr\_destaque, }\AttributeTok{fill =} \StringTok{"orange"}\NormalTok{) }\SpecialCharTok{+}
  
  \CommentTok{\# Usar geom\_text\_repel }
  \FunctionTok{geom\_text\_repel}\NormalTok{(}
    \AttributeTok{data =}\NormalTok{ pr\_destaque,}
    \FunctionTok{aes}\NormalTok{(}\AttributeTok{label =}\NormalTok{ name\_muni, }\AttributeTok{geometry =}\NormalTok{ geom),}
    \AttributeTok{stat =} \StringTok{"sf\_coordinates"}\NormalTok{, }\CommentTok{\# Calcula o X/Y do centroide automaticamente}
    \AttributeTok{min.segment.length =} \DecValTok{0}\NormalTok{,}
    \AttributeTok{box.padding =} \DecValTok{1}\NormalTok{,}
    \AttributeTok{size =} \DecValTok{3}
\NormalTok{  ) }\SpecialCharTok{+}
  \FunctionTok{theme\_void}\NormalTok{() }\SpecialCharTok{+}
  \FunctionTok{labs}\NormalTok{(}\AttributeTok{title =} \StringTok{"Municípios em Destaque no Paraná"}\NormalTok{)}
\end{Highlighting}
\end{Shaded}

\begin{figure}[H]

\centering{

\pandocbounded{\includegraphics[keepaspectratio]{fundEstspatial_files/figure-pdf/fig-ggrepel-mapa-1.pdf}}

}

\caption{\label{fig-ggrepel-mapa}Rótulos em mapas com ggrepel}

\end{figure}%

\begin{tcolorbox}[enhanced jigsaw, left=2mm, toptitle=1mm, colback=white, colframe=quarto-callout-tip-color-frame, colbacktitle=quarto-callout-tip-color!10!white, opacityback=0, rightrule=.15mm, bottomtitle=1mm, arc=.35mm, title=\textcolor{quarto-callout-tip-color}{\faLightbulb}\hspace{0.5em}{Segmentos e Setas}, titlerule=0mm, bottomrule=.15mm, leftrule=.75mm, coltitle=black, toprule=.15mm, breakable, opacitybacktitle=0.6]

Você pode estilizar as linhas que conectam o texto ao ponto usando
argumentos como \texttt{arrow}, \texttt{segment.size},
\texttt{segment.color} e \texttt{curvature} (para linhas curvas). Isso é
útil para criar anotações elegantes em gráficos de publicação.

\end{tcolorbox}

\subsection{Pacote ggspatial}\label{pacote-ggspatial}

O \texttt{ggplot2} é excelente para plotar geometrias, mas falta-lhe
``vocabulário cartográfico'' nativo. O pacote \texttt{ggspatial}
Dunnington (2025), desenvolvido por
\href{https://scholar.google.com/citations?user=Ik_72RsAAAAJ&hl=en}{Dewey
Dunnington}, preenche essa lacuna fornecendo geometrias e anotações
específicas para mapas.

Ele permite adicionar facilmente barras de escala, setas de norte e,
crucialmente, mapas de fundo (basemaps) provenientes de serviços como
OpenStreetMap, sem sair da sintaxe do \texttt{ggplot}.

\textbf{Instalação e Carregamento}

Para este exemplo, usaremos os dados do estado do Rio de Janeiro
provenientes do pacote \texttt{geobr}.

\begin{Shaded}
\begin{Highlighting}[]
\ControlFlowTok{if}\NormalTok{ (}\SpecialCharTok{!}\FunctionTok{require}\NormalTok{(}\StringTok{"pacman"}\NormalTok{)) }\FunctionTok{install.packages}\NormalTok{(}\StringTok{"pacman"}\NormalTok{)}
\NormalTok{pacman}\SpecialCharTok{::}\FunctionTok{p\_load}\NormalTok{(ggspatial, ggplot2, sf, terra, geobr, prettymapr, rosm)}

\CommentTok{\# Carregar dados do Rio de Janeiro}
\NormalTok{rj\_mun }\OtherTok{\textless{}{-}} \FunctionTok{read\_municipality}\NormalTok{(}\AttributeTok{code\_muni =} \StringTok{"RJ"}\NormalTok{, }\AttributeTok{year =} \DecValTok{2020}\NormalTok{, }\AttributeTok{showProgress =} \ConstantTok{FALSE}\NormalTok{)}
\end{Highlighting}
\end{Shaded}

\begin{enumerate}
\def\labelenumi{\arabic{enumi}.}
\tightlist
\item
  \textbf{Mapas Base (Basemaps)} (\texttt{annotation\_map\_tile})
\end{enumerate}

Uma das funcionalidades mais desejadas é adicionar um mapa de ruas ou
satélite como fundo. A função \texttt{annotation\_map\_tile()} faz isso
automaticamente, baixando as ``telhas'' (\emph{tiles}) necessárias para
a área do seu gráfico.

\begin{itemize}
\item
  \texttt{type}: O tipo de mapa. Padrão é \texttt{"osm"}
  (\href{https://www.openstreetmap.org/\#map=5/-15.13/-53.19}{OpenStreetMap}).
  Outras opções comuns incluem \texttt{"cartolight"},
  \texttt{"cartodark"}, \texttt{"hotstyle"}.
\item
  \texttt{zoom}: Nível de detalhe. Se omitido, o pacote calcula
  automaticamente. Zooms altos baixam muitos arquivos.
\item
  \texttt{progress}: Defina como \texttt{"none"} para evitar poluição
  visual no console durante o download.
\end{itemize}

\begin{Shaded}
\begin{Highlighting}[]
\FunctionTok{ggplot}\NormalTok{(rj\_mun) }\SpecialCharTok{+}
  \CommentTok{\#Adicionar o mapa base (deve vir primeiro para ficar no fundo)}
  \FunctionTok{annotation\_map\_tile}\NormalTok{(}\AttributeTok{type =} \StringTok{"osm"}\NormalTok{, }\AttributeTok{progress =} \StringTok{"none"}\NormalTok{) }\SpecialCharTok{+}
  
  \CommentTok{\#Adicionar os dados vetoriais por cima (com transparência)}
  \FunctionTok{geom\_sf}\NormalTok{(}\AttributeTok{fill =} \StringTok{"orange"}\NormalTok{, }\AttributeTok{alpha =} \FloatTok{0.4}\NormalTok{, }\AttributeTok{color =} \StringTok{"black"}\NormalTok{, }\AttributeTok{size =} \FloatTok{0.1}\NormalTok{) }\SpecialCharTok{+}
  
  \FunctionTok{theme\_minimal}\NormalTok{()}
\end{Highlighting}
\end{Shaded}

\begin{figure}[H]

\centering{

\pandocbounded{\includegraphics[keepaspectratio]{fundEstspatial_files/figure-pdf/fig-ggspatial-basemap-1.pdf}}

}

\caption{\label{fig-ggspatial-basemap}Municípios do RJ sobre base do
OpenStreetMap}

\end{figure}%

\begin{enumerate}
\def\labelenumi{\arabic{enumi}.}
\setcounter{enumi}{1}
\tightlist
\item
  \textbf{Barra de Escala (\texttt{annotation\_scale})}
\end{enumerate}

Diferente do \texttt{mapsf}, o \texttt{ggspatial} adiciona a escala como
uma camada (\texttt{layer}) do \texttt{ggplot}.

\begin{itemize}
\item
  \texttt{location}: Posição (``tl'' = top-left, ``br'' = bottom-right,
  etc.).
\item
  \texttt{width\_hint}: Proporção da largura do mapa que a escala deve
  ocupar (ex: 0.2 para 20\%).
\item
  \texttt{style}: Estilo visual (``bar'' ou ``ticks'').
\end{itemize}

\begin{Shaded}
\begin{Highlighting}[]
\FunctionTok{ggplot}\NormalTok{(rj\_mun) }\SpecialCharTok{+}
    \FunctionTok{annotation\_map\_tile}\NormalTok{(}\AttributeTok{type =} \StringTok{"osm"}\NormalTok{, }\AttributeTok{progress =} \StringTok{"none"}\NormalTok{) }\SpecialCharTok{+}
  \FunctionTok{geom\_sf}\NormalTok{(}\AttributeTok{fill =} \StringTok{"orange"}\NormalTok{, }\AttributeTok{color =} \StringTok{"white"}\NormalTok{) }\SpecialCharTok{+}
  
  \FunctionTok{annotation\_scale}\NormalTok{(}
    \AttributeTok{location =} \StringTok{"br"}\NormalTok{,    }\CommentTok{\# Bottom right (Canto inferior direito)}
    \AttributeTok{width\_hint =} \FloatTok{0.3}\NormalTok{,   }\CommentTok{\# Tamanho relativo}
    \AttributeTok{style =} \StringTok{"bar"}       \CommentTok{\# Estilo barra sólida}
\NormalTok{  ) }\SpecialCharTok{+}
  
  \FunctionTok{theme\_minimal}\NormalTok{()}
\end{Highlighting}
\end{Shaded}

\begin{figure}[H]

\centering{

\pandocbounded{\includegraphics[keepaspectratio]{fundEstspatial_files/figure-pdf/fig-ggspatial-scale-1.pdf}}

}

\caption{\label{fig-ggspatial-scale}Adicionando barra de escala ao mapa
do RJ}

\end{figure}%

\begin{enumerate}
\def\labelenumi{\arabic{enumi}.}
\setcounter{enumi}{2}
\tightlist
\item
  \textbf{Seta Norte (\texttt{annotation\_north\_arrow})}
\end{enumerate}

Adiciona a rosa dos ventos ou seta norte.

\begin{itemize}
\item
  \texttt{location}: Posição.
\item
  \texttt{which\_north}: \texttt{"true"} (Norte geográfico) ou
  \texttt{"grid"} (Norte da grade).
\item
  \texttt{style}: Estilo visual. O pacote oferece várias funções de
  estilo:
\item
  \texttt{north\_arrow\_orienteering} (Padrão)
\item
  \texttt{north\_arrow\_fancy\_orienteering} (Mais elaborado)
\item
  \texttt{north\_arrow\_minimal} (Simples)
\item
  \texttt{north\_arrow\_nautical} (Estilo náutico)
\end{itemize}

\begin{Shaded}
\begin{Highlighting}[]
\FunctionTok{ggplot}\NormalTok{(rj\_mun) }\SpecialCharTok{+}
    \FunctionTok{annotation\_map\_tile}\NormalTok{(}\AttributeTok{type =} \StringTok{"osm"}\NormalTok{, }\AttributeTok{progress =} \StringTok{"none"}\NormalTok{) }\SpecialCharTok{+}
  \FunctionTok{geom\_sf}\NormalTok{(}\AttributeTok{fill =} \StringTok{"orange"}\NormalTok{, }\AttributeTok{color =} \StringTok{"white"}\NormalTok{) }\SpecialCharTok{+}
  
  \FunctionTok{annotation\_scale}\NormalTok{(}
    \AttributeTok{location =} \StringTok{"br"}\NormalTok{,    }
    \AttributeTok{width\_hint =} \FloatTok{0.3}\NormalTok{,   }
    \AttributeTok{style =} \StringTok{"bar"}       
\NormalTok{  ) }\SpecialCharTok{+}
  \CommentTok{\# Norte Simples no topo esquerdo}
  \FunctionTok{annotation\_north\_arrow}\NormalTok{(}
    \AttributeTok{location =} \StringTok{"tl"}\NormalTok{, }
    \AttributeTok{which\_north =} \StringTok{"true"}\NormalTok{,}
    \AttributeTok{height =} \FunctionTok{unit}\NormalTok{(}\DecValTok{1}\NormalTok{, }\StringTok{"cm"}\NormalTok{),}
    \AttributeTok{width =} \FunctionTok{unit}\NormalTok{(}\DecValTok{1}\NormalTok{, }\StringTok{"cm"}\NormalTok{),}
    \AttributeTok{style =} \FunctionTok{north\_arrow\_minimal}\NormalTok{()}
\NormalTok{  ) }\SpecialCharTok{+}
  
  \CommentTok{\# Norte Náutico no fundo direito (apenas demonstração)}
  \FunctionTok{annotation\_north\_arrow}\NormalTok{(}
    \AttributeTok{location =} \StringTok{"br"}\NormalTok{, }
    \AttributeTok{style =} \FunctionTok{north\_arrow\_nautical}\NormalTok{(),}
    \AttributeTok{pad\_y=}\FunctionTok{unit}\NormalTok{(}\FloatTok{0.55}\NormalTok{, }\StringTok{"cm"}\NormalTok{)}
\NormalTok{  ) }\SpecialCharTok{+}
  \FunctionTok{theme\_minimal}\NormalTok{()}
\end{Highlighting}
\end{Shaded}

\begin{figure}[H]

\centering{

\pandocbounded{\includegraphics[keepaspectratio]{fundEstspatial_files/figure-pdf/fig-ggspatial-north-1.pdf}}

}

\caption{\label{fig-ggspatial-north}Estilos de Seta Norte}

\end{figure}%

\begin{enumerate}
\def\labelenumi{\arabic{enumi}.}
\setcounter{enumi}{3}
\tightlist
\item
  \textbf{Trabalhando com Rasters (\texttt{layer\_spatial})}
\end{enumerate}

O \texttt{ggplot2} nativo (\texttt{geom\_raster}) exige que convertamos
imagens para data.frames, o que é lento e consome memória. O
\texttt{ggspatial} fornece a função \texttt{layer\_spatial()}, que
aceita objetos \texttt{SpatRaster} (do pacote \texttt{terra}) ou
\texttt{stars} e os plota corretamente, lidando com projeções
automaticamente.

\begin{Shaded}
\begin{Highlighting}[]
\FunctionTok{p\_load}\NormalTok{(terra, sf, ggplot2,ggspatial,geobr)}

\NormalTok{rj\_mun }\OtherTok{\textless{}{-}} \FunctionTok{read\_municipality}\NormalTok{(}\AttributeTok{code\_muni =} \StringTok{"RJ"}\NormalTok{, }\AttributeTok{year =} \DecValTok{2020}\NormalTok{, }\AttributeTok{showProgress =} \ConstantTok{FALSE}\NormalTok{)}
\NormalTok{rj\_estado }\OtherTok{\textless{}{-}} \FunctionTok{st\_union}\NormalTok{(rj\_mun)}

\NormalTok{v\_rj }\OtherTok{\textless{}{-}} \FunctionTok{vect}\NormalTok{(rj\_estado) }

\NormalTok{r }\OtherTok{\textless{}{-}} \FunctionTok{rast}\NormalTok{(}\FunctionTok{ext}\NormalTok{(v\_rj), }\AttributeTok{resolution =} \FloatTok{0.01}\NormalTok{, }\AttributeTok{crs =} \FunctionTok{st\_crs}\NormalTok{(rj\_mun)}\SpecialCharTok{$}\NormalTok{wkt)}
\FunctionTok{values}\NormalTok{(r) }\OtherTok{\textless{}{-}} \FunctionTok{runif}\NormalTok{(}\FunctionTok{ncell}\NormalTok{(r), }\DecValTok{0}\NormalTok{, }\DecValTok{1000}\NormalTok{)}

\CommentTok{\#Recorte}
\NormalTok{r\_rj }\OtherTok{\textless{}{-}} \FunctionTok{crop}\NormalTok{(r, v\_rj)}
\NormalTok{r\_rj }\OtherTok{\textless{}{-}} \FunctionTok{mask}\NormalTok{(r\_rj, v\_rj)}

\FunctionTok{ggplot}\NormalTok{() }\SpecialCharTok{+}
  \FunctionTok{annotation\_map\_tile}\NormalTok{(}\AttributeTok{type =} \StringTok{"osm"}\NormalTok{, }\AttributeTok{zoom =} \DecValTok{8}\NormalTok{, }\AttributeTok{progress =} \StringTok{"none"}\NormalTok{) }\SpecialCharTok{+}
  \FunctionTok{layer\_spatial}\NormalTok{(r\_rj, }\AttributeTok{alpha =} \FloatTok{0.8}\NormalTok{) }\SpecialCharTok{+}
  \FunctionTok{scale\_fill\_viridis\_c}\NormalTok{(}\AttributeTok{name =} \StringTok{"Altitude (m)"}\NormalTok{, }\AttributeTok{option =} \StringTok{"B"}\NormalTok{, }\AttributeTok{na.value =} \ConstantTok{NA}\NormalTok{) }\SpecialCharTok{+}
  \FunctionTok{geom\_sf}\NormalTok{(}\AttributeTok{data =}\NormalTok{ rj\_estado, }\AttributeTok{fill =} \ConstantTok{NA}\NormalTok{, }\AttributeTok{color =} \StringTok{"black"}\NormalTok{, }\AttributeTok{size =} \FloatTok{0.8}\NormalTok{) }\SpecialCharTok{+}
  \FunctionTok{annotation\_scale}\NormalTok{(}
    \AttributeTok{location =} \StringTok{"br"}\NormalTok{,    }
    \AttributeTok{width\_hint =} \FloatTok{0.3}\NormalTok{,   }
    \AttributeTok{style =} \StringTok{"bar"}       
\NormalTok{  ) }\SpecialCharTok{+}
    \FunctionTok{annotation\_north\_arrow}\NormalTok{(}
    \AttributeTok{location =} \StringTok{"br"}\NormalTok{, }
    \AttributeTok{style =} \FunctionTok{north\_arrow\_nautical}\NormalTok{(),}
    \AttributeTok{pad\_y=}\FunctionTok{unit}\NormalTok{(}\FloatTok{0.55}\NormalTok{, }\StringTok{"cm"}\NormalTok{)}
\NormalTok{  ) }\SpecialCharTok{+}
  \FunctionTok{theme\_minimal}\NormalTok{()}
\end{Highlighting}
\end{Shaded}

\begin{figure}[H]

\centering{

\pandocbounded{\includegraphics[keepaspectratio]{fundEstspatial_files/figure-pdf/fig-recorte-rj-fixed-1.pdf}}

}

\caption{\label{fig-recorte-rj-fixed}Plotando raster diretamente com
layer\_spatial}

\end{figure}%

\begin{tcolorbox}[enhanced jigsaw, left=2mm, toptitle=1mm, colback=white, colframe=quarto-callout-tip-color-frame, colbacktitle=quarto-callout-tip-color!10!white, opacityback=0, rightrule=.15mm, bottomtitle=1mm, arc=.35mm, title=\textcolor{quarto-callout-tip-color}{\faLightbulb}\hspace{0.5em}{Dica}, titlerule=0mm, bottomrule=.15mm, leftrule=.75mm, coltitle=black, toprule=.15mm, breakable, opacitybacktitle=0.6]

Ao usar \texttt{annotation\_map\_tile}, o R baixa as imagens da internet
a cada plotagem se elas não estiverem em cache. Se o servidor de tiles
(ex: OSM) estiver lento, tente outro \texttt{type} como
\texttt{"cartolight"} ou \texttt{"hotstyle"}. Certifique-se também de
que seus dados \texttt{sf} tenham um CRS definido (\texttt{st\_crs()}).

\end{tcolorbox}

\subsection{\texorpdfstring{Mapas Interativos na Web: O pacote
\texttt{leaflet}}{Mapas Interativos na Web: O pacote leaflet}}\label{mapas-interativos-na-web-o-pacote-leaflet}

O pacote \texttt{leaflet} Cheng et al. (2025) é a ponte entre a
linguagem R e a biblioteca JavaScript homônima, que é o padrão da
indústria para cartografia web. Diferente do \texttt{mapsf} ou
\texttt{ggplot2}, que geram uma imagem estática (como uma foto), o
\texttt{leaflet} gera um \emph{widget HTML}.

\begin{enumerate}
\def\labelenumi{\arabic{enumi}.}
\item
  \texttt{Tiles} (Base): O fundo do mapa (ruas, satélite). São imagens
  estáticas carregadas da internet.
\item
  \texttt{Markers}/Polygons (Vetores): Seus dados desenhados sobre a
  base.
\item
  \texttt{Popups/Labels}: Interatividade que surge ao clicar ou passar o
  mouse.
\item
  \texttt{Controls:} Elementos de interface (zoom, seletor de camadas,
  legenda).
\end{enumerate}

\textbf{Instalação e Dados (SP)}

\begin{Shaded}
\begin{Highlighting}[]
\ControlFlowTok{if}\NormalTok{ (}\SpecialCharTok{!}\FunctionTok{require}\NormalTok{(}\StringTok{"pacman"}\NormalTok{)) }\FunctionTok{install.packages}\NormalTok{(}\StringTok{"pacman"}\NormalTok{)}
\NormalTok{pacman}\SpecialCharTok{::}\FunctionTok{p\_load}\NormalTok{(leaflet, sf, dplyr, geobr)}

\CommentTok{\#Base Cartográfica: Região Metropolitana de SP}
\ControlFlowTok{if}\NormalTok{ (}\SpecialCharTok{!}\FunctionTok{exists}\NormalTok{(}\StringTok{"rm\_sp"}\NormalTok{)) \{}
\NormalTok{  sp\_mun }\OtherTok{\textless{}{-}} \FunctionTok{read\_municipality}\NormalTok{(}\AttributeTok{code\_muni =} \StringTok{"SP"}\NormalTok{, }\AttributeTok{year =} \DecValTok{2020}\NormalTok{, }\AttributeTok{showProgress =} \ConstantTok{FALSE}\NormalTok{)}
\NormalTok{  capital }\OtherTok{\textless{}{-}}\NormalTok{ sp\_mun[sp\_mun}\SpecialCharTok{$}\NormalTok{name\_muni }\SpecialCharTok{==} \StringTok{"São Paulo"}\NormalTok{, ]}
\NormalTok{  vizinhos }\OtherTok{\textless{}{-}} \FunctionTok{st\_filter}\NormalTok{(sp\_mun, capital, }\AttributeTok{.predicate =}\NormalTok{ st\_touches)}
\NormalTok{  rm\_sp }\OtherTok{\textless{}{-}} \FunctionTok{rbind}\NormalTok{(capital, vizinhos)}
\NormalTok{\}}

\CommentTok{\#Escolas (INEP)}
\NormalTok{escolas\_br }\OtherTok{\textless{}{-}} \FunctionTok{read\_schools}\NormalTok{(}\AttributeTok{year =} \DecValTok{2020}\NormalTok{, }\AttributeTok{showProgress =} \ConstantTok{FALSE}\NormalTok{)}

\NormalTok{escolas\_rm }\OtherTok{\textless{}{-}} \FunctionTok{st\_filter}\NormalTok{(escolas\_br, rm\_sp) }\SpecialCharTok{|\textgreater{}} 
  \FunctionTok{select}\NormalTok{(}
    \AttributeTok{nome =}\NormalTok{ name\_school,}
    \AttributeTok{endereco =}\NormalTok{ address,}
    \AttributeTok{nivel =}\NormalTok{ education\_level,}
    \AttributeTok{admin =}\NormalTok{ admin\_category}
\NormalTok{  ) }\SpecialCharTok{|\textgreater{}} 
  \FunctionTok{filter}\NormalTok{(}\SpecialCharTok{!}\FunctionTok{is.na}\NormalTok{(nome), }\SpecialCharTok{!}\FunctionTok{is.na}\NormalTok{(geom))}

\FunctionTok{leaflet}\NormalTok{(escolas\_rm) }\SpecialCharTok{|\textgreater{}} 
  \CommentTok{\# Mapas Base}
  \FunctionTok{addTiles}\NormalTok{(}\AttributeTok{group =} \StringTok{"OSM (Rua)"}\NormalTok{) }\SpecialCharTok{|\textgreater{}} 
  \FunctionTok{addProviderTiles}\NormalTok{(providers}\SpecialCharTok{$}\NormalTok{CartoDB.Positron, }\AttributeTok{group =} \StringTok{"Clean (Claro)"}\NormalTok{) }\SpecialCharTok{|\textgreater{}} 
  
  \CommentTok{\# Camada de Contexto (Municípios)}
  \FunctionTok{addPolygons}\NormalTok{(}
    \AttributeTok{data =}\NormalTok{ rm\_sp,}
    \AttributeTok{color =} \StringTok{"\#444"}\NormalTok{, }\AttributeTok{weight =} \DecValTok{1}\NormalTok{, }\AttributeTok{fillOpacity =} \FloatTok{0.1}\NormalTok{,}
    \AttributeTok{group =} \StringTok{"Limites Municipais"}\NormalTok{,}
    \AttributeTok{label =} \SpecialCharTok{\textasciitilde{}}\NormalTok{name\_muni}
\NormalTok{  ) }\SpecialCharTok{|\textgreater{}} 
  
  \CommentTok{\# Camada de Escolas (Clusters)}
  \FunctionTok{addCircleMarkers}\NormalTok{(}
    \AttributeTok{radius =} \DecValTok{5}\NormalTok{,}
    \AttributeTok{color =} \StringTok{"\#2E86C1"}\NormalTok{, }\CommentTok{\# Azul}
    \AttributeTok{stroke =} \ConstantTok{FALSE}\NormalTok{, }\AttributeTok{fillOpacity =} \FloatTok{0.8}\NormalTok{,}
    
    \CommentTok{\# Popup Rico com dados reais da escola}
    \AttributeTok{popup =} \SpecialCharTok{\textasciitilde{}}\FunctionTok{paste0}\NormalTok{(}
      \StringTok{"\textless{}b\textgreater{}Escola:\textless{}/b\textgreater{} "}\NormalTok{, nome, }\StringTok{"\textless{}br\textgreater{}"}\NormalTok{,}
      \StringTok{"\textless{}b\textgreater{}Nível:\textless{}/b\textgreater{} "}\NormalTok{, nivel, }\StringTok{"\textless{}br\textgreater{}"}\NormalTok{,}
      \StringTok{"\textless{}b\textgreater{}Administração:\textless{}/b\textgreater{} "}\NormalTok{, admin, }\StringTok{"\textless{}br\textgreater{}"}\NormalTok{,}
      \StringTok{"\textless{}b\textgreater{}Endereço:\textless{}/b\textgreater{} "}\NormalTok{, endereco}
\NormalTok{    ),}
    
    \CommentTok{\# Clusterização Automática (Essencial para muitos pontos)}
    \AttributeTok{clusterOptions =} \FunctionTok{markerClusterOptions}\NormalTok{(),}
    \AttributeTok{group =} \StringTok{"Escolas"}
\NormalTok{  ) }\SpecialCharTok{|\textgreater{}} 
  
  \CommentTok{\# Controle de Camadas}
  \FunctionTok{addLayersControl}\NormalTok{(}
    \AttributeTok{baseGroups =} \FunctionTok{c}\NormalTok{(}\StringTok{"Clean (Claro)"}\NormalTok{, }\StringTok{"OSM (Rua)"}\NormalTok{),}
    \AttributeTok{overlayGroups =} \FunctionTok{c}\NormalTok{(}\StringTok{"Escolas"}\NormalTok{, }\StringTok{"Limites Municipais"}\NormalTok{),}
    \AttributeTok{options =} \FunctionTok{layersControlOptions}\NormalTok{(}\AttributeTok{collapsed =} \ConstantTok{FALSE}\NormalTok{)}
\NormalTok{  )}
\end{Highlighting}
\end{Shaded}

\begin{figure}[H]

\centering{

\pandocbounded{\includegraphics[keepaspectratio]{fundEstspatial_files/figure-pdf/fig-leaflet-real-fixed-1.pdf}}

}

\caption{\label{fig-leaflet-real-fixed}Distribuição Real de Escolas na
RM de São Paulo (Dados INEP/geobr)}

\end{figure}%

\begin{enumerate}
\def\labelenumi{\arabic{enumi}.}
\setcounter{enumi}{1}
\tightlist
\item
  \textbf{O Mapa Base e a Sintaxe Pipe}
  (\texttt{\textbar{}\textgreater{}})
\end{enumerate}

\begin{itemize}
\item
  \texttt{leaflet()}: Instancia o objeto do mapa.
\item
  \texttt{addTiles()}: Adiciona o mapa base padrão (OpenStreetMap).
\item
  \texttt{setView()}: (Opcional) Define onde a câmera do mapa começa
  posicionada.
\end{itemize}

\begin{Shaded}
\begin{Highlighting}[]
\FunctionTok{leaflet}\NormalTok{() }\SpecialCharTok{|\textgreater{}} 
  \FunctionTok{addTiles}\NormalTok{() }\SpecialCharTok{|\textgreater{}}  \CommentTok{\# Padrão: OpenStreetMap (estilo \textquotesingle{}ruas\textquotesingle{})}
  \FunctionTok{setView}\NormalTok{(}\AttributeTok{lng =} \SpecialCharTok{{-}}\FloatTok{46.63}\NormalTok{, }\AttributeTok{lat =} \SpecialCharTok{{-}}\FloatTok{23.55}\NormalTok{, }\AttributeTok{zoom =} \DecValTok{9}\NormalTok{)}
\end{Highlighting}
\end{Shaded}

\begin{figure}[H]

\centering{

\pandocbounded{\includegraphics[keepaspectratio]{fundEstspatial_files/figure-pdf/fig-leaflet-basico-1.pdf}}

}

\caption{\label{fig-leaflet-basico}Mapa base inicial (OpenStreetMap)}

\end{figure}%

\begin{enumerate}
\def\labelenumi{\arabic{enumi}.}
\setcounter{enumi}{2}
\tightlist
\item
  \textbf{Polígonos e Mapas Coropléticos}
\end{enumerate}

Para colorir os municípios baseados em dados (ex: Área), precisamos
mapear a variável numérica para uma paleta de cores.

\begin{tcolorbox}[enhanced jigsaw, left=2mm, toptitle=1mm, colback=white, colframe=quarto-callout-important-color-frame, colbacktitle=quarto-callout-important-color!10!white, opacityback=0, rightrule=.15mm, bottomtitle=1mm, arc=.35mm, title=\textcolor{quarto-callout-important-color}{\faExclamation}\hspace{0.5em}{Operador Til (\textasciitilde)}, titlerule=0mm, bottomrule=.15mm, leftrule=.75mm, coltitle=black, toprule=.15mm, breakable, opacitybacktitle=0.6]

No \texttt{leaflet}, diferentemente do \texttt{ggplot2} (que usa
\texttt{aes()}), usamos o símbolo \texttt{\textasciitilde{}} (til) para
indicar que um argumento deve ser lido de dentro dos dados.

\begin{itemize}
\item
  \texttt{color\ =\ "red"}: Todos os polígonos ficam vermelhos.
\item
  \texttt{color\ =\ \textasciitilde{}paleta(variavel)}: A cor depende da
  variável de cada polígono.
\end{itemize}

\end{tcolorbox}

\begin{Shaded}
\begin{Highlighting}[]
\CommentTok{\# Definir a paleta de cores (Quantis)}
\NormalTok{rm\_sp}\SpecialCharTok{$}\NormalTok{area\_km2 }\OtherTok{\textless{}{-}} \FunctionTok{as.numeric}\NormalTok{(}\FunctionTok{st\_area}\NormalTok{(rm\_sp)) }\SpecialCharTok{/} \FloatTok{1e6}
\NormalTok{pal\_area }\OtherTok{\textless{}{-}} \FunctionTok{colorQuantile}\NormalTok{(}\StringTok{"YlOrRd"}\NormalTok{, }\AttributeTok{domain =}\NormalTok{ rm\_sp}\SpecialCharTok{$}\NormalTok{area\_km2, }\AttributeTok{n =} \DecValTok{5}\NormalTok{)}

\NormalTok{labels\_html }\OtherTok{\textless{}{-}} \FunctionTok{paste0}\NormalTok{(}
  \StringTok{"\textless{}b\textgreater{}"}\NormalTok{, rm\_sp}\SpecialCharTok{$}\NormalTok{name\_muni, }\StringTok{"\textless{}/b\textgreater{}: "}\NormalTok{, }
  \FunctionTok{round}\NormalTok{(rm\_sp}\SpecialCharTok{$}\NormalTok{area\_km2), }\StringTok{" km\textless{}sup\textgreater{}2\textless{}/sup\textgreater{}"}
\NormalTok{) }\SpecialCharTok{|\textgreater{}} 
  \FunctionTok{lapply}\NormalTok{(htmltools}\SpecialCharTok{::}\NormalTok{HTML) }\CommentTok{\# Importante: Converte texto para objeto HTML}


\FunctionTok{leaflet}\NormalTok{(rm\_sp) }\SpecialCharTok{|\textgreater{}} 
  \CommentTok{\#ESCOLHA DO MAPA BASE}
  \FunctionTok{addProviderTiles}\NormalTok{(providers}\SpecialCharTok{$}\NormalTok{CartoDB.Positron) }\SpecialCharTok{|\textgreater{}}  \CommentTok{\#tem outras além de CartoDB.Positron}
  \CommentTok{\#CAMADA DE POLÍGONOS}
  \FunctionTok{addPolygons}\NormalTok{(}
    \AttributeTok{fillColor =} \SpecialCharTok{\textasciitilde{}}\FunctionTok{pal\_area}\NormalTok{(area\_km2), }\CommentTok{\# O til (\textasciitilde{}) mapeia a coluna aos dados}
    \AttributeTok{weight =} \DecValTok{1}\NormalTok{,         }
    \AttributeTok{color =} \StringTok{"white"}\NormalTok{,     }\CommentTok{\# Cor da linha de contorno}
    \AttributeTok{fillOpacity =} \FloatTok{0.7}\NormalTok{,   }\CommentTok{\# Transparência do preenchimento}
    
    \CommentTok{\# Comportamento ao passar o mouse (Realce)}
    \AttributeTok{highlightOptions =} \FunctionTok{highlightOptions}\NormalTok{(}
      \AttributeTok{color =} \StringTok{"\#666"}\NormalTok{, }
      \AttributeTok{weight =} \DecValTok{3}\NormalTok{,}
      \AttributeTok{bringToFront =} \ConstantTok{TRUE}
\NormalTok{    ),}
    
    \CommentTok{\# Rótulo (Label/Tooltip)}
    \AttributeTok{label =} \SpecialCharTok{\textasciitilde{}}\NormalTok{ labels\_html}
\NormalTok{  ) }\SpecialCharTok{|\textgreater{}} 
  
  \CommentTok{\# Legenda}
  \FunctionTok{addLegend}\NormalTok{(}
    \AttributeTok{position =} \StringTok{"bottomright"}\NormalTok{, }
    \AttributeTok{pal =}\NormalTok{ pal\_area, }
    \AttributeTok{values =} \SpecialCharTok{\textasciitilde{}}\NormalTok{area\_km2, }
    \AttributeTok{title =} \StringTok{"Área (Quantis)"}
\NormalTok{  )}
\end{Highlighting}
\end{Shaded}

\begin{figure}[H]

\centering{

\pandocbounded{\includegraphics[keepaspectratio]{fundEstspatial_files/figure-pdf/fig-leaflet-poly-1.pdf}}

}

\caption{\label{fig-leaflet-poly}Mapa Coroplético Interativo: Área dos
Municípios}

\end{figure}%

\begin{enumerate}
\def\labelenumi{\arabic{enumi}.}
\setcounter{enumi}{3}
\tightlist
\item
  \textbf{Marcadores, Popups HTML e Clusters}
\end{enumerate}

Uma das maiores vantagens do \texttt{leaflet} é o suporte a HTML dentro
dos popups e a capacidade de agrupar pontos automaticamente
(\texttt{Cluster}) para evitar poluição visual quando há milhares de
registros.

\begin{Shaded}
\begin{Highlighting}[]
\ControlFlowTok{if}\NormalTok{ (}\SpecialCharTok{!}\FunctionTok{exists}\NormalTok{(}\StringTok{"rm\_sp"}\NormalTok{)) \{}
  \FunctionTok{library}\NormalTok{(geobr)}
\NormalTok{  sp\_mun }\OtherTok{\textless{}{-}} \FunctionTok{read\_municipality}\NormalTok{(}\AttributeTok{code\_muni =} \StringTok{"SP"}\NormalTok{, }\AttributeTok{year =} \DecValTok{2020}\NormalTok{, }\AttributeTok{showProgress =} \ConstantTok{FALSE}\NormalTok{)}
\NormalTok{  capital }\OtherTok{\textless{}{-}}\NormalTok{ sp\_mun[sp\_mun}\SpecialCharTok{$}\NormalTok{name\_muni }\SpecialCharTok{==} \StringTok{"São Paulo"}\NormalTok{, ]}
\NormalTok{  rm\_sp }\OtherTok{\textless{}{-}}\NormalTok{ capital }
\NormalTok{\}}

\CommentTok{\#}
\FunctionTok{set.seed}\NormalTok{(}\DecValTok{123}\NormalTok{) }
\NormalTok{pontos\_poi }\OtherTok{\textless{}{-}} \FunctionTok{st\_sample}\NormalTok{(rm\_sp, }\AttributeTok{size =} \DecValTok{100}\NormalTok{) }\SpecialCharTok{|\textgreater{}} 
  \FunctionTok{st\_as\_sf}\NormalTok{() }\SpecialCharTok{|\textgreater{}} 
  \FunctionTok{mutate}\NormalTok{(}
    \AttributeTok{id =} \DecValTok{1}\SpecialCharTok{:}\DecValTok{100}\NormalTok{,}
    \AttributeTok{tipo =} \FunctionTok{sample}\NormalTok{(}\FunctionTok{c}\NormalTok{(}\StringTok{"Escola"}\NormalTok{, }\StringTok{"Hospital"}\NormalTok{, }\StringTok{"UBS"}\NormalTok{), }\DecValTok{100}\NormalTok{, }\AttributeTok{replace =} \ConstantTok{TRUE}\NormalTok{),}
    \AttributeTok{atendimentos =} \FunctionTok{round}\NormalTok{(}\FunctionTok{runif}\NormalTok{(}\DecValTok{100}\NormalTok{, }\DecValTok{50}\NormalTok{, }\DecValTok{500}\NormalTok{))}
\NormalTok{  )}

\NormalTok{pal\_tipo }\OtherTok{\textless{}{-}} \FunctionTok{colorFactor}\NormalTok{(}\FunctionTok{c}\NormalTok{(}\StringTok{"\#1f77b4"}\NormalTok{, }\StringTok{"\#d62728"}\NormalTok{, }\StringTok{"\#2ca02c"}\NormalTok{), }\AttributeTok{domain =}\NormalTok{ pontos\_poi}\SpecialCharTok{$}\NormalTok{tipo)}

\FunctionTok{leaflet}\NormalTok{(pontos\_poi) }\SpecialCharTok{|\textgreater{}} 
  \FunctionTok{addProviderTiles}\NormalTok{(providers}\SpecialCharTok{$}\NormalTok{CartoDB.Positron) }\SpecialCharTok{|\textgreater{}} 
  
  \FunctionTok{addCircleMarkers}\NormalTok{(}
    \AttributeTok{radius =} \DecValTok{6}\NormalTok{,}
    \AttributeTok{color =} \SpecialCharTok{\textasciitilde{}}\FunctionTok{pal\_tipo}\NormalTok{(tipo),}
    \AttributeTok{stroke =} \ConstantTok{FALSE}\NormalTok{, }\AttributeTok{fillOpacity =} \FloatTok{0.8}\NormalTok{,}
    \AttributeTok{label =} \SpecialCharTok{\textasciitilde{}}\NormalTok{tipo,}
    
    \CommentTok{\# HTML dentro do Popup}
    \AttributeTok{popup =} \SpecialCharTok{\textasciitilde{}}\FunctionTok{paste0}\NormalTok{(}
      \StringTok{"\textless{}strong\textgreater{}ID:\textless{}/strong\textgreater{} "}\NormalTok{, id, }\StringTok{"\textless{}br\textgreater{}"}\NormalTok{,}
      \StringTok{"\textless{}strong\textgreater{}Tipo:\textless{}/strong\textgreater{} "}\NormalTok{, tipo, }\StringTok{"\textless{}br\textgreater{}"}\NormalTok{,}
      \StringTok{"\textless{}strong\textgreater{}Média Atend.:\textless{}/strong\textgreater{} "}\NormalTok{, atendimentos}
\NormalTok{    ),}
    
    \AttributeTok{clusterOptions =} \FunctionTok{markerClusterOptions}\NormalTok{() }
\NormalTok{  )}
\end{Highlighting}
\end{Shaded}

\pandocbounded{\includegraphics[keepaspectratio]{fundEstspatial_files/figure-pdf/unnamed-chunk-44-1.pdf}}

\begin{enumerate}
\def\labelenumi{\arabic{enumi}.}
\setcounter{enumi}{4}
\item
  \textbf{Controle de Camadas (\texttt{addLayersControl})}
\item
  \texttt{BaseGroups:} Camadas de fundo mutuamente exclusivas (ex: Dia
  ou Noite).
\item
  \texttt{OverlayGroups:} Camadas de dados que podem ser
  ligadas/desligadas independentemente.
\end{enumerate}

\begin{Shaded}
\begin{Highlighting}[]
\FunctionTok{leaflet}\NormalTok{() }\SpecialCharTok{|\textgreater{}} 
  \FunctionTok{addTiles}\NormalTok{(}\AttributeTok{group =} \StringTok{"OSM (Ruas)"}\NormalTok{) }\SpecialCharTok{|\textgreater{}} 
  \FunctionTok{addProviderTiles}\NormalTok{(providers}\SpecialCharTok{$}\NormalTok{Esri.WorldImagery, }\AttributeTok{group =} \StringTok{"Satélite"}\NormalTok{) }\SpecialCharTok{|\textgreater{}} 
  
  \FunctionTok{addPolygons}\NormalTok{(}\AttributeTok{data =}\NormalTok{ rm\_sp, }\AttributeTok{color =} \StringTok{"white"}\NormalTok{, }\AttributeTok{weight =} \DecValTok{2}\NormalTok{, }\AttributeTok{fillColor =} \StringTok{"transparent"}\NormalTok{,}
              \AttributeTok{group =} \StringTok{"Limites Municipais"}\NormalTok{) }\SpecialCharTok{|\textgreater{}} 
  
  \FunctionTok{addCircleMarkers}\NormalTok{(}\AttributeTok{data =}\NormalTok{ pontos\_poi, }\AttributeTok{radius =} \DecValTok{5}\NormalTok{, }\AttributeTok{color =} \StringTok{"orange"}\NormalTok{, }
                   \AttributeTok{fillOpacity =} \FloatTok{0.8}\NormalTok{, }\AttributeTok{stroke =} \ConstantTok{FALSE}\NormalTok{,}
                   \AttributeTok{group =} \StringTok{"Equipamentos Públicos"}\NormalTok{) }\SpecialCharTok{|\textgreater{}} 
  
  \FunctionTok{addLayersControl}\NormalTok{(}
    \AttributeTok{baseGroups =} \FunctionTok{c}\NormalTok{(}\StringTok{"OSM (Ruas)"}\NormalTok{, }\StringTok{"Satélite"}\NormalTok{),}
    \AttributeTok{overlayGroups =} \FunctionTok{c}\NormalTok{(}\StringTok{"Limites Municipais"}\NormalTok{, }\StringTok{"Equipamentos Públicos"}\NormalTok{),}
    \AttributeTok{options =} \FunctionTok{layersControlOptions}\NormalTok{(}\AttributeTok{collapsed =} \ConstantTok{FALSE}\NormalTok{) }\CommentTok{\# Menu sempre aberto}
\NormalTok{  )}
\end{Highlighting}
\end{Shaded}

\begin{figure}[H]

\centering{

\pandocbounded{\includegraphics[keepaspectratio]{fundEstspatial_files/figure-pdf/fig-leaflet-layers-1.pdf}}

}

\caption{\label{fig-leaflet-layers}Controle de Camadas Completo}

\end{figure}%

\begin{Shaded}
\begin{Highlighting}[]
\CommentTok{\# Fonte: Este exemplo foi retirado no link: https://r{-}spatial.github.io/mapview/articles/mapview\_06{-}add.html}
\FunctionTok{p\_load}\NormalTok{(leafem, mapview)}
\NormalTok{m }\OtherTok{\textless{}{-}} \FunctionTok{mapview}\NormalTok{(breweries)}

\NormalTok{leafem}\SpecialCharTok{::}\FunctionTok{addLogo}\NormalTok{(m, }\StringTok{"https://jeroenooms.github.io/images/banana.gif"}\NormalTok{, }\CommentTok{\#pode adicionar imagem aqui}
                \AttributeTok{position =} \StringTok{"bottomleft"}\NormalTok{,}
                \AttributeTok{offset.x =} \DecValTok{5}\NormalTok{,}
                \AttributeTok{offset.y =} \DecValTok{40}\NormalTok{,}
                \AttributeTok{width =} \DecValTok{100}\NormalTok{,}
                \AttributeTok{height =} \DecValTok{100}\NormalTok{)}
\end{Highlighting}
\end{Shaded}

\pandocbounded{\includegraphics[keepaspectratio]{fundEstspatial_files/figure-pdf/unnamed-chunk-45-1.pdf}}

\begin{tcolorbox}[enhanced jigsaw, left=2mm, toptitle=1mm, colback=white, colframe=quarto-callout-tip-color-frame, colbacktitle=quarto-callout-tip-color!10!white, opacityback=0, rightrule=.15mm, bottomtitle=1mm, arc=.35mm, title=\textcolor{quarto-callout-tip-color}{\faLightbulb}\hspace{0.5em}{Exportação e Salvamento}, titlerule=0mm, bottomrule=.15mm, leftrule=.75mm, coltitle=black, toprule=.15mm, breakable, opacitybacktitle=0.6]

O \texttt{leaflet} gera HTML. Para salvar seu mapa:

Use o pacote \texttt{mapview}:
\texttt{mapview::mapshot(mapa,\ file\ =\ "mapa.png")}.

\end{tcolorbox}

\begin{tcolorbox}[enhanced jigsaw, left=2mm, toptitle=1mm, colback=white, colframe=quarto-callout-tip-color-frame, colbacktitle=quarto-callout-tip-color!10!white, opacityback=0, rightrule=.15mm, bottomtitle=1mm, arc=.35mm, title=\textcolor{quarto-callout-tip-color}{\faLightbulb}\hspace{0.5em}{Para ir além}, titlerule=0mm, bottomrule=.15mm, leftrule=.75mm, coltitle=black, toprule=.15mm, breakable, opacitybacktitle=0.6]

O ecossistema de análise espacial no R é vasto e dinâmico, com novos
pacotes surgindo constantemente. A seleção apresentada neste capítulo
focou nas ferramentas essenciais para resolver os problemas mais
frequentes do dia a dia.

Vale mencionar também o excelente pacote
\href{https://r-tmap.github.io/tmap/articles/examples_topo_Africa}{tmap},
que oferece uma sintaxe flexível para mapas temáticos (similar ao
ggplot2). Para se manter atualizado e buscar inspiração, recomendo duas
fontes indispensáveis:

\href{https://r-spatial.org/}{R-Spatial.org:} O blog oficial com as
últimas novidades sobre a evolução espacial do R.

\href{https://r-graph-gallery.com/}{The R Graph Gallery:} Uma coleção
com os melhores códigos e exemplos visuais, cobrindo desde gráficos
básicos até mapas complexos.

\end{tcolorbox}

\part{Geostatística}

\chapter{Geoestatística}\label{sec-geoest}

Imagine o desafio enfrentado por um engenheiro de minas na África do
Sul, na década de 1950. A sua tarefa consistia em estimar o teor de ouro
num bloco de rocha ainda não escavado, baseando-se apenas em algumas
amostras recolhidas através de furos de sondagem Ecker (2003). Ao
aplicar os métodos estatísticos convencionais da época, baseados em
médias aritméticas, os engenheiros deparavam-se consistentemente com um
erro sistemático: as previsões tendiam a superestimar as áreas ricas e a
subestimar as áreas pobres. Quando a exploração real avançava, o ouro
recuperado não correspondia à estimativa inicial, gerando prejuízos
avultados Krige e Kleingeld (2005).

A raiz deste problema não estava na precisão dos instrumentos, mas na
premissa estatística utilizada. A estatística clássica assume,
frequentemente, que as observações são independentes e identicamente
distribuídas (i.i.d.) Noel Cressie e Moores (2022). No entanto, na
natureza, esta independência raramente existe. O engenheiro sul-africano
\href{https://www.nae.edu/190431/DANIE-G-KRIGE-19192013}{Danie Gerhardus
Krige} percebeu empiricamente que uma amostra geológica não é um valor
isolado; ela carrega consigo uma influência da sua vizinhança Scalon
(2024). Uma amostra com elevado teor de minério sugere que a rocha
imediatamente adjacente tem também uma alta probabilidade de ser rica
Oliver e Webster (2014) . Ao ignorar a localização espacial das amostras
e tratá-las como eventos aleatórios independentes, perdia-se a
informação mais valiosa para a predição: a continuidade espacial
Nhancololo et al. (2024).

Foi com base nesta intuição que o matemático francês
\href{https://en.wikipedia.org/wiki/Georges_Matheron}{Georges Matheron},
na década de 1960, formalizou a disciplina que hoje conhecemos como
Geoestatística (Andre G. Journel e Huijbregts 1976; Scalon 2024).
Matheron sistematizou as observações de Krige através da \textbf{Teoria
das Variáveis Regionalizadas} (G. Matheron 1963; George Matheron 1971),
estabelecendo que fenômenos naturais não são nem puramente aleatórios,
como o lançamento de um dado, nem puramente determinísticos,
descritíveis por uma equação geométrica simples. Eles exibem uma
estrutura mista: possuem continuidade estruturada (dependência
espacial), mas também uma componente localmente imprevisível
(aleatoriedade local) (Noel Cressie 1990; Wackernagel 2003).

A Geoestatística define-se, portanto, como o ramo da estatística
espacial dedicado à modelagem e predição destes fenômenos contínuos
Myers (1994). Assume-se que a variável de interesse, denotada por
\(Y(\mathbf{s})\), existe em qualquer ponto de um domínio contínuo fixo
\(D^{G} \subseteq \mathbb{R}^d\), mas é observada apenas num conjunto
finito de locais \(\{\mathbf{s}_1, \dots, \mathbf{s}_n\}\) (Scalon 2024;
Noel Cressie e Moores 2022).

Neste contexto de informação incompleta, o objetivo central passa a ser
a utilização da estrutura de dependência espacial identificada nas
amostras para inferir, com o menor erro possível, o comportamento da
variável nos locais onde não foi realizada qualquer medição. Foi
precisamente esta mudança de paradigma e robustez preditiva que permitiu
resolver o problema original das minas de ouro, impulsionando a
subsequente expansão da disciplina para a hidrologia, ciências do solo,
meteorologia e epidemiologia, onde se consolidou como a ferramenta
padrão para lidar com variáveis contínuas no espaço (Yamamoto e Landim
2013; Nhancololo et al. 2024).

\begin{Shaded}
\begin{Highlighting}[]
\ControlFlowTok{if}\NormalTok{ (}\SpecialCharTok{!}\FunctionTok{require}\NormalTok{(}\StringTok{"pacman"}\NormalTok{)) }\FunctionTok{install.packages}\NormalTok{(}\StringTok{"pacman"}\NormalTok{)}
\NormalTok{pacman}\SpecialCharTok{::}\FunctionTok{p\_load}\NormalTok{(ggplot2, gstat, sf, viridis, gridExtra, knitr)}

\NormalTok{grid\_df }\OtherTok{\textless{}{-}} \FunctionTok{expand.grid}\NormalTok{(}\AttributeTok{x =} \DecValTok{1}\SpecialCharTok{:}\DecValTok{50}\NormalTok{, }\AttributeTok{y =} \DecValTok{1}\SpecialCharTok{:}\DecValTok{50}\NormalTok{)}
\NormalTok{grid\_sf }\OtherTok{\textless{}{-}} \FunctionTok{st\_as\_sf}\NormalTok{(grid\_df, }\AttributeTok{coords =} \FunctionTok{c}\NormalTok{(}\StringTok{"x"}\NormalTok{, }\StringTok{"y"}\NormalTok{))}

\FunctionTok{set.seed}\NormalTok{(}\DecValTok{42}\NormalTok{)}
\NormalTok{grid\_df}\SpecialCharTok{$}\NormalTok{iid }\OtherTok{\textless{}{-}} \FunctionTok{rnorm}\NormalTok{(}\FunctionTok{nrow}\NormalTok{(grid\_df))}


\NormalTok{modelo\_vgm }\OtherTok{\textless{}{-}}\NormalTok{ gstat}\SpecialCharTok{::}\FunctionTok{vgm}\NormalTok{(}\AttributeTok{psill =} \DecValTok{1}\NormalTok{, }\AttributeTok{model =} \StringTok{"Sph"}\NormalTok{, }\AttributeTok{range =} \DecValTok{20}\NormalTok{, }\AttributeTok{nugget =} \DecValTok{0}\NormalTok{)}
\NormalTok{g\_dummy }\OtherTok{\textless{}{-}}\NormalTok{ gstat}\SpecialCharTok{::}\FunctionTok{gstat}\NormalTok{(}\AttributeTok{formula =}\NormalTok{ z}\SpecialCharTok{\textasciitilde{}}\DecValTok{1}\NormalTok{, }\AttributeTok{locations =}\NormalTok{ grid\_sf, }\AttributeTok{dummy =}\NormalTok{ T, }\AttributeTok{beta =} \DecValTok{0}\NormalTok{, }\AttributeTok{model =}\NormalTok{ modelo\_vgm, }\AttributeTok{nmax =} \DecValTok{20}\NormalTok{)}
\FunctionTok{invisible}\NormalTok{(}\FunctionTok{capture.output}\NormalTok{(yy }\OtherTok{\textless{}{-}} \FunctionTok{predict}\NormalTok{(g\_dummy, }\AttributeTok{newdata =}\NormalTok{ grid\_sf, }\AttributeTok{nsim =} \DecValTok{1}\NormalTok{)))}
\NormalTok{grid\_df}\SpecialCharTok{$}\NormalTok{geo }\OtherTok{\textless{}{-}}\NormalTok{ yy}\SpecialCharTok{$}\NormalTok{sim1}

\NormalTok{p1 }\OtherTok{\textless{}{-}} \FunctionTok{ggplot}\NormalTok{(grid\_df, }\FunctionTok{aes}\NormalTok{(x, y, }\AttributeTok{fill =}\NormalTok{ iid)) }\SpecialCharTok{+}
  \FunctionTok{geom\_tile}\NormalTok{() }\SpecialCharTok{+}
  \FunctionTok{scale\_fill\_viridis\_c}\NormalTok{(}\AttributeTok{option =} \StringTok{"magma"}\NormalTok{, }\AttributeTok{name =} \StringTok{"Teor"}\NormalTok{) }\SpecialCharTok{+}
  \FunctionTok{coord\_fixed}\NormalTok{() }\SpecialCharTok{+} \FunctionTok{theme\_void}\NormalTok{() }\SpecialCharTok{+}
  \FunctionTok{labs}\NormalTok{(}\AttributeTok{title =} \StringTok{"Estatística Clássica"}\NormalTok{, }\AttributeTok{subtitle =} \StringTok{"Amostras Independentes (Ruído)"}\NormalTok{) }\SpecialCharTok{+}
  \FunctionTok{theme}\NormalTok{(}\AttributeTok{plot.title =} \FunctionTok{element\_text}\NormalTok{(}\AttributeTok{hjust =} \FloatTok{0.5}\NormalTok{), }\AttributeTok{plot.subtitle =} \FunctionTok{element\_text}\NormalTok{(}\AttributeTok{hjust =} \FloatTok{0.5}\NormalTok{))}

\NormalTok{p2 }\OtherTok{\textless{}{-}} \FunctionTok{ggplot}\NormalTok{(grid\_df, }\FunctionTok{aes}\NormalTok{(x, y, }\AttributeTok{fill =}\NormalTok{ geo)) }\SpecialCharTok{+}
  \FunctionTok{geom\_tile}\NormalTok{() }\SpecialCharTok{+}
  \FunctionTok{scale\_fill\_viridis\_c}\NormalTok{(}\AttributeTok{option =} \StringTok{"magma"}\NormalTok{, }\AttributeTok{name =} \StringTok{"Teor"}\NormalTok{) }\SpecialCharTok{+}
  \FunctionTok{coord\_fixed}\NormalTok{() }\SpecialCharTok{+} \FunctionTok{theme\_void}\NormalTok{() }\SpecialCharTok{+}
  \FunctionTok{labs}\NormalTok{(}\AttributeTok{title =} \StringTok{"Geoestatística"}\NormalTok{, }\AttributeTok{subtitle =} \StringTok{"Dependência Espacial (Continuidade)"}\NormalTok{) }\SpecialCharTok{+}
  \FunctionTok{theme}\NormalTok{(}\AttributeTok{plot.title =} \FunctionTok{element\_text}\NormalTok{(}\AttributeTok{hjust =} \FloatTok{0.5}\NormalTok{), }\AttributeTok{plot.subtitle =} \FunctionTok{element\_text}\NormalTok{(}\AttributeTok{hjust =} \FloatTok{0.5}\NormalTok{))}

\FunctionTok{grid.arrange}\NormalTok{(p1, p2, }\AttributeTok{ncol =} \DecValTok{2}\NormalTok{)}
\end{Highlighting}
\end{Shaded}

\begin{figure}[H]

\centering{

\pandocbounded{\includegraphics[keepaspectratio]{geostat_files/figure-pdf/fig-krige-intro-1.pdf}}

}

\caption{\label{fig-krige-intro}Intuição de Krige: A diferença entre
Ruído Branco (Independente) e Continuidade Espacial (Estruturado).}

\end{figure}%

\section{Variável Regionalizada e o Processo
Estocástico}\label{variuxe1vel-regionalizada-e-o-processo-estocuxe1stico}

Para operar matematicamente sobre um fenômeno natural único, utilizamos
o conceito de Variável Regionalizada. É crucial estabelecer uma
distinção notacional usada aqui: denotamos por \(y(\mathbf{s})\) o valor
numérico observado do fenômeno no local \(\mathbf{s}\) (a realização
concreta, com letra minúscula), e por \(Y(\mathbf{s})\) a variável
aleatória no local \(\mathbf{s}\) (o processo probabilístico antes da
observação, com letra maiúscula). Embora na realidade \(y(\mathbf{s})\)
seja único e fixo, a geoestatística modela-o como uma realização de um
Processo Estocástico (ou Função Aleatória/Campo Aleatório)
\(Y(\mathbf{s})\).

\textbf{Variável Regionalizada}

Uma variável regionalizada é uma função numérica \(f(\mathbf{s})\) que
descreve a distribuição espacial de uma grandeza física (ex: teor de
ouro, pH do solo) num domínio \(D^{G}\). Esta função possui propriedades
duais: um aspeto estruturado (refletindo tendências geológicas ou
climáticas de larga escala) e um aspeto aleatório (refletindo
irregularidades locais imprevisíveis) (George Matheron 1971).
Matematicamente, tratamos estas variáveis regionalizadas como
realizações de um processo estocástico (ou campo aleatório)
\(\{Y(s) : s \in D^{G}\}\), onde \(s\) denota um vetor de coordenadas em
um domínio espacial \(D^G \subseteq \mathbb{R}^d\) (geralmente \(d=2\)
ou \(3\)) (Noel Cressie 1991)

\textbf{Processo Estocástico}

Um processo estocástico espacial é definido como uma coleção de
variáveis aleatórias \(\{Y(\mathbf{s}) : \mathbf{s} \in D^{G}\}\), onde
\(D^{G} \subseteq D \subseteq \mathbb{R}^d\) é um conjunto de índices
contínuo (uma área ou volume) com
\href{https://pt.wikipedia.org/wiki/Medida_de_Lebesgue}{medida de
Lebesgue} positiva (área \(>0\)) (Noel Cressie 1993). O objetivo da
inferência geoestatística é reconstruir a lei de probabilidade do
processo \(Y(\mathbf{s})\) a partir de um conjunto finito de observações
\(\{y(\mathbf{s}_1), \dots, y(\mathbf{s}_n)\}\).

Ao contrário da análise de séries temporais, onde o índice \(t\) possui
uma ordenação natural (passado \(\to\) futuro), o índice espacial \(s\)
não possui uma ordenação única em \(\mathbb{R}^d\) para \(d \ge 2\), o
que introduz complexidades adicionais na modelação da dependência
multidirecional (Noel Cressie e Moores 2022).

Para tornar o processo estocástico tratável, decompomos a variável
aleatória \(Y(\mathbf{s})\) em componentes que explicam diferentes
escalas de variação. Existem duas formulações principais: a decomposição
simples e a decomposição estrutural completa.

Na decomposição simples assumimos que o processo estocástico é
constituído por uma média determinística e um erro correlacionado:

\begin{equation}\phantomsection\label{eq-decomp-simples}{Y(\mathbf{s}) = \mu(\mathbf{s}) + \delta(\mathbf{s}),}\end{equation}

Onde \(\mu(\mathbf{s}) \equiv E[Y(\mathbf{s})]\) representa a tendência
de larga escala (\emph{Trend} ou \emph{Drift}), isto é, variação
sistemática do fenômeno sobre o domínio \(DĜ\). Assume-se frequentemente
que \(\mu(\mathbf{s})\) é uma função suave das coordenadas ou uma
combinação linear de covariáveis externas \(X(\mathbf{s})\), tal que
\(\mu(\mathbf{s}) = \mathbf{x}(\mathbf{s})^\top \boldsymbol{\beta}\)
(Noel Cressie e Moores 2022). \(\delta(\mathbf{s})\): Representa a
variação de pequena escala ou o erro estocástico. Assume-se que este
termo tem média zero, \(E[\delta(\mathbf{s})] = 0\), e captura a
dependência espacial estatística (correlação) entre locais vizinhos.

A formulação usada na Eq.~\ref{eq-decomp-simples} é insuficiente pois
não distingue entre a variabilidade natural do fenômeno e o erro humano.
Noel Cressie (1991) e Peter J. Diggle, Tawn, e Moyeed (1998), sugerem
expandir o termo de erro (\(\delta(\mathbf{s})\):

\begin{equation}\phantomsection\label{eq-decomp-completa}{Y(\mathbf{s}) = \mu(\mathbf{s}) + W(\mathbf{s}) + \eta(\mathbf{s}) + \epsilon(\mathbf{s})}\end{equation}

Onde:

\begin{itemize}
\item
  \(\mu(\mathbf{s})\) continua sendo componente determinística. Podendo
  ser modelada como uma constante, \(\mu(\mathbf{s})=\mu\) (Krigagem
  simples e ordinária), um polinómio das coordenadas,
  \(Y(s) =\beta_0 + \beta_1 s_x + \beta_2 s_y\), (Krigagem Universal) ou
  função de covariáveis externas, como elevação (Krigagem com Deriva
  Externa) (Wackernagel 2003). Captura forçantes globais (ex: o
  gradiente de temperatura causado pela latitude).
\item
  \(W(\mathbf{s})\)) é o componente estocástico de interesse principal.
  É um processo estocástico com média zero e continuidade em média
  quadrática (\(L_2\)-contínuo).
  \(E[(W(\mathbf{s+h}) - W(\mathbf{s}))^2] \to 0\) quando
  \(\|\mathbf{h}\| \to 0\). Este componente captura a estrutura de
  dependência espacial observável na escala da amostragem. Em abordagens
  modernas de baixo posto este termo é frequentemente modelado por uma
  expansão de funções de base,
  \(W(\mathbf{s}) \approx \sum \alpha_j \phi_j(\mathbf{s})\) (sugestão
  de leitura Noel Cressie, Sainsbury-Dale, e Zammit-Mangion (2022) ).
\item
  \(\eta(\mathbf{s})\)) representa a variabilidade do fenômeno que
  ocorre a distâncias menores do que a menor distância de separação
  entre as observações disponíveis
  (\(\min \|\mathbf{s}_i - \mathbf{s}_j\|\)). É uma variação intrínseca
  e física do fenômeno, não um erro. No entanto, dado que não temos
  dados suficientes para resolver esta continuidade, modelamo-la
  estatisticamente como um ruído branco espacialmente não correlacionado
  na escala de observação. \textbf{Exemplo:} Se medirmos o teor de ouro
  a cada 10 metros, a variação extrema que ocorre dentro de uma
  \href{https://pt.wikipedia.org/wiki/Pepita}{pepita} de 1 cm é
  classificada como variação de microescala (\(\eta\)).
\end{itemize}

Conforme descrito por Peter J. Diggle, Tawn, e Moyeed (1998) referem-se
a componentes similares num contexto de modelos lineares generalizados
mistos, onde a heterogeneidade latente de pequena escala deve ser
contabilizada.

\begin{itemize}
\tightlist
\item
  \(\epsilon(\mathbf{s})\) é ruído branco puro
  (\(\epsilon \overset{iid}{\sim} N[0, \text{Var}(\epsilon(\mathbf{s}))]\)),
  introduzido pelo processo de observação (precisão do instrumento, erro
  de laboratório, erro de localização). É puramente aleatório e não tem
  realidade no fenômeno natural \(S(\mathbf{s})\) que estamos a tentar
  estudar.
\end{itemize}

A soma das variâncias dos dois últimos componentes (\(\eta (s)\) e
\(\epsilon (s)\)) constitui o que chamamos de Efeito Pepita (\(c_0\)),
visível no variograma experimental (assunto da próxima seção) como uma
descontinuidade na origem (\(\gamma(h) \to c_0\) quando \(h \to 0\)):

\begin{Shaded}
\begin{Highlighting}[]
\NormalTok{pacman}\SpecialCharTok{::}\FunctionTok{p\_load}\NormalTok{(tidyr)}

\NormalTok{grid\_df}\SpecialCharTok{$}\NormalTok{tendencia }\OtherTok{\textless{}{-}}\NormalTok{ (grid\_df}\SpecialCharTok{$}\NormalTok{x }\SpecialCharTok{+}\NormalTok{ grid\_df}\SpecialCharTok{$}\NormalTok{y) }\SpecialCharTok{/} \DecValTok{20} 

\NormalTok{grid\_df}\SpecialCharTok{$}\NormalTok{residuo }\OtherTok{\textless{}{-}}\NormalTok{ grid\_df}\SpecialCharTok{$}\NormalTok{geo}

\NormalTok{grid\_df}\SpecialCharTok{$}\NormalTok{Y }\OtherTok{\textless{}{-}}\NormalTok{ grid\_df}\SpecialCharTok{$}\NormalTok{tendencia }\SpecialCharTok{+}\NormalTok{ grid\_df}\SpecialCharTok{$}\NormalTok{residuo}

\NormalTok{df\_long }\OtherTok{\textless{}{-}} \FunctionTok{pivot\_longer}\NormalTok{(grid\_df, }\AttributeTok{cols =} \FunctionTok{c}\NormalTok{(Y, tendencia, residuo), }\AttributeTok{names\_to =} \StringTok{"Componente"}\NormalTok{, }\AttributeTok{values\_to =} \StringTok{"Valor"}\NormalTok{)}

\NormalTok{df\_long}\SpecialCharTok{$}\NormalTok{Componente }\OtherTok{\textless{}{-}} \FunctionTok{factor}\NormalTok{(df\_long}\SpecialCharTok{$}\NormalTok{Componente, }
                             \AttributeTok{levels =} \FunctionTok{c}\NormalTok{(}\StringTok{"Y"}\NormalTok{, }\StringTok{"tendencia"}\NormalTok{, }\StringTok{"residuo"}\NormalTok{),}
                             \AttributeTok{labels =} \FunctionTok{c}\NormalTok{(}\StringTok{"\textquotesingle{}Fenômeno Observado\textquotesingle{} \textasciitilde{} Y(s)"}\NormalTok{, }
                                        \StringTok{"\textquotesingle{}Tendência\textquotesingle{} \textasciitilde{} mu(s)"}\NormalTok{, }
                                        \StringTok{"\textquotesingle{}Resíduo estocástico\textquotesingle{} \textasciitilde{} delta(s)"}\NormalTok{))}

\FunctionTok{ggplot}\NormalTok{(df\_long, }\FunctionTok{aes}\NormalTok{(x, y, }\AttributeTok{fill =}\NormalTok{ Valor)) }\SpecialCharTok{+}
  \FunctionTok{geom\_tile}\NormalTok{() }\SpecialCharTok{+}
  \FunctionTok{facet\_wrap}\NormalTok{(}\SpecialCharTok{\textasciitilde{}}\NormalTok{Componente, }\AttributeTok{labeller =}\NormalTok{ label\_parsed) }\SpecialCharTok{+}
  \FunctionTok{scale\_fill\_viridis\_c}\NormalTok{(}\AttributeTok{option =} \StringTok{"C"}\NormalTok{) }\SpecialCharTok{+}
  \FunctionTok{coord\_fixed}\NormalTok{() }\SpecialCharTok{+} 
  \FunctionTok{theme\_void}\NormalTok{() }\SpecialCharTok{+}
  \FunctionTok{theme}\NormalTok{(}\AttributeTok{strip.text =} \FunctionTok{element\_text}\NormalTok{(}\AttributeTok{size =} \DecValTok{12}\NormalTok{, }\AttributeTok{face =} \StringTok{"bold"}\NormalTok{))}
\end{Highlighting}
\end{Shaded}

\begin{figure}[H]

\centering{

\pandocbounded{\includegraphics[keepaspectratio]{geostat_files/figure-pdf/fig-decomposicao-1.pdf}}

}

\caption{\label{fig-decomposicao}Decomposição de uma variável
regionalizada: Tendência global + resíduo estocástico}

\end{figure}%

\section{Estacionariedade e Inferência
Estatística}\label{estacionariedade-e-inferuxeancia-estatuxedstica}

O obstáculo central na inferência espacial é a unicidade da realização:
na prática, possuímos apenas um único conjunto de dados observados
\(\{y(\mathbf{s}) : \mathbf{s} \in D\}\), que constitui apenas uma das
infinitas realizações possíveis do processo estocástico gerador
\(\{Y(\mathbf{s})\}\). Diferentemente de experiências laboratoriais
controladas, não é possível replicar o processo geológico ou climático
sob as mesmas condições para gerar múltiplas realizações e, assim,
estimar a função de distribuição conjunta para qualquer conjunto finito
de \(k\) localizações:
\[F(y_1, \dots, y_k; \mathbf{s}_1, \dots, \mathbf{s}_k) = P\left(Y(\mathbf{s}_1) \le y_1, \dots, Y(\mathbf{s}_k) \le y_k\right),\]

Nem tão-pouco é possível calcular empiricamente os seus momentos de
primeira \(\mu(\mathbf{s})\),
\[\mu(\mathbf{s}) = E[Y(\mathbf{s})] = \int_{-\infty}^{+\infty} y \cdot f(y; \mathbf{s}) \, dy\:, \]
e segunda ordem \(\sigma^2(\mathbf{s})\),
\[\sigma^2(\mathbf{s}) = \text{Var}(Y(\mathbf{s})) = E\left[(Y(\mathbf{s}) - \mu(\mathbf{s}))^2\right], \]
em cada local \(\mathbf{s}\) (Noel Cressie 1993).

Para viabilizar a inferência estatística (i.e., estimar os parâmetros do
processo a partir de uma única realização), é imperativo invocar a
hipótese de estacionariedade. Este conceito assume a invariância das
propriedades estatísticas (momentos da distribuição) do processo sob
translação no domínio espacial \(D^G\) Seção~\ref{sec-estacionaridade}.
Sob esta premissa, assumimos que a estrutura de dependência é homogénea
em todo o domínio, o que nos permite utilizar repetições espaciais,
diferentes pares de pontos separados pelo mesmo vetor \(\mathbf{h}\) em
locais distintos, como substitutos válidos para as inexistentes
repetições estocásticas. Este procedimento fundamenta-se na hipótese de
\href{https://en.wikipedia.org/wiki/Ergodicity}{ergodicidade}, que
estabelece a condição necessária para estimar parâmetros probabilísticos
a partir de uma única realização observada. Sob condições específicas de
mistura (onde a correlação espacial decai suficientemente rápido com a
distância), a ergodicidade garante que as médias espaciais calculadas
sobre o domínio \(D^G\) convirjam para a esperança matemática (média
populacional) à medida que o volume ou área do domínio de observação
cresce indefinidamente (\(|D^G| \to \infty\)) (Noel Cressie 1993). Esta
convergência, é em média quadrática (ou convergência em \(L^2\)). Seja
\(\bar{Y}_D\) a média espacial (se existir) do processo no domínio
\(D\), definida como
\(\bar{Y}_{D^G} = \frac{1}{|D|} \int_{D^G} Y(\mathbf{s}) d\mathbf{s}\).
A propriedade ergódica assegura que o erro quadrático médio entre a
média espacial e a média teórica \(\mu\) tende para zero:

\[\lim_{|D^G| \to \infty} E\left[ \left( \bar{Y}_{D^G} - \mu \right)^2 \right] = 0 , \Longleftrightarrow \bar{Y}_{D^G} \xrightarrow{L^2} \mu\]
Esta convergência implica que a variância da média espacial diminui
assintoticamente, permitindo que as estatísticas calculadas sobre uma
única realização extensa sejam estimadores consistentes dos momentos do
processo estocástico gerador (George Matheron 1971; Noel Cressie 1989).

Conforme descrito na Seção~\ref{sec-estacionaridade} existem dois níveis
principais de estacionariedade utilizados na modelação geoestatística:

A estacionaridade mais comum em análise de séries temporais e
estatística espacial é a estacionariedade de segunda ordem (ou fraca).
Um processo estocástico \(\{Y(\mathbf{s})\}\) diz-se estacionário de
segunda ordem se satisfizer duas condições simultâneas:

\begin{itemize}
\tightlist
\item
  A esperança matemática deve ser constante e finita em todo o domínio
  \(D^G\), independentemente da localização \(\mathbf{s}\). O parâmetro
  \(\mu\) representa o nível global em torno do qual o processo flutua:
\end{itemize}

\[E[Y(\mathbf{s})] = \mu, \forall \mathbf{s} \in D^G\].

\begin{itemize}
\tightlist
\item
  A covariância entre dois pontos não depende das suas localizações
  absolutas \(\mathbf{s}\) e \(\mathbf{s}+\mathbf{h}\), mas apenas do
  vetor de separação \(\mathbf{h}\) (que define a distância e a direção
  entre eles):
\end{itemize}

\begin{equation}\phantomsection\label{eq-covariancia}{Cov(Y(\mathbf{s}), Y(\mathbf{s}+\mathbf{h})) = E[(Y(\mathbf{s}) - \mu)(Y(\mathbf{s}+\mathbf{h}) - \mu)] = C(\mathbf{h})}\end{equation}

onde \(C(\mathbf{h})\) denota a função de covariância. Esta definição
implica necessariamente que a variância do processo é finita e
constante, dada por \(Var(Y(\mathbf{s})) = C(\mathbf{0}) < \infty\).

Uma consequência analítica imediata da definição acima
(Eq.~\ref{eq-covariancia}) é a estabilidade da variância. Se
considerarmos o caso particular onde o vetor de separação é nulo
(\(\mathbf{h} = \mathbf{0}\)), a covariância de um ponto com ele próprio
torna-se, por definição, a sua variância. Como \(C(\mathbf{h})\) não
depende da localização \(\mathbf{s}\), segue-se que \(C(\mathbf{0})\)
também não depende.

Portanto, a variância do processo é finita e constante em todo o domínio
(propriedade de homocedasticidade espacial), sendo definida por:

\[Cov(Y(\mathbf{s}), Y(\mathbf{s+0}))=Var(Y(\mathbf{s})) = C(\mathbf{0}) = \sigma^2\]

Esta relação (\(C(\mathbf{0}) = \sigma^2\)) estabelece que o ``patamar''
máximo de variabilidade do processo é fixo e conhecido a priori. Se a
variabilidade do fenômeno crescer indefinidamente com a área (como na
topografia de uma cadeia montanhosa), a condição
\(C(\mathbf{0}) < \infty\) é violada, e a estacionariedade de segunda
ordem não pode ser assumida, exigindo a adoção da hipótese intrínseca.

Muitos fenômenos naturais, como a dispersão de poluentes ou a
topografia, apresentam uma variabilidade que cresce sem limites à medida
que a área de estudo aumenta, violando a condição de variância a priori
finita. Para acomodar estes processos, G. Matheron (1963) introduziu uma
condição mais fraca e generalista: a estacionariedade intrínseca. Esta
hipótese exige apenas a estacionariedade dos incrementos do processo
\(Y(\mathbf{s}+\mathbf{h}) - Y(\mathbf{s})\), definindo-se pelas
seguintes propriedades:

\[\begin{aligned}
E[Y(\mathbf{s}+\mathbf{h}) - Y(\mathbf{s})] &= 0 \\
Var(Y(\mathbf{s}+\mathbf{h}) - Y(\mathbf{s})) &= 2\gamma(\mathbf{h})
\end{aligned}\]

A formulação intrínseca é vantajosa pois abrange uma classe mais ampla
de processos estocásticos, incluindo aqueles com variância infinita
(como o movimento Browniano), onde a função de covariância
\(C(\mathbf{h})\) não estaria definida, mas o variograma está
perfeitamente caracterizado.

A função \(2\gamma(\mathbf{h})\) é definida formalmente como o
variograma, e \(\gamma(\mathbf{h})\) como o semivariograma.

Partindo da definição do variograma \(\gamma(\mathbf{h})\) como a
variância do incremento, o desenvolvimento segue:

\begin{equation}\phantomsection\label{eq-semivariograma}{\begin{aligned}
2\gamma(\mathbf{h}) &= Var(Y(\mathbf{s}+\mathbf{h}) - Y(\mathbf{s})) \\
&= E\left[ \left( \{Y(\mathbf{s}+\mathbf{h}) - Y(\mathbf{s})\} - E[Y(\mathbf{s}+\mathbf{h}) - Y(\mathbf{s})] \right)^2 \right] && \text{(Definição de Variância)} \\
&= E\left[ \left( Y(\mathbf{s}+\mathbf{h}) - Y(\mathbf{s}) \right)^2 \right] && \text{(Pois } E[Y(\mathbf{s}+\mathbf{h}) - Y(\mathbf{s})] = 0 \text{)} \\
&= E\left[ \left( \{Y(\mathbf{s}+\mathbf{h}) - \mu\} - \{Y(\mathbf{s}) - \mu\} \right)^2 \right] && \text{(Centrando na média } \mu \text{)} \\
&= E\left[ (Y(\mathbf{s}+\mathbf{h}) - \mu)^2 - 2(Y(\mathbf{s}+\mathbf{h}) - \mu)(Y(\mathbf{s}) - \mu) + (Y(\mathbf{s}) - \mu)^2 \right] && \text{(Expandindo o quadrado)} \\
&= E[(Y(\mathbf{s}+\mathbf{h}) - \mu)^2] - 2E[(Y(\mathbf{s}+\mathbf{h}) - \mu)(Y(\mathbf{s}) - \mu)] + E[(Y(\mathbf{s}) - \mu)^2] && \text{(Linearidade da Esperança)} \\
&= Var(Y(\mathbf{s}+\mathbf{h})) - 2Cov(Y(\mathbf{s}+\mathbf{h}), Y(\mathbf{s})) + Var(Y(\mathbf{s})) && \text{(Definições de Var e Cov)} \\
&= C(0) - 2C(\mathbf{h}) + C(0) && \text{(Estacionariedade de 2ª ordem)} \\
&= 2(C(0) - C(\mathbf{h}))
\end{aligned}}\end{equation}

Portanto, simplificando por 2, obtemos a relação fundamental para
processos estacionários:

\begin{equation}\phantomsection\label{eq-relacao_cov_var}{\gamma(\mathbf{h})= C(0) - C(\mathbf{h})}\end{equation}

A relação entre estas duas abordagens é hierárquica. Se um processo for
estacionário de segunda ordem, ele é necessariamente intrínseco,
existindo uma relação analítica direta que conecta o semivariograma
\(\gamma(\mathbf{h})\) à função de covariância
Eq.~\ref{eq-relacao_cov_var}. Esta equação revela que o semivariograma é
a imagem espelhada da covariância: enquanto a covariância
\(C(\mathbf{h})\) decresce com a distância (de \(\sigma^2\) para
assintoticamente 0), o semivariograma \(\gamma(\mathbf{h})\) cresce com
a distância (de 0 para um patamar \(\sigma^2\)). No entanto, a
formulação intrínseca é mais robusta, pois o variograma
\(2\gamma(\mathbf{h})\) permanece definido mesmo para processos com
variância infinita onde a covariância \(C(\mathbf{h})\) não existe, o
que justifica a preferência de Matheron pelo variograma como ferramenta
fundamental de análise estrutural Figura~\ref{fig-cov-vs-vario} .

\begin{Shaded}
\begin{Highlighting}[]
\NormalTok{nugget }\OtherTok{\textless{}{-}} \DecValTok{2}
\NormalTok{partial\_sill }\OtherTok{\textless{}{-}} \DecValTok{8}
\NormalTok{range\_val }\OtherTok{\textless{}{-}} \DecValTok{20}
\NormalTok{sill }\OtherTok{\textless{}{-}}\NormalTok{ nugget }\SpecialCharTok{+}\NormalTok{ partial\_sill}

\NormalTok{h }\OtherTok{\textless{}{-}} \FunctionTok{seq}\NormalTok{(}\DecValTok{0}\NormalTok{, }\DecValTok{30}\NormalTok{, }\AttributeTok{by =} \FloatTok{0.1}\NormalTok{)}

\NormalTok{gamma }\OtherTok{\textless{}{-}} \FunctionTok{ifelse}\NormalTok{(h }\SpecialCharTok{==} \DecValTok{0}\NormalTok{, }\DecValTok{0}\NormalTok{,}
         \FunctionTok{ifelse}\NormalTok{(h }\SpecialCharTok{\textless{}=}\NormalTok{ range\_val, }
\NormalTok{                nugget }\SpecialCharTok{+}\NormalTok{ partial\_sill }\SpecialCharTok{*}\NormalTok{ (}\FloatTok{1.5} \SpecialCharTok{*}\NormalTok{ (h}\SpecialCharTok{/}\NormalTok{range\_val) }\SpecialCharTok{{-}} \FloatTok{0.5} \SpecialCharTok{*}\NormalTok{ (h}\SpecialCharTok{/}\NormalTok{range\_val)}\SpecialCharTok{\^{}}\DecValTok{3}\NormalTok{),}
\NormalTok{                sill))}

\NormalTok{gamma[}\DecValTok{1}\NormalTok{] }\OtherTok{\textless{}{-}}\NormalTok{ nugget }

\NormalTok{df\_vgm }\OtherTok{\textless{}{-}} \FunctionTok{data.frame}\NormalTok{(}\AttributeTok{Distancia =}\NormalTok{ h, }\AttributeTok{Semivariancia =}\NormalTok{ gamma)}


\NormalTok{df\_vgm}\SpecialCharTok{$}\NormalTok{Covariancia }\OtherTok{\textless{}{-}}\NormalTok{ sill }\SpecialCharTok{{-}}\NormalTok{ df\_vgm}\SpecialCharTok{$}\NormalTok{Semivariancia}

\NormalTok{df\_comp }\OtherTok{\textless{}{-}} \FunctionTok{pivot\_longer}\NormalTok{(df\_vgm, }\AttributeTok{cols =} \FunctionTok{c}\NormalTok{(Semivariancia, Covariancia), }\AttributeTok{names\_to =} \StringTok{"Funcao"}\NormalTok{, }\AttributeTok{values\_to =} \StringTok{"Valor"}\NormalTok{)}
\FunctionTok{ggplot}\NormalTok{(df\_comp, }\FunctionTok{aes}\NormalTok{(}\AttributeTok{x =}\NormalTok{ Distancia, }\AttributeTok{y =}\NormalTok{ Valor, }\AttributeTok{color =}\NormalTok{ Funcao)) }\SpecialCharTok{+}
  \FunctionTok{geom\_line}\NormalTok{(}\AttributeTok{size =} \DecValTok{1}\NormalTok{) }\SpecialCharTok{+}
  \FunctionTok{scale\_color\_manual}\NormalTok{(}\AttributeTok{values =} \FunctionTok{c}\NormalTok{(}\StringTok{"Covariancia"} \OtherTok{=} \StringTok{"darkred"}\NormalTok{, }\StringTok{"Semivariancia"} \OtherTok{=} \StringTok{"steelblue"}\NormalTok{),}
                     \AttributeTok{labels =} \FunctionTok{c}\NormalTok{(}\FunctionTok{expression}\NormalTok{(}\FunctionTok{C}\NormalTok{(h)), }\FunctionTok{expression}\NormalTok{(}\FunctionTok{gamma}\NormalTok{(h)))) }\SpecialCharTok{+}
  \FunctionTok{labs}\NormalTok{(}\AttributeTok{y =} \StringTok{"Valor"}\NormalTok{, }\AttributeTok{x =} \StringTok{"Distância (h)"}\NormalTok{, }\AttributeTok{color =} \StringTok{"Função:"}\NormalTok{) }\SpecialCharTok{+}
  \FunctionTok{theme\_minimal}\NormalTok{() }\SpecialCharTok{+}
  \FunctionTok{theme}\NormalTok{(}\AttributeTok{legend.position =} \StringTok{"bottom"}\NormalTok{)}
\end{Highlighting}
\end{Shaded}

\begin{figure}[H]

\centering{

\pandocbounded{\includegraphics[keepaspectratio]{geostat_files/figure-pdf/fig-cov-vs-vario-1.pdf}}

}

\caption{\label{fig-cov-vs-vario}Covariância vs.~Semivariograma em
Processo Estacionário.}

\end{figure}%

\section{Efeito Pepita e suas Implicações na
Predição}\label{sec-efeito_pepita}

A caracterização da continuidade espacial de um fenômeno é
frequentemente realizada através do variograma experimental (ver secção
seguinte). Em muitos casos, observa-se que, à medida que a distância de
separação (\(\|\mathbf{h}\|\)) entre dois pontos tende a zero, o valor
do semivariograma (\(\gamma(\mathbf{h})\)) não converge para zero, mas
sim para um valor positivo. Esta descontinuidade na origem é denominada
Efeito Pepita (\(c_0\) ou
\href{https://stats.stackexchange.com/questions/324825/what-is-the-nugget-effect}{nugget
effect}). O termo Efeito Pepita (ou
\href{https://en.wikipedia.org/wiki/Gold_nugget}{Nugget Effect}) tem
origem histórica nas minas de ouro sul-africanas. Matheron observou que,
mesmo quando duas amostras eram recolhidas muito próximas uma da outra,
os seus valores podiam diferir devido à presença de
\href{https://pt.wikipedia.org/wiki/Pepita}{pepitas}, microscópicas de
ouro distribuídas aleatoriamente.

Noel Cressie (1993) define este parâmetro como a soma das variâncias das
duas componentes descritas acima:

\[c_0 = \lim_{\|\mathbf{h}\| \to 0} \gamma(\mathbf{h}) = \text{Var}(\eta(\mathbf{s})) + \text{Var}(\epsilon(\mathbf{s}))\]

Esta distinção permite-nos definir o ``sinal'' ou processo latente
\(S(\mathbf{s})\) que desejamos recuperar, livre do erro de medição
instrumental:

\[S(\mathbf{s}) = \mu(\mathbf{s}) + W(\mathbf{s}) + \eta(\mathbf{s})\]

Consequentemente, os dados observados são compostos pelo sinal mais o
erro de medição:
\(Y(\mathbf{s}) = S(\mathbf{s}) + \epsilon(\mathbf{s})\). Esta
formulação alinha-se com a Geoestatística Baseada em Modelos (Peter J.
Diggle, Tawn, e Moyeed 1998), que assume uma hierarquia:

\begin{enumerate}
\def\labelenumi{\arabic{enumi}.}
\item
  Modelo de Processo: \([S(\cdot)]\) (descreve a física do fenômeno).
\item
  Modelo de Dados: \([Y(\cdot) | S(\cdot)]\) (descreve a observação
  ruidosa desse fenômeno).
\end{enumerate}

Noel Cressie (1993) e Laslett (1994) destacam uma implicação fundamental
desta decomposição. As equações de predição (Krigagem) devem ser
ajustadas dependendo do nosso objetivo final:

\begin{enumerate}
\def\labelenumi{\arabic{enumi}.}
\item
  \textbf{Predição do dado observável} (\(Y\)): Se o objetivo é prever o
  valor que um sensor mediria no local \(\mathbf{s}_0\) (incluindo o
  erro inerente ao sensor), utilizamos a Krigagem Exata. Este
  interpolador respeita os dados originais e incorpora todo o \(c_0\) na
  variabilidade estimada.
\item
  \textbf{Predição do processo latente} (\(S\)): Se o objetivo é prever
  o valor físico real do fenômeno, filtrado do ruído instrumental,
  utilizamos a Krigagem com Suavização. Isto é feito subtraindo a
  variância do erro de medição (\(\text{Var}(\epsilon)\)) da diagonal da
  matriz de covariância do sistema de krigagem.
\end{enumerate}

Se ignorarmos esta distinção e tratarmos todo o efeito pepita como
variabilidade natural (assumindo \(\text{Var}(\epsilon)=0\)), os nossos
mapas serão desnecessariamente ruidosos (``rugosos'') e respeitarão
erros de medição como se fossem verdades. Por outro lado, se tratarmos
todo o pepita como erro, obteremos mapas mais suaves.

\begin{tcolorbox}[enhanced jigsaw, left=2mm, toptitle=1mm, colback=white, colframe=quarto-callout-important-color-frame, colbacktitle=quarto-callout-important-color!10!white, opacityback=0, rightrule=.15mm, bottomtitle=1mm, arc=.35mm, title=\textcolor{quarto-callout-important-color}{\faExclamation}\hspace{0.5em}{O Problema da Identificabilidade}, titlerule=0mm, bottomrule=.15mm, leftrule=.75mm, coltitle=black, toprule=.15mm, breakable, opacitybacktitle=0.6]

É importante notar que, sem medições repetidas no mesmo local
(co-localizadas), é impossível separar estatisticamente o quanto do
\(c_0\) se deve a \(\eta(s)\) (microescala) e o quanto se deve a
\(\epsilon (s)\) (erro). Noel Cressie (1993) alerta que esta distinção é
frequentemente uma escolha de modelagem baseada no conhecimento do
equipamento, e não uma inferência puramente estatística.

\end{tcolorbox}

\begin{tcolorbox}[enhanced jigsaw, left=2mm, toptitle=1mm, colback=white, colframe=quarto-callout-note-color-frame, colbacktitle=quarto-callout-note-color!10!white, opacityback=0, rightrule=.15mm, bottomtitle=1mm, arc=.35mm, title=\textcolor{quarto-callout-note-color}{\faInfo}\hspace{0.5em}{Tendência e Erro}, titlerule=0mm, bottomrule=.15mm, leftrule=.75mm, coltitle=black, toprule=.15mm, breakable, opacitybacktitle=0.6]

Wackernagel (2003) demonstra que, em domínios pequenos, uma tendência
local pode ser indistinguível de uma flutuação estocástica de baixa
frequência: A estrutura determinística média de uma pessoa pode ser a
estrutura de erro correlacionado de outra.

\end{tcolorbox}

\section{Variograma e funções de covariância}\label{sec-variograma}

Uma vez assumida a hipótese de estacionariedade, a inferência
geoestatística exige a quantificação da dependência espacial do processo
estocástico \(Y(\mathbf{s})\). Diferentemente da estatística clássica,
onde a correlação é frequentemente um escalar único, na geoestatística a
dependência é modelada como uma função contínua do vetor de separação
\(\mathbf{h} \in \mathbb{R}^d\) entre pares de pontos.

A caracterização desta estrutura é realizada através de duas ferramentas
fundamentais, cuja aplicabilidade depende do nível de estacionariedade
assumido: a Função de Covariância (Estacionariedade de 2.ª Ordem) e o
Variograma (Estacionariedade Intrínseca).

\textbf{Função de Covariância}

A função de covariância \(C(\mathbf{h})\) quantifica a covariância
linear entre os valores do processo estocástico \(Y(\mathbf{s})\) em
dois locais \(\mathbf{s}\) e \(\mathbf{s} + \mathbf{h}\), separados por
um vetor de distância \(\mathbf{h} \in \mathbb{R}^d\):

\[C(\mathbf{h}) = \text{Cov}(Y(\mathbf{s}), Y(\mathbf{s} + \mathbf{h})) = E[(Y(\mathbf{s}) - \mu)(Y(\mathbf{s} + \mathbf{h}) - \mu)]\]

onde \(\mu\) é a média estacionária do processo Noel Cressie (1993). Sob
a hipótese de estacionariedade de segunda ordem, a covariância depende
apenas de \(\mathbf{h}\) e é simétrica em relação à origem, ou seja,
\(C(\mathbf{h}) = C(-\mathbf{h})\).

Como descrito anteriormente, uma consequência fundamental desta
definição é que a covariância na origem, \(C(\mathbf{0})\), corresponde
à variância a priori do processo, \(\sigma^2_Y\), que se assume finita e
constante em todo o domínio \(D^G\),
\(C(\mathbf{0}) = \text{Var}(Y(\mathbf{s})) = \sigma^2_Y\).

Para garantir a validade estatística das predições (Krigagem), a função
de covariância deve ser positiva definida, o que implica que a variância
de qualquer combinação linear das variáveis aleatórias seja não negativa
George Matheron (1971).

\textbf{Variograma}

Em situações nas quais a variância do processo não é finita, a função de
covariância não pode ser definida (por exemplo: movimento Browniano,
topografia em grandes escalas, etc.). Esta condição de não-finitude,
central na formulação da hipótese intrínseca por George Matheron (1971),
não implica que um valor pontual seja infinito, mas sim que a dispersão
a priori do processo cresce indefinidamente à medida que o domínio de
observação se expande (\(|D| \to \infty\)). Um exemplo clássico é o
movimento Browniano unidimensional (processo de Wiener), denotado por
\(Y(t)\), em que a variância da posição da partícula no instante \(t\) é
linearmente proporcional ao tempo decorrido, expressa por
\(\text{Var}(Y(t)) = \sigma^2 t\). Consequentemente, num domínio
temporal ilimitado (\(t \to \infty\)), a variância global do processo
tende ao infinito (\(\sigma^2_{global} \to \infty\)), tornando
matematicamente impossível a definição de um patamar \(C(0)\) ou de uma
função de covariância estacionária.

Nesses casos, onde a estacionariedade de segunda ordem é violada pela
ausência de variância finita, utiliza-se o variograma
\(2\gamma(\mathbf{h})\). Essa ferramenta baseia-se na hipótese de
estacionaridade intrínseca introduzida por G. Matheron (1963), assumindo
estacionariedade apenas para os incrementos do processo. Isto implica
que os incrementos permanecem finitos e estáveis, mesmo quando a
variância absoluta diverge. O variograma é definido como a variância da
diferença entre os valores da variável em dois locais separados por um
vetor \(\mathbf{h}\) e, assumindo que a média dos incrementos é zero,
equivale ao valor esperado do quadrado dessas diferenças:

\[2\gamma(\mathbf{h}) = \text{Var}(Y(\mathbf{s} + \mathbf{h}) - Y(\mathbf{s})) = E[(Y(\mathbf{s} + \mathbf{h}) - Y(\mathbf{s}))^2]\]

O semivariograma \(\gamma(\mathbf{h})\) corresponde à metade do
variograma. Sob a hipótese intrínseca, assume-se que o variograma
depende apenas do vetor de separação \(\mathbf{h}\), garantindo a
caracterização da continuidade espacial mesmo sem variância global
definida. Contudo, se o processo satisfizer a estacionariedade de
segunda ordem (onde \(C(\mathbf{0})\) existe e é finito), estabelece-se
uma relação analítica direta entre o variograma e a função de
covariância (ver a dedução na Eq.~\ref{eq-semivariograma}).

\textbf{Elementos do Semivariograma}

Antes de procedermos à modelagem teórica, é crucial compreender os
parâmetros que constituem o perfil de um variograma pois a correta
identificação destes parâmetros condiciona diretamente a estrutura de
covariância utilizada na predição espacial (Yamamoto e Landim 2013).

\begin{enumerate}
\def\labelenumi{\arabic{enumi}.}
\tightlist
\item
  \textbf{Alcance (}\(a\) ou Range)
\end{enumerate}

O alcance define o limite da dependência espacial. É a distância no eixo
das abcissas (\(\|\mathbf{h}\|\)) a partir da qual a correlação entre
observações se torna desprezível e o variograma estabiliza. Em termos
práticos, pontos separados por uma distância \(\|\mathbf{h}\| \ge a\)
são considerados estatisticamente independentes (ou não
correlacionados). Este parâmetro impõe o critério crítico para a
amostragem: para capturar a estrutura do fenômeno, a malha de amostragem
deve possuir um espaçamento inferior ao alcance. Caso contrário,
qualquer interpolação entre os pontos será meramente especulativa, uma
vez que, além desta fronteira, a predição reverte estatisticamente para
a média global do processo.

\begin{enumerate}
\def\labelenumi{\arabic{enumi}.}
\setcounter{enumi}{1}
\tightlist
\item
  \textbf{Efeito Pepita (}\(C_0\) ou Nugget Effect)
\end{enumerate}

Teoricamente, pela definição de variância de um incremento nulo,
\(\gamma(\mathbf{0}) = 0\). Contudo, o variograma experimental
frequentemente exibe uma descontinuidade na origem, intercetando o eixo
das ordenadas num valor positivo \(C_0 > 0\). Como descrito na secção
Seção~\ref{sec-efeito_pepita}, Noel Cressie (1993) e Peter J. Diggle,
Tawn, e Moyeed (1998) decompõem este parâmetro na soma de duas fontes de
variabilidade que operam em escalas sub-amostrais: a variabilidade de
microescala e o erro de medição
(\(C_0 =\text{Var}(\eta(\mathbf{s})) + \text{Var}(\epsilon(\mathbf{s}))\)).
Embora seja um parâmetro a modelar, é desejável que a sua magnitude seja
reduzida em comparação com a variância total (\(C_0 <C\)), indicando que
o ruído não domina o sinal espacial.

\begin{enumerate}
\def\labelenumi{\arabic{enumi}.}
\setcounter{enumi}{2}
\tightlist
\item
  \textbf{Contribuição (}\(C\) ou Partial Sill)
\end{enumerate}

A Contribuição representa a componente da variância que é explicitamente
explicada pela estrutura de dependência espacial. Corresponde à
amplitude do crescimento do variograma, ou seja, a diferença entre o
valor onde a função estabiliza e o ponto onde intercepta o eixo \(Y\)
(Efeito Pepita). É neste segmento da curva que reside a informação
espacial útil para a krigagem: quanto maior for o valor de \(C\) em
relação a \(C_0\), mais forte e contínuo é o padrão espacial do
fenômeno.

\begin{enumerate}
\def\labelenumi{\arabic{enumi}.}
\setcounter{enumi}{3}
\tightlist
\item
  \textbf{Patamar (}\(C_0 + C\) ou Sill)
\end{enumerate}

O Patamar é o valor assintótico onde a função \(\gamma(\mathbf{h})\)
estabiliza. Sob a hipótese de estacionariedade de segunda ordem, este
valor corresponde teoricamente à variância total a priori do processo
(\(\sigma^2 = C(\mathbf{0})\)). Existe uma relação analítica de
aditividade que conecta os elementos de variância descritos acima:

\[\text{Patamar} = \text{Efeito Pepita} (C_0) + \text{Contribuição} (C)\]

Se o variograma não estabilizar num patamar e continuar a crescer
indefinidamente, isso indica que o processo não possui variância finita
(é intrínseco) ou que existe uma tendência (drift) não removida nos
dados.

\textbf{Interpretação Gráfica}

Para visualizar estes componentes no gráfico do semivariograma,
observa-se o comportamento da curva \(\gamma(\mathbf{h})\) em relação
aos eixos cartesianos. O gráfico inicia-se no eixo Y à altura do Efeito
Pepita (\(C_0\)). À medida que a distância \(\mathbf{h}\) (eixo X)
aumenta, a semivariância cresce com uma amplitude igual à Contribuição
(\(C\)). A curva cessa o seu crescimento quando a distância atinge o
Alcance (\(a\)), momento em que a função se torna horizontal, fixando-se
no valor do Patamar (\(C_0 + C\)). Portanto, a altura total da curva
representa a variabilidade total dos dados, a qual é particionada em
variabilidade não explicada (pepita) e variabilidade estruturada
(contribuição) Figura~\ref{fig-variograma-teorico}.

\begin{tcolorbox}[enhanced jigsaw, left=2mm, toptitle=1mm, colback=white, colframe=quarto-callout-note-color-frame, colbacktitle=quarto-callout-note-color!10!white, opacityback=0, rightrule=.15mm, bottomtitle=1mm, arc=.35mm, title=\textcolor{quarto-callout-note-color}{\faInfo}\hspace{0.5em}{Nota}, titlerule=0mm, bottomrule=.15mm, leftrule=.75mm, coltitle=black, toprule=.15mm, breakable, opacitybacktitle=0.6]

Uma discussão aprofundada sobre as implicações do efeito pepita na
escolha entre Krigagem Exata e Krigagem com Suavização encontra-se na
Seção~\ref{sec-efeito_pepita}.

\end{tcolorbox}

\begin{Shaded}
\begin{Highlighting}[]
\FunctionTok{ggplot}\NormalTok{(df\_vgm, }\FunctionTok{aes}\NormalTok{(}\AttributeTok{x =}\NormalTok{ Distancia, }\AttributeTok{y =}\NormalTok{ Semivariancia)) }\SpecialCharTok{+}
  \FunctionTok{geom\_line}\NormalTok{(}\AttributeTok{color =} \StringTok{"black"}\NormalTok{, }\AttributeTok{size =} \DecValTok{1}\NormalTok{) }\SpecialCharTok{+}
  \FunctionTok{geom\_hline}\NormalTok{(}\AttributeTok{yintercept =}\NormalTok{ sill, }\AttributeTok{linetype =} \StringTok{"dashed"}\NormalTok{, }\AttributeTok{color =} \StringTok{"black"}\NormalTok{) }\SpecialCharTok{+}
  \FunctionTok{geom\_vline}\NormalTok{(}\AttributeTok{xintercept =}\NormalTok{ range\_val, }\AttributeTok{linetype =} \StringTok{"dotted"}\NormalTok{, }\AttributeTok{color =} \StringTok{"black"}\NormalTok{) }\SpecialCharTok{+}
  \FunctionTok{geom\_hline}\NormalTok{(}\AttributeTok{yintercept =}\NormalTok{ nugget, }\AttributeTok{linetype =} \StringTok{"dashed"}\NormalTok{, }\AttributeTok{color =} \StringTok{"black"}\NormalTok{) }\SpecialCharTok{+}
  \FunctionTok{annotate}\NormalTok{(}\StringTok{"text"}\NormalTok{, }\AttributeTok{x =} \DecValTok{25}\NormalTok{, }\AttributeTok{y =}\NormalTok{ sill }\SpecialCharTok{+} \FloatTok{0.5}\NormalTok{, }\AttributeTok{label =} \StringTok{"Patamar (Sill)"}\NormalTok{, }\AttributeTok{color =} \StringTok{"black"}\NormalTok{) }\SpecialCharTok{+}
  \FunctionTok{annotate}\NormalTok{(}\StringTok{"text"}\NormalTok{, }\AttributeTok{x =}\NormalTok{ range\_val }\SpecialCharTok{+} \FloatTok{0.5}\NormalTok{, }\AttributeTok{y =} \DecValTok{2}\NormalTok{, }\AttributeTok{label =} \StringTok{"Alcance (Range)"}\NormalTok{, }\AttributeTok{angle =} \DecValTok{90}\NormalTok{, }\AttributeTok{color =} \StringTok{"black"}\NormalTok{) }\SpecialCharTok{+}
  \FunctionTok{geom\_point}\NormalTok{(}\FunctionTok{aes}\NormalTok{(}\AttributeTok{x =} \DecValTok{0}\NormalTok{, }\AttributeTok{y =}\NormalTok{ nugget), }\AttributeTok{color =} \StringTok{"red"}\NormalTok{, }\AttributeTok{size =} \DecValTok{3}\NormalTok{) }\SpecialCharTok{+}
  \FunctionTok{annotate}\NormalTok{(}\StringTok{"text"}\NormalTok{, }\AttributeTok{x =} \DecValTok{1}\NormalTok{, }\AttributeTok{y =}\NormalTok{ nugget }\SpecialCharTok{{-}} \FloatTok{0.5}\NormalTok{, }\AttributeTok{label =} \StringTok{"Efeito Pepita (C0)"}\NormalTok{, }\AttributeTok{color =} \StringTok{"black"}\NormalTok{, }\AttributeTok{hjust =} \DecValTok{0}\NormalTok{) }\SpecialCharTok{+}
  \FunctionTok{annotate}\NormalTok{(}\StringTok{"curve"}\NormalTok{, }
           \AttributeTok{x =} \FloatTok{1.5}\NormalTok{, }\AttributeTok{y =}\NormalTok{ nugget }\SpecialCharTok{{-}} \FloatTok{0.2}\NormalTok{,       }
           \AttributeTok{xend =} \FloatTok{0.15}\NormalTok{, }\AttributeTok{yend =}\NormalTok{ nugget }\SpecialCharTok{{-}} \FloatTok{0.05}\NormalTok{, }
           \AttributeTok{arrow =} \FunctionTok{arrow}\NormalTok{(}\AttributeTok{length =} \FunctionTok{unit}\NormalTok{(}\FloatTok{0.2}\NormalTok{, }\StringTok{"cm"}\NormalTok{), }\AttributeTok{type =} \StringTok{"closed"}\NormalTok{), }
           \AttributeTok{color =} \StringTok{"black"}\NormalTok{, }
           \AttributeTok{curvature =} \FloatTok{0.3}\NormalTok{) }\SpecialCharTok{+}
  \FunctionTok{annotate}\NormalTok{(}\StringTok{"segment"}\NormalTok{, }\AttributeTok{x =} \DecValTok{28}\NormalTok{, }\AttributeTok{xend =} \DecValTok{28}\NormalTok{, }\AttributeTok{y =}\NormalTok{ nugget, }\AttributeTok{yend =}\NormalTok{ sill, }
           \AttributeTok{arrow =} \FunctionTok{arrow}\NormalTok{(}\AttributeTok{ends =} \StringTok{"both"}\NormalTok{, }\AttributeTok{length =} \FunctionTok{unit}\NormalTok{(}\FloatTok{0.2}\NormalTok{, }\StringTok{"cm"}\NormalTok{)), }\AttributeTok{color =} \StringTok{"black"}\NormalTok{) }\SpecialCharTok{+}
  \FunctionTok{annotate}\NormalTok{(}\StringTok{"text"}\NormalTok{, }\AttributeTok{x =} \FloatTok{28.5}\NormalTok{, }\AttributeTok{y =}\NormalTok{ (nugget }\SpecialCharTok{+}\NormalTok{ sill)}\SpecialCharTok{/}\DecValTok{2} \SpecialCharTok{{-}}\DecValTok{2}\NormalTok{ , }\AttributeTok{label =} \StringTok{"Contribuição (C)"}\NormalTok{, }\AttributeTok{angle =} \DecValTok{90}\NormalTok{, }\AttributeTok{color =} \StringTok{"black"}\NormalTok{, }\AttributeTok{hjust =} \DecValTok{0}\NormalTok{) }\SpecialCharTok{+}
  \FunctionTok{scale\_y\_continuous}\NormalTok{(}\AttributeTok{limits =} \FunctionTok{c}\NormalTok{(}\DecValTok{0}\NormalTok{, }\DecValTok{12}\NormalTok{), }\AttributeTok{breaks =} \FunctionTok{seq}\NormalTok{(}\DecValTok{0}\NormalTok{, }\DecValTok{12}\NormalTok{, }\DecValTok{2}\NormalTok{)) }\SpecialCharTok{+}
  \FunctionTok{scale\_x\_continuous}\NormalTok{(}\AttributeTok{expand =} \FunctionTok{c}\NormalTok{(}\DecValTok{0}\NormalTok{, }\DecValTok{0}\NormalTok{), }\AttributeTok{limits =} \FunctionTok{c}\NormalTok{(}\DecValTok{0}\NormalTok{, }\DecValTok{30}\NormalTok{)) }\SpecialCharTok{+}
  \FunctionTok{labs}\NormalTok{(}\AttributeTok{x =} \StringTok{"Distância de Separação (h)"}\NormalTok{, }\AttributeTok{y =} \FunctionTok{expression}\NormalTok{(}\FunctionTok{gamma}\NormalTok{(h))) }\SpecialCharTok{+}
  \FunctionTok{theme\_classic}\NormalTok{()}
\end{Highlighting}
\end{Shaded}

\begin{figure}[H]

\centering{

\pandocbounded{\includegraphics[keepaspectratio]{geostat_files/figure-pdf/fig-variograma-teorico-1.pdf}}

}

\caption{\label{fig-variograma-teorico}Elementos Teóricos do
Semivariograma: Alcance, patamar, efeito pepita e contribuição.}

\end{figure}%

\section{Estimadores do Variograma}\label{estimadores-do-variograma}

A definição teórica do variograma,
\(2\gamma(\mathbf{h}) = E[(Y(\mathbf{s}+\mathbf{h}) - Y(\mathbf{s}))^2]\),
pressupõe o conhecimento da distribuição de probabilidade conjunta do
processo estocástico \(Y(\mathbf{s})\). Contudo, na prática
geoestatística, o investigador dispõe apenas de uma única realização
finita do processo, materializada num conjunto discreto de observações
amostrais
\(\mathbf{y} = \{y(\mathbf{s}_1), y(\mathbf{s}_2), \dots, y(\mathbf{s}_n)\}\).
Consequentemente, a função teórica \(\gamma(\mathbf{h})\) é desconhecida
e deve ser estimada empiricamente a partir destes dados. A qualidade
desta estimativa é crítica, uma vez que o variograma experimental
constitui a base para o ajuste do modelo teórico que alimentará as
equações de krigagem.

\textbf{Estimador Experimental Clássico (Matheron)}

Proposto por G. Matheron (1963), este estimador baseia-se no método dos
momentos. Ele calcula a média dos quadrados das diferenças entre pares
de pontos separados por um vetor \(\mathbf{h}\) (dentro de uma
determinada tolerância de distância e direção). O semivariograma
experimental \(\hat{\gamma}_{M}(\mathbf{h})\) é dado por:

\begin{equation}\phantomsection\label{eq-estimador_classico}{\hat{\gamma}_{M}(\mathbf{h}) = \frac{1}{2|N(\mathbf{h})|} \sum_{(\mathbf{s}_i, \mathbf{s}_j) \in N(\mathbf{h})} (y(\mathbf{s}_i) - y(\mathbf{s}_j))^2}\end{equation}

Onde \(N(\mathbf{h})\) representa o conjunto de pares de localizações
\((\mathbf{s}_i, \mathbf{s}_j)\) tal que a separação
\(\mathbf{s}_i - \mathbf{s}_j\) se aproxima do vetor \(\mathbf{h}\), e
\(|N(\mathbf{h})|\) denota a cardinalidade (número de pares) desse
conjunto.

\textbf{Estimador Robusto (Cressie-Hawkins)}

Embora o estimador clássico Eq.~\ref{eq-estimador_classico} seja
não-viesado para processos Gaussianos, ele apresenta uma vulnerabilidade
intrínseca: a elevação das diferenças ao quadrado,
\((y(\mathbf{s}_i) - y(\mathbf{s}_j))^2\), amplifica
desproporcionalmente o impacto de valores extremos. A presença de um
único outlier ou erro grosseiro na amostragem pode inflar a variância
estimada em determinadas classes de distância, distorcendo a estrutura
de continuidade espacial e dificultando a modelagem do patamar e do
alcance.

Em resposta à sensibilidade do estimador clássico a dados contaminados e
a distribuições de cauda longa, Noel Cressie e Hawkins (1980)
desenvolveram uma alternativa mais resiliente, conhecida como Estimador
Robusto de Cressie-Hawkins. A premissa deste método reside na suavização
das diferenças extremas através de uma transformação de raiz quadrada,
aproximando a distribuição dos incrementos à normalidade antes do
cálculo da média. A expressão para o estimador robusto
\(\hat{\gamma}_{CH}(\mathbf{h})\) é definida como:

\begin{equation}\phantomsection\label{eq-estimador_robusto}{\hat{\gamma}_{CH}(\mathbf{h}) = \frac{\left( \frac{1}{|N(\mathbf{h})|} \sum_{(\mathbf{s}_i, \mathbf{s}_j) \in N(\mathbf{h})} |y(\mathbf{s}_i) - y(\mathbf{s}_j)|^{1/2} \right)^4}{0.457 + \frac{0.494}{|N(\mathbf{h})|}}}\end{equation}

O denominador na Eq.~\ref{eq-estimador_robusto} atua como um fator de
correção de viés para amostras finitas, garantindo a consistência
estatística do estimador. A aplicação comparativa de ambos os
estimadores constitui uma prática recomendada na fase de Análise
Exploratória de Dados Espaciais (ESDA). Uma divergência significativa
entre o perfil do variograma clássico e o do robusto especificamente, se
\(\hat{\gamma}_{M}(\mathbf{h})\) apresentar flutuações erráticas ou
valores excessivamente elevados nas curtas distâncias em comparação com
\(\hat{\gamma}_{CH}(\mathbf{h})\), é um indicador forte da presença de
outliers ou de não-normalidade severa nos dados
Figura~\ref{fig-estimadores-comparacao}, sugerindo a necessidade de
adoção da abordagem robusta para a modelagem ou verificar possíveis
falhas resultantes da ação humana (Noel Cressie 1993) .

\begin{Shaded}
\begin{Highlighting}[]
\ControlFlowTok{if}\NormalTok{ (}\SpecialCharTok{!}\FunctionTok{require}\NormalTok{(}\StringTok{"pacman"}\NormalTok{)) }\FunctionTok{install.packages}\NormalTok{(}\StringTok{"pacman"}\NormalTok{)}
\NormalTok{pacman}\SpecialCharTok{::}\FunctionTok{p\_load}\NormalTok{(ggplot2, gstat, sf, viridis, dplyr)}

\FunctionTok{set.seed}\NormalTok{(}\DecValTok{123}\NormalTok{)}
\NormalTok{grid\_exemplo }\OtherTok{\textless{}{-}} \FunctionTok{expand.grid}\NormalTok{(}\AttributeTok{x =} \DecValTok{1}\SpecialCharTok{:}\DecValTok{50}\NormalTok{, }\AttributeTok{y =} \DecValTok{1}\SpecialCharTok{:}\DecValTok{50}\NormalTok{)}
\NormalTok{grid\_sf\_ex }\OtherTok{\textless{}{-}} \FunctionTok{st\_as\_sf}\NormalTok{(grid\_exemplo, }\AttributeTok{coords =} \FunctionTok{c}\NormalTok{(}\StringTok{"x"}\NormalTok{, }\StringTok{"y"}\NormalTok{))}

\NormalTok{modelo\_verdadeiro }\OtherTok{\textless{}{-}}\NormalTok{ gstat}\SpecialCharTok{::}\FunctionTok{vgm}\NormalTok{(}\AttributeTok{psill =} \DecValTok{10}\NormalTok{, }\AttributeTok{model =} \StringTok{"Exp"}\NormalTok{, }\AttributeTok{range =} \DecValTok{15}\NormalTok{, }\AttributeTok{nugget =} \DecValTok{2}\NormalTok{)}

\NormalTok{g\_sim }\OtherTok{\textless{}{-}}\NormalTok{ gstat}\SpecialCharTok{::}\FunctionTok{gstat}\NormalTok{(}\AttributeTok{formula =}\NormalTok{ z}\SpecialCharTok{\textasciitilde{}}\DecValTok{1}\NormalTok{, }\AttributeTok{locations =}\NormalTok{ grid\_sf\_ex, }\AttributeTok{dummy =} \ConstantTok{TRUE}\NormalTok{, }\AttributeTok{beta =} \DecValTok{0}\NormalTok{, }\AttributeTok{model =}\NormalTok{ modelo\_verdadeiro, }\AttributeTok{nmax =} \DecValTok{20}\NormalTok{)}
\NormalTok{simulacao }\OtherTok{\textless{}{-}} \FunctionTok{predict}\NormalTok{(g\_sim, }\AttributeTok{newdata =}\NormalTok{ grid\_sf\_ex, }\AttributeTok{nsim =} \DecValTok{1}\NormalTok{)}
\end{Highlighting}
\end{Shaded}

\begin{verbatim}
[using unconditional Gaussian simulation]
\end{verbatim}

\begin{Shaded}
\begin{Highlighting}[]
\NormalTok{simulacao}\SpecialCharTok{$}\NormalTok{z\_contaminado }\OtherTok{\textless{}{-}}\NormalTok{ simulacao}\SpecialCharTok{$}\NormalTok{sim1}

\CommentTok{\# 2. Introduzindo Outliers (Contaminação)}
\CommentTok{\# Adicionamos um erro grosseiro (+50) em apenas 0.2\% dos dados}
\FunctionTok{set.seed}\NormalTok{(}\DecValTok{999}\NormalTok{) }
\NormalTok{idx\_outliers }\OtherTok{\textless{}{-}} \FunctionTok{sample}\NormalTok{(}\DecValTok{1}\SpecialCharTok{:}\FunctionTok{nrow}\NormalTok{(simulacao), }\DecValTok{5}\NormalTok{)}
\NormalTok{simulacao}\SpecialCharTok{$}\NormalTok{z\_contaminado[idx\_outliers] }\OtherTok{\textless{}{-}}\NormalTok{ simulacao}\SpecialCharTok{$}\NormalTok{z\_contaminado[idx\_outliers] }\SpecialCharTok{+} \DecValTok{50}

\CommentTok{\#Variogramas}
\NormalTok{vgm\_classico }\OtherTok{\textless{}{-}}\NormalTok{ gstat}\SpecialCharTok{::}\FunctionTok{variogram}\NormalTok{(z\_contaminado }\SpecialCharTok{\textasciitilde{}} \DecValTok{1}\NormalTok{, }\AttributeTok{data =}\NormalTok{ simulacao, }\AttributeTok{cressie =} \ConstantTok{FALSE}\NormalTok{)}
\NormalTok{vgm\_classico}\SpecialCharTok{$}\NormalTok{Estimador }\OtherTok{\textless{}{-}} \StringTok{"Clássico (Matheron)"}

\NormalTok{vgm\_robusto }\OtherTok{\textless{}{-}}\NormalTok{ gstat}\SpecialCharTok{::}\FunctionTok{variogram}\NormalTok{(z\_contaminado }\SpecialCharTok{\textasciitilde{}} \DecValTok{1}\NormalTok{, }\AttributeTok{data =}\NormalTok{ simulacao, }\AttributeTok{cressie =} \ConstantTok{TRUE}\NormalTok{)}
\NormalTok{vgm\_robusto}\SpecialCharTok{$}\NormalTok{Estimador }\OtherTok{\textless{}{-}} \StringTok{"Robusto (Cressie)"}

\NormalTok{vgm\_comp }\OtherTok{\textless{}{-}} \FunctionTok{rbind}\NormalTok{(vgm\_classico, vgm\_robusto)}


\FunctionTok{ggplot}\NormalTok{() }\SpecialCharTok{+}
  \FunctionTok{geom\_point}\NormalTok{(}\AttributeTok{data =}\NormalTok{ vgm\_comp, }\FunctionTok{aes}\NormalTok{(}\AttributeTok{x =}\NormalTok{ dist, }\AttributeTok{y =}\NormalTok{ gamma, }\AttributeTok{color =}\NormalTok{ Estimador), }\AttributeTok{size =} \DecValTok{3}\NormalTok{, }\AttributeTok{alpha =} \FloatTok{0.8}\NormalTok{) }\SpecialCharTok{+} 
  \FunctionTok{geom\_line}\NormalTok{(}\AttributeTok{data =}\NormalTok{ vgm\_comp, }\FunctionTok{aes}\NormalTok{(}\AttributeTok{x =}\NormalTok{ dist, }\AttributeTok{y =}\NormalTok{ gamma, }\AttributeTok{color =}\NormalTok{ Estimador), }
            \AttributeTok{linewidth =} \FloatTok{0.6}\NormalTok{, }\AttributeTok{alpha =} \FloatTok{0.6}\NormalTok{) }\SpecialCharTok{+}
  \FunctionTok{stat\_function}\NormalTok{(}\AttributeTok{fun =} \ControlFlowTok{function}\NormalTok{(h) }\DecValTok{2} \SpecialCharTok{+} \DecValTok{10} \SpecialCharTok{*}\NormalTok{ (}\DecValTok{1} \SpecialCharTok{{-}} \FunctionTok{exp}\NormalTok{(}\SpecialCharTok{{-}}\NormalTok{h}\SpecialCharTok{/}\DecValTok{15}\NormalTok{)), }\FunctionTok{aes}\NormalTok{(}\AttributeTok{linetype =} \StringTok{"Modelo Verdadeiro (Gerador)"}\NormalTok{), }
                \AttributeTok{color =} \StringTok{"black"}\NormalTok{, }\AttributeTok{linewidth =} \DecValTok{1}\NormalTok{) }\SpecialCharTok{+}
  \FunctionTok{scale\_color\_manual}\NormalTok{(}\AttributeTok{values =} \FunctionTok{c}\NormalTok{(}\StringTok{"Clássico (Matheron)"} \OtherTok{=} \StringTok{"\#D55E00"}\NormalTok{, }\StringTok{"Robusto (Cressie)"} \OtherTok{=} \StringTok{"\#0072B2"}\NormalTok{)) }\SpecialCharTok{+}
  \FunctionTok{scale\_linetype\_manual}\NormalTok{(}\AttributeTok{name =} \StringTok{""}\NormalTok{, }\AttributeTok{values =} \FunctionTok{c}\NormalTok{(}\StringTok{"Modelo Verdadeiro (Gerador)"} \OtherTok{=} \StringTok{"dashed"}\NormalTok{)) }\SpecialCharTok{+}
  \FunctionTok{coord\_cartesian}\NormalTok{(}\AttributeTok{ylim =} \FunctionTok{c}\NormalTok{(}\DecValTok{0}\NormalTok{, }\DecValTok{25}\NormalTok{)) }\SpecialCharTok{+} 
  \FunctionTok{labs}\NormalTok{(}\AttributeTok{x =} \StringTok{"Distância (h)"}\NormalTok{, }
       \AttributeTok{y =} \FunctionTok{expression}\NormalTok{(Semivariância }\SpecialCharTok{\textasciitilde{}} \FunctionTok{gamma}\NormalTok{(h)), }\AttributeTok{color=}\StringTok{"Estimador:"}\NormalTok{) }\SpecialCharTok{+}
  \FunctionTok{theme\_minimal}\NormalTok{() }\SpecialCharTok{+}
  \FunctionTok{theme}\NormalTok{(}\AttributeTok{legend.position =} \StringTok{"bottom"}\NormalTok{)}
\end{Highlighting}
\end{Shaded}

\begin{figure}[H]

\centering{

\pandocbounded{\includegraphics[keepaspectratio]{geostat_files/figure-pdf/fig-estimadores-comparacao-1.pdf}}

}

\caption{\label{fig-estimadores-comparacao}Comparação de Robustez: O
estimador de Matheron (laranja) é sensível aos outliers, superestimando
a variância. O estimador de Cressie (azul) ignora a contaminação e
ajusta-se quase perfeitamente ao Modelo Verdadeiro (tracejado).}

\end{figure}%

Na definição teórica, o variograma é uma função contínua calculada para
vetores de distância exatos \(\mathbf{h}\). Contudo, a realidade
operacional dos levantamentos de campo impõe uma restrição fundamental:
as amostras raramente estão dispostas numa grelha regular perfeita. Em
dados reais, a probabilidade de encontrar múltiplos pares de pontos
separados por uma distância vetorial exata (por exemplo,
\(\|\mathbf{h}\| = 10.00\) metros a \(0^\circ\)) é, para todos os
efeitos práticos, nula.

Se tentássemos calcular o estimador \(\hat{\gamma}(\mathbf{h})\)
exigindo distâncias exatas, obteríamos apenas um par de pontos (ou
nenhum) para cada distância única, resultando num gráfico caótico de
ruído puro. Para viabilizar a inferência estrutural e garantir robustez
estatística, é necessário proceder à regularização dos dados, agrupando
os pares de pontos em classes de distância e direção, denominadas Lags
(ou passos).

Cada ponto no variograma experimental representa, portanto, não uma
distância única, mas a média estatística de todos os pares contidos num
intervalo de tolerância
\([\mathbf{h} - \epsilon, \mathbf{h} + \epsilon]\). A
Figura~\ref{fig-lags-concept} ilustra este processo de agrupamento: em
torno de um ponto de referência \(\mathbf{s}_i\) (a vermelho),
definem-se anéis concêntricos com uma espessura definida. Todos os
vizinhos que caem dentro de um determinado anel (a azul) são
considerados como estando aproximadamente à mesma distância,
contribuindo conjuntamente para o cálculo da semivariância média daquele
lag.

A definição correta da largura do lag e da tolerância angular constitui
um compromisso delicado entre resolução e estabilidade:

\begin{enumerate}
\def\labelenumi{\arabic{enumi}.}
\item
  \textbf{Intervalos demasiado estreitos:} Resultam em classes com
  poucos pares (\(|N(\mathbf{h})|\) baixo). Como a variância do
  estimador é inversamente proporcional ao número de pares, isto gera um
  variograma ruidoso, instável e de difícil interpretação.
\item
  \textbf{Intervalos demasiado largos:} Suavizam excessivamente a
  estrutura espacial. Ao fazer a média de pares muito distantes entre
  si, mascara-se o comportamento do variograma nas curtas distâncias, o
  que pode levar a uma estimativa incorreta do Efeito Pepita e da
  microestrutura do fenômeno.
\end{enumerate}

Andre G. Journel e Huijbregts (1976) propõe, como regra empírica
amplamente aceite, que o número de pares por lag não deve ser inferior a
30 para garantir a fiabilidade estatística da estimativa (Teorema do
Limite Central). Adicionalmente, recomenda-se que a largura do lag seja
coincidente com a distância média entre amostras vizinhas, maximizando
assim o aproveitamento da informação disponível.

\begin{Shaded}
\begin{Highlighting}[]
\NormalTok{pacman}\SpecialCharTok{::}\FunctionTok{p\_load}\NormalTok{(ggplot2,dplyr)}


\FunctionTok{set.seed}\NormalTok{(}\DecValTok{42}\NormalTok{)}
\NormalTok{n\_points }\OtherTok{\textless{}{-}} \DecValTok{60}
\NormalTok{df\_points }\OtherTok{\textless{}{-}} \FunctionTok{data.frame}\NormalTok{(}
  \AttributeTok{id =} \DecValTok{1}\SpecialCharTok{:}\NormalTok{n\_points,}
  \AttributeTok{x =} \FunctionTok{runif}\NormalTok{(n\_points, }\SpecialCharTok{{-}}\DecValTok{10}\NormalTok{, }\DecValTok{10}\NormalTok{),}
  \AttributeTok{y =} \FunctionTok{runif}\NormalTok{(n\_points, }\SpecialCharTok{{-}}\DecValTok{10}\NormalTok{, }\DecValTok{10}\NormalTok{)}
\NormalTok{)}

\NormalTok{center\_pt }\OtherTok{\textless{}{-}} \FunctionTok{data.frame}\NormalTok{(}\AttributeTok{x =} \DecValTok{0}\NormalTok{, }\AttributeTok{y =} \DecValTok{0}\NormalTok{)}
\NormalTok{df\_points}\SpecialCharTok{$}\NormalTok{dist }\OtherTok{\textless{}{-}} \FunctionTok{sqrt}\NormalTok{((df\_points}\SpecialCharTok{$}\NormalTok{x }\SpecialCharTok{{-}}\NormalTok{ center\_pt}\SpecialCharTok{$}\NormalTok{x)}\SpecialCharTok{\^{}}\DecValTok{2} \SpecialCharTok{+}\NormalTok{ (df\_points}\SpecialCharTok{$}\NormalTok{y }\SpecialCharTok{{-}}\NormalTok{ center\_pt}\SpecialCharTok{$}\NormalTok{y)}\SpecialCharTok{\^{}}\DecValTok{2}\NormalTok{)}
\NormalTok{lag\_width }\OtherTok{\textless{}{-}} \FloatTok{2.5}   \CommentTok{\# Largura do Lag}
\NormalTok{n\_lags }\OtherTok{\textless{}{-}} \DecValTok{3}        \CommentTok{\# Número de anéis para desenhar}

\NormalTok{df\_points}\SpecialCharTok{$}\NormalTok{lag\_group }\OtherTok{\textless{}{-}} \FunctionTok{cut}\NormalTok{(df\_points}\SpecialCharTok{$}\NormalTok{dist, }
                           \AttributeTok{breaks =} \FunctionTok{seq}\NormalTok{(}\DecValTok{0}\NormalTok{, }\DecValTok{15}\NormalTok{, }\AttributeTok{by =}\NormalTok{ lag\_width),}
                           \AttributeTok{labels =} \ConstantTok{FALSE}\NormalTok{)}

\NormalTok{target\_lag }\OtherTok{\textless{}{-}} \DecValTok{2}
\NormalTok{df\_points}\SpecialCharTok{$}\NormalTok{status }\OtherTok{\textless{}{-}} \FunctionTok{case\_when}\NormalTok{(}
\NormalTok{  df\_points}\SpecialCharTok{$}\NormalTok{lag\_group }\SpecialCharTok{==}\NormalTok{ target\_lag }\SpecialCharTok{\textasciitilde{}} \StringTok{"Pares no Lag Alvo"}\NormalTok{,}
  \ConstantTok{TRUE} \SpecialCharTok{\textasciitilde{}} \StringTok{"Outros Pares"}
\NormalTok{)}

\NormalTok{create\_circle }\OtherTok{\textless{}{-}} \ControlFlowTok{function}\NormalTok{(r, }\AttributeTok{center\_x=}\DecValTok{0}\NormalTok{, }\AttributeTok{center\_y=}\DecValTok{0}\NormalTok{, }\AttributeTok{npoints=}\DecValTok{100}\NormalTok{)\{}
\NormalTok{  tt }\OtherTok{\textless{}{-}} \FunctionTok{seq}\NormalTok{(}\DecValTok{0}\NormalTok{, }\DecValTok{2}\SpecialCharTok{*}\NormalTok{pi, }\AttributeTok{length.out =}\NormalTok{ npoints)}
  \FunctionTok{data.frame}\NormalTok{(}\AttributeTok{x =}\NormalTok{ center\_x }\SpecialCharTok{+}\NormalTok{ r }\SpecialCharTok{*} \FunctionTok{cos}\NormalTok{(tt), }\AttributeTok{y =}\NormalTok{ center\_y }\SpecialCharTok{+}\NormalTok{ r }\SpecialCharTok{*} \FunctionTok{sin}\NormalTok{(tt), }\AttributeTok{r =}\NormalTok{ r)}
\NormalTok{\}}

\NormalTok{circles }\OtherTok{\textless{}{-}} \FunctionTok{do.call}\NormalTok{(rbind, }\FunctionTok{lapply}\NormalTok{(}\FunctionTok{seq}\NormalTok{(lag\_width, lag\_width}\SpecialCharTok{*}\DecValTok{4}\NormalTok{, }\AttributeTok{by=}\NormalTok{lag\_width), create\_circle))}
\NormalTok{radius\_inner }\OtherTok{\textless{}{-}}\NormalTok{ (target\_lag }\SpecialCharTok{{-}} \DecValTok{1}\NormalTok{) }\SpecialCharTok{*}\NormalTok{ lag\_width}
\NormalTok{radius\_outer }\OtherTok{\textless{}{-}}\NormalTok{ target\_lag }\SpecialCharTok{*}\NormalTok{ lag\_width}

\FunctionTok{ggplot}\NormalTok{() }\SpecialCharTok{+}
  \FunctionTok{annotate}\NormalTok{(}\StringTok{"path"}\NormalTok{, }\AttributeTok{x=}\NormalTok{circles}\SpecialCharTok{$}\NormalTok{x[circles}\SpecialCharTok{$}\NormalTok{r}\SpecialCharTok{==}\NormalTok{radius\_outer], }\AttributeTok{y=}\NormalTok{circles}\SpecialCharTok{$}\NormalTok{y[circles}\SpecialCharTok{$}\NormalTok{r}\SpecialCharTok{==}\NormalTok{radius\_outer], }\AttributeTok{color=}\StringTok{"gray80"}\NormalTok{) }\SpecialCharTok{+}
  \FunctionTok{annotate}\NormalTok{(}\StringTok{"path"}\NormalTok{, }\AttributeTok{x=}\NormalTok{circles}\SpecialCharTok{$}\NormalTok{x[circles}\SpecialCharTok{$}\NormalTok{r}\SpecialCharTok{==}\NormalTok{radius\_inner], }\AttributeTok{y=}\NormalTok{circles}\SpecialCharTok{$}\NormalTok{y[circles}\SpecialCharTok{$}\NormalTok{r}\SpecialCharTok{==}\NormalTok{radius\_inner], }\AttributeTok{color=}\StringTok{"gray80"}\NormalTok{) }\SpecialCharTok{+}
  \FunctionTok{geom\_path}\NormalTok{(}\AttributeTok{data =}\NormalTok{ circles, }\FunctionTok{aes}\NormalTok{(x, y, }\AttributeTok{group =}\NormalTok{ r), }\AttributeTok{color =} \StringTok{"black"}\NormalTok{, }\AttributeTok{linetype =} \StringTok{"dashed"}\NormalTok{, }\AttributeTok{alpha =} \FloatTok{0.5}\NormalTok{) }\SpecialCharTok{+}
  \FunctionTok{geom\_point}\NormalTok{(}\AttributeTok{data =}\NormalTok{ df\_points, }\FunctionTok{aes}\NormalTok{(x, y, }\AttributeTok{color =}\NormalTok{ status, }\AttributeTok{shape =}\NormalTok{ status), }\AttributeTok{size =} \DecValTok{3}\NormalTok{) }\SpecialCharTok{+}
  \FunctionTok{geom\_point}\NormalTok{(}\AttributeTok{data =}\NormalTok{ center\_pt, }\FunctionTok{aes}\NormalTok{(x, y), }\AttributeTok{color =} \StringTok{"red"}\NormalTok{, }\AttributeTok{size =} \DecValTok{5}\NormalTok{, }\AttributeTok{shape =} \DecValTok{18}\NormalTok{) }\SpecialCharTok{+}
  \FunctionTok{annotate}\NormalTok{(}\StringTok{"text"}\NormalTok{, }\AttributeTok{x =} \FloatTok{0.5}\NormalTok{, }\AttributeTok{y =} \FloatTok{0.5}\NormalTok{, }\AttributeTok{label =} \StringTok{"s\_i"}\NormalTok{, }\AttributeTok{color =} \StringTok{"red"}\NormalTok{, }\AttributeTok{vjust =} \SpecialCharTok{{-}}\DecValTok{1}\NormalTok{, }\AttributeTok{fontface=}\StringTok{"bold"}\NormalTok{) }\SpecialCharTok{+}
  \FunctionTok{annotate}\NormalTok{(}\StringTok{"text"}\NormalTok{, }\AttributeTok{x =} \DecValTok{0}\NormalTok{, }\AttributeTok{y =} \SpecialCharTok{{-}}\NormalTok{lag\_width }\SpecialCharTok{*} \FloatTok{1.5}\NormalTok{, }\AttributeTok{label =} \StringTok{"Lag 1"}\NormalTok{) }\SpecialCharTok{+}
  \FunctionTok{annotate}\NormalTok{(}\StringTok{"text"}\NormalTok{, }\AttributeTok{x =} \DecValTok{0}\NormalTok{, }\AttributeTok{y =} \SpecialCharTok{{-}}\NormalTok{lag\_width }\SpecialCharTok{*} \FloatTok{2.5}\NormalTok{, }\AttributeTok{label =} \FunctionTok{paste}\NormalTok{(}\StringTok{"Lag"}\NormalTok{, target\_lag, }\StringTok{"}\SpecialCharTok{\textbackslash{}n}\StringTok{(Intervalo Ativo)"}\NormalTok{)) }\SpecialCharTok{+}
  \FunctionTok{annotate}\NormalTok{(}\StringTok{"segment"}\NormalTok{, }\AttributeTok{x =} \DecValTok{0}\NormalTok{, }\AttributeTok{y =}\NormalTok{ lag\_width, }\AttributeTok{xend =} \DecValTok{0}\NormalTok{, }\AttributeTok{yend =}\NormalTok{ lag\_width }\SpecialCharTok{*} \DecValTok{2}\NormalTok{, }
           \AttributeTok{arrow =} \FunctionTok{arrow}\NormalTok{(}\AttributeTok{length =} \FunctionTok{unit}\NormalTok{(}\FloatTok{0.2}\NormalTok{, }\StringTok{"cm"}\NormalTok{), }\AttributeTok{ends =} \StringTok{"both"}\NormalTok{)) }\SpecialCharTok{+}
  \FunctionTok{annotate}\NormalTok{(}\StringTok{"text"}\NormalTok{, }\AttributeTok{x =} \FloatTok{0.5}\NormalTok{, }\AttributeTok{y =}\NormalTok{ lag\_width }\SpecialCharTok{*} \FloatTok{1.5}\NormalTok{, }\AttributeTok{label =} \StringTok{"Largura}\SpecialCharTok{\textbackslash{}n}\StringTok{do Lag"}\NormalTok{, }\AttributeTok{hjust =} \DecValTok{0}\NormalTok{, }\AttributeTok{size =} \DecValTok{3}\NormalTok{) }\SpecialCharTok{+}
  \FunctionTok{scale\_color\_manual}\NormalTok{(}\AttributeTok{values =} \FunctionTok{c}\NormalTok{(}\StringTok{"Pares no Lag Alvo"} \OtherTok{=} \StringTok{"blue"}\NormalTok{, }\StringTok{"Outros Pares"} \OtherTok{=} \StringTok{"gray70"}\NormalTok{)) }\SpecialCharTok{+}
  \FunctionTok{scale\_shape\_manual}\NormalTok{(}\AttributeTok{values =} \FunctionTok{c}\NormalTok{(}\StringTok{"Pares no Lag Alvo"} \OtherTok{=} \DecValTok{19}\NormalTok{, }\StringTok{"Outros Pares"} \OtherTok{=} \DecValTok{1}\NormalTok{)) }\SpecialCharTok{+}
  \FunctionTok{coord\_fixed}\NormalTok{(}\AttributeTok{xlim =} \FunctionTok{c}\NormalTok{(}\SpecialCharTok{{-}}\DecValTok{8}\NormalTok{, }\DecValTok{8}\NormalTok{), }\AttributeTok{ylim =} \FunctionTok{c}\NormalTok{(}\SpecialCharTok{{-}}\DecValTok{8}\NormalTok{, }\DecValTok{8}\NormalTok{)) }\SpecialCharTok{+}
  \FunctionTok{theme\_void}\NormalTok{() }\SpecialCharTok{+}
  \FunctionTok{labs}\NormalTok{(}\AttributeTok{color =} \StringTok{"Classificação:"}\NormalTok{, }\AttributeTok{shape =} \StringTok{"Classificação"}\NormalTok{) }\SpecialCharTok{+}
  \FunctionTok{theme}\NormalTok{(}\AttributeTok{legend.position =} \StringTok{"bottom"}\NormalTok{, }\AttributeTok{plot.title =} \FunctionTok{element\_text}\NormalTok{(}\AttributeTok{hjust=}\FloatTok{0.5}\NormalTok{), }\AttributeTok{plot.subtitle =} \FunctionTok{element\_text}\NormalTok{(}\AttributeTok{hjust=}\FloatTok{0.5}\NormalTok{))}
\end{Highlighting}
\end{Shaded}

\begin{figure}[H]

\centering{

\pandocbounded{\includegraphics[keepaspectratio]{geostat_files/figure-pdf/fig-lags-concept-1.pdf}}

}

\caption{\label{fig-lags-concept}Esquematização do cálculo do Variograma
Experimental: Os pontos observados (y) não estão a distâncias exatas,
pelo que são agrupados em anéis concêntricos (Lags). Todos os pontos na
área azul contribuem para o cálculo da variância média daquele Lag
específico.}

\end{figure}%

\begin{Shaded}
\begin{Highlighting}[]
\ControlFlowTok{if}\NormalTok{ (}\SpecialCharTok{!}\FunctionTok{require}\NormalTok{(}\StringTok{"pacman"}\NormalTok{)) }\FunctionTok{install.packages}\NormalTok{(}\StringTok{"pacman"}\NormalTok{)}
\NormalTok{pacman}\SpecialCharTok{::}\FunctionTok{p\_load}\NormalTok{(gstat, sf, ggplot2, sp, patchwork)}

\CommentTok{\#Carregar dados exemplo (Meuse {-} metais pesados)}
\FunctionTok{data}\NormalTok{(meuse)}
\CommentTok{\# Converter para objeto sf}
\NormalTok{meuse\_sf }\OtherTok{\textless{}{-}} \FunctionTok{st\_as\_sf}\NormalTok{(meuse, }\AttributeTok{coords =} \FunctionTok{c}\NormalTok{(}\StringTok{"x"}\NormalTok{, }\StringTok{"y"}\NormalTok{), }\AttributeTok{crs =} \DecValTok{28992}\NormalTok{)}

\CommentTok{\#Variograma Cloud (Todos os pares de pontos possíveis)}
\CommentTok{\# Mostra a dispersão bruta das diferenças ao quadrado}
\NormalTok{v\_cloud }\OtherTok{\textless{}{-}} \FunctionTok{variogram}\NormalTok{(}\FunctionTok{log}\NormalTok{(zinc) }\SpecialCharTok{\textasciitilde{}} \DecValTok{1}\NormalTok{, meuse\_sf, }\AttributeTok{cloud =} \ConstantTok{TRUE}\NormalTok{)}

\NormalTok{p1 }\OtherTok{\textless{}{-}} \FunctionTok{ggplot}\NormalTok{(v\_cloud, }\FunctionTok{aes}\NormalTok{(}\AttributeTok{x =}\NormalTok{ dist, }\AttributeTok{y =}\NormalTok{ gamma)) }\SpecialCharTok{+}
  \FunctionTok{geom\_point}\NormalTok{(}\AttributeTok{alpha =} \FloatTok{0.2}\NormalTok{, }\AttributeTok{size =} \FloatTok{0.5}\NormalTok{) }\SpecialCharTok{+}
  \FunctionTok{labs}\NormalTok{(}\AttributeTok{title =} \StringTok{"Variograma Cloud (Bruto)"}\NormalTok{, }
       \AttributeTok{x =} \StringTok{"Distância"}\NormalTok{, }\AttributeTok{y =} \StringTok{"Semivariância"}\NormalTok{) }\SpecialCharTok{+}
  \FunctionTok{theme\_minimal}\NormalTok{()}

\CommentTok{\#Variograma Experimental (Experimental / Binned)}
\CommentTok{\# Agrupa a nuvem em lags (classes de distância) para obter a média}
\NormalTok{v\_exp }\OtherTok{\textless{}{-}} \FunctionTok{variogram}\NormalTok{(}\FunctionTok{log}\NormalTok{(zinc) }\SpecialCharTok{\textasciitilde{}} \DecValTok{1}\NormalTok{, meuse\_sf, }\AttributeTok{cutoff =} \DecValTok{1500}\NormalTok{, }\AttributeTok{width =} \DecValTok{100}\NormalTok{)}

\NormalTok{p2 }\OtherTok{\textless{}{-}} \FunctionTok{ggplot}\NormalTok{(v\_exp, }\FunctionTok{aes}\NormalTok{(}\AttributeTok{x =}\NormalTok{ dist, }\AttributeTok{y =}\NormalTok{ gamma)) }\SpecialCharTok{+}
  \FunctionTok{geom\_point}\NormalTok{(}\AttributeTok{size =} \DecValTok{1}\NormalTok{, }\AttributeTok{color =} \StringTok{"blue"}\NormalTok{) }\SpecialCharTok{+}
  \FunctionTok{geom\_text}\NormalTok{(}\FunctionTok{aes}\NormalTok{(}\AttributeTok{label =}\NormalTok{ np), }\AttributeTok{vjust =} \SpecialCharTok{{-}}\FloatTok{0.5}\NormalTok{, }\AttributeTok{size =} \DecValTok{3}\NormalTok{) }\SpecialCharTok{+} \CommentTok{\# Número de pares}
  \FunctionTok{labs}\NormalTok{(}\AttributeTok{title =} \StringTok{"Variograma Experimental (Médias)"}\NormalTok{, }
       \AttributeTok{subtitle =} \StringTok{"Números indicam pares por lag"}\NormalTok{,}
       \AttributeTok{x =} \StringTok{"Distância"}\NormalTok{, }\AttributeTok{y =} \StringTok{"Semivariância"}\NormalTok{) }\SpecialCharTok{+}
  \FunctionTok{theme\_minimal}\NormalTok{()}

\NormalTok{p1 }\SpecialCharTok{+}\NormalTok{ p2}
\end{Highlighting}
\end{Shaded}

\begin{figure}[H]

\centering{

\pandocbounded{\includegraphics[keepaspectratio]{geostat_files/figure-pdf/fig-variograma-experimental-1.pdf}}

}

\caption{\label{fig-variograma-experimental}Cálculo do Variograma
Experimental: Nuvem (Cloud) vs.~Experimental (Binned)}

\end{figure}%

\begin{tcolorbox}[enhanced jigsaw, left=2mm, toptitle=1mm, colback=white, colframe=quarto-callout-important-color-frame, colbacktitle=quarto-callout-important-color!10!white, opacityback=0, rightrule=.15mm, bottomtitle=1mm, arc=.35mm, title=\textcolor{quarto-callout-important-color}{\faExclamation}\hspace{0.5em}{Saiba mais}, titlerule=0mm, bottomrule=.15mm, leftrule=.75mm, coltitle=black, toprule=.15mm, breakable, opacitybacktitle=0.6]

Para compreender melhor os diferentes tipos de variograma e suas
aplicações práticas, recomenda-se a leitura do Capítulo 3 do livro
Scalon (2024).

\end{tcolorbox}

\section{Modelagem do Variograma}\label{modelagem-do-variograma}

O cálculo do variograma experimental, conforme detalhado na secção
anterior, resulta num conjunto discreto de estimativas pontuais
\(\hat{\gamma}(\mathbf{h}_k)\) para distâncias de separação específicas.
No entanto, o sistema de equações da Krigagem exige o conhecimento do
valor da semivariância para qualquer distância contínua \(\mathbf{h}\)
dentro do domínio de estudo, e não apenas para os intervalos amostrados.
Poder-se-ia intuir que uma simples interpolação linear ou spline entre
os pontos experimentais seria suficiente para resolver esta lacuna.
Contudo, tal procedimento é inválido e perigoso para a inferência.

Para garantir a existência e unicidade da solução do sistema de Krigagem
e, crucialmente, para assegurar que a variância de estimativa calculada
seja sempre não-negativa (\(\sigma_E^2 \ge 0\)), a função utilizada para
modelar a dependência espacial deve satisfazer a condição de definição
condicionalmente negativa (no caso do variograma) ou definição positiva
(no caso da covariância) (George Matheron 1971; Noel Cressie 1993).
Funções arbitrárias ou interpolações empíricas raramente satisfazem
estas desigualdades. Consequentemente, a prática geoestatística impõe a
substituição dos pontos experimentais por um modelo teórico paramétrico
\(\gamma(\mathbf{h}; \boldsymbol{\theta})\) que seja válido
(admissível). O processo de modelagem consiste, portanto, em ajustar uma
curva teórica aos dados empíricos, estimando o vetor de parâmetros
\(\boldsymbol{\theta} = (C_0, C, a)\) que melhor representa a estrutura
de continuidade do fenômeno.

Entre a vasta família de funções admissíveis (Entre a vasta família de
funções admissíveis (ver Banerjee, Carlin, e Gelfand (2003), p.~27-28),
três modelos isotrópicos destacam-se na literatura aplicada devido à sua
interpretabilidade: o modelo Esférico, o Exponencial e o Gaussiano
(Wackernagel 2003).

\textbf{Modelo Esférico}

O Modelo Esférico descreve fenômenos com uma transição clara e abrupta
entre a dependência espacial e a independência. O seu comportamento
caracteriza-se por um crescimento quase linear na origem que se curva
progressivamente até atingir o patamar exatamente na distância definida
pelo alcance \(a\). Para uma distância escalar \(h = \|\mathbf{h}\|\), é
definido por:

\begin{equation}\phantomsection\label{eq-modelo_esferico}{\gamma(h) = \begin{cases} C_0 + c \left( \frac{3h}{2a} - \frac{1}{2}\left(\frac{h}{a}\right)^3 \right) & \text{se } 0 < h \le a \\ C_0 + c & \text{se } h > a \end{cases}}\end{equation}

Em contraste, muitos fenômenos ambientais, como a dispersão de poluentes
ou propriedades do solo, exibem uma continuidade mais suave e
persistente, melhor descrita pelo Modelo Exponencial.

\textbf{Modelo Exponencial}

Diferentemente do modelo esférico, o exponencial é assintótico: ele
cresce rapidamente na origem mas nunca atinge o patamar, aproximando-se
dele indefinidamente. A sua formulação é dada por

\begin{equation}\phantomsection\label{eq-modelo_esponencial}{\gamma(h) = C_0 + C(1 - \exp(-h/a))}\end{equation}

Devido a esta natureza assintótica, o parâmetro \(a\) a
Eq.~\ref{eq-modelo_esponencial} não representa o alcance geométrico onde
a correlação se anula, mas sim um parâmetro de escala. Convenciona-se,
portanto, definir o Alcance Prático (\(a' \approx 3a\)) como a distância
na qual o variograma atinge \(95\%\) do valor do patamar.

\textbf{Modelo Gaussiano}

No extremo da continuidade encontra-se o Modelo Gaussiano, utilizado
para representar fenômenos extremamente regulares e infinitamente
diferenciáveis. A sua característica distintiva é o comportamento
parabólico na origem (\(h^2\)), indicando uma variação muito suave a
curtas distâncias. A equação define-se como

\begin{equation}\phantomsection\label{eq-modleo_gaussiano}{\gamma(h) = C_0 + c(1 - \exp(-h^2/a^2)),}\end{equation}

com um alcance prático de aproximadamente \(\sqrt{3}a\). Laslett (1994)
e Wackernagel (2003) demonstram que a utilização do modelo Gaussiano sem
um efeito pepita (\(C_0 = 0\)) pode conduzir a instabilidades numéricas
severas na inversão da matriz de krigagem (singularidade) e gerar
artefactos irrealistas na predição, devendo a sua aplicação ser sempre
acompanhada de uma componente de ruído, ainda que infinitesimal.

\begin{Shaded}
\begin{Highlighting}[]
\NormalTok{nugget }\OtherTok{\textless{}{-}} \DecValTok{0}
\NormalTok{sill\_total }\OtherTok{\textless{}{-}} \DecValTok{10}  \CommentTok{\# C0 + C}
\NormalTok{alcance\_pratico }\OtherTok{\textless{}{-}} \DecValTok{30} 

\NormalTok{h }\OtherTok{\textless{}{-}} \FunctionTok{seq}\NormalTok{(}\DecValTok{0}\NormalTok{, }\DecValTok{50}\NormalTok{, }\AttributeTok{by =} \FloatTok{0.5}\NormalTok{)}

\NormalTok{gamma\_sph }\OtherTok{\textless{}{-}} \FunctionTok{ifelse}\NormalTok{(h }\SpecialCharTok{\textless{}=}\NormalTok{ alcance\_pratico,}
\NormalTok{                    nugget }\SpecialCharTok{+}\NormalTok{ sill\_total }\SpecialCharTok{*}\NormalTok{ (}\FloatTok{1.5} \SpecialCharTok{*}\NormalTok{ (h}\SpecialCharTok{/}\NormalTok{alcance\_pratico) }\SpecialCharTok{{-}} \FloatTok{0.5} \SpecialCharTok{*}\NormalTok{ (h}\SpecialCharTok{/}\NormalTok{alcance\_pratico)}\SpecialCharTok{\^{}}\DecValTok{3}\NormalTok{),}
\NormalTok{                    sill\_total)}

\CommentTok{\#Modelo Exponencial }
\NormalTok{a\_exp }\OtherTok{\textless{}{-}}\NormalTok{ alcance\_pratico }\SpecialCharTok{/} \DecValTok{3}
\NormalTok{gamma\_exp }\OtherTok{\textless{}{-}}\NormalTok{ nugget }\SpecialCharTok{+}\NormalTok{ sill\_total }\SpecialCharTok{*}\NormalTok{ (}\DecValTok{1} \SpecialCharTok{{-}} \FunctionTok{exp}\NormalTok{(}\SpecialCharTok{{-}}\NormalTok{h}\SpecialCharTok{/}\NormalTok{a\_exp))}

\CommentTok{\#Modelo Gaussiano}
\NormalTok{a\_gau }\OtherTok{\textless{}{-}}\NormalTok{ alcance\_pratico }\SpecialCharTok{/} \FunctionTok{sqrt}\NormalTok{(}\DecValTok{3}\NormalTok{)}
\NormalTok{gamma\_gau }\OtherTok{\textless{}{-}}\NormalTok{ nugget }\SpecialCharTok{+}\NormalTok{ sill\_total }\SpecialCharTok{*}\NormalTok{ (}\DecValTok{1} \SpecialCharTok{{-}} \FunctionTok{exp}\NormalTok{(}\SpecialCharTok{{-}}\NormalTok{(h}\SpecialCharTok{\^{}}\DecValTok{2}\NormalTok{)}\SpecialCharTok{/}\NormalTok{(a\_gau}\SpecialCharTok{\^{}}\DecValTok{2}\NormalTok{)))}

\NormalTok{df\_modelos }\OtherTok{\textless{}{-}} \FunctionTok{data.frame}\NormalTok{(}\AttributeTok{Distancia =}\NormalTok{ h,}
                         \AttributeTok{Esferico =}\NormalTok{ gamma\_sph,}
                         \AttributeTok{Exponencial =}\NormalTok{ gamma\_exp,}
                         \AttributeTok{Gaussiano =}\NormalTok{ gamma\_gau)}

\NormalTok{df\_long }\OtherTok{\textless{}{-}} \FunctionTok{pivot\_longer}\NormalTok{(df\_modelos, }\AttributeTok{cols =} \SpecialCharTok{{-}}\NormalTok{Distancia, }\AttributeTok{names\_to =} \StringTok{"Modelo"}\NormalTok{, }\AttributeTok{values\_to =} \StringTok{"Gamma"}\NormalTok{)}

\FunctionTok{ggplot}\NormalTok{(df\_long, }\FunctionTok{aes}\NormalTok{(}\AttributeTok{x =}\NormalTok{ Distancia, }\AttributeTok{y =}\NormalTok{ Gamma, }\AttributeTok{color =}\NormalTok{ Modelo, }\AttributeTok{linetype =}\NormalTok{ Modelo)) }\SpecialCharTok{+}
  \FunctionTok{geom\_line}\NormalTok{(}\AttributeTok{size =} \DecValTok{1}\NormalTok{) }\SpecialCharTok{+}
  \FunctionTok{geom\_hline}\NormalTok{(}\AttributeTok{yintercept =}\NormalTok{ sill\_total, }\AttributeTok{linetype =} \StringTok{"dashed"}\NormalTok{, }\AttributeTok{color =} \StringTok{"black"}\NormalTok{) }\SpecialCharTok{+}
  \FunctionTok{annotate}\NormalTok{(}\StringTok{"text"}\NormalTok{, }\AttributeTok{x =} \DecValTok{45}\NormalTok{, }\AttributeTok{y =}\NormalTok{ sill\_total }\SpecialCharTok{+} \FloatTok{0.3}\NormalTok{, }\AttributeTok{label =} \StringTok{"Patamar (Sill)"}\NormalTok{, }\AttributeTok{color =} \StringTok{"black"}\NormalTok{) }\SpecialCharTok{+}
  
  \FunctionTok{scale\_color\_manual}\NormalTok{(}\AttributeTok{values =} \FunctionTok{c}\NormalTok{(}\StringTok{"Esferico"} \OtherTok{=} \StringTok{"black"}\NormalTok{, }\StringTok{"Exponencial"} \OtherTok{=} \StringTok{"red"}\NormalTok{, }\StringTok{"Gaussiano"} \OtherTok{=} \StringTok{"blue"}\NormalTok{)) }\SpecialCharTok{+}
  \FunctionTok{scale\_linetype\_manual}\NormalTok{(}\AttributeTok{values =} \FunctionTok{c}\NormalTok{(}\StringTok{"solid"}\NormalTok{, }\StringTok{"solid"}\NormalTok{, }\StringTok{"solid"}\NormalTok{)) }\SpecialCharTok{+}
  
  \FunctionTok{labs}\NormalTok{(}\AttributeTok{x =} \StringTok{"Distância (h)"}\NormalTok{, }\AttributeTok{y =} \FunctionTok{expression}\NormalTok{(}\FunctionTok{gamma}\NormalTok{(h))) }\SpecialCharTok{+}
  \FunctionTok{theme\_classic}\NormalTok{() }\SpecialCharTok{+}
  \FunctionTok{theme}\NormalTok{(}\AttributeTok{legend.position =} \StringTok{"bottom"}\NormalTok{)}
\end{Highlighting}
\end{Shaded}

\begin{figure}[H]

\centering{

\pandocbounded{\includegraphics[keepaspectratio]{geostat_files/figure-pdf/fig-modelos-teoricos-1.pdf}}

}

\caption{\label{fig-modelos-teoricos}Comparação dos Modelos Teóricos: O
Esférico (linear na origem) atinge o patamar abruptamente. O Exponencial
sobe rápido mas estabiliza lentamente (assintótico). O Gaussiano
(parabólico na origem) representa fenômenos muito suaves.}

\end{figure}%

Para unificar estas abordagens, Guttorp e Gneiting (2006) defendem que
modelos da família Matérn devem ser preferidos devido à sua
flexibilidade superior. Enquanto outros modelos impõem uma suavidade
fixa ao processo, a família Matérn possui um parâmetro de forma \(\nu\)
(suavidade) que permite aos dados ditar o grau de diferenciabilidade do
campo aleatório \(Y(\mathbf{s})\).

Como definido anteriormente, para processos estacionários de segunda
ordem, o semivariograma relaciona-se diretamente com a função de
covariância através de \(\gamma(h) = C(\mathbf{0}) - C(h)\).

A formulação da família Matérn distingue-se pela introdução de um
parâmetro adicional que controla a suavidade do processo estocástico,
permitindo que o modelo transite entre as formas exponencial e gaussiana
de acordo com a evidência dos dados Sahu (2022).

\[\gamma(h) = C_0 + C \left[ 1 - \frac{1}{2^{\nu-1}\Gamma(\nu)} \left(\frac{2\sqrt{\nu} \cdot h}{a}\right)^\nu K_\nu\left(\frac{2\sqrt{\nu} \cdot h}{a}\right) \right]\]

Onde,

\begin{itemize}
\item
  \(C\) (Contribuição ou Variância Estrutural) representa a parte do
  patamar explicada pela continuidade espacial. No limite
  \(h \to \infty\), o termo de correlação anula-se e o semivariograma
  estabiliza no patamar total \(C_0 + C\).
\item
  \(a\) (Alcance) define a escala de distância da dependência. Tal como
  nos modelos exponencial e gaussiano, este parâmetro dita quão rápido a
  semivariância cresce em direção ao patamar.
\item
  \(\nu\) (Parâmetro de suavidade) dita a diferenciabilidade do campo
  aleatório \(Y(\mathbf{s})\).
\item
  \(K_\nu(\cdot)\) representa a função de Bessel modificada de segunda
  espécie de ordem \(\nu\), que garante que o modelo seja válido
  (positivo definido) em espaços multidimensionais.
\end{itemize}

\begin{Shaded}
\begin{Highlighting}[]
\NormalTok{matern\_semivgm }\OtherTok{\textless{}{-}} \ControlFlowTok{function}\NormalTok{(h, }\AttributeTok{C0 =} \DecValTok{0}\NormalTok{, }\AttributeTok{C =} \DecValTok{1}\NormalTok{, }\AttributeTok{a =} \DecValTok{10}\NormalTok{, }\AttributeTok{nu =} \FloatTok{0.5}\NormalTok{) \{}
  \ControlFlowTok{if}\NormalTok{ (h }\SpecialCharTok{==} \DecValTok{0}\NormalTok{) }\FunctionTok{return}\NormalTok{(}\DecValTok{0}\NormalTok{)}
  
  \CommentTok{\#correlação Matérn}
  \CommentTok{\# arg = (2 * sqrt(nu) * h) / a}
\NormalTok{  arg }\OtherTok{\textless{}{-}}\NormalTok{ (}\DecValTok{2} \SpecialCharTok{*} \FunctionTok{sqrt}\NormalTok{(nu) }\SpecialCharTok{*}\NormalTok{ h) }\SpecialCharTok{/}\NormalTok{ a}
  
\NormalTok{  term1 }\OtherTok{\textless{}{-}} \DecValTok{1} \SpecialCharTok{/}\NormalTok{ (}\DecValTok{2}\SpecialCharTok{\^{}}\NormalTok{(nu }\SpecialCharTok{{-}} \DecValTok{1}\NormalTok{) }\SpecialCharTok{*} \FunctionTok{gamma}\NormalTok{(nu))}
\NormalTok{  term2 }\OtherTok{\textless{}{-}}\NormalTok{ (arg)}\SpecialCharTok{\^{}}\NormalTok{nu}
\NormalTok{  term3 }\OtherTok{\textless{}{-}} \FunctionTok{besselK}\NormalTok{(arg, nu)}
  
\NormalTok{  correlacao }\OtherTok{\textless{}{-}}\NormalTok{ term1 }\SpecialCharTok{*}\NormalTok{ term2 }\SpecialCharTok{*}\NormalTok{ term3}
  \CommentTok{\# gamma(h) = C0 + C * (1 {-} rho(h))}
  \FunctionTok{return}\NormalTok{(C0 }\SpecialCharTok{+}\NormalTok{ C }\SpecialCharTok{*}\NormalTok{ (}\DecValTok{1} \SpecialCharTok{{-}}\NormalTok{ correlacao))}
\NormalTok{\}}

\NormalTok{h\_seq }\OtherTok{\textless{}{-}} \FunctionTok{seq}\NormalTok{(}\DecValTok{0}\NormalTok{, }\DecValTok{50}\NormalTok{, }\AttributeTok{by =} \FloatTok{0.2}\NormalTok{)}
\NormalTok{contribuicao }\OtherTok{\textless{}{-}} \DecValTok{10}
\NormalTok{alcance\_a }\OtherTok{\textless{}{-}} \DecValTok{15} 

\NormalTok{df\_matern }\OtherTok{\textless{}{-}} \FunctionTok{data.frame}\NormalTok{()}

\ControlFlowTok{for}\NormalTok{ (nu\_val }\ControlFlowTok{in} \FunctionTok{c}\NormalTok{(}\FloatTok{0.5}\NormalTok{, }\FloatTok{1.5}\NormalTok{, }\DecValTok{10}\NormalTok{)) \{}
\NormalTok{  gamma\_vals }\OtherTok{\textless{}{-}} \FunctionTok{sapply}\NormalTok{(h\_seq, matern\_semivgm, }\AttributeTok{C0 =} \DecValTok{0}\NormalTok{, }\AttributeTok{C =}\NormalTok{ contribuicao, }\AttributeTok{a =}\NormalTok{ alcance\_a, }\AttributeTok{nu =}\NormalTok{ nu\_val)}
\NormalTok{  nome\_modelo }\OtherTok{\textless{}{-}} \FunctionTok{case\_when}\NormalTok{(}
\NormalTok{    nu\_val }\SpecialCharTok{==} \FloatTok{0.5} \SpecialCharTok{\textasciitilde{}} \StringTok{"nu = 0.5 (Exponencial)"}\NormalTok{,}
\NormalTok{    nu\_val }\SpecialCharTok{==} \FloatTok{1.5} \SpecialCharTok{\textasciitilde{}} \StringTok{"nu = 1.5 (Matern 3/2)"}\NormalTok{,}
\NormalTok{    nu\_val }\SpecialCharTok{==} \DecValTok{10}  \SpecialCharTok{\textasciitilde{}} \StringTok{"nu = 10 (Aprox. Gaussiano)"}
\NormalTok{  )}
  
\NormalTok{  df\_matern }\OtherTok{\textless{}{-}} \FunctionTok{rbind}\NormalTok{(df\_matern, }\FunctionTok{data.frame}\NormalTok{(}\AttributeTok{h =}\NormalTok{ h\_seq, }\AttributeTok{gamma =}\NormalTok{ gamma\_vals, }\AttributeTok{Modelo =}\NormalTok{ nome\_modelo))}
\NormalTok{\}}

\FunctionTok{ggplot}\NormalTok{(df\_matern, }\FunctionTok{aes}\NormalTok{(}\AttributeTok{x =}\NormalTok{ h, }\AttributeTok{y =}\NormalTok{ gamma, }\AttributeTok{color =}\NormalTok{ Modelo, }\AttributeTok{linetype =}\NormalTok{ Modelo)) }\SpecialCharTok{+}
  \FunctionTok{geom\_line}\NormalTok{(}\AttributeTok{linewidth =} \DecValTok{1}\NormalTok{) }\SpecialCharTok{+}
  \FunctionTok{geom\_hline}\NormalTok{(}\AttributeTok{yintercept =}\NormalTok{ contribuicao, }\AttributeTok{linetype =} \StringTok{"dashed"}\NormalTok{, }\AttributeTok{alpha =} \FloatTok{0.5}\NormalTok{) }\SpecialCharTok{+}
  \FunctionTok{scale\_color\_manual}\NormalTok{(}\AttributeTok{values =} \FunctionTok{c}\NormalTok{(}\StringTok{"nu = 0.5 (Exponencial)"} \OtherTok{=} \StringTok{"red"}\NormalTok{, }
                                \StringTok{"nu = 1.5 (Matern 3/2)"} \OtherTok{=} \StringTok{"\#009E73"}\NormalTok{, }
                                \StringTok{"nu = 10 (Aprox. Gaussiano)"} \OtherTok{=} \StringTok{"blue"}\NormalTok{)) }\SpecialCharTok{+}
  \FunctionTok{labs}\NormalTok{(}\AttributeTok{x =} \StringTok{"Distância de separação (h)"}\NormalTok{, }
       \AttributeTok{y =} \FunctionTok{expression}\NormalTok{(}\FunctionTok{gamma}\NormalTok{(h))) }\SpecialCharTok{+}
  \FunctionTok{theme\_classic}\NormalTok{() }\SpecialCharTok{+}
  \FunctionTok{theme}\NormalTok{(}\AttributeTok{legend.position =} \StringTok{"bottom"}\NormalTok{,}
        \AttributeTok{legend.title =} \FunctionTok{element\_blank}\NormalTok{())}
\end{Highlighting}
\end{Shaded}

\begin{figure}[H]

\centering{

\pandocbounded{\includegraphics[keepaspectratio]{geostat_files/figure-pdf/fig-matern-comparacao-1.pdf}}

}

\caption{\label{fig-matern-comparacao}A Família Matérn e a Flexibilidade
de Suavidade (nu). O parâmetro nu controla o comportamento na origem:
nu=0.5 recupera o modelo Exponencial, enquanto nu elevados aproximam-se
do Gaussiano.}

\end{figure}%

A flexibilidade desta família reside no facto de englobar os modelos
clássicos como casos particulares. Sahu (2022) destaca que quando
\(\nu = 0.5\), a função simplifica-se analiticamente para o modelo
exponencial, \(\gamma(d) = C_0 + C(1 - \exp(-h/a))\), descrevendo
processos contínuos mas rugosos (não diferenciáveis na origem). À medida
que \(\nu \to \infty\), a função converge para o modelo Gaussiano,
descrevendo processos infinitamente suaves e diferenciáveis
Figura~\ref{fig-matern-comparacao}.

Esta capacidade de transitar entre o rugoso e o suave permite que os
próprios dados informem o grau de regularidade do fenômeno, evitando
suposições arbitrárias. Além disso, a modelagem respeita a Primeira lei
da geografia de Tobler, assegurando que a semivariância cresça
monotonicamente com a distância. Para o caso específico de
\(\nu = 0.5\), o alcance prático (onde se atinge \(95\%\) do patamar)
mantém a relação clássica de \(3a\), facilitando a interpretação dos
parâmetros estimados.

\section{Diagnóstico e
Validação}\label{diagnuxf3stico-e-validauxe7uxe3o}

\subsection{Diagnóstico via Derivada na Origem (Teoria
Espectral).}\label{diagnuxf3stico-via-derivada-na-origem-teoria-espectral.}

A distinção visual entre modelos teóricos, particularmente entre as
famílias Exponencial e Gaussiana (ou Matérn com diferentes parâmetros de
suavidade), é frequentemente ambígua na presença de ruído experimental.
Gorsich e Genton (2000) propõem uma metodologia objetiva baseada na
análise da derivada do variograma na origem, \(\gamma'(0)\),
fundamentada na teoria espectral de campos aleatórios.

Pelo Teorema de Bochner, a função de covariância \(C(\mathbf{h})\) de um
campo aleatório estacionário e isotrópico em \(\mathbb{R}^d\) possui uma
representação espectral dada pela transformada de Hankel da densidade
espectral \(f(\omega)\). Para a dimensão \(d=2\), tem-se:

\[C(h) = 2\pi \int_0^{\infty} J_0(\omega h) f(\omega) \omega \, d\omega\]

Onde \(J_0(\cdot)\) é a função de Bessel de primeira espécie de ordem
zero. Sabemos que o semivariograma se relaciona com a covariância por
\(\gamma(h) = C(0) - C(h)\). Consequentemente, a derivada do
semivariograma é o simétrico da derivada da covariância:
\(\gamma'(h) = -C'(h)\).

A suavidade do processo estocástico (a existência de derivadas em média
quadrática do processo \(Y(\mathbf{s})\)) é determinada pela taxa de
decaimento da densidade espectral \(f(\omega)\) em altas frequências.
Gorsich e Genton (2000) demonstram que o comportamento de \(\gamma'(h)\)
quando \(h \to 0\) discrimina classes de diferenciabilidade.

\textbf{Análise Assintótica dos Modelos}

1.\textbf{Modelo Gaussiano}: A função de covariância é dada por
\(C(h) = \sigma^2 \exp(-h^2/a^2)\). A expansão de Taylor de segunda
ordem em torno de zero é:

\[C(h) \approx \sigma^2 \left( 1 - \frac{h^2}{a^2} + O(h^4) \right)\]

O semivariograma correspondente é
\(\gamma(h) = \sigma^2 - C(h) \approx \sigma^2 \frac{h^2}{a^2}\). A
derivada em relação a \(h\) é:
\(\gamma'(h) \approx \frac{2\sigma^2 h}{a^2}\). Tomando o limite na
origem: \(\lim_{h \to 0} \gamma'(h) = 0\). Uma derivada nula na origem
implica que o processo \(Y(\mathbf{s})\) é infinitamente diferenciável
em média quadrática, caracterizando uma estrutura espacial extremamente
suave.

\begin{enumerate}
\def\labelenumi{\arabic{enumi}.}
\setcounter{enumi}{1}
\tightlist
\item
  \textbf{Modelo Exponencial}: A função de covariância é
  \(C(h) = \sigma^2 \exp(-h/a)\). A expansão de Taylor de primeira ordem
  é:
\end{enumerate}

\[C(h) \approx \sigma^2 \left( 1 - \frac{h}{a} + O(h^2) \right)\]

O semivariograma comporta-se como
\(\gamma(h) \approx \sigma^2 \frac{h}{a}\). A derivada é:
\(\gamma'(h) \approx \frac{\sigma^2}{a}\). Tomando o
limite:\(\lim_{h \to 0} \gamma'(h) = \frac{\sigma^2}{a} > 0\). Uma
derivada positiva constante na origem indica que o processo é contínuo
em média quadrática, mas não diferenciável.

A aplicação prática deste diagnóstico requer o uso de um estimador
não-paramétrico para \(\gamma(h)\) (como expansões de Bessel ou splines
de suavização) e o cálculo numérico da sua derivada em \(h=0\).

\begin{Shaded}
\begin{Highlighting}[]
\CommentTok{\# Permite ver se as curvas de nível formam elipses}
\FunctionTok{plot}\NormalTok{(}
  \FunctionTok{variogram}\NormalTok{(}\FunctionTok{log}\NormalTok{(zinc) }\SpecialCharTok{\textasciitilde{}} \DecValTok{1}\NormalTok{, meuse\_sf, }\AttributeTok{map =} \ConstantTok{TRUE}\NormalTok{, }\AttributeTok{cutoff =} \DecValTok{1000}\NormalTok{, }\AttributeTok{width =} \DecValTok{50}\NormalTok{),}
  \AttributeTok{main =} \StringTok{"Mapa de Semivariância"}
\NormalTok{)}

\CommentTok{\#Variogramas Direcionais}
\CommentTok{\# alpha = direção (0=Norte, 45=Nordeste, 90=Leste, 135=Sudeste)}
\NormalTok{v\_dir }\OtherTok{\textless{}{-}} \FunctionTok{variogram}\NormalTok{(}\FunctionTok{log}\NormalTok{(zinc) }\SpecialCharTok{\textasciitilde{}} \DecValTok{1}\NormalTok{, meuse\_sf, }\AttributeTok{alpha =} \FunctionTok{c}\NormalTok{(}\DecValTok{0}\NormalTok{, }\DecValTok{45}\NormalTok{, }\DecValTok{90}\NormalTok{, }\DecValTok{135}\NormalTok{))}

\FunctionTok{ggplot}\NormalTok{(v\_dir, }\FunctionTok{aes}\NormalTok{(}\AttributeTok{x =}\NormalTok{ dist, }\AttributeTok{y =}\NormalTok{ gamma, }\AttributeTok{color =} \FunctionTok{factor}\NormalTok{(dir.hor))) }\SpecialCharTok{+}
  \FunctionTok{geom\_point}\NormalTok{() }\SpecialCharTok{+}
  \FunctionTok{geom\_line}\NormalTok{(}\AttributeTok{linewidth =} \FloatTok{0.5}\NormalTok{) }\SpecialCharTok{+}
  \FunctionTok{labs}\NormalTok{(}
    \AttributeTok{title =} \StringTok{"Variogramas Direcionais"}\NormalTok{,}
    \AttributeTok{color =} \StringTok{"Direção (graus):"}\NormalTok{,}
    \AttributeTok{x =} \StringTok{"Distância"}\NormalTok{,}
    \AttributeTok{y =} \StringTok{"Semivariância"}
\NormalTok{  ) }\SpecialCharTok{+}
  \FunctionTok{theme\_minimal}\NormalTok{() }\SpecialCharTok{+}
  \FunctionTok{theme}\NormalTok{(}\AttributeTok{legend.position =} \StringTok{"bottom"}\NormalTok{)}

\CommentTok{\# Nota: Se as curvas divergirem muito em altura (Patamar) = Anisotropia Zonal}
\CommentTok{\# Se divergirem no alcance (distância onde estabiliza) = Anisotropia Geométrica}
\end{Highlighting}
\end{Shaded}

\begin{figure}[H]

\centering{

\pandocbounded{\includegraphics[keepaspectratio]{geostat_files/figure-pdf/fig-anisotropia-1.pdf}}

}

\caption{\label{fig-anisotropia-1}Diagnóstico de Anisotropia: (a) mapa
de semivariância e (b) variogramas direcionais}

\end{figure}%

\begin{figure}[H]

\centering{

\pandocbounded{\includegraphics[keepaspectratio]{geostat_files/figure-pdf/fig-anisotropia-2.pdf}}

}

\caption{\label{fig-anisotropia-2}Diagnóstico de Anisotropia: (a) mapa
de semivariância e (b) variogramas direcionais}

\end{figure}%

\section{Construção de Modelos Válidos via Médias
Móveis}\label{construuxe7uxe3o-de-modelos-vuxe1lidos-via-muxe9dias-muxf3veis}

Uma limitação fundamental na modelagem geoestatística é a necessidade de
garantir que a função de variograma escolhida seja condicionalmente
negativa definida (ou que a covariância seja positiva definida). O uso
de funções arbitrárias pode levar a variâncias de krigagem negativas. Em
vez de se restringir a uma lista fixa de modelos paramétricos
pré-aprovados (como o Esférico ou Exponencial), Ver Hoef e Barry (1998)
propõem uma abordagem construtiva baseada em médias móveis (convolução)
que garante a validade matemática do modelo a priori.

Assuma-se que o processo espacial \(Y(\mathbf{s})\) é gerado pela
suavização (convolução) de um ruído branco subjacente \(W(\mathbf{u})\)
através de uma função de ponderação ou kernel \(g(\cdot)\) integrável ao
quadrado.

Seja \(W(\mathbf{u})\) um processo de ruído branco em \(\mathbb{R}^d\)
tal que \(E[dW(\mathbf{u})] = 0\),
\(\text{Var}[dW(\mathbf{u})] = d\mathbf{u}\) (ou seja,
\(\text{Cov}(dW(\mathbf{u}), dW(\mathbf{v})) = 0\) se
\(\mathbf{u} \neq \mathbf{v}\))

O processo \(Y(\mathbf{s})\) define-se como a integral estocástica:

\[Y(\mathbf{s}) = \int_{\mathbb{R}^d} g(\mathbf{u} - \mathbf{s}) \, dW(\mathbf{u})\]

O nosso objetivo é encontrar a expressão teórica do variograma
\(2\gamma(\mathbf{h})\) resultante deste processo construtivo.

\[2\gamma(\mathbf{h}) = \text{Var}[Y(\mathbf{s}+\mathbf{h}) - Y(\mathbf{s})] = E\left[ (Y(\mathbf{s}+\mathbf{h}) - Y(\mathbf{s}))^2 \right]\]

Substituímos \(Y(\cdot)\) pela sua definição integral:

\[
\begin{aligned}
Y(\mathbf{s}+\mathbf{h}) - Y(\mathbf{s}) &= \int_{\mathbb{R}^d} g(\mathbf{u} - (\mathbf{s}+\mathbf{h})) \, dW(\mathbf{u}) - \int_{\mathbb{R}^d} g(\mathbf{u} - \mathbf{s}) \, dW(\mathbf{u}) \\
&= \int_{\mathbb{R}^d} \left[ g(\mathbf{u} - \mathbf{s} - \mathbf{h}) - g(\mathbf{u} - \mathbf{s}) \right] \, dW(\mathbf{u})
\end{aligned}
\]

Uma propriedade fundamental das integrais estocásticas em relação ao
ruído branco (Movimento Browniano) é a Isometria de Ito, que estabelece
que a variância da integral estocástica é igual à integral do quadrado
da função determinística integranda:

\[
\text{Var}\left[ \int_{\mathbb{R}^d} f(\mathbf{u}) \, dW(\mathbf{u}) \right] = \int_{\mathbb{R}^d} [f(\mathbf{u})]^2 \, d\mathbf{u}
\]

\[
2\gamma(\mathbf{h}) = \int_{\mathbb{R}^d} \left[ g(\mathbf{u} - \mathbf{s} - \mathbf{h}) - g(\mathbf{u} - \mathbf{s}) \right]^2 \, d\mathbf{u}
\] Para demonstrar que o variograma depende apenas da separação
\(\mathbf{h}\) e não da localização \(\mathbf{s}\), fazemos uma mudança
de variável.

Seja \(\mathbf{x} = \mathbf{u} - \mathbf{s}\). Então,
\(d\mathbf{x} = d\mathbf{u}\) e os limites de integração
(\(\mathbb{R}^d\)) permanecem inalterados.

\(g(\mathbf{u} - \mathbf{s}) \rightarrow g(\mathbf{x})\) e
\(g(\mathbf{u} - \mathbf{s} - \mathbf{h}) \rightarrow g(\mathbf{x} - \mathbf{h})\),

\[2\gamma(\mathbf{h}) = \int_{\mathbb{R}^d} \left[ g(\mathbf{x} - \mathbf{h}) - g(\mathbf{x}) \right]^2 \, d\mathbf{x}\]

Esta equação demonstra que o variograma é a autoconvolução da diferença
do kernel. A implicação teórica mais poderosa deste resultado é a
garantia de validade:

\textbf{Teorema:} Qualquer função \(g(\cdot)\) que seja de quadrado
integrável (\(\int g(\mathbf{x})^2 d\mathbf{x} < \infty\)) gera
automaticamente um modelo de variograma matematicamente válido
(condicionalmente negativo definido). Isto elimina a necessidade de
verificar a positividade da matriz de covariância a posteriori.

\section{Anisotropia}\label{sec-anisotropia}

A hipótese de isotropia assume que a estrutura de dependência espacial
do processo estocástico \(Y(\mathbf{s})\) depende apenas da distância
euclidiana entre os pontos, \(\|\mathbf{h}\|\), e não da direção do
vetor de separação \(\mathbf{h}\). Isto implica que as isolinhas (ou
isossuperfícies) do variograma \(\gamma(\mathbf{h})\) formam círculos
(em \(\mathbb{R}^2\)) ou esferas (em \(\mathbb{R}^3\)).

Contudo, processos físicos geológicos e ambientais raramente são
isotrópicos. A sedimentação, o fluxo de águas subterrâneas ou a
dispersão eólica de poluentes criam direções preferenciais de
continuidade. Quando a variabilidade espacial muda com a direção, o
processo é denominado anisotrópico.

A modelagem da anisotropia não requer a criação de novas funções de
variograma, mas sim a aplicação de transformações lineares afins sobre o
sistema de coordenadas, mapeando o espaço anisotrópico original num
espaço isotrópico equivalente. Classificamos a anisotropia em três
categorias fundamentais: Geométrica, Zonal e Mista(Yamamoto e Landim
2013).

\subsection{Anisotropia Geométrica}\label{anisotropia-geomuxe9trica}

A anisotropia geométrica ocorre quando o alcance (\(a\)) da dependência
espacial varia com a direção, mas o patamar (\(C_0 + C\)) permanece
constante em todas as direções. As isolinhas de semivariância formam
elipses, cujos eixos principais correspondem às direções de maior e
menor continuidade.

Seja \(\gamma_{iso}(h; a)\) um modelo isotrópico com alcance \(a\). Num
processo anisotrópico geométrico em \(\mathbb{R}^2\), definimos:

\(a_{max}\): O alcance máximo (direção de maior continuidade).

\(a_{min}\): O alcance mínimo (direção de menor variabilidade).

\(\phi\): O ângulo de azimute da direção de maior continuidade.

A razão de anisotropia é definida como o escalar
\(\lambda = a_{max} / a_{min} \ge 1\) (ou o seu inverso, dependendo da
convenção de software, aqui usaremos a definição de estiramento).

O objetivo é transformar o vetor de separação original
\(\mathbf{h} = [h_x, h_y]^T\) num vetor transformado \(\mathbf{h}'\) tal
que a estrutura se torne isotrópica com alcance padronizado (geralmente
\(a_{min}\) ou 1). Esta transformação
\(\mathbf{h}' = \mathbf{A}\mathbf{h}\) compõe-se de uma rotação e um
escalonamento.

\begin{itemize}
\tightlist
\item
  Passo 1: Rotação
\end{itemize}

Alinhamos o sistema de coordenadas com os eixos principais da
anisotropia através da matriz de rotação \(\mathbf{R}(\phi)\):

\[\mathbf{R}(\phi) = \begin{bmatrix} \cos\phi & \sin\phi \\ -\sin\phi & \cos\phi \end{bmatrix}\]

\begin{itemize}
\tightlist
\item
  Passo 2: Escalonamento
\end{itemize}

Reduzimos a distância ao longo do eixo maior para que corresponda à
escala do eixo menor. A matriz de escalonamento \(\mathbf{S}\) é:

\[\mathbf{S} = \begin{bmatrix} 1/\lambda & 0 \\ 0 & 1 \end{bmatrix} \quad \text{onde } \lambda = \frac{a_{max}}{a_{min}}\]

Esta operação comprime o eixo maior, transformando a elipse de alcance
num círculo de raio \(a_{min}\).

\begin{itemize}
\tightlist
\item
  Passo 3: Variograma Anisotrópico
\end{itemize}

O vetor transformado é
\(\mathbf{h}' = \mathbf{S} \mathbf{R}(\phi) \mathbf{h}\). A matriz de
transformação completa \(\mathbf{A}\) é:

\[\mathbf{A} = \mathbf{S} \mathbf{R}(\phi) = \begin{bmatrix} \frac{\cos\phi}{\lambda} & \frac{\sin\phi}{\lambda} \\ -\sin\phi & \cos\phi \end{bmatrix}\]

O modelo de variograma anisotrópico \(\gamma_{aniso}(\mathbf{h})\) é
obtido avaliando o modelo isotrópico na norma do vetor transformado:

\[\gamma_{aniso}(\mathbf{h}) = \gamma_{iso}(\|\mathbf{A}\mathbf{h}\|; a_{min})\]

Como \(\mathbf{A}\) é uma matriz não-singular, a transformação é
bijectiva. Uma vez que \(\gamma_{iso}\) é uma função condicionalmente
negativa definida (CND) em \(\mathbb{R}^d\), a composição
\(\gamma_{iso}(\|\mathbf{A}\cdot\|)\) preserva a propriedade CND,
garantindo que o modelo anisotrópico é válido para a krigagem (Noel
Cressie 1993).

\subsection{Anisotropia Zonal}\label{anisotropia-zonal}

A anisotropia zonal ocorre quando o patamar (variância total) varia
consoante a direção. Isto é teoricamente problemático para a hipótese de
estacionariedade de segunda ordem, pois implica que a covariância na
origem \(C(\mathbf{0})\) (a variância do processo) não é única.

Na prática, isto ocorre quando a variabilidade vertical é muito superior
à horizontal, de tal forma que o variograma vertical atinge um patamar
muito mais alto do que o horizontal, ou o variograma horizontal parece
nunca atingir o patamar total do processo.

Para modelar a anisotropia zonal mantendo a validade do modelo, Andre G.
Journel e Huijbregts (1976) propõem decompor o processo
\(Y(\mathbf{s})\) na soma de processos independentes (estruturas
aninhadas), onde alguns componentes atuam apenas em subespaços do
domínio.

Seja \(\mathbf{h} = (h_x, h_y, h_z)\). O modelo é construído como:

\[\gamma(\mathbf{h}) = \gamma_{iso}(\|\mathbf{h}\|) + \gamma_{zonal}(|h_z|)\]

Onde:

\(\gamma_{iso}(\|\mathbf{h}\|)\): É uma componente isotrópica (ou
geometricamente anisotrópica) que contribui para a variabilidade em
todas as direções.

\(\gamma_{zonal}(|h_z|)\): É uma componente que depende apenas da
distância vertical \(h_z\).

\begin{itemize}
\tightlist
\item
  Na direção vertical (\(h_x=0, h_y=0\)), o vetor é
  \(\mathbf{h} = (0,0, h_z)\). O variograma total é:
\end{itemize}

\[\gamma(0,0, h_z) = \gamma_{iso}(h_z) + \gamma_{zonal}(h_z)\]

O patamar nesta direção é a soma dos patamares das duas estruturas:
\(C_{total} = C_{iso} + C_{zonal}\).

\begin{itemize}
\tightlist
\item
  Na direção horizontal (\(h_z=0\)), o vetor é
  \(\mathbf{h} = (h_x, h_y, 0)\). Como \(\gamma_{zonal}(0) = 0\) (por
  definição de variograma na origem), a equação reduz-se a:
\end{itemize}

\[\gamma(h_x, h_y, 0) = \gamma_{iso}(\sqrt{h_x^2 + h_y^2}) + 0\]

O patamar aparente na horizontal é apenas \(C_{iso}\).

A componente \(\gamma_{zonal}\) possui um alcance horizontal infinito.
Modela-se como uma estrutura cujo alcance no plano \(xy\) tende a
\(\infty\), contribuindo para a variância total apenas quando existe
separação vertical. Isto reflete a realidade de camadas sedimentares
onde a variação litológica é intensa verticalmente, mas as propriedades
persistem lateralmente por longas distâncias.

\subsection{Anisotropia Mista (ou
Combinada)}\label{anisotropia-mista-ou-combinada}

A anisotropia mista é a generalização que permite modelar sistemas
complexos onde diferentes escalas de variação possuem diferentes
direções de continuidade. Por exemplo, a microvariabilidade (curta
distância) pode ser isotrópica, enquanto a tendência regional (longa
distância) segue uma direção preferencial.

Assume-se que o processo \(Y(\mathbf{s})\) é a soma de \(K\) componentes
ortogonais independentes (escalas espaciais),
\(Y(\mathbf{s}) = \sum_{k=1}^K Y_k(\mathbf{s})\).

Pela propriedade de aditividade da variância de variáveis independentes,
o variograma total é a soma dos variogramas individuais:

\[\gamma_{total}(\mathbf{h}) = \sum_{k=1}^{K} \gamma_k(\mathbf{h})\]

Cada estrutura \(\gamma_k(\mathbf{h})\) pode ter a sua própria definição
de anisotropia geométrica, com a sua própria matriz de transformação
\(\mathbf{A}_k\). A equação geral para a anisotropia mista é (Goovaerts
(1997)):

\[\gamma(\mathbf{h}) = C_0 + \sum_{k=1}^{K} C_k \cdot \rho_k\left( \|\mathbf{A}_k \mathbf{h}\| \right)\]

Onde:

\(C_0\) é feito pepita (geralmente isotrópico, pois é ruído); \(C_k\) é
contribuição da estrutura \(k\); \(\rho_k(\cdot)\) é função de
correlação básica; \(\mathbf{A}_k\) é matriz de transformação específica
para a estrutura \(k\).

\section{Diagnóstico da
Anisotropia}\label{diagnuxf3stico-da-anisotropia}

A anisotropia em um processo espacial \(Y(\mathbf{s})\) manifesta-se
quando a estrutura de dependência espacial varia consoante a direção.
Como discutido na Seção~\ref{sec-anisotropia}, a modelagem correta da
anisotropia é crucial para garantir a precisão da predição espacial
(krigagem) e a validade estatística das inferências. O diagnóstico da
anisotropia envolve a detecção e a caracterização da dependência
direcional, tipicamente através da análise de variogramas direcionais e
mapas de variograma.

\subsection{Variogramas Direcionais}\label{variogramas-direcionais}

A forma mais comum de diagnosticar a anisotropia é calcular variogramas
experimentais \(\hat{\gamma}(\mathbf{h})\) para diferentes direções do
vetor de separação \(\mathbf{h}\). Em vez de considerar todas as
distâncias omnidirecionais, restringimos o cálculo a pares de pontos
cuja separação vetorial cai dentro de setores angulares específicos
(Andre G. Journel e Huijbregts 1976; Isaaks, Srivastava, et al. 1989).

Seja \(\theta\) o ângulo de direção
(\href{https://pt.wikipedia.org/wiki/Azimute}{azimute}) e
\(\Delta\theta\) a
\href{https://www.gdandtbasics.com/angularity/}{tolerância angular}
(meia-janela). O estimador do variograma direcional para a direção
\(\theta\) é dado por:

\[\hat{\gamma}(h, \theta) = \frac{1}{2|N(h, \theta)|} \sum_{(\mathbf{s}_i, \mathbf{s}_j) \in N(h, \theta)} (y(\mathbf{s}_i) - y(\mathbf{s}_j))^2\]

Onde \(N(h, \theta)\) é o conjunto de pares de locais
\((\mathbf{s}_i, \mathbf{s}_j)\) tal que a distância
\(\|\mathbf{s}_i - \mathbf{s}_j\| \approx h\) e o ângulo do vetor
\(\mathbf{s}_i - \mathbf{s}_j\) está em
\([\theta - \Delta\theta, \theta + \Delta\theta]\).

Tipicamente, calculam-se variogramas para quatro direções principais:
\(0^\circ\) (Norte-Sul), \(45^\circ\) (Nordeste-Sudoeste), \(90^\circ\)
(Leste-Oeste) e \(135^\circ\) (Sudeste-Noroeste) Scalon (2024). A
comparação visual destes variogramas permite identificar o tipo de
anisotropia (ver seção Seção~\ref{sec-anisotropia}):

\begin{itemize}
\item
  Anisotropia Geométrica: Se os variogramas atingem o mesmo patamar
  (\(C_0 + C\)) mas com alcances diferentes (\(a(\theta)\)), estamos
  perante uma anisotropia geométrica. O alcance varia com a direção
  segundo uma elipse.
\item
  Anisotropia Zonal: Se os variogramas estabilizam em patamares
  diferentes dependendo da direção, ou se numa direção específica o
  variograma não estabiliza (indicando uma tendência), temos anisotropia
  zonal.
\item
  Anisotropia mista se ocorre a anisotropia geometrica e zonal.
\end{itemize}

Uma ferramenta visual poderosa para detetar anisotropia é o mapa de
variograma ou superfície de variograma. Em vez de traçar curvas
\(\gamma(h)\) para direções discretas, representamos
\(\hat{\gamma}(\mathbf{h})\) como uma superfície em função das
coordenadas do vetor de separação \(\mathbf{h} = (h_x, h_y)\) (Goovaerts
1997).

O mapa é construído calculando a semivariância média para células de uma
grelha no espaço dos vetores de separação. O centro do mapa corresponde
a \(\mathbf{h} = (0,0)\) (semivariância zero). As cores ou curvas de
nível representam a magnitude de \(\gamma(\mathbf{h})\).

\textbf{Isotropia:} é o contrario da anisotropia.Aqui, as curvas de
nível formam círculos concêntricos em redor da origem.

\begin{itemize}
\item
  Anisotropia Geométrica: As curvas de nível formam elipses. O eixo
  maior da elipse no mapa de variograma corresponde à direção de menor
  variabilidade (maior continuidade), que é a direção do alcance máximo
  (\(a_{max}\)). O eixo menor corresponde à direção de maior
  variabilidade (alcance mínimo \(a_{min}\)).
\item
  Anisotropia Zonal: As curvas de nível não fecham ou mostram
  comportamentos muito distintos em direções ortogonais, indicando
  diferenças nos patamares.
\end{itemize}

Para anisotropia geométrica, pode-se ajustar modelos teóricos aos
variogramas direcionais experimentais e plotar os alcances estimados
\(a(\theta)\) num diagrama polar
(\href{https://en.wikipedia.org/wiki/Wind_rose}{Rose Diagram}). A forma
resultante deve aproximar-se de uma elipse descrita pela equação polar:

\[a(\theta) = \frac{a_{max} a_{min}}{\sqrt{a_{min}^2 \cos^2(\theta - \phi) + a_{max}^2 \sin^2(\theta - \phi)}}\]

Onde \(\phi\) é o ângulo da direção de maior continuidade (eixo maior).
Este diagnóstico permite estimar os parâmetros da transformação de
coordenadas (rotação \(\phi\) e razão de anisotropia
\(\lambda = a_{max}/a_{min}\)) necessários para corrigir a anisotropia e
aplicar a krigagem num espaço isotrópico equivalente (Chiles e Delfiner
2012).

\section{Ajuste de Modelos de
semivariograma}\label{sec-ajuste-diagnostico}

A inferência da estrutura de dependência espacial é uma etapa crítica na
geoestatística. Dado um conjunto de dados observados
\(\mathbf{y} = (y(\mathbf{s}_1), \dots, y(\mathbf{s}_n))^\top\), em
locais \(\mathbf{s}_1, \dots, \mathbf{s}_n\), o objetivo é estimar o
vetor de parâmetros \(\boldsymbol{\theta}\) (efeito pepita, patamar,
alcance, suavidade) de um modelo de variograma teórico
\(2\gamma(\mathbf{h}; \boldsymbol{\theta})\) que descreva adequadamente
o processo estocástico subjacente.

\subsection{Mínimos Quadrados
Ponderados}\label{muxednimos-quadrados-ponderados}

O método de mínimos quadrados ordinários é inadequada para o ajuste de
variogramas devido à heterocedasticidade intrínseca dos estimadores.
Noel Cressie (1985) formalizou a dedução da variância assintótica do
estimador de Matheron, justificando a necessidade de pesos.

Assumindo que o campo aleatório \(Y(\mathbf{s})\) é um processo
Gaussiano estacionário. Definamos a variável aleatória da diferença
entre dois pontos separados pelo vetor \(\mathbf{h}\) como
\(D(\mathbf{h}) = Y(\mathbf{s} + \mathbf{h}) - Y(\mathbf{s})\), pela
hipótese estacionaridade intrínseca, esta diferença tem média zero e
variância definida pelo variograma teórico:
\(D(\mathbf{h}) \sim \mathcal{N}(0, 2\gamma(\mathbf{h}))\). O estimador
de variograma baseia-se no quadrado desta diferença. Vamos padronizar
\(D(\mathbf{h})\) dividindo pelo desvio padrão
\(\sqrt{2\gamma(\mathbf{h})}\) para obter uma normal padrão
\(Y | X \sim \mathcal{N}(0,1)\):

\[\frac{D(\mathbf{h})}{\sqrt{2\gamma(\mathbf{h})}} \sim \mathcal{N}(0, 1)\]

Elevando ambos os lados ao quadrado, obtemos uma variável que segue uma
distribuição Qui-quadrado com 1 grau de liberdade (\(\chi^2_1\)):

\[\left( \frac{D(\mathbf{h})}{\sqrt{2\gamma(\mathbf{h})}} \right)^2 \sim \chi^2_1 \implies [D(\mathbf{h})]^2 \sim 2\gamma(\mathbf{h}) \cdot \chi^2_1\]

Sabemos que a variância de uma variável \(\chi^2_1\) é exatamente 2.
Usando a propriedade de variância
\(\text{Var}(kX) = k^2 \text{Var}(X)\), a variância da diferença
quadrática para um único par de pontos é:

\[
\begin{aligned}
\text{Var}\left[ (Y(\mathbf{s}+\mathbf{h}) - Y(\mathbf{s}))^2 \right] &= \text{Var}\left[ 2\gamma(\mathbf{h}) \cdot \chi^2_1 \right] \\
&= [2\gamma(\mathbf{h})]^2 \cdot \text{Var}(\chi^2_1) \\
&= 4[\gamma(\mathbf{h})]^2 \cdot 2 \\
&= 8[\gamma(\mathbf{h})]^2
\end{aligned}
\]

O estimador do semivariograma \(\hat{\gamma}(\mathbf{h})\) para um lag
\(\mathbf{h}\) é a média de \(N(\mathbf{h})\) diferenças quadráticas,
dividida por 2. Assumindo independência aproximada entre os pares
(necessário para a derivação dos pesos Noel Cressie (1985)):

\[
\begin{aligned}
\text{Var}[\hat{\gamma}(\mathbf{h})] &= \text{Var}\left[ \frac{1}{2 N(\mathbf{h})} \sum_{i=1}^{N(\mathbf{h})} (Y(\mathbf{s}_i) - Y(\mathbf{s}*j))^2 \right] \\
&= \frac{1}{4 [N(\mathbf{h})]^2} \sum_{i=1}^{N(\mathbf{h})} \text{Var}\left[ (Y(\mathbf{s}_i) - Y(\mathbf{s}_j))^2 \right] \quad \text{(Soma de variâncias)} \\
&= \frac{1}{4 [N(\mathbf{h})]^2} \cdot N(\mathbf{h}) \cdot 8[\gamma(\mathbf{h})]^2 \quad \text{(Substituindo o resultado anterior)} \\
&= \frac{2[\gamma(\mathbf{h})]^2}{N(\mathbf{h})}
\end{aligned}
\]

Pelo princípio de Aitken, os pesos ótimos são o inverso da variância
(\(w_k = 1/\text{Var}_k\)) (McBratney e Webster 1986; Noel Cressie
1985). Ignorando a constante 2 (que não afeta a minimização) temos,
\(w_k = \frac{N(\mathbf{h}_k)}{[\gamma(\mathbf{h}_k \boldsymbol{\theta})]^2}\).
Substituindo os pesos na soma dos erros quadráticos, obtemos a função a
ser minimizada para encontrar \(\boldsymbol{\theta}\):

\[
\begin{aligned}
\hat{\boldsymbol{\theta}}_{WLS} &= \arg \min_{\boldsymbol{\theta}} \sum_{k=1}^{K} w(\mathbf{h}_k) \left[ \hat{\gamma}(\mathbf{h}_k) - \gamma(\mathbf{h}_k; \boldsymbol{\theta}) \right]^2, \quad   w(\mathbf{h}_k) = \frac{|N(\mathbf{h}_k)|}{[\gamma(\mathbf{h}_k; \boldsymbol{\theta})]^2}
\end{aligned}
\]

onde \(K\) é número total de lags (classes de distância) considerados;
\(N(\mathbf{h}_k)\) é número de pares de pontos no lag \(k\).
\(\hat{\gamma}(\mathbf{h}_k)\) é o valor do semivariograma experimental
(observado); \(\gamma(\mathbf{h}_k; \boldsymbol{\theta})\) é o valor do
modelo teórico (predito);
\([\gamma(\mathbf{h}_k; \boldsymbol{\theta})]^{-2}\) (implícito no
denominador) penaliza fortemente erros em distâncias curtas (onde
\(\gamma\) é pequeno), garantindo que o modelo se ajuste bem na origem,
o que é crucial para a Krigagem.

\section{Mínimos Quadrados Generalizados Explícitos
(GLSE)}\label{muxednimos-quadrados-generalizados-expluxedcitos-glse}

O método de mínimos quadrados ponderados (WLS) pressupõe que as
estimativas do variograma em diferentes lags são estatisticamente
independentes
(\(\text{Cov}(\hat{\gamma}(\mathbf{h}_i), \hat{\gamma}(\mathbf{h}_j)) = 0\)
para \(i \neq j\)). Na realidade, como os mesmos dados espaciais são
reutilizados para calcular múltiplos lags, existe uma correlação
significativa entre eles. Genton (1998) formalizou o problema como um
ajuste de Mínimos Quadrados Generalizados, deduzindo a matriz de
covariância completa \(\boldsymbol{\Sigma}_{\hat{\gamma}}\) dos
estimadores.

Para deduzir a covariância entre dois estimadores do variograma,
reescrevemos primeiro o estimador de Matheron em notação matricial. Seja
\(\mathbf{y}\) o vetor de observações e \(\mathbf{A}(\mathbf{h})\) a
matriz de incidência espacial para o lag \(\mathbf{h}\) (uma matriz
esparsa que seleciona os pares de pontos separados por \(\mathbf{h}\)).
O estimador pode ser expresso como uma forma quadrática:

\[2\hat{\gamma}(\mathbf{h}) = \frac{1}{|N(\mathbf{h})|} \mathbf{y}^\top \mathbf{A}(\mathbf{h}) \mathbf{y}\]

A matriz \(\mathbf{A}(\mathbf{h})\) é definida tal que
\(\mathbf{y}^\top \mathbf{A}(\mathbf{h}) \mathbf{y} = \sum_{(\mathbf{s}_i, \mathbf{s}_j) \in N(\mathbf{h})} (y(\mathbf{s}_i) - y(\mathbf{s}_j))^2\).

Para calcular a covariância entre as estimativas em dois lags distintos,
\(\mathbf{h}_u\) e \(\mathbf{h}_v\), recorremos a um teorema fundamental
para formas quadráticas de vetores Gaussianos. Se
\(\mathbf{y} \sim \mathcal{N}(\boldsymbol{\mu}, \boldsymbol{\Sigma}_y)\),
então a covariância entre duas formas quadráticas
\(\mathbf{y}^\top \mathbf{A} \mathbf{y}\) e
\(\mathbf{y}^\top \mathbf{B} \mathbf{y}\) é dada por:

\[\text{Cov}(\mathbf{y}^\top \mathbf{A} \mathbf{y}, \mathbf{y}^\top \mathbf{B} \mathbf{y}) = 2 \text{tr}(\mathbf{A} \boldsymbol{\Sigma}_y \mathbf{B} \boldsymbol{\Sigma}_y) + 4\boldsymbol{\mu}^\top \mathbf{A} \boldsymbol{\Sigma}_y \mathbf{B} \boldsymbol{\mu}\]

Assumindo um processo de média zero (ou trabalhando com resíduos), o
termo da média anula-se. Aplicando este teorema às matrizes de
incidência espacial \(\mathbf{A}(\mathbf{h}_u)\) e
\(\mathbf{A}(\mathbf{h}_v)\), obtemos a covariância entre os estimadores
do variograma (elemento \(uv\) da matriz
\(\boldsymbol{\Sigma}_{\hat{\gamma}}\)):

\[\begin{aligned}
[\boldsymbol{\Sigma}_{\hat{\gamma}}]_{uv} &= \text{Cov}(2\hat{\gamma}(\mathbf{h}_u), 2\hat{\gamma}(\mathbf{h}_v)) \\
&= \text{Cov}\left( \frac{1}{|N(\mathbf{h}_u)|} \mathbf{y}^\top \mathbf{A}(\mathbf{h}_u) \mathbf{y}, \frac{1}{|N(\mathbf{h}_v)|} \mathbf{y}^\top \mathbf{A}(\mathbf{h}_v) \mathbf{y} \right) \\
&= \frac{1}{|N(\mathbf{h}_u)| |N(\mathbf{h}_v)|} \text{Cov}(\mathbf{y}^\top \mathbf{A}(\mathbf{h}_u) \mathbf{y}, \mathbf{y}^\top \mathbf{A}(\mathbf{h}_v) \mathbf{y}) \\
&= \frac{2}{|N(\mathbf{h}_u)| |N(\mathbf{h}_v)|} \text{tr}\left( \mathbf{A}(\mathbf{h}_u) \boldsymbol{\Sigma}_y(\boldsymbol{\theta}) \mathbf{A}(\mathbf{h}_v) \boldsymbol{\Sigma}_y(\boldsymbol{\theta}) \right)
\end{aligned}\]

A matriz \(\boldsymbol{\Sigma}_{\hat{\gamma}}\) captura explicitamente a
interdependência estatística entre os lags, dependendo tanto da
geometria da amostragem (via matrizes \(\mathbf{A}\)) quanto da
estrutura de covariância real dos dados
(\(\boldsymbol{\Sigma}_y(\boldsymbol{\theta})\)).

A função objetivo a ser minimizada no método GLSE, que leva em conta
esta estrutura de correlação completa, é a distância de Mahalanobis
entre o vetor de estimativas empíricas \(\hat{\boldsymbol{\gamma}}\) e o
vetor do modelo teórico \(\boldsymbol{\gamma}(\boldsymbol{\theta})\):

\[\begin{aligned}
\hat{\boldsymbol{\theta}}_{GLSE} &= \arg \min_{\boldsymbol{\theta}} (\hat{\boldsymbol{\gamma}} - \boldsymbol{\gamma}(\boldsymbol{\theta}))^\top [\boldsymbol{\Sigma}_{\hat{\gamma}}(\boldsymbol{\theta})]^{-1} (\hat{\boldsymbol{\gamma}} - \boldsymbol{\gamma}(\boldsymbol{\theta}))
\end{aligned}\]

onde \(\hat{\boldsymbol{\gamma}}\) é o vetor contendo as estimativas
\(\hat{\gamma}(\mathbf{h}_k)\) para todos os \(K\) lags;
\(\boldsymbol{\gamma}(\boldsymbol{\theta})\) é o vetor correspondente
dos valores teóricos;
\([\boldsymbol{\Sigma}_{\hat{\gamma}}(\boldsymbol{\theta})]^{-1}\) é a
inversa da matriz de covariância dos estimadores, que atua como uma
matriz de pesos generalizada, penalizando não apenas a variância
(elementos diagonais, como no WLS), mas também a redundância de
informação entre lags correlacionados (elementos fora da diagonal). Este
método é iterativo, pois a matriz de pesos depende dos próprios
parâmetros \(\boldsymbol{\theta}\) que estamos a estimar.

\subsection{Métodos de Máxima Verossimilhança (ML) e Máxima
Verossimilhança Restrita
(REML)}\label{muxe9todos-de-muxe1xima-verossimilhanuxe7a-ml-e-muxe1xima-verossimilhanuxe7a-restrita-reml}

Enquanto o método dos Mínimos Quadrados Ponderados (WLS) minimiza a
distância entre o variograma experimental e o modelo teórico, os métodos
baseados em verossimilhança operam diretamente sobre o vetor de dados
observados \(\mathbf{y(s)}\), sem a necessidade de calcular estatísticas
intermédias (como a semivariância experimental). Lark (2000) e Peter J.
Diggle, Tawn, e Moyeed (1998) argumentam que esta abordagem é
teoricamente mais eficiente, especialmente quando os dados são escassos
ou a amostragem é irregular, pois utiliza toda a informação contida na
distribuição conjunta dos dados.

Assuma-se que o campo aleatório \(Y(\mathbf{s})\) segue um modelo linear
misto, composto por uma tendência determinística
(\(\mathbf{X}\boldsymbol{\beta}\) - média ) e um componente estocástico
espacialmente correlacionado (\(\boldsymbol{\eta}\)). Seja
\(\mathbf{y} = (y(\mathbf{s}_1), \dots, y(\mathbf{s}_n))^\top\) o vetor
de observações:

\[\mathbf{y} = \mathbf{X}\boldsymbol{\beta} + \boldsymbol{\eta}\]

onde \(\mathbf{X}\) é a matriz de desenho (\(n \times p\)) contendo as
covariáveis (coordenadas, elevação, etc.); \(\boldsymbol{\beta}\) é o
vetor (\(p \times 1\)) de parâmetros de tendência desconhecidos (efeitos
fixos) e, \(\boldsymbol{\eta}\) é o vetor de resíduos aleatórios,
assumido seguir uma distribuição normal multivariada com média zero e
matriz de covariância \(\mathbf{V}(\boldsymbol{\theta})\).

A matriz de covariância \(\mathbf{V}(\boldsymbol{\theta})\) é
parametrizada pelo vetor \(\boldsymbol{\theta}\) (alcance, patamar,
efeito pepita) que desejamos estimar. O elemento \((i,j)\) desta matriz
é dado por:

\[[\mathbf{V}(\boldsymbol{\theta})]_{ij} = \text{Cov}(Y(\mathbf{s}_i), Y(\mathbf{s}_j)) = C(\mathbf{s}_i - \mathbf{s}_j; \boldsymbol{\theta})\]

A função de densidade de probabilidade conjunta para o vetor
\(\mathbf{y(s)}\), dado os parâmetros \(\boldsymbol{\beta}\) e
\(\boldsymbol{\theta}\), é:

\[f(\mathbf{y} | \boldsymbol{\beta}, \boldsymbol{\theta}) = (2\pi)^{-n/2} |\mathbf{V}(\boldsymbol{\theta})|^{-1/2} \exp\left( -\frac{1}{2} (\mathbf{y} - \mathbf{X}\boldsymbol{\beta})^\top \mathbf{V}(\boldsymbol{\theta})^{-1} (\mathbf{y} - \mathbf{X}\boldsymbol{\beta}) \right)\]

A função de log-verossimilhança (\(L_{ML}\)) é o logaritmo natural desta
densidade. Para estimar \(\boldsymbol{\theta}\), primeiro perfilamos a
verossimilhança em relação a \(\boldsymbol{\beta}\). O estimador de
Máxima Verossimilhança para \(\boldsymbol{\beta}\), fixado
\(\boldsymbol{\theta}\), é o estimador de Mínimos Quadrados
Generalizados (GLS):

\[\hat{\boldsymbol{\beta}}_{GLS} = (\mathbf{X}^\top \mathbf{V}(\boldsymbol{\theta})^{-1} \mathbf{X})^{-1} \mathbf{X}^\top \mathbf{V}(\boldsymbol{\theta})^{-1} \mathbf{y}\]

Substituindo \(\hat{\boldsymbol{\beta}}_{GLS}\) na equação da densidade,
obtemos a verossimilhança perfilada a ser maximizada em relação a
\(\boldsymbol{\theta}\):

\[\begin{aligned}
L_{ML}(\boldsymbol{\theta}) &= -\frac{n}{2}\ln(2\pi) - \frac{1}{2}\ln|\mathbf{V}(\boldsymbol{\theta})| - \frac{1}{2}(\mathbf{y} - \mathbf{X}\hat{\boldsymbol{\beta}}_{GLS})^\top \mathbf{V}(\boldsymbol{\theta})^{-1} (\mathbf{y} - \mathbf{X}\hat{\boldsymbol{\beta}}_{GLS})
\end{aligned}\]

Noel Cressie (1993) e Marchant e Lark (2007) destacam que existe uma
tendência nos dados (\(\mu(\mathbf{s}) \neq \text{constante}\)), o
estimador de Máxima Verossimilhança (ML) subestima a variância e o
variograma, pois assume que os parâmetros da tendência
(\(\boldsymbol{\beta}\)) são conhecidos, ignorando os graus de liberdade
perdidos na sua estimação.

Para corrigir este viés, utiliza-se a Máxima Verossimilhança Restrita
(REML).

Seja \(\mathbf{K}\) uma matriz de contrastes de dimensão
\(n \times (n-p)\) tal que suas colunas geram o espaço ortogonal ao
espaço das colunas de \(\mathbf{X}\).

\[\mathbf{K}^\top \mathbf{X} = \mathbf{0} \quad \text{e} \quad \text{posto}(\mathbf{K}) = n-p\]

Definimos o vetor de contrastes de erro transformados como
\(\mathbf{w} = \mathbf{K}^\top \mathbf{y}\). A distribuição de
\(\mathbf{w}\) depende apenas de \(\boldsymbol{\theta}\) e não de
\(\boldsymbol{\beta}\):

\[
\begin{aligned}
E[\mathbf{w}] &= E[\mathbf{K}^\top (\mathbf{X}\boldsymbol{\beta} + \boldsymbol{\eta})] = \mathbf{K}^\top \mathbf{X} \boldsymbol{\beta} + \mathbf{K}^\top E[\boldsymbol{\eta}] = \mathbf{0} \cdot \boldsymbol{\beta} + \mathbf{0} = \mathbf{0} \\
\text{Var}[\mathbf{w}] &= \mathbf{K}^\top \text{Var}[\mathbf{y}] \mathbf{K} = \mathbf{K}^\top \mathbf{V}(\boldsymbol{\theta}) \mathbf{K}
\end{aligned}
\]

A log-verossimilhança baseada nestes contrastes (Verossimilhança
Restrita) é:

\[L_R(\boldsymbol{\theta}) = -\frac{1}{2} \ln|\mathbf{K}^\top \mathbf{V}(\boldsymbol{\theta}) \mathbf{K}| - \frac{1}{2} \mathbf{w}^\top (\mathbf{K}^\top \mathbf{V}(\boldsymbol{\theta}) \mathbf{K})^{-1} \mathbf{w} + \text{constante}\]

A implementação computacional direta desta fórmula é ineficiente devido
à necessidade de construir a matriz \(\mathbf{K}\). Contudo, Harville
(1977) provou duas identidades algébricas fundamentais que relacionam os
componentes da verossimilhança restrita com as matrizes originais:

\begin{enumerate}
\def\labelenumi{\arabic{enumi}.}
\tightlist
\item
  Identidade do Determinante:
  \(\ln|\mathbf{K}^\top \mathbf{V} \mathbf{K}| = \ln|\mathbf{V}| + \ln|\mathbf{X}^\top \mathbf{V}^{-1} \mathbf{X}| - \ln|\mathbf{X}^\top \mathbf{X}| - 2\ln|\mathbf{K}|\)
\end{enumerate}

Ignorando os termos que não dependem de \(\boldsymbol{\theta}\)
(\(\mathbf{X}\) e \(\mathbf{K}\) são fixos), o termo relevante é
\(\ln|\mathbf{V}| + \ln|\mathbf{X}^\top \mathbf{V}^{-1} \mathbf{X}|\).

\begin{enumerate}
\def\labelenumi{\arabic{enumi}.}
\setcounter{enumi}{1}
\tightlist
\item
  Identidade da Forma Quadrática:
  \(\mathbf{w}^\top (\mathbf{K}^\top \mathbf{V} \mathbf{K})^{-1} \mathbf{w} = (\mathbf{y} - \mathbf{X}\hat{\boldsymbol{\beta}}_{GLS})^\top \mathbf{V}^{-1} (\mathbf{y} - \mathbf{X}\hat{\boldsymbol{\beta}}_{GLS})\)
\end{enumerate}

Isto demonstra que a forma quadrática nos contrastes é equivalente à
soma ponderada dos quadrados dos resíduos GLS no espaço original.

Substituindo estas identidades na equação de \(L_R\), obtemos a função
objetivo do REML apresentada por Marchant e Lark (2007):

\begin{equation}\phantomsection\label{eq-REML}{
\begin{aligned}
L_{REML}(\boldsymbol{\theta}) \propto -\frac{1}{2} \Bigg( & \underbrace{\ln|\mathbf{V}(\boldsymbol{\theta})|}_{\text{Ajuste da Covariância}} + \underbrace{(\mathbf{y} - \mathbf{X}\hat{\boldsymbol{\beta}}_{GLS})^\top \mathbf{V}(\boldsymbol{\theta})^{-1} (\mathbf{y} - \mathbf{X}\hat{\boldsymbol{\beta}}_{GLS})}_{\text{Ajuste aos Resíduos}} + \underbrace{\ln|\mathbf{X}^\top \mathbf{V}(\boldsymbol{\theta})^{-1} \mathbf{X}|}_{\text{Penalidade de Complexidade}} \Bigg)
\end{aligned}
}\end{equation}

O termo adicional \(\ln|\mathbf{X}^\top \mathbf{V}^{-1} \mathbf{X}|\)
atua como uma penalidade. Quanto maior a incerteza na estimação da
tendência (refletida na variância do estimador
\(\hat{\boldsymbol{\beta}}\), que é proporcional a
\((\mathbf{X}^\top \mathbf{V}^{-1} \mathbf{X})^{-1}\)), maior será este
termo logarítmico (pois estamos a tomar o log da inversa da variância,
ou seja, da precisão/informação). Esta penalidade corrige a subestimação
da variância inerente ao método ML, fornecendo estimativas não-viesadas
para o variograma.

\subsection{REML Robusto}\label{reml-robusto}

O estimador REML padrão assume que os dados seguem uma distribuição
Gaussiana. Consequentemente, a função de log-verossimilhança inclui um
termo quadrático (a distância de Mahalanobis dos resíduos) que penaliza
desvios em relação à média. Na presença de outliers (valores
provenientes de um processo de contaminação com caudas pesadas), este
termo quadrático cresce rapidamente, dominando a função de
verossimilhança e forçando o modelo a inflacionar a variância (efeito
pepita ou patamar) para acomodar os dados anómalos.

Marchant e Lark (2007), baseando-se no trabalho de Richardson e Welsh
(1995), propõem o REML Robusto, que substitui a norma \(L_2\)
(quadrática) por uma função de perda robusta (Huber) aplicada aos
resíduos decorrelacionados.

Em dados geoestatísticos, os resíduos não são independentes; são
correlacionados pela estrutura espacial
\(\mathbf{V}(\boldsymbol{\theta})\). Para aplicar uma função de robustez
(que geralmente assume independência), é necessário primeiro
decorrelacionar os resíduos. Seja o modelo linear misto
\(\mathbf{y} = \mathbf{X}\boldsymbol{\beta} + \boldsymbol{\eta}\), com
\(\boldsymbol{\eta} \sim N(\mathbf{0}, \mathbf{V}(\boldsymbol{\theta}))\).
Definimos a raiz quadrada inversa da matriz de covariância,
\(\mathbf{V}^{-1/2}\), tal que
\(\mathbf{V}^{-1/2} \mathbf{V} (\mathbf{V}^{-1/2})^T = \mathbf{I}\).

O vetor de resíduos padronizados e decorrelacionados
\(\boldsymbol{\epsilon}^*\) é dado por:

\[\boldsymbol{\epsilon}^*(\boldsymbol{\beta}, \boldsymbol{\theta}) = \mathbf{V}(\boldsymbol{\theta})^{-1/2} (\mathbf{y} - \mathbf{X}\boldsymbol{\beta})\]

Desta forma, sob a hipótese nula (sem contaminação),
\(\boldsymbol{\epsilon}^* \sim N(\mathbf{0}, \mathbf{I})\).

No REML padrão, \(\boldsymbol{\beta}\) é estimado por GLS
(\(\hat{\boldsymbol{\beta}}_{GLS}\)), que minimiza a soma dos quadrados
de \(\boldsymbol{\epsilon}^*\). No REML Robusto,
\(\hat{\boldsymbol{\beta}}_{R}\) é obtido minimizando uma função de
perda robusta \(\rho_c\) sobre estes resíduos transformados.

Dado um \(\boldsymbol{\theta}\) fixo, \(\hat{\boldsymbol{\beta}}_{R}\) é
o vetor que minimiza:

\[Q_{\beta}(\boldsymbol{\beta}) = \sum_{i=1}^{n} \rho_c \left( [\mathbf{V}^{-1/2} (\mathbf{y} - \mathbf{X}\boldsymbol{\beta})]_i \right)\]

Onde \(\rho_c(\cdot)\) é a função de Huber (definida abaixo). Esta etapa
garante que a estimativa da tendência (média local) não seja ``puxada''
por valores extremos, fornecendo uma base estável para a estimativa da
variância.

A função objetivo a ser minimizada para estimar \(\boldsymbol{\theta}\)
substitui o termo quadrático da verossimilhança restrita
((\textbf{Robust?}) Estimation of a Location Parameter) pela soma dos
escores de Huber dos resíduos decorrelacionados, calculados com base no
\(\hat{\boldsymbol{\beta}}_{R}\).

A função objetivo negativa de log-verossimilhança robusta é definida
como:

\[L_{Rob}(\boldsymbol{\theta}) = \underbrace{\frac{1}{2}\ln|\mathbf{V}(\boldsymbol{\theta})| + \frac{1}{2}\ln|\mathbf{X}^T \mathbf{V}(\boldsymbol{\theta})^{-1} \mathbf{X}|}_{\text{Penalidades de Complexidade e Viés (Inalteradas)}} + \underbrace{\sum_{i=1}^{n} \rho_c \left( \left[ \mathbf{V}(\boldsymbol{\theta})^{-1/2} (\mathbf{y} - \mathbf{X}\hat{\boldsymbol{\beta}}_{R}) \right]_i \right)}_{\text{Termo de Ajuste Robusto aos Resíduos}}\]

onde \(\mathbf{V}(\boldsymbol{\theta})^{-1/2}\) garante que a métrica de
distância considere a correlação espacial. Um outlier espacial não é
apenas um valor alto, mas um valor que difere do que seria esperado dada
a sua vizinhança. A decorrelação expõe estes valores.
\(\hat{\boldsymbol{\beta}}_{R}\) é o estimador robusto da tendência. Se
usássemos o GLS padrão aqui, o viés na média propagar-se-ia para a
variância. \(\rho_c(u)\) (Função de Huber) é uma função híbrida que
limita a influência de resíduos grandes. É definida por:

\[
\rho_c(u) = \begin{cases}
\frac{1}{2}u^2 & \text{se } |u| \le c \\
c|u| - \frac{1}{2}c^2 & \text{se } |u| > c
\end{cases}
\]

\(|u| \le c\), para resíduos pequenos (dados normais), o método
comporta-se como o REML padrão (máxima eficiência estatística sob
normalidade). \(|u| > c\), para resíduos grandes (outliers), a
penalidade cresce linearmente em vez de quadraticamente. Isso significa
que a derivada da função de perda (a função de influência \(\psi(u)\))
torna-se constante, limitando o peso que uma única observação anômala
pode exercer sobre a estimativa dos parâmetros \(\boldsymbol{\theta}\);
\(c\) define o limiar entre dados normais e anómalos. Valores comuns
situam-se entre \(1.345\) e \(2.0\). Marchant e Lark (2007) recomendam
testar diferentes valores de \(c\) e selecionar aquele que minimiza o
erro na validação cruzada, pois o grau de contaminação é desconhecido a
priori.

Ao minimizar \(L_{Rob}(\boldsymbol{\theta})\), obtemos estimativas dos
parâmetros do variograma (alcance, patamar) que representam a estrutura
espacial dominante do processo \(Y(\mathbf{s})\), ignorando as
perturbações locais causadas por contaminação.

\begin{Shaded}
\begin{Highlighting}[]
\NormalTok{pacman}\SpecialCharTok{::}\FunctionTok{p\_load}\NormalTok{(gt)}
\CommentTok{\#Calcular Variograma Experimental}
\NormalTok{v\_exp }\OtherTok{\textless{}{-}} \FunctionTok{variogram}\NormalTok{(}\FunctionTok{log}\NormalTok{(zinc) }\SpecialCharTok{\textasciitilde{}} \DecValTok{1}\NormalTok{, meuse\_sf)}

\CommentTok{\#Definir um modelo inicial (Chute inicial)}
\CommentTok{\# psill = patamar parcial, range = alcance, nugget = efeito pepita}
\NormalTok{modelo\_inicial }\OtherTok{\textless{}{-}} \FunctionTok{vgm}\NormalTok{(}\AttributeTok{psill =} \FloatTok{0.5}\NormalTok{, }\AttributeTok{model =} \StringTok{"Sph"}\NormalTok{, }\AttributeTok{range =} \DecValTok{900}\NormalTok{, }\AttributeTok{nugget =} \FloatTok{0.1}\NormalTok{)}

\CommentTok{\# Ajuste Automático (Mínimos Quadrados Ponderados {-} WLS)}
\NormalTok{modelo\_ajustado }\OtherTok{\textless{}{-}} \FunctionTok{fit.variogram}\NormalTok{(v\_exp, }\AttributeTok{model =}\NormalTok{ modelo\_inicial)}

\NormalTok{modelo\_ajustado}\SpecialCharTok{\%\textgreater{}\%}
\NormalTok{  knitr}\SpecialCharTok{::}\FunctionTok{kable}\NormalTok{()}
\end{Highlighting}
\end{Shaded}

\begin{figure}

\centering{

\begin{longtable*}[]{@{}lrrrrrrrr@{}}
\toprule\noalign{}
model & psill & range & kappa & ang1 & ang2 & ang3 & anis1 & anis2 \\
\midrule\noalign{}
\endhead
\bottomrule\noalign{}
\endlastfoot
Nug & 0.0506602 & 0.0000 & 0.0 & 0 & 0 & 0 & 1 & 1 \\
Sph & 0.5906056 & 897.0044 & 0.5 & 0 & 0 & 0 & 1 & 1 \\
\end{longtable*}

}

\caption{\label{fig-ajuste-modelo-1}Ajuste do modelo teórico sobre o
experimental usando WLS}

\end{figure}%

\begin{Shaded}
\begin{Highlighting}[]
\NormalTok{preds }\OtherTok{\textless{}{-}} \FunctionTok{variogramLine}\NormalTok{(modelo\_ajustado, }\AttributeTok{maxdist =} \FunctionTok{max}\NormalTok{(v\_exp}\SpecialCharTok{$}\NormalTok{dist))}

\FunctionTok{ggplot}\NormalTok{() }\SpecialCharTok{+}
  \FunctionTok{geom\_point}\NormalTok{(}\AttributeTok{data =}\NormalTok{ v\_exp, }\FunctionTok{aes}\NormalTok{(}\AttributeTok{x =}\NormalTok{ dist, }\AttributeTok{y =}\NormalTok{ gamma), }\AttributeTok{size =} \DecValTok{3}\NormalTok{) }\SpecialCharTok{+}
  \FunctionTok{geom\_line}\NormalTok{(}\AttributeTok{data =}\NormalTok{ preds, }\FunctionTok{aes}\NormalTok{(}\AttributeTok{x =}\NormalTok{ dist, }\AttributeTok{y =}\NormalTok{ gamma),}
            \AttributeTok{color =} \StringTok{"red"}\NormalTok{, }\AttributeTok{linewidth =}\NormalTok{.}\DecValTok{5}\NormalTok{) }\SpecialCharTok{+}
  \FunctionTok{labs}\NormalTok{(}
    \AttributeTok{title =} \StringTok{"Modelo Esférico"}\NormalTok{,}
    \AttributeTok{subtitle =} \FunctionTok{paste0}\NormalTok{(}
      \StringTok{"Nugget: "}\NormalTok{, }\FunctionTok{round}\NormalTok{(modelo\_ajustado}\SpecialCharTok{$}\NormalTok{psill[}\DecValTok{1}\NormalTok{], }\DecValTok{3}\NormalTok{),}
      \StringTok{" | Sill Total: "}\NormalTok{, }\FunctionTok{round}\NormalTok{(}\FunctionTok{sum}\NormalTok{(modelo\_ajustado}\SpecialCharTok{$}\NormalTok{psill), }\DecValTok{3}\NormalTok{),}
      \StringTok{" | Range: "}\NormalTok{, }\FunctionTok{round}\NormalTok{(modelo\_ajustado}\SpecialCharTok{$}\NormalTok{range[}\DecValTok{2}\NormalTok{], }\DecValTok{1}\NormalTok{)}
\NormalTok{    ),}
    \AttributeTok{x =} \StringTok{"Distância"}\NormalTok{,}
    \AttributeTok{y =} \StringTok{"Semivariância"}
\NormalTok{  ) }\SpecialCharTok{+}
  \FunctionTok{theme\_bw}\NormalTok{()}
\end{Highlighting}
\end{Shaded}

\begin{figure}[H]

\centering{

\pandocbounded{\includegraphics[keepaspectratio]{geostat_files/figure-pdf/fig-ajuste-modelo-1.pdf}}

}

\caption{\label{fig-ajuste-modelo-2}Ajuste do modelo teórico sobre o
experimental usando WLS}

\end{figure}%

\section{Métodos de Predição}\label{sec-metodos_predicao}

A predição espacial é o objetivo central da geoestatística: estimar o
valor da variável regionalizada \(Y(\mathbf{s}_0)\) num local não
amostrado \(\mathbf{s}_0 \in D^G\), baseando-se nos valores observados
\(\mathbf{y} = (y(\mathbf{s}_1), \dots, y(\mathbf{s}_n))^\top\) em
locais vizinhos. O método universalmente consagrado para esta tarefa é a
Krigagem (termo cunhado por Matheron em homenagem a D.G. Krige).

A Krigagem é definida como o Melhor Estimador Linear Não Viciado (BLUE -
Best Linear Unbiased Estimator) da variável \(Y(\mathbf{s}_0)\). Vamos
decompor esta definição para compreender a sua fundamentação teórica
(Noel Cressie 1993; Wackernagel 2003).

\begin{itemize}
\tightlist
\item
  Estimador Linear: O preditor \(\hat{Y}(\mathbf{s}_0)\) é uma
  combinação linear ponderada das observações disponíveis:
\end{itemize}

\[
\hat{Y}(\mathbf{s}_0) = \sum_{i=1}^{n} \lambda_i Y(\mathbf{s}_i) = \boldsymbol{\lambda}^\top \mathbf{Y}
\]

onde \(\lambda_i\) são os pesos atribuídos a cada observação
\(Y(\mathbf{s}_i)\) e
\(\boldsymbol{\lambda} = (\lambda_1, \dots, \lambda_n)^\top\) é o vetor
de pesos. O objetivo da krigagem é determinar os pesos ótimos
\(\lambda_i\).

\begin{itemize}
\tightlist
\item
  Não Viciado (Unbiasedness): Exige-se que, em média, o estimador não
  cometa erros sistemáticos. A esperança do erro de predição deve ser
  nula:
\end{itemize}

\[
E[\hat{Y}(\mathbf{s}_0) - Y(\mathbf{s}_0)] = 0 \implies E[\hat{Y}(\mathbf{s}_0)] = E[Y(\mathbf{s}_0)]
\]

Substituindo a forma linear:

\[E\left[ \sum_{i=1}^{n} \lambda_i Y(\mathbf{s}_i) \right] = \sum_{i=1}^{n} \lambda_i E[Y(\mathbf{s}_i)] = E[Y(\mathbf{s}_0)]\]

Esta condição impõe restrições sobre os pesos \(\lambda_i\), dependendo
do modelo assumido para a média \(E[Y(\mathbf{s})]\) (média constante,
tendência polinomial, etc.).

\begin{itemize}
\tightlist
\item
  Melhor (Best - Minimum Variance): Entre todos os estimadores lineares
  não viciados possíveis (aqueles que satisfazem as condições acima), a
  krigagem escolhe aquele que minimiza a variância do erro de predição
  (ou variância de estimação).
\end{itemize}

O erro de estimação é
\(\varepsilon = \hat{Y}(\mathbf{s}_0) - Y(\mathbf{s}_0)\). A variância
deste erro, que queremos minimizar, é:

\[
\sigma_E^2(\mathbf{s}_0) = \text{Var}(\hat{Y}(\mathbf{s}_0) - Y(\mathbf{s}_0)) = E\left[ (\hat{Y}(\mathbf{s}_0) - Y(\mathbf{s}_0))^2 \right]
\]

A minimização desta função objetivo quadrática em relação aos pesos
\(\lambda_i\), sujeita às restrições de não enviesamento, é realizada
através do método dos Multiplicadores de Lagrange, resultando no Sistema
de Krigagem.

Para minimizar a variância do erro sob restrições lineares, construímos
a função de Lagrange. A forma geral do sistema depende das suposições
sobre a média \(\mu(\mathbf{s})\). Contudo, a estrutura fundamental
deriva da expansão da variância do erro em termos da covariância
espacial \(C(\mathbf{h})\) ou do variograma \(\gamma(\mathbf{h})\).

\begin{equation}\phantomsection\label{eq-varerro}{
\begin{aligned}
\sigma_E^2 &= \text{Var}\left( \sum_{i=1}^n \lambda_i Y(\mathbf{s}_i) - Y(\mathbf{s}_0) \right) \\
&= \text{Var}\left( \sum_{i=1}^n \lambda_i Y(\mathbf{s}_i) \right) + \text{Var}(Y(\mathbf{s}_0)) - 2\text{Cov}\left( \sum_{i=1}^n \lambda_i Y(\mathbf{s}_i), Y(\mathbf{s}_0) \right) \\
&= \sum_{i=1}^n \sum_{j=1}^n \lambda_i \lambda_j C(\mathbf{s}_i, \mathbf{s}_j) + C(0) - 2 \sum_{i=1}^n \lambda_i C(\mathbf{s}_i, \mathbf{s}_0)
\end{aligned}
}\end{equation} onde
\(C(\mathbf{s}_i, \mathbf{s}_j) = \text{Cov}(Y(\mathbf{s}_i), Y(\mathbf{s}_j))\)
é a covariância entre dados, e \(C(0) = \sigma^2\) é a variância a
priori do processo (Noel Cressie 1993).

A minimização desta expressão (com restrições) conduz a um sistema de
equações lineares da forma \(\mathbf{A}\mathbf{x} = \mathbf{b}\), onde:

\begin{itemize}
\item
  \(\mathbf{A}\): Matriz de covariâncias (ou variogramas) entre as
  observações amostrais (redundância de informação).
\item
  \(\mathbf{x}\): Vetor das incógnitas (os pesos \(\lambda_i\) e os
  multiplicadores de Lagrange).
\item
  \(\mathbf{b}\): Vetor de covariâncias (ou variogramas) entre as
  observações e o ponto a estimar \(\mathbf{s}_0\) (proximidade da
  informação).
\end{itemize}

\subsection{Krigagem Simples (KS)}\label{krigagem-simples-ks}

A Krigagem Simples (KS) é a forma mais básica, assumindo que a média do
processo \(E[Y(\mathbf{s})] = \mu\) é conhecida e constante em todo o
domínio (Andre G. Journel e Huijbregts 1976). Como a média \(\mu\) é
conhecida, o estimador baseia-se nos resíduos em relação à média:

\[\hat{Y}_{KS}(\mathbf{s}_0) = \mu + \sum_{i=1}^n \lambda_i (Y(\mathbf{s}_i) - \mu)\]

\textbf{Não enviesamento}

O estimador é não viciado por construção para quaisquer pesos
\(\lambda_i\), pois \(E[Y(\mathbf{s}_i) - \mu] = 0\).

\[E[\hat{Y}_{KS}(\mathbf{s}_0) - Y(\mathbf{s}_0)] = \mu + \sum \lambda_i E[Y(\mathbf{s}_i) - \mu] - E[Y(\mathbf{s}_0)] = \mu + 0 - \mu = 0\]
Portanto, não existem restrições de soma de pesos (sem multiplicadores
de Lagrange).

\textbf{Variância mínima}

Para minimizar \(\sigma_E^2\) derivamos Eq.~\ref{eq-varerro} em relação
a cada peso \(\lambda_k\) (para \(k=1,\ldots, n\)) e igualamos a zero:

\begin{itemize}
\tightlist
\item
  Derivada do termo linear:
\end{itemize}

\[
  \frac{\partial}{\partial \lambda_k} \left( -2 \sum_{i=1}^n \lambda_i C(\mathbf{s}_i, \mathbf{s}_0) \right) = -2 C(\mathbf{s}_k, \mathbf{s}_0)
\]

\begin{itemize}
\tightlist
\item
  Derivada do termo quadrático (dupla soma):
\end{itemize}

Aqui, a derivada gera dois termos idênticos devido à simetria da
covariância e à regra do produto:

\[
  \frac{\partial}{\partial \lambda_k} \left( \sum_{i=1}^n \sum_{j=1}^n \lambda_i \lambda_j C(\mathbf{s}_i, \mathbf{s}_j) \right)   = \sum_{j=1}^n \lambda_j C(\mathbf{s}_k, \mathbf{s}_j) + \sum_{i=1}^n \lambda_i C(\mathbf{s}_i, \mathbf{s}_k)
\] Como
\(C(\mathbf{s}_i, \mathbf{s}_k) = C(\mathbf{s}_k, \mathbf{s}_i)\), as
duas somas são iguais.

\begin{itemize}
\tightlist
\item
  A derivada de \(C(\mathbf{s}_0, \mathbf{s}_0)\) (constante) é zero.
\end{itemize}

Assim, temos:

\[
\begin{aligned}
\frac{\partial \sigma_E^2}{\partial \lambda_k} &= \sum_{j=1}^n \lambda_j C(\mathbf{s}_k, \mathbf{s}_j) + \sum_{i=1}^n \lambda_i C(\mathbf{s}_i, \mathbf{s}_k) - 2C(\mathbf{s}_k, \mathbf{s}_0) = 0\\
&\text{Como as duas somas são idênticas, podemos combiná-las}\\
2 \sum_{j=1}^n \lambda_j C_{kj} - 2C_{k0} &= 0 \implies \sum_{j=1}^n \lambda_j C_{kj} = C_{k0}, \quad \forall k=1,\dots,n
\end{aligned}
\]

Em notação matricial:
\(\mathbf{\Sigma} \boldsymbol{\lambda} = \mathbf{c}\), onde
\(\mathbf{\Sigma}\) é a matriz de covariância dados-dados e
\(\mathbf{c}\) é o vetor de covariância dados-alvo. A solução é:
\(\boldsymbol{\lambda} = \mathbf{\Sigma}^{-1} \mathbf{c}\).

A Krigagem Simples é raramente usada na prática pois o conhecimento
exato da média global \(\mu\) é incomum. No entanto, é o fundamento
teórico para métodos mais avançados como a Krigagem Gaussiana e
Simulação condicional.

\begin{Shaded}
\begin{Highlighting}[]
\NormalTok{pacman}\SpecialCharTok{::}\FunctionTok{p\_load}\NormalTok{(stars, gstat, ggplot2, sf, patchwork, viridis)}

\FunctionTok{data}\NormalTok{(meuse)}
\FunctionTok{data}\NormalTok{(meuse.area) }

\NormalTok{meuse\_sf }\OtherTok{\textless{}{-}} \FunctionTok{st\_as\_sf}\NormalTok{(meuse, }\AttributeTok{coords =} \FunctionTok{c}\NormalTok{(}\StringTok{"x"}\NormalTok{, }\StringTok{"y"}\NormalTok{), }\AttributeTok{crs =} \DecValTok{28992}\NormalTok{)}

\NormalTok{area\_sf }\OtherTok{\textless{}{-}} \FunctionTok{st\_polygon}\NormalTok{(}\FunctionTok{list}\NormalTok{(}\FunctionTok{as.matrix}\NormalTok{(meuse.area))) }\SpecialCharTok{|\textgreater{}} 
  \FunctionTok{st\_sfc}\NormalTok{(}\AttributeTok{crs =} \DecValTok{28992}\NormalTok{) }\SpecialCharTok{|\textgreater{}} 
  \FunctionTok{st\_as\_sf}\NormalTok{()}

\CommentTok{\#Criar Grid de Predição }
\CommentTok{\# Precisamos definir ONDE queremos estimar}
\NormalTok{grid\_pred }\OtherTok{\textless{}{-}} \FunctionTok{st\_bbox}\NormalTok{(area\_sf) }\SpecialCharTok{|\textgreater{}}       
  \FunctionTok{st\_as\_stars}\NormalTok{(}\AttributeTok{dx =} \DecValTok{40}\NormalTok{, }\AttributeTok{dy =} \DecValTok{40}\NormalTok{) }\SpecialCharTok{|\textgreater{}}     \CommentTok{\# Define a resolução}
  \FunctionTok{st\_crop}\NormalTok{(area\_sf)                     }\CommentTok{\# RECORTA usando o polígono}


\CommentTok{\#Ajuste do Variograma}
\NormalTok{v\_ord }\OtherTok{\textless{}{-}} \FunctionTok{variogram}\NormalTok{(}\FunctionTok{log}\NormalTok{(zinc) }\SpecialCharTok{\textasciitilde{}} \DecValTok{1}\NormalTok{, meuse\_sf)}
\NormalTok{m\_ord }\OtherTok{\textless{}{-}} \FunctionTok{fit.variogram}\NormalTok{(v\_ord, }\FunctionTok{vgm}\NormalTok{(}\FloatTok{0.6}\NormalTok{, }\StringTok{"Sph"}\NormalTok{, }\DecValTok{900}\NormalTok{, }\FloatTok{0.05}\NormalTok{))}

\NormalTok{media\_conhecida }\OtherTok{\textless{}{-}} \FunctionTok{mean}\NormalTok{(}\FunctionTok{log}\NormalTok{(meuse}\SpecialCharTok{$}\NormalTok{zinc))}

\CommentTok{\#Krigagem Simples}
\CommentTok{\# O resultado terá duas camadas: var1.pred e var1.var}
\NormalTok{ks }\OtherTok{\textless{}{-}} \FunctionTok{krige}\NormalTok{(}\FunctionTok{log}\NormalTok{(zinc) }\SpecialCharTok{\textasciitilde{}} \DecValTok{1}\NormalTok{, }
            \AttributeTok{locations =}\NormalTok{ meuse\_sf, }
            \AttributeTok{newdata =}\NormalTok{ grid\_pred, }
            \AttributeTok{model =}\NormalTok{ m\_ord, }
            \AttributeTok{beta =}\NormalTok{ media\_conhecida, }\AttributeTok{debug.level =} \DecValTok{0}\NormalTok{)}

\CommentTok{\# Mapa de Predição}
\NormalTok{ks\_masked }\OtherTok{\textless{}{-}}\NormalTok{ ks[area\_sf]}
\NormalTok{p1 }\OtherTok{\textless{}{-}} \FunctionTok{ggplot}\NormalTok{() }\SpecialCharTok{+}
  \FunctionTok{geom\_stars}\NormalTok{(}\AttributeTok{data =}\NormalTok{ ks\_masked, }\FunctionTok{aes}\NormalTok{(}\AttributeTok{fill =}\NormalTok{ var1.pred)) }\SpecialCharTok{+}
  \FunctionTok{geom\_sf}\NormalTok{(}\AttributeTok{data =}\NormalTok{ area\_sf, }\AttributeTok{fill =} \ConstantTok{NA}\NormalTok{, }\AttributeTok{color =} \StringTok{"black"}\NormalTok{, }\AttributeTok{linewidth =} \FloatTok{0.5}\NormalTok{) }\SpecialCharTok{+}
  \FunctionTok{scale\_fill\_viridis\_c}\NormalTok{(}\AttributeTok{option =} \StringTok{"B"}\NormalTok{, }\AttributeTok{na.value =} \StringTok{"transparent"}\NormalTok{) }\SpecialCharTok{+}
  \FunctionTok{labs}\NormalTok{(}\AttributeTok{title =} \StringTok{"Predição (Log Zinc)"}\NormalTok{) }\SpecialCharTok{+}
  \FunctionTok{theme\_void}\NormalTok{() }\SpecialCharTok{+}
  \FunctionTok{theme}\NormalTok{(}\AttributeTok{plot.title =} \FunctionTok{element\_text}\NormalTok{(}\AttributeTok{hjust =} \FloatTok{0.5}\NormalTok{))}

\CommentTok{\# Mapa de Variância (Erro)}
\CommentTok{\# A variável de erro se chama \textquotesingle{}var1.var\textquotesingle{}}
\NormalTok{p2 }\OtherTok{\textless{}{-}} \FunctionTok{ggplot}\NormalTok{() }\SpecialCharTok{+}
  \FunctionTok{geom\_stars}\NormalTok{(}\AttributeTok{data =}\NormalTok{ ks\_masked, }\FunctionTok{aes}\NormalTok{(}\AttributeTok{fill =}\NormalTok{ var1.var)) }\SpecialCharTok{+}
 \FunctionTok{geom\_sf}\NormalTok{(}\AttributeTok{data =}\NormalTok{ area\_sf, }\AttributeTok{fill =} \ConstantTok{NA}\NormalTok{, }\AttributeTok{color =} \StringTok{"black"}\NormalTok{, }\AttributeTok{size =} \FloatTok{0.5}\NormalTok{) }\SpecialCharTok{+} \CommentTok{\# Contorno}
  \FunctionTok{scale\_fill\_viridis\_c}\NormalTok{(}\AttributeTok{option =} \StringTok{"B"}\NormalTok{, }\AttributeTok{na.value =} \StringTok{"transparent"}\NormalTok{) }\SpecialCharTok{+} 
  \FunctionTok{labs}\NormalTok{(}\AttributeTok{title =} \StringTok{"Variância de Krigagem"}\NormalTok{, }\AttributeTok{fill =} \StringTok{"Var"}\NormalTok{, }\AttributeTok{x =} \ConstantTok{NULL}\NormalTok{, }\AttributeTok{y =} \ConstantTok{NULL}\NormalTok{) }\SpecialCharTok{+}
  \FunctionTok{theme\_void}\NormalTok{() }\SpecialCharTok{+}
  \FunctionTok{theme}\NormalTok{(}\AttributeTok{plot.title =} \FunctionTok{element\_text}\NormalTok{(}\AttributeTok{hjust =} \FloatTok{0.5}\NormalTok{))}

\NormalTok{p1 }\SpecialCharTok{+}\NormalTok{ p2}
\end{Highlighting}
\end{Shaded}

\begin{figure}[H]

\centering{

\pandocbounded{\includegraphics[keepaspectratio]{geostat_files/figure-pdf/fig-krigagem_simples-1.pdf}}

}

\caption{\label{fig-krigagem_simples}Krigagem simples: (a) predição e
(b) variância de krigagem}

\end{figure}%

\subsection{Krigagem Ordinária (KO)}\label{krigagem-ordinuxe1ria-ko}

A Krigagem Ordinária (KO) assume que a média é constante
(\(E[Y(\mathbf{s})] = \mu\)) mas desconhecida (Noel Cressie 1993). O
modelo deve estimar a média implicitamente localmente, adaptando-se a
flutuações locais do nível da variável.

O estimador é uma combinação linear direta:

\[\hat{Y}_{KO}(\mathbf{s}_0) = \sum_{i=1}^n \lambda_i Y(\mathbf{s}_i)\]

Para garantir o não enviesamento sem conhecer \(\mu\):

\begin{equation}\phantomsection\label{eq-restriuxe7uxe3o}{E[\hat{Y}_{KO}(\mathbf{s}_0)] = \sum \lambda_i E[Y(\mathbf{s}_i)] = \mu \sum \lambda_i = \mu \implies \sum_{i=1}^n \lambda_i = 1}\end{equation}

Para que o erro seja zero para qualquer \(\mu\), é necessário impor a
restrição \(\sum_{i=1}^n \lambda_i = 1\).

A esperança na Eq.~\ref{eq-restrição} pode ser rescrita por:

\[E[\hat{Y}_{KO} - Y_0] = \sum \lambda_i E[Y_i] - E[Y_0] = \mu \sum \lambda_i - \mu = \mu \left( \sum_{i=1}^n \lambda_i - 1 \right)\]

Como nosso problema agora não é simplesmente encontrar o mínimo de \$
\sigma\_E\^{}2(\lambda\_1, \ldots, \lambda\_n)\$, e sim, encontrar o
mínimo sujeito a uma restrição, isto é,

\[
\begin{aligned}
& \underset{\lambda_1, \dots, \lambda_n}{\text{minimizar}}
& & \sigma_E^2(\lambda_1, ..., \lambda_n) \\
& \text{sujeito a}
& & \sum_{i=1}^n \lambda_i - 1 = 0
\end{aligned}
\] temos que somos obrigados a introduzir uma nova variável, o
multiplicador \(\nu\), e criamos a função Lagrangiana \(L\), que combina
a função objetivo original e a restrição. O fator 2 é pura conveniência
algébrica, pois cancela o 2 que surgirá das derivadas, simplificando as
equações finais.

O
\href{https://pt.wikipedia.org/wiki/Multiplicadores_de_Lagrange}{Multiplicador
de Lagrange} é a ferramenta matemática usada para transformar um
problema de otimização com restrições em um problema de otimização sem
restrições, facilitando a solução. Assim, temos:

\[
\begin{aligned}
L(\boldsymbol{\lambda}, \nu) &= \sigma_E^2 + 2\nu \left( \sum_{i=1}^n \lambda_i - 1 \right)\\
L(\boldsymbol{\lambda}, \nu) &= \sum_i \sum_j \lambda_i \lambda_j C_{ij} + C(0) - 2 \sum_i \lambda_i C_{i0} + 2\nu \left( \sum_i \lambda_i - 1 \right)
\end{aligned}
\]

Derivando em relação a \(\lambda_k\) e \(\nu\) obtemos:

\begin{itemize}
\item
  \(\frac{\partial L}{\partial \lambda_k} = 2 \sum_{j=1}^n \lambda_j C_{kj} - 2 C_{k0} + 2\nu = 0 \implies \sum_{j=1}^n \lambda_j C_{kj} + \nu = C_{k0}\)
\item
  \(\frac{\partial L}{\partial \nu} = 2 \left( \sum_{i=1}^n \lambda_i - 1 \right) = 0 \implies \sum_{i=1}^n \lambda_i = 1\)
\end{itemize}

Isto resulta num sistema de Krigagem Ordinária equações:

O sistema de \((n+1)\) equações resultante (em termos de covariograma
\(C(\cdot)\)) é:

\[\begin{cases}
\sum_{j=1}^n \lambda_j  C_{kj} + \nu = C_{k0}, & i=1, \dots, n \\
\sum_{j=1}^n \lambda_j = 1
\end{cases}\]

Em notação matricial
\(\mathbf{A}_{KO} \mathbf{x}_{KO} = \mathbf{b}_{KO}\):

\[
\begin{bmatrix}
C_{11} & \dots & C_{1n} & 1 \\
\vdots & \ddots & \vdots & \vdots \\
C_{n1} & \dots & C_{nn} & 1 \\
1 & \dots & 1 & 0
\end{bmatrix}
\begin{bmatrix}
\lambda_1 \\ \vdots \\ \lambda_n \\ \nu
\end{bmatrix}
=
\begin{bmatrix}
C_{10} \\ \vdots \\ C_{n0} \\ 1
\end{bmatrix}
\] A Krigagem Ordinária é robusta a tendências locais (drift) se a
vizinhança de krigagem for restrita, pois reestima a média localmente em
cada janela de busca.

\textbf{Variância de Krigagem}

Para obter a variância de krigagem ordinária \(\sigma_{KO}^2\),
substituímos a relação de otimalidade
(\(\sum \lambda_j C_{ij} = C_{i0} - \nu\)) na expressão original da
variância do erro:

\[
\begin{aligned}
\sigma_{KO}^2 &= \sum_{i=1}^n \lambda_i \underbrace{\left( \sum_{j=1}^n \lambda_j C_{ij} \right)}_{C_{i0} - \nu} + C(0) - 2 \sum_{i=1}^n \lambda_i C_{i0} \\
&= \sum_{i=1}^n \lambda_i (C_{i0} - \nu) + C(0) - 2 \sum_{i=1}^n \lambda_i C_{i0} \\
&= \sum_{i=1}^n \lambda_i C_{i0} - \nu \underbrace{\sum_{i=1}^n \lambda_i}_{1} + C(0) - 2 \sum_{i=1}^n \lambda_i C_{i0} \\
&= C(0) - \sum_{i=1}^n \lambda_i C_{i0} - \nu
\end{aligned}
\]

Em termos de variograma (\(\gamma (h)\)), e usando a relação
\(C(h) = C(0) - \gamma(h)\), a expressão equivalente é (Andre G. Journel
e Huijbregts 1976):

\[\sigma_{KO}^2 = \sum_{i=1}^n \lambda_i \gamma(\mathbf{s}_i - \mathbf{s}_0) + \nu - \gamma(\mathbf{0})\]

Note-se que \(\nu\) é o multiplicador de Lagrange, que pode ser
interpretado como o custo em variância de não conhecermos a média
verdadeira.

Se assumirmos que \(\gamma(\mathbf{0}) = 0\) (o variograma teórico na
origem é nulo, o efeito pepita \(C_0\) é o limite quando \(h \to 0\)), a
expressão simplifica-se. Em termos matriciais:

\[\sigma_{KO}^2(\mathbf{s}_0) = \mathbf{b}_{KO}^\top \mathbf{x}_{KO} = \sum_{i=1}^n \lambda_i \gamma_{i0} + \nu\]

A variância de krigagem depende apenas da configuração geométrica das
amostras e do modelo de variograma, mas não depende dos valores
observados \(y(\mathbf{s}_i)\) (propriedade de homocedasticidade da
krigagem linear sob normalidade). O mapa de variância de krigagem
reflete a densidade da amostragem:

\begin{itemize}
\item
  É zero nos locais amostrados (se não houver efeito pepita/erro de
  medição e usarmos krigagem exata).
\item
  Aumenta à medida que nos afastamos dos pontos de dados.
\item
  Depende da estrutura espacial: modelos com maior efeito pepita ou
  menor alcance resultam em maior incerteza de predição.
\end{itemize}

\begin{Shaded}
\begin{Highlighting}[]
\NormalTok{pacman}\SpecialCharTok{::}\FunctionTok{p\_load}\NormalTok{(stars, gstat, ggplot2, sf, patchwork, viridis, sp)}


\FunctionTok{data}\NormalTok{(meuse)}
\FunctionTok{data}\NormalTok{(meuse.area) }

\NormalTok{meuse\_sf }\OtherTok{\textless{}{-}} \FunctionTok{st\_as\_sf}\NormalTok{(meuse, }\AttributeTok{coords =} \FunctionTok{c}\NormalTok{(}\StringTok{"x"}\NormalTok{, }\StringTok{"y"}\NormalTok{), }\AttributeTok{crs =} \DecValTok{28992}\NormalTok{)}

\NormalTok{area\_sf }\OtherTok{\textless{}{-}} \FunctionTok{st\_polygon}\NormalTok{(}\FunctionTok{list}\NormalTok{(}\FunctionTok{as.matrix}\NormalTok{(meuse.area))) }\SpecialCharTok{|\textgreater{}} 
  \FunctionTok{st\_sfc}\NormalTok{(}\AttributeTok{crs =} \DecValTok{28992}\NormalTok{) }\SpecialCharTok{|\textgreater{}} 
  \FunctionTok{st\_as\_sf}\NormalTok{()}

\CommentTok{\#Criar Grid de Predição }
\CommentTok{\# Precisamos definir ONDE queremos estimar}
\NormalTok{grid\_pred }\OtherTok{\textless{}{-}} \FunctionTok{st\_bbox}\NormalTok{(area\_sf) }\SpecialCharTok{|\textgreater{}}       
  \FunctionTok{st\_as\_stars}\NormalTok{(}\AttributeTok{dx =} \DecValTok{40}\NormalTok{, }\AttributeTok{dy =} \DecValTok{40}\NormalTok{) }\SpecialCharTok{|\textgreater{}}     
  \FunctionTok{st\_crop}\NormalTok{(area\_sf)                     }

\CommentTok{\#Krigagem}
\CommentTok{\# modelo\_ajustado \textless{}{-} vgm(0.5, "Sph", 900, 0.1)}

\NormalTok{krigagem }\OtherTok{\textless{}{-}} \FunctionTok{krige}\NormalTok{(}
  \FunctionTok{log}\NormalTok{(zinc) }\SpecialCharTok{\textasciitilde{}} \DecValTok{1}\NormalTok{,}
  \AttributeTok{locations =}\NormalTok{ meuse\_sf,}
  \AttributeTok{newdata   =}\NormalTok{ grid\_pred,}
  \AttributeTok{model     =}\NormalTok{ modelo\_ajustado, }\AttributeTok{debug.level =} \DecValTok{0}
\NormalTok{)}

\CommentTok{\# var1.pred = Valor Estimado }
\CommentTok{\# var1.var = Variância de Krigagem (Erro)}
\CommentTok{\# Mapa 1: Predição}
\NormalTok{p1 }\OtherTok{\textless{}{-}} \FunctionTok{ggplot}\NormalTok{() }\SpecialCharTok{+}
  \FunctionTok{geom\_stars}\NormalTok{(}\AttributeTok{data =}\NormalTok{ krigagem, }\FunctionTok{aes}\NormalTok{(}\AttributeTok{fill =}\NormalTok{ var1.pred)) }\SpecialCharTok{+}
  \FunctionTok{geom\_sf}\NormalTok{(}\AttributeTok{data =}\NormalTok{ area\_sf, }\AttributeTok{fill =} \ConstantTok{NA}\NormalTok{, }\AttributeTok{color =} \StringTok{"black"}\NormalTok{, }\AttributeTok{size =} \FloatTok{0.5}\NormalTok{) }\SpecialCharTok{+} \CommentTok{\# Contorno}
  \FunctionTok{scale\_fill\_viridis\_c}\NormalTok{(}\AttributeTok{option =} \StringTok{"B"}\NormalTok{, }\AttributeTok{name =} \StringTok{"Log(Zinc)"}\NormalTok{, }\AttributeTok{na.value =} \StringTok{"transparent"}\NormalTok{) }\SpecialCharTok{+}
  \FunctionTok{labs}\NormalTok{(}\AttributeTok{title =} \StringTok{"Predição"}\NormalTok{) }\SpecialCharTok{+}
  \FunctionTok{theme\_void}\NormalTok{()}\SpecialCharTok{+}
  \FunctionTok{theme}\NormalTok{(}\AttributeTok{plot.title =} \FunctionTok{element\_text}\NormalTok{(}\AttributeTok{hjust =} \FloatTok{0.5}\NormalTok{))}

\CommentTok{\# Mapa 2: Variância (Erro)}
\NormalTok{p2 }\OtherTok{\textless{}{-}} \FunctionTok{ggplot}\NormalTok{() }\SpecialCharTok{+}
  \FunctionTok{geom\_stars}\NormalTok{(}\AttributeTok{data =}\NormalTok{ krigagem, }\FunctionTok{aes}\NormalTok{(}\AttributeTok{fill =}\NormalTok{ var1.var)) }\SpecialCharTok{+}
  \FunctionTok{geom\_sf}\NormalTok{(}\AttributeTok{data =}\NormalTok{ area\_sf, }\AttributeTok{fill =} \ConstantTok{NA}\NormalTok{, }\AttributeTok{color =} \StringTok{"black"}\NormalTok{, }\AttributeTok{size =} \FloatTok{0.5}\NormalTok{) }\SpecialCharTok{+}
  \FunctionTok{scale\_fill\_viridis\_c}\NormalTok{(}\AttributeTok{option =} \StringTok{"B"}\NormalTok{, }\AttributeTok{name =} \StringTok{"Variância"}\NormalTok{, }\AttributeTok{na.value =} \StringTok{"transparent"}\NormalTok{) }\SpecialCharTok{+}
  \FunctionTok{labs}\NormalTok{(}\AttributeTok{title =} \StringTok{"Incerteza (Variância)"}\NormalTok{) }\SpecialCharTok{+}
  \FunctionTok{theme\_minimal}\NormalTok{() }\SpecialCharTok{+} \CommentTok{\#para colocar coordenadas}
  \FunctionTok{theme}\NormalTok{(}\AttributeTok{plot.title =} \FunctionTok{element\_text}\NormalTok{(}\AttributeTok{hjust =} \FloatTok{0.5}\NormalTok{),}
        \AttributeTok{axis.title =} \FunctionTok{element\_blank}\NormalTok{()}
\NormalTok{        )}

\NormalTok{p1 }\SpecialCharTok{+}\NormalTok{ p2}
\end{Highlighting}
\end{Shaded}

\begin{figure}[H]

\centering{

\pandocbounded{\includegraphics[keepaspectratio]{geostat_files/figure-pdf/fig-krigagem-1.pdf}}

}

\caption{\label{fig-krigagem}Krigagem Ordinária: (a) predição e (b)
variância de krigagem}

\end{figure}%

\subsection{Krigagem Universal (KU)}\label{krigagem-universal-ku}

Quando a média do processo não é constante, mas apresenta uma tendência
sistemática (drift) que pode ser modelada como uma função determinística
das coordenadas (ex:
\(E[Y(\mathbf{s})] = \mu(\mathbf{s}) = \beta_0 + \beta_1 x + \beta_2 y\)),
utilizamos a Krigagem Universal (KU) (George Matheron 1971).

Assumimos que a média é uma combinação linear de \(L+1\) funções de base
conhecidas \(f_l(\mathbf{s})\) (monómios):

\[\mu(\mathbf{s}) = \sum_{l=0}^L a_l f_l(\mathbf{s}), \quad \text{com } f_0(\mathbf{s}) \equiv 1\]
Isto pode ser reescrito como: \[
\begin{aligned}
&Y(\mathbf{s}) = \mu(\mathbf{s}) + \delta(\mathbf{s}), \quad \text{com } E[\delta(\mathbf{s})] = 0, \quad \text{Cov}(\delta(\mathbf{s}_i), \delta(\mathbf{s}_j)) = C(\mathbf{s}_i, \mathbf{s}_j)\\
&\mu(\mathbf{s}) = \sum_{l=0}^{L} a_l f_l(\mathbf{s}), \quad f_0(\mathbf{s}) \equiv 1\\
&\hat{Y}(\mathbf{s}_0) = \sum_{i=1}^n \lambda_i Y(\mathbf{s}_i)
\end{aligned}
\] \textbf{Condição de não enviesamento:}

\[
\begin{aligned}
E[\hat{Y}(\mathbf{s}_0) - Y(\mathbf{s}_0)] &= \sum_{i=1}^n \lambda_i \mu(\mathbf{s}_i) - \mu(\mathbf{s}_0)\\
&= \sum_{i=1}^n \lambda_i \sum_{l=0}^L a_l f_l(\mathbf{s}_i) - \sum_{l=0}^L a_l f_l(\mathbf{s}_0)\\
&= \sum_{l=0}^L a_l \left[ \sum_{i=1}^n \lambda_i f_l(\mathbf{s}_i) - f_l(\mathbf{s}_0) \right] \overset{!}{=} 0
\end{aligned}
\]

Como isto deve ser verdade para quaisquer coeficientes \(a_l\)
desconhecidos, os termos entre parênteses devem ser zero
individualmente. Isto gera \(L+1\) restrições:

\[\sum_{i=1}^n \lambda_i f_l(\mathbf{s}_i) = f_l(\mathbf{s}_0), \quad \text{para } l=0, \dots, L\]

\textbf{Variância do erro:}

\[
\begin{aligned}
\sigma_E^2 &= \text{Var}\left(\hat{Y}(\mathbf{s}_0) - Y(\mathbf{s}_0)\right)\\
&= \text{Var}\left(\sum_{i=1}^n \lambda_i \delta(\mathbf{s}_i) - \delta(\mathbf{s}_0)\right)\\
&= \sum_{i=1}^n \sum_{j=1}^n \lambda_i \lambda_j C(\mathbf{s}_i, \mathbf{s}_j) - 2\sum_{i=1}^n \lambda_i C(\mathbf{s}_i, \mathbf{s}_0) + C(\mathbf{s}_0, \mathbf{s}_0)
\end{aligned}
\]

De igual modo como feito na KO, introduzimos \(L+1\) multiplicadores de
Lagrange \(\nu_l\).

\[
\begin{aligned}
L(\boldsymbol{\lambda}, \boldsymbol{\nu}) &= \sigma_E^2 - 2\sum_{l=0}^L \nu_l \left(\sum_{i=1}^n \lambda_i f_l(\mathbf{s}_i) - f_l(\mathbf{s}_0)\right)\\
&= \sum_{i=1}^n\sum_{j=1}^n \lambda_i\lambda_j C_{ij} - 2\sum_{i=1}^n \lambda_i C_{i0} + C_{00} - 2\sum_{l=0}^L \nu_l\left(\sum_{i=1}^n \lambda_i f_{il} - f_{0l}\right)
\end{aligned}
\] Para procedermos com a minimização da variância, novamente calculamos
as derivadas em relação a \(\lambda_i\) e \(\nu_l\) :

\[
\begin{aligned}
\frac{\partial L}{\partial \lambda_k} &= \frac{\partial}{\partial \lambda_k}\left(\sum_{i=1}^n\sum_{j=1}^n \lambda_i\lambda_j C_{ij}\right) - 2C_{k0} - 2\sum_{l=0}^L \nu_l f_{kl}\\
&= \sum_{j=1}^n \lambda_j C_{kj} + \sum_{i=1}^n \lambda_i C_{ik} - 2C_{k0} - 2\sum_{l=0}^L \nu_l f_{kl}\\
2\sum_{j=1}^n \lambda_j C_{kj} - 2C_{k0} - 2\sum_{l=0}^L \nu_l f_{kl} &=  0\\
\sum_{j=1}^n \lambda_j C_{kj} - \sum_{l=0}^L \nu_l f_l(\mathbf{s}_k) &= C_{k0}, \quad k = 1,\dots,n
\end{aligned}
\]

\[
\begin{aligned}
\frac{\partial L}{\partial \nu_l} = -2\left(\sum_{i=1}^n \lambda_i f_l(\mathbf{s}_i) - f_l(\mathbf{s}_0)\right) &= 0\\
\sum_{i=1}^n \lambda_i f_l(\mathbf{s}_i) &= f_l(\mathbf{s}_0), \quad l = 0,\dots,L
\end{aligned}
\]

Para a notação matricial definimos:

\[
\begin{aligned}
&\mathbf{C}_{nn} = [C_{ij}]_{n\times n}, \quad \mathbf{F}_{n\times(L+1)} = [f_l(\mathbf{s}_i)]\\
&\mathbf{c}_{n0} = [C_{i0}]_{n\times 1}, \quad \mathbf{f}_{L0} = [f_l(\mathbf{s}_0)]_{(L+1)\times 1}\\
&\boldsymbol{\lambda} = [\lambda_1, \dots, \lambda_n]^\top, \quad \boldsymbol{\nu} = [\nu_0, \dots, \nu_L]^\top
\end{aligned}
\] e obtemos:

\[
\begin{bmatrix}
\mathbf{C}_{nn} & -\mathbf{F} \\
-\mathbf{F}^\top & \mathbf{0}
\end{bmatrix}
\begin{bmatrix}
\boldsymbol{\lambda} \\ \boldsymbol{\nu}
\end{bmatrix}
=
\begin{bmatrix}
\mathbf{c}_{n0} \\ -\mathbf{f}_{L0}
\end{bmatrix}
\] É comum alguns autores definem \(\mu_l = -\nu_l\), obtendo:

\[
\begin{bmatrix}
\mathbf{C}_{nn} & \mathbf{F} \\
\mathbf{F}^\top & \mathbf{0}
\end{bmatrix}
\begin{bmatrix}
\boldsymbol{\lambda} \\ \boldsymbol{\mu}
\end{bmatrix}
=
\begin{bmatrix}
\mathbf{c}_{n0} \\ \mathbf{f}_{L0}
\end{bmatrix}
\]

Onde \(\mathbf{F}\) contém os valores das funções de deriva nos pontos
amostrais.

A variância mínima é dada por:

\[
\sigma_{\min}^2 = C_{00} - \sum_{i=1}^n \lambda_i C_{i0} + \sum_{l=0}^L \nu_l f_l(\mathbf{s}_0)
\]

A matriz \(\mathbf{C}_{nn}\) deve representar a covariância dos resíduos
\(\delta(\mathbf{s})\), não dos dados originais \(Y(\mathbf{s})\). Como
os resíduos são desconhecidos, na prática estima-se frequentemente o
variograma na direção ortogonal à tendência ou usa-se REML para estimar
o variograma dos resíduos diretamente.

Quando \(L=0\) (apenas \(f_0(\mathbf{s})=1\)), recupera-se o sistema da
KO com um multiplicador \(\nu_0\).

\begin{Shaded}
\begin{Highlighting}[]
\NormalTok{pacman}\SpecialCharTok{::}\FunctionTok{p\_load}\NormalTok{(stars, gstat, ggplot2, sf, patchwork, viridis)}

\FunctionTok{data}\NormalTok{(meuse)}
\FunctionTok{data}\NormalTok{(meuse.grid)}


\NormalTok{meuse\_sf }\OtherTok{\textless{}{-}} \FunctionTok{st\_as\_sf}\NormalTok{(meuse, }\AttributeTok{coords =} \FunctionTok{c}\NormalTok{(}\StringTok{"x"}\NormalTok{, }\StringTok{"y"}\NormalTok{), }\AttributeTok{crs =} \DecValTok{28992}\NormalTok{, }\AttributeTok{remove =} \ConstantTok{FALSE}\NormalTok{)}

\NormalTok{grid\_sf }\OtherTok{\textless{}{-}} \FunctionTok{st\_as\_sf}\NormalTok{(meuse.grid, }\AttributeTok{coords =} \FunctionTok{c}\NormalTok{(}\StringTok{"x"}\NormalTok{, }\StringTok{"y"}\NormalTok{), }\AttributeTok{crs =} \DecValTok{28992}\NormalTok{, }\AttributeTok{remove =} \ConstantTok{FALSE}\NormalTok{)}

\CommentTok{\#fórmula: \textasciitilde{} x + y (tendência linear nas coordenadas)}
\NormalTok{v\_uni }\OtherTok{\textless{}{-}} \FunctionTok{variogram}\NormalTok{(}\FunctionTok{log}\NormalTok{(zinc) }\SpecialCharTok{\textasciitilde{}}\NormalTok{ x }\SpecialCharTok{+}\NormalTok{ y, meuse\_sf)}
\NormalTok{m\_uni }\OtherTok{\textless{}{-}} \FunctionTok{fit.variogram}\NormalTok{(v\_uni, }\FunctionTok{vgm}\NormalTok{(}\FloatTok{0.5}\NormalTok{, }\StringTok{"Sph"}\NormalTok{, }\DecValTok{800}\NormalTok{, }\FloatTok{0.05}\NormalTok{))}

\CommentTok{\#Krigagem Universal}
\NormalTok{ku\_points }\OtherTok{\textless{}{-}} \FunctionTok{krige}\NormalTok{(}\FunctionTok{log}\NormalTok{(zinc) }\SpecialCharTok{\textasciitilde{}}\NormalTok{ x }\SpecialCharTok{+}\NormalTok{ y, }
                   \AttributeTok{locations =}\NormalTok{ meuse\_sf, }
                   \AttributeTok{newdata =}\NormalTok{ grid\_sf, }
                   \AttributeTok{model =}\NormalTok{ m\_uni, }\AttributeTok{debug.level =} \DecValTok{0}\NormalTok{)}

\CommentTok{\# dx e dy definem o tamanho do pixel (40m para o dataset meuse)}
\NormalTok{ku\_stars }\OtherTok{\textless{}{-}} \FunctionTok{st\_rasterize}\NormalTok{(ku\_points, }\AttributeTok{dx =} \DecValTok{40}\NormalTok{, }\AttributeTok{dy =} \DecValTok{40}\NormalTok{)}

\CommentTok{\# Mapa de Predição}
\NormalTok{p1 }\OtherTok{\textless{}{-}} \FunctionTok{ggplot}\NormalTok{() }\SpecialCharTok{+}
  \FunctionTok{geom\_stars}\NormalTok{(}\AttributeTok{data =}\NormalTok{ ku\_stars, }\FunctionTok{aes}\NormalTok{(}\AttributeTok{fill =}\NormalTok{ var1.pred)) }\SpecialCharTok{+}
  \FunctionTok{scale\_fill\_viridis\_c}\NormalTok{(}\AttributeTok{option =} \StringTok{"B"}\NormalTok{, }\AttributeTok{na.value =} \StringTok{"transparent"}\NormalTok{) }\SpecialCharTok{+}
  \FunctionTok{labs}\NormalTok{(}\AttributeTok{title =} \StringTok{"Krigagem Universal (Pred)"}\NormalTok{, }\AttributeTok{fill =} \StringTok{"Log(Zn)"}\NormalTok{, }\AttributeTok{x =} \ConstantTok{NULL}\NormalTok{, }\AttributeTok{y =} \ConstantTok{NULL}\NormalTok{) }\SpecialCharTok{+}
  \FunctionTok{theme\_void}\NormalTok{() }\SpecialCharTok{+}
  \FunctionTok{theme}\NormalTok{(}\AttributeTok{plot.title =} \FunctionTok{element\_text}\NormalTok{(}\AttributeTok{hjust =} \FloatTok{0.5}\NormalTok{))}

\CommentTok{\# Mapa de Variância}
\NormalTok{p2 }\OtherTok{\textless{}{-}} \FunctionTok{ggplot}\NormalTok{() }\SpecialCharTok{+}
  \FunctionTok{geom\_stars}\NormalTok{(}\AttributeTok{data =}\NormalTok{ ku\_stars, }\FunctionTok{aes}\NormalTok{(}\AttributeTok{fill =}\NormalTok{ var1.var)) }\SpecialCharTok{+}
  \FunctionTok{scale\_fill\_viridis\_c}\NormalTok{(}\AttributeTok{option =} \StringTok{"B"}\NormalTok{, }\AttributeTok{na.value =} \StringTok{"transparent"}\NormalTok{) }\SpecialCharTok{+}
  \FunctionTok{labs}\NormalTok{(}\AttributeTok{title =} \StringTok{"Variância (Erro)"}\NormalTok{, }\AttributeTok{fill =} \StringTok{"Var"}\NormalTok{, }\AttributeTok{x =} \ConstantTok{NULL}\NormalTok{, }\AttributeTok{y =} \ConstantTok{NULL}\NormalTok{) }\SpecialCharTok{+}
  \FunctionTok{theme\_void}\NormalTok{() }\SpecialCharTok{+}
  \FunctionTok{theme}\NormalTok{(}\AttributeTok{plot.title =} \FunctionTok{element\_text}\NormalTok{(}\AttributeTok{hjust =} \FloatTok{0.5}\NormalTok{))}

\NormalTok{p1 }\SpecialCharTok{+}\NormalTok{ p2}
\end{Highlighting}
\end{Shaded}

\begin{figure}[H]

\centering{

\pandocbounded{\includegraphics[keepaspectratio]{geostat_files/figure-pdf/fig-krigagem_Universal-1.pdf}}

}

\caption{\label{fig-krigagem_Universal}Krigagem Universal: (a) predição
e (b) variância de krigagem}

\end{figure}%

\subsection{Krigagem com Deriva Externa
(KED)}\label{krigagem-com-deriva-externa-ked}

A Krigagem com Deriva Externa (KED) é uma variante da Krigagem Universal
onde a tendência \(\mu(\mathbf{s})\) não é modelada por coordenadas, mas
sim por uma ou mais variáveis auxiliares (covariáveis)
\(x_k(\mathbf{s})\) conhecidas exaustivamente em todo o domínio (ex: um
Modelo Digital de Elevação para interpolar temperatura ou precipitação)
(Goovaerts 1997).

O modelo para a média é:
\(E[Y(\mathbf{s})] = a_0 + a_1 x_1(\mathbf{s})\)

\[
\begin{aligned}
&Y(\mathbf{s}) = \mu(\mathbf{s}) + \delta(\mathbf{s}), \quad \text{com } E[\delta(\mathbf{s})] = 0, \quad \text{Cov}(\delta(\mathbf{s}_i), \delta(\mathbf{s}_j)) = C(\mathbf{s}_i, \mathbf{s}_j)\\
&\mu(\mathbf{s}) = a_0 + a_1 x(\mathbf{s}) \quad \text{(modelo linear com covariável } x\text{)}\\
&\hat{Y}(\mathbf{s}_0) = \sum_{i=1}^n \lambda_i Y(\mathbf{s}_i)
\end{aligned}
\] \textbf{Condição de não enviesamento:}

\[
\begin{aligned}
E[\hat{Y}(\mathbf{s}_0) - Y(\mathbf{s}_0)] &= \sum_{i=1}^n \lambda_i \mu(\mathbf{s}_i) - \mu(\mathbf{s}_0)\\
&= \sum_{i=1}^n \lambda_i (a_0 + a_1 x(\mathbf{s}_i)) - (a_0 + a_1 x(\mathbf{s}_0))\\
&= a_0\left(\sum_{i=1}^n \lambda_i - 1\right) + a_1\left(\sum_{i=1}^n \lambda_i x(\mathbf{s}_i) - x(\mathbf{s}_0)\right) \overset{!}{=} 0
\end{aligned}
\] Para que valha para quaisquer \(a_0, a_1\) desconhecidos é necessário
que:

\[
\sum_{i=1}^n \lambda_i = 1 \quad \text{e} \quad \sum_{i=1}^n \lambda_i x(\mathbf{s}_i) = x(\mathbf{s}_0)
\]

\textbf{Variância do erro}

\[
\begin{aligned}
\sigma_E^2 &= \text{Var}\left(\hat{Y}(\mathbf{s}_0) - Y(\mathbf{s}_0)\right)\\
&= \text{Var}\left(\sum_{i=1}^n \lambda_i \delta(\mathbf{s}_i) - \delta(\mathbf{s}_0)\right)\\
&= \sum_{i=1}^n\sum_{j=1}^n \lambda_i\lambda_j C_{ij} - 2\sum_{i=1}^n \lambda_i C_{i0} + C_{00}
\end{aligned}
\] Novamente, introduzindo multiplicadores \(\nu_0\) e \(\nu_1\) e para
minimizar derivamos derivamos:

\[
\begin{aligned}
L(\boldsymbol{\lambda}, \nu_0, \nu_1) &= \sigma_E^2 - 2\nu_0\left(\sum_{i=1}^n \lambda_i - 1\right) - 2\nu_1\left(\sum_{i=1}^n \lambda_i x(\mathbf{s}_i) - x(\mathbf{s}_0)\right)\\
&= \sum_{i=1}^n\sum_{j=1}^n \lambda_i\lambda_j C_{ij} - 2\sum_{i=1}^n \lambda_i C_{i0} + C_{00} - 2\nu_0\left(\sum_{i=1}^n \lambda_i - 1\right) - 2\nu_1\left(\sum_{i=1}^n \lambda_i x_i - x_0\right)
\end{aligned}
\]

\[
\begin{aligned}
\frac{\partial L}{\partial \lambda_k} &= \frac{\partial}{\partial \lambda_k}\left(\sum_{i=1}^n\sum_{j=1}^n \lambda_i\lambda_j C_{ij}\right) - 2C_{k0} - 2\nu_0 - 2\nu_1 x_k\\
&= \sum_{j=1}^n \lambda_j C_{kj} + \sum_{i=1}^n \lambda_i C_{ik} - 2C_{k0} - 2\nu_0 - 2\nu_1 x_k\\
&= 2\sum_{j=1}^n \lambda_j C_{kj} - 2C_{k0} - 2\nu_0 - 2\nu_1 x_k = 0\\
\sum_{j=1}^n \lambda_j C_{kj} - \nu_0 - \nu_1 x_k &= C_{k0}, \quad k = 1,\dots,n
\end{aligned}
\] \[
\begin{aligned}
\frac{\partial L}{\partial \nu_0} &= -2\left(\sum_{i=1}^n \lambda_i - 1\right) = 0 \quad \Rightarrow \quad \sum_{i=1}^n \lambda_i = 1\\
\frac{\partial L}{\partial \nu_1} &= -2\left(\sum_{i=1}^n \lambda_i x_i - x_0\right) = 0 \quad \Rightarrow \quad \sum_{i=1}^n \lambda_i x_i = x_0
\end{aligned}
\] Novamente definimos:

\[
\begin{aligned}
&\mathbf{C}_{nn} = [C_{ij}]_{n\times n}, \quad \mathbf{X} = [1 \ x_i]_{n\times 2}, \quad \mathbf{c}_{n0} = [C_{i0}]_{n\times 1}\\
&\mathbf{x}_0 = \begin{bmatrix} 1 \\ x_0 \end{bmatrix}_{2\times 1}, \quad \boldsymbol{\lambda} = [\lambda_1, \dots, \lambda_n]^\top, \quad \boldsymbol{\nu} = [\nu_0, \nu_1]^\top
\end{aligned}
\] para obter o sistema:

\[
\begin{bmatrix}
\mathbf{C}_{nn} & -\mathbf{X} \\
-\mathbf{X}^\top & \mathbf{0}
\end{bmatrix}
\begin{bmatrix}
\boldsymbol{\lambda} \\ \boldsymbol{\nu}
\end{bmatrix}
=
\begin{bmatrix}
\mathbf{c}_{n0} \\ -\mathbf{z}_0
\end{bmatrix}
\]

Alternativamente, com \(\boldsymbol{\mu} = -\boldsymbol{\nu}\):

\[
\begin{bmatrix}
\mathbf{C}_{nn} & \mathbf{X} \\
\mathbf{X}^\top & \mathbf{0}
\end{bmatrix}
\begin{bmatrix}
\boldsymbol{\lambda} \\ \boldsymbol{\mu}
\end{bmatrix}
=
\begin{bmatrix}
\mathbf{c}_{n0} \\ \mathbf{x}_0
\end{bmatrix}
\] \textbf{Variância mínima da krigagem:}

\[
\sigma_{\text{KED}}^2 = C_{00} - \sum_{i=1}^n \lambda_i C_{i0} + \nu_0 + \nu_1 x_0
\] A KED é um caso particular da KU onde as funções de base são
substituídas por covariáveis conhecidas. Para p covariáveis:

\[
\mu(\mathbf{s}) = a_0 + \sum_{k=1}^p a_k x_k(\mathbf{s}) \Rightarrow \sum_{i=1}^n \lambda_i x_k(\mathbf{s}_i) = x_k(\mathbf{s}_0), \ k=0,\dots,p
\] com \(x_0(\mathbf{s}) \equiv 1\).

Uma vantagem da KED é que ela incorpora eficientemente informação
auxiliar correlacionada com a variável primária, melhorando a precisão
da interpolação quando a covariável captura padrões de larga escala.
Note que esta covariável deve ser conhecida em todos os pontos
(amostrados e não amostrados). Covariância \(C\) deve representar a
dependência espacial dos resíduos após remover o efeito da covariável

\begin{Shaded}
\begin{Highlighting}[]
\NormalTok{pacman}\SpecialCharTok{::}\FunctionTok{p\_load}\NormalTok{(stars, gstat, ggplot2, sf, patchwork, viridis)}

\FunctionTok{data}\NormalTok{(meuse)}
\FunctionTok{data}\NormalTok{(meuse.grid)}

\CommentTok{\# Precisamos garantir que a coluna \textquotesingle{}dist\textquotesingle{} (nossa covariável x\_k) esteja presente}
\NormalTok{meuse\_sf }\OtherTok{\textless{}{-}} \FunctionTok{st\_as\_sf}\NormalTok{(meuse, }\AttributeTok{coords =} \FunctionTok{c}\NormalTok{(}\StringTok{"x"}\NormalTok{, }\StringTok{"y"}\NormalTok{), }\AttributeTok{crs =} \DecValTok{28992}\NormalTok{)}

\CommentTok{\#Preparar o Grid (Newdata)}
\CommentTok{\# Aqui usamos covariavel \textquotesingle{}dist\textquotesingle{} pois TAMBÉM precisa existir para todos os pixels}
\NormalTok{grid\_sf }\OtherTok{\textless{}{-}} \FunctionTok{st\_as\_sf}\NormalTok{(meuse.grid, }\AttributeTok{coords =} \FunctionTok{c}\NormalTok{(}\StringTok{"x"}\NormalTok{, }\StringTok{"y"}\NormalTok{), }\AttributeTok{crs =} \DecValTok{28992}\NormalTok{)}

\NormalTok{grid\_stars }\OtherTok{\textless{}{-}} \FunctionTok{st\_rasterize}\NormalTok{(grid\_sf, }\AttributeTok{dx =} \DecValTok{40}\NormalTok{, }\AttributeTok{dy =} \DecValTok{40}\NormalTok{)}

\CommentTok{\#Variograma}
\NormalTok{v\_ked }\OtherTok{\textless{}{-}} \FunctionTok{variogram}\NormalTok{(}\FunctionTok{log}\NormalTok{(zinc) }\SpecialCharTok{\textasciitilde{}}\NormalTok{ dist, meuse\_sf)}
\NormalTok{m\_ked }\OtherTok{\textless{}{-}} \FunctionTok{fit.variogram}\NormalTok{(v\_ked, }\FunctionTok{vgm}\NormalTok{(}\FloatTok{0.5}\NormalTok{, }\StringTok{"Exp"}\NormalTok{, }\DecValTok{800}\NormalTok{, }\FloatTok{0.05}\NormalTok{))}

\FunctionTok{plot}\NormalTok{(v\_ked, m\_ked, }\AttributeTok{main =} \StringTok{"Variograma (KED)"}\NormalTok{)}

\CommentTok{\# Krigagem com Deriva Externa (KED)}
\NormalTok{ked }\OtherTok{\textless{}{-}} \FunctionTok{krige}\NormalTok{(}\FunctionTok{log}\NormalTok{(zinc) }\SpecialCharTok{\textasciitilde{}}\NormalTok{ dist, }
             \AttributeTok{locations =}\NormalTok{ meuse\_sf, }
             \AttributeTok{newdata =}\NormalTok{ grid\_stars, }
             \AttributeTok{model =}\NormalTok{ m\_ked, }\AttributeTok{debug.level =} \DecValTok{0}\NormalTok{)}

\CommentTok{\#}
\CommentTok{\# Predição}
\NormalTok{p1 }\OtherTok{\textless{}{-}} \FunctionTok{ggplot}\NormalTok{() }\SpecialCharTok{+}
  \FunctionTok{geom\_stars}\NormalTok{(}\AttributeTok{data =}\NormalTok{ ked, }\FunctionTok{aes}\NormalTok{(}\AttributeTok{fill =}\NormalTok{ var1.pred)) }\SpecialCharTok{+}
  \FunctionTok{scale\_fill\_viridis\_c}\NormalTok{(}\AttributeTok{option =} \StringTok{"plasma"}\NormalTok{, }\AttributeTok{na.value =} \StringTok{"transparent"}\NormalTok{) }\SpecialCharTok{+}
  \FunctionTok{labs}\NormalTok{(}\AttributeTok{title =} \StringTok{"KED Predição (Covariável: Dist)"}\NormalTok{, }
       \AttributeTok{subtitle =} \StringTok{"Tendência guiada pela distância ao rio"}\NormalTok{,}
       \AttributeTok{fill =} \StringTok{"Log(Zn)"}\NormalTok{, }\AttributeTok{x =} \ConstantTok{NULL}\NormalTok{, }\AttributeTok{y =} \ConstantTok{NULL}\NormalTok{) }\SpecialCharTok{+}
  \FunctionTok{theme\_void}\NormalTok{() }\SpecialCharTok{+}
  \FunctionTok{theme}\NormalTok{(}\AttributeTok{plot.title =} \FunctionTok{element\_text}\NormalTok{(}\AttributeTok{hjust =} \FloatTok{0.5}\NormalTok{), }\AttributeTok{plot.subtitle =} \FunctionTok{element\_text}\NormalTok{(}\AttributeTok{size =} \DecValTok{9}\NormalTok{))}

\CommentTok{\# Variância (Erro)}
\NormalTok{p2 }\OtherTok{\textless{}{-}} \FunctionTok{ggplot}\NormalTok{() }\SpecialCharTok{+}
  \FunctionTok{geom\_stars}\NormalTok{(}\AttributeTok{data =}\NormalTok{ ked, }\FunctionTok{aes}\NormalTok{(}\AttributeTok{fill =}\NormalTok{ var1.var)) }\SpecialCharTok{+}
  \FunctionTok{scale\_fill\_viridis\_c}\NormalTok{(}\AttributeTok{option =} \StringTok{"cividis"}\NormalTok{, }\AttributeTok{na.value =} \StringTok{"transparent"}\NormalTok{) }\SpecialCharTok{+}
  \FunctionTok{labs}\NormalTok{(}\AttributeTok{title =} \StringTok{"Variância KED"}\NormalTok{, }\AttributeTok{fill =} \StringTok{"Var"}\NormalTok{, }\AttributeTok{x =} \ConstantTok{NULL}\NormalTok{, }\AttributeTok{y =} \ConstantTok{NULL}\NormalTok{) }\SpecialCharTok{+}
  \FunctionTok{theme\_void}\NormalTok{() }\SpecialCharTok{+}
  \FunctionTok{theme}\NormalTok{(}\AttributeTok{plot.title =} \FunctionTok{element\_text}\NormalTok{(}\AttributeTok{hjust =} \FloatTok{0.5}\NormalTok{))}

\NormalTok{p1 }\SpecialCharTok{+}\NormalTok{ p2}
\end{Highlighting}
\end{Shaded}

\begin{figure}[H]

\centering{

\pandocbounded{\includegraphics[keepaspectratio]{geostat_files/figure-pdf/fig-krigagem_KDE-1.pdf}}

}

\caption{\label{fig-krigagem_KDE-1}Krigagem com deriva externa: (a)
predição e (b) variância de krigagem}

\end{figure}%

\begin{figure}[H]

\centering{

\pandocbounded{\includegraphics[keepaspectratio]{geostat_files/figure-pdf/fig-krigagem_KDE-2.pdf}}

}

\caption{\label{fig-krigagem_KDE-2}Krigagem com deriva externa: (a)
predição e (b) variância de krigagem}

\end{figure}%

\subsection{Co-Krigagem}\label{co-krigagem}

A Co-Krigagem (CK) é a extensão multivariada da krigagem. Utiliza-se
quando queremos estimar uma variável primária \(Y_1(\mathbf{s}_0)\)
utilizando a correlação espacial não só com as suas próprias amostras,
mas também com amostras de uma ou mais variáveis secundárias
\(Y_2(\mathbf{s}), \dots, Y_k(\mathbf{s})\), que estão correlacionadas
espacialmente com \(Y_1\) (co-regionalização) (Ver Hoef e Barry 1998;
Wackernagel 2003). Assim, temos:

\[
\begin{aligned}
&Y_1(\mathbf{s}) = \mu_1 + \delta_1(\mathbf{s}), \quad E[\delta_1(\mathbf{s})] = 0 \\
&Y_2(\mathbf{s}) = \mu_2 + \delta_2(\mathbf{s}), \quad E[\delta_2(\mathbf{s})] = 0 \\
&\text{Cov}(\delta_a(\mathbf{s}_i), \delta_b(\mathbf{s}_j)) = C_{ab}(\mathbf{s}_i, \mathbf{s}_j), \quad a,b = 1,2\\
&\hat{Y}_1(\mathbf{s}_0) = \sum_{i=1}^{n_1} \lambda_{1i} Y_1(\mathbf{s}_{1i}) + \sum_{j=1}^{n_2} \lambda_{2j} Y_2(\mathbf{s}_{2j})
\end{aligned}
\]

\textbf{Condição de não enviesamento:}

\[
\begin{aligned}
E[\hat{Y}_1(\mathbf{s}_0) - Y_1(\mathbf{s}_0)] &= \sum_{i=1}^{n_1} \lambda_{1i} \mu_1 + \sum_{j=1}^{n_2} \lambda_{2j} \mu_2 - \mu_1 \\
&= \mu_1\left(\sum_{i=1}^{n_1} \lambda_{1i} - 1\right) + \mu_2 \sum_{j=1}^{n_2} \lambda_{2j} \overset{!}{=} 0
\end{aligned}
\] Como sempre para que valha para quaisquer \(\mu_1, \mu_2\)
desconhecidos, definimos:

\[
\sum_{i=1}^{n_1} \lambda_{1i} = 1 \quad \text{e} \quad \sum_{j=1}^{n_2} \lambda_{2j} = 0
\]

\textbf{Variância do erro:}

\[
\begin{aligned}
\sigma_E^2 &= \text{Var}\left(\hat{Y}_1(\mathbf{s}_0) - Y_1(\mathbf{s}_0)\right) \\
&= \text{Var}\left(\sum_{i=1}^{n_1} \lambda_{1i} \delta_1(\mathbf{s}_{1i}) + \sum_{j=1}^{n_2} \lambda_{2j} \delta_2(\mathbf{s}_{2j}) - \delta_1(\mathbf{s}_0)\right) \\
&= \sum_{i=1}^{n_1}\sum_{k=1}^{n_1} \lambda_{1i}\lambda_{1k} C_{11}(\mathbf{s}_{1i}, \mathbf{s}_{1k}) \\
&\quad + \sum_{j=1}^{n_2}\sum_{l=1}^{n_2} \lambda_{2j}\lambda_{2l} C_{22}(\mathbf{s}_{2j}, \mathbf{s}_{2l}) \\
&\quad + 2\sum_{i=1}^{n_1}\sum_{j=1}^{n_2} \lambda_{1i}\lambda_{2j} C_{12}(\mathbf{s}_{1i}, \mathbf{s}_{2j}) \\
&\quad - 2\sum_{i=1}^{n_1} \lambda_{1i} C_{11}(\mathbf{s}_{1i}, \mathbf{s}_0) - 2\sum_{j=1}^{n_2} \lambda_{2j} C_{12}(\mathbf{s}_{2j}, \mathbf{s}_0) \\
&\quad + C_{11}(\mathbf{s}_0, \mathbf{s}_0)
\end{aligned}
\] Definindo o Langrangiano temos:

\[
\begin{aligned}
L(\boldsymbol{\lambda}_1, \boldsymbol{\lambda}_2, \nu_1, \nu_2) &= \sigma_E^2 - 2\nu_1\left(\sum_{i=1}^{n_1} \lambda_{1i} - 1\right) - 2\nu_2\sum_{j=1}^{n_2} \lambda_{2j}
\end{aligned}
\] cujas derivadas são:

\[
\begin{aligned}
\frac{\partial L}{\partial \lambda_{1i}} = 2\sum_{k=1}^{n_1} \lambda_{1k} C_{11}(\mathbf{s}_{1i}, \mathbf{s}_{1k}) + 2\sum_{j=1}^{n_2} \lambda_{2j} C_{12}(\mathbf{s}_{1i}, \mathbf{s}_{2j}) - 2C_{11}(\mathbf{s}_{1i}, \mathbf{s}_0) - 2\nu_1 &= 0\\
\sum_{k=1}^{n_1} \lambda_{1k} C_{11}(\mathbf{s}_{1i}, \mathbf{s}_{1k}) + \sum_{j=1}^{n_2} \lambda_{2j} C_{12}(\mathbf{s}_{1i}, \mathbf{s}_{2j}) - \nu_1 &= C_{11}(\mathbf{s}_{1i}, \mathbf{s}_0)
\end{aligned}
\]

\[
\begin{aligned}
\frac{\partial L}{\partial \lambda_{2j}} = 2\sum_{l=1}^{n_2} \lambda_{2l} C_{22}(\mathbf{s}_{2j}, \mathbf{s}_{2l}) + 2\sum_{i=1}^{n_1} \lambda_{1i} C_{12}(\mathbf{s}_{1i}, \mathbf{s}_{2j}) - 2C_{12}(\mathbf{s}_{2j}, \mathbf{s}_0) - 2\nu_2 &= 0\\
\sum_{l=1}^{n_2} \lambda_{2l} C_{22}(\mathbf{s}_{2j}, \mathbf{s}_{2l}) + \sum_{i=1}^{n_1} \lambda_{1i} C_{12}(\mathbf{s}_{1i}, \mathbf{s}_{2j}) - \nu_2 &= C_{12}(\mathbf{s}_{2j}, \mathbf{s}_0)
\end{aligned}
\] \[
\begin{aligned}
\frac{\partial L}{\partial \nu_1} &= -2\left(\sum_{i=1}^{n_1} \lambda_{1i} - 1\right) = 0 \quad \Rightarrow \quad\sum_{i=1}^{n_1} \lambda_{1i} = 1 \\
\frac{\partial L}{\partial \nu_2} &= -2\sum_{j=1}^{n_2} \lambda_{2j} = 0 \quad \Rightarrow \quad \sum_{j=1}^{n_2} \lambda_{2j} = 0
\end{aligned}
\]

Novamente, definimos:

\[
\begin{aligned}
&\mathbf{C}_{11} = [C_{11}(\mathbf{s}_{1i}, \mathbf{s}_{1k})]_{n_1 \times n_1}, \quad
\mathbf{C}_{22} = [C_{22}(\mathbf{s}_{2j}, \mathbf{s}_{2l})]_{n_2 \times n_2} \\
&\mathbf{C}_{12} = [C_{12}(\mathbf{s}_{1i}, \mathbf{s}_{2j})]_{n_1 \times n_2}, \quad
\mathbf{C}_{21} = \mathbf{C}_{12}^\top \\
&\mathbf{c}_{10} = [C_{11}(\mathbf{s}_{1i}, \mathbf{s}_0)]_{n_1 \times 1}, \quad \quad 
\mathbf{c}_{20} = [C_{12}(\mathbf{s}_{2j}, \mathbf{s}_0)]_{n_2 \times 1} \\
&\mathbf{1}_{n_1} = [1, \dots, 1]^\top_{n_1 \times 1}, \quad \quad \quad \:
\mathbf{1}_{n_2} = [1, \dots, 1]^\top_{n_2 \times 1} \\
&\boldsymbol{\lambda}_1 = [\lambda_{11}, \dots, \lambda_{1n_1}]^\top, \quad \quad \:\:
\boldsymbol{\lambda}_2 = [\lambda_{21}, \dots, \lambda_{2n_2}]^\top
\end{aligned}
\] para obter o seguinte sistema:

\[
\begin{bmatrix}
\mathbf{C}_{11} & \mathbf{C}_{12} & -\mathbf{1}_{n_1} & \mathbf{0}_{n_1} \\
\mathbf{C}_{21} & \mathbf{C}_{22} & \mathbf{0}_{n_2} & -\mathbf{1}_{n_2} \\
\mathbf{1}_{n_1}^\top & \mathbf{0}_{n_2}^\top & 0 & 0 \\
\mathbf{0}_{n_1}^\top & \mathbf{1}_{n_2}^\top & 0 & 0
\end{bmatrix}
\begin{bmatrix}
\boldsymbol{\lambda}_1 \\ \boldsymbol{\lambda}_2 \\ \nu_1 \\ \nu_2
\end{bmatrix}
=
\begin{bmatrix}
\mathbf{c}_{10} \\ \mathbf{c}_{20} \\ 1 \\ 0
\end{bmatrix}
\] \textbf{Variância mínima da co-krigagem:}

\[
\sigma_{\text{CK}}^2 = C_{11}(\mathbf{s}_0, \mathbf{s}_0) - \left(\sum_{i=1}^{n_1} \lambda_{1i} C_{11}(\mathbf{s}_{1i}, \mathbf{s}_0) + \sum_{j=1}^{n_2} \lambda_{2j} C_{12}(\mathbf{s}_{2j}, \mathbf{s}_0)\right) + \nu_1
\]

\begin{tcolorbox}[enhanced jigsaw, left=2mm, toptitle=1mm, colback=white, colframe=quarto-callout-important-color-frame, colbacktitle=quarto-callout-important-color!10!white, opacityback=0, rightrule=.15mm, bottomtitle=1mm, arc=.35mm, title=\textcolor{quarto-callout-important-color}{\faExclamation}\hspace{0.5em}{Importante}, titlerule=0mm, bottomrule=.15mm, leftrule=.75mm, coltitle=black, toprule=.15mm, breakable, opacitybacktitle=0.6]

\begin{itemize}
\tightlist
\item
  Para garantir que a matriz de covariâncias cruzadas seja positiva
  definida, as covariâncias devem ser modeladas como:
\end{itemize}

\[
C_{ab}(\mathbf{h}) = \sum_{k=1}^K B_{ab}^k \rho_k(\mathbf{h})
\] onde \(\rho_k(\mathbf{h})\) são funções de correlação básicas e
\([B_{ab}^k]\) são matrizes de coregionalização definidas não-negativas.

\begin{itemize}
\item
  Como descrito nas seções anteriores em geral,
  \(C_{12}(\mathbf{h}) = C_{21}(-\mathbf{h})\). Para processos
  estacionários, \(C_{12}(\mathbf{h}) = C_{21}(\mathbf{h})\) se a
  covariância cruzada é simétrica.
\item
  Quando as variáveis secundárias estão disponíveis apenas nos mesmos
  locais que a primária (ou em grades finas), simplificações são
  possíveis.
\item
  A abordagem condicional de Noel Cressie e Zammit-Mangion (2016)
  oferece uma alternativa flexível para construir modelos de covariância
  multivariados válidos e assimétricos. Noel Cressie e Zammit-Mangion
  (2016) modela a distribuição conjunta condicionando \(Y_1\) a \(Y_2\),
  evitando a necessidade de modelar explicitamente a covariância
  cruzada:
\end{itemize}

\[Y_1(\mathbf{s}) = \mu_1(\mathbf{s}) + \beta(\mathbf{s})[Y_2(\mathbf{s}) - \mu_2(\mathbf{s})] + \delta(\mathbf{s})\]
onde \(\beta(\mathbf{s})\) é um coeficiente espacialmente variante.

\end{tcolorbox}

\begin{Shaded}
\begin{Highlighting}[]
\NormalTok{pacman}\SpecialCharTok{::}\FunctionTok{p\_load}\NormalTok{(stars, gstat, ggplot2, sf, patchwork, viridis)}

\FunctionTok{data}\NormalTok{(meuse)}
\FunctionTok{data}\NormalTok{(meuse.grid)}


\NormalTok{meuse\_sf }\OtherTok{\textless{}{-}} \FunctionTok{st\_as\_sf}\NormalTok{(meuse, }\AttributeTok{coords =} \FunctionTok{c}\NormalTok{(}\StringTok{"x"}\NormalTok{, }\StringTok{"y"}\NormalTok{), }\AttributeTok{crs =} \DecValTok{28992}\NormalTok{)}
\NormalTok{grid\_sf }\OtherTok{\textless{}{-}} \FunctionTok{st\_as\_sf}\NormalTok{(meuse.grid, }\AttributeTok{coords =} \FunctionTok{c}\NormalTok{(}\StringTok{"x"}\NormalTok{, }\StringTok{"y"}\NormalTok{), }\AttributeTok{crs =} \DecValTok{28992}\NormalTok{)}
\NormalTok{grid\_stars }\OtherTok{\textless{}{-}} \FunctionTok{st\_rasterize}\NormalTok{(grid\_sf, }\AttributeTok{dx =} \DecValTok{40}\NormalTok{, }\AttributeTok{dy =} \DecValTok{40}\NormalTok{)}

\CommentTok{\# Adicionamos as variáveis uma por uma ao objeto gstat}

\CommentTok{\# Variável 1 (Primária): Zinco}
\NormalTok{g }\OtherTok{\textless{}{-}} \FunctionTok{gstat}\NormalTok{(}\ConstantTok{NULL}\NormalTok{, }\AttributeTok{id =} \StringTok{"zinc"}\NormalTok{, }\AttributeTok{formula =} \FunctionTok{log}\NormalTok{(zinc) }\SpecialCharTok{\textasciitilde{}} \DecValTok{1}\NormalTok{, }\AttributeTok{data =}\NormalTok{ meuse\_sf); g}
\CommentTok{\# Variável 2 (Secundária): Chumbo (Lead)}
\NormalTok{g1 }\OtherTok{\textless{}{-}} \FunctionTok{gstat}\NormalTok{(g, }\AttributeTok{id =} \StringTok{"lead"}\NormalTok{, }\AttributeTok{formula =} \FunctionTok{log}\NormalTok{(lead) }\SpecialCharTok{\textasciitilde{}} \DecValTok{1}\NormalTok{, }\AttributeTok{data =}\NormalTok{ meuse\_sf);g1}

\CommentTok{\#Variograma Cruzado}
\CommentTok{\# O gstat calcula automaticamente: Var(Zn), Var(Pb) e Cov(Zn, Pb)}
\NormalTok{v\_cross }\OtherTok{\textless{}{-}} \FunctionTok{variogram}\NormalTok{(g1)}

\CommentTok{\# Plotar os variogramas (Diretos e Cruzado) para inspeção}
\FunctionTok{plot}\NormalTok{(v\_cross, }\AttributeTok{main =} \StringTok{"Variogramas Cruzados: Zinco x Chumbo"}\NormalTok{)}

\CommentTok{\#Ajuste do Modelo Linear de Coregionalização (LMC)}
\CommentTok{\# fit.lmc ajusta o modelo aos variogramas direto e cruzado simultaneamente}
\NormalTok{model\_base }\OtherTok{\textless{}{-}} \FunctionTok{vgm}\NormalTok{(}\FloatTok{0.6}\NormalTok{, }\StringTok{"Exp"}\NormalTok{, }\DecValTok{800}\NormalTok{, }\FloatTok{0.05}\NormalTok{)}
\NormalTok{m\_cross }\OtherTok{\textless{}{-}} \FunctionTok{fit.lmc}\NormalTok{(v\_cross, g, }\AttributeTok{model =}\NormalTok{ model\_base)}


\FunctionTok{plot}\NormalTok{(v\_cross, m\_cross, }\AttributeTok{main =} \StringTok{"Ajuste do LMC (Zn + Pb)"}\NormalTok{)}

\CommentTok{\#Realizar a Co{-}Krigagem}
\NormalTok{ck }\OtherTok{\textless{}{-}} \FunctionTok{predict}\NormalTok{(m\_cross, }\AttributeTok{newdata =}\NormalTok{ grid\_stars)}

\CommentTok{\# Predição}
\NormalTok{p1 }\OtherTok{\textless{}{-}} \FunctionTok{ggplot}\NormalTok{() }\SpecialCharTok{+}
  \FunctionTok{geom\_stars}\NormalTok{(}\AttributeTok{data =}\NormalTok{ ck, }\FunctionTok{aes}\NormalTok{(}\AttributeTok{fill =}\NormalTok{ zinc.pred)) }\SpecialCharTok{+}
  \FunctionTok{scale\_fill\_viridis\_c}\NormalTok{(}\AttributeTok{option =} \StringTok{"plasma"}\NormalTok{, }\AttributeTok{na.value =} \StringTok{"transparent"}\NormalTok{) }\SpecialCharTok{+}
  \FunctionTok{labs}\NormalTok{(}\AttributeTok{title =} \StringTok{"Co{-}Krigagem Ordinária (Pred)"}\NormalTok{, }
       \AttributeTok{subtitle =} \StringTok{"Zinco auxiliado por Chumbo"}\NormalTok{,}
       \AttributeTok{fill =} \StringTok{"Log(Zn)"}\NormalTok{, }\AttributeTok{x =} \ConstantTok{NULL}\NormalTok{, }\AttributeTok{y =} \ConstantTok{NULL}\NormalTok{) }\SpecialCharTok{+}
  \FunctionTok{theme\_void}\NormalTok{() }\SpecialCharTok{+}
  \FunctionTok{theme}\NormalTok{(}\AttributeTok{plot.title =} \FunctionTok{element\_text}\NormalTok{(}\AttributeTok{hjust =} \FloatTok{0.5}\NormalTok{), }\AttributeTok{plot.subtitle =} \FunctionTok{element\_text}\NormalTok{(}\AttributeTok{size =} \DecValTok{9}\NormalTok{))}

\CommentTok{\# Variância}
\NormalTok{p2 }\OtherTok{\textless{}{-}} \FunctionTok{ggplot}\NormalTok{() }\SpecialCharTok{+}
  \FunctionTok{geom\_stars}\NormalTok{(}\AttributeTok{data =}\NormalTok{ ck, }\FunctionTok{aes}\NormalTok{(}\AttributeTok{fill =}\NormalTok{ zinc.var)) }\SpecialCharTok{+}
  \FunctionTok{scale\_fill\_viridis\_c}\NormalTok{(}\AttributeTok{option =} \StringTok{"cividis"}\NormalTok{, }\AttributeTok{na.value =} \StringTok{"transparent"}\NormalTok{) }\SpecialCharTok{+}
  \FunctionTok{labs}\NormalTok{(}\AttributeTok{title =} \StringTok{"Variância CK"}\NormalTok{, }\AttributeTok{fill =} \StringTok{"Var"}\NormalTok{, }\AttributeTok{x =} \ConstantTok{NULL}\NormalTok{, }\AttributeTok{y =} \ConstantTok{NULL}\NormalTok{) }\SpecialCharTok{+}
  \FunctionTok{theme\_void}\NormalTok{() }\SpecialCharTok{+}
  \FunctionTok{theme}\NormalTok{(}\AttributeTok{plot.title =} \FunctionTok{element\_text}\NormalTok{(}\AttributeTok{hjust =} \FloatTok{0.5}\NormalTok{))}

\NormalTok{p1 }\SpecialCharTok{+}\NormalTok{ p2}
\end{Highlighting}
\end{Shaded}

\begin{figure}[H]

\centering{

\pandocbounded{\includegraphics[keepaspectratio]{geostat_files/figure-pdf/fig-krigagem_co-1.pdf}}

}

\caption{\label{fig-krigagem_co-1}Co-Krigagem: (a) predição e (b)
variância de krigagem}

\end{figure}%

\begin{figure}[H]

\centering{

\pandocbounded{\includegraphics[keepaspectratio]{geostat_files/figure-pdf/fig-krigagem_co-2.pdf}}

}

\caption{\label{fig-krigagem_co-2}Co-Krigagem: (a) predição e (b)
variância de krigagem}

\end{figure}%

\begin{figure}[H]

\centering{

\pandocbounded{\includegraphics[keepaspectratio]{geostat_files/figure-pdf/fig-krigagem_co-3.pdf}}

}

\caption{\label{fig-krigagem_co-3}Co-Krigagem: (a) predição e (b)
variância de krigagem}

\end{figure}%

\section{Avaliação da Performance
Preditiva}\label{avaliauxe7uxe3o-da-performance-preditiva}

Uma vez ajustado o modelo de variograma e realizada a krigagem, é
fundamental verificar se as predições são precisas e se a incerteza
(variância de krigagem) está bem calibrada. Para tal, recorre-se a
métodos de reamostragem e estatísticas de consistência, sendo a
validação cruzada (leave-one-out) a mais usada.

\subsection{Validação Cruzada}\label{validauxe7uxe3o-cruzada}

A validação cruzada
\href{https://en.wikipedia.org/wiki/Cross-validation_statistics}{leave-one-out}
é o método padrão para verificar a consistência entre a incerteza
predita pelo modelo (variância de krigagem) e o erro real de predição. A
estatística para este diagnóstico é a Razão de Desvio Quadrático Médio
(MSDR - Mean Squared Deviation Ratio), discutida em profundidade por
McBratney e Webster (1986) e Lark (2000).

Seja \(Y(\mathbf{s}_i)\) o valor observado no local \(\mathbf{s}_i\).
Retiramos este ponto do conjunto de dados e realizamos a krigagem usando
os \(n-1\) vizinhos restantes para obter a estimativa
\(\hat{Y}_{-i}(\mathbf{s}_i)\) e a variância de krigagem associada
\(\sigma_E^2(\mathbf{s}_i)\).

\textbf{Erro de predição}

Definimos o erro de predição (resíduo) como:
\[\varepsilon_i = Y(\mathbf{s}_i) - \hat{Y}_{-i}(\mathbf{s}_i)\]

Sob a hipótese nula de que o modelo de variograma
\(\gamma(\mathbf{h}; \boldsymbol{\theta})\) e as premissas de
estacionariedade estão corretos, o estimador de krigagem é não-viesado e
a variância do erro é exata. Se assumirmos adicionalmente que o processo
é Gaussiano:

\[\varepsilon_i \sim \mathcal{N}(0, \sigma_E^2(\mathbf{s}_i))\]

Para avaliar a validade da variância, padronizamos o erro dividindo-o
pelo desvio padrão estimado (variância de krigagem). Definimos o resíduo
padronizado \(\theta_i\):

\[\theta_i = \frac{Y(\mathbf{s}_i) - \hat{Y}_{-i}(\mathbf{s}_i)}{\sigma_k(\mathbf{s}_i)}\]

Dada a distribuição de \(\varepsilon_i\), a variável \(\theta_i\) deve
seguir uma distribuição normal padrão,
\(\theta_i \sim \mathcal{N}(0, 1)\). Consequentemente, o quadrado do
resíduo padronizado segue uma distribuição Qui-quadrado com 1 grau de
liberdade:

\[\theta_i^2 \sim \chi^2_{(1)}\]

A esperança de uma variável \(\chi^2_{(1)}\) é igual aos seus graus de
liberdade. Portanto, o valor esperado teórico para o quadrado do erro
padronizado é:

\[E[\theta_i^2] = \text{Var}(\theta_i) + (E[\theta_i])^2 = 1 + 0 = 1\]

A MSDR é definida como a média amostral dos quadrados dos resíduos
padronizados sobre todos os pontos de validação \(n\):

\[\text{MSDR} = \frac{1}{n} \sum_{i=1}^n \theta_i^2 = \frac{1}{n} \sum_{i=1}^n \frac{[Y(\mathbf{s}_i) - \hat{Y}_{-i}(\mathbf{s}_i)]^2}{\sigma_E^2(\mathbf{s}_i)}\]

A análise do valor de MSDR permite diagnosticar a calibração da
incerteza do modelo:

\begin{enumerate}
\def\labelenumi{\arabic{enumi}.}
\tightlist
\item
  \textbf{Consistência (}\(\text{MSDR} \approx 1\)):
\end{enumerate}

A variância média dos erros reais (numerador) é aproximadamente igual à
variância média predita pela krigagem (denominador). O modelo descreve
corretamente a variabilidade espacial.

\begin{enumerate}
\def\labelenumi{\arabic{enumi}.}
\setcounter{enumi}{1}
\tightlist
\item
  \textbf{Subestimação da Incerteza (}\(\text{MSDR} > 1\)):
\end{enumerate}

\[\frac{1}{n}\sum \varepsilon_i^2 > \frac{1}{n}\sum \sigma_E^2\]

O erro real é sistematicamente maior do que o modelo prevê. Isso ocorre
frequentemente quando o variograma subestima o efeito pepita ou a
variância total (patamar), ou quando há outliers não modelados.

\begin{enumerate}
\def\labelenumi{\arabic{enumi}.}
\setcounter{enumi}{2}
\tightlist
\item
  \textbf{Superestimação da Incerteza (}\(\text{MSDR} < 1\)):
\end{enumerate}

\[\frac{1}{n}\sum \varepsilon_i^2 < \frac{1}{n}\sum \sigma_E^2\]

A variância de krigagem é maior do que o erro efetivo. O modelo assume
uma variabilidade espacial ou um ruído (pepita) superior ao que
realmente existe nos dados. O modelo é ``pessimista'' ou conservador.

Com base na fundamentação teórica de Journel \& Huijbregts (1978),
Cressie (1993), Goovaerts (1997) e Chilès \& Delfiner (1999), apresento
a secção detalhada sobre Anisotropia, mantendo o rigor matemático e a
dedução das transformações lineares necessárias para a modelagem.

\begin{Shaded}
\begin{Highlighting}[]
\NormalTok{pacman}\SpecialCharTok{::}\FunctionTok{p\_load}\NormalTok{(stars, gstat, ggplot2, sf, patchwork, viridis, dplyr, tibble, gt)}

\FunctionTok{data}\NormalTok{(meuse)}
\NormalTok{meuse\_sf }\OtherTok{\textless{}{-}} \FunctionTok{st\_as\_sf}\NormalTok{(meuse, }\AttributeTok{coords =} \FunctionTok{c}\NormalTok{(}\StringTok{"x"}\NormalTok{, }\StringTok{"y"}\NormalTok{), }\AttributeTok{crs =} \DecValTok{28992}\NormalTok{)}

\NormalTok{v\_ord }\OtherTok{\textless{}{-}} \FunctionTok{variogram}\NormalTok{(}\FunctionTok{log}\NormalTok{(zinc) }\SpecialCharTok{\textasciitilde{}} \DecValTok{1}\NormalTok{, meuse\_sf)}
\NormalTok{m\_ord }\OtherTok{\textless{}{-}} \FunctionTok{fit.variogram}\NormalTok{(v\_ord, }\FunctionTok{vgm}\NormalTok{(}\FloatTok{0.6}\NormalTok{, }\StringTok{"Exp"}\NormalTok{, }\DecValTok{800}\NormalTok{, }\FloatTok{0.05}\NormalTok{))}

\CommentTok{\#Executar Validação Cruzada (Leave{-}One{-}Out)}
\CommentTok{\# A função krige.cv faz o loop automaticamente.}
\NormalTok{cv\_results }\OtherTok{\textless{}{-}} \FunctionTok{krige.cv}\NormalTok{(}\FunctionTok{log}\NormalTok{(zinc) }\SpecialCharTok{\textasciitilde{}} \DecValTok{1}\NormalTok{, }
                       \AttributeTok{locations =}\NormalTok{ meuse\_sf, }
                       \AttributeTok{model =}\NormalTok{ m\_ord, }\AttributeTok{debug.level =} \DecValTok{0}\NormalTok{); }

\NormalTok{cv\_results}\SpecialCharTok{|\textgreater{}}
  \FunctionTok{head}\NormalTok{()}\SpecialCharTok{|\textgreater{}}
\NormalTok{  knitr}\SpecialCharTok{::}\FunctionTok{kable}\NormalTok{()}
\end{Highlighting}
\end{Shaded}

\begin{longtable}[]{@{}
  >{\raggedleft\arraybackslash}p{(\linewidth - 12\tabcolsep) * \real{0.1282}}
  >{\raggedleft\arraybackslash}p{(\linewidth - 12\tabcolsep) * \real{0.1282}}
  >{\raggedleft\arraybackslash}p{(\linewidth - 12\tabcolsep) * \real{0.1154}}
  >{\raggedleft\arraybackslash}p{(\linewidth - 12\tabcolsep) * \real{0.1410}}
  >{\raggedleft\arraybackslash}p{(\linewidth - 12\tabcolsep) * \real{0.1410}}
  >{\raggedleft\arraybackslash}p{(\linewidth - 12\tabcolsep) * \real{0.0641}}
  >{\raggedright\arraybackslash}p{(\linewidth - 12\tabcolsep) * \real{0.2821}}@{}}
\toprule\noalign{}
\begin{minipage}[b]{\linewidth}\raggedleft
var1.pred
\end{minipage} & \begin{minipage}[b]{\linewidth}\raggedleft
var1.var
\end{minipage} & \begin{minipage}[b]{\linewidth}\raggedleft
observed
\end{minipage} & \begin{minipage}[b]{\linewidth}\raggedleft
residual
\end{minipage} & \begin{minipage}[b]{\linewidth}\raggedleft
zscore
\end{minipage} & \begin{minipage}[b]{\linewidth}\raggedleft
fold
\end{minipage} & \begin{minipage}[b]{\linewidth}\raggedright
geometry
\end{minipage} \\
\midrule\noalign{}
\endhead
\bottomrule\noalign{}
\endlastfoot
6.833605 & 0.1614145 & 6.929517 & 0.0959118 & 0.2387267 & 1 & POINT
(181072 333611) \\
6.786747 & 0.1604868 & 7.039660 & 0.2529135 & 0.6313240 & 2 & POINT
(181025 333558) \\
6.291128 & 0.1836067 & 6.461468 & 0.1703405 & 0.3975334 & 3 & POINT
(181165 333537) \\
6.051561 & 0.2405948 & 5.549076 & -0.5024847 & -1.0244238 & 4 & POINT
(181298 333484) \\
5.576407 & 0.1717014 & 5.594711 & 0.0183041 & 0.0441734 & 5 & POINT
(181307 333330) \\
5.455516 & 0.2381790 & 5.638355 & 0.1828387 & 0.3746419 & 6 & POINT
(181390 333260) \\
\end{longtable}

\begin{Shaded}
\begin{Highlighting}[]
\CommentTok{\#Estatísticas de Diagnóstico}

\CommentTok{\# MSDR (Razão de Desvio Quadrático Médio)}
\NormalTok{msdr }\OtherTok{\textless{}{-}} \FunctionTok{mean}\NormalTok{(cv\_results}\SpecialCharTok{$}\NormalTok{zscore}\SpecialCharTok{\^{}}\DecValTok{2}\NormalTok{)}

\CommentTok{\# ME (Erro Médio) {-} Deve ser próximo de 0 (viés)}
\NormalTok{me }\OtherTok{\textless{}{-}} \FunctionTok{mean}\NormalTok{(cv\_results}\SpecialCharTok{$}\NormalTok{residual)}

\CommentTok{\# RMSE (Raiz do Erro Quadrático Médio)}
\NormalTok{rmse }\OtherTok{\textless{}{-}} \FunctionTok{sqrt}\NormalTok{(}\FunctionTok{mean}\NormalTok{(cv\_results}\SpecialCharTok{$}\NormalTok{residual}\SpecialCharTok{\^{}}\DecValTok{2}\NormalTok{))}

\NormalTok{resultados }\OtherTok{\textless{}{-}} \FunctionTok{tibble}\NormalTok{(}
  \AttributeTok{Indicador =} \FunctionTok{c}\NormalTok{(}\StringTok{"Mean Error (Viés)"}\NormalTok{, }
                \StringTok{"RMSE (Precisão)"}\NormalTok{, }
                \StringTok{"MSDR (Calibração)"}\NormalTok{),}
  \AttributeTok{Valor =} \FunctionTok{round}\NormalTok{(}\FunctionTok{c}\NormalTok{(me, rmse, msdr), }\DecValTok{4}\NormalTok{)}
\NormalTok{)}

\NormalTok{resultados}\SpecialCharTok{|\textgreater{}}
\NormalTok{  knitr}\SpecialCharTok{::}\FunctionTok{kable}\NormalTok{()}
\end{Highlighting}
\end{Shaded}

\begin{longtable}[]{@{}lr@{}}
\toprule\noalign{}
Indicador & Valor \\
\midrule\noalign{}
\endhead
\bottomrule\noalign{}
\endlastfoot
Mean Error (Viés) & 0.0021 \\
RMSE (Precisão) & 0.3935 \\
MSDR (Calibração) & 0.8657 \\
\end{longtable}

\begin{Shaded}
\begin{Highlighting}[]
\CommentTok{\# Interpretação automática simples}
\ControlFlowTok{if}\NormalTok{(msdr }\SpecialCharTok{\textgreater{}} \FloatTok{1.1}\NormalTok{) \{}
  \FunctionTok{cat}\NormalTok{(}\StringTok{"MSDR \textgreater{} 1: Subestimação da incerteza (Variograma muito \textquotesingle{}otimista\textquotesingle{} ou outliers).}\SpecialCharTok{\textbackslash{}n}\StringTok{"}\NormalTok{)}
\NormalTok{\} }\ControlFlowTok{else} \ControlFlowTok{if}\NormalTok{(msdr }\SpecialCharTok{\textless{}} \FloatTok{0.9}\NormalTok{) \{}
  \FunctionTok{cat}\NormalTok{(}\StringTok{"MSDR \textless{} 1: Superestimação da incerteza (Variograma muito \textquotesingle{}pessimista\textquotesingle{}).}\SpecialCharTok{\textbackslash{}n}\StringTok{"}\NormalTok{)}
\NormalTok{\} }\ControlFlowTok{else}\NormalTok{ \{}
  \FunctionTok{cat}\NormalTok{(}\StringTok{"MSDR \textasciitilde{} 1: Incerteza bem calibrada.}\SpecialCharTok{\textbackslash{}n}\StringTok{"}\NormalTok{)}
\NormalTok{\}}
\end{Highlighting}
\end{Shaded}

\begin{verbatim}
MSDR < 1: Superestimação da incerteza (Variograma muito 'pessimista').
\end{verbatim}

\begin{Shaded}
\begin{Highlighting}[]
\NormalTok{p1 }\OtherTok{\textless{}{-}} \FunctionTok{ggplot}\NormalTok{(cv\_results, }\FunctionTok{aes}\NormalTok{(}\AttributeTok{x =}\NormalTok{ observed, }\AttributeTok{y =}\NormalTok{ var1.pred)) }\SpecialCharTok{+}
  \FunctionTok{geom\_point}\NormalTok{(}\AttributeTok{alpha =} \FloatTok{0.5}\NormalTok{) }\SpecialCharTok{+}
  \FunctionTok{geom\_abline}\NormalTok{(}\AttributeTok{slope =} \DecValTok{1}\NormalTok{, }\AttributeTok{intercept =} \DecValTok{0}\NormalTok{, }\AttributeTok{color =} \StringTok{"red"}\NormalTok{, }\AttributeTok{linetype =} \StringTok{"dashed"}\NormalTok{) }\SpecialCharTok{+}
  \FunctionTok{labs}\NormalTok{(}\AttributeTok{title =} \StringTok{"Acurácia: Observado vs Predito"}\NormalTok{,}
       \AttributeTok{subtitle =} \FunctionTok{paste}\NormalTok{(}\StringTok{"RMSE:"}\NormalTok{, }\FunctionTok{round}\NormalTok{(rmse, }\DecValTok{3}\NormalTok{)),}
       \AttributeTok{x =} \StringTok{"Log(Zinc) Observado"}\NormalTok{, }\AttributeTok{y =} \StringTok{"Log(Zinc) Predito"}\NormalTok{) }\SpecialCharTok{+}
  \FunctionTok{theme\_bw}\NormalTok{()}

\CommentTok{\#Histograma dos Z{-}Scores (Deve parecer uma Normal(0,1))}
\NormalTok{p2 }\OtherTok{\textless{}{-}} \FunctionTok{ggplot}\NormalTok{(cv\_results, }\FunctionTok{aes}\NormalTok{(}\AttributeTok{x =}\NormalTok{ zscore)) }\SpecialCharTok{+}
  \FunctionTok{geom\_histogram}\NormalTok{(}\FunctionTok{aes}\NormalTok{(}\AttributeTok{y =}\NormalTok{ ..density..), }\AttributeTok{bins =} \DecValTok{20}\NormalTok{, }\AttributeTok{fill =} \StringTok{"steelblue"}\NormalTok{, }\AttributeTok{color =} \StringTok{"white"}\NormalTok{, }\AttributeTok{alpha =} \FloatTok{0.7}\NormalTok{) }\SpecialCharTok{+}
  \FunctionTok{stat\_function}\NormalTok{(}\AttributeTok{fun =}\NormalTok{ dnorm, }\AttributeTok{args =} \FunctionTok{list}\NormalTok{(}\AttributeTok{mean =} \DecValTok{0}\NormalTok{, }\AttributeTok{sd =} \DecValTok{1}\NormalTok{), }\AttributeTok{color =} \StringTok{"red"}\NormalTok{, }\AttributeTok{size =} \DecValTok{1}\NormalTok{) }\SpecialCharTok{+}
  \FunctionTok{labs}\NormalTok{(}\AttributeTok{title =} \StringTok{"Calibração: Z{-}Scores"}\NormalTok{,}
       \AttributeTok{subtitle =} \FunctionTok{paste}\NormalTok{(}\StringTok{"MSDR:"}\NormalTok{, }\FunctionTok{round}\NormalTok{(msdr, }\DecValTok{3}\NormalTok{), }\StringTok{"(Ideal = 1.0)"}\NormalTok{),}
       \AttributeTok{x =} \StringTok{"Resíduo Padronizado"}\NormalTok{, }\AttributeTok{y =} \StringTok{"Densidade"}\NormalTok{) }\SpecialCharTok{+}
  \FunctionTok{theme\_bw}\NormalTok{()}

\NormalTok{p1 }\SpecialCharTok{+}\NormalTok{ p2}
\end{Highlighting}
\end{Shaded}

\pandocbounded{\includegraphics[keepaspectratio]{geostat_files/figure-pdf/unnamed-chunk-1-1.pdf}}

Com base na sua solicitação, elaborei esta seção detalhada sobre
Krigagem Indicatriz. Mantive a notação rigorosa \(Y(\mathbf{s})\),
integrei a evolução histórica e matemática com base nos documentos
fornecidos (de 1983 a 2025) e diferenciei o método das abordagens
lineares anteriores.

\section{Krigagem Indicatriz (IK)}\label{krigagem-indicatriz-ik}

As técnicas de krigagem discutidas anteriormente (Simples, Ordinária,
Universal) são classificadas como estimadores lineares. Elas são ótimas
para estimar o valor esperado da variável \(Y(\mathbf{s}_0)\) sob a
condição de que a distribuição dos dados seja razoavelmente simétrica
(dados normais) e livre de outliers extremos. No entanto, em geociências
e estudos ambientais, frequentemente lidamos com distribuições altamente
assimétricas (ex: concentrações de ouro ou poluentes) onde a média não é
uma boa medida de tendência central e a variância de krigagem não
reflete a verdadeira incerteza local, pois é independente dos valores
dos dados (propriedade da homocedasticidade) (André G. Journel 1983).

A Krigagem Indicatriz (IK), introduzida por André G. Journel (1983),
representa uma mudança de paradigma: em vez de estimar o valor da
variável \(Y(\mathbf{s}_0)\) diretamente, estimamos a Função de
Distribuição Cumulativa Condicional (ccdf) local em \(\mathbf{s}_0\).

A importância fundamental da IK em relação à KO ou KS reside na sua
natureza não-paramétrica. Ela não assume normalidade dos dados e é
robusta a outliers, pois transforma os dados em binários (0 ou 1)
baseados em limiares de corte (\emph{cut-offs}). Um valor extremamente
alto não explode a estimativa, pois é tratado apenas como acima do corte
(Carvalho e Deutsch 2017).

\textbf{Transformada Indicadora e a ccdf}

Seja \(Y(\mathbf{s})\) uma variável regionalizada e \(y_k\) um valor de
corte (threshold) escolhido. A transformada indicadora
\(I(\mathbf{s}; y_k)\) é definida como uma variável binária:

\[
I(\mathbf{s}; z_k) = \begin{cases} 
1, & \text{se } Y(\mathbf{s}) \le y_k \\
0, & \text{se } Y(\mathbf{s}) > y_k
\end{cases}
\]

Para variáveis categóricas (como espécies biológicas ou litologia), a
definição muda para uma igualdade estrita (\(= y_k\)), conforme descrito
por Guimaraes et al. (2012).

A propriedade estatística que fundamenta a IK é que a esperança da
variável indicadora é igual à probabilidade cumulativa:

\[E[I(\mathbf{s}; y_k)] = 1 \cdot P(Y(\mathbf{s}) \le y_k) + 0 \cdot P(Y(\mathbf{s}) > y_k) = P(Y(\mathbf{s}) \le y_k) = F(y_k)\]

Portanto, ao estimar o valor esperado do indicador num local não
amostrado \(\mathbf{s}_0\), estamos estimando a probabilidade de que a
variável real seja menor ou igual ao corte \(y_k\), condicionada aos
dados vizinhos \((n)\). Esta é a ccdf local:

\[\hat{F}(u; y_k | (n)) = E^*[I(\mathbf{s}_0; y_k) | (n)] = \text{Prob}^* \{ Y(\mathbf{s}_0) \le y_k | (n) \}\]

\textbf{Estimador e o Sistema de Krigagem Indicatriz}

A estimativa da probabilidade para um corte \(y_k\) é uma combinação
linear dos indicadores observados nos locais \(\mathbf{s}_i\):

\[\hat{I}(\mathbf{s}_0; y_k) = \sum_{i=1}^{n} \lambda_i(y_k) I(\mathbf{s}_i; y_k)\]

Note que os pesos \(\lambda_i(y_k)\) dependem do corte \(y_k\). Isso
significa que a estrutura de continuidade espacial (variograma) pode
mudar para diferentes teores. O ouro de alto teor pode ter uma
continuidade espacial muito menor (alcance curto) do que o ouro de baixo
teor (alcance longo). Esta flexibilidade é uma vantagem crucial sobre a
Krigagem Ordinária, que assume um único variograma para todo o processo
(Mohammadpour et al. 2019).

O sistema de equações para obter os pesos é idêntico ao da Krigagem
Ordinária, mas aplicado aos dados transformados e ao Variograma
Indicador \(\gamma_I(\mathbf{h}; z_k)\):

\[\gamma_I(\mathbf{h}; y_k) = \frac{1}{2} E \left[ \{ I(\mathbf{s} + \mathbf{h}; y_k) - I(\mathbf{s}; y_k) \}^2 \right]\]

O sistema \(\mathbf{A}_{IK} \mathbf{x}_{IK} = \mathbf{b}_{IK}\) para
cada corte \(y_k\) é:

\[
\begin{cases}
\sum_{j=1}^n \lambda_j(z_k) \gamma_I(\mathbf{s}_i - \mathbf{s}_j; z_k) + \nu(z_k) = \gamma_I(\mathbf{s}_i - \mathbf{s}_0; z_k), & i=1, \dots, n \\
\sum_{j=1}^n \lambda_j(z_k) = 1
\end{cases}
\]

Myers (1994) classifica a IK como uma extensão não-linear que evita as
premissas fortes de distribuição bivariada exigidas pela
\href{https://www.teses.usp.br/teses/disponiveis/11/11134/tde-12052003-151635/publico/maria.pdf}{Krigagem
Disjuntiva}, embora exija a hipótese de estacionariedade forte para a
inferência da ccdf.

\textbf{Soft Kriging: Incorporando Incertezas}

Uma generalização poderosa da IK é a capacidade de incorporar dados
imprecisos ou qualitativos, conhecidos como \emph{Soft Data}. Andre G.
Journel (1986) formalizou o conceito de \emph{Soft Kriging}.

Enquanto um dado exato em \(\mathbf{s}_i\) é codificado como um degrau
abrupto na ccdf (0 ou 1), um dado suave (ex: o valor está entre \(a\) e
\(b\)) pode ser codificado como uma probabilidade
\(y(\mathbf{s}_i; y_k) \in [0, 1]\). O estimador se torna:

\[\hat{I}(\mathbf{s}_0; y_k) = \sum_{i \in \text{hard}} \lambda_i I(\mathbf{s}_i; y_k) + \sum_{j \in \text{soft}} \nu_j y(\mathbf{s}_j; y_k)\]

Ungaro et al. (2008) expandem este conceito utilizando \emph{Simple
Indicator Kriging with Varying Local Means (SIK-VLM)}, onde a média
local varia conforme mapas de solo auxiliares, permitindo distinguir
anomalias naturais de contaminação antrópica.

\textbf{Estatísticas E-type e Risco}

A IK resulta em um conjunto de probabilidades discretas para \(K\)
cortes: \(\hat{F}(y_1), \hat{F}(y_2), \dots, \hat{F}(y_K)\). Para obter
uma estimativa de valor (teor) único para o local \(\mathbf{s}_0\),
calculamos a esperança da ccdf, conhecida como estimativa \emph{E-type}:

\[\hat{Y}_{E\text{-type}}(\mathbf{s}_0) \approx \sum_{k=0}^{K} \bar{y}_k \left[ \hat{F}(y_{k+1}) - \hat{F}(y_k) \right]\]

Onde \(\bar{y}_k\) é a média da classe entre os cortes \(y_k\) e
\(y_{k+1}\). Carvalho e Deutsch (2017) destaca a importância de modelar
corretamente as caudas da distribuição (abaixo do primeiro corte e acima
do último), sugerindo ajustes hiperbólicos para evitar subestimativa de
valores extremos.

Além da média, a IK permite quantificar o risco de decisão. Juang e Lee
(1998) desenvolvem as equações para calcular a probabilidade de Falso
Positivo (\(\alpha\), classificar como perigoso o que é seguro) e Falso
Negativo (\(\beta\), classificar como seguro o que é perigoso), cruciais
para remediação ambiental.

\textbf{Desafios e Desenvolvimentos Recentes}

\begin{itemize}
\item
  Como as krigagens são independentes para cada corte, pode ocorrer que
  \(P(Y \le 10) < P(Y \le 5)\), o que é matematicamente impossível.
  Algoritmos de correção (médias ascendentes/descendentes) são aplicados
  a posteriori (Deutsch e Journel 1997).
\item
  Para reduzir o esforço computacional, Hill (1998) propõem a
  \emph{Median Indicator Kriging}, assumindo que o variograma da mediana
  é representativo para todos os cortes. Embora eficiente, Mohammadpour
  et al. (2019) argumentam que para anomalias sutis, deve-se usar
  variogramas específicos definidos por modelos Multifractais (FMIK)
  para separar corretamente as populações geológicas.
\item
  O avanço mais recente, proposto por Ji et al. (2025), introduz a
  \emph{Ordered Indicator Kriging} com parâmetros de campo. Este método
  adapta a anisotropia da krigagem localmente baseada na estrutura
  geológica (dobras, falhas) e na lógica deposicional (proximal-distal),
  superando a limitação de estacionaridade da IK em ambientes complexos.
\end{itemize}

\begin{Shaded}
\begin{Highlighting}[]
\NormalTok{pacman}\SpecialCharTok{::}\FunctionTok{p\_load}\NormalTok{(stars, gstat, ggplot2, sf, viridis, dplyr)}

\FunctionTok{data}\NormalTok{(meuse)}
\FunctionTok{data}\NormalTok{(meuse.grid)}


\CommentTok{\#Preparação dos Dados}
\CommentTok{\# Definir o corte (Threshold). Vamos usar o 3º quartil do Zinco como "perigo".}
\NormalTok{threshold }\OtherTok{\textless{}{-}} \FunctionTok{quantile}\NormalTok{(meuse}\SpecialCharTok{$}\NormalTok{zinc, }\FloatTok{0.75}\NormalTok{) }
\NormalTok{meuse}\SpecialCharTok{$}\NormalTok{zinc\_ind }\OtherTok{\textless{}{-}} \FunctionTok{ifelse}\NormalTok{(meuse}\SpecialCharTok{$}\NormalTok{zinc }\SpecialCharTok{\textgreater{}}\NormalTok{ threshold, }\DecValTok{1}\NormalTok{, }\DecValTok{0}\NormalTok{) }\CommentTok{\# 1 se perigoso, 0 se seguro}

\NormalTok{meuse\_sf }\OtherTok{\textless{}{-}} \FunctionTok{st\_as\_sf}\NormalTok{(meuse, }\AttributeTok{coords =} \FunctionTok{c}\NormalTok{(}\StringTok{"x"}\NormalTok{, }\StringTok{"y"}\NormalTok{), }\AttributeTok{crs =} \DecValTok{28992}\NormalTok{)}
\NormalTok{grid\_sf }\OtherTok{\textless{}{-}} \FunctionTok{st\_as\_sf}\NormalTok{(meuse.grid, }\AttributeTok{coords =} \FunctionTok{c}\NormalTok{(}\StringTok{"x"}\NormalTok{, }\StringTok{"y"}\NormalTok{), }\AttributeTok{crs =} \DecValTok{28992}\NormalTok{)}
\NormalTok{grid\_stars }\OtherTok{\textless{}{-}} \FunctionTok{st\_rasterize}\NormalTok{(grid\_sf, }\AttributeTok{dx =} \DecValTok{40}\NormalTok{, }\AttributeTok{dy =} \DecValTok{40}\NormalTok{)}


\CommentTok{\#Variograma Indicador}

\NormalTok{v\_ind }\OtherTok{\textless{}{-}} \FunctionTok{variogram}\NormalTok{(zinc\_ind }\SpecialCharTok{\textasciitilde{}} \DecValTok{1}\NormalTok{, meuse\_sf)}
\NormalTok{m\_ind }\OtherTok{\textless{}{-}} \FunctionTok{fit.variogram}\NormalTok{(v\_ind, }\FunctionTok{vgm}\NormalTok{(}\FloatTok{0.15}\NormalTok{, }\StringTok{"Exp"}\NormalTok{, }\DecValTok{600}\NormalTok{, }\FloatTok{0.05}\NormalTok{))}

\FunctionTok{plot}\NormalTok{(v\_ind, m\_ind, }\AttributeTok{main =} \StringTok{"Variograma Indicador (Zinco \textgreater{} Q75)"}\NormalTok{)}


\CommentTok{\#Krigagem Indicatriz (Ordinária)}

\CommentTok{\# O resultado (var1.pred) será a PROBABILIDADE de ser 1 (acima do corte)}
\NormalTok{ik }\OtherTok{\textless{}{-}} \FunctionTok{krige}\NormalTok{(zinc\_ind }\SpecialCharTok{\textasciitilde{}} \DecValTok{1}\NormalTok{, }
            \AttributeTok{locations =}\NormalTok{ meuse\_sf, }
            \AttributeTok{newdata =}\NormalTok{ grid\_stars, }
            \AttributeTok{model =}\NormalTok{ m\_ind, }\AttributeTok{debug.level =} \DecValTok{0}\NormalTok{)}


\NormalTok{ik}\SpecialCharTok{$}\NormalTok{probabilidade }\OtherTok{\textless{}{-}} \FunctionTok{pmin}\NormalTok{(}\FunctionTok{pmax}\NormalTok{(ik}\SpecialCharTok{$}\NormalTok{var1.pred, }\DecValTok{0}\NormalTok{), }\DecValTok{1}\NormalTok{)}

\FunctionTok{ggplot}\NormalTok{() }\SpecialCharTok{+}
  \FunctionTok{geom\_stars}\NormalTok{(}\AttributeTok{data =}\NormalTok{ ik, }\FunctionTok{aes}\NormalTok{(}\AttributeTok{fill =}\NormalTok{ probabilidade)) }\SpecialCharTok{+}
  \FunctionTok{geom\_sf}\NormalTok{(}\AttributeTok{data =}\NormalTok{ meuse\_sf, }\FunctionTok{aes}\NormalTok{(}\AttributeTok{color =} \FunctionTok{as.factor}\NormalTok{(zinc\_ind)), }\AttributeTok{size =} \DecValTok{1}\NormalTok{) }\SpecialCharTok{+}
  \FunctionTok{scale\_fill\_viridis\_c}\NormalTok{(}\AttributeTok{option =} \StringTok{"turbo"}\NormalTok{, }\AttributeTok{name =} \StringTok{"Prob. \textgreater{} Corte"}\NormalTok{, }\AttributeTok{na.value =} \StringTok{"transparent"}\NormalTok{) }\SpecialCharTok{+}
  \FunctionTok{scale\_color\_manual}\NormalTok{(}\AttributeTok{values =} \FunctionTok{c}\NormalTok{(}\StringTok{"white"}\NormalTok{, }\StringTok{"black"}\NormalTok{), }\AttributeTok{name =} \StringTok{"Dados Reais"}\NormalTok{, }\AttributeTok{labels =} \FunctionTok{c}\NormalTok{(}\StringTok{"\textless{}= Corte"}\NormalTok{, }\StringTok{"\textgreater{} Corte"}\NormalTok{))}\SpecialCharTok{+}
  \FunctionTok{labs}\NormalTok{(}\AttributeTok{title =} \StringTok{""}\NormalTok{,}
       \AttributeTok{subtitle =} \StringTok{"Probabilidade do teor de Zinco exceder o limiar de 75\%"}\NormalTok{,}
       \AttributeTok{x =} \ConstantTok{NULL}\NormalTok{, }\AttributeTok{y =} \ConstantTok{NULL}\NormalTok{) }\SpecialCharTok{+}
  \FunctionTok{theme\_void}\NormalTok{() }\SpecialCharTok{+}
  \FunctionTok{theme}\NormalTok{(}\AttributeTok{plot.title =} \FunctionTok{element\_text}\NormalTok{(}\AttributeTok{hjust =} \FloatTok{0.5}\NormalTok{), }\AttributeTok{legend.position =} \StringTok{"right"}\NormalTok{)}
\end{Highlighting}
\end{Shaded}

\begin{figure}[H]

\centering{

\pandocbounded{\includegraphics[keepaspectratio]{geostat_files/figure-pdf/fig-krigagem_indicatriz-1.pdf}}

}

\caption{\label{fig-krigagem_indicatriz-1}Krigagem Indicatriz}

\end{figure}%

\begin{figure}[H]

\centering{

\pandocbounded{\includegraphics[keepaspectratio]{geostat_files/figure-pdf/fig-krigagem_indicatriz-2.pdf}}

}

\caption{\label{fig-krigagem_indicatriz-2}Krigagem Indicatriz}

\end{figure}%

\section{Desafios Computacionais e a Abordagem
SPDE}\label{desafios-computacionais-e-a-abordagem-spde}

Imagine que você é um climatologista modelando a temperatura da
superfície do mar no Atlântico Norte usando dados de satélite. Você tem
\(n = 100.000\) pontos de observação
\(\mathbf{s}_1, \dots, \mathbf{s}_n\) e deseja prever a temperatura em
locais não medidos e entender a variabilidade espacial.

Na geoestatística, assumimos que o vetor de temperaturas observadas
\(\mathbf{y}\) é uma realização de um Campo Aleatório Gaussiano (GRF)
\(x(\mathbf{s})\). A dependência espacial é regida por uma Matriz de
Covariância densa \(\mathbf{\Sigma}\), onde o elemento \(\Sigma_{ij}\)
descreve a correlação entre \(\mathbf{s}_i\) e \(\mathbf{s}_j\).

Um Campo Aleatório Gaussiano (GRF - Gaussian Random Field), denotado por
\(\{x(\mathbf{s}) : \mathbf{s} \in D^G \subset \mathbb{R}^d\}\), é um
processo estocástico onde, para qualquer conjunto finito de locais
\(\{\mathbf{s}_1, \dots, \mathbf{s}_n\}\), o vetor
\(\mathbf{x} = (x(\mathbf{s}_1), \dots, x(\mathbf{s}_n))^\top\) segue
uma distribuição Normal Multivariada:

\[\mathbf{x} \sim \mathcal{N}(\boldsymbol{\mu}, \boldsymbol{\Sigma})\]

Onde \(\boldsymbol{\mu}\) é o vetor de médias e \(\boldsymbol{\Sigma}\)
é a matriz de covariância, cujos elementos são dados por uma função de
covariância definida positiva \(C(\cdot)\):

\[\Sigma_{ij} = \text{Cov}(x(\mathbf{s}_i), x(\mathbf{s}_j)) = C(\|\mathbf{s}_i - \mathbf{s}_j\|)\]

A função \(C\) é função de covariância da Matérn, que controla a
variância e a suavidade do campo. Note que ao longo do texto descrevemos
funções de semivariograma, mas poderiamos ter o feito para funções de
covariância.

Para estimar parâmetros (como o alcance da correlação) via Máxima
Verossimilhança, precisamos calcular a log-verossimilhança:

\[\log L(\theta) \propto -\frac{1}{2} \log |\mathbf{\Sigma}| - \frac{1}{2} \mathbf{y}^\top \mathbf{\Sigma}^{-1} \mathbf{y}\]

Isso exige calcular o determinante \(|\mathbf{\Sigma}|\) e a inversa
\(\mathbf{\Sigma}^{-1}\). O método padrão é a Fatoração de Cholesky
(\(\mathbf{\Sigma} = \mathbf{L}\mathbf{L}^\top\)), que decompõe a matriz
\(\boldsymbol{\Sigma}\) em um produto
\(\boldsymbol{\Sigma} = \mathbf{L}\mathbf{L}^\top\), onde \(\mathbf{L}\)
é uma matriz triangular inferior única com elementos diagonais
estritamente positivos.

A matriz \(\mathbf{\Sigma}\) é densa. Em processos espaciais, tudo se
correlaciona com tudo (Primeira Lei da Geografia), mesmo que a
correlação seja ínfima a longas distâncias.

O problema é que a matriz \(\boldsymbol{\Sigma}\) é densa (quase todos
os elementos são não-nulos, pois a correlação decai com a distância mas
nunca é exatamente zero). Portanto, \(\Sigma_{ij} \neq 0\) para quase
todos os pares. Como consequência, o custo computacional da fatoração de
Cholesky para uma matriz densa seja de ordem \(\mathcal{O}(n^3)\) e o
custo de armazenamento (memória) de ordem \(\mathcal{O}(n^2)\). Para
\(n=100.000\), isso exige da ordem de \(10^{15}\) operações e cerca de
80 GB de RAM apenas para armazenar a matriz em precisão dupla. Isso
torna a geoestatística clássica computacionalmente proibitiva para
grandes conjuntos de dados (Abdulah et al. 2023).

A solução reside na mudança de paradigma: modelar a Precisão
(dependência local) em vez da Covariância (dependência global).

\begin{itemize}
\item
  Covariância (\(\mathbf{\Sigma}\)) descreve a correlação marginal. Se o
  ponto A afeta B, e B afeta C, então A está correlacionado com C. O
  grafo de dependência é completo (denso).
\item
  Precisão (\(\mathbf{Q} = \mathbf{\Sigma}^{-1}\)) descreve a correlação
  condicional. Se fixarmos o valor de B, A e C tornam-se independentes
  (propriedade de Markov). Em um Campo Aleatório de Markov Gaussiano
  (GMRF), um ponto só interage com seus vizinhos imediatos. A matriz é
  esparsa (cheia de zeros).
\end{itemize}

Para construir uma matriz esparsa \(\mathbf{Q}\) que corresponda a um
modelo de covariância contínuo e válido, recorre-se a abordagem
\emph{Stochastic Partial Differential Equation} ( SPDE ou Equação
Diferencial Parcial Estocástica).

\subsection{Equação Diferencial Parcial Estocástica
(SPDE)}\label{equauxe7uxe3o-diferencial-parcial-estocuxe1stica-spde}

Uma SPDE é uma equação diferencial onde um ou mais termos são processos
estocásticos. No contexto da geoestatística, usamos uma SPDE linear para
descrever como o campo \(x(\mathbf{s})\) é gerado a partir de um ruído
branco.

Whittle (1954) provou um campo aleatório Gaussiano estacionário
\(x(\mathbf{s})\) em \(\mathbb{R}^d\) com função de covariância Matérn é
a solução estacionária da seguinte Equação Diferencial Parcial
Estocástica (SPDE) linear fracionária impulsionada por um ruído branco
Gaussiano:

\begin{equation}\phantomsection\label{eq-SPDE}{(\kappa^2 - \Delta)^{\alpha/2} (\tau x(\mathbf{s})) = \mathcal{W}(\mathbf{s}), \quad \mathbf{s} \in \mathbb{R}^d}\end{equation}

Onde:

\begin{itemize}
\item
  \(\Delta\) (Laplaciano) é operador diferencial definido como
  \(\Delta = \sum_{i=1}^d \frac{\partial^2}{\partial s_i^2}\). Mede a
  concavidade local.
\item
  \(\mathcal{W}(\mathbf{s})\) (Ruído Branco Espacial) é um processo
  estocástico Gaussiano com média zero e densidade espectral constante
  \(S_{\mathcal{W}}(\boldsymbol{\omega}) = 1\) em todas as frequências.
  Para quaisquer funções de teste \(f, g \in L^2(\mathbb{R}^d)\), a
  covariância é dada pelo produto interno:
  \(\text{Cov}(\langle f, \mathcal{W} \rangle, \langle g, \mathcal{W} \rangle) = \int f(\mathbf{s})g(\mathbf{s}) d\mathbf{s}\).
\item
  \(\kappa\) é parâmetro de escala espacial (\(>0\)). Controla o
  decaimento da correlação. Relacionado ao Alcance (\(\rho\)) por
  \(\rho = \sqrt{8\nu}/\kappa\).
\item
  \(\alpha\) é parâmetro de suavidade que determina a ordem da equação
  diferencial. Relacionado ao parâmetro \(\nu\) da Matérn por
  \(\alpha = \nu + d/2\)
\end{itemize}

\(\tau\) é parâmetro de escala da variância. Controla a magnitude das
flutuações. Relacionado à variância marginal \(\sigma^2\).

Para provar que a solução da equação acima realmente gera uma
covariância Matérn, utilizamos a análise espectral no domínio da
frequência (Lindgren, Rue, e Lindström 2011). Que fundamenta-se na
transformada de Fourier \(\mathcal{F}\), que consiste em converter uma
função do domínio espacial \(\mathbf{s}\) para o domínio da frequência
\(\boldsymbol{\omega}\). Uma propriedade crucial é como ela afeta o
operador Laplaciano:

\[\mathcal{F}(\Delta f(\mathbf{s})) = -\|\boldsymbol{\omega}\|^2 \hat{f}(\boldsymbol{\omega})\]

Aplicando a transformada de Fourier em ambos os lados da SPDE
(Eq.~\ref{eq-SPDE}), temos:

\[
\begin{aligned}
\mathcal{F}\left[ (\kappa^2 - \Delta)^{\alpha/2} (\tau x(\mathbf{s})) \right] &= \mathcal{F}[\mathcal{W}(\mathbf{s})] \\
\tau (\kappa^2 - \mathcal{F}[\Delta])^{\alpha/2} \hat{x}(\boldsymbol{\omega}) &= \hat{\mathcal{W}}(\boldsymbol{\omega}) \\
\tau (\kappa^2 + \|\boldsymbol{\omega}\|^2)^{\alpha/2} \hat{x}(\boldsymbol{\omega}) &= \hat{\mathcal{W}}(\boldsymbol{\omega})
\end{aligned}
\]

Isolamos \(\hat{x}(\boldsymbol{\omega})\) para encontrar a função de
transferência do sistema:

\[\hat{x}(\boldsymbol{\omega}) = \frac{1}{\tau (\kappa^2 + \|\boldsymbol{\omega}\|^2)^{\alpha/2}} \hat{\mathcal{W}}(\boldsymbol{\omega})\]

A densidade espectral de potência \(S_x(\boldsymbol{\omega})\) do campo
\(x\) é dada pelo quadrado do módulo da função de transferência
multiplicado pela densidade espectral do ruído de entrada (que é 1):

\[\begin{aligned}
S_x(\boldsymbol{\omega}) &= \left| \frac{1}{\tau (\kappa^2 + \|\boldsymbol{\omega}\|^2)^{\alpha/2}} \right|^2 \cdot S_{\mathcal{W}}(\boldsymbol{\omega}) \\
S_x(\boldsymbol{\omega}) &= \frac{1}{\tau^2 (\kappa^2 + \|\boldsymbol{\omega}\|^2)^{\alpha}}
\end{aligned}\]

Este resultado,
\(S_x(\boldsymbol{\omega}) \propto (\kappa^2 + \|\boldsymbol{\omega}\|^2)^{-\alpha}\),
é exatamente a definição da densidade espectral de um campo isotrópico
com covariância da classe Matérn em \(\mathbb{R}^d\), com parâmetro de
suavidade \(\nu = \alpha - d/2\) Lindgren, Bolin, e Rue (2022).

\textbf{Discretização via Método de Elementos Finitos}

A SPDE é uma equação definida no contínuo. Para implementá-la
computacionalmente e obter a almejada esparsidade, precisamos
discretizar o domínio espacial \(D^G\). Lindgren, Rue, e Lindström
(2011) propõem o uso do Método de Elementos Finitos (FEM), uma técnica
numérica para encontrar soluções aproximadas de equações diferenciais
parciais. O método consiste em subdividir um domínio contínuo em um
conjunto de subdomínios discretos (elementos, formando uma malha) e
aproximar a solução da equação como uma combinação linear de funções de
base simples definidas nesses elementos (Noel Cressie, Sainsbury-Dale, e
Zammit-Mangion 2022).

\textbf{A Malha e as Funções de Base}

Subdividimos o domínio \(D^G\) em uma malha (\emph{mesh}) de triângulos
não sobrepostos. Seja \(V\) o número de vértices (nós) da malha.
Definimos uma aproximação do campo contínuo \(x(\mathbf{s})\) como:

\[x(\mathbf{s}) \approx \sum_{k=1}^V w_k \psi_k(\mathbf{s})\]

Onde:

\begin{itemize}
\item
  \(w_k\) são pesos aleatórios (Gaussianos) associados a cada vértice
  \(k\). O objetivo da inferência passa a ser encontrar a distribuição
  conjunta do vetor de pesos \(\mathbf{w} = (w_1, \dots, w_V)^\top\).
\item
  \(\psi_k(\mathbf{s})\) são funções de base determinísticas, lineares
  por partes. A função \(\psi_k\) vale 1 no vértice \(k\), decai
  linearmente para 0 nos vértices vizinhos e é identicamente nula em
  todo o restante do domínio. Esta propriedade de suporte compacto é
  crucial para a esparsidade.
\end{itemize}

Não podemos substituir a aproximação diretamente na SPDE
\((\kappa^2 - \Delta)x = \mathcal{W}\) (assumindo \(\alpha=2\) para
simplificar, o que equivale a \(\nu=1\) em 2D), pois a segunda derivada
\(\Delta\) de funções lineares por partes resulta em funções indefinidas
(Deltas de Dirac) nas arestas dos triângulos.

Para contornar isso, usamos a formulação fraca (ou variacional):
multiplicamos a SPDE por uma função de teste (escolhemos a própria base
\(\psi_i\)) e integramos sobre o domínio \(\Omega\):

\[\int_{\Omega} \psi_i(\mathbf{s}) (\kappa^2 - \Delta) x(\mathbf{s}) \, d\mathbf{s} = \frac{1}{\tau} \int_{\Omega} \psi_i(\mathbf{s}) \mathcal{W}(\mathbf{s}) \, d\mathbf{s}, \quad \text{para } i=1,\dots,V\]

Para resolver o termo com o Laplaciano (\(\int \psi_i \Delta x\)),
aplicamos a Primeira Identidade de Green, que é a generalização da
integração por partes para dimensões superiores. Assumindo condições de
fronteira de Neumann (derivada normal nula na borda do domínio), a
identidade transfere uma derivada do campo \(x\) para a função de teste
\(\psi_i\):

\[-\int_{\Omega} \psi_i \Delta x \, d\mathbf{s} = \int_{\Omega} \nabla \psi_i \cdot \nabla x \, d\mathbf{s}\]

Substituindo a expansão
\(x(\mathbf{s}) = \sum_{j=1}^V w_j \psi_j(\mathbf{s})\) na equação
integral:

\[
\begin{aligned}
\int_{\Omega} \psi_i \left[ \kappa^2 \sum_j w_j \psi_j \right] d\mathbf{s} + \int_{\Omega} \nabla \psi_i \cdot \left[ \sum_j w_j \nabla \psi_j \right] d\mathbf{s} &= \frac{1}{\tau} \int_{\Omega} \psi_i \mathcal{W} d\mathbf{s} \\
\sum_{j=1}^V w_j \underbrace{\int_{\Omega} \kappa^2 \psi_i \psi_j d\mathbf{s}}_{\text{Termo dependente de C}} + \sum_{j=1}^V w_j \underbrace{\int_{\Omega} \nabla \psi_i \cdot \nabla \psi_j d\mathbf{s}}_{\text{Termo dependente de G}} &= \frac{1}{\tau} \int_{\Omega} \psi_i \mathcal{W} d\mathbf{s}
\end{aligned}
\]

Isso nos permite definir as matrizes esparsas fundamentais que dependem
apenas da geometria da malha:

\begin{itemize}
\item
  Matriz de Massa (\(\mathbf{C}\)):
  \(C_{ij} = \int_{\Omega} \psi_i(\mathbf{s}) \psi_j(\mathbf{s}) \, d\mathbf{s} = \langle \psi_i, \psi_j \rangle\).
\item
  Matriz de Rigidez (\(\mathbf{G}\)):
  \(G_{ij} = \int_{\Omega} \nabla \psi_i(\mathbf{s}) \cdot \nabla \psi_j(\mathbf{s}) \, d\mathbf{s} = \langle \nabla \psi_i, \nabla \psi_j \rangle\).
\end{itemize}

A equação matricial resultante para o vetor de pesos \(\mathbf{w}\) é:

\[(\kappa^2 \mathbf{C} + \mathbf{G}) \mathbf{w} = \tilde{\mathbf{W}}\]

onde \(\tilde{\mathbf{W}}\) é um vetor Gaussiano com média zero. Pela
propriedade do ruído branco espacial, a matriz de covariância de
\(\tilde{\mathbf{W}}\) é igual à matriz de massa \(\mathbf{C}\) (pois
\(\text{Cov}(\int f \mathcal{W}, \int g \mathcal{W}) = \int fg\)).

Nosso objetivo final é encontrar a matriz de precisão \(\mathbf{Q}\) dos
pesos \(\mathbf{w}\). Da equação linear derivada acima, definimos a
matriz do sistema como
\(\mathbf{K} = \kappa^2 \mathbf{C} + \mathbf{G}\). Temos então:
\(\mathbf{K}\mathbf{w} = \tilde{\mathbf{W}}\)

A matriz de covariância de \(\mathbf{w}\), denotada por
\(\boldsymbol{\Sigma}_w\), é dada por:

\[
\begin{aligned}
\boldsymbol{\Sigma}_w &= \text{Var}(\mathbf{w}) \\
&= \text{Var}(\mathbf{K}^{-1}\tilde{\mathbf{W}}) \\
&= \mathbf{K}^{-1} \text{Var}(\tilde{\mathbf{W}}) \mathbf{K}^{-T} \quad \text{(Propriedade da variância linear)} \\
&= \mathbf{K}^{-1} \mathbf{C} \mathbf{K}^{-T} \quad \text{(Pois Var}(\tilde{\mathbf{W}}) = \mathbf{C})
\end{aligned}
\]

A matriz de precisão \(\mathbf{Q}\) é, por definição, a inversa da
matriz de covariância:

\[
\begin{aligned}
\mathbf{Q} &= \boldsymbol{\Sigma}_w^{-1} \\
&= (\mathbf{K}^{-1} \mathbf{C} \mathbf{K}^{-T})^{-1} \\
&= \mathbf{K}^T \mathbf{C}^{-1} \mathbf{K}
\end{aligned}
\]

As matrizes \(\mathbf{K}\) e \(\mathbf{C}\) são esparsas, pois as
funções de base \(\psi_i\) e \(\psi_j\) só se sobrepoem se os vértices
\(i\) e \(j\) forem vizinhos na malha. Se não forem vizinhos, as
integrais são zero. No entanto, a inversa de uma matriz esparsa
(\(\mathbf{C}^{-1}\)) é, em geral, densa. Se usássemos
\(\mathbf{C}^{-1}\) na fórmula acima, \(\mathbf{Q}\) se tornaria densa,
destruindo todo o benefício computacional da abordagem.

Para resolver isso, Lindgren, Rue, e Lindström (2011) utilizam a técnica
de \href{https://en.wikipedia.org/wiki/Lumped-element_model}{Mass
Lumping} (comum em métodos numéricos): substituímos a matriz de massa
consistente \(\mathbf{C}\) por uma matriz diagonal
\(\tilde{\mathbf{C}}\), onde os elementos diagonais são a soma das
linhas de \(\mathbf{C}\):

\[\tilde{C}_{ii} = \sum_j C_{ij} = \int_{\Omega} \psi_i(\mathbf{s}) d\mathbf{s}\]

Como a inversa de uma matriz diagonal é trivial e também diagonal
(portanto, esparsa), podemos calcular a precisão final preservando a
esparsidade. Substituindo
\(\mathbf{K} = \kappa^2 \mathbf{C} + \mathbf{G}\) e usando a aproximação
diagonal \(\tilde{\mathbf{C}}\), obtemos a matriz de precisão explícita
para \(\alpha=2\):

\[\mathbf{Q}_{\text{SPDE}} = \tau^2 (\kappa^2 \tilde{\mathbf{C}} + \mathbf{G})^T \tilde{\mathbf{C}}^{-1} (\kappa^2 \tilde{\mathbf{C}} + \mathbf{G})\]

A matriz \(\mathbf{Q}_{\text{SPDE}}\) é construída apenas por somas e
multiplicações de matrizes esparsas e diagonais. Portanto,
\(\mathbf{Q}_{\text{SPDE}}\) é esparsa. Esta esparsidade permite o uso
de algoritmos eficientes de fatoração de Cholesky esparsa, reduzindo a
complexidade computacional de \(\mathcal{O}(n^3)\) para aproximadamente
\(\mathcal{O}(n^{3/2})\) em problemas espaciais 2D. Isso viabiliza a
análise bayesiana (via INLA) (Bakka et al. 2018; Lindgren e Rue 2015) ou
a estimação de máxima verossimilhança (via ExaGeoStatR) para grandes
conjuntos de dados geoestatísticos (Big Data) que eram anteriormente
intratáveis (Abdulah et al. 2023).

\begin{Shaded}
\begin{Highlighting}[]
\NormalTok{pacman}\SpecialCharTok{::}\FunctionTok{p\_load}\NormalTok{(sf, ggplot2)}

\FunctionTok{set.seed}\NormalTok{(}\DecValTok{123}\NormalTok{)}
\NormalTok{fronteira }\OtherTok{\textless{}{-}} \FunctionTok{st\_polygon}\NormalTok{(}\FunctionTok{list}\NormalTok{(}\FunctionTok{rbind}\NormalTok{(}\FunctionTok{c}\NormalTok{(}\DecValTok{0}\NormalTok{,}\DecValTok{0}\NormalTok{), }\FunctionTok{c}\NormalTok{(}\DecValTok{10}\NormalTok{,}\DecValTok{0}\NormalTok{), }\FunctionTok{c}\NormalTok{(}\DecValTok{10}\NormalTok{,}\DecValTok{10}\NormalTok{), }\FunctionTok{c}\NormalTok{(}\DecValTok{0}\NormalTok{,}\DecValTok{10}\NormalTok{), }\FunctionTok{c}\NormalTok{(}\DecValTok{0}\NormalTok{,}\DecValTok{0}\NormalTok{))))}
\NormalTok{pontos }\OtherTok{\textless{}{-}} \FunctionTok{st\_sample}\NormalTok{(fronteira, }\AttributeTok{size =} \DecValTok{30}\NormalTok{)}

\NormalTok{borda\_externa }\OtherTok{\textless{}{-}} \FunctionTok{st\_buffer}\NormalTok{(}\FunctionTok{st\_sfc}\NormalTok{(fronteira), }\AttributeTok{dist =} \DecValTok{2}\NormalTok{) }
\NormalTok{pontos\_borda }\OtherTok{\textless{}{-}} \FunctionTok{st\_sample}\NormalTok{(}\FunctionTok{st\_cast}\NormalTok{(borda\_externa, }\StringTok{"LINESTRING"}\NormalTok{), }\AttributeTok{size =} \DecValTok{20}\NormalTok{)}
\NormalTok{todos\_pontos }\OtherTok{\textless{}{-}} \FunctionTok{c}\NormalTok{(pontos, pontos\_borda)}

\NormalTok{malha }\OtherTok{\textless{}{-}} \FunctionTok{st\_triangulate}\NormalTok{(}\FunctionTok{st\_combine}\NormalTok{(todos\_pontos)) }\SpecialCharTok{|\textgreater{}} 
  \FunctionTok{st\_collection\_extract}\NormalTok{(}\StringTok{"POLYGON"}\NormalTok{) }\SpecialCharTok{|\textgreater{}} 
  \FunctionTok{st\_sf}\NormalTok{()}

\FunctionTok{ggplot}\NormalTok{() }\SpecialCharTok{+}
  \FunctionTok{geom\_sf}\NormalTok{(}\AttributeTok{data =}\NormalTok{ malha, }\AttributeTok{fill =} \ConstantTok{NA}\NormalTok{, }\AttributeTok{color =} \StringTok{"grey60"}\NormalTok{, }\AttributeTok{size =} \FloatTok{0.3}\NormalTok{) }\SpecialCharTok{+}
  \FunctionTok{geom\_sf}\NormalTok{(}\AttributeTok{data =}\NormalTok{ fronteira, }\AttributeTok{fill =} \ConstantTok{NA}\NormalTok{, }\AttributeTok{color =} \StringTok{"blue"}\NormalTok{, }\AttributeTok{size =} \DecValTok{1}\NormalTok{) }\SpecialCharTok{+}
  \FunctionTok{geom\_sf}\NormalTok{(}\AttributeTok{data =}\NormalTok{ pontos, }\AttributeTok{color =} \StringTok{"red"}\NormalTok{, }\AttributeTok{size =} \DecValTok{2}\NormalTok{) }\SpecialCharTok{+}
  \FunctionTok{theme\_void}\NormalTok{() }\SpecialCharTok{+}
  \FunctionTok{labs}\NormalTok{(}\AttributeTok{title =} \StringTok{""}\NormalTok{,}
       \AttributeTok{subtitle =} \StringTok{"(1) Triângulos cinza sao elementos finitos; (2) pontos vermelhos }\SpecialCharTok{\textbackslash{}n}\StringTok{são dados observados; (3) linha azul é o domínio de estudo"}\NormalTok{) }\SpecialCharTok{+}
  \FunctionTok{theme}\NormalTok{(}\AttributeTok{plot.title =} \FunctionTok{element\_text}\NormalTok{(}\AttributeTok{hjust =} \FloatTok{0.5}\NormalTok{),}
        \AttributeTok{plot.subtitle =} \FunctionTok{element\_text}\NormalTok{(}\AttributeTok{hjust =} \FloatTok{0.5}\NormalTok{, }\AttributeTok{color =} \StringTok{"grey40"}\NormalTok{))}
\end{Highlighting}
\end{Shaded}

\begin{figure}[H]

\centering{

\pandocbounded{\includegraphics[keepaspectratio]{geostat_files/figure-pdf/fig-spde-mesh-1.pdf}}

}

\caption{\label{fig-spde-mesh}Conceito de Malha (Mesh) para SPDE. A
precisão é maior onde há dados e menor nas bordas.}

\end{figure}%

\begin{tcolorbox}[enhanced jigsaw, left=2mm, toptitle=1mm, colback=white, colframe=quarto-callout-important-color-frame, colbacktitle=quarto-callout-important-color!10!white, opacityback=0, rightrule=.15mm, bottomtitle=1mm, arc=.35mm, title=\textcolor{quarto-callout-important-color}{\faExclamation}\hspace{0.5em}{Aprofundamento Teórico e Prático}, titlerule=0mm, bottomrule=.15mm, leftrule=.75mm, coltitle=black, toprule=.15mm, breakable, opacitybacktitle=0.6]

Além dos métodos abordados aqui, existem diversas variantes de Krigagem
desenvolvidas para lidar com características específicas dos dados.
Entre elas destacam-se: a Krigagem Lognormal (aplicada aos logaritmos
dos dados); a Krigagem Multi-Gaussiana (aplicada após a transformação
normal score), que é uma generalização da lognormal; a Krigagem de
Postos (\emph{Rank Kriging}, baseada na transformação uniforme);
Krigagem Disjuntiva, entre outras. Para maior detalhes e aplicações,
recomenda-se a leitura do capítulo 4 de Deutsch e Journel (1997).

Muitas vezes, a aplicação correta de transformações nos dados é
suficiente para resolver problemas de estacionaridade ou
não-normalidade. Aos interessados neste tópico, recomenda-se a leitura
do capítulo 3 de Yamamoto e Landim (2013).

Por fim, para uma visão histórica completa da geoestatística e um guia
prático de ferramentas computacionais, sugere-se o capítulo 3 de Scalon
(2024). A obra abrange o uso do \texttt{R/RStudio}, análise
exploratória, outros tipos de semivariogramas, validação de modelos e
inclui um tutorial detalhado sobre o pacote \texttt{geoR}.

\end{tcolorbox}

\section{Pacote gstat}\label{pacote-gstat}

O pacote \texttt{gstat} (E. J. Pebesma 2004; Gräler, Pebesma, e
Heuvelink 2016) é de autoria do professor
\href{https://scholar.google.com/citations?user=d6jdqdQAAAAJ&hl=en}{Edzer
Pebesma}, junto com o pacote \texttt{geoR} (Paulo Justiniano Ribeiro Jr
e Diggle 2025; Paulo J. Ribeiro Jr e Diggle 2006) da autoria dos
professores
\href{https://scholar.google.com/citations?user=-Odk2HgAAAAJ&hl=pt-BR}{Paulo
Justiniano Ribeiro Jr} e
\href{https://scholar.google.com/citations?user=GcTKrPIAAAAJ&hl=en}{Peter
J. Diggle} se estabeleceram como a pacotes de referência no ambiente
\texttt{R} para a modelagem geoestatística. Aqui é descrito apenas o
pacote \texttt{gstat}. Aos interressados na descrição completa do pacote
\texttt{geoR} sugere-se o capítulo 3 de Scalon (2024).

Embora suas raízes estejam no antigo pacote \texttt{sp}, as versões
contemporâneas do gstat integram-se nativamente com os pacotes
\texttt{sf} (\emph{Simple Features}) e \texttt{stars}
(\emph{Spatiotemporal Arrays}), permitindo um fluxo de trabalho moderno
e computacionalmente eficiente.

Utilizaremos o conjunto de dados meuse (concentração de metais pesados
na planície de inundação do rio Meuse, Holanda) para demonstração.

\textbf{Preparação do Ambiente e Dados}

Para a execução de rotinas geoestatísticas, dois componentes geométricos
são imprescindíveis:

\begin{enumerate}
\def\labelenumi{\arabic{enumi}.}
\item
  Dados Observados (Suporte Pontual): As amostras coletadas em campo.
\item
  Malha de Predição (Suporte de Área/Grade): O domínio espacial
  discretizado onde as estimativas serão realizadas.
\end{enumerate}

\begin{Shaded}
\begin{Highlighting}[]
\ControlFlowTok{if}\NormalTok{ (}\SpecialCharTok{!}\FunctionTok{require}\NormalTok{(}\StringTok{"pacman"}\NormalTok{)) }\FunctionTok{install.packages}\NormalTok{(}\StringTok{"pacman"}\NormalTok{)}

\NormalTok{pacman}\SpecialCharTok{::}\FunctionTok{p\_load}\NormalTok{(gstat, sf, stars, ggplot2, viridis, dplyr,gt, gridExtra)}


\CommentTok{\#Carregamento e conversão dos dados amostrais}
\FunctionTok{data}\NormalTok{(meuse)}
\CommentTok{\# O CRS 28992 refere{-}se à projeção holandesa Amersfoort / RD New, vc nos seus dados usaraa crs=4326}
\NormalTok{meuse\_sf }\OtherTok{\textless{}{-}} \FunctionTok{st\_as\_sf}\NormalTok{(meuse, }\AttributeTok{coords =} \FunctionTok{c}\NormalTok{(}\StringTok{"x"}\NormalTok{, }\StringTok{"y"}\NormalTok{), }\AttributeTok{crs =} \DecValTok{28992}\NormalTok{)}
\FunctionTok{glimpse}\NormalTok{(meuse\_sf)}
\end{Highlighting}
\end{Shaded}

\begin{verbatim}
Rows: 155
Columns: 13
$ cadmium  <dbl> 11.7, 8.6, 6.5, 2.6, 2.8, 3.0, 3.2, 2.8, 2.4, 1.6, 1.4, 1.8, ~
$ copper   <dbl> 85, 81, 68, 81, 48, 61, 31, 29, 37, 24, 25, 25, 93, 31, 27, 8~
$ lead     <dbl> 299, 277, 199, 116, 117, 137, 132, 150, 133, 80, 86, 97, 285,~
$ zinc     <dbl> 1022, 1141, 640, 257, 269, 281, 346, 406, 347, 183, 189, 251,~
$ elev     <dbl> 7.909, 6.983, 7.800, 7.655, 7.480, 7.791, 8.217, 8.490, 8.668~
$ dist     <dbl> 0.00135803, 0.01222430, 0.10302900, 0.19009400, 0.27709000, 0~
$ om       <dbl> 13.6, 14.0, 13.0, 8.0, 8.7, 7.8, 9.2, 9.5, 10.6, 6.3, 6.4, 9.~
$ ffreq    <fct> 1, 1, 1, 1, 1, 1, 1, 1, 1, 1, 1, 1, 1, 1, 1, 1, 1, 1, 1, 1, 1~
$ soil     <fct> 1, 1, 1, 2, 2, 2, 2, 1, 1, 2, 2, 1, 1, 1, 1, 1, 1, 1, 1, 1, 1~
$ lime     <fct> 1, 1, 1, 0, 0, 0, 0, 0, 0, 0, 0, 0, 1, 0, 0, 1, 1, 1, 1, 1, 1~
$ landuse  <fct> Ah, Ah, Ah, Ga, Ah, Ga, Ah, Ab, Ab, W, Fh, Ag, W, Ah, Ah, W, ~
$ dist.m   <dbl> 50, 30, 150, 270, 380, 470, 240, 120, 240, 420, 400, 300, 20,~
$ geometry <POINT [m]> POINT (181072 333611), POINT (181025 333558), POINT (18~
\end{verbatim}

\textbf{Criação da Malha de Predição (Grid)}

A krigagem precisa de locais para onde serão feitas as predições. Para
tal criaremos uma grade regular (\texttt{raster}) baseada nos limites da
área.

\begin{Shaded}
\begin{Highlighting}[]
\CommentTok{\# O pacote traz um polígono \textquotesingle{}meuse.area\textquotesingle{}, vamos convertê{-}lo para sf}

\FunctionTok{data}\NormalTok{(meuse.area)}

\NormalTok{limite\_sf }\OtherTok{\textless{}{-}} \FunctionTok{st\_polygon}\NormalTok{(}\FunctionTok{list}\NormalTok{(}\FunctionTok{as.matrix}\NormalTok{(meuse.area))) }\SpecialCharTok{|\textgreater{}} 
  \FunctionTok{st\_sfc}\NormalTok{(}\AttributeTok{crs =} \DecValTok{28992}\NormalTok{) }\SpecialCharTok{|\textgreater{}} 
  \FunctionTok{st\_sf}\NormalTok{() }\CommentTok{\#provavelmente vc terá um arquivo shapfile, use arquivo .shp}


\CommentTok{\#Gerar a Grade Regular (Rasterização Vetorial)}
\CommentTok{\# st\_make\_grid cria a geometria. \textquotesingle{}cellsize\textquotesingle{} define a resolução (ex: 40x40 metros)}

\NormalTok{grid\_vetorial }\OtherTok{\textless{}{-}} \FunctionTok{st\_make\_grid}\NormalTok{(limite\_sf, }\AttributeTok{cellsize =} \DecValTok{40}\NormalTok{, }\AttributeTok{what =} \StringTok{"centers"}\NormalTok{) }\SpecialCharTok{|\textgreater{}}
  \FunctionTok{st\_as\_sf}\NormalTok{() }\SpecialCharTok{|\textgreater{}}
  \FunctionTok{st\_filter}\NormalTok{(limite\_sf) }\CommentTok{\# Recorta a grade para ficar apenas dentro do polígono}


\CommentTok{\#Conversão para STARS (Mais eficiente para o gstat)}

\NormalTok{grid\_stars }\OtherTok{\textless{}{-}} \FunctionTok{st\_as\_stars}\NormalTok{(}\FunctionTok{st\_bbox}\NormalTok{(limite\_sf), }\AttributeTok{dx =} \DecValTok{40}\NormalTok{, }\AttributeTok{dy =} \DecValTok{40}\NormalTok{)}
\NormalTok{grid\_stars }\OtherTok{\textless{}{-}} \FunctionTok{st\_crop}\NormalTok{(grid\_stars, limite\_sf) }\CommentTok{\# Mascara o que está fora do limite}


\FunctionTok{ggplot}\NormalTok{() }\SpecialCharTok{+}
  \FunctionTok{geom\_sf}\NormalTok{(}\AttributeTok{data =}\NormalTok{ grid\_vetorial, }\AttributeTok{color=}\StringTok{"white"}\NormalTok{) }\SpecialCharTok{+} \CommentTok{\# vc pode trocar \textquotesingle{}grid\_vetorial\textquotesingle{} por \textquotesingle{}grid\_stars\textquotesingle{}}
  \FunctionTok{geom\_sf}\NormalTok{(}\AttributeTok{data =}\NormalTok{ meuse\_sf, }\AttributeTok{color =} \StringTok{"red"}\NormalTok{, }\AttributeTok{size =} \FloatTok{0.5}\NormalTok{) }\SpecialCharTok{+}\CommentTok{\# e aqui trocar \textquotesingle{}geom\_sf\textquotesingle{} por geom\_stars(data=grid\_stars)}
  \FunctionTok{geom\_sf}\NormalTok{(}\AttributeTok{data=}\NormalTok{limite\_sf, }\AttributeTok{color=}\StringTok{"black"}\NormalTok{, }\AttributeTok{fill=}\ConstantTok{NA}\NormalTok{)}\SpecialCharTok{+}
  \FunctionTok{labs}\NormalTok{(}\AttributeTok{title =} \StringTok{"Domínio de Predição (Grade) e Amostras (Vermelho)"}\NormalTok{) }\SpecialCharTok{+}
  \FunctionTok{theme\_void}\NormalTok{()}
\end{Highlighting}
\end{Shaded}

\pandocbounded{\includegraphics[keepaspectratio]{geostat_files/figure-pdf/unnamed-chunk-3-1.pdf}}

\textbf{Análise exploratória espacial}

Antes de modelar, é necessário verificar a existência de dependência
espacial.

\texttt{hscat:} Gráficos de Dispersão Defasados

A função \texttt{hscat} (\texttt{h-scatterplots}) confronta o valor de
\(Z(s)\) com \(Z(s+h)\). Se houver estrutura espacial, espera-se que,
para distâncias (\(h\)) pequenas, os pontos se alinhem à diagonal (alta
correlação). Conforme \(h\) aumenta, a nuvem deve se dispersar.

\begin{itemize}
\item
  \texttt{formula:} Define a variável e/ou variável resposta
  (\texttt{ex:\ log(zinc)\ \textasciitilde{}\ 1}) (recomenda-se
  transformação logarítmica para dados assimétricos de concentração).
\item
  \texttt{data:} O objeto espacial contendo as observações.
\item
  \texttt{breaks:} Vetor numérico definindo os limites dos intervalos de
  distância (lags).
\end{itemize}

\begin{Shaded}
\begin{Highlighting}[]
\FunctionTok{hscat}\NormalTok{(}\FunctionTok{log}\NormalTok{(zinc) }\SpecialCharTok{\textasciitilde{}} \DecValTok{1}\NormalTok{, }\AttributeTok{data=}\NormalTok{meuse\_sf, }\AttributeTok{breaks =} \FunctionTok{c}\NormalTok{(}\DecValTok{0}\NormalTok{, }\DecValTok{100}\NormalTok{, }\DecValTok{200}\NormalTok{, }\DecValTok{400}\NormalTok{, }\DecValTok{800}\NormalTok{))}
\end{Highlighting}
\end{Shaded}

\begin{figure}[H]

\centering{

\pandocbounded{\includegraphics[keepaspectratio]{geostat_files/figure-pdf/fig-hscat-1.pdf}}

}

\caption{\label{fig-hscat}Dispersão espacial do log(Zinco) em diferentes
lags}

\end{figure}%

\textbf{Modelagem da Covariância}

A etapa central da geoestatística é a determinação da função de
semivariância \(\gamma(h)\), que quantifica a dissimilaridade espacial.

\begin{itemize}
\tightlist
\item
  O Variograma Experimental: \texttt{variogram}
\end{itemize}

A função \texttt{variogram} calcula a semivariância média para classes
de distância discretas a partir dos dados observados.

\begin{enumerate}
\def\labelenumi{\arabic{enumi}.}
\item
  Sintaxe: \texttt{variogram(object,\ locations,\ ...)}
\item
  Argumentos:
\end{enumerate}

\begin{itemize}
\item
  \texttt{formula:} Para Krigagem Ordinária, utiliza-se
  \texttt{z\ \textasciitilde{}\ 1} (média constante). Para Krigagem
  Universal, define-se a tendência,
  ex:\texttt{z\ \textasciitilde{}\ x\ +\ y}, para as demais krigagens
  consulte a seção Seção~\ref{sec-metodos_predicao}.
\item
  \texttt{cutoff:} A distância máxima de investigação. Por convenção,
  limita-se a 1/3 da diagonal da área de estudo para garantir
  representatividade amostral nos lags.
\item
  \texttt{width:} A largura do intervalo de classe (tamanho do lag).
\item
  \texttt{cloud:} Se TRUE, retorna a nuvem variográfica (semivariância
  de todos os pares individuais), útil para detecção de outliers.
\item
  \texttt{map:} Se TRUE, gera um mapa variográfico para inspeção de
  anisotropia.
\item
  \texttt{alpha:} Direção em graus (ex: c(0, 45, 90, 135)) para
  investigar anisotropia.
\end{itemize}

\begin{Shaded}
\begin{Highlighting}[]
\CommentTok{\# Variograma Omnidirecional (Isotrópico)}
\NormalTok{v\_exp }\OtherTok{\textless{}{-}} \FunctionTok{variogram}\NormalTok{(}\FunctionTok{log}\NormalTok{(zinc) }\SpecialCharTok{\textasciitilde{}} \DecValTok{1}\NormalTok{, meuse\_sf, }\AttributeTok{cutoff =} \DecValTok{1200}\NormalTok{, }\AttributeTok{width =} \DecValTok{100}\NormalTok{)}

\FunctionTok{plot}\NormalTok{(v\_exp, }\AttributeTok{main =} \StringTok{"Semivariograma Experimental"}\NormalTok{, }
     \AttributeTok{xlab =} \StringTok{"Distância (m)"}\NormalTok{, }\AttributeTok{ylab =} \StringTok{"Semivariância"}\NormalTok{)}
\end{Highlighting}
\end{Shaded}

\pandocbounded{\includegraphics[keepaspectratio]{geostat_files/figure-pdf/unnamed-chunk-4-1.pdf}}

\textbf{Definição do Modelo Teórico}: \texttt{vgm}

O variograma experimental fornece pontos discretos. A krigagem exige uma
função contínua e positiva definida. A função \texttt{vgm} estrutura
este modelo.

\begin{itemize}
\item
  \texttt{psill}: Patamar parcial (variância estrutural).
\item
  \texttt{model}: Família da curva.

  \begin{itemize}
  \item
    ``Sph'' (Esférico): Crescimento linear na origem, atinge patamar
    definido.
  \item
    ``Exp'' (Exponencial): Crescimento abrupto, atinge patamar
    assintoticamente.
  \item
    ``Gau'' (Gaussiano): Suave na origem (parabólico), indica alta
    continuidade.
  \end{itemize}
\item
  \texttt{range:} Alcance.
\item
  \texttt{nugget:} Efeito pepita (erro na origem).
\end{itemize}

Para visualizar as famílias disponíveis, utiliza-se
\texttt{show.vgms()}.

\begin{Shaded}
\begin{Highlighting}[]
\CommentTok{\# Definição inicial baseada na inspeção visual do gráfico anterior}

\NormalTok{modelo\_inicial }\OtherTok{\textless{}{-}} \FunctionTok{vgm}\NormalTok{(}\AttributeTok{psill =} \FloatTok{0.6}\NormalTok{, }\AttributeTok{model =} \StringTok{"Sph"}\NormalTok{, }\AttributeTok{range =} \DecValTok{800}\NormalTok{, }\AttributeTok{nugget =} \FloatTok{0.05}\NormalTok{) }

\CommentTok{\#range 800 porque parece se estabilizar nele}
\CommentTok{\# no mesmo lugar que se estabiliza range 800, temos uma patamar (psill) de \textasciitilde{} 0.6 a 0.7}
\CommentTok{\#começa meio como linha reta obliquo, então aparenta ser esférico (Sph)}
\CommentTok{\# se olhar para onde se interceta o eixo y parece ser \textasciitilde{}0.05 esse é efeito pepita (nugget)}
\end{Highlighting}
\end{Shaded}

\textbf{Ajuste de Parâmetros}: \texttt{fit.variogram}

A função \texttt{fit.variogram} ajusta os parâmetros do modelo teórico
(\texttt{vgm}) aos pontos experimentais (\texttt{v\_exp}) utilizando
Mínimos Quadrados Ponderados (WLS). O método pondera mais fortemente os
lags iniciais (menores distâncias), que contêm mais pares de pontos e
são cruciais para a interpolação.

\begin{Shaded}
\begin{Highlighting}[]
\NormalTok{modelo\_ajustado }\OtherTok{\textless{}{-}} \FunctionTok{fit.variogram}\NormalTok{(v\_exp, modelo\_inicial)}
\NormalTok{modelo\_ajustado}\SpecialCharTok{|\textgreater{}}
\NormalTok{  knitr}\SpecialCharTok{::}\FunctionTok{kable}\NormalTok{()}
\end{Highlighting}
\end{Shaded}

\begin{figure}

\centering{

\begin{longtable*}[]{@{}lrrrrrrrr@{}}
\toprule\noalign{}
model & psill & range & kappa & ang1 & ang2 & ang3 & anis1 & anis2 \\
\midrule\noalign{}
\endhead
\bottomrule\noalign{}
\endlastfoot
Nug & 0.0635564 & 0.000 & 0.0 & 0 & 0 & 0 & 1 & 1 \\
Sph & 0.6042690 & 979.887 & 0.5 & 0 & 0 & 0 & 1 & 1 \\
\end{longtable*}

}

\caption{\label{fig-var-fit-1}Ajuste do modelo teórico}

\end{figure}%

\begin{Shaded}
\begin{Highlighting}[]
\FunctionTok{plot}\NormalTok{(v\_exp, modelo\_ajustado)}
\end{Highlighting}
\end{Shaded}

\begin{figure}[H]

\centering{

\pandocbounded{\includegraphics[keepaspectratio]{geostat_files/figure-pdf/fig-var-fit-1.pdf}}

}

\caption{\label{fig-var-fit-2}Ajuste do modelo teórico}

\end{figure}%

\textbf{Extração de Valores}: \texttt{variogramLine}

Se desejar reproduzir a curva teórica no ggplot2, esta função gera os
dados da linha.

\begin{Shaded}
\begin{Highlighting}[]
\NormalTok{linha\_teorica }\OtherTok{\textless{}{-}} \FunctionTok{variogramLine}\NormalTok{(modelo\_ajustado, }\AttributeTok{maxdist =} \DecValTok{1200}\NormalTok{)}

\FunctionTok{head}\NormalTok{(linha\_teorica) }\SpecialCharTok{|\textgreater{}}
\NormalTok{  knitr}\SpecialCharTok{::}\FunctionTok{kable}\NormalTok{()}
\end{Highlighting}
\end{Shaded}

\begin{longtable}[]{@{}rr@{}}
\toprule\noalign{}
dist & gamma \\
\midrule\noalign{}
\endhead
\bottomrule\noalign{}
\endlastfoot
0.001200 & 0.0635575 \\
6.031345 & 0.0691354 \\
12.061489 & 0.0747128 \\
18.091634 & 0.0802894 \\
24.121779 & 0.0858648 \\
30.151924 & 0.0914384 \\
\end{longtable}

\textbf{Krigagem: Interpolação espacial}: \texttt{krige}

A função \texttt{krige} seleciona automaticamente o método adequado
(Simples, Ordinária, Universal) baseando-se nos argumentos fornecidos.

\begin{itemize}
\item
  Sintaxe: \texttt{krige(formula,\ locations,\ newdata,\ model)}
\item
  Argumentos:

  \begin{itemize}
  \item
    \texttt{formula:} log(zinc) \textasciitilde{} 1 indica ausência de
    variáveis explicativas exógenas (apenas intercepto).
  \item
    \texttt{locations:} Objeto \texttt{sf} com os dados observados.
  \item
    \texttt{newdata}: Objeto \texttt{sf} ou \texttt{stars} com os locais
    de predição.
  \item
    \texttt{model:} O modelo de variograma ajustado.
  \item
    \texttt{block:} (Opcional) Se fornecido um vetor, ex: c(40, 40),
    realiza Krigagem de Bloco, estimando o valor médio dentro da célula,
    resultando em mapas mais suaves e menor variância de predição.
  \end{itemize}
\end{itemize}

\begin{Shaded}
\begin{Highlighting}[]
\CommentTok{\# Execução da Krigagem Ordinária}

\NormalTok{krigagem\_ord }\OtherTok{\textless{}{-}} \FunctionTok{krige}\NormalTok{(}\FunctionTok{log}\NormalTok{(zinc) }\SpecialCharTok{\textasciitilde{}} \DecValTok{1}\NormalTok{, }
                      \AttributeTok{locations =}\NormalTok{ meuse\_sf,  }\CommentTok{\# Dados}
                      \AttributeTok{newdata =}\NormalTok{grid\_vetorial ,  }\CommentTok{\# Grade de destino, vc poderia usar tambem grid\_stars}
                      \AttributeTok{model =}\NormalTok{ modelo\_ajustado, }\AttributeTok{debug.level =} \DecValTok{0}\NormalTok{) }\CommentTok{\# Modelo espacial}
\end{Highlighting}
\end{Shaded}

O objeto resultante (krigagem\_ord) contém duas variáveis fundamentais:

\begin{itemize}
\item
  \texttt{var1.pred:} O valor predito (estimativa).
\item
  \texttt{var1.var:} A variância da krigagem (incerteza da estimativa).
\end{itemize}

\textbf{Visualização de Mapas de Predição e Incerteza}

A apresentação correta dos resultados exige a exibição da estimativa
juntamente com sua incerteza associada.

\begin{Shaded}
\begin{Highlighting}[]
\NormalTok{pacman}\SpecialCharTok{::}\FunctionTok{p\_load}\NormalTok{(stars)}

\NormalTok{krigagem\_raster }\OtherTok{\textless{}{-}} \FunctionTok{st\_rasterize}\NormalTok{(krigagem\_ord) }\SpecialCharTok{|\textgreater{}}
  \FunctionTok{st\_crop}\NormalTok{( limite\_sf)  }\CommentTok{\#corta apenas a area do shapfile}
\CommentTok{\#}
\NormalTok{g1 }\OtherTok{\textless{}{-}} \FunctionTok{ggplot}\NormalTok{() }\SpecialCharTok{+}
  \FunctionTok{geom\_stars}\NormalTok{(}\AttributeTok{data =}\NormalTok{ krigagem\_raster, }\FunctionTok{aes}\NormalTok{(}\AttributeTok{fill =}\NormalTok{ var1.pred, }\AttributeTok{x =}\NormalTok{ x, }\AttributeTok{y =}\NormalTok{ y)) }\SpecialCharTok{+} 
  \FunctionTok{scale\_fill\_viridis\_c}\NormalTok{(}\AttributeTok{option =} \StringTok{"B"}\NormalTok{, }\AttributeTok{name =} \StringTok{"log(Zn)"}\NormalTok{, }\AttributeTok{na.value =} \StringTok{"transparent"}\NormalTok{) }\SpecialCharTok{+}
  \FunctionTok{geom\_sf}\NormalTok{(}\AttributeTok{data =}\NormalTok{ limite\_sf, }\AttributeTok{fill =} \ConstantTok{NA}\NormalTok{, }\AttributeTok{color =} \StringTok{"black"}\NormalTok{) }\SpecialCharTok{+}
  \FunctionTok{labs}\NormalTok{(}\AttributeTok{title =} \StringTok{"Predição (Superfície Raster)"}\NormalTok{) }\SpecialCharTok{+}
  \FunctionTok{theme\_minimal}\NormalTok{()}

\NormalTok{g2 }\OtherTok{\textless{}{-}} \FunctionTok{ggplot}\NormalTok{() }\SpecialCharTok{+}
  \FunctionTok{geom\_stars}\NormalTok{(}\AttributeTok{data =}\NormalTok{ krigagem\_raster, }\FunctionTok{aes}\NormalTok{(}\AttributeTok{fill =} \FunctionTok{sqrt}\NormalTok{(var1.var), }\AttributeTok{x =}\NormalTok{ x, }\AttributeTok{y =}\NormalTok{ y)) }\SpecialCharTok{+} 
  \FunctionTok{scale\_fill\_viridis\_c}\NormalTok{(}\AttributeTok{option =} \StringTok{"B"}\NormalTok{, }\AttributeTok{name =} \StringTok{"SD"}\NormalTok{, }\AttributeTok{na.value =} \StringTok{"transparent"}\NormalTok{) }\SpecialCharTok{+}
  \FunctionTok{geom\_sf}\NormalTok{(}\AttributeTok{data =}\NormalTok{ limite\_sf, }\AttributeTok{fill =} \ConstantTok{NA}\NormalTok{, }\AttributeTok{color =} \StringTok{"black"}\NormalTok{) }\SpecialCharTok{+}
  \FunctionTok{labs}\NormalTok{(}\AttributeTok{title =} \StringTok{"Incerteza (Desvio Padrão)"}\NormalTok{) }\SpecialCharTok{+}
  \FunctionTok{theme\_minimal}\NormalTok{()}\SpecialCharTok{+}
  \FunctionTok{theme}\NormalTok{(}\AttributeTok{axis.title =} \FunctionTok{element\_blank}\NormalTok{()) }\CommentTok{\#remove os eixos x e y (veja no da esquerda existem porque não removemos)}

\NormalTok{g1; g2}
\end{Highlighting}
\end{Shaded}

\begin{figure}

\begin{minipage}{0.50\linewidth}

\begin{figure}[H]

\centering{

\pandocbounded{\includegraphics[keepaspectratio]{geostat_files/figure-pdf/fig-kriging-maps-1.pdf}}

}

\caption{\label{fig-kriging-maps-1}Mapas de Predição e Incerteza (Desvio
Padrão)}

\end{figure}%

\end{minipage}%
%
\begin{minipage}{0.50\linewidth}

\begin{figure}[H]

\centering{

\pandocbounded{\includegraphics[keepaspectratio]{geostat_files/figure-pdf/fig-kriging-maps-2.pdf}}

}

\caption{\label{fig-kriging-maps-2}Mapas de Predição e Incerteza (Desvio
Padrão)}

\end{figure}%

\end{minipage}%

\end{figure}%

Alternativamente poderia fazer:

\begin{Shaded}
\begin{Highlighting}[]
\CommentTok{\# st\_bbox pega os limites da área}
\NormalTok{bb }\OtherTok{\textless{}{-}} \FunctionTok{st\_bbox}\NormalTok{(limite\_sf) }\CommentTok{\# onde limite\_sf é seu shapfile}

\CommentTok{\# Criar objeto stars vazio com resolução de 40m}
\NormalTok{grid\_raster }\OtherTok{\textless{}{-}} \FunctionTok{st\_as\_stars}\NormalTok{(bb, }\AttributeTok{dx =} \DecValTok{40}\NormalTok{, }\AttributeTok{dy =} \DecValTok{40}\NormalTok{)}

\CommentTok{\# Recortar (Crop) o raster usando o polígono limite}
\NormalTok{grid\_raster }\OtherTok{\textless{}{-}} \FunctionTok{st\_crop}\NormalTok{(grid\_raster, limite\_sf)}


\NormalTok{krigagem\_direta }\OtherTok{\textless{}{-}} \FunctionTok{krige}\NormalTok{(}\FunctionTok{log}\NormalTok{(zinc) }\SpecialCharTok{\textasciitilde{}} \DecValTok{1}\NormalTok{, }
                         \AttributeTok{locations =}\NormalTok{ meuse\_sf, }
                         \AttributeTok{newdata =}\NormalTok{ grid\_raster, }
                         \AttributeTok{model =}\NormalTok{ modelo\_ajustado, }\AttributeTok{debug.level =} \DecValTok{0}\NormalTok{)}

\FunctionTok{ggplot}\NormalTok{() }\SpecialCharTok{+}
  \FunctionTok{geom\_stars}\NormalTok{(}\AttributeTok{data =}\NormalTok{ krigagem\_direta, }\FunctionTok{aes}\NormalTok{(}\AttributeTok{fill =}\NormalTok{ var1.pred)) }\SpecialCharTok{+}
  \FunctionTok{scale\_fill\_viridis\_c}\NormalTok{(}\AttributeTok{option =} \StringTok{"plasma"}\NormalTok{, }\AttributeTok{name =} \StringTok{"log(Zn)"}\NormalTok{, }\AttributeTok{na.value =} \StringTok{"transparent"}\NormalTok{) }\SpecialCharTok{+}
  \FunctionTok{geom\_sf}\NormalTok{(}\AttributeTok{data =}\NormalTok{ limite\_sf, }\AttributeTok{fill =} \ConstantTok{NA}\NormalTok{, }\AttributeTok{color =} \StringTok{"black"}\NormalTok{, }\AttributeTok{size =} \FloatTok{0.8}\NormalTok{) }\SpecialCharTok{+}
  \FunctionTok{labs}\NormalTok{(}\AttributeTok{title =} \StringTok{"Krigagem Ordinária"}\NormalTok{) }\SpecialCharTok{+}
  \FunctionTok{theme\_void}\NormalTok{() }\SpecialCharTok{+}
  \FunctionTok{coord\_sf}\NormalTok{(}\AttributeTok{expand =} \ConstantTok{FALSE}\NormalTok{) }\CommentTok{\# Remove espaços em branco extras nas margens}
\end{Highlighting}
\end{Shaded}

\pandocbounded{\includegraphics[keepaspectratio]{geostat_files/figure-pdf/unnamed-chunk-8-1.pdf}}

\textbf{Validação Cruzada:} \texttt{krige.cv}

Para aferir a qualidade preditiva do modelo, utiliza-se a função
\texttt{krige.cv}. Esta função executa o procedimento
\texttt{leave-one-out} (ou k-fold), que remove um ponto, estima-o com os
vizinhos, compara o real com o estimado, repete para todos.

\begin{itemize}
\item
  Sintaxe: \texttt{krige.cv(formula,\ locations,\ model,\ nfold,\ ...)}
\item
  Argumentos:

  \begin{itemize}
  \tightlist
  \item
    \texttt{nfold:} Se omitido ou igual ao número de observações, faz
    \texttt{leave-one-out}. Se definido (ex: 5 ou 10), faz validação
    cruzada em k-partes.
  \end{itemize}
\end{itemize}

\begin{Shaded}
\begin{Highlighting}[]
\NormalTok{validacao }\OtherTok{\textless{}{-}} \FunctionTok{krige.cv}\NormalTok{(}\FunctionTok{log}\NormalTok{(zinc) }\SpecialCharTok{\textasciitilde{}} \DecValTok{1}\NormalTok{, }\AttributeTok{locations =}\NormalTok{ meuse\_sf, }\AttributeTok{model =}\NormalTok{ modelo\_ajustado, }\AttributeTok{debug.level =} \DecValTok{0}\NormalTok{)}
\end{Highlighting}
\end{Shaded}

\textbf{Extração de Métricas de Diagnóstico}

\begin{Shaded}
\begin{Highlighting}[]
\CommentTok{\# Resíduo = Observado {-} Predito}
\CommentTok{\# Z{-}score = Resíduo / Desvio Padrão da Krigagem}

\NormalTok{metricas }\OtherTok{\textless{}{-}}\NormalTok{ validacao }\SpecialCharTok{|\textgreater{}}
  \FunctionTok{st\_drop\_geometry}\NormalTok{() }\SpecialCharTok{|\textgreater{}}
  \FunctionTok{summarise}\NormalTok{(}
    \AttributeTok{ME =} \FunctionTok{mean}\NormalTok{(residual),              }\CommentTok{\# Erro Médio (Viés): Ideal {-}\textgreater{} 0}
    \AttributeTok{RMSE =} \FunctionTok{sqrt}\NormalTok{(}\FunctionTok{mean}\NormalTok{(residual}\SpecialCharTok{\^{}}\DecValTok{2}\NormalTok{)),    }\CommentTok{\# Acurácia: Ideal {-}\textgreater{} Baixo}
    \AttributeTok{MSNE =} \FunctionTok{mean}\NormalTok{(zscore),              }\CommentTok{\# Viés Normalizado: Ideal {-}\textgreater{} 0}
    \AttributeTok{RMSNE =} \FunctionTok{sqrt}\NormalTok{(}\FunctionTok{mean}\NormalTok{(zscore}\SpecialCharTok{\^{}}\DecValTok{2}\NormalTok{))      }\CommentTok{\# Consistência da Variância: Ideal {-}\textgreater{} 1}
\NormalTok{  )}


\NormalTok{metricas}\SpecialCharTok{|\textgreater{}}
\NormalTok{  knitr}\SpecialCharTok{::}\FunctionTok{kable}\NormalTok{()}
\end{Highlighting}
\end{Shaded}

\begin{longtable}[]{@{}rrrr@{}}
\toprule\noalign{}
ME & RMSE & MSNE & RMSNE \\
\midrule\noalign{}
\endhead
\bottomrule\noalign{}
\endlastfoot
-0.0004748 & 0.3978521 & -0.0003306 & 0.8972854 \\
\end{longtable}

\textbf{Simulação Estocástica}

A Krigagem suaviza a realidade. Para aplicações que exigem a reprodução
da textura real da variabilidade (ex: fluxo em meios porosos, análise de
risco ambiental), utiliza-se simulação.

Basta adicionar o argumento \texttt{nsim} e definir \texttt{nmax} (para
Simulação Sequencial Gaussiana local).

\begin{Shaded}
\begin{Highlighting}[]
\CommentTok{\# nsim = 4 gera quatro mapas possíveis da realidade}
\NormalTok{simulacoes }\OtherTok{\textless{}{-}} \FunctionTok{krige}\NormalTok{(}\FunctionTok{log}\NormalTok{(zinc) }\SpecialCharTok{\textasciitilde{}} \DecValTok{1}\NormalTok{, }
                    \AttributeTok{locations =}\NormalTok{ meuse\_sf, }
                    \AttributeTok{newdata =}\NormalTok{ grid\_stars, }
                    \AttributeTok{model =}\NormalTok{ modelo\_ajustado, }
                    \AttributeTok{nsim =} \DecValTok{4}\NormalTok{, }
                    \AttributeTok{nmax =} \DecValTok{30}\NormalTok{, }\AttributeTok{debug.level =} \DecValTok{0}\NormalTok{)}

\NormalTok{dados\_df }\OtherTok{\textless{}{-}} \FunctionTok{as.data.frame}\NormalTok{(}\FunctionTok{merge}\NormalTok{(simulacoes))}

\FunctionTok{ggplot}\NormalTok{() }\SpecialCharTok{+}
  \FunctionTok{geom\_stars}\NormalTok{(}\AttributeTok{data =} \FunctionTok{merge}\NormalTok{(simulacoes)) }\SpecialCharTok{+}
\FunctionTok{geom\_contour}\NormalTok{(}\AttributeTok{data=}\NormalTok{dados\_df,}\FunctionTok{aes}\NormalTok{(}\AttributeTok{x=}\NormalTok{x, }\AttributeTok{y=}\NormalTok{y,}\AttributeTok{z=}\NormalTok{var1),}\AttributeTok{color=}\StringTok{"white"}\NormalTok{,}\AttributeTok{size=}\FloatTok{0.2}\NormalTok{,}\AttributeTok{alpha=}\FloatTok{0.5}\NormalTok{)}\SpecialCharTok{+} \CommentTok{\#linhas de contorno (isolines)}
  \FunctionTok{facet\_wrap}\NormalTok{(}\SpecialCharTok{\textasciitilde{}}\NormalTok{attributes) }\SpecialCharTok{+}
  \FunctionTok{geom\_sf}\NormalTok{(}\AttributeTok{data =}\NormalTok{ limite\_sf, }\AttributeTok{fill =} \ConstantTok{NA}\NormalTok{, }\AttributeTok{color =} \StringTok{"black"}\NormalTok{, }\AttributeTok{size =} \FloatTok{0.5}\NormalTok{) }\SpecialCharTok{+}
  \FunctionTok{scale\_fill\_viridis\_c}\NormalTok{(}\AttributeTok{option =} \StringTok{"inferno"}\NormalTok{, }\AttributeTok{name =} \StringTok{"log(Zn)"}\NormalTok{, }\AttributeTok{na.value =} \StringTok{"transparent"}\NormalTok{) }\SpecialCharTok{+}
  \FunctionTok{theme\_minimal}\NormalTok{() }\SpecialCharTok{+}
  \FunctionTok{labs}\NormalTok{(}\AttributeTok{title =} \StringTok{"Simulação com Curvas de Nível"}\NormalTok{)}\SpecialCharTok{+}
  \FunctionTok{theme}\NormalTok{(}\AttributeTok{axis.title =} \FunctionTok{element\_blank}\NormalTok{())}
\end{Highlighting}
\end{Shaded}

\begin{figure}[H]

\centering{

\pandocbounded{\includegraphics[keepaspectratio]{geostat_files/figure-pdf/fig-sim-1.pdf}}

}

\caption{\label{fig-sim}Cenários equiprováveis (Realizações)}

\end{figure}%

\section{Pacote automap}\label{pacote-automap}

Enquanto no pacote \texttt{gstat} o usuário deve fornecer estimativas
iniciais (chutes) para os parâmetros do variograma
\texttt{(patamar,\ alcance\ e\ efeito\ pepita)} e testar manualmente
diferentes funções de covariância
\texttt{(Esférico,\ Exponencial,\ Matérn,\ etc.)}, o pacote
\texttt{automap} foi desenvolvido para automatizar essas etapas de
ajuste e krigagem.

Baseado na metodologia descrita por Hiemstra et al. (2008), o pacote é
ideal para situações onde se deseja realizar interpolação espacial sem a
necessidade de definir manualmente os parâmetros iniciais, ou quando é
necessário processar múltiplos conjuntos de dados em lote. O pacote
estima os valores iniciais a partir dos dados e itera sobre diferentes
modelos para encontrar o melhor ajuste baseado na menor soma dos
quadrados dos resíduos.

\textbf{Ajuste Automático do Variograma}: \texttt{autofitVariogram}

Esta função elimina a necessidade de tentativa e erro manual. Ao
contrário da função \texttt{fit.variogram} do \texttt{gstat}, que exige
parâmetros iniciais, a \texttt{autofitVariogram} calcula esses valores
automaticamente: o alcance inicial é definido como 0,10 vezes a diagonal
da área dos dados (\texttt{bounding\ box}), o \texttt{efeito\ pepita}
inicial é o mínimo da semivariância amostral, e o \texttt{patamar} é uma
média entre o máximo e a mediana da semivariância . O algoritmo então
ajusta e testa automaticamente os modelos Esférico, Exponencial,
Gaussiano e Stein (Matérn), retornando aquele com o melhor ajuste
estatístico . No canto inferior direito mostra o modelo ajustado e
respetivos parâmetros.

\begin{Shaded}
\begin{Highlighting}[]
\NormalTok{pacman}\SpecialCharTok{::}\FunctionTok{p\_load}\NormalTok{(automap, sf, gstat)}

\FunctionTok{data}\NormalTok{(meuse)}
\NormalTok{meuse\_sf }\OtherTok{\textless{}{-}} \FunctionTok{st\_as\_sf}\NormalTok{(meuse, }\AttributeTok{coords =} \FunctionTok{c}\NormalTok{(}\StringTok{"x"}\NormalTok{, }\StringTok{"y"}\NormalTok{), }\AttributeTok{crs =} \DecValTok{28992}\NormalTok{)}

\NormalTok{variograma\_auto }\OtherTok{\textless{}{-}} \FunctionTok{autofitVariogram}\NormalTok{(}\FunctionTok{log}\NormalTok{(zinc) }\SpecialCharTok{\textasciitilde{}} \DecValTok{1}\NormalTok{, }\AttributeTok{input\_data =} \FunctionTok{as}\NormalTok{(meuse\_sf, }\StringTok{"Spatial"}\NormalTok{))}

\FunctionTok{plot}\NormalTok{(variograma\_auto)}
\end{Highlighting}
\end{Shaded}

\pandocbounded{\includegraphics[keepaspectratio]{geostat_files/figure-pdf/unnamed-chunk-11-1.pdf}}

\textbf{Interpolação Automática}: \texttt{autoKrige}

A função \texttt{autoKrige} é chama internamente a
\texttt{autofitVariogram} para ajustar o modelo e, em seguida, utiliza
esse modelo otimizado para realizar a predição espacial nos novos locais
. Isso resolve o problema de ter que passar manualmente os parâmetros do
variograma para a função de krigagem. Ela suporta Krigagem Ordinária
(padrão), Universal (inserindo covariáveis na fórmula) e Krigagem de
Bloco .

\begin{Shaded}
\begin{Highlighting}[]
\NormalTok{pacman}\SpecialCharTok{::}\FunctionTok{p\_load}\NormalTok{(stars, sp)}
\FunctionTok{data}\NormalTok{(meuse)}
\FunctionTok{data}\NormalTok{(meuse.grid)}

\NormalTok{meuse\_sf }\OtherTok{\textless{}{-}} \FunctionTok{st\_as\_sf}\NormalTok{(meuse, }\AttributeTok{coords =} \FunctionTok{c}\NormalTok{(}\StringTok{"x"}\NormalTok{, }\StringTok{"y"}\NormalTok{), }\AttributeTok{crs =} \DecValTok{28992}\NormalTok{)}
\NormalTok{meuse\_grid\_sf }\OtherTok{\textless{}{-}} \FunctionTok{st\_as\_sf}\NormalTok{(meuse.grid, }\AttributeTok{coords =} \FunctionTok{c}\NormalTok{(}\StringTok{"x"}\NormalTok{, }\StringTok{"y"}\NormalTok{), }\AttributeTok{crs =} \DecValTok{28992}\NormalTok{)}
\NormalTok{meuse\_grid\_stars }\OtherTok{\textless{}{-}} \FunctionTok{st\_rasterize}\NormalTok{(meuse\_grid\_sf, }\AttributeTok{dx =} \DecValTok{40}\NormalTok{, }\AttributeTok{dy =} \DecValTok{40}\NormalTok{)}

\NormalTok{krigagem }\OtherTok{\textless{}{-}} \FunctionTok{autoKrige}\NormalTok{(}\FunctionTok{log}\NormalTok{(zinc)}\SpecialCharTok{\textasciitilde{}}\DecValTok{1}\NormalTok{, }
                      \AttributeTok{input\_data =}\NormalTok{ meuse\_sf, }
                      \AttributeTok{new\_data =}\NormalTok{ meuse\_grid\_stars, }\AttributeTok{debug.level =} \DecValTok{0}\NormalTok{)}

\FunctionTok{plot}\NormalTok{(krigagem)}
\end{Highlighting}
\end{Shaded}

\pandocbounded{\includegraphics[keepaspectratio]{geostat_files/figure-pdf/unnamed-chunk-12-1.pdf}}

\textbf{Extraindo o resultado para plotar com ggplot}

\begin{Shaded}
\begin{Highlighting}[]
\NormalTok{pacman}\SpecialCharTok{::}\FunctionTok{p\_load}\NormalTok{(patchwork)}

\NormalTok{resultado\_sf }\OtherTok{\textless{}{-}} \FunctionTok{st\_as\_sf}\NormalTok{(krigagem}\SpecialCharTok{$}\NormalTok{krige\_output)}

\NormalTok{p1 }\OtherTok{\textless{}{-}} \FunctionTok{ggplot}\NormalTok{(resultado\_sf) }\SpecialCharTok{+}
  \FunctionTok{geom\_sf}\NormalTok{(}\FunctionTok{aes}\NormalTok{(}\AttributeTok{fill =}\NormalTok{ var1.pred), }\AttributeTok{color =} \ConstantTok{NA}\NormalTok{) }\SpecialCharTok{+} \CommentTok{\# color=NA é crucial aqui}
  \FunctionTok{scale\_fill\_viridis\_c}\NormalTok{(}\AttributeTok{option =} \StringTok{"plasma"}\NormalTok{, }\AttributeTok{name =} \StringTok{"Log(Zn)"}\NormalTok{) }\SpecialCharTok{+}
  \FunctionTok{labs}\NormalTok{(}\AttributeTok{title =} \StringTok{"Predição"}\NormalTok{) }\SpecialCharTok{+} 
  \FunctionTok{theme\_void}\NormalTok{()}\SpecialCharTok{+}
  \FunctionTok{theme}\NormalTok{(}\AttributeTok{plot.title =} \FunctionTok{element\_text}\NormalTok{(}\AttributeTok{hjust =} \FloatTok{0.5}\NormalTok{))}

\NormalTok{p2 }\OtherTok{\textless{}{-}} \FunctionTok{ggplot}\NormalTok{(resultado\_sf) }\SpecialCharTok{+}
  \FunctionTok{geom\_sf}\NormalTok{(}\FunctionTok{aes}\NormalTok{(}\AttributeTok{fill =} \FunctionTok{sqrt}\NormalTok{(var1.var)), }\AttributeTok{color =} \ConstantTok{NA}\NormalTok{) }\SpecialCharTok{+} 
  \FunctionTok{scale\_fill\_viridis\_c}\NormalTok{(}\AttributeTok{option =} \StringTok{"cividis"}\NormalTok{, }\AttributeTok{name =} \StringTok{"SD"}\NormalTok{) }\SpecialCharTok{+}
  \FunctionTok{labs}\NormalTok{(}\AttributeTok{title =} \StringTok{"Erro Padrão"}\NormalTok{) }\SpecialCharTok{+} 
  \FunctionTok{theme\_minimal}\NormalTok{()}\SpecialCharTok{+}
  \FunctionTok{theme}\NormalTok{(}\AttributeTok{plot.title =} \FunctionTok{element\_text}\NormalTok{(}\AttributeTok{hjust =} \FloatTok{0.5}\NormalTok{))}

\NormalTok{p1 }\SpecialCharTok{+}\NormalTok{ p2}
\end{Highlighting}
\end{Shaded}

\pandocbounded{\includegraphics[keepaspectratio]{geostat_files/figure-pdf/unnamed-chunk-13-1.pdf}}

\textbf{Validação Cruzada Automática}: \texttt{autoKrige.cv}

Para garantir que a automação não comprometeu a qualidade da predição, a
função \texttt{autoKrige.cv} realiza a validação cruzada. Ela ajusta o
variograma automaticamente e aplica a validação
(\texttt{leave-one-out\ ou\ k-fold}) usando a função \texttt{krige.cv}
do \texttt{gstat}. Isso permite verificar rapidamente se a estratégia
automática está gerando resíduos aceitáveis sem a necessidade de
configurar loops manuais de validação.

Para avaliar a qualidade do modelo ajustado automaticamente, utiliza-se
a autoKrige.cv. Esta função ajusta o variograma aos dados e, em seguida,
utiliza a função krige.cv do gstat para realizar a validação cruzada
(cross-validation)7. Ela suporta tanto o método leave-one-out quanto o
k-fold (validar subconjuntos de dados) através do argumento nfold8.

\begin{Shaded}
\begin{Highlighting}[]
\CommentTok{\# Validação cruzada automática (10{-}fold) para KO}
\NormalTok{cv\_ordinaria }\OtherTok{\textless{}{-}} \FunctionTok{autoKrige.cv}\NormalTok{(}\FunctionTok{log}\NormalTok{(zinc) }\SpecialCharTok{\textasciitilde{}} \DecValTok{1}\NormalTok{,}
                             \AttributeTok{input\_data =} \FunctionTok{as}\NormalTok{(meuse\_sf, }\StringTok{"Spatial"}\NormalTok{),}
                             \AttributeTok{nfold =} \DecValTok{10}\NormalTok{)}

\CommentTok{\# Validação cruzada automática para Krigagem Universal (usando distância como covariável)}
\NormalTok{cv\_universal }\OtherTok{\textless{}{-}} \FunctionTok{autoKrige.cv}\NormalTok{(}\FunctionTok{log}\NormalTok{(zinc) }\SpecialCharTok{\textasciitilde{}} \FunctionTok{sqrt}\NormalTok{(dist),}
                             \AttributeTok{input\_data =} \FunctionTok{as}\NormalTok{(meuse\_sf, }\StringTok{"Spatial"}\NormalTok{),}
                             \AttributeTok{nfold =} \DecValTok{10}\NormalTok{, }\AttributeTok{debug.level =} \DecValTok{0}\NormalTok{)}

\CommentTok{\# Resumo dos resíduos}
\FunctionTok{summary}\NormalTok{(cv\_ordinaria}\SpecialCharTok{$}\NormalTok{krige.cv\_output)}
\end{Highlighting}
\end{Shaded}

\begin{verbatim}
Object of class SpatialPointsDataFrame
Coordinates:
             min    max
coords.x1 178605 181390
coords.x2 329714 333611
Is projected: TRUE 
proj4string :
[+proj=sterea +lat_0=52.1561605555556 +lon_0=5.38763888888889
+k=0.9999079 +x_0=155000 +y_0=463000 +ellps=bessel +units=m +no_defs]
Number of points: 155
Data attributes:
   var1.pred        var1.var         observed        residual        
 Min.   :4.891   Min.   :0.1151   Min.   :4.727   Min.   :-0.969586  
 1st Qu.:5.369   1st Qu.:0.1564   1st Qu.:5.288   1st Qu.:-0.199327  
 Median :5.862   Median :0.1808   Median :5.787   Median :-0.002832  
 Mean   :5.882   Mean   :0.1895   Mean   :5.886   Mean   : 0.004181  
 3rd Qu.:6.347   3rd Qu.:0.2017   3rd Qu.:6.514   3rd Qu.: 0.218404  
 Max.   :7.265   Max.   :0.5400   Max.   :7.517   Max.   : 1.387144  
     zscore               fold       
 Min.   :-2.251356   Min.   : 1.000  
 1st Qu.:-0.467268   1st Qu.: 4.000  
 Median :-0.006538   Median : 6.000  
 Mean   : 0.010897   Mean   : 5.768  
 3rd Qu.: 0.515972   3rd Qu.: 8.000  
 Max.   : 3.108253   Max.   :10.000  
\end{verbatim}

\textbf{Comparação de Modelos}: \texttt{compare.cv}

Dado que a escolha entre Krigagem Ordinária e Universal pode ser
difícil, a função \texttt{compare.cv} permite comparar diretamente os
resultados de múltiplas validações cruzadas. Ela gera diagnósticos
estatísticos (como RMSE e Correlação) e gráficos espaciais de bolhas
(\texttt{bubble\ plots}) para identificar visualmente qual abordagem
automática produziu menores erros. O argumento \texttt{plot.diff}
destaca onde um modelo supera o outro.

\begin{Shaded}
\begin{Highlighting}[]
\NormalTok{comparacao }\OtherTok{\textless{}{-}} \FunctionTok{compare.cv}\NormalTok{(cv\_ordinaria, cv\_universal,}
                         \AttributeTok{col.names =} \FunctionTok{c}\NormalTok{(}\StringTok{"Ordinária"}\NormalTok{, }\StringTok{"Universal"}\NormalTok{),}
                         \AttributeTok{bubbleplots =} \ConstantTok{TRUE}\NormalTok{, }\CommentTok{\# Gera os gráficos na janela de plotagem}
                         \AttributeTok{plot.diff =} \ConstantTok{FALSE}\NormalTok{)   }
\end{Highlighting}
\end{Shaded}

\pandocbounded{\includegraphics[keepaspectratio]{geostat_files/figure-pdf/unnamed-chunk-15-1.pdf}}

\begin{Shaded}
\begin{Highlighting}[]
\FunctionTok{print}\NormalTok{(comparacao}\SpecialCharTok{$}\NormalTok{spatial)}
\end{Highlighting}
\end{Shaded}

\begin{verbatim}
NULL
\end{verbatim}

\begin{Shaded}
\begin{Highlighting}[]
\NormalTok{comparacao1 }\OtherTok{\textless{}{-}} \FunctionTok{compare.cv}\NormalTok{(cv\_ordinaria, cv\_universal,}
                         \AttributeTok{col.names =} \FunctionTok{c}\NormalTok{(}\StringTok{"Ordinária"}\NormalTok{, }\StringTok{"Universal"}\NormalTok{),}
                         \AttributeTok{bubbleplots =} \ConstantTok{FALSE}\NormalTok{, }
                         \AttributeTok{plot.diff =} \ConstantTok{FALSE}\NormalTok{)   }

\NormalTok{comparacao1 }\SpecialCharTok{|\textgreater{}}
\NormalTok{  knitr}\SpecialCharTok{::}\FunctionTok{kable}\NormalTok{()}
\end{Highlighting}
\end{Shaded}

\begin{longtable}[]{@{}lll@{}}
\toprule\noalign{}
& Ordinária & Universal \\
\midrule\noalign{}
\endhead
\bottomrule\noalign{}
\endlastfoot
mean\_error & 0.004181 & 0.001329 \\
me\_mean & 0.0007103 & 0.0002258 \\
MAE & 0.2933 & 0.2709 \\
MSE & 0.1551 & 0.1442 \\
MSNE & 0.8247 & 1.088 \\
cor\_obspred & 0.8378 & 0.8499 \\
cor\_predres & 0.06733 & -0.0531 \\
RMSE & 0.3938 & 0.3797 \\
RMSE\_sd & 0.5455 & 0.526 \\
URMSE & 0.3938 & 0.3797 \\
iqr & 0.4177 & 0.3999 \\
\end{longtable}

\textbf{Intervalos de Predição de Posição}:
\texttt{posPredictionInterval}

Esta função oferece uma ferramenta prática para tomada de decisão
baseada na incerteza da krigagem automática. Ela calcula a posição do
intervalo de predição (padrão 95\%) em relação a um valor limite
(cutoff). O mapa resultante classifica as áreas como potencialmente
acima, potencialmente abaixo ou indistinguível do limite, facilitando a
interpretação de riscos sem exigir cálculos manuais de intervalos de
confiança.

\begin{Shaded}
\begin{Highlighting}[]
\CommentTok{\#Extrair os resultados da krigagem e remover NA}
\NormalTok{resultado\_pontos }\OtherTok{\textless{}{-}} \FunctionTok{st\_as\_sf}\NormalTok{(krigagem}\SpecialCharTok{$}\NormalTok{krige\_output, }\AttributeTok{as\_points =} \ConstantTok{TRUE}\NormalTok{)}
\NormalTok{resultado\_limpo }\OtherTok{\textless{}{-}}\NormalTok{ resultado\_pontos[}\SpecialCharTok{!}\FunctionTok{is.na}\NormalTok{(resultado\_pontos}\SpecialCharTok{$}\NormalTok{var1.pred), ]}

\NormalTok{krigagem\_pontos }\OtherTok{\textless{}{-}}\NormalTok{ krigagem}
\NormalTok{krigagem\_pontos}\SpecialCharTok{$}\NormalTok{krige\_output }\OtherTok{\textless{}{-}}\NormalTok{ resultado\_limpo}

\NormalTok{intervalos }\OtherTok{\textless{}{-}} \FunctionTok{posPredictionInterval}\NormalTok{(krigagem\_pontos, }
                                    \AttributeTok{p =} \DecValTok{95}\NormalTok{, }
                                    \AttributeTok{value =} \FloatTok{6.0}\NormalTok{)}

\FunctionTok{plot}\NormalTok{(intervalos, }\AttributeTok{main =} \StringTok{"Classificação vs Limiar (6.0)"}\NormalTok{)}
\end{Highlighting}
\end{Shaded}

\begin{figure}[H]

\centering{

\pandocbounded{\includegraphics[keepaspectratio]{geostat_files/figure-pdf/fig-pospi-1.pdf}}

}

\caption{\label{fig-pospi}Áreas estatisticamente acima ou abaixo do
limiar (Log(Zn) = 6.0)}

\end{figure}%

\section{Pacote geoR}\label{pacote-geor}

Veja no capítulo 3 de Scalon (2024).

\part{Dados de Área (Lattice)}

\chapter{Dados de Área}\label{sec-dados_area}

A análise de dados de área (ou \emph{lattice data}) lida com processos
estocásticos cujo domínio espacial é fixo, discreto e contável.
Denotamos esse domínio por \(D^L\). Enquanto na geoestatística
(Capítulo~\ref{sec-geoest}) o suporte é contínuo, permitindo observações
em qualquer localização \(\mathbf{s} \in D^G\), nos dados de área as
observações estão ancoradas em unidades espaciais predefinidas e não
sobrepostas, como regiões administrativas, células de uma grelha ou
zonas censitárias.

Seja \(\{D_i\}_{i=1}^{n}\) uma coleção finita de \(n\) unidades
espaciais (por exemplo, municípios, distritos ou pixels). Os dados de
área são definidos como uma coleção de variáveis aleatórias indexadas
por essas unidades: \(\{Y(\mathbf{s}_i): \mathbf{s}_i \in D^L\}\), onde
\(D^L = \{\mathbf{s}_1, \dots, \mathbf{s}_n\}\) é um subconjunto fixo e
contável do espaço Euclidiano \(\mathbb{R}^d\) (Noel Cressie e Moores
2022). A incerteza reside exclusivamente no valor do atributo
\(Y(\mathbf{s}_i)\), e não na localização \(\mathbf{s}_i\) (que é fixa e
conhecida), diferenciando-se dos processos pontuais
(Capítulo~\ref{sec-proc_pont}). Para as \(n\) regiões, o vetor de
observações é \(\mathbf{y} = (y_1, \dots, y_n)^\top\), onde
\(y_i \equiv y(\mathbf{s}_i)\).

Conforme destacado por Julian Besag (1974), a dependência espacial neste
contexto não é necessariamente governada por uma métrica de distância
Euclidiana contínua, como na geoestatística (Capítulo~\ref{sec-geoest}),
mas sim pela topologia ou estrutura de vizinhança definida entre as
unidades discretas \(\{D_i\}\).

A questão inferencial central também se desloca. Em vez de interpolar
(prever) um valor em um local não observado \(s_0\) (krigagem,
Capítulo~\ref{sec-geoest}), o foco passa a ser compreender e quantificar
como o valor observado na unidade \(D_i\) é influenciado pelos valores
nas unidades vizinhas \(\{D_j\}\) (interação espacial). Por
simplicidade, as unidades são frequentemente denotadas apenas por seus
índices \(i\) e \(j\).

Uma característica fundamental dos dados de área é a agregação. O valor
observado \(y_i\) na unidade \(i\) é tipicamente o resultado da
integração (ou média) de um processo contínuo latente \(Y(\mathbf{s})\)
sobre a área geográfica \(A_i\) daquela unidade. Formalmente, se
\(Y(\mathbf{s})\) representa, por exemplo, densidade ou intensidade,
então:

\[
y_i = \int_{A_i} Y(\mathbf{s}) \, d\mathbf{s} \: \text{(para contagens ou volumes)}, \: \text{ ou }
y_i = \frac{1}{|A_i|} \int_{A_i} Y(\mathbf{s}) \, d\mathbf{s} \: \text{(para médias ou intensidades)},
\] onde \(|A_i|\) é a área da região \(i\).

Esta natureza agregada implica que a inferência estatística é
condicional à partição específica do espaço
(\(A_1 \cup \dots \cup A_n\)). Alterar essa partição (escala ou limites)
pode alterar as propriedades estatísticas (média, variância, correlação)
dos dados. Este fenômeno é conhecido como o Problema da Unidade de Área
Modificável
(\href{https://en.wikipedia.org/wiki/Modifiable_areal_unit_problem}{MAUP})
(Openshaw 1984). Estatisticamente, a agregação introduz uma
heterocedasticidade intrínseca: unidades com áreas \(|A_i|\) ou
populações-base diferentes terão variâncias de amostragem distintas, um
aspecto que deve ser cuidadosamente considerado na modelagem da matriz
de covariância \(\mathbf{\Sigma}\).

\section{Representação espacial e construção de estruturas de
vizinhança}\label{sec-lattice}

A representação dos dados de área pode se dar em estruturas regulares
(grelhas ou \emph{grids}) ou irregulares (divisões políticas ou
administrativas, Seção~\ref{sec-grid}). \emph{Grids} regulares são
comuns em análise de imagens, sensoriamento remoto e dados climáticos,
onde cada célula (pixel) tem uma forma e tamanho constantes, facilitando
a computação e a definição de vizinhança. Polígonos irregulares, que
representam entidades como municípios ou distritos, são comuns em
ciências sociais e saúde pública. A heterogeneidade no tamanho e forma
dessas regiões introduz desafios adicionais, como a já mencionada
variância desigual e a definição não trivial de proximidade (Noel
Cressie e Chan 1989).

Para modelar a dependência espacial, é fundamental definir formalmente
como as unidades se relacionam. Essa relação baseia-se na matriz de
pesos espaciais ou matriz de vizinhança:

\[
\mathbf{W}_{n \times n} =
\begin{bmatrix}
w_{11} & w_{12} & \cdots & w_{1n} \\
w_{21} & w_{22} & \cdots & w_{2n} \\
\vdots & \vdots & \ddots & \vdots \\
w_{n1} & w_{n2} & \cdots & w_{nn}
\end{bmatrix},
\]

onde cada elemento \(w_{ij}\) quantifica a conexão espacial entre a
unidade \(j\) e unidade \(i\). Por convenção, assume-se que
\(w_{ii} = 0\), impedindo que uma unidade seja vizinha de si (Anselin
2001). Note ainda que é comum descrever a vizinhança entre unidades
\(i\) e \(j\) se existe, simplesmente \(i\sim j\), para referir que
\(w_{ij} \neq 0\) (Julian Besag e Kooperberg 1995).

A construção de \(\mathbf{W}\) envolve duas etapas conceituais
distintas: 1) a definição da topologia ou critério de vizinhança (quem é
vizinho de quem); e 2) a ponderação (a intensidade atribuída a cada
conexão). Enquanto a primeira é predominantemente geométrica, a segunda
frequentemente envolve uma operação de normalização, crucial para a
estabilidade numérica e interpretabilidade dos modelos.

Diferente das séries temporais, onde a dependência é unidirecional e
sequencial (o passado influencia o futuro), nos dados de área a
dependência é multidirecional e simultânea. A unidade \(i\) influencia
\(j\), que influencia \(k\), que pode, por sua vez, influenciar \(i\)
novamente através de outras conexões, criando um sistema de
\href{https://en.wikipedia.org/wiki/Feedback}{feedback} espacial.

\subsection{Critérios de
Vizinhança}\label{crituxe9rios-de-vizinhanuxe7a}

A definição operacional de proximidade ou vizinhança é um passo
fundamental e teórico. Anselin (2002) discutem os critérios mais comuns:

\begin{enumerate}
\def\labelenumi{\arabic{enumi}.}
\item
  \textbf{Contiguidade (Adjacência):} Baseia-se no compartilhamento de
  fronteiras (ver Seção~\ref{sec-dependencia}).

  \begin{itemize}
  \item
    Torre (\emph{Rook}): As unidades \(i\) e \(j\) são vizinhas se
    compartilham um segmento de fronteira
    (\href{https://pt.wikipedia.org/wiki/Aresta}{aresta}). Formalmente,
    \(\text{dim}(\partial A_i \cap \partial A_j) = 1\).
  \item
    Rainha (\emph{Queen}): As unidades \(i\) e \(j\) são vizinhas se
    compartilham qualquer ponto de fronteira, seja um
    \href{https://pt.wikipedia.org/wiki/V\%C3\%A9rtice}{vértice} ou uma
    aresta. Formalmente, \(A_i \cap \partial A_j \neq \emptyset\). Este
    critério é mais abrangente e é particularmente útil para malhas
    irregulares, pois evita que unidades que se tocam apenas em um canto
    (como municípios separados por um rio que se encontram em uma
    \href{https://pt.wikipedia.org/wiki/Conflu\%C3\%AAncia}{confluência})
    sejam consideradas desconectadas. Por exemplo, ao estudar a
    propagação de um fenômeno social entre municípios, dois que são
    separados por um rio mas cujos centros urbanos estão próximos na
    confluência podem ter intensa interação. Usar o critério \emph{Rook}
    os trataria como isolados, enquanto o critério \emph{Queen}
    capturaria essa potencial conexão, resultando em uma matriz de
    conectividade mais robusta e evitando subestimar a dependência
    espacial.
  \end{itemize}
\end{enumerate}

\begin{Shaded}
\begin{Highlighting}[]
\ControlFlowTok{if}\NormalTok{ (}\SpecialCharTok{!}\FunctionTok{require}\NormalTok{(}\StringTok{"pacman"}\NormalTok{)) }\FunctionTok{install.packages}\NormalTok{(}\StringTok{"pacman"}\NormalTok{)}
\NormalTok{pacman}\SpecialCharTok{::}\FunctionTok{p\_load}\NormalTok{(sf, spdep, ggplot2, patchwork, dplyr, geodata)}

\CommentTok{\# Baixar dados dos EUA (Nível 1 = Estados)}
\NormalTok{usa\_sf }\OtherTok{\textless{}{-}} \FunctionTok{tryCatch}\NormalTok{(\{}
  \CommentTok{\# Tenta baixar direto}
\NormalTok{  usa\_vect }\OtherTok{\textless{}{-}}\NormalTok{ geodata}\SpecialCharTok{::}\FunctionTok{gadm}\NormalTok{(}\AttributeTok{country =} \StringTok{"USA"}\NormalTok{, }\AttributeTok{level =} \DecValTok{1}\NormalTok{, }\AttributeTok{path =} \FunctionTok{tempdir}\NormalTok{(), }\AttributeTok{version=}\StringTok{"latest"}\NormalTok{)}
\NormalTok{  sf}\SpecialCharTok{::}\FunctionTok{st\_as\_sf}\NormalTok{(usa\_vect)}
\NormalTok{\}, }\AttributeTok{error =} \ControlFlowTok{function}\NormalTok{(e) \{}
  \FunctionTok{message}\NormalTok{(}\StringTok{"Erro ao baixar dados, gadm está com problemas, baixa direto no site: https://gadm.org/maps.html."}\NormalTok{)}
\NormalTok{\})}

\CommentTok{\# FILTRAR apenas Utah, Colorado, Arizona, New Mexico}
\NormalTok{four\_corners }\OtherTok{\textless{}{-}}\NormalTok{ usa\_sf }\SpecialCharTok{\%\textgreater{}\%} 
  \FunctionTok{filter}\NormalTok{(NAME\_1 }\SpecialCharTok{\%in\%} \FunctionTok{c}\NormalTok{(}\StringTok{"Utah"}\NormalTok{, }\StringTok{"Colorado"}\NormalTok{, }\StringTok{"Arizona"}\NormalTok{, }\StringTok{"New Mexico"}\NormalTok{)) }\SpecialCharTok{\%\textgreater{}\%}
  \FunctionTok{st\_make\_valid}\NormalTok{()}

\CommentTok{\# Extrair centroides para o grafo}
\NormalTok{coords }\OtherTok{\textless{}{-}} \FunctionTok{suppressWarnings}\NormalTok{(}\FunctionTok{st\_coordinates}\NormalTok{(}\FunctionTok{st\_centroid}\NormalTok{(four\_corners))) }\CommentTok{\#suppressWarnings() era para tirar}
\NormalTok{coords\_df }\OtherTok{\textless{}{-}} \FunctionTok{as.data.frame}\NormalTok{(coords)}


\CommentTok{\# Queen (Rainha)}

\CommentTok{\#identificar quais polígonos são vizinhos e constroir lista de vizinhos}
\NormalTok{nb\_queen }\OtherTok{\textless{}{-}} \FunctionTok{poly2nb}\NormalTok{(four\_corners, }\AttributeTok{queen =} \ConstantTok{TRUE}\NormalTok{)}

\CommentTok{\#criar segmentos de reta ligando os centroides das áreas vizinhas}
\NormalTok{nb\_lines\_queen }\OtherTok{\textless{}{-}} \FunctionTok{nb2lines}\NormalTok{(nb\_queen, }\AttributeTok{coords =}\NormalTok{ coords, }\AttributeTok{as\_sf =} \ConstantTok{TRUE}\NormalTok{)}
\FunctionTok{st\_crs}\NormalTok{(nb\_lines\_queen) }\OtherTok{\textless{}{-}} \FunctionTok{st\_crs}\NormalTok{(four\_corners) }\CommentTok{\# atribuir a nb\_lines\_queen CRS igual do four\_corners}

\CommentTok{\# Rook (Torre)}
\NormalTok{nb\_rook }\OtherTok{\textless{}{-}} \FunctionTok{poly2nb}\NormalTok{(four\_corners, }\AttributeTok{queen =} \ConstantTok{FALSE}\NormalTok{)}
\NormalTok{nb\_lines\_rook }\OtherTok{\textless{}{-}} \FunctionTok{nb2lines}\NormalTok{(nb\_rook, }\AttributeTok{coords =}\NormalTok{ coords, }\AttributeTok{as\_sf =} \ConstantTok{TRUE}\NormalTok{)}
\FunctionTok{st\_crs}\NormalTok{(nb\_lines\_rook) }\OtherTok{\textless{}{-}} \FunctionTok{st\_crs}\NormalTok{(four\_corners)}

\NormalTok{theme\_comp }\OtherTok{\textless{}{-}} \FunctionTok{theme\_void}\NormalTok{() }\SpecialCharTok{+}
  \FunctionTok{theme}\NormalTok{(}\AttributeTok{plot.title =} \FunctionTok{element\_text}\NormalTok{(}\AttributeTok{hjust =} \FloatTok{0.5}\NormalTok{, }\AttributeTok{face =} \StringTok{"bold"}\NormalTok{),}
        \AttributeTok{legend.position =} \StringTok{"bottom"}\NormalTok{)}\SpecialCharTok{+}
  \FunctionTok{theme}\NormalTok{(}\AttributeTok{legedn.title=}\FunctionTok{element\_text}\NormalTok{(}\AttributeTok{hjust=}\FloatTok{0.5}\NormalTok{))}

\CommentTok{\# Queen}
\NormalTok{p\_queen }\OtherTok{\textless{}{-}} \FunctionTok{ggplot}\NormalTok{() }\SpecialCharTok{+}
  \FunctionTok{geom\_sf}\NormalTok{(}\AttributeTok{data =}\NormalTok{ four\_corners, }\AttributeTok{fill =} \StringTok{"white"}\NormalTok{, }\AttributeTok{color =} \StringTok{"gray20"}\NormalTok{, }\AttributeTok{linewidth =} \FloatTok{0.5}\NormalTok{) }\SpecialCharTok{+}
  \FunctionTok{geom\_sf}\NormalTok{(}\AttributeTok{data =}\NormalTok{ nb\_lines\_queen, }\FunctionTok{aes}\NormalTok{(}\AttributeTok{color =} \StringTok{"Conexão (Queen)"}\NormalTok{), }\AttributeTok{linewidth =} \FloatTok{1.2}\NormalTok{) }\SpecialCharTok{+}
  \FunctionTok{geom\_point}\NormalTok{(}\AttributeTok{data =}\NormalTok{ coords\_df, }\FunctionTok{aes}\NormalTok{(X, Y), }\AttributeTok{size =} \DecValTok{3}\NormalTok{) }\SpecialCharTok{+}
  \FunctionTok{geom\_sf\_text}\NormalTok{(}\AttributeTok{data =}\NormalTok{ four\_corners, }\FunctionTok{aes}\NormalTok{(}\AttributeTok{label =}\NormalTok{ NAME\_1), }\AttributeTok{size =} \DecValTok{3}\NormalTok{, }\AttributeTok{nudge\_y =} \SpecialCharTok{{-}}\FloatTok{0.5}\NormalTok{, }\AttributeTok{nudge\_x=}\DecValTok{1}\NormalTok{) }\SpecialCharTok{+}
  \FunctionTok{scale\_color\_manual}\NormalTok{(}\AttributeTok{values =} \StringTok{"steelblue"}\NormalTok{, }\AttributeTok{name =} \StringTok{""}\NormalTok{) }\SpecialCharTok{+}
  \FunctionTok{labs}\NormalTok{(}\AttributeTok{title =} \StringTok{"Critério Rainha (Queen)"}\NormalTok{, }
       \AttributeTok{subtitle =} \StringTok{"Utah, Colorado, Arizona, New Mexico}\SpecialCharTok{\textbackslash{}n}\StringTok{ tem 1 ponto em comum"}\NormalTok{) }\SpecialCharTok{+}
\NormalTok{  theme\_comp}

\CommentTok{\# Rook}
\NormalTok{p\_rook }\OtherTok{\textless{}{-}} \FunctionTok{ggplot}\NormalTok{() }\SpecialCharTok{+}
  \FunctionTok{geom\_sf}\NormalTok{(}\AttributeTok{data =}\NormalTok{ four\_corners, }\AttributeTok{fill =} \StringTok{"white"}\NormalTok{, }\AttributeTok{color =} \StringTok{"gray20"}\NormalTok{, }\AttributeTok{linewidth =} \FloatTok{0.5}\NormalTok{) }\SpecialCharTok{+}
  \FunctionTok{geom\_sf}\NormalTok{(}\AttributeTok{data =}\NormalTok{ nb\_lines\_rook, }\FunctionTok{aes}\NormalTok{(}\AttributeTok{color =} \StringTok{"Conexão (Rook)"}\NormalTok{), }\AttributeTok{linewidth =} \FloatTok{1.2}\NormalTok{) }\SpecialCharTok{+}
  \FunctionTok{geom\_point}\NormalTok{(}\AttributeTok{data =}\NormalTok{ coords\_df, }\FunctionTok{aes}\NormalTok{(X, Y), }\AttributeTok{size =} \DecValTok{3}\NormalTok{) }\SpecialCharTok{+}
  \FunctionTok{geom\_sf\_text}\NormalTok{(}\AttributeTok{data =}\NormalTok{ four\_corners, }\FunctionTok{aes}\NormalTok{(}\AttributeTok{label =}\NormalTok{ NAME\_1), }\AttributeTok{size =} \DecValTok{3}\NormalTok{, }\AttributeTok{nudge\_y =} \SpecialCharTok{{-}}\FloatTok{0.5}\NormalTok{, }\AttributeTok{nudge\_x=}\DecValTok{1}\NormalTok{) }\SpecialCharTok{+}
  \FunctionTok{scale\_color\_manual}\NormalTok{(}\AttributeTok{values =} \StringTok{"firebrick"}\NormalTok{, }\AttributeTok{name =} \StringTok{""}\NormalTok{) }\SpecialCharTok{+}
  \FunctionTok{labs}\NormalTok{(}\AttributeTok{title =} \StringTok{"Critério Torre (Rook)"}\NormalTok{, }
       \AttributeTok{subtitle =} \StringTok{"Utah e New Mexico; Colorado e Arizona}\SpecialCharTok{\textbackslash{}n}\StringTok{ śo tem 1 ponto em comum"}\NormalTok{) }\SpecialCharTok{+}
\NormalTok{  theme\_comp}

\NormalTok{p\_queen }\SpecialCharTok{+}\NormalTok{ p\_rook}
\end{Highlighting}
\end{Shaded}

\begin{figure}[H]

\centering{

\pandocbounded{\includegraphics[keepaspectratio]{lattice_data_files/figure-pdf/fig-vizinhanca-usa-fourcorners-1.pdf}}

}

\caption{\label{fig-vizinhanca-usa-fourcorners}Comparação das Estruturas
de Vizinhança Queen (Rainha) e Rook (Torre) entre Utah, Colorado,
Arizona, New Mexico (Estados Unidos)}

\end{figure}%

\begin{enumerate}
\def\labelenumi{\arabic{enumi}.}
\setcounter{enumi}{1}
\item
  \textbf{Baseado em distância:}

  \begin{itemize}
  \tightlist
  \item
    \(k\)-Vizinhos mais próximos (\(k\)-NN): Define como vizinhos de
    \(i\), as \(k\) unidades cujos centroides (ou outro ponto
    representativo) estão mais próximos, segundo a distância euclidiana.
    Garante que cada unidade tenha exatamente \(k\) vizinhos, criando
    uma matriz esparsa e evitando ilhas de desconexão (Anselin 2001).
  \end{itemize}
\end{enumerate}

\begin{Shaded}
\begin{Highlighting}[]
\ControlFlowTok{if}\NormalTok{ (}\SpecialCharTok{!}\FunctionTok{require}\NormalTok{(}\StringTok{"pacman"}\NormalTok{)) }\FunctionTok{install.packages}\NormalTok{(}\StringTok{"pacman"}\NormalTok{)}
\NormalTok{pacman}\SpecialCharTok{::}\FunctionTok{p\_load}\NormalTok{(sf, spdep, ggplot2, geobr, dplyr, patchwork)}

\CommentTok{\# Baixar mapa municipal de Mato Grosso (MT)}
\NormalTok{mt\_sf }\OtherTok{\textless{}{-}} \FunctionTok{read\_municipality}\NormalTok{(}\AttributeTok{code\_muni =} \StringTok{"MT"}\NormalTok{, }\AttributeTok{year =} \DecValTok{2020}\NormalTok{, }\AttributeTok{showProgress =} \ConstantTok{FALSE}\NormalTok{)}

\NormalTok{coords\_mt }\OtherTok{\textless{}{-}} \FunctionTok{suppressWarnings}\NormalTok{(}\FunctionTok{st\_coordinates}\NormalTok{(}\FunctionTok{st\_centroid}\NormalTok{(mt\_sf)))}

\NormalTok{theme\_map }\OtherTok{\textless{}{-}} \FunctionTok{theme\_void}\NormalTok{() }\SpecialCharTok{+}
  \FunctionTok{theme}\NormalTok{(}\AttributeTok{plot.title =} \FunctionTok{element\_text}\NormalTok{(}\AttributeTok{hjust =} \FloatTok{0.5}\NormalTok{, }\AttributeTok{face =} \StringTok{"bold"}\NormalTok{, }\AttributeTok{size =} \DecValTok{12}\NormalTok{),}
        \AttributeTok{plot.subtitle =} \FunctionTok{element\_text}\NormalTok{(}\AttributeTok{hjust =} \FloatTok{0.5}\NormalTok{, }\AttributeTok{size =} \DecValTok{14}\NormalTok{))}
\end{Highlighting}
\end{Shaded}

\begin{Shaded}
\begin{Highlighting}[]
\CommentTok{\# Calcular os k=4 vizinhos mais próximos}
\NormalTok{k }\OtherTok{\textless{}{-}} \DecValTok{4}
\NormalTok{knn\_nb }\OtherTok{\textless{}{-}} \FunctionTok{knearneigh}\NormalTok{(coords\_mt, }\AttributeTok{k =}\NormalTok{ k)}
\NormalTok{nb\_knn }\OtherTok{\textless{}{-}} \FunctionTok{knn2nb}\NormalTok{(knn\_nb)}

\CommentTok{\# Converter para linhas espaciais para plotar}
\NormalTok{lines\_knn }\OtherTok{\textless{}{-}} \FunctionTok{nb2lines}\NormalTok{(nb\_knn, }\AttributeTok{coords =}\NormalTok{ coords\_mt, }\AttributeTok{as\_sf =} \ConstantTok{TRUE}\NormalTok{)}
\FunctionTok{st\_crs}\NormalTok{(lines\_knn) }\OtherTok{\textless{}{-}} \FunctionTok{st\_crs}\NormalTok{(mt\_sf)}

\CommentTok{\# Plot}
\FunctionTok{ggplot}\NormalTok{() }\SpecialCharTok{+}
  \FunctionTok{geom\_sf}\NormalTok{(}\AttributeTok{data =}\NormalTok{ mt\_sf, }\AttributeTok{fill =} \StringTok{"gray95"}\NormalTok{, }\AttributeTok{color =} \StringTok{"gray80"}\NormalTok{) }\SpecialCharTok{+}
  \FunctionTok{geom\_sf}\NormalTok{(}\AttributeTok{data =}\NormalTok{ lines\_knn, }\AttributeTok{color =} \StringTok{"purple"}\NormalTok{, }\AttributeTok{linewidth =} \FloatTok{0.5}\NormalTok{, }\AttributeTok{alpha =} \FloatTok{0.6}\NormalTok{) }\SpecialCharTok{+}
  \FunctionTok{geom\_point}\NormalTok{(}\AttributeTok{data =} \FunctionTok{as.data.frame}\NormalTok{(coords\_mt), }\FunctionTok{aes}\NormalTok{(X, Y), }\AttributeTok{size =} \FloatTok{0.8}\NormalTok{) }\SpecialCharTok{+}
  \FunctionTok{labs}\NormalTok{(}\AttributeTok{title =} \FunctionTok{paste0}\NormalTok{(}\StringTok{"k{-}Vizinhos Mais Próximos (k="}\NormalTok{, k, }\StringTok{")"}\NormalTok{),}
       \AttributeTok{subtitle =} \StringTok{"Cada município conecta{-}se aos 4 centroides mais próximos"}\NormalTok{) }\SpecialCharTok{+}
\NormalTok{  theme\_map}
\end{Highlighting}
\end{Shaded}

\begin{figure}[H]

\centering{

\pandocbounded{\includegraphics[keepaspectratio]{lattice_data_files/figure-pdf/fig-knn-mt-1.pdf}}

}

\caption{\label{fig-knn-mt}Vizinhança k-NN (k=4) em Mato Grosso.}

\end{figure}%

\begin{verbatim}
-   Limiar de distância (*Threshold*): $w_{ij} = 1$ se $d_{ij} \le d_{\max}$, e $0$ caso contrário, onde $d_{ij}$ é a distância entre centroides.
\end{verbatim}

\begin{Shaded}
\begin{Highlighting}[]
\CommentTok{\# Para precisão, vamos projetar para SIRGAS 2000 / Brazil Polyconic (EPSG 5880) para usar metros.}
\NormalTok{mt\_proj }\OtherTok{\textless{}{-}} \FunctionTok{st\_transform}\NormalTok{(mt\_sf, }\DecValTok{5880}\NormalTok{)}
\NormalTok{coords\_proj }\OtherTok{\textless{}{-}} \FunctionTok{st\_coordinates}\NormalTok{(}\FunctionTok{st\_centroid}\NormalTok{(mt\_proj))}

\CommentTok{\# Definir raio de 120 km (120000 metros)}
\NormalTok{dist\_nb }\OtherTok{\textless{}{-}} \FunctionTok{dnearneigh}\NormalTok{(coords\_proj, }\DecValTok{0}\NormalTok{, }\DecValTok{120000}\NormalTok{)}

\CommentTok{\# Converter para linhas}
\NormalTok{lines\_dist }\OtherTok{\textless{}{-}} \FunctionTok{nb2lines}\NormalTok{(dist\_nb, }\AttributeTok{coords =}\NormalTok{ coords\_proj, }\AttributeTok{as\_sf =} \ConstantTok{TRUE}\NormalTok{)}
\FunctionTok{st\_crs}\NormalTok{(lines\_dist) }\OtherTok{\textless{}{-}} \FunctionTok{st\_crs}\NormalTok{(mt\_proj)}

\FunctionTok{ggplot}\NormalTok{() }\SpecialCharTok{+}
  \FunctionTok{geom\_sf}\NormalTok{(}\AttributeTok{data =}\NormalTok{ mt\_proj, }\AttributeTok{fill =} \StringTok{"gray95"}\NormalTok{, }\AttributeTok{color =} \StringTok{"gray80"}\NormalTok{) }\SpecialCharTok{+}
  \FunctionTok{geom\_sf}\NormalTok{(}\AttributeTok{data =}\NormalTok{ lines\_dist, }\AttributeTok{color =} \StringTok{"darkorange"}\NormalTok{, }\AttributeTok{linewidth =} \FloatTok{0.5}\NormalTok{, }\AttributeTok{alpha =} \FloatTok{0.6}\NormalTok{) }\SpecialCharTok{+}
  \FunctionTok{geom\_point}\NormalTok{(}\AttributeTok{data =} \FunctionTok{as.data.frame}\NormalTok{(coords\_proj), }\FunctionTok{aes}\NormalTok{(X, Y), }\AttributeTok{size =} \FloatTok{0.8}\NormalTok{) }\SpecialCharTok{+}
  \FunctionTok{labs}\NormalTok{(}\AttributeTok{title =} \StringTok{"Limiar de Distância Fixa (120 km)"}\NormalTok{,}
       \AttributeTok{subtitle =} \StringTok{"Conexões apenas se d \textless{} 120km (Note as ilhas isoladas)"}\NormalTok{) }\SpecialCharTok{+}
\NormalTok{  theme\_map}
\end{Highlighting}
\end{Shaded}

\begin{figure}[H]

\centering{

\pandocbounded{\includegraphics[keepaspectratio]{lattice_data_files/figure-pdf/fig-dist-threshold-1.pdf}}

}

\caption{\label{fig-dist-threshold}Vizinhança por Limiar de Distância
(120km).}

\end{figure}%

\begin{verbatim}
-   Decaimento por distância: Atribui pesos que decrescem com a distância, ex: $w_{ij} = d_{ij}^{-\alpha}$ ou $w_{ij} = \exp(-\beta d_{ij})$. Atribui maior influência a unidades mais próximas.
\end{verbatim}

\begin{Shaded}
\begin{Highlighting}[]
\CommentTok{\# Identificar Cuiabá}
\NormalTok{id\_cuiaba }\OtherTok{\textless{}{-}} \FunctionTok{which}\NormalTok{(mt\_sf}\SpecialCharTok{$}\NormalTok{name\_muni }\SpecialCharTok{==} \StringTok{"Cuiabá"}\NormalTok{)}

\CommentTok{\# Calcular distâncias de Cuiabá para TODOS os outros municípios}
\NormalTok{nb\_all }\OtherTok{\textless{}{-}} \FunctionTok{dnearneigh}\NormalTok{(coords\_proj, }\DecValTok{0}\NormalTok{, }\DecValTok{900000}\NormalTok{) }\CommentTok{\# Raio grande para pegar quase todo estado}
\NormalTok{dists }\OtherTok{\textless{}{-}} \FunctionTok{nbdists}\NormalTok{(nb\_all, coords\_proj)}

\CommentTok{\# Calcular Pesos (Inverso da Distância: 1/d)}
\NormalTok{weights\_list }\OtherTok{\textless{}{-}} \FunctionTok{lapply}\NormalTok{(dists, }\ControlFlowTok{function}\NormalTok{(x) }\DecValTok{1}\SpecialCharTok{/}\NormalTok{(x}\SpecialCharTok{/}\DecValTok{1000}\NormalTok{)) }\CommentTok{\# /1000 para km}

\CommentTok{\# Preparar dados apenas para Cuiabá para visualização}
\NormalTok{vizinhos\_cuiaba }\OtherTok{\textless{}{-}}\NormalTok{ nb\_all[[id\_cuiaba]]}
\NormalTok{pesos\_cuiaba }\OtherTok{\textless{}{-}}\NormalTok{ weights\_list[[id\_cuiaba]]}

\CommentTok{\# Criar linhas saindo de Cuiabá}

\NormalTok{lines\_cuiaba }\OtherTok{\textless{}{-}} \FunctionTok{vector}\NormalTok{(}\StringTok{"list"}\NormalTok{, }\FunctionTok{length}\NormalTok{(vizinhos\_cuiaba))}

\ControlFlowTok{for}\NormalTok{(i }\ControlFlowTok{in} \FunctionTok{seq\_along}\NormalTok{(vizinhos\_cuiaba)) \{}
\NormalTok{  dest\_idx }\OtherTok{\textless{}{-}}\NormalTok{ vizinhos\_cuiaba[i]}
\NormalTok{  lines\_cuiaba[[i]] }\OtherTok{\textless{}{-}} \FunctionTok{st\_linestring}\NormalTok{(}\FunctionTok{rbind}\NormalTok{(coords\_proj[id\_cuiaba,], coords\_proj[dest\_idx,]))}
\NormalTok{\}}

\NormalTok{sf\_decay }\OtherTok{\textless{}{-}} \FunctionTok{st\_sf}\NormalTok{(}\AttributeTok{peso =}\NormalTok{ pesos\_cuiaba, }\AttributeTok{geometry =} \FunctionTok{st\_sfc}\NormalTok{(lines\_cuiaba), }\AttributeTok{crs =} \DecValTok{5880}\NormalTok{)}

\FunctionTok{ggplot}\NormalTok{() }\SpecialCharTok{+}
  \FunctionTok{geom\_sf}\NormalTok{(}\AttributeTok{data =}\NormalTok{ mt\_proj, }\AttributeTok{fill =} \StringTok{"gray95"}\NormalTok{, }\AttributeTok{color =} \StringTok{"white"}\NormalTok{) }\SpecialCharTok{+}
  \FunctionTok{geom\_sf}\NormalTok{(}\AttributeTok{data =}\NormalTok{ sf\_decay, }\FunctionTok{aes}\NormalTok{(}\AttributeTok{color =}\NormalTok{ peso, }\AttributeTok{linewidth =}\NormalTok{ peso), }\AttributeTok{alpha =} \FloatTok{0.8}\NormalTok{) }\SpecialCharTok{+}
  \FunctionTok{geom\_point}\NormalTok{(}\FunctionTok{aes}\NormalTok{(}\AttributeTok{x=}\NormalTok{coords\_proj[id\_cuiaba,}\DecValTok{1}\NormalTok{], }\AttributeTok{y=}\NormalTok{coords\_proj[id\_cuiaba,}\DecValTok{2}\NormalTok{]), }\AttributeTok{color=}\StringTok{"red"}\NormalTok{, }\AttributeTok{size=}\DecValTok{3}\NormalTok{) }\SpecialCharTok{+}
  \FunctionTok{scale\_color\_viridis\_c}\NormalTok{(}\AttributeTok{option =} \StringTok{"magma"}\NormalTok{, }\AttributeTok{name =} \StringTok{"Peso (1/d)"}\NormalTok{) }\SpecialCharTok{+}
  \FunctionTok{scale\_linewidth}\NormalTok{(}\AttributeTok{range =} \FunctionTok{c}\NormalTok{(}\FloatTok{0.1}\NormalTok{, }\DecValTok{2}\NormalTok{), }\AttributeTok{guide =} \StringTok{"none"}\NormalTok{) }\SpecialCharTok{+}
  \FunctionTok{labs}\NormalTok{(}\AttributeTok{title =} \StringTok{"Decaimento por Distância (Foco: Cuiabá)"}\NormalTok{,}
       \AttributeTok{subtitle =} \StringTok{"A espessura e cor indicam a força da influência"}\NormalTok{) }\SpecialCharTok{+}
\NormalTok{  theme\_map}
\end{Highlighting}
\end{Shaded}

\begin{figure}[H]

\centering{

\pandocbounded{\includegraphics[keepaspectratio]{lattice_data_files/figure-pdf/fig-distance-decay-1.pdf}}

}

\caption{\label{fig-distance-decay}Decaimento por Distância Inversa a
partir de Cuiabá.}

\end{figure}%

\begin{enumerate}
\def\labelenumi{\arabic{enumi}.}
\setcounter{enumi}{2}
\tightlist
\item
  \textbf{Vizinhança econômica ou social:} Harris, Moffat, e Kravtsova
  (2011) argumentam que a contiguidade física pode ser insuficiente ou
  enganosa em muitos contextos. Em estudos regionais, a conexão
  funcional frequentemente supera a proximidade geográfica. Por exemplo,
  no Brasil, um município do agronegócio no Centro-Oeste (ex.:
  \href{https://pt.wikipedia.org/wiki/Sorriso}{Sorriso/MT}) pode estar
  economicamente mais conectado aos
  \href{https://pt.wikipedia.org/wiki/Porto_de_Santos}{portos de Santos
  (SP)} ou
  \href{https://pt.wikipedia.org/wiki/Paranagu\%C3\%A1}{Paranaguá (PR)}
  por onde escoa sua produção do que aos municípios geograficamente
  adjacentes em seu próprio estado que possuem economias de base
  diferente. Da mesma forma, para análises de mercado de trabalho ou
  inovação, a
  \href{https://pt.wikipedia.org/wiki/Regi\%C3\%A3o_Metropolitana_de_S\%C3\%A3o_Paulo}{região
  metropolitana de São Paulo} pode ter uma interação mais intensa com
  polos tecnológicos como
  \href{https://pt.wikipedia.org/wiki/Regi\%C3\%A3o_Metropolitana_de_Campinas}{Campinas}
  ou até com outros centros globais do que com municípios vizinhos de
  baixa intensidade tecnológica. Matrizes baseadas em fluxos
  (comerciais, migratórios, de passageiros), similaridade socioeconômica
  (\href{https://pt.wikipedia.org/wiki/Produto_interno_bruto}{PIB} per
  capita, estrutura produtiva) ou redes de infraestrutura (rodovias,
  linhas de voo) são, portanto, alternativas teóricas mais ricas e
  adequadas a fenômenos específicos.
\end{enumerate}

\begin{Shaded}
\begin{Highlighting}[]
\CommentTok{\# Carregar mapa do Brasil (Estados)}
\NormalTok{br\_states }\OtherTok{\textless{}{-}} \FunctionTok{read\_state}\NormalTok{(}\AttributeTok{year =} \DecValTok{2020}\NormalTok{, }\AttributeTok{showProgress =} \ConstantTok{FALSE}\NormalTok{)}

\CommentTok{\# Coordenadas aproximadas das cidades de interesse}
\CommentTok{\# (Sorriso{-}MT, Santos{-}SP, Paranaguá{-}PR)}

\NormalTok{cidades\_df }\OtherTok{\textless{}{-}} \FunctionTok{data.frame}\NormalTok{(}
  \AttributeTok{cidade =} \FunctionTok{c}\NormalTok{(}\StringTok{"Sorriso (MT)"}\NormalTok{, }\StringTok{"Porto de Santos (SP)"}\NormalTok{, }\StringTok{"Porto de Paranaguá (PR)"}\NormalTok{),}
  \AttributeTok{lat =} \FunctionTok{c}\NormalTok{(}\SpecialCharTok{{-}}\FloatTok{12.5427}\NormalTok{, }\SpecialCharTok{{-}}\FloatTok{23.9618}\NormalTok{, }\SpecialCharTok{{-}}\FloatTok{25.5205}\NormalTok{),}
  \AttributeTok{lon =} \FunctionTok{c}\NormalTok{(}\SpecialCharTok{{-}}\FloatTok{55.7211}\NormalTok{, }\SpecialCharTok{{-}}\FloatTok{46.3322}\NormalTok{, }\SpecialCharTok{{-}}\FloatTok{48.5095}\NormalTok{),}
  \AttributeTok{tipo =} \FunctionTok{c}\NormalTok{(}\StringTok{"Origem"}\NormalTok{, }\StringTok{"Destino"}\NormalTok{, }\StringTok{"Destino"}\NormalTok{)}
\NormalTok{)}

\NormalTok{cidades\_sf }\OtherTok{\textless{}{-}} \FunctionTok{st\_as\_sf}\NormalTok{(cidades\_df, }\AttributeTok{coords =} \FunctionTok{c}\NormalTok{(}\StringTok{"lon"}\NormalTok{, }\StringTok{"lat"}\NormalTok{), }\AttributeTok{crs =} \DecValTok{4326}\NormalTok{)}

\CommentTok{\# Criar conexões (Arcos)}
\NormalTok{sorriso\_coords }\OtherTok{\textless{}{-}} \FunctionTok{subset}\NormalTok{(cidades\_df, cidade }\SpecialCharTok{==} \StringTok{"Sorriso (MT)"}\NormalTok{)}

\NormalTok{destinos }\OtherTok{\textless{}{-}} \FunctionTok{subset}\NormalTok{(cidades\_df, tipo }\SpecialCharTok{==} \StringTok{"Destino"}\NormalTok{)}

\NormalTok{conexoes }\OtherTok{\textless{}{-}} \FunctionTok{lapply}\NormalTok{(}\DecValTok{1}\SpecialCharTok{:}\FunctionTok{nrow}\NormalTok{(destinos), }\ControlFlowTok{function}\NormalTok{(i) \{}
  \FunctionTok{st\_linestring}\NormalTok{(}\FunctionTok{rbind}\NormalTok{(}
    \FunctionTok{c}\NormalTok{(sorriso\_coords}\SpecialCharTok{$}\NormalTok{lon, sorriso\_coords}\SpecialCharTok{$}\NormalTok{lat),}
    \FunctionTok{c}\NormalTok{(destinos}\SpecialCharTok{$}\NormalTok{lon[i], destinos}\SpecialCharTok{$}\NormalTok{lat[i])}
\NormalTok{  ))}
\NormalTok{\})}

\NormalTok{conexoes\_sf }\OtherTok{\textless{}{-}} \FunctionTok{st\_sf}\NormalTok{(}\AttributeTok{geometry =} \FunctionTok{st\_sfc}\NormalTok{(conexoes), }\AttributeTok{crs =} \DecValTok{4326}\NormalTok{)}

\CommentTok{\# Plot}
\FunctionTok{ggplot}\NormalTok{() }\SpecialCharTok{+}
  \FunctionTok{geom\_sf}\NormalTok{(}\AttributeTok{data =}\NormalTok{ br\_states, }\AttributeTok{fill =} \StringTok{"gray95"}\NormalTok{, }\AttributeTok{color =} \StringTok{"white"}\NormalTok{) }\SpecialCharTok{+}
  \CommentTok{\# Destacar Estados envolvidos}
  \FunctionTok{geom\_sf}\NormalTok{(}\AttributeTok{data =} \FunctionTok{subset}\NormalTok{(br\_states, abbrev\_state }\SpecialCharTok{\%in\%} \FunctionTok{c}\NormalTok{(}\StringTok{"MT"}\NormalTok{, }\StringTok{"SP"}\NormalTok{, }\StringTok{"PR"}\NormalTok{)), }
          \AttributeTok{fill =} \StringTok{"gray85"}\NormalTok{, }\AttributeTok{color =} \StringTok{"white"}\NormalTok{) }\SpecialCharTok{+}
  
  \CommentTok{\# Linhas de Fluxo (Curvas para indicar movimento/distância)}
  \FunctionTok{geom\_curve}\NormalTok{(}\AttributeTok{data =} \FunctionTok{data.frame}\NormalTok{(}\AttributeTok{x1 =}\NormalTok{ sorriso\_coords}\SpecialCharTok{$}\NormalTok{lon, }\AttributeTok{y1 =}\NormalTok{ sorriso\_coords}\SpecialCharTok{$}\NormalTok{lat,}
                               \AttributeTok{x2 =}\NormalTok{ destinos}\SpecialCharTok{$}\NormalTok{lon, }\AttributeTok{y2 =}\NormalTok{ destinos}\SpecialCharTok{$}\NormalTok{lat),}
             \FunctionTok{aes}\NormalTok{(}\AttributeTok{x =}\NormalTok{ x1, }\AttributeTok{y =}\NormalTok{ y1, }\AttributeTok{xend =}\NormalTok{ x2, }\AttributeTok{yend =}\NormalTok{ y2),}
             \AttributeTok{color =} \StringTok{"darkgreen"}\NormalTok{, }\AttributeTok{size =} \DecValTok{1}\NormalTok{, }\AttributeTok{curvature =} \FloatTok{0.2}\NormalTok{, }
             \AttributeTok{arrow =} \FunctionTok{arrow}\NormalTok{(}\AttributeTok{length =} \FunctionTok{unit}\NormalTok{(}\FloatTok{0.03}\NormalTok{, }\StringTok{"npc"}\NormalTok{))) }\SpecialCharTok{+}
  
  \FunctionTok{geom\_point}\NormalTok{(}\AttributeTok{data =}\NormalTok{ cidades\_df, }\FunctionTok{aes}\NormalTok{(}\AttributeTok{x =}\NormalTok{ lon, }\AttributeTok{y =}\NormalTok{ lat, }\AttributeTok{color =}\NormalTok{ tipo), }\AttributeTok{size =} \DecValTok{3}\NormalTok{) }\SpecialCharTok{+}
  \FunctionTok{scale\_color\_manual}\NormalTok{(}\AttributeTok{values =} \FunctionTok{c}\NormalTok{(}\StringTok{"red"}\NormalTok{, }\StringTok{"blue"}\NormalTok{)) }\SpecialCharTok{+}
  
  \FunctionTok{geom\_text}\NormalTok{(}\AttributeTok{data =}\NormalTok{ cidades\_df, }\FunctionTok{aes}\NormalTok{(}\AttributeTok{x =}\NormalTok{ lon, }\AttributeTok{y =}\NormalTok{ lat, }\AttributeTok{label =}\NormalTok{ cidade), }
            \AttributeTok{vjust =} \SpecialCharTok{{-}}\DecValTok{1}\NormalTok{, }\AttributeTok{fontface =} \StringTok{"bold"}\NormalTok{, }\AttributeTok{size =} \DecValTok{3}\NormalTok{, }\AttributeTok{nudge\_x=}\DecValTok{8}\NormalTok{, }\AttributeTok{nudge\_y=}\SpecialCharTok{{-}}\DecValTok{2}\NormalTok{) }\SpecialCharTok{+} \CommentTok{\#usei nudge pra mover legenda}
  \FunctionTok{labs}\NormalTok{(}\AttributeTok{title =} \StringTok{"Vizinhança Econômica (Fluxo de Commodities)"}\NormalTok{,}
       \AttributeTok{subtitle =} \StringTok{"A conexão funcional supera a proximidade geográfica"}\NormalTok{) }\SpecialCharTok{+}
  \FunctionTok{theme\_void}\NormalTok{() }\SpecialCharTok{+}
  \FunctionTok{theme}\NormalTok{(}\AttributeTok{legend.position =} \StringTok{"none"}\NormalTok{, }\AttributeTok{plot.title =} \FunctionTok{element\_text}\NormalTok{(}\AttributeTok{hjust =} \FloatTok{0.5}\NormalTok{))}
\end{Highlighting}
\end{Shaded}

\begin{figure}[H]

\centering{

\pandocbounded{\includegraphics[keepaspectratio]{lattice_data_files/figure-pdf/fig-vizinhanca-economica-1.pdf}}

}

\caption{\label{fig-vizinhanca-economica}Vizinhança Econômica/Funcional:
O Agronegócio conectando Sorriso-MT aos Portos.}

\end{figure}%

\subsection{\texorpdfstring{Matriz de pesos espaciais (\(\mathbf{W}\)) e
normalização}{Matriz de pesos espaciais (\textbackslash mathbf\{W\}) e normalização}}\label{sec-matriz_pesos}

A matriz binária de adjacência \(\mathbf{W} = [w_{ij}]_{n \times n}\)
(com elementos 0 ou 1) é frequentemente transformada em uma matriz de
pesos para refletir a intensidade relativa das conexões. A escolha dos
pesos é exógena ao modelo (ou seja, deve ser definida a priori com base
em teoria ou no desenho do estudo) e tem implicações na estimação e
interpretação (H. Kelejian e Piras 2017).

A necessidade de normalização surge por razões estatísticas e de
interpretação. Em modelos autorregressivos espaciais (por serem vistos
mais adiante), o parâmetro de dependência \(\rho\) deve geralmente estar
em um intervalo que garanta a invertibilidade da matriz
\(( \mathbf{I} - \rho \mathbf{W} )\). Se \(\mathbf{W}\) não for
normalizada, os autovalores podem ser muito grandes ou desiguais,
restringindo o espaço paramétrico válido para \(\rho\) a um intervalo
desconhecido e difícil de interpretar. A normalização estabiliza o
comportamento numérico do modelo.

\begin{itemize}
\tightlist
\item
  \textbf{Normalização por linha (\emph{row-standardization})}
\end{itemize}

É a abordagem mais comum. Cada peso é dividido pela soma da linha
correspondente:

\[ w_{ij}^{r} = \frac{w_{ij}}{\sum_{j=1}^n w_{ij}}.\]

O resultado é que cada linha de \(\mathbf{W}^{r}\) soma 1. A operação
\(\mathbf{W}^{r}\mathbf{y}\) gera uma defasagem espacial
(\href{https://en.wikipedia.org/wiki/Spatial_weight_matrix}{spatial
lag}) que é interpretado como a média ponderada dos valores dos vizinhos
de cada unidade. Esta normalização equaliza a capacidade de receber
influência de cada unidade, independentemente do seu número de vizinhos.
Garante também que o maior autovalor de \(\mathbf{W}^{r}\) seja 1,
facilitando a definição do intervalo \((-1, 1)\) para \(\rho\) em
modelos SAR.

Exemplo: Considere uma matriz de vizinhança/adjacência binária
\(\mathbf{W}\) para quatro unidades/estados (A, B, C, D), onde estado A
é vizinho de B e C; B é vizinha apenas de A; C é vizinho apenas de A e,
D é isolado.

A normalização por linha transforma a matriz da seguinte forma:

\[
\mathbf{W} =
\begin{array}{c|cccc}
& A & B & C & D \\
\hline
A & 0 & 1 & 1 & 0 \\
B & 1 & 0 & 0 & 0 \\
C & 1 & 0 & 0 & 0 \\
D & 0 & 0 & 0 & 0 \\
\end{array}
\quad \rightarrow \quad
\mathbf{W}^{r} =
\begin{array}{c|cccc}
& A & B & C & D \\
\hline
A & 0 & 0.5 & 0.5 & 0 \\
B & 1 & 0 & 0 & 0 \\
C & 1 & 0 & 0 & 0 \\
D & 0 & 0 & 0 & 0 \\
\end{array}
\]

Cada unidade recebe uma influência total igual a 1 de seus vizinhos. A
unidade A (com dois vizinhos) recebe 50\% de sua influência de B e 50\%
de C. B e C (cada um com um único vizinho) recebem 100\% de sua
influência de A. D não recebe influência. A defasagem espacial para a
unidade A, \((\mathbf{W}^{r}\mathbf{y})_A\), é
\(0.5 \cdot y_B + 0.5 \cdot y_C\), a média simples dos valores de seus
vizinhos.

\begin{Shaded}
\begin{Highlighting}[]
\ControlFlowTok{if}\NormalTok{ (}\SpecialCharTok{!}\FunctionTok{require}\NormalTok{(}\StringTok{"pacman"}\NormalTok{)) }\FunctionTok{install.packages}\NormalTok{(}\StringTok{"pacman"}\NormalTok{)}
\NormalTok{pacman}\SpecialCharTok{::}\FunctionTok{p\_load}\NormalTok{(sf, spdep, geobr, dplyr)}

\CommentTok{\# Carregar mapa do estado de Sergipe }
\NormalTok{se\_sf }\OtherTok{\textless{}{-}} \FunctionTok{read\_municipality}\NormalTok{(}\AttributeTok{code\_muni =} \StringTok{"SE"}\NormalTok{, }\AttributeTok{year =} \DecValTok{2020}\NormalTok{, }\AttributeTok{showProgress =} \ConstantTok{FALSE}\NormalTok{)}

\CommentTok{\# Criar vizinhança (Queen)}
\NormalTok{nb }\OtherTok{\textless{}{-}} \FunctionTok{poly2nb}\NormalTok{(se\_sf, }\AttributeTok{queen =} \ConstantTok{TRUE}\NormalTok{)}

\CommentTok{\# Criar Matriz Binária (0 e 1)}
\CommentTok{\# Necessária para os cálculos manuais de Coluna, Espectral e CAR}
\NormalTok{W\_binaria }\OtherTok{\textless{}{-}} \FunctionTok{nb2mat}\NormalTok{(nb, }\AttributeTok{style =} \StringTok{"B"}\NormalTok{, }\AttributeTok{zero.policy =} \ConstantTok{TRUE}\NormalTok{)}

\FunctionTok{paste}\NormalTok{(}\StringTok{"Dimensão da Matriz W:"}\NormalTok{, }\FunctionTok{nrow}\NormalTok{(W\_binaria), }\StringTok{"x"}\NormalTok{, }\FunctionTok{ncol}\NormalTok{(W\_binaria))}
\end{Highlighting}
\end{Shaded}

\begin{verbatim}
[1] "Dimensão da Matriz W: 75 x 75"
\end{verbatim}

\begin{Shaded}
\begin{Highlighting}[]
\CommentTok{\# Normalização por linha, style = "W"}
\NormalTok{lw\_row }\OtherTok{\textless{}{-}} \FunctionTok{nb2listw}\NormalTok{(nb, }\AttributeTok{style =} \StringTok{"W"}\NormalTok{, }\AttributeTok{zero.policy =} \ConstantTok{TRUE}\NormalTok{)}

\CommentTok{\# Extrair a matriz de pesos para verificação}
\NormalTok{W\_row }\OtherTok{\textless{}{-}} \FunctionTok{listw2mat}\NormalTok{(lw\_row)}

\CommentTok{\# A soma dos pesos de cada linha deve ser 1 (para quem tem vizinhos)}
\NormalTok{soma\_linhas }\OtherTok{\textless{}{-}} \FunctionTok{rowSums}\NormalTok{(W\_row)}
\FunctionTok{print}\NormalTok{(}\FunctionTok{head}\NormalTok{(soma\_linhas)) }\CommentTok{\# Deve mostrar 1, 1, 1... (por baixo), }
\end{Highlighting}
\end{Shaded}

\begin{verbatim}
1 2 3 4 5 6 
1 1 1 1 1 1 
\end{verbatim}

\begin{Shaded}
\begin{Highlighting}[]
                        \CommentTok{\#os de cima sao indices que identificam Municipios}
\end{Highlighting}
\end{Shaded}

\begin{itemize}
\tightlist
\item
  \textbf{Normalização por Coluna (\emph{Column-Standardization})}
\end{itemize}

Cada peso é dividido pela soma da coluna correspondente:

\[ w_{ij}^{c} = \frac{w_{ij}}{\sum_{i=1}^n w_{ij}}.\]

Esta abordagem equaliza a capacidade de emitir influência de cada
unidade. Enquanto a normalização por linha controla o impacto recebido,
a normalização por coluna controla o impacto causado.

Exemplo: Usando a mesma matriz \(\mathbf{W}\) definida anteriormente, a
normalização por coluna resulta em:

\[
\mathbf{W} =
\begin{array}{c|cccc}
& A & B & C & D \\
\hline
A & 0 & 1 & 1 & 0 \\
B & 1 & 0 & 0 & 0 \\
C & 1 & 0 & 0 & 0 \\
D & 0 & 0 & 0 & 0 \\
\end{array}
\quad \rightarrow \quad
\mathbf{W}^{c} =
\begin{array}{c|cccc}
& A & B & C & D \\
\hline
A & 0 & 1 & 1 & 0 \\
B & 0.5 & 0 & 0 & 0 \\
C & 0.5 & 0 & 0 & 0 \\
D & 0 & 0 & 0 & 0 \\
\end{array}
\]

A influência total que cada unidade emite é normalizada para 1. A
unidade A é alvo da influência de B e C; portanto, a coluna A
(influência emitida para A) soma 2 (vinda de B e C). Cada conexão para A
recebe peso \(1/2\). A unidade B emite influência apenas para A (coluna
B soma 1), logo, a conexão de A para B recebe peso 1. Assim, a defasagem
espacial agora é um vetor onde o valor para cada unidade é a soma dos
valores das unidades que ela influencia, ponderada pela intensidade.
Para a unidade A,
\((\mathbf{W}^{c}\mathbf{y})_A = 1 \cdot y_B + 1 \cdot y_C\). Esta
abordagem é menos comum, mas pode ser relevante em modelos de difusão ou
análise de redes, onde o
\href{https://en.wikipedia.org/wiki/Directed_graph}{out-degree}
(influência emitida) é um objeto de interesse central.

\begin{Shaded}
\begin{Highlighting}[]
\CommentTok{\# Calcular a soma de cada coluna da matriz binária}
\NormalTok{col\_somas }\OtherTok{\textless{}{-}} \FunctionTok{colSums}\NormalTok{(W\_binaria)}
\CommentTok{\# Proteção contra divisão por zero (caso haja ilhas)}
\NormalTok{col\_somas[col\_somas }\SpecialCharTok{==} \DecValTok{0}\NormalTok{] }\OtherTok{\textless{}{-}} \DecValTok{1} 

\CommentTok{\# Dividir cada elemento pela soma da sua coluna}
\CommentTok{\# A função sweep aplica a operação na MARGIN=2 (colunas)}
\NormalTok{W\_col }\OtherTok{\textless{}{-}} \FunctionTok{sweep}\NormalTok{(W\_binaria, }\AttributeTok{MARGIN =} \DecValTok{2}\NormalTok{, }\AttributeTok{STATS =}\NormalTok{ col\_somas, }\AttributeTok{FUN =} \StringTok{"/"}\NormalTok{)}

\CommentTok{\# A soma da primeira coluna deve ser 1}
\FunctionTok{paste}\NormalTok{(}\StringTok{"Soma da Coluna 1:"}\NormalTok{, }\FunctionTok{sum}\NormalTok{(W\_col[,}\DecValTok{1}\NormalTok{]))}
\end{Highlighting}
\end{Shaded}

\begin{verbatim}
[1] "Soma da Coluna 1: 1"
\end{verbatim}

\begin{itemize}
\tightlist
\item
  \textbf{Normalização espectral (ou por autovalor máximo):} Para
  preservar as proporções relativas originais entre os pesos
  (especialmente importante em matrizes baseadas em distância),
  normaliza-se toda a matriz por seu autovalor de maior módulo,
  \(\lambda_{max}\):
\end{itemize}

\[ \mathbf{W}^{spectral} = \frac{\mathbf{W}^0}{\lambda_{max}}.\]

Esta abordagem mantém a simetria da matriz (se originalmente simétrica)
e preserva o significado físico original dos pesos (ex., um decaimento
por distância). É recomendada por autores como H. H. Kelejian e Prucha
(2010) e Elhorst et al. (2014) para evitar distorções na estrutura de
dependência.

\begin{Shaded}
\begin{Highlighting}[]
\CommentTok{\# Calcular autovalores da matriz binária}
\NormalTok{autovalores }\OtherTok{\textless{}{-}} \FunctionTok{eigen}\NormalTok{(W\_binaria, }\AttributeTok{only.values =} \ConstantTok{TRUE}\NormalTok{)}\SpecialCharTok{$}\NormalTok{values}

\CommentTok{\# Encontrar o maior autovalor absoluto (Raio Espectral)}
\NormalTok{lambda\_max }\OtherTok{\textless{}{-}} \FunctionTok{max}\NormalTok{(}\FunctionTok{abs}\NormalTok{(autovalores))}

\CommentTok{\# Normalizar a matriz}
\NormalTok{W\_spec }\OtherTok{\textless{}{-}}\NormalTok{ W\_binaria }\SpecialCharTok{/}\NormalTok{ lambda\_max}

\CommentTok{\# O maior autovalor da nova matriz deve ser 1}
\FunctionTok{print}\NormalTok{(}\FunctionTok{paste}\NormalTok{(}\StringTok{"Novo Lambda Max:"}\NormalTok{, }\FunctionTok{max}\NormalTok{(}\FunctionTok{abs}\NormalTok{(}\FunctionTok{eigen}\NormalTok{(W\_spec, }\AttributeTok{only.values=}\ConstantTok{TRUE}\NormalTok{)}\SpecialCharTok{$}\NormalTok{values))))}
\end{Highlighting}
\end{Shaded}

\begin{verbatim}
[1] "Novo Lambda Max: 1"
\end{verbatim}

\begin{itemize}
\tightlist
\item
  \textbf{Normalização de variância escalar (para modelos CAR):} Em
  modelos autorregressivos condicionais (CAR) bayesianos, busca-se
  frequentemente uma matriz simétrica para definir uma matriz de
  precisão válida. Uma normalização comum é:
\end{itemize}

\[\mathbf{W}^{CAR} = \mathbf{D}^{-1/2} \mathbf{W} \mathbf{D}^{-1/2},\]

onde \(\mathbf{D}\) é uma matriz diagonal com
\(d_{ii} = \sum_j w_{ij}\). Esta forma estabiliza a variância e preserva
a simetria.

\begin{Shaded}
\begin{Highlighting}[]
\CommentTok{\# Fórmula: D\^{}({-}1/2) * W * D\^{}({-}1/2)}

\CommentTok{\# Obter número de vizinhos (D) de cada área}
\NormalTok{num\_vizinhos }\OtherTok{\textless{}{-}} \FunctionTok{rowSums}\NormalTok{(W\_binaria)}

\CommentTok{\# Calcular a matriz diagonal inversa da raiz quadrada (D\^{}{-}1/2)}
\CommentTok{\# Se vizinhos = 0, mantemos 0 para evitar Infinito}
\NormalTok{inv\_sqrt\_D }\OtherTok{\textless{}{-}} \FunctionTok{ifelse}\NormalTok{(num\_vizinhos }\SpecialCharTok{\textgreater{}} \DecValTok{0}\NormalTok{, }\DecValTok{1} \SpecialCharTok{/} \FunctionTok{sqrt}\NormalTok{(num\_vizinhos), }\DecValTok{0}\NormalTok{)}
\NormalTok{M\_diag }\OtherTok{\textless{}{-}} \FunctionTok{diag}\NormalTok{(inv\_sqrt\_D)}

\CommentTok{\# Multiplicação Matricial (\%*\%)}
\NormalTok{W\_car }\OtherTok{\textless{}{-}}\NormalTok{ M\_diag }\SpecialCharTok{\%*\%}\NormalTok{ W\_binaria }\SpecialCharTok{\%*\%}\NormalTok{ M\_diag}

\CommentTok{\# Visualizar o canto da matriz (Note que ela é simétrica)}
\FunctionTok{print}\NormalTok{(}\FunctionTok{round}\NormalTok{(W\_car[}\DecValTok{1}\SpecialCharTok{:}\DecValTok{5}\NormalTok{, }\DecValTok{1}\SpecialCharTok{:}\DecValTok{5}\NormalTok{], }\DecValTok{3}\NormalTok{))}
\end{Highlighting}
\end{Shaded}

\begin{verbatim}
      [,1]  [,2] [,3] [,4] [,5]
[1,] 0.000 0.192    0    0    0
[2,] 0.192 0.000    0    0    0
[3,] 0.000 0.000    0    0    0
[4,] 0.000 0.000    0    0    0
[5,] 0.000 0.000    0    0    0
\end{verbatim}

\subsection{\texorpdfstring{Críticas e escolha da matriz
\(\mathbf{W}\)}{Críticas e escolha da matriz \textbackslash mathbf\{W\}}}\label{cruxedticas-e-escolha-da-matriz-mathbfw}

A escolha de \(\mathbf{W}\) é frequentemente o ponto mais subjetivo e
crítico da modelagem espacial. H. Kelejian e Piras (2017) e Elhorst et
al. (2014) apresentam críticas à aplicação da normalização por linha:

\begin{enumerate}
\def\labelenumi{\arabic{enumi}.}
\item
  Perda da interpretação de distância: Se \(\mathbf{W}\) é baseada no
  inverso da distância (\(w_{ij} = d_{ij}^{-\alpha}\)), a normalização
  por linha destrói a estrutura de decaimento absoluto. Uma unidade
  central com muitos vizinhos próximos (\(\sum_j w_{ij}\) grande) terá
  seus pesos reduzidos drasticamente, enquanto uma unidade periférica
  com poucos vizinhos distantes (\(\sum_j w_{ij}\) pequeno) terá seus
  pesos inflacionados.
\item
  Indução de assimetria: Uma matriz de contiguidade ou distância é
  frequentemente simétrica (\(w_{ij} = w_{ji}\)). A normalização por
  linha gera uma matriz assimétrica (\(w_{ij}^{r} \neq w_{ji}^{r}\)), o
  que pode ser contra-intuitivo para noções de vizinhança e complica a
  interpretação em alguns modelos.
\item
  A Falácia da seleção por \(R^2\): Uma prática comum é escolher a
  matriz \(\mathbf{W}\) (ou seu critério de construção) que maximiza uma
  medida de ajuste como o \(R^2\) ou a verossimilhança do modelo. H.
  Kelejian e Piras (2017) demonstram analiticamente que este
  procedimento é enviesado. Eles provam que o \(R^2\) é maximizado
  quando os pesos se aproximam de uma matriz de pesos uniformes. Nesse
  cenário, o parâmetro espacial \(\hat{\rho}\) absorve toda a variação,
  e os coeficientes das covariáveis \(\hat{\boldsymbol{\beta}}\)
  colapsam para zero, produzindo um modelo sem poder explicativo real.
\end{enumerate}

Assim, a seleção de \(\mathbf{W}\) deve ser guiada pela teoria
substantiva do fenômeno em estudo. Quando várias especificações são
plausíveis, pode-se usar:

\begin{itemize}
\tightlist
\item
  Critérios de seleção de modelo: Como proposto por Zhang e Yu (2018),
  que adaptam um critério do tipo \(C_p\) de
  \href{https://en.wikipedia.org/wiki/Mallows\%27s_Cp}{Mallows} para
  selecionar a matriz dentro de um conjunto candidato, visando minimizar
  o erro de previsão. Em sua forma clássica, o \(C_p\) de Mallows
  fornece uma estimativa do erro quadrático médio de previsão para um
  modelo de regressão com \(p\) parâmetros (Colin L. Mallows 1973; Cohn
  L. Mallows 1995). Sua expressão é dada por:
\end{itemize}

\[
C_p = \frac{\text{SSE}_p}{\hat{\sigma}^2} - n + 2p,
\]

onde = \(\text{SSE}_p\) é a soma dos quadrados dos resíduos do modelo
candidato; \(\hat{\sigma}^2\) é uma estimativa não viciada da variância
do erro do modelo mais completo (ou do modelo considerado verdadeiro);
\(n\) é o número de observações e, \(p\) é o número de parâmetros do
modelo (incluindo o intercepto), que atua como penalização pela
complexidade.

Um valor menor de \(C_p\) indica um melhor equilíbrio entre qualidade de
ajuste (SSE baixo) e parcimônia (penalidade \(p\) baixa), guiando a
seleção do modelo.

Zhang e Yu (2018) estende este princípio para modelos de defasagem
espacial (SAR). A ideia central é tratar cada matriz candidata
\(\mathbf{W}_k\) como um modelo distinto. Para um SAR da forma
\(\mathbf{y} = \rho_k \mathbf{W}_k \mathbf{y} + \mathbf{X} \boldsymbol{\beta}_k + \boldsymbol{\varepsilon}_k\),
uma estatística \(C_p\) adaptada é derivada.

Essa adaptação considera que a complexidade efetiva do modelo espacial
não depende apenas do número de covariáveis em \(\mathbf{X}\), mas
também da estrutura de dependência induzida por \(\mathbf{W}_k\) e do
parâmetro espacial \(\rho_k\). O
\href{https://pt.wikipedia.org/wiki/Tra\%C3\%A7o_(\%C3\%A1lgebra_linear)}{traço
da matriz} de
\href{https://pt.wikipedia.org/wiki/Proje\%C3\%A7\%C3\%A3o_(\%C3\%A1lgebra_linear)}{projeção}
(ou \emph{hat matrix}) do modelo SAR, \(\text{tr}(\mathbf{H}_k)\), que
generaliza o número de parâmetros \(p\), é tipicamente utilizado na
penalização. A estatística resultante pode ser aproximada por:

\[
C_p(\mathbf{W}_k) \approx \frac{\text{SSE}_k}{\hat{\sigma}^2} - n + 2 \, \text{tr}(\mathbf{H}_k),
\]

onde \(\text{SSE}_k\) e \(\text{tr}(\mathbf{H}_k)\) são calculados para
o modelo estimado com a matriz \(\mathbf{W}_k\). A matriz
\(\mathbf{W}_k\) que minimiza \(C_p(\mathbf{W}_k)\) no conjunto
candidato é então selecionada.

Zhang e Yu (2018) demonstra que este procedimento é assintoticamente
ótimo no sentido de minimizar o erro quadrático médio de previsão, mesmo
que a verdadeira matriz de pesos (geradora dos dados) não esteja
incluída no conjunto \(\{\mathbf{W}_1, \ldots, \mathbf{W}_K\}\).

\begin{itemize}
\tightlist
\item
  Média de modelos: Uma evolução natural deste paradigma é reconhecer a
  incerteza inerente à escolha de uma única matriz. Em vez de selecionar
  um único \(\mathbf{W}_k\), a abordagem de média de modelos combina as
  previsões de todos os modelos candidatos, atribuindo-lhes pesos que
  refletem seu suporte empírico. Miao et al. (2025) estendem esta lógica
  para modelos espaciais multivariados (MSAR). Eles propõem estimar
  pesos \(\pi_k\) para cada modelo (cada um com sua matriz
  \(\mathbf{W}_k\)) de modo a minimizar o risco de predição. A previsão
  final é uma média ponderada:
\end{itemize}

\[
\hat{\mathbf{y}}^* = \sum_{k=1}^K \pi_k \, \hat{\mathbf{y}}_k,
\]

onde \(\hat{\mathbf{y}}_k\) é a previsão do modelo com matriz
\(\mathbf{W}_k\). Esta estratégia geralmente produz previsões mais
robustas e estáveis do que qualquer modelo individual, pois incorpora a
incerteza sobre a estrutura de dependência espacial correta.

\section{Implicações estatísticas da discretização
espacial}\label{implicauxe7uxf5es-estatuxedsticas-da-discretizauxe7uxe3o-espacial}

A agregação de um processo contínuo em unidades de área discretas
introduz desafios inferenciais profundos que vão além do MAUP.

\begin{enumerate}
\def\labelenumi{\arabic{enumi}.}
\item
  O MAUP possui duas dimensões Openshaw (1984) . O efeito de escala
  refere-se à mudança nos resultados ao se alterar o nível de agregação
  (ex.: de bairros para municípios). O efeito de zoneamento refere-se à
  mudança nos resultados ao se redesenhar os limites das unidades no
  mesmo nível de agregação. Ambos podem alterar ou até inverter o sinal
  de correlações e parâmetros espaciais, pois a discretização atua como
  um filtro não linear na estrutura de covariância do processo
  subjacente.
\item
  Em dados de contagem (ex.: casos de doença), a variabilidade observada
  é uma combinação da variação do processo espacial latente de risco
  (\(Y(\mathbf{s})\)) e da variação inerente ao mecanismo de amostragem
  (ex.: distribuição de Poisson). Em áreas com populações pequenas, a
  flutuação amostral pode dominar, criando padrões espúrios. Noel
  Cressie e Chan (1989) mostram, no estudo da Síndrome da Morte Súbita
  Infantil (SIDS), como a heterogeneidade do tamanho da população-base
  pode gerar autocorrelação espacial aparente. Modelos hierárquicos que
  incorporam um offset populacional ou usam distribuições como a
  Binomial negativa são essenciais para separar esses efeitos.
\item
  A agregação tende a alisar a variação local de um processo contínuo,
  podendo atenuar \emph{hotspots} reais. Além disso, como alertado por
  Reich, Hodges, e Zadnik (2006), a inclusão de termos de dependência
  espacial (como um processo CAR ou SAR) para capturar correlação nos
  resíduos pode introduzir colinearidade com as covariáveis fixas do
  modelo, inflando a variância das estimativas dos coeficientes
  \(\boldsymbol{\beta}\) e complicando a inferência.
\item
  A estrutura de dependência inferida é altamente condicional à matriz
  \(\mathbf{W}\) especificada. Duas observações fisicamente próximas,
  mas separadas por uma fronteira administrativa que não é considerada
  no critério de vizinhança, serão modeladas como independentes.
  Portanto, a dependência espacial estimada é, em grande parte, uma
  função da discretização e da definição de vizinhança adotada, e não
  apenas uma propriedade intrínseca do fenômeno (Hodges e Reich 2010).
\end{enumerate}

\section{Análise exploratória em dados de área}\label{sec-esda}

A Análise Exploratória de Dados Espaciais (ESDA --
\href{https://cran.r-project.org/web/packages/geostan/vignettes/measuring-sa.html}{Exploratory
Spatial Data Analysis}) é definida por Anselin (1995) como um conjunto
de técnicas destinadas a: (i) descrever e visualizar distribuições
espaciais; (ii) identificar localizações atípicas (\emph{spatial
outliers}); (iii) detectar padrões de associação espacial
(\emph{clusters}); e (iv) sugerir regimes de heterogeneidade espacial.
Diferentemente da estatística descritiva clássica, a ESDA não assume
independência entre as observações. O seu objetivo central é,
justamente, quantificar a natureza e a intensidade da dependência
espacial, que é definida pela estrutura de vizinhança
\(\mathbf{W}=[w_{ij}]_{n \times n}\) (ver Seção~\ref{sec-lattice}). A
ESDA é uma extensão da Análise Exploratória de Dados (EDA) para o
contexto espacial, mantendo seu caráter visual e robusto, mas com a
adição fundamental do mapa como ferramenta central para responder a
perguntas como ``onde estão esses casos no mapa?'' ou ``quais áreas
nesta sub-região atendem a critérios específicos de atributo?'' R. H.
Haining, Wise, e Ma (1998) .

Em dados de área (\emph{lattice}), onde \(y_i\) representa um valor
agregado na unidade discreta \(i\), a análise divide-se fundamentalmente
em duas categorias: indicadores globais (que resumem o padrão de todo o
mapa num único escalar) e indicadores locais (que decompõem a estrutura
de dependência para cada \(i\)-unidade individualmente). A implementação
prática da ESDA frequentemente ocorre em ambientes de Sistemas de
Informação Geográfica (GIS), que integram capacidades de visualização
cartográfica, gestão de dados e análise estatística interativa, como
exemplificado pelo sistema SAGE descrito por R. H. Haining, Wise, e Ma
(1998). Este capítulo, assim como feito nos outros capítulos, usaremos o
\texttt{R/Rstudio}.

\subsection{Estatísticas globais de
autocorrelação}\label{estatuxedsticas-globais-de-autocorrelauxe7uxe3o}

As estatísticas globais testam a hipótese nula de aleatoriedade espacial
completa (CSR - \emph{Complete Spatial Randomness}). Sob \(H_0\), os
valores \(\{y_i\}\) são distribuídos aleatoriamente pelas localizações
fixas, sem respeitar a topologia definida por \(\mathbf{W}\).

\textbf{Índice I de Moran}

O Índice de Moran (\(I\)) é a medida de autocorrelação espacial mais
amplamente utilizada, introduzida formalmente por Moran (1950) . A sua
estrutura é análoga ao coeficiente de
\href{https://pt.wikipedia.org/wiki/Coeficiente_de_correla\%C3\%A7\%C3\%A3o_de_Pearson}{correlação
de Pearson}, mas ponderada pela matriz de pesos espaciais. Para um vetor
de observações \(\mathbf{y} = (y_1, \dots, y_n)^\top\) com \(n\)
unidades:

\begin{equation}\phantomsection\label{eq-moran}{
I = \frac{n}{S_0} \cdot \frac{\sum_{i=1}^n \sum_{j=1}^n w_{ij} (y_i - \bar{y})(y_j - \bar{y})}{\sum_{i=1}^n (y_i - \bar{y})^2} = \frac{n}{S_0} \cdot \frac{\mathbf{z}^\top \mathbf{W} \mathbf{z}}{\mathbf{z}^\top \mathbf{z}},
}\end{equation}

onde \(\mathbf{z} = (y_1 - \bar{y}, \dots, y_n - \bar{y})^\top\) é o
vetor dos desvios em relação à média \(\bar{y}\), \(w_{ij}\) são os
elementos da matriz de pesos espaciais \(\mathbf{W}\) (tipicamente
normalizada por linha, ver Seção~\ref{sec-matriz_pesos}), e
\(S_0 = \sum_{i=1}^n \sum_{j=1}^n w_{ij}\) é a soma de todos os pesos
(que iguala \(n\) no caso de normalização por linha).

O valor esperado de \(I\) sob \(H_0\) é \(E[I] = -1/(n-1)\), que tende a
zero quando \(n\) aumenta.

\begin{itemize}
\item
  \(I > E[I]\): Indica autocorrelação espacial positiva (agrupamento de
  valores semelhantes no espaço).
\item
  \(I < E[I]\): Indica autocorrelação espacial negativa (dispersão
  perfeita ou padrão de xadrez).
\end{itemize}

A inferência é geralmente realizada através de uma abordagem de
permutação condicional (Monte Carlo), uma vez que a aproximação à
normalidade depende de pressupostos assintóticos que podem não se
verificar em matrizes de pesos irregulares, como discutido por Getis
(1995).

\textbf{Propriedades estatísticas do Índice de Moran}

Embora ``não exista'' um teste uniformemente mais poderoso (UMP) para
autocorrelação espacial em todos os cenários, Tiefelsdorf (2000)
demonstrou que o \(I\) de Moran é um teste Localmente Melhor Invariante
(LBI). Isso significa que, na vizinhança da hipótese nula
(\(\rho \approx 0\)), a função de poder do \(I\) de Moran possui a
inclinação mais acentuada em comparação a outros testes. Isso torna o
\(I\) de Moran a ferramenta mais sensível para detectar pequenos desvios
da aleatoriedade, sendo eficaz tanto contra hipóteses alternativas de
processos autorregressivos (AR) quanto de médias móveis (MA)18.

Burridge (1980) provou que o \(I\) de Moran é assintoticamente
equivalente a um teste de Multiplicador de Lagrange (LM) para processos
gaussianos. A estatística LM, calculada a partir da função de
verossimilhança restrita, é proporcional ao valor de \(I\),
compartilhando assim as propriedades de eficiência computacional dos
testes LM, que exigem estimação apenas sob a hipótese nula e são mais
conservadores que os testes de Razão de Verossimilhança (LR).

\begin{Shaded}
\begin{Highlighting}[]
\ControlFlowTok{if}\NormalTok{ (}\SpecialCharTok{!}\FunctionTok{require}\NormalTok{(}\StringTok{"pacman"}\NormalTok{)) }\FunctionTok{install.packages}\NormalTok{(}\StringTok{"pacman"}\NormalTok{)}
\NormalTok{pacman}\SpecialCharTok{::}\FunctionTok{p\_load}\NormalTok{(sf, spdep, ggplot2, patchwork, dplyr, geobr)}

\CommentTok{\#Carregar dados: Malha de Minas Gerais (MG)}
\NormalTok{mg\_sf }\OtherTok{\textless{}{-}} \FunctionTok{read\_municipality}\NormalTok{(}\AttributeTok{code\_muni =} \StringTok{"MG"}\NormalTok{, }\AttributeTok{year =} \DecValTok{2020}\NormalTok{, }\AttributeTok{showProgress =} \ConstantTok{FALSE}\NormalTok{)}
\CommentTok{\#simulando dados}

\NormalTok{coords }\OtherTok{\textless{}{-}} \FunctionTok{st\_coordinates}\NormalTok{(}\FunctionTok{st\_centroid}\NormalTok{(mg\_sf))}
\FunctionTok{set.seed}\NormalTok{(}\DecValTok{123}\NormalTok{)}
\NormalTok{mg\_sf}\SpecialCharTok{$}\NormalTok{indicador }\OtherTok{\textless{}{-}}\NormalTok{ (}\SpecialCharTok{{-}}\NormalTok{coords[,}\DecValTok{2}\NormalTok{]) }\SpecialCharTok{*} \DecValTok{10} \SpecialCharTok{+} \FunctionTok{rnorm}\NormalTok{(}\FunctionTok{nrow}\NormalTok{(mg\_sf), }\AttributeTok{mean =} \DecValTok{0}\NormalTok{, }\AttributeTok{sd =} \DecValTok{15}\NormalTok{)}

\CommentTok{\# Definir Vizinhança e Pesos}
\CommentTok{\# Vizinhança Queen}
\NormalTok{nb }\OtherTok{\textless{}{-}} \FunctionTok{poly2nb}\NormalTok{(mg\_sf, }\AttributeTok{queen =} \ConstantTok{TRUE}\NormalTok{)}

\CommentTok{\# normalizar por linha (style W)}
\NormalTok{lw }\OtherTok{\textless{}{-}} \FunctionTok{nb2listw}\NormalTok{(nb, }\AttributeTok{style =} \StringTok{"W"}\NormalTok{, }\AttributeTok{zero.policy =} \ConstantTok{TRUE}\NormalTok{) }\CommentTok{\#recomendo usar sempre zero.policy = TRUE}

\CommentTok{\# Cálculo do Índice de Moran}
\CommentTok{\# A) Teste Rápido para variavel de interresse "indicador"}
\NormalTok{moran\_analitico }\OtherTok{\textless{}{-}} \FunctionTok{moran.test}\NormalTok{(mg\_sf}\SpecialCharTok{$}\NormalTok{indicador, }\AttributeTok{listw=}\NormalTok{lw, }\AttributeTok{randomisation =} \ConstantTok{TRUE}\NormalTok{)}
\FunctionTok{print}\NormalTok{(moran\_analitico)}
\end{Highlighting}
\end{Shaded}

\begin{verbatim}

    Moran I test under randomisation

data:  mg_sf$indicador  
weights: lw    

Moran I statistic standard deviate = 29.13, p-value < 2.2e-16
alternative hypothesis: greater
sample estimates:
Moran I statistic       Expectation          Variance 
     0.6193722870     -0.0011737089      0.0004537973 
\end{verbatim}

\begin{Shaded}
\begin{Highlighting}[]
\CommentTok{\# B) Teste Monte Carlo (Robusto)}
\CommentTok{\# Simula 999 permutações aleatórias}
\NormalTok{moran\_mc }\OtherTok{\textless{}{-}} \FunctionTok{moran.mc}\NormalTok{(mg\_sf}\SpecialCharTok{$}\NormalTok{indicador, }\AttributeTok{listw=}\NormalTok{lw, }\AttributeTok{nsim =} \DecValTok{999}\NormalTok{)}

\FunctionTok{print}\NormalTok{(moran\_mc)}
\end{Highlighting}
\end{Shaded}

\begin{verbatim}

    Monte-Carlo simulation of Moran I

data:  mg_sf$indicador 
weights: lw  
number of simulations + 1: 1000 

statistic = 0.61937, observed rank = 1000, p-value = 0.001
alternative hypothesis: greater
\end{verbatim}

\textbf{Interpretação}

Como normalizamos a matriz \(\mathbf{W}\) por linha, o intervalo do
índice de Moran é limitado, tipicamente variando entre -1 e 1.

O valor da estatística I de Moran calculado foi de aproximadamente
0,619. Este valor positivo e de magnitude elevada indica uma forte
autocorrelação espacial positiva, sugerindo que municípios com indicador
socioeconômico alto tendem a estar geograficamente agrupados com outros
de indicador alto (neste caso, no Sul), enquanto municípios com
indicador baixo formam aglomerados com seus vizinhos de indicador baixo
(neste caso, no Norte).

Para validar se este agrupamento é meramente fruto do acaso, compara-se
o valor observado com o valor esperado sob a hipótese nula de
aleatoriedade espacial completa (Expectation), que neste caso é -0,0012
(calculado como \(-1/(n-1) = -1/(853-1) \approx -0,0012\)). A diferença
substancial entre o valor observado (Statistic \(\approx 0,62\)) e o
esperado (-0,0012) fornece a evidência primária de que o processo
gerador dos dados não é aleatório. A confirmação estatística desta
observação na abordagem analítica é dada pelo desvio padrão padronizado
(standard deviate ou Z-score) de 29,13. Este valor é extremamente alto
muito além do corte crítico de 1,96 para 95\% de confiança resultando em
um valor-p virtualmente nulo (\(< 2.2e-16\)). Isso nos permite rejeitar
a hipótese nula com um nível de confiança de 95\% (considerando que
fixamos nível de significância em 5\%) e confirmar a existência de
dependência espacial significativa.

A abordagem via simulação de Monte Carlo fortalece essa conclusão, sendo
tecnicamente preferível por não depender de pressupostos de normalidade
distributiva dos dados. O resultado mostra que, ao realizar 999
permutações aleatórias dos valores do indicador pelo mapa de Minas
Gerais, a estatística observada nos dados originais (statistic =
0,61937) foi superior a absolutamente todas as simulações geradas,
ocupando a posição máxima (observed rank) de 1000. Isso resulta em um
pseudo valor-p de 0,001, indicando que a probabilidade de se obter um
padrão espacial tão ou mais organizado quanto este por mero acaso é de
apenas 1 em 1000. Portanto, conclui-se que a variável analisada
apresenta dependência espacial positiva significativa.

\textbf{Índice C de Geary}

Proposto por \href{https://en.wikipedia.org/wiki/Roy_C._Geary}{Robert
Charles Geary} e desenvolvido por Cliff e Ord (1981), este indicador
(\(c\)) foca na dissimilaridade quadrática entre vizinhos, em vez da
covariância (produto cruzado):

\[
c = \frac{(n-1)}{2 S_0} \cdot \frac{\sum_{i=1}^n \sum_{j=1}^n w_{ij} (y_i - y_j)^2}{\sum_{i=1}^n (y_i - \bar{y})^2}.
\]

Enquanto o \(I\) de Moran é uma medida de covariância global, o \(c\) de
Geary assemelha-se ao variograma da geoestatística
(Seção~\ref{sec-variograma}), medindo a variância local das diferenças.

\begin{itemize}
\item
  \(0 < c < 1\): Autocorrelação positiva (vizinhos são similares,
  diferenças ao quadrado são pequenas).
\item
  \(c > 1\): Autocorrelação negativa (vizinhos são dissimilares).
\item
  \(c \approx 1\): Ausência de autocorrelação espacial.
\end{itemize}

Anselin (2001) nota que o \(I\) de Moran é mais sensível a tendências
globais e \emph{clusters}, enquanto o \(c\) de Geary é mais sensível a
diferenças locais e \emph{outliers} espaciais.

\begin{Shaded}
\begin{Highlighting}[]
\NormalTok{pacman}\SpecialCharTok{::}\FunctionTok{p\_load}\NormalTok{(ggspatial)}
\CommentTok{\#Cálculo do Índice C de Geary}

\CommentTok{\# A) Teste Analítico}
\NormalTok{geary\_analitico }\OtherTok{\textless{}{-}} \FunctionTok{geary.test}\NormalTok{(mg\_sf}\SpecialCharTok{$}\NormalTok{indicador, }\AttributeTok{listw=}\NormalTok{lw, }\AttributeTok{randomisation =} \ConstantTok{TRUE}\NormalTok{)}

\CommentTok{\# B) Teste Monte Carlo}
\FunctionTok{set.seed}\NormalTok{(}\DecValTok{123}\NormalTok{)}
\NormalTok{geary\_mc }\OtherTok{\textless{}{-}} \FunctionTok{geary.mc}\NormalTok{(mg\_sf}\SpecialCharTok{$}\NormalTok{indicador, }\AttributeTok{listw=}\NormalTok{lw, }\AttributeTok{nsim =} \DecValTok{999}\NormalTok{)}

\FunctionTok{print}\NormalTok{(geary\_analitico)}
\end{Highlighting}
\end{Shaded}

\begin{verbatim}

    Geary C test under randomisation

data:  mg_sf$indicador 
weights: lw   

Geary C statistic standard deviate = 26.149, p-value < 2.2e-16
alternative hypothesis: Expectation greater than statistic
sample estimates:
Geary C statistic       Expectation          Variance 
      0.377124962       1.000000000       0.000567393 
\end{verbatim}

\begin{Shaded}
\begin{Highlighting}[]
\FunctionTok{print}\NormalTok{(geary\_mc)}
\end{Highlighting}
\end{Shaded}

\begin{verbatim}

    Monte-Carlo simulation of Geary C

data:  mg_sf$indicador 
weights: lw  
number of simulations + 1: 1000 

statistic = 0.37712, observed rank = 1, p-value = 0.001
alternative hypothesis: greater
\end{verbatim}

\begin{Shaded}
\begin{Highlighting}[]
\CommentTok{\# Calculamos o Geary Local (localC) para ver onde vizinhos diferem muito.}
\CommentTok{\# Valores altos no mapa indicam vizinhos muito diferentes (outliers locais).}
\NormalTok{mg\_sf}\SpecialCharTok{$}\NormalTok{geary\_local }\OtherTok{\textless{}{-}} \FunctionTok{localC}\NormalTok{(mg\_sf}\SpecialCharTok{$}\NormalTok{indicador, }\AttributeTok{listw=}\NormalTok{lw)}

\CommentTok{\# Mapa da Variável Original}
\NormalTok{p1 }\OtherTok{\textless{}{-}} \FunctionTok{ggplot}\NormalTok{(mg\_sf) }\SpecialCharTok{+}
  \FunctionTok{geom\_sf}\NormalTok{(}\FunctionTok{aes}\NormalTok{(}\AttributeTok{fill =}\NormalTok{ indicador), }\AttributeTok{color =} \ConstantTok{NA}\NormalTok{) }\SpecialCharTok{+}
  \FunctionTok{scale\_fill\_viridis\_c}\NormalTok{(}\AttributeTok{option =} \StringTok{"magma"}\NormalTok{, }\AttributeTok{name =} \StringTok{"Valor"}\NormalTok{) }\SpecialCharTok{+}
  \FunctionTok{labs}\NormalTok{(}\AttributeTok{title =} \StringTok{"A. Variável Original"}\NormalTok{, }\AttributeTok{subtitle =} \StringTok{"Padrão Norte{-}Sul"}\NormalTok{) }\SpecialCharTok{+}
  \FunctionTok{theme\_void}\NormalTok{()}\SpecialCharTok{+} 
  \FunctionTok{annotation\_scale}\NormalTok{(}\AttributeTok{location =} \StringTok{"bl"}\NormalTok{, }\AttributeTok{width\_hint =} \FloatTok{0.3}\NormalTok{, }\AttributeTok{bar\_cols =} \FunctionTok{c}\NormalTok{(}\StringTok{"black"}\NormalTok{, }\StringTok{"white"}\NormalTok{)) }\SpecialCharTok{+}
  \FunctionTok{annotation\_north\_arrow}\NormalTok{(}\AttributeTok{location =} \StringTok{"tl"}\NormalTok{, }\AttributeTok{which\_north =} \StringTok{"true"}\NormalTok{, }
                         \AttributeTok{pad\_x =} \FunctionTok{unit}\NormalTok{(}\FloatTok{0.2}\NormalTok{, }\StringTok{"in"}\NormalTok{), }\AttributeTok{pad\_y =} \FunctionTok{unit}\NormalTok{(}\FloatTok{0.2}\NormalTok{, }\StringTok{"in"}\NormalTok{),}
                         \AttributeTok{style =}\NormalTok{ north\_arrow\_fancy\_orienteering)}

\CommentTok{\# Mapa de Geary Local}
\NormalTok{p2 }\OtherTok{\textless{}{-}} \FunctionTok{ggplot}\NormalTok{(mg\_sf) }\SpecialCharTok{+}
  \FunctionTok{geom\_sf}\NormalTok{(}\FunctionTok{aes}\NormalTok{(}\AttributeTok{fill =}\NormalTok{ geary\_local), }\AttributeTok{color =} \ConstantTok{NA}\NormalTok{) }\SpecialCharTok{+}
  \FunctionTok{scale\_fill\_viridis\_c}\NormalTok{(}\AttributeTok{option =} \StringTok{"cividis"}\NormalTok{, }\AttributeTok{name =} \StringTok{"Local C"}\NormalTok{) }\SpecialCharTok{+}
  \FunctionTok{labs}\NormalTok{(}\AttributeTok{title =} \StringTok{"B. Dissimilaridade Local (Geary)"}\NormalTok{, }
       \AttributeTok{subtitle =} \StringTok{"Áreas claras = Vizinhos muito diferentes"}\NormalTok{) }\SpecialCharTok{+}
  \FunctionTok{theme\_void}\NormalTok{()}\SpecialCharTok{+} 
  \FunctionTok{annotation\_scale}\NormalTok{(}\AttributeTok{location =} \StringTok{"bl"}\NormalTok{, }\AttributeTok{width\_hint =} \FloatTok{0.3}\NormalTok{, }\AttributeTok{bar\_cols =} \FunctionTok{c}\NormalTok{(}\StringTok{"black"}\NormalTok{, }\StringTok{"white"}\NormalTok{)) }\SpecialCharTok{+}
  \FunctionTok{annotation\_north\_arrow}\NormalTok{(}\AttributeTok{location =} \StringTok{"tl"}\NormalTok{, }\AttributeTok{which\_north =} \StringTok{"true"}\NormalTok{, }
                         \AttributeTok{pad\_x =} \FunctionTok{unit}\NormalTok{(}\FloatTok{0.2}\NormalTok{, }\StringTok{"in"}\NormalTok{), }\AttributeTok{pad\_y =} \FunctionTok{unit}\NormalTok{(}\FloatTok{0.2}\NormalTok{, }\StringTok{"in"}\NormalTok{),}
                         \AttributeTok{style =}\NormalTok{ north\_arrow\_fancy\_orienteering)}

\NormalTok{p1 }\SpecialCharTok{+}\NormalTok{ p2}
\end{Highlighting}
\end{Shaded}

\begin{figure}[H]

\centering{

\pandocbounded{\includegraphics[keepaspectratio]{lattice_data_files/figure-pdf/fig-geary-global-1.pdf}}

}

\caption{\label{fig-geary-global}Índice C de Geary: Teste Global e Mapa
de Dissimilaridade Local.}

\end{figure}%

\textbf{Interpretação}

A análise do Índice C de Geary confirma a existência de autocorrelação
espacial positiva fort. O valor estatístico observado (\emph{Geary C
statistic 0,377}) situa-se substancialmente abaixo do valor esperado sob
a hipótese de aleatoriedade espacial (\emph{Expectation = 1,0}). Como
\(0<c<1\), conclui-se que a variância local (diferença entre vizinhos) é
muito menor do que a variância global, ou seja, vizinhos tendem a ser
parecidos.

A significância estatística deste padrão é atestada pelo elevado desvio
padrão padronizado (\emph{standard deviate = 26,15}), que resulta em um
valor-p baixo (\emph{p-value \textless{} 2.2e-16}). Isso permite
rejeitar a hipótese nula com um nível de confiança de 95\% e confirmar
que a similaridade entre as unidades espaciais não é aleatória.

A validação via Monte Carlo reforça o resultado obtido. A estatística
observada (\emph{statistic = 0,377}) foi mais extrema (menor) do que
todas as 999 permutações aleatórias geradas, ocupando a primeira posição
no ranking de dissimilaridade (\emph{observed rank = 1}). Isso gera um
pseudo valor-p significativo (\emph{p-value = 0,001}), confirmando que a
probabilidade de tal padrão de aglomeração surgir ao acaso é desprezível
(1 em 1000).

\subsection{\texorpdfstring{Suporte das estatísticas \(I\) e
\(c\)}{Suporte das estatísticas I e c}}\label{suporte-das-estatuxedsticas-i-e-c}

É comum assumir incorretamente que o Índice de Moran varia estritamente
no intervalo \([-1, 1]\), tal como o coeficiente de correlação de
Pearson. No entanto, o suporte (intervalo de valores possíveis) das
estatísticas de autocorrelação espacial não é fixo (Scalon 2024); ele é
intrinsecamente dependente da topologia da rede definida por
\(\mathbf{W}=[w_{ij}]_{n \times n}\) (matriz de pesos ou vizinhança) e
pode ultrapassar esses limites dependendo da geometria dos vizinhos.

\textbf{Suporte do Índice de Moran}

Jong, Sprenger, e Veen (1984) discutem este problema abordando-o através
da álgebra matricial. Eles demonstram que os valores extremos (limites
mínimo e máximo) de \(I\) correspondem aos autovalores da matriz de
pesos, sujeitos à restrição de que os dados são centrados na média
(\(\mathbf{z}^\top \mathbf{1} = 0\), ver Eq.~\ref{eq-moran}).

Para uma dada matriz de pesos \(\mathbf{W}\), os limites inferior
(\(I_{min}\)) e superior (\(I_{max}\)), necessários para definir o
suporte do índice de Moran são dados por:

\[
I_{min} = \frac{n}{S_0} \lambda_{min} \quad \text{e} \quad I_{max} = \frac{n}{S_0} \lambda_{max}
\]

Onde \(\lambda_{min}\) e \(\lambda_{max}\) são, respectivamente, o menor
e o maior autovalor da matriz de pesos projetada no subespaço ortogonal
ao vetor constante.

Se \(\mathbf{W}\) for simétrica, estes são simplesmente os autovalores
extremos (menor e maior autovalor) de \(\mathbf{W}\) (excluindo o
autovalor trivial associado à média constante). Caso \(\mathbf{W}\) seja
assimétrica (comum em vizinhança por \(k\)-vizinhos), os autovalores são
calculados a partir da parte simétrica da matriz,
\(\mathbf{S} = \frac{1}{2}(\mathbf{W} + \mathbf{W}^\top)\).

\textbf{Extremos do Índice de Geary}

De forma análoga, Jong, Sprenger, e Veen (1984) derivaram os limites
para a estatística \(c\) de Geary reformulando o seu numerador como uma
forma quadrática. Os valores extremos são definidos como:

\[
c_{min} = \frac{n-1}{2 S_0} \gamma_{min} \quad \text{e} \quad c_{max} = \frac{n-1}{2 S_0} \gamma_{max}
\]

Neste caso, \(\gamma_{min}\) e \(\gamma_{max}\) representam os
autovalores extremos (menor e maior autovalor) de uma matriz auxiliar
\(\mathbf{B}\), cujos elementos \(b_{ij}\) são construídos combinando a
estrutura de conectividade com os totais marginais dos pesos6:

\[
b_{ij} = (R_i + K_j)\delta_{ij} - 2w_{ij}
\]

Sendo \(R_i\) a soma da linha \(i\) de \(\mathbf{W}\), \(K_j\) a soma da
coluna \(j\), e \(\delta_{ij}\) o delta de Kronecker (1 se \(i=j\), 0
caso contrário).

O cálculo destes limites exatos é crucial para a validação estatística.
Se utilizarmos uma aproximação teórica (como a distribuição Normal) que
atribua probabilidade a valores fora do intervalo
\([I_{min}, I_{max}]\), estaremos cometendo um erro de especificação,
atribuindo probabilidade a eventos impossíveis para aquela configuração
espacial específica. Além disso, Jong, Sprenger, e Veen (1984) notam que
a simetria desses limites fornece informação sobre a estrutura da rede;
por exemplo, grafos do tipo estrela possuem limites muito distintos de
reticulados regulares (grids), o que afeta a interpretação da magnitude
da autocorrelação.

\subsection{Estatísticas Locais}\label{estatuxedsticas-locais}

A dependência espacial raramente é estacionária sobre todo o domínio
\(D^L\). Anselin (1995) introduziu o conceito de LISA (\emph{Local
Indicators of Spatial Association}) para decompor a estatística global
em contribuições individuais \(I_i\), satisfazendo duas propriedades:

\begin{enumerate}
\def\labelenumi{\arabic{enumi}.}
\item
  O indicador \(I_i\) permite avaliar a significância do padrão espacial
  local em torno da unidade \(i\).
\item
  A soma dos indicadores locais é proporcional à estatística global (ex:
  \(\sum_i I_i \propto I\)).
\end{enumerate}

\textbf{Índice I de Moran local}

O Moran local (\(I_i\)) avalia a correlação entre o valor de uma unidade
e a média dos seus vizinhos (o \emph{spatial lag})
Figura~\ref{fig-IndiceMoranLocal}. É definido como:

\[
I_i = \frac{(y_i - \bar{y})}{S^2} \cdot \sum_{j=1}^n w_{ij} (y_j - \bar{y}),
\]

onde \(S^2 = \sum_{j=1}^n (y_j - \bar{y})^2 / n\) é a variância
amostral.

Esta estatística é a base para o mapa de clusters LISA, que classifica
cada localidade estatisticamente significativa em quatro quadrantes,
baseados no sinal de \(z_i = (y_i - \bar{y})\) (valor local) e do seu
defasamento espacial (\emph{spatial lag})
\(\sum_j w_{ij} (y_j - \bar{y})\) (valor dos vizinhos):

\begin{enumerate}
\def\labelenumi{\arabic{enumi}.}
\item
  Alto-Alto (High-High \textbar{} HH): Este quadrante (superior direito)
  representa o regime de associação espacial positiva onde uma unidade
  com valor acima da média é circundada por vizinhos que também possuem
  valores altos. Estatisticamente, identifica-se aqui a formação de
  clusters (agrupamentos) de alta intensidade, conhecidos como \emph{hot
  spots}. A sua presença sugere fortes fenômenos de contágio ou
  transbordamento (\emph{spillover}), indicando que os fatores que
  elevam a variável numa localidade estão também presentes e ativos na
  sua vizinhança imediata.
\item
  Baixo-Baixo (Low-Low \textbar{} LL): Define-se pelo agrupamento de uma
  unidade com valor abaixo da média cercada por \(j\) vizinhos que
  compartilham essa característica de baixa intensidade. Este padrão,
  denominado \emph{cold spot}, indica também autocorrelação espacial
  positiva, mas na direção oposta aos \emph{hot spots}. Podendo refletir
  barreiras geográficas à difusão de um fenômeno ou áreas
  estruturalmente desfavorecidas (ou protegidas, dependendo se a
  variável é benéfica ou maléfica) em bloco.
\item
  Alto-Baixo (High-Low \textbar{} HL): Caracteriza-se por uma unidade
  com valor alto que está isolada em meio a uma vizinhança de valores
  predominantemente baixos, configurando um outlier espacial. Este
  padrão de autocorrelação negativa sugere que o processo gerador de
  dados na unidade central é distinto do seu entorno. Frequentemente
  indica vulnerabilidades locais específicas, erros de medição, ou um
  surto localizado que, por algum motivo de contenção, não transbordou
  para as regiões adjacentes.
\item
  Baixo-Alto (Low-High \textbar{} LH): Refere-se a uma unidade com valor
  abaixo da média que está cercada por vizinhos com valores altos. Como
  um outlier espacial inverso, esta configuração indica autocorrelação
  negativa. Pode sinalizar a eficácia de políticas de contenção locais
  (resiliência), subnotificação de dados em relação aos vizinhos, ou
  características intrínsecas que tornam a unidade impermeável à
  influência do seu entorno de alta intensidade.
\end{enumerate}

\begin{Shaded}
\begin{Highlighting}[]
\NormalTok{pacman}\SpecialCharTok{::}\FunctionTok{p\_load}\NormalTok{(tidyverse,sf,spdep,geobr,patchwork, ggtext, ggspatial)  }


\CommentTok{\# Cálculo do Moran Local}
\NormalTok{loc\_m }\OtherTok{\textless{}{-}} \FunctionTok{localmoran}\NormalTok{(mg\_sf}\SpecialCharTok{$}\NormalTok{indicador, }\AttributeTok{listw=}\NormalTok{lw)}

\CommentTok{\# Preparar dados para plotagem}
\NormalTok{mg\_sf}\SpecialCharTok{$}\NormalTok{z\_score }\OtherTok{\textless{}{-}} \FunctionTok{as.numeric}\NormalTok{(}\FunctionTok{scale}\NormalTok{(mg\_sf}\SpecialCharTok{$}\NormalTok{indicador)) }
\NormalTok{mg\_sf}\SpecialCharTok{$}\NormalTok{lag\_z   }\OtherTok{\textless{}{-}} \FunctionTok{lag.listw}\NormalTok{(lw, mg\_sf}\SpecialCharTok{$}\NormalTok{z\_score)    }
\NormalTok{mg\_sf}\SpecialCharTok{$}\NormalTok{p\_value }\OtherTok{\textless{}{-}}\NormalTok{ loc\_m[, }\DecValTok{5}\NormalTok{] }\CommentTok{\# P{-}valor do teste local}

\CommentTok{\# Classificação dos Quadrantes}
\NormalTok{sig\_level }\OtherTok{\textless{}{-}} \FloatTok{0.05}

\NormalTok{mg\_sf }\OtherTok{\textless{}{-}}\NormalTok{ mg\_sf }\SpecialCharTok{\%\textgreater{}\%} 
  \FunctionTok{mutate}\NormalTok{(}\AttributeTok{quadrante =} \FunctionTok{case\_when}\NormalTok{(}
\NormalTok{    p\_value }\SpecialCharTok{\textgreater{}}\NormalTok{ sig\_level }\SpecialCharTok{\textasciitilde{}} \StringTok{"NS"}\NormalTok{,}
\NormalTok{    z\_score }\SpecialCharTok{\textgreater{}} \DecValTok{0} \SpecialCharTok{\&}\NormalTok{ lag\_z }\SpecialCharTok{\textgreater{}} \DecValTok{0} \SpecialCharTok{\textasciitilde{}} \StringTok{"HH"}\NormalTok{,}
\NormalTok{    z\_score }\SpecialCharTok{\textless{}} \DecValTok{0} \SpecialCharTok{\&}\NormalTok{ lag\_z }\SpecialCharTok{\textless{}} \DecValTok{0} \SpecialCharTok{\textasciitilde{}} \StringTok{"LL"}\NormalTok{,}
\NormalTok{    z\_score }\SpecialCharTok{\textgreater{}} \DecValTok{0} \SpecialCharTok{\&}\NormalTok{ lag\_z }\SpecialCharTok{\textless{}} \DecValTok{0} \SpecialCharTok{\textasciitilde{}} \StringTok{"HL"}\NormalTok{,}
\NormalTok{    z\_score }\SpecialCharTok{\textless{}} \DecValTok{0} \SpecialCharTok{\&}\NormalTok{ lag\_z }\SpecialCharTok{\textgreater{}} \DecValTok{0} \SpecialCharTok{\textasciitilde{}} \StringTok{"LH"}
\NormalTok{  )) }\SpecialCharTok{\%\textgreater{}\%}
  \FunctionTok{mutate}\NormalTok{(}\AttributeTok{quadrante =} \FunctionTok{factor}\NormalTok{(quadrante, }
                            \AttributeTok{levels =} \FunctionTok{c}\NormalTok{(}\StringTok{"HH"}\NormalTok{, }\StringTok{"LL"}\NormalTok{, }\StringTok{"HL"}\NormalTok{, }\StringTok{"LH"}\NormalTok{, }\StringTok{"NS"}\NormalTok{),}
                            \AttributeTok{labels =} \FunctionTok{c}\NormalTok{(}\StringTok{"Alto{-}Alto (HH)"}\NormalTok{, }\StringTok{"Baixo{-}Baixo (LL)"}\NormalTok{, }
                                       \StringTok{"Alto{-}Baixo (HL)"}\NormalTok{, }\StringTok{"Baixo{-}Alto (LH)"}\NormalTok{, }
                                       \StringTok{"Não Significativo"}\NormalTok{)))}

\CommentTok{\# Cores}
\NormalTok{cores\_lisa }\OtherTok{\textless{}{-}} \FunctionTok{c}\NormalTok{(}
  \StringTok{"Alto{-}Alto (HH)"} \OtherTok{=} \StringTok{"\#FF0000"}\NormalTok{,    }
  \StringTok{"Baixo{-}Baixo (LL)"} \OtherTok{=} \StringTok{"\#0000FF"}\NormalTok{, }
  \StringTok{"Alto{-}Baixo (HL)"} \OtherTok{=} \StringTok{"\#FFA500"}\NormalTok{,  }
  \StringTok{"Baixo{-}Alto (LH)"} \OtherTok{=} \StringTok{"\#87CEFA"}\NormalTok{,  }
  \StringTok{"Não Significativo"} \OtherTok{=} \StringTok{"\#eeeeee"}
\NormalTok{)}

\CommentTok{\# Mapa}
\NormalTok{g\_map }\OtherTok{\textless{}{-}} \FunctionTok{ggplot}\NormalTok{(mg\_sf) }\SpecialCharTok{+}
  \FunctionTok{geom\_sf}\NormalTok{(}\FunctionTok{aes}\NormalTok{(}\AttributeTok{fill =}\NormalTok{ quadrante), }\AttributeTok{color =} \StringTok{"black"}\NormalTok{, }\AttributeTok{size =} \FloatTok{0.05}\NormalTok{) }\SpecialCharTok{+}
  \FunctionTok{scale\_fill\_manual}\NormalTok{(}\AttributeTok{values =}\NormalTok{ cores\_lisa) }\SpecialCharTok{+}
  \FunctionTok{theme\_void}\NormalTok{() }\SpecialCharTok{+}
  \FunctionTok{labs}\NormalTok{(}\AttributeTok{title =} \StringTok{"A. LISA"}\NormalTok{,}
       \AttributeTok{subtitle =} \StringTok{"Identificação de regimes locais (p \textless{} 0.05)"}\NormalTok{, }
       \AttributeTok{fill=}\StringTok{"Legenda"}\NormalTok{) }\SpecialCharTok{+}
  \FunctionTok{theme}\NormalTok{(}\AttributeTok{plot.title =} \FunctionTok{element\_text}\NormalTok{(}\AttributeTok{size =} \DecValTok{12}\NormalTok{, }\AttributeTok{face =} \StringTok{"bold"}\NormalTok{))}\SpecialCharTok{+}
  
  \FunctionTok{annotation\_scale}\NormalTok{(}
    \AttributeTok{location =} \StringTok{"bl"}\NormalTok{,           }
    \AttributeTok{width\_hint =} \FloatTok{0.3}\NormalTok{,          }
    \AttributeTok{bar\_cols =} \FunctionTok{c}\NormalTok{(}\StringTok{"black"}\NormalTok{, }\StringTok{"white"}\NormalTok{), }
    \AttributeTok{text\_family =} \StringTok{"sans"}       
\NormalTok{  ) }\SpecialCharTok{+}
  
  \FunctionTok{annotation\_north\_arrow}\NormalTok{(}
    \AttributeTok{location =} \StringTok{"tl"}\NormalTok{,           }
    \AttributeTok{which\_north =} \StringTok{"true"}\NormalTok{,      }
    \AttributeTok{pad\_x =} \FunctionTok{unit}\NormalTok{(}\FloatTok{0.2}\NormalTok{, }\StringTok{"in"}\NormalTok{),   }
    \AttributeTok{pad\_y =} \FunctionTok{unit}\NormalTok{(}\FloatTok{0.2}\NormalTok{, }\StringTok{"in"}\NormalTok{),   }
    \AttributeTok{style =}\NormalTok{ north\_arrow\_fancy\_orienteering }
\NormalTok{  )}

\CommentTok{\#}
\NormalTok{g\_scatter }\OtherTok{\textless{}{-}} \FunctionTok{ggplot}\NormalTok{(mg\_sf, }\FunctionTok{aes}\NormalTok{(}\AttributeTok{x =}\NormalTok{ z\_score, }\AttributeTok{y =}\NormalTok{ lag\_z)) }\SpecialCharTok{+}
  \FunctionTok{geom\_hline}\NormalTok{(}\AttributeTok{yintercept =} \DecValTok{0}\NormalTok{, }\AttributeTok{linetype =} \StringTok{"dashed"}\NormalTok{, }\AttributeTok{color =} \StringTok{"gray50"}\NormalTok{) }\SpecialCharTok{+}
  \FunctionTok{geom\_vline}\NormalTok{(}\AttributeTok{xintercept =} \DecValTok{0}\NormalTok{, }\AttributeTok{linetype =} \StringTok{"dashed"}\NormalTok{, }\AttributeTok{color =} \StringTok{"gray50"}\NormalTok{) }\SpecialCharTok{+}
  \FunctionTok{geom\_point}\NormalTok{(}\FunctionTok{aes}\NormalTok{(}\AttributeTok{color =}\NormalTok{ quadrante), }\AttributeTok{size =} \FloatTok{1.5}\NormalTok{, }\AttributeTok{alpha =} \FloatTok{0.6}\NormalTok{) }\SpecialCharTok{+}
  \CommentTok{\# Linha de regressão (Moran Global)}
  \FunctionTok{geom\_smooth}\NormalTok{(}\AttributeTok{method =} \StringTok{"lm"}\NormalTok{, }\AttributeTok{se =} \ConstantTok{FALSE}\NormalTok{, }\AttributeTok{color =} \StringTok{"black"}\NormalTok{, }\AttributeTok{size =} \FloatTok{0.8}\NormalTok{) }\SpecialCharTok{+}
  \CommentTok{\# Anotações dos Quadrantes}
  \FunctionTok{annotate}\NormalTok{(}\StringTok{"text"}\NormalTok{, }\AttributeTok{x =} \DecValTok{2}\NormalTok{, }\AttributeTok{y =} \DecValTok{2}\NormalTok{, }\AttributeTok{label =} \StringTok{"HH"}\NormalTok{, }\AttributeTok{color =} \StringTok{"red"}\NormalTok{, }\AttributeTok{fontface=}\StringTok{"bold"}\NormalTok{) }\SpecialCharTok{+}
  \FunctionTok{annotate}\NormalTok{(}\StringTok{"text"}\NormalTok{, }\AttributeTok{x =} \SpecialCharTok{{-}}\DecValTok{2}\NormalTok{, }\AttributeTok{y =} \SpecialCharTok{{-}}\DecValTok{2}\NormalTok{, }\AttributeTok{label =} \StringTok{"LL"}\NormalTok{, }\AttributeTok{color =} \StringTok{"blue"}\NormalTok{, }\AttributeTok{fontface=}\StringTok{"bold"}\NormalTok{) }\SpecialCharTok{+}
  \FunctionTok{annotate}\NormalTok{(}\StringTok{"text"}\NormalTok{, }\AttributeTok{x =} \DecValTok{2}\NormalTok{, }\AttributeTok{y =} \SpecialCharTok{{-}}\DecValTok{1}\NormalTok{, }\AttributeTok{label =} \StringTok{"HL"}\NormalTok{, }\AttributeTok{color =} \StringTok{"orange"}\NormalTok{, }\AttributeTok{fontface=}\StringTok{"bold"}\NormalTok{) }\SpecialCharTok{+}
  \FunctionTok{annotate}\NormalTok{(}\StringTok{"text"}\NormalTok{, }\AttributeTok{x =} \SpecialCharTok{{-}}\DecValTok{2}\NormalTok{, }\AttributeTok{y =} \DecValTok{1}\NormalTok{, }\AttributeTok{label =} \StringTok{"LH"}\NormalTok{, }\AttributeTok{color =} \StringTok{"\#87CEFA"}\NormalTok{, }\AttributeTok{fontface=}\StringTok{"bold"}\NormalTok{) }\SpecialCharTok{+}
  \FunctionTok{scale\_color\_manual}\NormalTok{(}\AttributeTok{values =}\NormalTok{ cores\_lisa) }\SpecialCharTok{+}
  \FunctionTok{labs}\NormalTok{(}\AttributeTok{title =} \StringTok{"B. Diagrama de Dispersão"}\NormalTok{,}
       \AttributeTok{subtitle =} \FunctionTok{paste}\NormalTok{(}\StringTok{"I de Moran Global:"}\NormalTok{, }\FunctionTok{round}\NormalTok{(}\FunctionTok{moran.test}\NormalTok{(mg\_sf}\SpecialCharTok{$}\NormalTok{indicador, lw)}\SpecialCharTok{$}\NormalTok{estimate[}\DecValTok{1}\NormalTok{], }\DecValTok{3}\NormalTok{)),}
       \AttributeTok{x =} \StringTok{"Valor Padronizado (y)"}\NormalTok{,}
       \AttributeTok{y =} \StringTok{"Defasagem Espacial (Wy)"}\NormalTok{) }\SpecialCharTok{+}
  \FunctionTok{theme\_minimal}\NormalTok{() }\SpecialCharTok{+}
  \FunctionTok{theme}\NormalTok{(}\AttributeTok{legend.position =} \StringTok{"none"}\NormalTok{,}
        \AttributeTok{plot.title =} \FunctionTok{element\_text}\NormalTok{(}\AttributeTok{size =} \DecValTok{12}\NormalTok{, }\AttributeTok{face =} \StringTok{"bold"}\NormalTok{))}


\NormalTok{g\_map }\SpecialCharTok{+}\NormalTok{ g\_scatter}
\end{Highlighting}
\end{Shaded}

\begin{figure}[H]

\centering{

\pandocbounded{\includegraphics[keepaspectratio]{lattice_data_files/figure-pdf/fig-IndiceMoranLocal-1.pdf}}

}

\caption{\label{fig-IndiceMoranLocal}Clusters Espaciais LISA (Local
Moran) para Minas Gerais}

\end{figure}%

\textbf{Interpretação}

Os resultados (Figura~\ref{fig-IndiceMoranLocal}) confirmam e localizam
a intensa polarização espacial sugerida pelas estatísticas globais. O
mapa de LISA (Figura~\ref{fig-IndiceMoranLocal} A) decompõe a
dependência, evidenciando um aglomerado do tipo Alto-Alto (HH) na porção
sul do estado (em vermelho), onde municípios com altos valores do
indicador estão circundados por vizinhos com valores igualmente
elevados. Em contrapartida, observa-se um vasto aglomerado do tipo
Baixo-Baixo (LL) dominando as regiões Norte e Nordeste (em azul),
caracterizando um regime espacial de valores baixos cercados por
vizinhos também com baixos valores.

O Diagrama de dispersão de Moran (Figura~\ref{fig-IndiceMoranLocal} B)
mostra a quase totalidade das observações significativas (pontos
coloridos) se concentra nos quadrantes de associação positiva (HH e LL).
A escassez de pontos nos quadrantes de transição ou outliers espaciais
(HL e LH) reflete a alta continuidade do fenômeno e a existência de
fronteiras nítidas entre os regimes. A inclinação da reta de regressão,
correspondente ao I de Moran Global de 0,619, atesta que o valor de um
município é um forte preditor positivo da média de seus vizinhos,
validando estatisticamente o gradiente Norte-Sul presente nos dados.

\textbf{Estatísticas Getis-Ord (}\(G_i\) e \(G_i^*\))

Desenvolvidas por Getis e Ord (1992) e Getis (2010), estas estatísticas
focam especificamente na deteção de agrupamentos de valores altos ou
baixos baseados em distância, sem a componente de covariância negativa
do Moran.

\begin{itemize}
\item
  \(G_i\) (sem auto-inclusão): Mede a concentração de valores na
  vizinhança, excluindo \(y_i\).
\item
  \(G_i^*\) (com auto-inclusão): Mede a concentração incluindo o valor
  da própria unidade \(i\):

  \[
  G_i^* = \frac{\sum_{j=1}^n w_{ij} y_j}{\sum_{j=1}^n y_j},
  \]
\end{itemize}

onde \(\mathbf{W}\) é tipicamente uma matriz de pesos baseada em
distância binária (1 se \(d_{ij} < d_{\max}\), 0 caso contrário).

Um valor de \(G_i^*\) significativamente positivo (Z-score alto) indica
um \emph{hot spot} (agrupamento de valores altos); um valor
significativamente negativo indica um \emph{cold spot}. Diferentemente
do Moran local, o \(G_i^*\) não distingue um outlier espacial de um
cluster fraco, sendo uma medida pura de intensidade local.

\begin{Shaded}
\begin{Highlighting}[]
\ControlFlowTok{if}\NormalTok{ (}\SpecialCharTok{!}\FunctionTok{require}\NormalTok{(}\StringTok{"pacman"}\NormalTok{)) }\FunctionTok{install.packages}\NormalTok{(}\StringTok{"pacman"}\NormalTok{)}
\NormalTok{pacman}\SpecialCharTok{::}\FunctionTok{p\_load}\NormalTok{(tidyverse, sf, spdep, geobr, ggspatial)}

\CommentTok{\#Cálculo do Getis{-}Ord Gi*}
\CommentTok{\# A função retorna os Z{-}scores (desvios padrão)}
\NormalTok{gi\_star }\OtherTok{\textless{}{-}} \FunctionTok{localG}\NormalTok{(mg\_sf}\SpecialCharTok{$}\NormalTok{indicador, }\AttributeTok{listw=}\NormalTok{lw)}

\CommentTok{\# Adicionar ao mapa}
\NormalTok{mg\_sf}\SpecialCharTok{$}\NormalTok{gi\_zscore }\OtherTok{\textless{}{-}} \FunctionTok{as.numeric}\NormalTok{(gi\_star)}

\CommentTok{\# Classificação para o Mapa (Níveis de Confiança)}
\CommentTok{\# Baseado na distribuição Normal Padrão}
\NormalTok{mg\_sf}\SpecialCharTok{$}\NormalTok{classificacao }\OtherTok{\textless{}{-}} \FunctionTok{case\_when}\NormalTok{(}
\NormalTok{  mg\_sf}\SpecialCharTok{$}\NormalTok{gi\_zscore }\SpecialCharTok{\textgreater{}=} \FloatTok{1.96}  \SpecialCharTok{\textasciitilde{}} \StringTok{"Hot Spot (95\%)"}\NormalTok{,}
\NormalTok{  mg\_sf}\SpecialCharTok{$}\NormalTok{gi\_zscore }\SpecialCharTok{\textless{}=} \SpecialCharTok{{-}}\FloatTok{1.96} \SpecialCharTok{\textasciitilde{}} \StringTok{"Cold Spot (95\%)"}\NormalTok{,}
  \ConstantTok{TRUE} \SpecialCharTok{\textasciitilde{}} \StringTok{"Não Significativo"}
\NormalTok{)}

\NormalTok{cores\_gi }\OtherTok{\textless{}{-}} \FunctionTok{c}\NormalTok{(}
  \StringTok{"Hot Spot (95\%)"} \OtherTok{=} \StringTok{"\#d7191c"}\NormalTok{,}
  \StringTok{"Cold Spot (95\%)"} \OtherTok{=} \StringTok{"\#2c7bb6"}\NormalTok{,}
  \StringTok{"Não Significativo"} \OtherTok{=} \StringTok{"gray90"}
\NormalTok{)}

\FunctionTok{ggplot}\NormalTok{(mg\_sf) }\SpecialCharTok{+}
  \FunctionTok{geom\_sf}\NormalTok{(}\FunctionTok{aes}\NormalTok{(}\AttributeTok{fill =}\NormalTok{ classificacao), }\AttributeTok{color =} \StringTok{"white"}\NormalTok{, }\AttributeTok{size =} \FloatTok{0.05}\NormalTok{) }\SpecialCharTok{+}
  \FunctionTok{scale\_fill\_manual}\NormalTok{(}\AttributeTok{values =}\NormalTok{ cores\_gi, }\AttributeTok{name =} \StringTok{"Intensidade (Gi*)"}\NormalTok{) }\SpecialCharTok{+}
  \FunctionTok{theme\_void}\NormalTok{() }\SpecialCharTok{+}
  \FunctionTok{labs}\NormalTok{(}\AttributeTok{title =} \StringTok{"Análise de Hot Spots (Getis{-}Ord Gi*)"}\NormalTok{,}
       \AttributeTok{subtitle =} \StringTok{"Identificação de aglomerados de alta e baixa intensidade"}\NormalTok{) }\SpecialCharTok{+}
  \FunctionTok{theme}\NormalTok{(}\AttributeTok{plot.title =} \FunctionTok{element\_text}\NormalTok{(}\AttributeTok{size =} \DecValTok{14}\NormalTok{, }\AttributeTok{face =} \StringTok{"bold"}\NormalTok{),}
        \AttributeTok{legend.position =} \StringTok{"right"}\NormalTok{) }\SpecialCharTok{+}
  
  \FunctionTok{annotation\_scale}\NormalTok{(}
    \AttributeTok{location =} \StringTok{"bl"}\NormalTok{,           }
    \AttributeTok{width\_hint =} \FloatTok{0.3}\NormalTok{,          }
    \AttributeTok{bar\_cols =} \FunctionTok{c}\NormalTok{(}\StringTok{"black"}\NormalTok{, }\StringTok{"white"}\NormalTok{), }
    \AttributeTok{text\_family =} \StringTok{"sans"}       
\NormalTok{  ) }\SpecialCharTok{+}
  
  \FunctionTok{annotation\_north\_arrow}\NormalTok{(}
    \AttributeTok{location =} \StringTok{"tl"}\NormalTok{,           }
    \AttributeTok{which\_north =} \StringTok{"true"}\NormalTok{,      }
    \AttributeTok{pad\_x =} \FunctionTok{unit}\NormalTok{(}\FloatTok{0.2}\NormalTok{, }\StringTok{"in"}\NormalTok{),   }
    \AttributeTok{pad\_y =} \FunctionTok{unit}\NormalTok{(}\FloatTok{0.2}\NormalTok{, }\StringTok{"in"}\NormalTok{),   }
    \AttributeTok{style =}\NormalTok{ north\_arrow\_fancy\_orienteering }
\NormalTok{  )}
\end{Highlighting}
\end{Shaded}

\begin{figure}[H]

\centering{

\pandocbounded{\includegraphics[keepaspectratio]{lattice_data_files/figure-pdf/fig-GetisOrd-MG-1.pdf}}

}

\caption{\label{fig-GetisOrd-MG}Análise de Hot Spots (Getis-Ord Gi*)
para Minas Gerais.}

\end{figure}%

\textbf{Interpretação}

O mapa (Figura~\ref{fig-GetisOrd-MG}) de análise de \emph{Hot Spots
(Getis-Ord Gi}) mostra uma clara polarização na distribuição espacial
dos valores de intensidade em Minas Gerais. Observa-se a formação de um
aglomerado significativo de alta intensidade (\emph{Hot Spot}),
representado em vermelho, concentrado na região sul do estado. Essa área
indica uma aglomeração de municípios com valores elevados cercados por
vizinhos também com valores elevados, sugerindo uma zona de maior
atividade ou ocorrência do fenômeno analisado com 95\% de confiança.

Em contrapartida, a região norte e nordeste do estado é caracterizada
por um aglomerado de baixa intensidade (\emph{Cold Spot}), representado
em azul. Essa configuração espacial aponta para uma área onde predominam
municípios com valores baixos, rodeados por outros de características
similares, indicando uma depressão ou menor intensidade do fenômeno
nesta porção do território. As áreas em cinza, classificadas como ``Não
Significativo'', representam regiões de transição ou de aleatoriedade
espacial, onde não se verifica um padrão de aglomeração significativo
nem para altas nem para baixas intensidades.

\section{Riscos e Suavização}\label{sec-risco}

A visualização de dados de área brutos, especialmente taxas (ex:
incidência de doenças), sofre de instabilidade intrínseca da variância,
um problema amplamente discutido por Noel Cressie e Chan (1989) na
análise de SIDS (Síndrome de Morte Súbita Infantil).

\textbf{Mapas de risco e razão padronizada (SMR)}

Para dados de contagem \(O_i\) (observados) com uma população em risco
\(P_i\), a taxa bruta é \(r_i = O_i / P_i\). Em epidemiologia,
utiliza-se frequentemente a razão de morbidade/mortalidade padronizada
(SMR - \emph{Standardized Mortality/Morbidity Ratio}):

\[
\text{SMR}_i = \frac{O_i}{E_i},
\]

onde \(E_i = P_i \cdot \bar{r}\) é o número esperado de casos sob a
hipótese de uma taxa constante global
\(\bar{r} = \sum_i O_i / \sum_i P_i\).

Em áreas com população \(P_i\) pequena, a variância da taxa \(r_i\) é
altíssima. Um único caso adicional pode duplicar a taxa, criando
\emph{outliers} espúrios que dominam visualmente o mapa coroplético.

Para mitigar a instabilidade, utiliza-se a suavização Bayesiana
empírica, onde a taxa estimada \(\theta_i\) é uma média ponderada entre
a taxa local (instável) e a média global (estável) ou a média da
vizinhança local (Banerjee 2016).

\[
\hat{\theta}_i^{EB} = \gamma_i \, r_i + (1 - \gamma_i) \, \bar{r}.
\]

O peso \(\gamma_i \in [0, 1]\) depende da população \(P_i\): áreas
populosas têm \(\gamma_i \approx 1\) (confia-se no dado local), áreas
despovoadas têm \(\gamma_i \approx 0\) (o valor é encolhido em direção à
média). Julian Besag e Green (1993) estendem isto para modelos
hierárquicos completos
(\href{https://sites.stat.columbia.edu/gelman/research/published/bym_article_SSTEproof.pdf}{BYM}),
utilizando a estrutura de vizinhança \(\mathbf{W}\) para suavizar
localmente.

\begin{Shaded}
\begin{Highlighting}[]
\ControlFlowTok{if}\NormalTok{ (}\SpecialCharTok{!}\FunctionTok{require}\NormalTok{(}\StringTok{"pacman"}\NormalTok{)) }\FunctionTok{install.packages}\NormalTok{(}\StringTok{"pacman"}\NormalTok{)}
\NormalTok{pacman}\SpecialCharTok{::}\FunctionTok{p\_load}\NormalTok{(tidyverse, sf, spdep, geobr, patchwork, viridis, ggspatial)}

\NormalTok{mg\_sf }\OtherTok{\textless{}{-}} \FunctionTok{read\_municipality}\NormalTok{(}\AttributeTok{code\_muni =} \StringTok{"MG"}\NormalTok{, }\AttributeTok{year =} \DecValTok{2020}\NormalTok{, }\AttributeTok{showProgress =} \ConstantTok{FALSE}\NormalTok{)}

\CommentTok{\#Vamos simulação de um dataframe de dados brutos (Como viria do DATASUS)}
\CommentTok{\# Cenário: Incidência de uma doença rara.}
\CommentTok{\# O usuário teria uma tabela com: ID do Município, População em Risco, Nº de Casos.}

\FunctionTok{set.seed}\NormalTok{(}\DecValTok{999}\NormalTok{)}

\CommentTok{\# A) Simular População (ni):}
\NormalTok{populacao\_simulada }\OtherTok{\textless{}{-}} \FunctionTok{floor}\NormalTok{(}\FunctionTok{rlnorm}\NormalTok{(}\FunctionTok{nrow}\NormalTok{(mg\_sf), }\AttributeTok{meanlog =} \DecValTok{9}\NormalTok{, }\AttributeTok{sdlog =} \FloatTok{1.2}\NormalTok{))}

\CommentTok{\# B) Definir um Risco Latente (Verdadeiro, mas desconhecido):}

\NormalTok{coords }\OtherTok{\textless{}{-}} \FunctionTok{st\_coordinates}\NormalTok{(}\FunctionTok{st\_centroid}\NormalTok{(mg\_sf))}
\NormalTok{padrao\_norte\_sul }\OtherTok{\textless{}{-}}\NormalTok{ (coords[,}\DecValTok{2}\NormalTok{] }\SpecialCharTok{{-}} \FunctionTok{min}\NormalTok{(coords[,}\DecValTok{2}\NormalTok{])) }\SpecialCharTok{/}\NormalTok{ (}\FunctionTok{max}\NormalTok{(coords[,}\DecValTok{2}\NormalTok{]) }\SpecialCharTok{{-}} \FunctionTok{min}\NormalTok{(coords[,}\DecValTok{2}\NormalTok{]))}

\NormalTok{risco\_real }\OtherTok{\textless{}{-}} \FloatTok{0.0005} \SpecialCharTok{*}\NormalTok{ (}\DecValTok{1} \SpecialCharTok{+}\NormalTok{ (padrao\_norte\_sul }\SpecialCharTok{*} \DecValTok{2}\NormalTok{)) }\CommentTok{\# Risco varia de 0.05\% a 0.15\%}

\CommentTok{\# Simular Nº de Casos (yi):}
\CommentTok{\# Os casos são uma realização de um processo de Poisson: Casos \textasciitilde{} Poisson(Pop * Risco)}
\NormalTok{casos\_simulados }\OtherTok{\textless{}{-}} \FunctionTok{rpois}\NormalTok{(}\FunctionTok{nrow}\NormalTok{(mg\_sf), }\AttributeTok{lambda =}\NormalTok{ populacao\_simulada }\SpecialCharTok{*}\NormalTok{ risco\_real)}

\CommentTok{\# D) Unir ao mapa }
\NormalTok{mg\_dados }\OtherTok{\textless{}{-}}\NormalTok{ mg\_sf }\SpecialCharTok{\%\textgreater{}\%}
  \FunctionTok{mutate}\NormalTok{(}
    \AttributeTok{populacao =}\NormalTok{ populacao\_simulada,  }
    \AttributeTok{casos =}\NormalTok{ casos\_simulados          }\CommentTok{\#Óbitos ou Doentes}
\NormalTok{  )}

\CommentTok{\# Cálculo das Taxas }

\CommentTok{\# Passo 1: Calcular Taxa Bruta (Incidência por 10.000 habitantes)}

\NormalTok{mg\_dados }\OtherTok{\textless{}{-}}\NormalTok{ mg\_dados }\SpecialCharTok{\%\textgreater{}\%}
  \FunctionTok{mutate}\NormalTok{(}\AttributeTok{taxa\_bruta =}\NormalTok{ (casos }\SpecialCharTok{/}\NormalTok{ populacao) }\SpecialCharTok{*} \DecValTok{10000}\NormalTok{)}

\CommentTok{\# Passo 2: Suavização Bayesiana Empírica Local }

\NormalTok{nb }\OtherTok{\textless{}{-}} \FunctionTok{poly2nb}\NormalTok{(mg\_dados, }\AttributeTok{queen =} \ConstantTok{TRUE}\NormalTok{)}

\CommentTok{\# A função EBlocal precisa dos CASOS (ri) e da POPULAÇÃO (ni)}

\NormalTok{eb\_resultado }\OtherTok{\textless{}{-}} \FunctionTok{EBlocal}\NormalTok{(}\AttributeTok{ri =}\NormalTok{ mg\_dados}\SpecialCharTok{$}\NormalTok{casos, }
                        \AttributeTok{ni =}\NormalTok{ mg\_dados}\SpecialCharTok{$}\NormalTok{populacao, }
                        \AttributeTok{nb =}\NormalTok{ nb, }
                        \AttributeTok{zero.policy =} \ConstantTok{TRUE}\NormalTok{)}

\CommentTok{\# Adicionamos a taxa suavizada ao mapa (multiplicando por 10k para ficar na mesma escala)}
\NormalTok{mg\_dados}\SpecialCharTok{$}\NormalTok{taxa\_suavizada }\OtherTok{\textless{}{-}}\NormalTok{ eb\_resultado}\SpecialCharTok{$}\NormalTok{est }\SpecialCharTok{*} \DecValTok{10000}


\CommentTok{\# Definir limites iguais para garantir que as cores representem os mesmos valores}
\NormalTok{escala\_limites }\OtherTok{\textless{}{-}} \FunctionTok{range}\NormalTok{(}\FunctionTok{c}\NormalTok{(mg\_dados}\SpecialCharTok{$}\NormalTok{taxa\_bruta, mg\_dados}\SpecialCharTok{$}\NormalTok{taxa\_suavizada), }\AttributeTok{na.rm =} \ConstantTok{TRUE}\NormalTok{)}


\NormalTok{p1 }\OtherTok{\textless{}{-}} \FunctionTok{ggplot}\NormalTok{(mg\_dados) }\SpecialCharTok{+}
  \FunctionTok{geom\_sf}\NormalTok{(}\FunctionTok{aes}\NormalTok{(}\AttributeTok{fill =}\NormalTok{ taxa\_bruta), }\AttributeTok{color =} \ConstantTok{NA}\NormalTok{) }\SpecialCharTok{+}
  \FunctionTok{scale\_fill\_viridis\_c}\NormalTok{(}\AttributeTok{option =} \StringTok{"turbo"}\NormalTok{, }\AttributeTok{limits =}\NormalTok{ escala\_limites, }\AttributeTok{name =} \StringTok{"Taxa/10k"}\NormalTok{) }\SpecialCharTok{+}
  \FunctionTok{theme\_void}\NormalTok{() }\SpecialCharTok{+}
  \FunctionTok{labs}\NormalTok{(}\AttributeTok{title =} \StringTok{"A. Taxa Bruta (Dados Observados)"}\NormalTok{,}
       \AttributeTok{subtitle =} \StringTok{"Ruído excessivo em municípios pequenos"}\NormalTok{) }\SpecialCharTok{+}
  \FunctionTok{theme}\NormalTok{(}\AttributeTok{plot.title =} \FunctionTok{element\_text}\NormalTok{(}\AttributeTok{size =} \DecValTok{12}\NormalTok{, }\AttributeTok{face =} \StringTok{"bold"}\NormalTok{))}\SpecialCharTok{+}
  
  \FunctionTok{annotation\_scale}\NormalTok{(}
    \AttributeTok{location =} \StringTok{"bl"}\NormalTok{,           }
    \AttributeTok{width\_hint =} \FloatTok{0.3}\NormalTok{,          }
    \AttributeTok{bar\_cols =} \FunctionTok{c}\NormalTok{(}\StringTok{"black"}\NormalTok{, }\StringTok{"white"}\NormalTok{), }
    \AttributeTok{text\_family =} \StringTok{"sans"}       
\NormalTok{  ) }\SpecialCharTok{+}
  
  \FunctionTok{annotation\_north\_arrow}\NormalTok{(}
    \AttributeTok{location =} \StringTok{"tl"}\NormalTok{,           }
    \AttributeTok{which\_north =} \StringTok{"true"}\NormalTok{,      }
    \AttributeTok{pad\_x =} \FunctionTok{unit}\NormalTok{(}\FloatTok{0.2}\NormalTok{, }\StringTok{"in"}\NormalTok{),   }
    \AttributeTok{pad\_y =} \FunctionTok{unit}\NormalTok{(}\FloatTok{0.2}\NormalTok{, }\StringTok{"in"}\NormalTok{),   }
    \AttributeTok{style =}\NormalTok{ north\_arrow\_fancy\_orienteering }
\NormalTok{  )}

\NormalTok{p2 }\OtherTok{\textless{}{-}} \FunctionTok{ggplot}\NormalTok{(mg\_dados) }\SpecialCharTok{+}
  \FunctionTok{geom\_sf}\NormalTok{(}\FunctionTok{aes}\NormalTok{(}\AttributeTok{fill =}\NormalTok{ taxa\_suavizada), }\AttributeTok{color =} \ConstantTok{NA}\NormalTok{) }\SpecialCharTok{+}
  \FunctionTok{scale\_fill\_viridis\_c}\NormalTok{(}\AttributeTok{option =} \StringTok{"turbo"}\NormalTok{, }\AttributeTok{limits =}\NormalTok{ escala\_limites, }\AttributeTok{name =} \StringTok{"Taxa/10k"}\NormalTok{) }\SpecialCharTok{+}
  \FunctionTok{theme\_void}\NormalTok{() }\SpecialCharTok{+}
  \FunctionTok{labs}\NormalTok{(}\AttributeTok{title =} \StringTok{"B. Taxa Suavizada (Empirical Bayes)"}\NormalTok{,}
       \AttributeTok{subtitle =} \StringTok{"Padrão espacial real recuperado"}\NormalTok{) }\SpecialCharTok{+}
  \FunctionTok{theme}\NormalTok{(}\AttributeTok{plot.title =} \FunctionTok{element\_text}\NormalTok{(}\AttributeTok{size =} \DecValTok{12}\NormalTok{, }\AttributeTok{face =} \StringTok{"bold"}\NormalTok{)) }\SpecialCharTok{+}
  
  \FunctionTok{annotation\_scale}\NormalTok{(}
    \AttributeTok{location =} \StringTok{"bl"}\NormalTok{,           }
    \AttributeTok{width\_hint =} \FloatTok{0.3}\NormalTok{,          }
    \AttributeTok{bar\_cols =} \FunctionTok{c}\NormalTok{(}\StringTok{"black"}\NormalTok{, }\StringTok{"white"}\NormalTok{), }
    \AttributeTok{text\_family =} \StringTok{"sans"}       
\NormalTok{  ) }\SpecialCharTok{+}
  
  \FunctionTok{annotation\_north\_arrow}\NormalTok{(}
    \AttributeTok{location =} \StringTok{"tl"}\NormalTok{,           }
    \AttributeTok{which\_north =} \StringTok{"true"}\NormalTok{,      }
    \AttributeTok{pad\_x =} \FunctionTok{unit}\NormalTok{(}\FloatTok{0.2}\NormalTok{, }\StringTok{"in"}\NormalTok{),   }
    \AttributeTok{pad\_y =} \FunctionTok{unit}\NormalTok{(}\FloatTok{0.2}\NormalTok{, }\StringTok{"in"}\NormalTok{),   }
    \AttributeTok{style =}\NormalTok{ north\_arrow\_fancy\_orienteering }
\NormalTok{  )}

\NormalTok{p1 }\SpecialCharTok{+}\NormalTok{ p2}
\end{Highlighting}
\end{Shaded}

\begin{figure}[H]

\centering{

\pandocbounded{\includegraphics[keepaspectratio]{lattice_data_files/figure-pdf/fig-suavizacao-eb-pratica-1.pdf}}

}

\caption{\label{fig-suavizacao-eb-pratica}Comparação: (A) Taxa Bruta
Instável vs.~(B) Taxa Suavizada (Empirical Bayes Local).}

\end{figure}%

\textbf{Interpretação}

A análise comparativa entre as taxas brutas
(Figura~\ref{fig-suavizacao-eb-pratica} A) e as taxas suavizadas por
meio do método Bayesiano Empírico Local
(Figura~\ref{fig-suavizacao-eb-pratica} B) evidencia a eficácia deste
último na correção de distorções estatísticas. O mapa das taxas brutas
(Figura~\ref{fig-suavizacao-eb-pratica} A) apresenta um padrão ruidoso e
fragmentado, com valores extremos (muito altos ou muito baixos)
dispersos aleatoriamente, refletindo a instabilidade das estimativas em
municípios com pequenas populações.

Em contraste, o mapa das taxas suavizadas
(Figura~\ref{fig-suavizacao-eb-pratica} B) mostra com clareza a
estrutura espacial subjacente do fenômeno, destacando um gradiente
norte-sul consistente. Ao incorporar informações da vizinhança, o método
Bayesiano Empírico estabiliza as estimativas, permitindo a identificação
de padrões geográficos reais que estavam obscurecidos pelo ruído nos
dados brutos.

\section{Diagnóstico de dependência
espacial}\label{diagnuxf3stico-de-dependuxeancia-espacial}

A ESDA não se aplica apenas aos dados brutos (\(\mathbf{y}\)), mas é
crucial no diagnóstico de modelos de regressão. Como alertado por Noel
Cressie e Chan (1989), a detecção de autocorrelação espacial em
\(\mathbf{y}\) não implica necessariamente um processo espacial
intrínseco (como contágio); pode ser resultado de heterogeneidade não
observada ou variáveis omitidas que possuem padrão espacial.

O procedimento padrão, detalhado por Anselin (2001), envolve:

\begin{enumerate}
\def\labelenumi{\arabic{enumi}.}
\item
  Ajustar um modelo de regressão (mínimos quadrados ordinários-OLS):
  \(\mathbf{y} = \mathbf{X}\boldsymbol{\beta} + \boldsymbol{\varepsilon}, : \mathbf{y} \| \mathbf{X} \sim N(\mathbf{\mu}, \Sigma)\).
\item
  Calcular os resíduos
  \(\hat{\boldsymbol{\varepsilon}} = \mathbf{y} - \mathbf{X}\hat{\boldsymbol{\beta}}\).
\item
  Aplicar o teste \(I\) de Moran sobre os resíduos
  \(\hat{\boldsymbol{\varepsilon}}\) utilizando uma matriz
  \(\mathbf{W}\) escolhida.
\end{enumerate}

Se \(I\) for significativo, o pressuposto de independência dos erros é
violado. No entanto, Zhang e Yu (2018) demonstram que a validade deste
diagnóstico depende criticamente da escolha correta de \(\mathbf{W}\). O
uso de uma matriz incorreta pode falhar em detetar a dependência
residual ou indicar falsamente a necessidade de um modelo espacial.

\begin{Shaded}
\begin{Highlighting}[]
\ControlFlowTok{if}\NormalTok{ (}\SpecialCharTok{!}\FunctionTok{require}\NormalTok{(}\StringTok{"pacman"}\NormalTok{)) }\FunctionTok{install.packages}\NormalTok{(}\StringTok{"pacman"}\NormalTok{)}
\NormalTok{pacman}\SpecialCharTok{::}\FunctionTok{p\_load}\NormalTok{(sf, spdep, ggplot2, patchwork, viridis)}


\ControlFlowTok{if}\NormalTok{ (}\SpecialCharTok{!}\FunctionTok{exists}\NormalTok{(}\StringTok{"mg\_dados"}\NormalTok{)) \{}
\NormalTok{  mg\_dados }\OtherTok{\textless{}{-}} \FunctionTok{read\_municipality}\NormalTok{(}\AttributeTok{code\_muni =} \StringTok{"MG"}\NormalTok{, }\AttributeTok{year =} \DecValTok{2020}\NormalTok{, }\AttributeTok{showProgress =} \ConstantTok{FALSE}\NormalTok{)}
\NormalTok{  coords }\OtherTok{\textless{}{-}} \FunctionTok{st\_coordinates}\NormalTok{(}\FunctionTok{st\_centroid}\NormalTok{(mg\_dados))}
\NormalTok{  mg\_dados}\SpecialCharTok{$}\NormalTok{taxa\_bruta }\OtherTok{\textless{}{-}}\NormalTok{ (}\SpecialCharTok{{-}}\NormalTok{coords[,}\DecValTok{2}\NormalTok{] }\SpecialCharTok{*} \DecValTok{10}\NormalTok{) }\SpecialCharTok{+} \FunctionTok{rnorm}\NormalTok{(}\FunctionTok{nrow}\NormalTok{(mg\_dados), }\DecValTok{0}\NormalTok{, }\DecValTok{5}\NormalTok{)}
\NormalTok{\}}

\CommentTok{\# Criar uma variável explicativa X aleatória (sem padrão espacial)}
\FunctionTok{set.seed}\NormalTok{(}\DecValTok{123}\NormalTok{)}
\NormalTok{mg\_dados}\SpecialCharTok{$}\NormalTok{variavel\_x }\OtherTok{\textless{}{-}} \FunctionTok{rnorm}\NormalTok{(}\FunctionTok{nrow}\NormalTok{(mg\_dados))}

\CommentTok{\#Ajuste do Modelo de Regressão Linear (OLS)}
\NormalTok{modelo\_ols }\OtherTok{\textless{}{-}} \FunctionTok{lm}\NormalTok{(taxa\_bruta }\SpecialCharTok{\textasciitilde{}}\NormalTok{ variavel\_x, }\AttributeTok{data =}\NormalTok{ mg\_dados)}

\CommentTok{\# Extrair os resíduos do modelo}
\NormalTok{mg\_dados}\SpecialCharTok{$}\NormalTok{residuos }\OtherTok{\textless{}{-}} \FunctionTok{residuals}\NormalTok{(modelo\_ols)}

\CommentTok{\#Teste I de Moran nos Resíduos}
\CommentTok{\# Necessário recriar a lista de pesos (W)}
\NormalTok{nb }\OtherTok{\textless{}{-}} \FunctionTok{poly2nb}\NormalTok{(mg\_dados, }\AttributeTok{queen =} \ConstantTok{TRUE}\NormalTok{)}
\NormalTok{lw }\OtherTok{\textless{}{-}} \FunctionTok{nb2listw}\NormalTok{(nb, }\AttributeTok{style =} \StringTok{"W"}\NormalTok{, }\AttributeTok{zero.policy =} \ConstantTok{TRUE}\NormalTok{)}

\CommentTok{\# Função para resíduos de regressão (lm.morantest)}
\NormalTok{moran\_residuos }\OtherTok{\textless{}{-}} \FunctionTok{lm.morantest}\NormalTok{(modelo\_ols, lw, }\AttributeTok{alternative =} \StringTok{"two.sided"}\NormalTok{)}

\FunctionTok{print}\NormalTok{(moran\_residuos)}
\end{Highlighting}
\end{Shaded}

\begin{verbatim}

    Global Moran I for regression residuals

data:  
model: lm(formula = taxa_bruta ~ variavel_x, data = mg_dados)
weights: lw

Moran I statistic standard deviate = 9.3389, p-value < 2.2e-16
alternative hypothesis: two.sided
sample estimates:
Observed Moran I      Expectation         Variance 
    0.1977385849    -0.0011812437     0.0004536919 
\end{verbatim}

\begin{Shaded}
\begin{Highlighting}[]
\NormalTok{p1 }\OtherTok{\textless{}{-}} \FunctionTok{ggplot}\NormalTok{(mg\_dados) }\SpecialCharTok{+}
  \FunctionTok{geom\_sf}\NormalTok{(}\FunctionTok{aes}\NormalTok{(}\AttributeTok{fill =}\NormalTok{ residuos), }\AttributeTok{color =} \ConstantTok{NA}\NormalTok{) }\SpecialCharTok{+}
  \FunctionTok{scale\_fill\_distiller}\NormalTok{(}\AttributeTok{palette =} \StringTok{"RdBu"}\NormalTok{, }\AttributeTok{name =} \StringTok{"Resíduos"}\NormalTok{) }\SpecialCharTok{+}
  \FunctionTok{labs}\NormalTok{(}\AttributeTok{title =} \StringTok{"A. Mapa dos Resíduos OLS"}\NormalTok{, }
       \AttributeTok{subtitle =} \StringTok{"Padrão visível (não aleatório)"}\NormalTok{) }\SpecialCharTok{+}
  
  \FunctionTok{annotation\_scale}\NormalTok{(}
    \AttributeTok{location =} \StringTok{"bl"}\NormalTok{,           }
    \AttributeTok{width\_hint =} \FloatTok{0.3}\NormalTok{,          }
    \AttributeTok{bar\_cols =} \FunctionTok{c}\NormalTok{(}\StringTok{"black"}\NormalTok{, }\StringTok{"white"}\NormalTok{), }
    \AttributeTok{text\_family =} \StringTok{"sans"}       
\NormalTok{  ) }\SpecialCharTok{+}
  
  \FunctionTok{annotation\_north\_arrow}\NormalTok{(}
    \AttributeTok{location =} \StringTok{"tl"}\NormalTok{,           }
    \AttributeTok{which\_north =} \StringTok{"true"}\NormalTok{,      }
    \AttributeTok{pad\_x =} \FunctionTok{unit}\NormalTok{(}\FloatTok{0.2}\NormalTok{, }\StringTok{"in"}\NormalTok{),   }
    \AttributeTok{pad\_y =} \FunctionTok{unit}\NormalTok{(}\FloatTok{0.2}\NormalTok{, }\StringTok{"in"}\NormalTok{),   }
    \AttributeTok{style =}\NormalTok{ north\_arrow\_fancy\_orienteering }
\NormalTok{  )}\SpecialCharTok{+}
  \FunctionTok{theme\_void}\NormalTok{()}

\CommentTok{\#}
\NormalTok{mg\_dados}\SpecialCharTok{$}\NormalTok{residuos\_z }\OtherTok{\textless{}{-}} \FunctionTok{scale}\NormalTok{(mg\_dados}\SpecialCharTok{$}\NormalTok{residuos)}
\NormalTok{mg\_dados}\SpecialCharTok{$}\NormalTok{lag\_residuos }\OtherTok{\textless{}{-}} \FunctionTok{lag.listw}\NormalTok{(lw, mg\_dados}\SpecialCharTok{$}\NormalTok{residuos\_z)}

\NormalTok{p2 }\OtherTok{\textless{}{-}} \FunctionTok{ggplot}\NormalTok{(mg\_dados, }\FunctionTok{aes}\NormalTok{(}\AttributeTok{x =}\NormalTok{ residuos\_z, }\AttributeTok{y =}\NormalTok{ lag\_residuos)) }\SpecialCharTok{+}
  \FunctionTok{geom\_point}\NormalTok{(}\AttributeTok{alpha =} \FloatTok{0.4}\NormalTok{) }\SpecialCharTok{+}
  \FunctionTok{geom\_smooth}\NormalTok{(}\AttributeTok{method =} \StringTok{"lm"}\NormalTok{, }\AttributeTok{se =} \ConstantTok{FALSE}\NormalTok{, }\AttributeTok{color =} \StringTok{"red"}\NormalTok{) }\SpecialCharTok{+}
  \FunctionTok{geom\_hline}\NormalTok{(}\AttributeTok{yintercept =} \DecValTok{0}\NormalTok{, }\AttributeTok{linetype=}\StringTok{"dashed"}\NormalTok{) }\SpecialCharTok{+}
  \FunctionTok{geom\_vline}\NormalTok{(}\AttributeTok{xintercept =} \DecValTok{0}\NormalTok{, }\AttributeTok{linetype=}\StringTok{"dashed"}\NormalTok{) }\SpecialCharTok{+}
  \FunctionTok{labs}\NormalTok{(}\AttributeTok{title =} \StringTok{"B. Moran dos Resíduos"}\NormalTok{,}
       \AttributeTok{subtitle =} \FunctionTok{paste}\NormalTok{(}\StringTok{"I de Moran ="}\NormalTok{, }\FunctionTok{round}\NormalTok{(moran\_residuos}\SpecialCharTok{$}\NormalTok{statistic, }\DecValTok{3}\NormalTok{)),}
       \AttributeTok{x =} \StringTok{"Resíduos (Z)"}\NormalTok{, }\AttributeTok{y =} \StringTok{"Lag Espacial dos Resíduos"}\NormalTok{) }\SpecialCharTok{+}
  \FunctionTok{theme\_bw}\NormalTok{()}

\NormalTok{p1 }\SpecialCharTok{+}\NormalTok{ p2}
\end{Highlighting}
\end{Shaded}

\begin{figure}[H]

\centering{

\pandocbounded{\includegraphics[keepaspectratio]{lattice_data_files/figure-pdf/fig-diagnostico-residuos-1.pdf}}

}

\caption{\label{fig-diagnostico-residuos}Diagnóstico de Resíduos: (A)
Mapa dos Resíduos e (B) Diagrama de Moran dos Resíduos.}

\end{figure}%

\textbf{Interpretação}

A análise dos resíduos do modelo de regressão linear clássico aponta
para a violação do pressuposto de independência dos resíduos. A
estatística I de Moran observada nos resíduos foi de aproximadamente
0,198, um valor substancialmente superior à esperança matemática sob
aleatoriedade (-0,001). A magnitude dessa dependência é confirmada pelo
desvio padrão padronizado de 9,34, resultando em um valor-p baixo
(\textless{} 2.2e-16), o que nos leva a rejeitar a hipótese nula de
ausência de autocorrelação espacial com alto grau de confiança.

Visualmente, essa dependência é observada na
Figura~\ref{fig-diagnostico-residuos}.
Figura~\ref{fig-diagnostico-residuos} A), a distribuição espacial dos
resíduos não se assemelha a um ruído branco aleatório; ao contrário,
observam-se nítidos aglomerados de resíduos positivos (em vermelho) e
negativos (em azul), indicando que o modelo subestima ou superestima
sistematicamente os valores em regiões vizinhas.
Figura~\ref{fig-diagnostico-residuos} B) reforça esse diagnóstico
através da inclinação positiva da reta de regressão entre os resíduos e
sua defasagem espacial. Essa estrutura remanescente nos erros sugere que
a variável explicativa aleatória introduzida foi incapaz de capturar o
padrão espacial da variável resposta, transferindo essa estrutura não
modelada para o termo de erro.

Como próximo passo, é o pesquisador abandonaria o estimador de Mínimos
Quadrados Ordinários, cujos testes de significância tornaram-se
inválidos, e adotar modelos de regressão espacial que incorporem
explicitamente a estrutura de dependência na matriz de covariância ou na
média.

\textbf{Limitações da ESDA}

\begin{enumerate}
\def\labelenumi{\arabic{enumi}.}
\item
  Os valores das estatísticas de Moran ou Geary podem mudar
  drasticamente com a alteração da escala de agregação ou do desenho das
  zonas. Um cluster identificado a nível municipal pode desaparecer a
  nível estadual (MAUP).
\item
  O cálculo de estatísticas locais (LISA) envolve a realização de \(n\)
  testes de hipótese simultâneos. Sem correção (como Bonferroni ou
  \href{https://en.wikipedia.org/wiki/False_discovery_rate}{False
  Discovery Rate}), a probabilidade de encontrar clusters falsos
  positivos (erro Tipo I) aumenta com o tamanho da amostra.
\item
  A descoberta de estrutura espacial nos resíduos via ESDA pode levar à
  inclusão de termos espaciais (CAR/SAR) que competem com as covariáveis
  fixas pela explicação da variância (colinearidade espacial),
  enviesando a inferência sobre os efeitos causais.
\item
  Unidades na periferia do mapa têm menos vizinhos (efeito da borda), o
  que distorce o cálculo dos momentos das estatísticas locais e globais,
  exigindo correções ou simulações de Monte Carlo para inferência válida
  (Julian Besag 1975).
\end{enumerate}

\section{Fundamentos probabilísticos: GMRF, matrizes de precisão e
operadores Laplacianos discretos}\label{sec-gmrf}

Esta seção estabelece a ponte formal entre a teoria dos processos
estocásticos contínuos e a implementação computacional eficiente em
dados de área, fundamentando-se na estrutura probabilística dos Campos
Aleatórios de Markov.

\textbf{Campos Aleatórios Gaussianos: GRF vs.~GMRF}

Um campo aleatório Gaussiano (GRF -
\href{https://en.wikipedia.org/wiki/Gaussian_random_field}{Gaussian
Random Field}) é um processo estocástico
\(\{Y(\mathbf{s}): \mathbf{s} \in D\}\) tal que, para qualquer conjunto
finito de localizações \(\{\mathbf{s}_1, \dots, \mathbf{s}_n\}\), o
vetor \((Y(\mathbf{s}_1), \dots, Y(\mathbf{s}_n))^\top\) segue uma
distribuição normal multivariada (Noel Cressie e Moores 2022). A
distinção fundamental na modelagem espacial reside em como essa
estrutura de dependência é parametrizada:

\begin{enumerate}
\def\labelenumi{\arabic{enumi}.}
\item
  GRF Padrão (Geoestatística): Especificado por sua função de média
  \(\mu(\mathbf{s})\) e sua função de covariância
  \(C(\mathbf{s}_i, \mathbf{s}_j) = \text{Cov}(Y(\mathbf{s}_i), Y(\mathbf{s}_j))\).
  Isto gera uma matriz de covariância \(\mathbf{\Sigma}\) densa, onde
  cada par de locais possui uma correlação teoricamente não nula,
  definida por uma função de decaimento (ex.: exponencial, Matérn). O
  custo computacional para inferência (como cálculo da verossimilhança)
  é \(\mathcal{O}(n^3)\) devido à necessidade de fatorar uma matriz
  \(n \times n\) densa.
\item
  Campo aleatório de Markov Gaussiano (GMRF): É um GRF definido sobre um
  domínio discreto \(D^L\) (um grafo ou \emph{lattice}), especificado
  diretamente pela sua matriz de precisão
  \(\mathbf{Q} = \mathbf{\Sigma}^{-1}\). A propriedade que define um
  GMRF é a independência condicional espacial (Håvard Rue e Held 2005;
  Håvard Rue, Martino, e Chopin 2009):
\end{enumerate}

\[
Y_i \perp Y_j \mid \mathbf{Y}_{-ij} \iff Q_{ij} = 0 \quad \text{para} \quad i \neq j.
\]

Ou seja, dadas as observações de todos os outros locais, \(Y_i\) e
\(Y_j\) são independentes se e somente se não forem vizinhos no grafo
(ou se a estrutura de dependência direta for modelada como nula). Esta é
a formalização probabilística da propriedade de Markov: o valor em um
local depende apenas dos valores em seus vizinhos diretos.

\textbf{Propriedade de Markov em grafos e esparsidade da matriz de
precisão}

A propriedade de Markov espacial formaliza a noção intuitiva de
vizinhança. Seja \(\mathcal{G} = (\mathcal{V}, \mathcal{E})\) o grafo
que representa o domínio espacial discreto, onde
\(\mathcal{V}=\{1, \dots, n\}\) são os vértices (unidades de área) e
\(\mathcal{E}\) são as arestas que definem vizinhanças. Um vetor
aleatório \(\mathbf{Y} = (Y_1, \dots, Y_n)^\top\) é um GMRF em relação a
\(\mathcal{G}\) se:

\[
p(Y_i \mid \mathbf{Y}_{-i}) = p(Y_i \mid \mathbf{Y}_{\partial i}),
\]

onde \(\partial i = \{j: (i, j) \in \mathcal{E}\}\) é o conjunto de
vizinhos de \(i\). Esta propriedade local é equivalente à esparsidade da
matriz de precisão \(\mathbf{Q}\): \(Q_{ij} = 0\) para todo par
\((i, j)\) tal que \(j \notin \partial i\) e \(j \neq i\).

Exemplo: Considere 5 regiões onde cada região é vizinha apenas da
anterior e da seguinte (assumindo que as regiões estão em fila única). A
matriz de precisão teria a forma:

\[
\mathbf{Q} =
\begin{bmatrix}
* & * & 0 & 0 & 0 \\
* & * & * & 0 & 0 \\
0 & * & * & * & 0 \\
0 & 0 & * & * & * \\
0 & 0 & 0 & * & *
\end{bmatrix},
\]

onde \texttt{*} denota um elemento não nulo. Para dados de área
bidimensionais com vizinhança por contiguidade, cada região tem
tipicamente entre 4 e 8 vizinhos, resultando em uma matriz
\(\mathbf{Q}\) com \(\mathcal{O}(n)\) elementos não nulos, em contraste
com os \(\mathcal{O}(n^2)\) de uma matriz densa. Esta esparsidade
permite o uso de algoritmos numéricos de álgebra linear esparsa (como a
fatoração de Cholesky esparsa), reduzindo o custo computacional da
inferência de \(\mathcal{O}(n^3)\) para aproximadamente
\(\mathcal{O}(n^{3/2})\) em domínios 2D, viabilizando métodos como a
inferência Bayesiana aproximada via \href{https://www.r-inla.org/}{INLA}
(Håvard Rue, Martino, e Chopin 2009).

\textbf{O Laplaciano do grafo}

A estrutura da matriz de precisão \(\mathbf{Q}\) em modelos espaciais
está intrinsecamente ligada ao conceito de Laplaciano discreto. O
operador Laplaciano contínuo \(\nabla^2\) mede a divergência do
gradiente, ou a curvatura local de uma função. Num grafo, seu análogo
mede a diferença entre o valor num nó e a média dos valores dos seus
vizinhos, atuando como um quantificador de suavidade ou rugosidade local
do campo.

A matriz Laplaciana \(\mathbf{L}\) de um grafo não direcionado é
definida como \(\mathbf{L} = \mathbf{D} - \mathbf{W},\) onde
\(\mathbf{W}=[w_{ij}]_{n \times n}\) é a matriz de adjacência/vizinhança
(com \(w_{ij}=1\) se \(i\) e \(j\) são vizinhos) e \(\mathbf{D}\) é a
matriz diagonal dos graus (com \(D_{ii} = \sum_j w_{ij}\), o número de
vizinhos de \(i\)). Para um vetor
\(\mathbf{y} = (y_1, \dots, y_n)^\top\), a forma quadrática associada ao
Laplaciano é:

\[\mathbf{y}^\top \mathbf{L} \mathbf{y} = \sum_{(i,j) \in \mathcal{E}} (y_i - y_j)^2.\]

Esta equação demonstra que \(\mathbf{y}^\top \mathbf{L} \mathbf{y}\) é
uma soma de diferenças quadráticas entre todos os pares de vizinhos. Um
valor baixo indica que \(\mathbf{y}\) é suave sobre o grafo (vizinhos
têm valores similares), enquanto um valor alto indica um campo rugoso ou
heterogêneo.

\textbf{Propriedades espectrais e singularidade}

As propriedades espectrais (autovalores e autovetores) de \(\mathbf{L}\)
revelam características fundamentais da conectividade do sistema.

\begin{itemize}
\item
  \(\mathbf{L}\) é sempre semidefinida positiva, ou seja,
  \(\mathbf{y}^\top \mathbf{L} \mathbf{y} \geq 0\) para todo
  \(\mathbf{y}\).
\item
  O menor autovalor de \(\mathbf{L}\) é sempre \(\lambda_1 = 0\), e seu
  autovetor correspondente é o vetor constante
  \(\mathbf{1} = (1, \dots, 1)^\top\). Isto decorre diretamente do fato
  de que \(\mathbf{L}\mathbf{1} = \mathbf{0}\) (a soma de cada linha é
  zero).
\item
  A existência do autovalor zero implica que \(\mathbf{L}\) é uma matriz
  singular (não invertível). Estatisticamente, isso significa que uma
  distribuição com matriz de precisão proporcional a \(\mathbf{L}\),
  como \(\mathbf{Y} \sim \mathcal{N}(\mathbf{0}, \mathbf{L}^-)\) (onde
  \(\mathbf{L}^-\) denota uma inversa generalizada), é imprópria. Ela
  define uma densidade de probabilidade válida apenas no subespaço
  ortogonal ao vetor constante (ou seja, para contrastes entre os
  \(y_i\)), pois a variância na direção do nível médio global é
  infinita.
\end{itemize}

\section{Modelos mais comuns em dados de área: CAR, ICAR, SAR e
BYM}\label{sec-modelos_classicos}

A modelagem de dados de área (ou \emph{lattice data}) fundamenta-se na
incorporação explícita da estrutura de vizinhança definida pela matriz
de pesos espaciais \(\mathbf{W}=[w_{ij}]_{n \times n}\) (ver
Seção~\ref{sec-lattice}) no mecanismo gerador de dados. A literatura
distingue duas formas canônicas de especificar esta dependência: a
especificação condicional (CAR), que modela a distribuição de uma área
dados os seus vizinhos, e a especificação simultânea (SAR), que modela o
sistema de equações de feedback instantâneo entre todas as áreas {[}Noel
Cressie (1993){]}.

\subsection{Modelo Condicional Autorregressivo (CAR)}\label{sec-car}

Introduzido por Julian Besag (1974), o modelo autorregressivo
condicional (CAR - \emph{Conditional Autoregressive}) especifica cada
observação \(Y_i\) como uma função linear dos valores de seus vizinhos
mais um termo de erro independente, mas a inferência é baseada na
distribuição condicional.

Seja \(Y_i\) a variável aleatória na unidade \(i\) e
\(\mathbf{Y}_{-i} = \{Y_j : j \neq i\}\) o conjunto de todas as outras
observações excluindo \(i\). O modelo CAR é definido por uma família de
distribuições condicionais Gaussianas:

\begin{equation}\phantomsection\label{eq-car}{
Y_i = \mu_i  + \sum_{j \neq i} w_{ij}(Y_j - \mu_j) + \epsilon_i,
}\end{equation}

onde \(\mu_i\) é uma tendência determinística (tendência, geralmente
\(\mathbf{x}_i^\top \boldsymbol{\beta}\)); \(w_{ij}\) são os pesos
espaciais normalizados, com \(w_{ij} \neq 0\) apenas se \(j\) é vizinho
de \(i\) e, \(\epsilon_i \sim \mathcal{N}(0, \sigma_i^2)\) são erros
independentes.

Aqui, o valor esperado em \(i\), condicional aos seus vizinhos, é a
média global ajustada por uma média ponderada dos desvios dos seus
vizinhos em relação à média global.

A partir Eq.~\ref{eq-car}, deriva-se a distribuição condicional que
caracteriza o CAR:

\[
Y_i \mid \mathbf{Y}_{-i} \sim \mathcal{N}\left( \mu_i + \sum_{j \neq i} w_{ij}(Y_j - \mu_j),\, \sigma_i^2 \right), \: \: \mathbb{E} [Y_i \mid \mathbf{Y}_{-i}]=\mu_i + \sum_{j \neq i} w_{ij}(Y_j - \mu_j), \:\: \text{Var}[Y_i \mid \mathbf{Y}_{-i}] =\sigma^2_i
\]

Para que estas condicionais definam uma distribuição conjunta válida
\(\mathbf{Y} \sim \mathcal{N}(\boldsymbol{\mu}, \mathbf{\sigma})\), o
\href{https://en.wikipedia.org/wiki/Hammersley\%E2\%80\%93Clifford_theorem}{Teorema
de Hammersley-Clifford} impõe condições de simetria. Julian Besag (1974)
mostrou que a matriz de precisão conjunta
\(\Sigma^-1=\mathbf{Q} = [Q_{ij}]_{n \times n}\) deve ser simétrica e
positiva definida. A distribuição conjunta é dada por:

\[
\mathbf{Y} \sim \mathcal{N}_n(\boldsymbol{\mu}, \mathbf{Q}_{CAR}), \quad \text{onde} \quad \mathbf{Q}_{CAR} = \mathbf{M}^{-1} (\mathbf{I} - \rho \mathbf{W})\: \text{ se a inversa existir }
\]

Aqui, \(\Sigma= \mathbf{M}(\mathbf{I} - \rho \mathbf{W})^{-1}\),
\(\mathbf{M}_{n \times n} =\text{diag}(\sigma_1^2, \dots, \sigma_n^2), \: \boldsymbol{\mu} = [\mu_1, \mu_2, \ldots, \mu_n]^\top\)
e \(\rho\) parâmetro espacial desconhecido. Para garantir a simetria de
\(\mathbf{Q}_{CAR}\), é necessário que
\(\frac{w_{ij}}{\sigma_i^2} = \frac{w_{ji}}{\sigma_j^2}\). Note ainda
que
\(w_{ii}=0, \, Q_{ii} = 1/\sigma_i^2 >0, \: Q_{ij}= -\rho w_{ij}/\sigma_i^2\: i \neq j\).
Lembre-se que quando \(w_{ij} \neq 0\), pode-se escrever \(i \sim j\)
(Julian Besag e Kooperberg 1995). Note que sem perda de generalidade, em
várias literaturas tem se assumido que \(\mu=\mu_i=\mu_j=0\), removendo
o efeito da tendência global.

\begin{table}

\caption{\label{tbl-modelo_car_comparacao}\pandocbounded{\includegraphics[keepaspectratio]{lattice_data_files/figure-pdf/tbl-modelo_car_comparacao-1.pdf}}}

\centering{

\begin{Shaded}
\begin{Highlighting}[]
\ControlFlowTok{if}\NormalTok{ (}\SpecialCharTok{!}\FunctionTok{require}\NormalTok{(}\StringTok{"pacman"}\NormalTok{)) }\FunctionTok{install.packages}\NormalTok{(}\StringTok{"pacman"}\NormalTok{)}
\NormalTok{pacman}\SpecialCharTok{::}\FunctionTok{p\_load}\NormalTok{(spatialreg, spdep, modelsummary,knitr, kableExtra, texreg,ggplot2, patchwork, sf)}

\CommentTok{\#Preparação dos Dados}
\ControlFlowTok{if}\NormalTok{ (}\SpecialCharTok{!}\FunctionTok{exists}\NormalTok{(}\StringTok{"mg\_dados"}\NormalTok{)) \{}
\NormalTok{  mg\_dados }\OtherTok{\textless{}{-}}\NormalTok{ geobr}\SpecialCharTok{::}\FunctionTok{read\_municipality}\NormalTok{(}\AttributeTok{code\_muni =} \StringTok{"MG"}\NormalTok{, }\AttributeTok{year =} \DecValTok{2020}\NormalTok{, }\AttributeTok{showProgress =} \ConstantTok{FALSE}\NormalTok{)}
\NormalTok{  coords }\OtherTok{\textless{}{-}} \FunctionTok{st\_coordinates}\NormalTok{(}\FunctionTok{st\_centroid}\NormalTok{(mg\_dados))}
  \FunctionTok{set.seed}\NormalTok{(}\DecValTok{123}\NormalTok{)}
\NormalTok{  mg\_dados}\SpecialCharTok{$}\NormalTok{taxa\_bruta }\OtherTok{\textless{}{-}}\NormalTok{ (}\SpecialCharTok{{-}}\NormalTok{coords[,}\DecValTok{2}\NormalTok{] }\SpecialCharTok{*} \DecValTok{10}\NormalTok{) }\SpecialCharTok{+} \FunctionTok{rnorm}\NormalTok{(}\FunctionTok{nrow}\NormalTok{(mg\_dados), }\DecValTok{0}\NormalTok{, }\DecValTok{5}\NormalTok{)}
\NormalTok{  mg\_dados}\SpecialCharTok{$}\NormalTok{variavel\_x }\OtherTok{\textless{}{-}} \FunctionTok{rnorm}\NormalTok{(}\FunctionTok{nrow}\NormalTok{(mg\_dados))}
\NormalTok{\}}

\CommentTok{\# Matriz de Pesos Espaciais}
\NormalTok{nb }\OtherTok{\textless{}{-}} \FunctionTok{poly2nb}\NormalTok{(mg\_dados, }\AttributeTok{queen =} \ConstantTok{TRUE}\NormalTok{)}
\NormalTok{lw }\OtherTok{\textless{}{-}} \FunctionTok{nb2listw}\NormalTok{(nb, }\AttributeTok{style =} \StringTok{"W"}\NormalTok{, }\AttributeTok{zero.policy =} \ConstantTok{TRUE}\NormalTok{)}

\CommentTok{\#Ajuste dos Modelos}
\CommentTok{\# Regressão linear classica }
\NormalTok{mod\_ols }\OtherTok{\textless{}{-}} \FunctionTok{lm}\NormalTok{(taxa\_bruta }\SpecialCharTok{\textasciitilde{}}\NormalTok{ variavel\_x, }\AttributeTok{data =}\NormalTok{ mg\_dados)}

\CommentTok{\# Modelo CAR (Incorpora dependência espacial condicional)}
\CommentTok{\# family = "CAR" ajusta via Máxima Verossimilhança}
\NormalTok{mod\_car }\OtherTok{\textless{}{-}} \FunctionTok{spautolm}\NormalTok{(taxa\_bruta }\SpecialCharTok{\textasciitilde{}}\NormalTok{ variavel\_x, }
                    \AttributeTok{data =}\NormalTok{ mg\_dados, }
                    \AttributeTok{listw =}\NormalTok{ lw, }
                    \AttributeTok{family =} \StringTok{"CAR"}\NormalTok{)}

\CommentTok{\# Função auxiliar para formatar valor}
\NormalTok{format\_coef }\OtherTok{\textless{}{-}} \ControlFlowTok{function}\NormalTok{(est, se, pval) \{}
\NormalTok{  stars }\OtherTok{\textless{}{-}} \FunctionTok{case\_when}\NormalTok{(pval }\SpecialCharTok{\textless{}} \FloatTok{0.001} \SpecialCharTok{\textasciitilde{}} \StringTok{"***"}\NormalTok{, pval }\SpecialCharTok{\textless{}} \FloatTok{0.01} \SpecialCharTok{\textasciitilde{}} \StringTok{"**"}\NormalTok{, pval }\SpecialCharTok{\textless{}} \FloatTok{0.05} \SpecialCharTok{\textasciitilde{}} \StringTok{"*"}\NormalTok{, }\ConstantTok{TRUE} \SpecialCharTok{\textasciitilde{}} \StringTok{""}\NormalTok{)}
  \FunctionTok{paste0}\NormalTok{(}\FunctionTok{format}\NormalTok{(}\FunctionTok{round}\NormalTok{(est, }\DecValTok{3}\NormalTok{), }\AttributeTok{nsmall=}\DecValTok{3}\NormalTok{), }\StringTok{" ("}\NormalTok{, }\FunctionTok{format}\NormalTok{(}\FunctionTok{round}\NormalTok{(se, }\DecValTok{3}\NormalTok{), }\AttributeTok{nsmall=}\DecValTok{3}\NormalTok{), }\StringTok{")"}\NormalTok{, stars)}
\NormalTok{\}}


\NormalTok{sum\_ols }\OtherTok{\textless{}{-}} \FunctionTok{summary}\NormalTok{(mod\_ols)}
\NormalTok{coef\_ols }\OtherTok{\textless{}{-}}\NormalTok{ sum\_ols}\SpecialCharTok{$}\NormalTok{coefficients}
\NormalTok{res\_ols }\OtherTok{\textless{}{-}} \FunctionTok{c}\NormalTok{(}
  \FunctionTok{format\_coef}\NormalTok{(coef\_ols[}\DecValTok{1}\NormalTok{,}\DecValTok{1}\NormalTok{], coef\_ols[}\DecValTok{1}\NormalTok{,}\DecValTok{2}\NormalTok{], coef\_ols[}\DecValTok{1}\NormalTok{,}\DecValTok{4}\NormalTok{]), }\CommentTok{\# Intercepto}
  \FunctionTok{format\_coef}\NormalTok{(coef\_ols[}\DecValTok{2}\NormalTok{,}\DecValTok{1}\NormalTok{], coef\_ols[}\DecValTok{2}\NormalTok{,}\DecValTok{2}\NormalTok{], coef\_ols[}\DecValTok{2}\NormalTok{,}\DecValTok{4}\NormalTok{]), }\CommentTok{\# Variavel X}
  \StringTok{"{-}"}\NormalTok{,                                                      }\CommentTok{\# Lambda (Não existe no OLS)}
  \FunctionTok{round}\NormalTok{(}\FunctionTok{AIC}\NormalTok{(mod\_ols), }\DecValTok{1}\NormalTok{)                                    }\CommentTok{\# AIC}
\NormalTok{)}


\NormalTok{sum\_car }\OtherTok{\textless{}{-}} \FunctionTok{summary}\NormalTok{(mod\_car)}
\NormalTok{coef\_car }\OtherTok{\textless{}{-}}\NormalTok{ sum\_car}\SpecialCharTok{$}\NormalTok{Coef}
\CommentTok{\# Lambda (parametro espacial) e seu SE}
\NormalTok{lambda\_val }\OtherTok{\textless{}{-}}\NormalTok{ mod\_car}\SpecialCharTok{$}\NormalTok{lambda}
\NormalTok{lambda\_se  }\OtherTok{\textless{}{-}}\NormalTok{ mod\_car}\SpecialCharTok{$}\NormalTok{lambda.se}
\CommentTok{\# Teste Z para o Lambda (aproximado)}
\NormalTok{lambda\_p   }\OtherTok{\textless{}{-}} \DecValTok{2} \SpecialCharTok{*}\NormalTok{ (}\DecValTok{1} \SpecialCharTok{{-}} \FunctionTok{pnorm}\NormalTok{(}\FunctionTok{abs}\NormalTok{(lambda\_val }\SpecialCharTok{/}\NormalTok{ lambda\_se)))}

\NormalTok{res\_car }\OtherTok{\textless{}{-}} \FunctionTok{c}\NormalTok{(}
  \FunctionTok{format\_coef}\NormalTok{(coef\_car[}\DecValTok{1}\NormalTok{,}\DecValTok{1}\NormalTok{], coef\_car[}\DecValTok{1}\NormalTok{,}\DecValTok{2}\NormalTok{], coef\_car[}\DecValTok{1}\NormalTok{,}\DecValTok{4}\NormalTok{]), }\CommentTok{\# Intercepto}
  \FunctionTok{format\_coef}\NormalTok{(coef\_car[}\DecValTok{2}\NormalTok{,}\DecValTok{1}\NormalTok{], coef\_car[}\DecValTok{2}\NormalTok{,}\DecValTok{2}\NormalTok{], coef\_car[}\DecValTok{2}\NormalTok{,}\DecValTok{4}\NormalTok{]), }\CommentTok{\# Variavel X}
  \FunctionTok{format\_coef}\NormalTok{(lambda\_val, lambda\_se, lambda\_p),             }\CommentTok{\# Lambda}
  \FunctionTok{round}\NormalTok{(}\FunctionTok{AIC}\NormalTok{(mod\_car), }\DecValTok{1}\NormalTok{)                                    }\CommentTok{\# AIC}
\NormalTok{)}

\CommentTok{\#}
\NormalTok{tabela\_final }\OtherTok{\textless{}{-}} \FunctionTok{data.frame}\NormalTok{(}
  \AttributeTok{Parametro =} \FunctionTok{c}\NormalTok{(}\StringTok{"Intercepto"}\NormalTok{, }\StringTok{"Variável X"}\NormalTok{, }\StringTok{"Lambda (Espacial)"}\NormalTok{, }\StringTok{"AIC"}\NormalTok{),}
  \AttributeTok{OLS =}\NormalTok{ res\_ols,}
  \AttributeTok{CAR =}\NormalTok{ res\_car}
\NormalTok{)}

\CommentTok{\#Gerar Tabela Bonita (HTML/LaTeX)}
\FunctionTok{kbl}\NormalTok{(tabela\_final, }
    \AttributeTok{format =} \StringTok{"latex"}\NormalTok{, }
    \AttributeTok{booktabs =} \ConstantTok{TRUE}\NormalTok{, }
    \AttributeTok{align =} \StringTok{"lcc"}\NormalTok{, }
    \AttributeTok{caption =} \ConstantTok{NULL}\NormalTok{) }\SpecialCharTok{\%\textgreater{}\%} 
  \FunctionTok{kable\_styling}\NormalTok{(}\AttributeTok{latex\_options =} \FunctionTok{c}\NormalTok{(}\StringTok{"HOLD\_position"}\NormalTok{), }
                \AttributeTok{full\_width =} \ConstantTok{FALSE}\NormalTok{, }
                \AttributeTok{position =} \StringTok{"center"}\NormalTok{) }\SpecialCharTok{\%\textgreater{}\%}
  \FunctionTok{add\_header\_above}\NormalTok{(}\FunctionTok{c}\NormalTok{(}\StringTok{" "} \OtherTok{=} \DecValTok{1}\NormalTok{, }\StringTok{"Modelos"} \OtherTok{=} \DecValTok{2}\NormalTok{)) }\SpecialCharTok{\%\textgreater{}\%}
  \FunctionTok{footnote}\NormalTok{(}\AttributeTok{general =} \StringTok{"* p\textless{}0.05; ** p\textless{}0.01; *** p\textless{}0.001. Valores em parênteses são erros{-}padrão."}\NormalTok{)}
\end{Highlighting}
\end{Shaded}

\centering
\begin{tabular}[t]{lcc}
\toprule
\multicolumn{1}{c}{ } & \multicolumn{2}{c}{Modelos} \\
\cmidrule(l{3pt}r{3pt}){2-3}
Parametro & OLS & CAR\\
\midrule
Intercepto & 8.973 (0.182)*** & 8.993 (0.325)***\\
Variável X & 0.058 (0.185) & -0.029 (0.172)\\
Lambda (Espacial) & - & 0.733 (0.059)***\\
AIC & 5275 & 5199\\
\bottomrule
\multicolumn{3}{l}{\rule{0pt}{1em}\textit{Note: }}\\
\multicolumn{3}{l}{\rule{0pt}{1em}* p<0.05; ** p<0.01; *** p<0.001. Valores em parênteses são erros-padrão.}\\
\end{tabular}

\begin{Shaded}
\begin{Highlighting}[]
\CommentTok{\# Diagnóstico dos Resíduos}
\NormalTok{mg\_dados}\SpecialCharTok{$}\NormalTok{resid\_ols }\OtherTok{\textless{}{-}} \FunctionTok{residuals}\NormalTok{(mod\_ols)}
\NormalTok{mg\_dados}\SpecialCharTok{$}\NormalTok{resid\_car }\OtherTok{\textless{}{-}} \FunctionTok{residuals}\NormalTok{(mod\_car)}

\CommentTok{\# Teste de Moran}
\NormalTok{moran\_car }\OtherTok{\textless{}{-}} \FunctionTok{moran.test}\NormalTok{(mg\_dados}\SpecialCharTok{$}\NormalTok{resid\_car, }\AttributeTok{listw=}\NormalTok{lw)}
\FunctionTok{print}\NormalTok{(}\FunctionTok{paste}\NormalTok{(}\StringTok{"I de Moran (Resíduos CAR):"}\NormalTok{, }\FunctionTok{round}\NormalTok{(moran\_car}\SpecialCharTok{$}\NormalTok{estimate[}\DecValTok{1}\NormalTok{], }\DecValTok{3}\NormalTok{), }
            \StringTok{"| p{-}valor:"}\NormalTok{, }\FunctionTok{round}\NormalTok{(moran\_car}\SpecialCharTok{$}\NormalTok{p.value, }\DecValTok{3}\NormalTok{)))}
\end{Highlighting}
\end{Shaded}

{[}1{]} ``I de Moran (Resíduos CAR): -0.187 \textbar{} p-valor: 1''

\begin{Shaded}
\begin{Highlighting}[]
\NormalTok{p1 }\OtherTok{\textless{}{-}} \FunctionTok{ggplot}\NormalTok{(mg\_dados) }\SpecialCharTok{+}
  \FunctionTok{geom\_sf}\NormalTok{(}\FunctionTok{aes}\NormalTok{(}\AttributeTok{fill =}\NormalTok{ resid\_ols), }\AttributeTok{color =} \ConstantTok{NA}\NormalTok{) }\SpecialCharTok{+}
  \FunctionTok{scale\_fill\_distiller}\NormalTok{(}\AttributeTok{palette =} \StringTok{"RdBu"}\NormalTok{, }\AttributeTok{name =} \StringTok{"Resíduos"}\NormalTok{) }\SpecialCharTok{+}
  \FunctionTok{labs}\NormalTok{(}\AttributeTok{title =} \StringTok{"A. Resíduos OLS"}\NormalTok{, }\AttributeTok{subtitle =} \StringTok{"Dependência visível"}\NormalTok{) }\SpecialCharTok{+}
  
  \FunctionTok{annotation\_scale}\NormalTok{(}
    \AttributeTok{location =} \StringTok{"bl"}\NormalTok{,           }
    \AttributeTok{width\_hint =} \FloatTok{0.3}\NormalTok{,          }
    \AttributeTok{bar\_cols =} \FunctionTok{c}\NormalTok{(}\StringTok{"black"}\NormalTok{, }\StringTok{"white"}\NormalTok{), }
    \AttributeTok{text\_family =} \StringTok{"sans"}       
\NormalTok{  ) }\SpecialCharTok{+}
  
  \FunctionTok{annotation\_north\_arrow}\NormalTok{(}
    \AttributeTok{location =} \StringTok{"tl"}\NormalTok{,           }
    \AttributeTok{which\_north =} \StringTok{"true"}\NormalTok{,      }
    \AttributeTok{pad\_x =} \FunctionTok{unit}\NormalTok{(}\FloatTok{0.2}\NormalTok{, }\StringTok{"in"}\NormalTok{),   }
    \AttributeTok{pad\_y =} \FunctionTok{unit}\NormalTok{(}\FloatTok{0.2}\NormalTok{, }\StringTok{"in"}\NormalTok{),   }
    \AttributeTok{style =}\NormalTok{ north\_arrow\_fancy\_orienteering }
\NormalTok{  )}\SpecialCharTok{+}
  \FunctionTok{theme\_void}\NormalTok{()}

\NormalTok{p2 }\OtherTok{\textless{}{-}} \FunctionTok{ggplot}\NormalTok{(mg\_dados) }\SpecialCharTok{+}
  \FunctionTok{geom\_sf}\NormalTok{(}\FunctionTok{aes}\NormalTok{(}\AttributeTok{fill =}\NormalTok{ resid\_car), }\AttributeTok{color =} \ConstantTok{NA}\NormalTok{) }\SpecialCharTok{+}
  \FunctionTok{scale\_fill\_distiller}\NormalTok{(}\AttributeTok{palette =} \StringTok{"RdBu"}\NormalTok{, }\AttributeTok{name =} \StringTok{"Resíduos"}\NormalTok{) }\SpecialCharTok{+}
  \FunctionTok{labs}\NormalTok{(}\AttributeTok{title =} \StringTok{"B. Resíduos CAR"}\NormalTok{, }\AttributeTok{subtitle =} \StringTok{"Padrão removido"}\NormalTok{) }\SpecialCharTok{+}
  
  \FunctionTok{annotation\_scale}\NormalTok{(}
    \AttributeTok{location =} \StringTok{"bl"}\NormalTok{,           }
    \AttributeTok{width\_hint =} \FloatTok{0.3}\NormalTok{,          }
    \AttributeTok{bar\_cols =} \FunctionTok{c}\NormalTok{(}\StringTok{"black"}\NormalTok{, }\StringTok{"white"}\NormalTok{), }
    \AttributeTok{text\_family =} \StringTok{"sans"}       
\NormalTok{  ) }\SpecialCharTok{+}
  
  \FunctionTok{annotation\_north\_arrow}\NormalTok{(}
    \AttributeTok{location =} \StringTok{"tl"}\NormalTok{,           }
    \AttributeTok{which\_north =} \StringTok{"true"}\NormalTok{,      }
    \AttributeTok{pad\_x =} \FunctionTok{unit}\NormalTok{(}\FloatTok{0.2}\NormalTok{, }\StringTok{"in"}\NormalTok{),   }
    \AttributeTok{pad\_y =} \FunctionTok{unit}\NormalTok{(}\FloatTok{0.2}\NormalTok{, }\StringTok{"in"}\NormalTok{),   }
    \AttributeTok{style =}\NormalTok{ north\_arrow\_fancy\_orienteering }
\NormalTok{  )}\SpecialCharTok{+}
  \FunctionTok{theme\_void}\NormalTok{()}

\NormalTok{p1 }\SpecialCharTok{+}\NormalTok{ p2}
\end{Highlighting}
\end{Shaded}

}

\end{table}%

\textbf{Interpretação}

A tabela de resultados demonstra que o modelo CAR proporcionou uma
redução substancial no critério de informação AIC (de 5275 para 5199),
indicando um ajuste melhor em relação a regressão linear (OLS). O
parâmetro espacial \(\lambda\) (Lambda), estimado em 0,733, foi
significativo (\(p < 0,001\)), confirmando que a vizinhança exerce
influência determinante no processo, capturando a variabilidade que o
modelo linear clássico falhou em explicar. Note-se ainda que a variável
explicativa aleatória permaneceu não significativa em ambos os modelos,
como esperado, mas o erro-padrão e o intercepto foram ajustados para
refletir a incerteza real do sistema.

Quanto ao diagnóstico final, o teste de Moran aplicado aos resíduos do
modelo CAR resultou em um índice de -0,187 com um valor-p de 1 (não
significativo). Indica a rejeição da hipótese de aglomeração espacial
positiva (clusters) nos erros. O valor-p unitário sugere que a forte
autocorrelação positiva, anteriormente presente no OLS, foi absorvida
pela estrutura condicional do modelo, restando resíduos que,
estatisticamente, não apresentam mais o padrão de agrupamento que
invalidava a inferência anterior.

\section{Modelo CAR Intrínseco
(ICAR)}\label{modelo-car-intruxednseco-icar}

Um caso limite do modelo CAR ocorre quando \(\rho \to 1\). Este modelo,
denominado CAR Intrínseco , é amplamente utilizado como prior para
efeitos aleatórios espaciais (Julian Besag, York, e Mollié 1991; Held e
Rue 2010; Matthew J. Keefe, Ferreira, e Franck 2018b; Michael J. Keefe,
Ferreira, e Franck 2019).

\subsection{Modelo Condicional Autorregressivo Intrínseco
(ICAR)}\label{sec-icar}

O modelo Condicional Autorregressivo Intrínseco (\emph{Intrinsic
Conditional Autoregressive},
(\href{https://cran.r-project.org/web/packages/ref.ICAR/vignettes/ref-icar-vignette.html}{ICAR}))
é um caso particular e amplamente utilizado do modelo CAR, no qual a
matriz de precisão (\(\mathbf{Q}\)) é singular (não é invertível),
refletindo uma prior espacial que apenas penaliza diferenças entre
vizinhos, sem especificar um nível absoluto para as variáveis. O ICAR
pode ser entendido como o limite de um CAR próprio quando o parâmetro de
dependência espacial se aproxima da fronteira do espaço paramétrico,
resultando em uma matriz de precisão (\(\mathbf{Q}\)) semidefinida
positiva com posto \(n-1\) (Julian Besag e Kooperberg 1995).

Partindo da especificação condicional do CAR dada na Eq.~\ref{eq-car}, o
ICAR é definido quando a dependência espacial atinge seu máximo
(\(\rho \approx 1\)), com a média condicional de cada área dependendo
apenas da média simples de seus vizinhos. Para pesos da matriz de
vizinhança (\(w_{ij}=1\) se \(i \sim j\), 0 caso contrário), as
distribuições condicionais são:

\begin{equation}\phantomsection\label{eq-icar-cond}{
Y_i \mid \mathbf{Y}_{-i} \sim \mathcal{N}\left( \bar{Y}_{\partial i},\, \frac{\sigma^2}{m_i} \right),
\qquad
\mathbb{E}[Y_i \mid \mathbf{Y}_{-i}] = \bar{Y}_{\partial i} = \frac{1}{m_i}\sum_{j \in \partial i} Y_j,
\qquad
\text{Var}[Y_i \mid \mathbf{Y}_{-i}] = \frac{\sigma^2}{m_i},
}\end{equation}

onde \(\partial i\) denota o conjunto de vizinhos da área \(i\),
\(m_i = |\partial i|\) é o número de vizinhos (cardinal de
\(\partial i\)), e \(\sigma^2 > 0\) é um parâmetro de escala que
controla a suavidade espacial. Nesta formulação, o valor esperado
condicional é a média aritmética dos valores nas áreas vizinhas
(\(\mathbb{E}[Y_i \mid \mathbf{Y}_{-i}]\)), e a variância condicional
(\(\text{Var}[Y_i \mid \mathbf{Y}_{-i}]\)) é inversamente proporcional
ao número de vizinhos, refletindo maior incerteza em áreas com menos
conexões/vizinhos (Held e Rue 2010).

A especificação condicional implica uma matriz de precisão conjunta
(\(\mathbf{Q}\)) singular (sem inversa), com posto \(n-1\). Seja
\(\mathbf{D} = \text{diag}(m_1, \dots, m_n)\) e
\(\mathbf{W}=[w_{ij}]_{n \times n}\) a matriz de de vizinhança. A matriz
de precisão do ICAR é proporcional a
\(\mathbf{Q} = \sigma^{-2} (\mathbf{D} - \mathbf{W})=\tau (\mathbf{D} - \mathbf{W})\),
que é semidefinida positiva, com um autovalor zero correspondente ao
autovetor \(\mathbf{1}\) (vetor de uns). Consequentemente, a densidade
conjunta é imprópria e pode ser escrita (a menos de uma constante) como:

\[
p(\mathbf{Y}) \propto \exp\left( -\frac{1}{2\sigma^2} \sum_{i \sim j} (Y_i - Y_j)^2 \right),
\]

onde a soma percorre todos os pares de áreas adjacentes. Esta forma é
conhecida como \emph{pairwise difference prior} e destaca que a
densidade só depende dos contrastes locais entre vizinhos, sendo
invariante à adição de uma constante global (Julian Besag e Kooperberg
1995; Held e Rue 2010).

Para obter uma distribuição própria e identificável, é comum impor uma
restrição de soma-zero, \(\sum_{i=1}^n Y_i = 0\). Esta restrição pode
ser aplicada formalmente projetando o vetor de efeitos espaciais no
subespaço ortogonal a \(\mathbf{1}\), resultando em uma distribuição
Gaussiana singular própria com matriz de covariância proporcional à
\href{https://en.wikipedia.org/wiki/Moore\%E2\%80\%93Penrose_inverse}{pseudoinversa
de Moore-Penrose} de \(\mathbf{Q}_{ICAR}\) (Matthew J. Keefe, Ferreira,
e Franck 2018a). Em modelos hierárquicos Bayesianos, esta restrição é
frequentemente implementada durante a amostragem MCMC, mas a
formalização via projeção assegura unicidade e propriedades matemáticas
bem definidas.

O ICAR é extensivamente utilizado como componente espacial estruturado
em modelos hierárquicos, como no modelo Besag-York-Mollié (BYM) (
Seção~\ref{sec-BYM}) (Julian Besag, York, e Mollié 1991), que combina um
efeito ICAR para variação espacial suave e um efeito aleatório
independente para variação não estruturada. Matthew J. Keefe, Ferreira,
e Franck (2019) derivaram priors de referência objetivas para modelos
com componentes ICAR, facilitando análise Bayesiana automática sem a
necessidade de especificação subjetiva de hiperparâmetros.

\begin{table}

\caption{\label{tbl-icar_bayes}\pandocbounded{\includegraphics[keepaspectratio]{lattice_data_files/figure-pdf/tbl-icar_bayes-1.pdf}}}

\centering{

\begin{Shaded}
\begin{Highlighting}[]
\ControlFlowTok{if}\NormalTok{ (}\SpecialCharTok{!}\FunctionTok{require}\NormalTok{(}\StringTok{"pacman"}\NormalTok{)) }\FunctionTok{install.packages}\NormalTok{(}\StringTok{"pacman"}\NormalTok{)}

\CommentTok{\# CARBayes é o pacote padrão para modelagem Bayesiana de áreas no CRAN}
\NormalTok{pacman}\SpecialCharTok{::}\FunctionTok{p\_load}\NormalTok{(CARBayes, spdep, sf, ggplot2, patchwork, coda, gt)}

\ControlFlowTok{if}\NormalTok{ (}\SpecialCharTok{!}\FunctionTok{exists}\NormalTok{(}\StringTok{"mg\_dados"}\NormalTok{)) \{}
\NormalTok{  mg\_dados }\OtherTok{\textless{}{-}}\NormalTok{ geobr}\SpecialCharTok{::}\FunctionTok{read\_municipality}\NormalTok{(}\AttributeTok{code\_muni =} \StringTok{"MG"}\NormalTok{, }\AttributeTok{year =} \DecValTok{2020}\NormalTok{, }\AttributeTok{showProgress =} \ConstantTok{FALSE}\NormalTok{)}
\NormalTok{  coords }\OtherTok{\textless{}{-}} \FunctionTok{st\_coordinates}\NormalTok{(}\FunctionTok{st\_centroid}\NormalTok{(mg\_dados))}
  \FunctionTok{set.seed}\NormalTok{(}\DecValTok{123}\NormalTok{)}
\NormalTok{  mg\_dados}\SpecialCharTok{$}\NormalTok{taxa\_bruta }\OtherTok{\textless{}{-}}\NormalTok{ (}\SpecialCharTok{{-}}\NormalTok{coords[,}\DecValTok{2}\NormalTok{] }\SpecialCharTok{*} \DecValTok{10}\NormalTok{) }\SpecialCharTok{+} \FunctionTok{rnorm}\NormalTok{(}\FunctionTok{nrow}\NormalTok{(mg\_dados), }\DecValTok{0}\NormalTok{, }\DecValTok{5}\NormalTok{)}
\NormalTok{  mg\_dados}\SpecialCharTok{$}\NormalTok{variavel\_x }\OtherTok{\textless{}{-}} \FunctionTok{rnorm}\NormalTok{(}\FunctionTok{nrow}\NormalTok{(mg\_dados))}
\NormalTok{\}}

\CommentTok{\#Matriz de Vizinhança Binária (W)}
\CommentTok{\# O pacote CARBayes exige uma matriz binária (0 e 1), não normalizada.}
\NormalTok{nb }\OtherTok{\textless{}{-}} \FunctionTok{poly2nb}\NormalTok{(mg\_dados, }\AttributeTok{queen =} \ConstantTok{TRUE}\NormalTok{)}
\NormalTok{W\_binaria }\OtherTok{\textless{}{-}} \FunctionTok{nb2mat}\NormalTok{(nb, }\AttributeTok{style =} \StringTok{"B"}\NormalTok{, }\AttributeTok{zero.policy =} \ConstantTok{TRUE}\NormalTok{)}

\CommentTok{\#Ajuste do Modelo ICAR (Bayesiano)}

\FunctionTok{set.seed}\NormalTok{(}\DecValTok{123}\NormalTok{)}
\NormalTok{modelo\_icar }\OtherTok{\textless{}{-}} \FunctionTok{S.CARleroux}\NormalTok{(}\AttributeTok{formula =}\NormalTok{ taxa\_bruta }\SpecialCharTok{\textasciitilde{}}\NormalTok{ variavel\_x, }
                           \AttributeTok{data =}\NormalTok{ mg\_dados, }
                           \AttributeTok{family =} \StringTok{"gaussian"}\NormalTok{, }\CommentTok{\#terias que ver a distr dos seus dados para decidir aqui}
                           \AttributeTok{W =}\NormalTok{ W\_binaria, }
                           \AttributeTok{burnin =} \DecValTok{2000}\NormalTok{,   }\CommentTok{\# Descarta as primeiras 2000 iterações (aquecimento)}
                           \AttributeTok{n.sample =} \DecValTok{10000}\NormalTok{, }\CommentTok{\# Total de amostras MCMC}
                           \AttributeTok{thin =} \DecValTok{10}\NormalTok{,        }\CommentTok{\# Salva a cada 10 para reduzir autocorrelação}
                           \AttributeTok{rho =} \DecValTok{1}\NormalTok{,          }\CommentTok{\# RHO = 1 define o ICAR}
                           \AttributeTok{verbose =} \ConstantTok{FALSE}\NormalTok{)}


\CommentTok{\#Extrair os resultados}
\CommentTok{\# O objeto mod\_icar$summary.results contém médias e quantis}
\NormalTok{summ }\OtherTok{\textless{}{-}} \FunctionTok{as.data.frame}\NormalTok{(modelo\_icar}\SpecialCharTok{$}\NormalTok{summary.results)}


\NormalTok{params\_interesse }\OtherTok{\textless{}{-}} \FunctionTok{c}\NormalTok{(}\StringTok{"(Intercept)"}\NormalTok{, }\StringTok{"variavel\_x"}\NormalTok{, }\StringTok{"tau2"}\NormalTok{, }\StringTok{"nu2"}\NormalTok{)}
\NormalTok{summ\_filt }\OtherTok{\textless{}{-}}\NormalTok{ summ[}\FunctionTok{rownames}\NormalTok{(summ) }\SpecialCharTok{\%in\%}\NormalTok{ params\_interesse, ]}

\CommentTok{\# Função para formatar: "Média [IC 2.5\%; IC 97.5\%]"}
\NormalTok{format\_bayes }\OtherTok{\textless{}{-}} \ControlFlowTok{function}\NormalTok{(mean, lower, upper) \{}
  \FunctionTok{paste0}\NormalTok{(}\FunctionTok{format}\NormalTok{(}\FunctionTok{round}\NormalTok{(mean, }\DecValTok{3}\NormalTok{), }\AttributeTok{nsmall=}\DecValTok{3}\NormalTok{), }\StringTok{" ["}\NormalTok{, }
         \FunctionTok{format}\NormalTok{(}\FunctionTok{round}\NormalTok{(lower, }\DecValTok{3}\NormalTok{), }\AttributeTok{nsmall=}\DecValTok{3}\NormalTok{), }\StringTok{"; "}\NormalTok{, }
         \FunctionTok{format}\NormalTok{(}\FunctionTok{round}\NormalTok{(upper, }\DecValTok{3}\NormalTok{), }\AttributeTok{nsmall=}\DecValTok{3}\NormalTok{), }\StringTok{"]"}\NormalTok{)}
\NormalTok{\}}

\NormalTok{tabela\_icar }\OtherTok{\textless{}{-}} \FunctionTok{data.frame}\NormalTok{(}
  \AttributeTok{Parametro =} \FunctionTok{rownames}\NormalTok{(summ\_filt),}
  \AttributeTok{Estimativa =} \FunctionTok{mapply}\NormalTok{(format\_bayes, summ\_filt}\SpecialCharTok{$}\NormalTok{Mean, summ\_filt}\SpecialCharTok{$}\StringTok{\textasciigrave{}}\AttributeTok{2.5\%}\StringTok{\textasciigrave{}}\NormalTok{, summ\_filt}\SpecialCharTok{$}\StringTok{\textasciigrave{}}\AttributeTok{97.5\%}\StringTok{\textasciigrave{}}\NormalTok{)}
\NormalTok{)}

\CommentTok{\# Renomear}
\NormalTok{tabela\_icar}\SpecialCharTok{$}\NormalTok{Parametro }\OtherTok{\textless{}{-}} \FunctionTok{recode}\NormalTok{(}
\NormalTok{  tabela\_icar}\SpecialCharTok{$}\NormalTok{Parametro,}
  \StringTok{"(Intercept)"} \OtherTok{=} \StringTok{"Intercepto"}\NormalTok{,}
  \StringTok{"variavel\_x"}  \OtherTok{=} \StringTok{"Variável X"}\NormalTok{,}
  \StringTok{"tau2"}         \OtherTok{=} \StringTok{"Variância Espacial $}\SpecialCharTok{\textbackslash{}\textbackslash{}}\StringTok{tau\^{}2$"}\NormalTok{,}
  \StringTok{"nu2"}          \OtherTok{=} \StringTok{"Variância do Erro $}\SpecialCharTok{\textbackslash{}\textbackslash{}}\StringTok{nu\^{}2$"}
\NormalTok{)}


\FunctionTok{kbl}\NormalTok{(tabela\_icar, }
    \AttributeTok{format =} \StringTok{"latex"}\NormalTok{,}
    \AttributeTok{booktabs =} \ConstantTok{TRUE}\NormalTok{, }
    \AttributeTok{align =} \StringTok{"lc"}\NormalTok{, }
    \AttributeTok{caption =} \ConstantTok{NULL}\NormalTok{, }
    \AttributeTok{escape =} \ConstantTok{FALSE}\NormalTok{) }\SpecialCharTok{\%\textgreater{}\%}
  \FunctionTok{kable\_styling}\NormalTok{(}\AttributeTok{latex\_options =} \FunctionTok{c}\NormalTok{(}\StringTok{"HOLD\_position"}\NormalTok{, }\StringTok{"striped"}\NormalTok{), }
                \AttributeTok{full\_width =} \ConstantTok{FALSE}\NormalTok{, }
                \AttributeTok{position =} \StringTok{"center"}\NormalTok{) }\SpecialCharTok{\%\textgreater{}\%}
  \FunctionTok{footnote}\NormalTok{(}\AttributeTok{general =} \StringTok{"Estimativas: Média a posteriori [IC 95\%]."}\NormalTok{) }
\end{Highlighting}
\end{Shaded}

\centering
\begin{tabular}[t]{lc}
\toprule
Parametro & Estimativa\\
\midrule
\cellcolor{gray!10}{Intercepto} & \cellcolor{gray!10}{8.975 [8.657; 9.285]}\\
Variável X & -0.001 [-0.332; 0.353]\\
\cellcolor{gray!10}{Variância do Erro $\nu^2$} & \cellcolor{gray!10}{21.944 [19.461; 24.745]}\\
Variância Espacial $\tau^2$ & 5.672 [3.028; 9.547]\\
\bottomrule
\multicolumn{2}{l}{\rule{0pt}{1em}\textit{Note: }}\\
\multicolumn{2}{l}{\rule{0pt}{1em}Estimativas: Média a posteriori [IC 95\%].}\\
\end{tabular}

\begin{Shaded}
\begin{Highlighting}[]
\CommentTok{\# O modelo estima um efeito aleatório (phi) para cada município.}
\NormalTok{mg\_dados}\SpecialCharTok{$}\NormalTok{efeito\_icar }\OtherTok{\textless{}{-}} \FunctionTok{apply}\NormalTok{(modelo\_icar}\SpecialCharTok{$}\NormalTok{samples}\SpecialCharTok{$}\NormalTok{phi, }\DecValTok{2}\NormalTok{, mean)}

\CommentTok{\#}
\FunctionTok{ggplot}\NormalTok{(mg\_dados) }\SpecialCharTok{+}
  \FunctionTok{geom\_sf}\NormalTok{(}\FunctionTok{aes}\NormalTok{(}\AttributeTok{fill =}\NormalTok{ efeito\_icar), }\AttributeTok{color =} \ConstantTok{NA}\NormalTok{) }\SpecialCharTok{+}
  \FunctionTok{scale\_fill\_distiller}\NormalTok{(}\AttributeTok{palette =} \StringTok{"RdBu"}\NormalTok{, }\AttributeTok{name =} \FunctionTok{expression}\NormalTok{(}\StringTok{"Efeito Espacial"} \SpecialCharTok{\textasciitilde{}}\NormalTok{Phi)) }\SpecialCharTok{+}
  \FunctionTok{labs}\NormalTok{(}\AttributeTok{title =} \StringTok{"Componente Espacial ICAR"}\NormalTok{, }
       \AttributeTok{subtitle =} \StringTok{"Padrão latente recuperado (Suavizado)"}\NormalTok{) }\SpecialCharTok{+}
  \FunctionTok{theme\_void}\NormalTok{() }\SpecialCharTok{+}
  \FunctionTok{theme}\NormalTok{(}\AttributeTok{plot.title =} \FunctionTok{element\_text}\NormalTok{(}\AttributeTok{face =} \StringTok{"bold"}\NormalTok{))}\SpecialCharTok{+}
  
  \FunctionTok{annotation\_scale}\NormalTok{(}
    \AttributeTok{location =} \StringTok{"bl"}\NormalTok{,           }
    \AttributeTok{width\_hint =} \FloatTok{0.3}\NormalTok{,          }
    \AttributeTok{bar\_cols =} \FunctionTok{c}\NormalTok{(}\StringTok{"black"}\NormalTok{, }\StringTok{"white"}\NormalTok{), }
    \AttributeTok{text\_family =} \StringTok{"sans"}       
\NormalTok{  ) }\SpecialCharTok{+}
  
  \FunctionTok{annotation\_north\_arrow}\NormalTok{(}
    \AttributeTok{location =} \StringTok{"tl"}\NormalTok{,           }
    \AttributeTok{which\_north =} \StringTok{"true"}\NormalTok{,      }
    \AttributeTok{pad\_x =} \FunctionTok{unit}\NormalTok{(}\FloatTok{0.2}\NormalTok{, }\StringTok{"in"}\NormalTok{),   }
    \AttributeTok{pad\_y =} \FunctionTok{unit}\NormalTok{(}\FloatTok{0.2}\NormalTok{, }\StringTok{"in"}\NormalTok{),   }
    \AttributeTok{style =}\NormalTok{ north\_arrow\_fancy\_orienteering }
\NormalTok{  )}
\end{Highlighting}
\end{Shaded}

}

\end{table}%

\textbf{Interpretação}

O intercepto do modelo foi estimado em 8,975 e é significativo uma vez
que seu intervalo de pois o intervalo credibilidade de 95\% (entre 8,657
e 9,285) encontra-se inteiramente acima de zero (não inclui zero). Em
contrapartida, a covariável aleatória (``Variável X'') não demonstrou
qualquer influência explicativa sobre a taxa, apresentando uma
estimativa pontual quase nula (-0,001) e um intervalo de credibilidade
que inclui zero, o que confirma sua irrelevância para o fenômeno
analisado.

A variância do erro não estruturado (\(\nu^2\)) foi estimada em 21,944,
indicando uma presença de ruído nos dados brutos, enquanto a variância
espacial (\(\tau^2\)), estimada em 5,672 com um intervalo de
credibilidade estritamente positivo (3,028 a 9,547), confirma que há um
componente de vizinhança. A visualização deste efeito espacial
(\(\phi\)) no mapa recupera o gradiente norte-sul latente, isolando-o
efetivamente da variabilidade aleatória capturada pelo termo de erro.

\begin{tcolorbox}[enhanced jigsaw, colframe=quarto-callout-color-frame, opacityback=0, rightrule=.15mm, arc=.35mm, bottomrule=.15mm, colback=white, toprule=.15mm, leftrule=.75mm, breakable, left=2mm]

\vspace{-3mm}\textbf{Saiba mais}\vspace{3mm}

Acesse ao \href{https://connordonegan.github.io/geostan/index.html}{link
1} e
\href{https://cran.r-project.org/web/packages/ref.ICAR/vignettes/ref-icar-vignette.html}{link
2} e veja outras formas de de ajuste do mesmo modelo.

\end{tcolorbox}

\section{Modelos Simultâneos Autorregressivos (SAR)}\label{sec-SAR}

O modelo Autorregressivo Simultâneo (SAR - \emph{Simultaneous
Autoregressive}), introduzido por Whittle (1954), tem raízes na análise
de séries temporais e é predominante na econometria espacial (Anselin
1988).

Ao contrário do CAR (Seção~\ref{sec-car}), o SAR modela a dependência
através de um sistema de equações simultâneas onde a variável \(Y_i\) é
função direta dos seus valores contemporâneos \(Y_j\) e de um termo de
erro:

\[Y_i = \rho \sum_{j=1}^n w_{ij} Y_j + \mathbf{x}_i^\top \boldsymbol{\beta} + \epsilon_i, \quad \epsilon_i \sim \mathcal{N}(0, \sigma^2)\]

Aqui, \(\rho \sum w_{ij} Y_j\) é o termo de defasagem espacial
(\emph{spatial lag}). Existe um efeito de feedback global: um choque em
\(\epsilon_i\) afeta \(Y_i\), que afeta os vizinhos \(Y_j\), que por sua
vez afetam de volta \(Y_i\) e se propagam por todo o sistema.

Em notação matricial, o modelo é:

\[
\mathbf{Y} = \rho \mathbf{W}\mathbf{Y} + \mathbf{X}\boldsymbol{\beta} + \boldsymbol{\epsilon} \implies (\mathbf{I} - \rho \mathbf{W})\mathbf{Y} = \mathbf{X}\boldsymbol{\beta} + \boldsymbol{\epsilon}, \: \: \mathbf{Y} = (\mathbf{I} - \rho \mathbf{W})^{-1}\mathbf{X}\boldsymbol{\beta} + \boldsymbol{\epsilon}, \: \: \text{ se a inversa existir}
\]

A forma reduzida para a média e a covariância é:

\[\mathbf{Y}|\mathbf{X} \sim \mathcal{N}\left( (\mathbf{I} - \rho \mathbf{W})^{-1}\mathbf{X}\boldsymbol{\beta}, \sigma^2 [(\mathbf{I} - \rho \mathbf{W})^\top (\mathbf{I} - \rho \mathbf{W})]^{-1} \right)\]

Ver Hoef, Hanks, e Hooten (2018) e Wall (2004) destacam as diferenças
fundamentais na estrutura de covariância:

\begin{enumerate}
\def\labelenumi{\arabic{enumi}.}
\item
  A matriz de precisão do SAR é
  \(\mathbf{Q}_{SAR} = \sigma^{-2}(\mathbf{I} - \rho \mathbf{W})^\top (\mathbf{I} - \rho \mathbf{W})\).
  Note o produto cruzado das matrizes. Isso implica que a correlação no
  SAR decai mais lentamente com a distância do grafo do que no CAR.
\item
  O parâmetro \(\rho\) no SAR tem uma interpretação de \emph{spillover}
  global (efeito multiplicador), enquanto no CAR é uma medida de
  correlação condicional local.
\end{enumerate}

\begin{tcolorbox}[enhanced jigsaw, left=2mm, toptitle=1mm, colback=white, colframe=quarto-callout-note-color-frame, colbacktitle=quarto-callout-note-color!10!white, opacityback=0, rightrule=.15mm, bottomtitle=1mm, arc=.35mm, title=\textcolor{quarto-callout-note-color}{\faInfo}\hspace{0.5em}{Nota}, titlerule=0mm, bottomrule=.15mm, leftrule=.75mm, coltitle=black, toprule=.15mm, breakable, opacitybacktitle=0.6]

\begin{enumerate}
\def\labelenumi{\arabic{enumi}.}
\item
  Outras especificações dos modelos SAR e CAR são discutidas no capítulo
  4 do livro Scalon (2024).
\item
  Consulte o capítulo 4 do livro Scalon (2024) para conhecer todos os
  procedimentos a seguir antes do ajuste do modelo, desde a análise
  exploratória até o ajuste propriamente dito.
\item
  A descrição completa dos pacotes \texttt{spdep} e \texttt{spatialreg}
  encontra-se no capítulo 4 do livro Scalon (2024).
\end{enumerate}

\end{tcolorbox}

\section{O Modelo BYM (Besag-York-Mollié) e sua reparametrização
(BYM2)}\label{sec-BYM}

Até este ponto, vimos os modelos CAR, SAR e ICAR que consideram que a
variável de interesse \(\mathbf{Y}\) segue, condicional aos parâmetros,
uma distribuição Normal multivariada. Contudo, em epidemiologia,
demografia e ciências sociais, é comum lidar com dados de contagem
\(y_i\) observados em uma região \(i\), as quais são comumente modeladas
como variáveis aleatórias seguindo uma distribuição de Poisson:

\[y_i \mid \theta_i \sim \text{Poisson}(E_i \theta_i), \quad i = 1, \dots, n\]

Nessa formulação, \(E_i\) representa o número esperado de casos,
ajustado por características populacionais, enquanto \(\theta_i\) denota
o risco relativo verdadeiro ( ver Seção~\ref{sec-risco}). Como os dados
são de contagem, é imperativo que a média (número esperado) do processo
seja positiva. No entanto, se utilizássemos um modelo linear direto para
o risco, poderíamos incorrer no erro de estimar valores negativos, uma
vez que, na distribuição Normal, o suporte do parâmetro de média
compreende todos os números reais. É importante enfatizar que o foco da
modelagem estatística reside na estrutura da média, embora existam
extensões importantes como os modelos lineares generalizados (GLM)
duplos, que modelam a média e a variância (ver Paula 2013), ou os
modelos GAMLSS, que permitem modelar simultaneamente a média, a
variância, a curtose e a assimetria (ver Stasinopoulos et al. 2017,
2024; Rigby et al. 2019).

Para resolver o problema da restrição de positividade, não se aplica uma
transformação diretamente à variável resposta, como se faria em modelos
de regressão clássicos para estabilização de variância, mas sim uma
função de ligação ao valor esperado do processo. Essa é a essência dos
Modelos Lineares Generalizados (GLM). No caso da distribuição Poisson, a
função de ligação canônica é o logaritmo natural (ver outras funções em
(Paula 2025)), o que nos permite definir o preditor linear
\(\eta_i = \log(\theta_i)\). Esse preditor é então modelado através de
uma estrutura aditiva que incorpora o intercepto \(\mu\), o efeito das
covariáveis \(\mathbf{x}_i^\top \boldsymbol{\beta}\) e um termo de
efeito aleatório espacial \(\varepsilon_i\), resultando na expressão:

\[\eta_i = \log(\theta_i)=\mu + \mathbf{x}_i^\top \boldsymbol{\beta} + \varepsilon_i .\]

\subsection{Modelo BYM}\label{modelo-bym}

Proposto por Julian Besag, York, e Mollié (1991), o modelo
Besag-York-Mollié (BYM) decompõe o efeito aleatório espacial
\(\varepsilon_i\) em dois componentes aditivos (Riebler et al. 2016):

\[
\varepsilon_i = u_i + v_i
\]

onde:

\begin{itemize}
\tightlist
\item
  \(u_i\) é componente estrutural que modela a dependência espacial.
  Assume-se uma prior ICAR (ver seção sobre Seção~\ref{sec-icar}), onde
  a distribuição condicional de \(u_i\) depende apenas dos vizinhos
  \(\partial i\):
\end{itemize}

\[u_i \mid \mathbf{u}_{-i}, \tau_u \sim \mathcal{N}\left( \frac{1}{m_i} \sum_{j \in \partial i} u_j, \frac{1}{m_i \tau_u} \right), \: \:\frac{1}{\tau_u} = \sigma_u^2\]

A distribuição conjunta é imprópria e dada por
\(\mathbf{u} \sim \mathcal{N}(\mathbf{0}, \tau_u^{-1}\mathbf{Q}^-)\),
onde \(\mathbf{Q}\) é a matriz de estrutura definida pela vizinhança
(\(Q_{ii} = m_i\) e \(Q_{ij} = -1\) se \(i \sim j\)) e \(\mathbf{Q}^-\)
denota a sua
\href{https://en.wikipedia.org/wiki/Generalized_inverse}{inversa
generalizada}. \(\tau_u\) é a precisão (inverso da variância) deste
componente.

\begin{itemize}
\tightlist
\item
  \(v_i\) é componente não estrutural que modela o ruído aleatório
  independente (heterogeneidade pura). Assume-se normalidade i.i.d.:
\end{itemize}

\[v_i \mid \tau_v \sim \mathcal{N}(0, \tau_v^{-1}), \: \: \frac{1}{\tau_v} = \sigma_v^2\]

A variância marginal do efeito total \(\varepsilon_i\) no modelo BYM
original é, portanto, a soma das variâncias dos componentes:

\[\text{Var}(\varepsilon_i \mid \tau_u, \tau_v) = \text{Var}(v_i) + \text{Var}(u_i) = \tau_v^{-1} + (\tau_u^{-1}\mathbf{Q}^-)_{ii}\]

Apesar de sua ampla utilização no mapeamento de doenças, Riebler et al.
(2016) identificaram limitações nesta parametrização que afetam a
inferência:

\begin{enumerate}
\def\labelenumi{\arabic{enumi}.}
\item
  Identificabilidade: Apenas a soma \(\varepsilon_i = u_i + v_i\) é
  identificável. A separação entre \(u\) e \(v\) depende inteiramente
  das distribuições a priori, e os hiperparâmetros \(\tau_u\) e
  \(\tau_v\) competem pela explicação da variância total. Lembre-se que
  um parâmetro \(\boldsymbol{\theta}\) é considerado identificável se
  valores distintos do parâmetro implicam necessariamente em
  distribuições de probabilidade distintas para os dados observados
  (formalmente, se
  \(P_{\boldsymbol{\theta}_1} = P_{\boldsymbol{\theta}_2} \implies \boldsymbol{\theta}_1 = \boldsymbol{\theta}_2\)).
  No modelo BYM, infinitas combinações dos componentes \(u_i\) e \(v_i\)
  podem resultar no mesmo valor latente \(\varepsilon_i\) e, por
  consequência, na mesma função de verossimilhança.
\item
  Escalonamento (Scaling): A variância marginal do componente espacial
  \((\mathbf{Q}^-)_{ii}\) não é constante e depende da geometria do
  grafo (matriz) de vizinhança. Em grafos mais conectados (regiões com
  muitos vizinhos), a variância marginal induzida por um mesmo
  \(\tau_u\) é menor do que em grafos menos conectados (regiões com
  poucos vizinhos). Isso torna as priors para \(\tau_u\) não
  transferíveis: uma prior que é vagamente informativa para um mapa pode
  ser fortemente informativa para outro, impedindo a criação de padrões
  de referência (Sigrunn H. Sørbye e Rue 2014).
\end{enumerate}

Exemplo: Imagine que você ajusta um modelo espacial para os municípios
de do estado de São Paulo (muitos vizinhos, malha densa) e outro para os
municípios do estado do Amazonas (áreas enormes, poucos vizinhos). Se
você usar a mesma prior para a precisão \(\tau\) (ex: \(\tau=1\)) em
ambos os mapas, no mapa denso, a variância induzida será pequena
enquanto no mapa esparso, a variância induzida será grande. Isso
significa que a prior não é transferível, ou seja, o significado de
\(\tau\) muda dependendo do mapa que você está usando.

\subsection{Modelo BYM2}\label{modelo-bym2}

Para solucionar os problemas de escalonamento e confundimento de
parâmetros, Riebler et al. (2016) propuseram o modelo BYM2. A inovação
consiste em escalonar a matriz de precisão espacial e reparametrizar o
modelo em termos de variância total e proporção de variância espacial.

Primeiramente, define-se uma matriz de estrutura escalonada
\(\mathbf{Q}_*\). O escalonamento é realizado de tal forma que a média
geométrica das variâncias marginais generalizadas do componente espacial
seja igual a 1:

\[\exp\left( \frac{1}{n} \sum_{i=1}^n \log((\mathbf{Q}^-)_{ii}) \right) = 1\]

Esta operação normaliza o tamanho médio do efeito espacial, tornando-o
comparável ao efeito não estruturado (cuja variância é 1 na escala
padronizada). Com a matriz \(\mathbf{Q}_*\) normalizada, o vetor de
efeitos aleatórios
\(\boldsymbol{\varepsilon} = (\varepsilon_1, \dots, \varepsilon_n)^\top\)
é reescrito como:

\[\boldsymbol{\varepsilon} = \frac{1}{\sqrt{\tau}} \left( \sqrt{1 - \phi}\,\mathbf{v} + \sqrt{\phi}\,\mathbf{u}_* \right)\]

onde \(\mathbf{v} \sim \mathcal{N}(\mathbf{0}, \mathbf{I})\) é o
componente não estruturado padronizado;
\(\mathbf{u}_* \sim \mathcal{N}(\mathbf{0}, \mathbf{Q}_*^-)\) é o
componente estruturado escalonado; \(\tau > 0\) é o parâmetro de
precisão marginal total e, \(\phi \in [0, 1]\) é o parâmetro de mistura.

A matriz de covariância do vetor \(\boldsymbol{\varepsilon}\) no modelo
BYM2 torna-se:

\[\text{Var}(\boldsymbol{\varepsilon} \mid \tau, \phi) = \frac{1}{\tau} \left[ (1 - \phi)\mathbf{I} + \phi \mathbf{Q}_*^- \right]\]

Esta formulação desacopla a magnitude da variabilidade da estrutura de
dependência, facilitando a interpretação. Assim, \(\tau\) controla a
variância marginal total do efeito latente. \(\sigma = \tau^{-1/2}\) é o
desvio padrão marginal de \(\varepsilon\), sendo invariante à estrutura
do grafo (região com muitos ou poucos vizinhos), enquanto \(\phi\)
representa a proporção da variância total que é explicada pela estrutura
espacial.

Note que se \(\phi=1\), temos um modelo puramente ICAR; se \(\phi=0\),
um modelo puramente i.i.d.

Esta parametrização permite o uso de PC Priors (Penalised Complexity
Priors, (Sigrunn Holbek Sørbye e Rue 2017; Simpson et al. 2017)), onde o
pesquisador pode expressar conhecimento a priori de forma intuitiva (ex:
acredito que a probabilidade do efeito espacial ser superior a 0.5 é
baixa).

\section{Priors de Penalização de Complexidade (PC
Priors)}\label{priors-de-penalizauxe7uxe3o-de-complexidade-pc-priors}

A especificação de distribuições a priori para parâmetros de variância e
correlação em modelos hierárquicos Bayesianos constitui, historicamente,
um desafio. As escolhas tradicionais, notadamente a família de
distribuições
\href{https://en.wikipedia.org/wiki/Inverse-gamma_distribution}{Gama-Inversa}
(\(\epsilon, \epsilon\)) para a variância \(\sigma^2\) (ou
\href{https://en.wikipedia.org/wiki/Gamma_distribution}{Gama} para a
precisão) com parâmetros vagos (por exemplo, \(\epsilon = 0.001\)),
foram severamente criticadas por Gelman (2006). Gelman demonstra que,
embora matematicamente convenientes devido à conjugação condicional,
tais distribuições são inadequadas como referência não informativa. O
autor argumenta que a família Gama-Inversa é patológica no limite
\(\epsilon \to 0\), pois não converge para uma distribuição a posteriori
própria em situações onde a verossimilhança permite variância nula.
Consequentemente, a inferência torna-se instável e altamente sensível à
escolha arbitrária do hiperparâmetro \(\epsilon\), especialmente em
conjuntos de dados com um número pequeno de grupos ou quando a variância
dos efeitos aleatórios é pequena. Na prática, a prior Gama-Inversa
introduz um viés que empurra a massa da posteriori para longe da origem,
distorcendo a inferência ao sugerir uma variabilidade entre grupos maior
do que a suportada pelos dados. Gelman (2006) recomenda, em vez disso, o
uso de prioris Uniformes ou da família
\href{https://en.wikipedia.org/wiki/Folded-t_and_half-t_distributions}{Half-t}
(como a Half-Cauchy) para o desvio padrão \(\sigma\), que possuem melhor
comportamento na origem e nas caudas.

Em resposta a essas e várias outras limitações, Simpson et al. (2017)
propuseram uma estrutura unificada para a construção de distribuições a
priori denominadas Priors de Penalização de Complexidade (\emph{PC
Priors}).

A construção das PC Priors fundamenta-se em quatro princípios. O
primeiro princípio, o
\href{https://pt.wikipedia.org/wiki/Navalha_de_Ockham}{princípio da
parcimônia}, estabelece a existência de um modelo base (denotado por
\(g\)) que é uma versão simplificada do modelo flexível (denotado por
\(f\)). A prior deve ser construída de modo a favorecer o modelo base,
penalizando o afastamento deste a menos que os dados forneçam evidência
robusta para justificar a complexidade adicional. O segundo princípio
define uma medida de complexidade baseada na Divergência de
Kullback-Leibler (KLD),

\[\text{KLD}(f \| g) = \int f(x) \log(f(x)/g(x)) dx\],

que quantifica a perda de informação ao aproximar o modelo flexível pelo
modelo base. Para conferir interpretabilidade, essa divergência é
transformada em uma distância unidirecional definida como

\[d(\xi) = \sqrt{2 \text{KLD}(f \| g)},\]

onde \(\xi\) é o parâmetro que governa a flexibilidade do modelo.

O terceiro princípio determina a taxa de penalização. Simpson et al.
(2017) argumentam que, na ausência de informações externas que sugiram o
contrário, a penalização deve ocorrer a uma taxa constante ao longo da
distância métrica \(d\). A única distribuição de probabilidade contínua
definida em \([0, \infty)\) que satisfaz a propriedade de falta de
memória (taxa de risco constante) é a distribuição exponencial.
Portanto, a prior na escala da distância deve ser

\[\pi(d) = \lambda e^{-\lambda d},\]

onde \(\lambda\) é a taxa de decaimento. A distribuição a priori para o
parâmetro original \(\xi\) é então obtida através da regra de
transformação de variáveis, dada por

\[\pi(\xi) = \lambda e^{-\lambda d(\xi)} \left| \partial d(\xi) / \partial \xi \right|.\]

O quarto princípio permite a calibração definida pelo usuário. Em vez de
escolher o parâmetro \(\lambda\) diretamente, o pesquisador especifica
uma regra probabilística intuitiva sobre a cauda da distribuição, da
forma

\[\text{Prob}(Q(\xi) > U) = \alpha,\]

onde \(Q(\xi)\) é uma transformação interpretável do parâmetro, \(U\) é
um limite superior e \(\alpha\) uma probabilidade pequena.

\textbf{Exemplo da Aplicação de PC Priors}

Considere um pesquisador que analisa a incidência de uma doença
infecciosa em \(n\) municípios. Os dados observados são contagens
\(y_i\) de casos, modeladas como
\(y_i \mid \theta_i \sim \text{Poisson}(E_i \theta_i)\), onde \(E_i\) é
o número esperado de casos e \(\theta_i\) o risco relativo. O preditor
linear é definido como
\(\eta_i = \log(\theta_i) = \mu + \mathbf{x}_i^\top \boldsymbol{\beta} + \varepsilon_i\).
O efeito aleatório espacial \(\varepsilon_i\) é modelado utilizando a
reparametrização BYM2:
\(\varepsilon_i = \tau^{-1/2} \left( \sqrt{1-\phi} \, v_i + \sqrt{\phi} \, u_i^* \right)\),
onde \(\tau\) é a precisão marginal total e \(\phi \in [0,1]\) é o
parâmetro de mistura. A especificação de distribuições a priori para
\(\tau\) e \(\phi\) é crucial. O uso de priors Gama para a precisão,
como \(\tau \sim \Gamma(1, 0.01)\), é comum, mas problemático, pois
coloca densidade zero em \(\tau \to \infty\) (o modelo base de variância
nula), podendo levar a sobreajuste ao não permitir o encolhimento
adequado.

A aplicação das PC Priors resolve este problema. O primeiro passo é
identificar os modelos base para cada parâmetro (se o objetivo é modelar
dependência espacial, o modelo base é aquele que não tem dependência
espacial). Para a precisão \(\tau\), o modelo base é a ausência de
efeitos aleatórios, correspondente a \(\tau \to \infty\) ou,
equivalentemente, ao desvio padrão marginal
\(\sigma = \tau^{-1/2} = 0\). Para o parâmetro de mistura \(\phi\), o
modelo base é a ausência de dependência espacial, ou seja, \(\phi = 0\).
As PC Priors serão construídas para penalizar o afastamento destes
modelos base.

Na prática, o pesquisador traduz conhecimento epidemiológico prévio em
declarações probabilísticas (isso é prior). Para a precisão \(\tau\), é
mais intuitivo pensar na escala do desvio padrão \(\sigma\) (o quão
afastados estão os casos de dengue em Lavras em relação média geral de
todo estado de Minas Gerais) . Suponha que, com base na literatura, o
pesquisador acredite ser improvável que a variação não explicada no
log-risco seja extrema. Um desvio padrão de \(\sigma = 0.5\) implica que
os riscos relativos (na escala original) variam tipicamente por um fator
de até \(e^{0.5} \approx 1.65\) para mais ou para menos. O pesquisador
pode considerar que é pouco plausível que a variação não explicada
supere este patamar. Assim, formula-se a declaração:
\(\text{Prob}(\sigma > 0.5) = 0.01\). Isto significa que se atribui
apenas 1\% de probabilidade a priori a valores de \(\sigma\) superiores
a 0.5.

Esta declaração define os parâmetros da PC Prior: \(U = 0.5\) e
\(\alpha = 0.01\). Conforme a derivação de Simpson et al. (2017), isso
implica uma distribuição exponencial para \(\sigma\) com taxa
\(\lambda = -\ln(\alpha)/U = -\ln(0.01)/0.5 \approx 9.21\). A prior
correspondente para a precisão \(\tau\) é uma distribuição tipo-2
Gumbel, mas o usuário não precisa manipulá-la diretamente.

Para o parâmetro de mistura \(\phi\), que representa a proporção da
variância atribuída à dependência espacial, o raciocínio é similar. O
modelo base é \(\phi=0\) (ausência de estrutura espacial). O pesquisador
deve refletir sobre a importância relativa esperada da dependência
espacial. Em muitas aplicações epidemiológicas, é razoável assumir, na
falta de informação forte, que a heterogeneidade não espacial (ruído)
pode ser tão ou mais importante que a estrutura espacial. Uma declaração
conservadora poderia ser: \(\text{Prob}(\phi > 0.5) = 0.5\). Isto
significa que se atribui igual probabilidade (50\%) a \(\phi\) estar
acima ou abaixo de 0.5, mas ainda assim a prior é construída para
encolher em direção a zero.

No pacote \texttt{R-INLA}, a especificação destas PC Priors é direta.
Suponha que \texttt{id\_regiao} é um vetor de índices que identificam
cada município do estado de Minas Gerais, \texttt{grafo} é a matriz de
vizinhança, e \texttt{y} é o vetor de contagens. A fórmula do modelo,
incorporando as PC Priors com os parâmetros \(U\) e \(\alpha\) definidos
acima, seria:

O argumento \texttt{param\ =\ c(U,\ α)} na prior \texttt{"pc.prec"} para
a precisão codifica exatamente a declaração
\(\text{Prob}(\sigma > U) = \alpha\). Para a prior \texttt{"pc"} no
parâmetro \texttt{phi}, \texttt{param\ =\ c(0.5,\ 0.5)} codifica
\(\text{Prob}(\phi > 0.5) = 0.5\). A opção \texttt{scale.model\ =\ TRUE}
garante o escalonamento automático da matriz de precisão do componente
estruturado, essencial para a interpretabilidade de \(\phi\).

A prior para \(\sigma\) (derivada de \texttt{pc.prec}) tem seu pico em
zero e decai exponencialmente. Assim, se os dados forem escassos ou
pouco informativos, a posteriori de \(\sigma\) será puxada para valores
próximos de zero, efetivamente encolhendo os efeitos aleatórios
\(\varepsilon_i\) e produzindo estimativas de risco mais suaves e
próximas da média global. Segundo, a prior para \(\phi\) também encolhe
em direção a zero. Portanto, na ausência de um sinal espacial claro nos
dados, o modelo tenderá a alocar mais variância ao componente não
estruturado (\(v_i\)), evitando a imposição de uma suavização espacial
injustificada.

\begin{table}

\caption{\label{tbl-bym2_inla}Resultados do Modelo BYM2 com PC Priors.}

\centering{

\begin{Shaded}
\begin{Highlighting}[]
\CommentTok{\# Instalação do INLA }
\FunctionTok{options}\NormalTok{(}\AttributeTok{timeout =} \DecValTok{600}\NormalTok{)}

\ControlFlowTok{if}\NormalTok{ (}\SpecialCharTok{!}\FunctionTok{requireNamespace}\NormalTok{(}\StringTok{"INLA"}\NormalTok{, }\AttributeTok{quietly =} \ConstantTok{TRUE}\NormalTok{)) \{}
   \FunctionTok{install.packages}\NormalTok{(}\StringTok{"INLA"}\NormalTok{, }\AttributeTok{repos=}\FunctionTok{c}\NormalTok{(}\FunctionTok{getOption}\NormalTok{(}\StringTok{"repos"}\NormalTok{), }\AttributeTok{INLA=}\StringTok{"https://inla.r{-}inla{-}download.org/R/stable"}\NormalTok{), }\AttributeTok{dep=}\ConstantTok{TRUE}\NormalTok{)}
\NormalTok{\}}


\CommentTok{\# Veja como instalar INLA no link:  https://www.r{-}inla.org/download{-}install}

\ControlFlowTok{if}\NormalTok{ (}\SpecialCharTok{!}\FunctionTok{require}\NormalTok{(}\StringTok{"pacman"}\NormalTok{)) }\FunctionTok{install.packages}\NormalTok{(}\StringTok{"pacman"}\NormalTok{)}
\NormalTok{pacman}\SpecialCharTok{::}\FunctionTok{p\_load}\NormalTok{(INLA, spdep, sf, dplyr, knitr, kableExtra, ggplot2, patchwork)}


\CommentTok{\# Preparação dos Dados}
\ControlFlowTok{if}\NormalTok{ (}\SpecialCharTok{!}\FunctionTok{exists}\NormalTok{(}\StringTok{"mg\_dados"}\NormalTok{)) \{}
\NormalTok{  mg\_dados }\OtherTok{\textless{}{-}}\NormalTok{ geobr}\SpecialCharTok{::}\FunctionTok{read\_municipality}\NormalTok{(}\AttributeTok{code\_muni =} \StringTok{"MG"}\NormalTok{, }\AttributeTok{year =} \DecValTok{2020}\NormalTok{, }\AttributeTok{showProgress =} \ConstantTok{FALSE}\NormalTok{)}
\NormalTok{  coords }\OtherTok{\textless{}{-}} \FunctionTok{st\_coordinates}\NormalTok{(}\FunctionTok{st\_centroid}\NormalTok{(mg\_dados))}
  \FunctionTok{set.seed}\NormalTok{(}\DecValTok{123}\NormalTok{)}
\NormalTok{  mg\_dados}\SpecialCharTok{$}\NormalTok{taxa\_bruta }\OtherTok{\textless{}{-}}\NormalTok{ (}\SpecialCharTok{{-}}\NormalTok{coords[,}\DecValTok{2}\NormalTok{] }\SpecialCharTok{*} \DecValTok{10}\NormalTok{) }\SpecialCharTok{+} \FunctionTok{rnorm}\NormalTok{(}\FunctionTok{nrow}\NormalTok{(mg\_dados), }\DecValTok{0}\NormalTok{, }\DecValTok{5}\NormalTok{)}
\NormalTok{  mg\_dados}\SpecialCharTok{$}\NormalTok{variavel\_x }\OtherTok{\textless{}{-}} \FunctionTok{rnorm}\NormalTok{(}\FunctionTok{nrow}\NormalTok{(mg\_dados))}
\NormalTok{\}}

\CommentTok{\# Criar um identificador numérico sequencial para as áreas (exigência do INLA)}
\NormalTok{mg\_dados}\SpecialCharTok{$}\NormalTok{ID\_AREA }\OtherTok{\textless{}{-}} \DecValTok{1}\SpecialCharTok{:}\FunctionTok{nrow}\NormalTok{(mg\_dados)}

\CommentTok{\#Matriz de Vizinhança e Grafo}
\NormalTok{nb }\OtherTok{\textless{}{-}} \FunctionTok{poly2nb}\NormalTok{(mg\_dados, }\AttributeTok{queen =} \ConstantTok{TRUE}\NormalTok{)}

\FunctionTok{plot}\NormalTok{(}\FunctionTok{st\_geometry}\NormalTok{(mg\_dados), }\AttributeTok{border =} \StringTok{"lightgrey"}\NormalTok{, }\AttributeTok{main =} \StringTok{"Estrutura de Vizinhança (Grafo)"}\NormalTok{)}
\FunctionTok{plot}\NormalTok{(nb, coords, }\AttributeTok{add =} \ConstantTok{TRUE}\NormalTok{, }\AttributeTok{col =} \StringTok{"red"}\NormalTok{, }\AttributeTok{pch =} \DecValTok{19}\NormalTok{, }\AttributeTok{cex =} \FloatTok{0.6}\NormalTok{, }\AttributeTok{lwd =} \FloatTok{0.5}\NormalTok{)}
\end{Highlighting}
\end{Shaded}

\pandocbounded{\includegraphics[keepaspectratio]{lattice_data_files/figure-pdf/tbl-bym2_inla-1.pdf}}

\begin{Shaded}
\begin{Highlighting}[]
\CommentTok{\# Converter para formato de grafo do INLA}
\FunctionTok{nb2INLA}\NormalTok{(}\StringTok{"mg\_graph.adj"}\NormalTok{, nb)}
\NormalTok{g }\OtherTok{\textless{}{-}} \FunctionTok{inla.read.graph}\NormalTok{(}\StringTok{"mg\_graph.adj"}\NormalTok{)}


\CommentTok{\#Definição dos PC Priors (Penalised Complexity)}
\CommentTok{\# Prior para a Precisão (Tau): Prob(desvio padrão \textgreater{} 1) = 0.01 , vc pode usar outros}
   \CommentTok{\#A probabilidade do Desvio Padrão (σ) ser maior que 1 é de apenas 1\% (0.01).}

\CommentTok{\# Prior para Mistura (Phi): Prob(phi \textless{} 0.5) = 0.5 (neutro), note que Phi = 1 (tudo é espacial), Phi=0 (tudo é aleatório).}
       \CommentTok{\#Você está dizendo ao modelo: "Eu não sei se o fenômeno é mais espacial ou }
       \CommentTok{\# mais aleatório, então vou deixar 50\% de chance para cada lado"}

\NormalTok{hyper\_pc }\OtherTok{\textless{}{-}} \FunctionTok{list}\NormalTok{(}
  \AttributeTok{prec =} \FunctionTok{list}\NormalTok{(}\AttributeTok{prior =} \StringTok{"pc.prec"}\NormalTok{, }\AttributeTok{param =} \FunctionTok{c}\NormalTok{(}\DecValTok{1}\NormalTok{, }\FloatTok{0.01}\NormalTok{)),}
  \AttributeTok{phi  =} \FunctionTok{list}\NormalTok{(}\AttributeTok{prior =} \StringTok{"pc"}\NormalTok{, }\AttributeTok{param =} \FunctionTok{c}\NormalTok{(}\FloatTok{0.5}\NormalTok{, }\FloatTok{0.5}\NormalTok{))}
\NormalTok{)}

\CommentTok{\#Ajuste do Modelo BYM2}

\NormalTok{formula\_bym2 }\OtherTok{\textless{}{-}}\NormalTok{ taxa\_bruta }\SpecialCharTok{\textasciitilde{}}\NormalTok{ variavel\_x }\SpecialCharTok{+} 
                \FunctionTok{f}\NormalTok{(ID\_AREA, }\AttributeTok{model =} \StringTok{"bym2"}\NormalTok{, }\AttributeTok{graph =}\NormalTok{ g, }\AttributeTok{scale.model =} \ConstantTok{TRUE}\NormalTok{, }
                  \AttributeTok{hyper =}\NormalTok{ hyper\_pc)}

\NormalTok{modelo\_inla }\OtherTok{\textless{}{-}} \FunctionTok{inla}\NormalTok{(formula\_bym2, }
                    \AttributeTok{data =}\NormalTok{ mg\_dados, }
                    \AttributeTok{family =} \StringTok{"gaussian"}\NormalTok{, }
                    \AttributeTok{control.predictor =} \FunctionTok{list}\NormalTok{(}\AttributeTok{compute =} \ConstantTok{TRUE}\NormalTok{), }
                    \AttributeTok{control.compute =} \FunctionTok{list}\NormalTok{(}\AttributeTok{dic =} \ConstantTok{TRUE}\NormalTok{, }\AttributeTok{waic =} \ConstantTok{TRUE}\NormalTok{))}

\CommentTok{\# Função auxiliar de formatação "Média [IC 95\%]"}
\NormalTok{fmt }\OtherTok{\textless{}{-}} \ControlFlowTok{function}\NormalTok{(m, l, u) \{}
  \FunctionTok{paste0}\NormalTok{(}\FunctionTok{format}\NormalTok{(}\FunctionTok{round}\NormalTok{(m, }\DecValTok{3}\NormalTok{), }\AttributeTok{nsmall=}\DecValTok{3}\NormalTok{), }\StringTok{" ["}\NormalTok{, }
         \FunctionTok{format}\NormalTok{(}\FunctionTok{round}\NormalTok{(l, }\DecValTok{3}\NormalTok{), }\AttributeTok{nsmall=}\DecValTok{3}\NormalTok{), }\StringTok{"; "}\NormalTok{, }
         \FunctionTok{format}\NormalTok{(}\FunctionTok{round}\NormalTok{(u, }\DecValTok{3}\NormalTok{), }\AttributeTok{nsmall=}\DecValTok{3}\NormalTok{), }\StringTok{"]"}\NormalTok{)}
\NormalTok{\}}

\CommentTok{\# Efeitos Fixos}
\NormalTok{fix }\OtherTok{\textless{}{-}}\NormalTok{ modelo\_inla}\SpecialCharTok{$}\NormalTok{summary.fixed[, }\FunctionTok{c}\NormalTok{(}\StringTok{"mean"}\NormalTok{, }\StringTok{"0.025quant"}\NormalTok{, }\StringTok{"0.975quant"}\NormalTok{)]}
\NormalTok{df\_fix }\OtherTok{\textless{}{-}} \FunctionTok{data.frame}\NormalTok{(}
  \AttributeTok{Parametro =} \FunctionTok{rownames}\NormalTok{(fix),}
  \AttributeTok{Valor =} \FunctionTok{mapply}\NormalTok{(fmt, fix}\SpecialCharTok{$}\NormalTok{mean, fix}\SpecialCharTok{$}\StringTok{\textasciigrave{}}\AttributeTok{0.025quant}\StringTok{\textasciigrave{}}\NormalTok{, fix}\SpecialCharTok{$}\StringTok{\textasciigrave{}}\AttributeTok{0.975quant}\StringTok{\textasciigrave{}}\NormalTok{)}
\NormalTok{)}

\CommentTok{\# renomear nomes}
\NormalTok{df\_fix}\SpecialCharTok{$}\NormalTok{Parametro }\OtherTok{\textless{}{-}} \FunctionTok{recode}\NormalTok{(df\_fix}\SpecialCharTok{$}\NormalTok{Parametro, }
                           \StringTok{"(Intercept)"} \OtherTok{=} \StringTok{"Intercepto"}\NormalTok{, }
                           \StringTok{"variavel\_x"} \OtherTok{=} \StringTok{"Variável X"}\NormalTok{)}

\CommentTok{\# Hiperparâmetros}
\NormalTok{hyp }\OtherTok{\textless{}{-}}\NormalTok{ modelo\_inla}\SpecialCharTok{$}\NormalTok{summary.hyperpar[, }\FunctionTok{c}\NormalTok{(}\StringTok{"mean"}\NormalTok{, }\StringTok{"0.025quant"}\NormalTok{, }\StringTok{"0.975quant"}\NormalTok{)]}
\NormalTok{df\_hyp }\OtherTok{\textless{}{-}} \FunctionTok{data.frame}\NormalTok{(}
  \AttributeTok{Parametro =} \FunctionTok{rownames}\NormalTok{(hyp),}
  \AttributeTok{Valor =} \FunctionTok{mapply}\NormalTok{(fmt, hyp}\SpecialCharTok{$}\NormalTok{mean, hyp}\SpecialCharTok{$}\StringTok{\textasciigrave{}}\AttributeTok{0.025quant}\StringTok{\textasciigrave{}}\NormalTok{, hyp}\SpecialCharTok{$}\StringTok{\textasciigrave{}}\AttributeTok{0.975quant}\StringTok{\textasciigrave{}}\NormalTok{)}
\NormalTok{)}

\NormalTok{df\_hyp}\SpecialCharTok{$}\NormalTok{Parametro }\OtherTok{\textless{}{-}} \FunctionTok{recode}\NormalTok{(df\_hyp}\SpecialCharTok{$}\NormalTok{Parametro,}
                           \StringTok{"Precision for the Gaussian observations"} \OtherTok{=} \StringTok{"Precisão (Likelihood)"}\NormalTok{,}
                           \StringTok{"Precision for ID\_AREA"} \OtherTok{=} \StringTok{"Precisão Marginal $}\SpecialCharTok{\textbackslash{}\textbackslash{}}\StringTok{tau$"}\NormalTok{,}
                           \StringTok{"Phi for ID\_AREA"} \OtherTok{=} \StringTok{"Dependência Espacial $}\SpecialCharTok{\textbackslash{}\textbackslash{}}\StringTok{phi$"}\NormalTok{)}

\CommentTok{\#Métricas de Ajuste}
\NormalTok{df\_met }\OtherTok{\textless{}{-}} \FunctionTok{data.frame}\NormalTok{(}
  \AttributeTok{Parametro =} \FunctionTok{c}\NormalTok{(}\StringTok{"DIC"}\NormalTok{, }\StringTok{"WAIC"}\NormalTok{),}
  \AttributeTok{Valor =} \FunctionTok{c}\NormalTok{(}\FunctionTok{format}\NormalTok{(}\FunctionTok{round}\NormalTok{(modelo\_inla}\SpecialCharTok{$}\NormalTok{dic}\SpecialCharTok{$}\NormalTok{dic, }\DecValTok{2}\NormalTok{), }\AttributeTok{nsmall=}\DecValTok{2}\NormalTok{),}
            \FunctionTok{format}\NormalTok{(}\FunctionTok{round}\NormalTok{(modelo\_inla}\SpecialCharTok{$}\NormalTok{waic}\SpecialCharTok{$}\NormalTok{waic, }\DecValTok{2}\NormalTok{), }\AttributeTok{nsmall=}\DecValTok{2}\NormalTok{))}
\NormalTok{)}

\CommentTok{\# Unir tudo}
\NormalTok{df\_final }\OtherTok{\textless{}{-}} \FunctionTok{rbind}\NormalTok{(df\_fix, df\_hyp, df\_met)}

\CommentTok{\# Gerar Tabela}
\FunctionTok{kbl}\NormalTok{(df\_final, }
    \AttributeTok{format =} \StringTok{"latex"}\NormalTok{,}
    \AttributeTok{booktabs =} \ConstantTok{TRUE}\NormalTok{, }
    \AttributeTok{align =} \StringTok{"lr"}\NormalTok{, }
    \AttributeTok{caption =} \ConstantTok{NULL}\NormalTok{) }\SpecialCharTok{\%\textgreater{}\%} 
  \FunctionTok{kable\_styling}\NormalTok{(}\AttributeTok{latex\_options =} \FunctionTok{c}\NormalTok{(}\StringTok{"HOLD\_position"}\NormalTok{), }
                \AttributeTok{full\_width =} \ConstantTok{FALSE}\NormalTok{, }
                \AttributeTok{position =} \StringTok{"center"}\NormalTok{) }\SpecialCharTok{\%\textgreater{}\%}
  \FunctionTok{pack\_rows}\NormalTok{(}\StringTok{"Efeitos Fixos (Média [IC 95\%])"}\NormalTok{, }\DecValTok{1}\NormalTok{, }\FunctionTok{nrow}\NormalTok{(df\_fix)) }\SpecialCharTok{\%\textgreater{}\%}
  \FunctionTok{pack\_rows}\NormalTok{(}\StringTok{"Hiperparâmetros (Média [IC 95\%])"}\NormalTok{, }\FunctionTok{nrow}\NormalTok{(df\_fix) }\SpecialCharTok{+} \DecValTok{1}\NormalTok{, }\FunctionTok{nrow}\NormalTok{(df\_fix) }\SpecialCharTok{+} \FunctionTok{nrow}\NormalTok{(df\_hyp)) }\SpecialCharTok{\%\textgreater{}\%}
  \FunctionTok{pack\_rows}\NormalTok{(}\StringTok{"Qualidade do Ajuste"}\NormalTok{, }\FunctionTok{nrow}\NormalTok{(df\_final) }\SpecialCharTok{{-}} \DecValTok{1}\NormalTok{, }\FunctionTok{nrow}\NormalTok{(df\_final)) }\SpecialCharTok{\%\textgreater{}\%}
  \FunctionTok{row\_spec}\NormalTok{((}\FunctionTok{nrow}\NormalTok{(df\_final)}\SpecialCharTok{{-}}\DecValTok{1}\NormalTok{)}\SpecialCharTok{:}\FunctionTok{nrow}\NormalTok{(df\_final), }\AttributeTok{bold =} \ConstantTok{TRUE}\NormalTok{) }
\end{Highlighting}
\end{Shaded}

\centering
\begin{tabular}[t]{lr}
\toprule
Parametro & Valor\\
\midrule
\addlinespace[0.3em]
\multicolumn{2}{l}{\textbf{Efeitos Fixos (Média [IC 95\%])}}\\
\hspace{1em}Intercepto & 8.974 [8.664; 9.283]\\
\hspace{1em}Variável X & -0.001 [-0.331; 0.328]\\
\addlinespace[0.3em]
\multicolumn{2}{l}{\textbf{Hiperparâmetros (Média [IC 95\%])}}\\
\hspace{1em}Precisão (Likelihood) & 20326.203 [1086.105; 70703.806]\\
\hspace{1em}Precisão Marginal \$\textbackslash{}tau\$ & 0.041 [0.037; 0.045]\\
\hspace{1em}Dependência Espacial \$\textbackslash{}phi\$ & 0.125 [0.069; 0.202]\\
\addlinespace[0.3em]
\multicolumn{2}{l}{\textbf{Qualidade do Ajuste}}\\
\textbf{\hspace{1em}DIC} & \textbf{-4822.42}\\
\textbf{\hspace{1em}WAIC} & \textbf{-5073.48}\\
\bottomrule
\end{tabular}

}

\end{table}%

\textbf{Interpretação}

A análise dos efeitos fixos (Tabela~\ref{tbl-bym2_inla}) mostra que o
intercepto do modelo foi estimado com uma média de 8,974, situando-se
num intervalo de credibilidade de 95\% entre 8,664 e 9,283 (não inclui
zero). Por outro lado, a covariável ``Variável X'' apresentou uma
estimativa média muito próxima de zero (-0,001) e, mais importante, o
seu intervalo de credibilidade varia de -0,331 a 0,328. Como este
intervalo inclui o valor zero, conclui-se que não existe evidência
estatística de uma associação significativa entre a Variável X e a
variável resposta; ou seja, a variável não contribui para explicar o
fenômeno estudado neste modelo.

No que nos hiperparâmetros e à estrutura aleatória, a precisão da
verossimilhança (\emph{Likelihood}) é extremamente alta (média de
21117,082), o que indica que a variância do erro residual dos dados em
torno da média predita é muito pequena. O parâmetro de dependência
espacial (\(\phi\)) foi estimado em 0,124 (com intervalo de 0,068 a
0,200), o que revela que apenas cerca de 12,4\% da variabilidade
capturada pelo efeito aleatório (campo latente) se deve à estrutura
espacial de vizinhança, enquanto a maior parte (os restantes 87,6\%) é
atribuída a ruído não estruturado (efeito iid). Por fim, os critérios de
informação DIC (-4822,46) e WAIC (-5056,00) indicam a qualidade do
ajuste, servindo como base de comparação para modelos alternativos, onde
valores menores indicariam um melhor ajuste.

\begin{Shaded}
\begin{Highlighting}[]
\CommentTok{\# DIAGNÓSTICO}

\NormalTok{mg\_dados}\SpecialCharTok{$}\NormalTok{efeito\_bym }\OtherTok{\textless{}{-}}\NormalTok{ modelo\_inla}\SpecialCharTok{$}\NormalTok{summary.random}\SpecialCharTok{$}\NormalTok{ID\_AREA}\SpecialCharTok{$}\NormalTok{mean[}\DecValTok{1}\SpecialCharTok{:}\FunctionTok{nrow}\NormalTok{(mg\_dados)]}
\NormalTok{mg\_dados}\SpecialCharTok{$}\NormalTok{ajustado   }\OtherTok{\textless{}{-}}\NormalTok{ modelo\_inla}\SpecialCharTok{$}\NormalTok{summary.fitted.values}\SpecialCharTok{$}\NormalTok{mean}
\NormalTok{mg\_dados}\SpecialCharTok{$}\NormalTok{residuos   }\OtherTok{\textless{}{-}}\NormalTok{ mg\_dados}\SpecialCharTok{$}\NormalTok{taxa\_bruta }\SpecialCharTok{{-}}\NormalTok{ mg\_dados}\SpecialCharTok{$}\NormalTok{ajustado}

\CommentTok{\#Mapa do Efeito Espacial (Risco/Nível estimado)}
\NormalTok{p1 }\OtherTok{\textless{}{-}} \FunctionTok{ggplot}\NormalTok{(mg\_dados) }\SpecialCharTok{+}
  \FunctionTok{geom\_sf}\NormalTok{(}\FunctionTok{aes}\NormalTok{(}\AttributeTok{fill =}\NormalTok{ efeito\_bym), }\AttributeTok{color =} \ConstantTok{NA}\NormalTok{) }\SpecialCharTok{+}
  \FunctionTok{scale\_fill\_distiller}\NormalTok{(}\AttributeTok{palette =} \StringTok{"RdBu"}\NormalTok{, }\AttributeTok{name =} \StringTok{"Efeito}\SpecialCharTok{\textbackslash{}n}\StringTok{Espacial"}\NormalTok{) }\SpecialCharTok{+}
  \FunctionTok{labs}\NormalTok{(}\AttributeTok{title =} \StringTok{"A. Componente Espacial (BYM2)"}\NormalTok{) }\SpecialCharTok{+}
  \FunctionTok{theme\_minimal}\NormalTok{()}\SpecialCharTok{+}
  
  \FunctionTok{annotation\_scale}\NormalTok{(}
    \AttributeTok{location =} \StringTok{"bl"}\NormalTok{,           }
    \AttributeTok{width\_hint =} \FloatTok{0.3}\NormalTok{,          }
    \AttributeTok{bar\_cols =} \FunctionTok{c}\NormalTok{(}\StringTok{"black"}\NormalTok{, }\StringTok{"white"}\NormalTok{), }
    \AttributeTok{text\_family =} \StringTok{"sans"}       
\NormalTok{  ) }\SpecialCharTok{+}
  
  \FunctionTok{annotation\_north\_arrow}\NormalTok{(}
    \AttributeTok{location =} \StringTok{"tl"}\NormalTok{,           }
    \AttributeTok{which\_north =} \StringTok{"true"}\NormalTok{,      }
    \AttributeTok{pad\_x =} \FunctionTok{unit}\NormalTok{(}\FloatTok{0.2}\NormalTok{, }\StringTok{"in"}\NormalTok{),   }
    \AttributeTok{pad\_y =} \FunctionTok{unit}\NormalTok{(}\FloatTok{0.2}\NormalTok{, }\StringTok{"in"}\NormalTok{),   }
    \AttributeTok{style =}\NormalTok{ north\_arrow\_fancy\_orienteering }
\NormalTok{  )}

\CommentTok{\# Mapa dos Resíduos }
\NormalTok{p2 }\OtherTok{\textless{}{-}} \FunctionTok{ggplot}\NormalTok{(mg\_dados) }\SpecialCharTok{+}
  \FunctionTok{geom\_sf}\NormalTok{(}\FunctionTok{aes}\NormalTok{(}\AttributeTok{fill =}\NormalTok{ residuos), }\AttributeTok{color =} \ConstantTok{NA}\NormalTok{) }\SpecialCharTok{+}
  \FunctionTok{scale\_fill\_distiller}\NormalTok{(}\AttributeTok{palette =} \StringTok{"PuOr"}\NormalTok{, }\AttributeTok{name =} \StringTok{"Resíduos"}\NormalTok{) }\SpecialCharTok{+}
  \FunctionTok{labs}\NormalTok{(}\AttributeTok{title =} \StringTok{"B. Mapa de Resíduos"}\NormalTok{) }\SpecialCharTok{+}
   \FunctionTok{theme\_minimal}\NormalTok{()}\SpecialCharTok{+}
  
  \FunctionTok{annotation\_scale}\NormalTok{(}
    \AttributeTok{location =} \StringTok{"bl"}\NormalTok{,           }
    \AttributeTok{width\_hint =} \FloatTok{0.3}\NormalTok{,          }
    \AttributeTok{bar\_cols =} \FunctionTok{c}\NormalTok{(}\StringTok{"black"}\NormalTok{, }\StringTok{"white"}\NormalTok{), }
    \AttributeTok{text\_family =} \StringTok{"sans"}       
\NormalTok{  ) }\SpecialCharTok{+}
  
  \FunctionTok{annotation\_north\_arrow}\NormalTok{(}
    \AttributeTok{location =} \StringTok{"tl"}\NormalTok{,           }
    \AttributeTok{which\_north =} \StringTok{"true"}\NormalTok{,      }
    \AttributeTok{pad\_x =} \FunctionTok{unit}\NormalTok{(}\FloatTok{0.2}\NormalTok{, }\StringTok{"in"}\NormalTok{),   }
    \AttributeTok{pad\_y =} \FunctionTok{unit}\NormalTok{(}\FloatTok{0.2}\NormalTok{, }\StringTok{"in"}\NormalTok{),   }
    \AttributeTok{style =}\NormalTok{ north\_arrow\_fancy\_orienteering }
\NormalTok{  )}

\CommentTok{\#Gráfico Observado vs Esperado}
\NormalTok{cor\_p }\OtherTok{\textless{}{-}} \FunctionTok{round}\NormalTok{(}\FunctionTok{cor}\NormalTok{(mg\_dados}\SpecialCharTok{$}\NormalTok{taxa\_bruta, mg\_dados}\SpecialCharTok{$}\NormalTok{ajustado), }\DecValTok{3}\NormalTok{)}
\NormalTok{p3 }\OtherTok{\textless{}{-}} \FunctionTok{ggplot}\NormalTok{(mg\_dados, }\FunctionTok{aes}\NormalTok{(}\AttributeTok{x =}\NormalTok{ ajustado, }\AttributeTok{y =}\NormalTok{ taxa\_bruta)) }\SpecialCharTok{+}
  \FunctionTok{geom\_point}\NormalTok{(}\AttributeTok{alpha =} \FloatTok{0.3}\NormalTok{, }\AttributeTok{color =} \StringTok{"darkblue"}\NormalTok{) }\SpecialCharTok{+}
  \FunctionTok{geom\_abline}\NormalTok{(}\AttributeTok{col =} \StringTok{"red"}\NormalTok{, }\AttributeTok{linetype =} \StringTok{"dashed"}\NormalTok{) }\SpecialCharTok{+}
  \FunctionTok{labs}\NormalTok{(}\AttributeTok{title =} \StringTok{"C. Ajuste do Modelo"}\NormalTok{, }
       \AttributeTok{subtitle =} \FunctionTok{paste0}\NormalTok{(}\StringTok{"Correlação: "}\NormalTok{, cor\_p),}
       \AttributeTok{x =} \StringTok{"Predito (INLA)"}\NormalTok{, }\AttributeTok{y =} \StringTok{"Observado"}\NormalTok{) }\SpecialCharTok{+}
  \FunctionTok{theme\_minimal}\NormalTok{()}


\NormalTok{(p1 }\SpecialCharTok{+}\NormalTok{ p2 }\SpecialCharTok{+}\NormalTok{ p3)}

\CommentTok{\# TESTE DE MORAN NOS RESÍDUOS}

\NormalTok{lw }\OtherTok{\textless{}{-}} \FunctionTok{nb2listw}\NormalTok{(nb, }\AttributeTok{style =} \StringTok{"W"}\NormalTok{)}
\NormalTok{moran\_test }\OtherTok{\textless{}{-}} \FunctionTok{moran.test}\NormalTok{(mg\_dados}\SpecialCharTok{$}\NormalTok{residuos, lw)}

\FunctionTok{print}\NormalTok{(moran\_test)}
\end{Highlighting}
\end{Shaded}

\begin{verbatim}

    Moran I test under randomisation

data:  mg_dados$residuos  
weights: lw    

Moran I statistic standard deviate = -1.3916, p-value = 0.918
alternative hypothesis: greater
sample estimates:
Moran I statistic       Expectation          Variance 
    -0.0307506114     -0.0011737089      0.0004517036 
\end{verbatim}

\begin{Shaded}
\begin{Highlighting}[]
\ControlFlowTok{if}\NormalTok{(moran\_test}\SpecialCharTok{$}\NormalTok{p.value }\SpecialCharTok{\textgreater{}} \FloatTok{0.05}\NormalTok{) \{}
  \FunctionTok{message}\NormalTok{(}\StringTok{"SUCESSO: Resíduos são aleatórios (p \textgreater{} 0.05). O modelo removeu a autocorrelação."}\NormalTok{)}
\NormalTok{\} }\ControlFlowTok{else}\NormalTok{ \{}
  \FunctionTok{message}\NormalTok{(}\StringTok{"ATENÇÃO: Ainda existe dependência espacial nos resíduos."}\NormalTok{)}
\NormalTok{\}}
\end{Highlighting}
\end{Shaded}

\begin{figure}[H]

\centering{

\pandocbounded{\includegraphics[keepaspectratio]{lattice_data_files/figure-pdf/fig-diagnosticoBYM-1.pdf}}

}

\caption{\label{fig-diagnosticoBYM}Diagnóstico do modelo BYM2 ajustado}

\end{figure}%

\textbf{Interpretação}

O componente espacial mapeado em Figura~\ref{fig-diagnosticoBYM} (A)
delineia a variabilidade latente, capturando a heterogeneidade
geográfica do fenômeno. O ajuste é evidenciado na
Figura~\ref{fig-diagnosticoBYM} (C), onde a relação entre os valores
observados e os preditos pelo INLA exibe uma correlação unitária, com os
pontos assentando-se perfeitamente sobre a linha de identidade,
indicando ausência de viés sistemático nas predições.

A validade do modelo é confirmada pela análise dos resíduos. O mapa em
Figura~\ref{fig-diagnosticoBYM} (B) sugere visualmente uma distribuição
espacial aleatória dos erros, sem padrões de aglomeração
(\emph{clustering}) remanescentes. Esta inspeção visual é confirmada
pelo Teste de I de Moran realizado nos resíduos (\(I = -0.0311\);
\(p = 0.9203\)). O valor-p elevado não permite rejeitar a hipótese nula
de aleatoriedade espacial, o que confirma que o modelo BYM2 foi eficaz
em incorporar toda a autocorrelação espacial significativa presente nos
dados, restando nos resíduos apenas ruído branco estocástico.

\section{Estatística Espacial vs.~Econometria
Espacial}\label{estatuxedstica-espacial-vs.-econometria-espacial}

A escolha entre as famílias de modelos condicionais (CAR/ICAR/BYM) e
simultâneos (SAR) é fundamentalmente guiada pelo objetivo, pela natureza
teórica da dependência espacial e pelo contexto disciplinar. A
Estatística Espacial, com seus modelos CAR e ICAR, adota uma abordagem
condicional ou markoviana (Julian Besag 1974), cujo objetivo primário é
a predição, suavização e mapeamento de superfícies latentes, como o
risco de doença ou a concentração de poluentes (Noel Cressie 1993).
Neste paradigma, o foco reside em modelar com precisão a estrutura de
covariância para capturar padrões espaciais, utilizando uma formulação
baseada em distribuições condicionais locais. Esta especificação resulta
em uma matriz de precisão esparsa, o que confere eficiência
computacional a métodos bayesianos e hierárquicos, como MCMC e INLA
(Håvard Rue, Martino, e Chopin 2009; Havard Rue e Held 2005). Esta
abordagem é predominante em campos como a epidemiologia, ecologia e
ciências ambientais, onde a interpretação do processo espacial latente e
a quantificação da incerteza de predição são objetivos centrais (Riebler
et al. 2016; Banerjee, Carlin, e Gelfand 2003).

Em contraste, a Econometria Espacial, fundamentada no modelo SAR, adota
uma abordagem simultânea ou estrutural (Anselin 1988). Seu objetivo
primário é a inferência causal e a estimação não enviesada de parâmetros
que capturem efeitos de interação e transbordamento (\emph{spillovers})
entre unidades (J. LeSage e Pace 2009). Sua formulação constitui um
sistema de equações com feedback, onde o valor em uma unidade depende
simultaneamente dos valores em outras, gerando uma estrutura de precisão
geralmente densa (Ver Hoef, Hanks, e Hooten 2018). Esta característica
associa-a mais fortemente a métodos de estimação frequentistas, como a
máxima verossimilhança ou o método dos momentos generalizado (GMM)
(Anselin 2010). Esta abordagem é predominante em campos como economia
regional, ciências políticas e estudos urbanos, onde a questão central é
testar teorias sobre interdependência estratégica, externalidades
espaciais ou difusão de políticas.

A seleção do modelo deve ser uma função direta da pergunta de pesquisa.
Deve-se optar por modelos CAR, ICAR ou BYM quando o interesse principal
for a predição, suavização ou recuperação de um campo aleatório espacial
subjacente (Wall 2004). Estes são a escolha natural para a criação de
mapas de risco, interpolação de superfícies ou para modelagem bayesiana
hierárquica que ``empreste força'' (\emph{borrowing strength}) entre
áreas vizinhas (Riebler et al. 2016). Por outro lado, modelos SAR (ou
modelos de erro espacial - SEM) são mais adequados quando o interesse é
a inferência causal sobre um parâmetro de interação espacial (\(\rho\))
ou quando a teoria subjacente postula um mecanismo de dependência
simultânea e \emph{feedback} entre unidades observadas, como em modelos
de competição ou difusão (Ver Hoef, Hanks, e Hooten 2018). Tais modelos
são essenciais para corrigir viés de variável omitida espacial e para
estimar efeitos diretos e indiretos de políticas. Ambas as abordagens
são complementares, e a fronteira entre elas tem se tornado mais
permeável, com avanços metodológicos permitindo, por exemplo, a
interpretação de priors CAR em termos de equilíbrio de sistemas
dinâmicos ou o uso de técnicas de álgebra esparsa para a estimação
eficiente de certas especificações de modelos SAR (Ver Hoef, Hanks, e
Hooten 2018; J. LeSage e Pace 2009).

\section{Modelos Espaciais Econométricos para Dados de
Área}\label{modelos-espaciais-economuxe9tricos-para-dados-de-uxe1rea}

A econometria espacial distingue-se da estatística espacial clássica
(como os modelos CAR, SAR Seção~\ref{sec-car}) por sua ênfase na
identificação de relações causais, na fundamentação teórica dos
processos de dependência (como interações estratégicas, efeitos de
contágio ou
\href{https://pt.wikipedia.org/wiki/Externalidades}{externalidades}) e
na garantia de propriedades assintóticas dos estimadores, especialmente
a consistência sob condições de endogeneidade. Enquanto os modelos CAR
são frequentemente usados para suavização e predição em contextos
Bayesianos hierárquicos, os modelos econométricos (SAR, SEM, SDM, etc.)
são desenhados para testar hipóteses sobre mecanismos de interação e
estimar o impacto de políticas ou choques exógenos em um sistema
interconectado.

A literatura econométrica, consolidada por Anselin (1988) e expandida
expressivamente por J. LeSage e Pace (2009), Elhorst et al. (2014) e H.
Kelejian e Piras (2017), organiza os modelos espaciais com base na
inclusão de três tipos fundamentais de interação entre unidades
espaciais:

\begin{itemize}
\tightlist
\item
  Defasagem espacial na variável dependente (\(\mathbf{W}\mathbf{y}\)):
  Refere-se ao fenômeno em que o resultado observado em uma região ou
  agente é diretamente influenciado pelos resultados observados em
  regiões ou agentes vizinhos. Isto é, o valor da variável de interesse
  na unidade \(i\) (\(y_i\)) é determinado simultaneamente pelos valores
  de \(y\) nas unidades vizinhas (\(\mathbf{W}\mathbf{y}\), ver Elhorst
  (2022)).
\end{itemize}

Exemplo: Competição fiscal entre municípios (a alíquota de imposto de um
município depende da alíquota dos vizinhos para não perder base
tributária) ou efeitos de pares em educação (o desempenho de um aluno
depende do desempenho dos colegas).

\begin{itemize}
\tightlist
\item
  Defasagem espacial nas variáveis explicativas
  (\(\mathbf{W}\mathbf{X}\)): Refere-se à influência que as
  características ou políticas de regiões vizinhas exercem sobre os
  resultados de uma região. É um efeito de transbordamento direto ou
  externalidade, onde o que acontece ao lado importa tanto quanto o que
  acontece aqui (ver Elhorst (2022)). Note que aqui \(\mathbf{X}\) é
  covariável.
\end{itemize}

Exemplo: A criminalidade em um bairro pode ser afetada não apenas pelo
policiamento local, mas também pelo policiamento nos bairros vizinhos
(deslocamento do crime).

\begin{itemize}
\tightlist
\item
  Dependência espacial no termo de erro (\(\mathbf{W}\mathbf{u}\)):
  Refere-se à dependência residual. A correlação espacial surge devido a
  variáveis omitidas que são espacialmente correlacionadas ou a erros de
  medição com padrão espacial. Isto é, mesmo após controlar pelas
  variáveis explicativas observadas (\(\mathbf{X}\)), os resíduos de
  unidades geograficamente próximas podem não ser independentes, podendo
  ser correlacionados. Isso indica que os fatores não observados que
  influenciam a variável resposta \(\mathbf{y}\) possuem, eles próprios,
  uma estrutura espacial.
\end{itemize}

Exemplo: Considere um modelo de preços de imóveis que inclui variáveis
como tamanho, idade, número de quartos e proximidade do centro. Se o
modelo omitir a poluição sonora (ruído de tráfego) um fator que varia
gradualmente no espaço (ruas adjacentes têm níveis similares), os
imóveis em ruas mais barulhentas terão sistematicamente preços abaixo do
previsto pelo modelo, enquanto aqueles em ruas mais silenciosas terão
preços acima do previsto. Como a poluição sonora é espacialmente
correlacionada, os erros do modelo (resíduos) também o serão: resíduos
negativos se agruparão em regiões barulhentas e positivos em regiões
quietas. Essa estrutura espacial nos resíduos é exatamente a dependência
espacial no termo de erro (\(W u\)).

O modelo mais geral que engloba essas três formas de interação é o
modelo geral de aninhamento espacial (GNS - \emph{General Nesting
Spatial Mode}l, Elhorst (2022)), por vezes referido como \emph{Spatial
Autoregressive model with Spatial Autoregressive errors} (SARAR) ou
\emph{Spatial Autoregressive Combined} (SAC) generalizado:

\begin{equation}\phantomsection\label{eq-GNS}{
\begin{aligned}
\mathbf{y} &= \rho \mathbf{W}\mathbf{y} + \mathbf{X}\boldsymbol{\beta} + \mathbf{W}\mathbf{X}\boldsymbol{\theta} + \mathbf{u} \\
\mathbf{u} &= \lambda \mathbf{W}\mathbf{u} + \boldsymbol{\epsilon}
\end{aligned}
}\end{equation}

Onde:

\begin{itemize}
\item
  \(\mathbf{y}\) é um vetor \(N \times 1\) da variável dependente.
\item
  \(\mathbf{X}\) é uma matriz \(N \times K\) de variáveis explicativas
  (covaráveis).
\item
  \(\mathbf{W}\) é a matriz de pesos espaciais (ou matriz de vizinhança)
  \(N \times N\). Pode-se usar matrizes diferentes para cada termo
  (\(\mathbf{W}_1, \mathbf{W}_2\)), mas frequentemente assume-se a mesma
  estrutura por parcimônia.
\item
  \(\rho\) é o coeficiente autorregressivo espacial (intensidade da
  dependência substantiva).
\item
  \(\boldsymbol{\beta}\) é o vetor \(K \times 1\) de parâmetros
  associados às covariáveis.
\item
  \(\boldsymbol{\theta}\) é o vetor \(K \times 1\) de parâmetros
  associados à defasagem espacial.
\item
  \(\lambda\) é o coeficiente de erro autorregressivo espacial.
\item
  \(\boldsymbol{\epsilon} \sim \mathcal{N}(\mathbf{0}, \sigma^2 \mathbf{I}_n)\)
  é o vetor de erros, tipicamente assumido i.i.d. com média zero e
  variância constante \(\sigma^2\) (embora métodos modernos relaxem a
  homocedasticidade).
\end{itemize}

A partir desta estrutura geral, impõem-se restrições aos parâmetros
(\(\rho=0\), \(\lambda=0\), ou \(\boldsymbol{\theta}=\mathbf{0}\)) para
derivar os modelos específicos mais utilizados na prática.

\subsection{\texorpdfstring{Modelo de Defasagem Espacial (SAR --
\emph{Spatial Autoregressive
Model})}{Modelo de Defasagem Espacial (SAR -- Spatial Autoregressive Model)}}\label{modelo-de-defasagem-espacial-sar-spatial-autoregressive-model}

O Modelo de Defasagem Espacial (SAR, o mesmo descrito na
Seção~\ref{sec-SAR}), também conhecido como \emph{Spatial Lag Model},
constitui um caso particular do Modelo Geral de Aninhamento Espacial
(GNS, Eq.~\ref{eq-GNS}) obtido ao impor as restrições paramétricas
\(\lambda = 0\) e \(\boldsymbol{\theta} = \mathbf{0}\).

Esta especificação pressupõe que a dependência espacial é um fenômeno
substantivo, resultante da interação direta e simultânea entre as
unidades de observação, onde o resultado de uma localidade é
condicionado pelos resultados de suas vizinhas. Formalmente, o modelo é
definido pela seguinte equação estrutural (Anselin 1988):

\begin{equation}\phantomsection\label{eq-SAR}{
\mathbf{y} = \rho \mathbf{W}\mathbf{y} + \mathbf{X}\boldsymbol{\beta} + \boldsymbol{\epsilon},
}\end{equation}

com erros i.i.d.
\(\boldsymbol{\epsilon} \sim \mathcal{N}(\mathbf{0}, \sigma^2 \mathbf{I}_n)\).
Na sua forma escalar, para uma unidade \(i\), o modelo é expresso como:

\[
y_i = \rho \sum_{j=1}^{n} w_{ij} y_j + \sum_{k=1}^{K} x_{ik}\beta_k + \epsilon_i,
\]

em que \(\rho\) é o coeficiente de autocorrelação espacial, que
quantifica a intensidade da dependência espacial.

A partir da forma reduzida do modelo, obtém-se a distribuição
condicional completa de \(\mathbf{y}\). Assumindo que
\((\mathbf{I}_n - \rho \mathbf{W})\) é não singular (tem inversa),
temos:

\[
\mathbf{y} \, | \, \mathbf{X}, \mathbf{W} \sim \mathcal{N}\left( (\mathbf{I}_n - \rho \mathbf{W})^{-1}\mathbf{X}\boldsymbol{\beta}, \quad \sigma^2 (\mathbf{I}_n - \rho \mathbf{W})^{-1} (\mathbf{I}_n - \rho \mathbf{W})^{-\top} \right).
\]

Esta expressão evidencia que tanto a média condicional quanto a
estrutura de covariância de \(\mathbf{y}\) são globalmente afetadas pela
configuração espacial através da matriz de pesos \(\mathbf{W}\) e do
parâmetro \(\rho\).

A especificação é motivada por processos de difusão, contágio,
competição estratégica, externalidades ou efeitos de aprendizagem entre
agentes ou regiões geograficamente próximas (J. LeSage e Pace 2009).

A natureza simultânea do modelo é explicitada por sua forma reduzida (ou
\href{https://en.wikipedia.org/wiki/Data_generating_process}{Data
Generating Process -- DGP}), obtida ao resolver a equação para
\(\mathbf{y}\):

\[
\begin{aligned}
(\mathbf{I}_n - \rho \mathbf{W})\mathbf{y} &= \mathbf{X}\boldsymbol{\beta} + \boldsymbol{\epsilon}, \\
\mathbf{y} &= (\mathbf{I}_n - \rho \mathbf{W})^{-1}\mathbf{X}\boldsymbol{\beta} + (\mathbf{I}_n - \rho \mathbf{W})^{-1}\boldsymbol{\epsilon}.
\end{aligned}
\]

A matriz
\(\mathbf{S}(\mathbf{W}) = (\mathbf{I}_n - \rho \mathbf{W})^{-1}\) é
denominada multiplicador espacial. Sua expansão em série de potências,

\[
(\mathbf{I}_n - \rho \mathbf{W})^{-1} = \mathbf{I}_n + \rho \mathbf{W} + \rho^2 \mathbf{W}^2 + \rho^3 \mathbf{W}^3 + \cdots,
\]

revela que um choque exógeno (seja via \(\mathbf{X}\) ou
\(\boldsymbol{\epsilon}\)) em uma unidade \(i\) não afeta apenas \(y_i\)
ou seus vizinhos diretos, mas propaga-se por toda a rede através de
efeitos de \emph{feedback} de ordem superior, envolvendo caminhos de
comprimento crescente na estrutura de vizinhança (Elhorst et al. 2014).

Consequentemente, a esperança condicional da variável dependente é
espacialmente estruturada:

\[
\mathbb{E}[\mathbf{y} \, | \, \mathbf{X}, \mathbf{W}] = (\mathbf{I}_n - \rho \mathbf{W})^{-1}\mathbf{X}\boldsymbol{\beta}.
\] A presença do multiplicador espacial implica que a interpretação dos
parâmetros \(\boldsymbol{\beta}\) no modelo SAR difere fundamentalmente
da regressão linear clássica. Estes coeficientes não representam os
efeitos marginais diretos de uma mudança nas variáveis explicativas.
Como qualquer alteração em uma covariável para a unidade \(i\) afeta sua
própria variável dependente e, via interdependência, as das demais
unidades, os efeitos são globais.

J. LeSage e Pace (2009) propõe uma decomposição da matriz de efeitos
totais das variáveis explicativas, derivada da forma reduzida do modelo.
Para uma variável explicativa \(k\), o efeito total de uma mudança
unitária em todo o sistema é dado pela matriz:

\[
\frac{\partial \mathbf{y}}{\partial \mathbf{x}_k'} = \beta_k (\mathbf{I}_n - \rho \mathbf{W})^{-1}.
\]

Os efeitos são então decompostos em:

\begin{itemize}
\item
  Efeito direto médio que captura o impacto médio de uma mudança em
  \(x_{ik}\) sobre o próprio \(y_i\). Este efeito inclui tanto o impacto
  inicial quanto os \emph{feedbacks} espaciais que retornam à unidade de
  origem.
\item
  Efeito indireto médio (ou \emph{Spillover}) que captura o impacto
  médio da mudança em \(x_{ik}\) sobre todas as outras unidades \(y_j\)
  (\(j \neq i\)).
\end{itemize}

A presença da variável dependente defasada espacialmente
(\(\mathbf{W}\mathbf{y}\)) no lado direito da equação gera um problema
de simultaneidade: como \(y_i\) depende de \(y_j\) e vice-versa, o termo
\(\mathbf{W}\mathbf{y}\) está correlacionado com o termo de erro
\(\boldsymbol{\epsilon}\). Isto implica que o estimador de Mínimos
Quadrados Ordinários (MQO) é viesado e inconsistente
(\(plim \, \hat{\rho}_{MQO} \neq \rho\)), tendendo a superestimar a
dependência espacial e enviesar os coeficientes \(\boldsymbol{\beta}\)
(Anselin 1988).

As abordagens mais comuns para a estimação consistente e eficiente são:

\begin{enumerate}
\def\labelenumi{\arabic{enumi}.}
\item
  máxima verossimilhança: Maximiza a função de verossimilhança que
  incorpora o log-determinante do Jacobiano da transformação,
  \(\ln|\mathbf{I}_n - \rho \mathbf{W}|\), para corrigir a
  simultaneidade.
\item
  Método dos Momentos Generalizados (GMM) / Variáveis Instrumentais
  (VI): Utiliza defasagens espaciais das covariáveis
  (\(\mathbf{W}\mathbf{X}, \mathbf{W}^2\mathbf{X}, \dots\)) como
  instrumentos válidos para \(\mathbf{W}\mathbf{y}\), uma estratégia
  robusta proposta por (H. H. Kelejian e Prucha 1998) .
\end{enumerate}

\begin{table}

\caption{\label{tbl-sar_estimacao}Resultados da Estimação do Modelo SAR
e Decomposição dos Efeitos. Estimativas {[}intervalo de confiança{]} e
abaixo o respetivo erro padrão.}

\centering{

\begin{Shaded}
\begin{Highlighting}[]
\ControlFlowTok{if}\NormalTok{ (}\SpecialCharTok{!}\FunctionTok{require}\NormalTok{(}\StringTok{"pacman"}\NormalTok{)) }\FunctionTok{install.packages}\NormalTok{(}\StringTok{"pacman"}\NormalTok{)}
\NormalTok{pacman}\SpecialCharTok{::}\FunctionTok{p\_load}\NormalTok{(spatialreg, spdep, sf, texreg, knitr, kableExtra, dplyr)}

\CommentTok{\# Preparação dos Dados}
\ControlFlowTok{if}\NormalTok{ (}\SpecialCharTok{!}\FunctionTok{exists}\NormalTok{(}\StringTok{"mg\_dados"}\NormalTok{)) \{}
\NormalTok{  mg\_dados }\OtherTok{\textless{}{-}}\NormalTok{ geobr}\SpecialCharTok{::}\FunctionTok{read\_municipality}\NormalTok{(}\AttributeTok{code\_muni =} \StringTok{"MG"}\NormalTok{, }\AttributeTok{year =} \DecValTok{2020}\NormalTok{, }\AttributeTok{showProgress =} \ConstantTok{FALSE}\NormalTok{)}
\NormalTok{  coords }\OtherTok{\textless{}{-}} \FunctionTok{st\_coordinates}\NormalTok{(}\FunctionTok{st\_centroid}\NormalTok{(mg\_dados))}
  \FunctionTok{set.seed}\NormalTok{(}\DecValTok{123}\NormalTok{)}
\NormalTok{  mg\_dados}\SpecialCharTok{$}\NormalTok{taxa\_bruta }\OtherTok{\textless{}{-}}\NormalTok{ (}\SpecialCharTok{{-}}\NormalTok{coords[,}\DecValTok{2}\NormalTok{] }\SpecialCharTok{*} \DecValTok{10}\NormalTok{) }\SpecialCharTok{+} \FunctionTok{rnorm}\NormalTok{(}\FunctionTok{nrow}\NormalTok{(mg\_dados), }\DecValTok{0}\NormalTok{, }\DecValTok{5}\NormalTok{)}
\NormalTok{  mg\_dados}\SpecialCharTok{$}\NormalTok{variavel\_x }\OtherTok{\textless{}{-}} \FunctionTok{rnorm}\NormalTok{(}\FunctionTok{nrow}\NormalTok{(mg\_dados))}
\NormalTok{\}}

\CommentTok{\# Matriz de Pesos Espaciais (Padronizada por linha \textquotesingle{}W\textquotesingle{} é o padrão para SAR)}
\NormalTok{nb }\OtherTok{\textless{}{-}} \FunctionTok{poly2nb}\NormalTok{(mg\_dados, }\AttributeTok{queen =} \ConstantTok{TRUE}\NormalTok{)}
\NormalTok{lw }\OtherTok{\textless{}{-}} \FunctionTok{nb2listw}\NormalTok{(nb, }\AttributeTok{style =} \StringTok{"W"}\NormalTok{, }\AttributeTok{zero.policy =} \ConstantTok{TRUE}\NormalTok{)}

\CommentTok{\# Ajuste do Modelo}
\CommentTok{\# A) OLS (Referência)}
\NormalTok{mod\_ols }\OtherTok{\textless{}{-}} \FunctionTok{lm}\NormalTok{(taxa\_bruta }\SpecialCharTok{\textasciitilde{}}\NormalTok{ variavel\_x, }\AttributeTok{data =}\NormalTok{ mg\_dados)}

\CommentTok{\# SAR (Spatial Autoregressive Model) {-} Máxima Verossimilhança}
\CommentTok{\# A função lagsarlm ajusta do modelo SAR via ML}
\NormalTok{mod\_sar }\OtherTok{\textless{}{-}} \FunctionTok{lagsarlm}\NormalTok{(taxa\_bruta }\SpecialCharTok{\textasciitilde{}}\NormalTok{ variavel\_x, }
                    \AttributeTok{data =}\NormalTok{ mg\_dados, }
                    \AttributeTok{listw =}\NormalTok{ lw)}

\CommentTok{\#}
\NormalTok{mapa\_vars }\OtherTok{\textless{}{-}} \FunctionTok{c}\NormalTok{(}
  \StringTok{"(Intercept)"} \OtherTok{=} \StringTok{"Intercepto"}\NormalTok{,}
  \StringTok{"variavel\_x"}  \OtherTok{=} \StringTok{"Variável X"}\NormalTok{,}
  \StringTok{"rho"}         \OtherTok{=} \StringTok{"$}\SpecialCharTok{\textbackslash{}\textbackslash{}}\StringTok{rho$ (Dependência Espacial)"}
\NormalTok{)}

\CommentTok{\#}
\NormalTok{mapa\_gof }\OtherTok{\textless{}{-}} \FunctionTok{list}\NormalTok{(}
  \FunctionTok{list}\NormalTok{(}\StringTok{"raw"} \OtherTok{=} \StringTok{"nobs"}\NormalTok{, }\StringTok{"clean"} \OtherTok{=} \StringTok{"N"}\NormalTok{, }\StringTok{"fmt"} \OtherTok{=} \DecValTok{0}\NormalTok{),}
  \FunctionTok{list}\NormalTok{(}\StringTok{"raw"} \OtherTok{=} \StringTok{"r.squared"}\NormalTok{, }\StringTok{"clean"} \OtherTok{=} \StringTok{"$}\SpecialCharTok{\textbackslash{}\textbackslash{}}\StringTok{ R\^{}2$"}\NormalTok{, }\StringTok{"fmt"} \OtherTok{=} \DecValTok{3}\NormalTok{),}
  \FunctionTok{list}\NormalTok{(}\StringTok{"raw"} \OtherTok{=} \StringTok{"aic"}\NormalTok{, }\StringTok{"clean"} \OtherTok{=} \StringTok{"AIC"}\NormalTok{, }\StringTok{"fmt"} \OtherTok{=} \DecValTok{1}\NormalTok{),}
  \FunctionTok{list}\NormalTok{(}\StringTok{"raw"} \OtherTok{=} \StringTok{"logLik"}\NormalTok{, }\StringTok{"clean"} \OtherTok{=} \StringTok{"Log Likelihood"}\NormalTok{, }\StringTok{"fmt"} \OtherTok{=} \DecValTok{1}\NormalTok{)}
\NormalTok{)}
\CommentTok{\# Gerar a Tabela Principal}
\FunctionTok{modelsummary}\NormalTok{(}
  \FunctionTok{list}\NormalTok{(}\StringTok{"OLS (Clássico)"} \OtherTok{=}\NormalTok{ mod\_ols, }\StringTok{"SAR (Lag Espacial)"} \OtherTok{=}\NormalTok{ mod\_sar),}
  \AttributeTok{coef\_map =}\NormalTok{ mapa\_vars,      }
  \AttributeTok{gof\_map =}\NormalTok{ mapa\_gof,      }
  \AttributeTok{estimate=}\StringTok{"\{estimate\} [\{conf.low\}, \{conf.high\}]"}\NormalTok{,}
  \AttributeTok{stars =} \FunctionTok{c}\NormalTok{(}\StringTok{\textquotesingle{}*\textquotesingle{}} \OtherTok{=}\NormalTok{ .}\DecValTok{05}\NormalTok{, }\StringTok{\textquotesingle{}**\textquotesingle{}} \OtherTok{=}\NormalTok{ .}\DecValTok{01}\NormalTok{, }\StringTok{\textquotesingle{}***\textquotesingle{}} \OtherTok{=}\NormalTok{ .}\DecValTok{001}\NormalTok{),}
  \AttributeTok{title =} \ConstantTok{NULL}\NormalTok{, }
  \AttributeTok{output =} \StringTok{"kableExtra"}\NormalTok{, }
  \AttributeTok{escape =} \ConstantTok{FALSE}
\NormalTok{) }\SpecialCharTok{\%\textgreater{}\%}
  \FunctionTok{kable\_styling}\NormalTok{(}\AttributeTok{latex\_options =} \FunctionTok{c}\NormalTok{(}\StringTok{"HOLD\_position"}\NormalTok{), }
                \AttributeTok{full\_width =} \ConstantTok{FALSE}\NormalTok{, }
                \AttributeTok{position =} \StringTok{"center"}\NormalTok{) }\SpecialCharTok{\%\textgreater{}\%}
  \FunctionTok{row\_spec}\NormalTok{(}\DecValTok{5}\NormalTok{, }\AttributeTok{bold =} \ConstantTok{TRUE}\NormalTok{) }\SpecialCharTok{\%\textgreater{}\%} 
  \FunctionTok{as.character}\NormalTok{() }\SpecialCharTok{\%\textgreater{}\%}
  \FunctionTok{cat}\NormalTok{()}
\end{Highlighting}
\end{Shaded}

\centering\centering
\begin{tabular}[t]{lcc}
\toprule
  & OLS (Clássico) & SAR (Lag Espacial)\\
\midrule
Intercepto & \num{8.973} [\num{8.616}, \num{9.330}] & \num{5.751} [\num{4.876}, \num{6.626}]\\
 & (\num{0.182}) & (\num{0.446})\\
Variável X & \num{0.058} [\num{-0.306}, \num{0.422}] & \num{0.029} [\num{-0.316}, \num{0.374}]\\
 & (\num{0.185}) & (\num{0.176})\\
\textbf{$\rho$ (Dependência Espacial)} & \textbf{} & \textbf{\num{0.359} [\num{0.269}, \num{0.449}]}\\
 &  & (\num{0.046})\\
\midrule
N & \num{853} & \\
$\ R^2$ & \num{0.000} & \\
AIC & \num{5275.0} & \num{5212.9}\\
Log Likelihood & \num{-2634.5} & \\
\bottomrule
\multicolumn{3}{l}{\rule{0pt}{1em}* p $<$ 0.05, ** p $<$ 0.01, *** p $<$ 0.001}\\
\end{tabular}

}

\end{table}%

\textbf{Interpretação}

A Tabela~\ref{tbl-sar_estimacao} para a comparação de modelos evidencia
a superioridade de ajuste do modelo de SAR em detrimento do linear
(OLS), como observa-se pela redução substancial no Critério de
Informação de Akaike (AIC), que decresceu de \(5275.0\) no modelo
clássico para \(5212.9\) na especificação espacial. Este ganho de ajuste
é atribuído à significância do parâmetro de autoregressão espacial
\(\rho\) (\(0.359\); \(IC_{95\%} [0.269, 0.449]\)), que confirma a
existência de dependência espacial positiva na variável resposta.

Não obstante a melhor adequação da estrutura espacial, a
Tabela~\ref{tbl-sar_estimacao} de decomposição dos impactos demonstra
que a covariável \(X\) não possui significância estatística. A análise
dos efeitos marginais revela que a variável explicativa não exerce
influência sobre o desfecho, apresentando valores-p não significativos
tanto para o efeito direto (\(0.030\); \(p=0.829\)), que mensura o
impacto local, quanto para o efeito indireto (\(0.015\); \(p=0.837\)),
que mensura o transbordamento espacial. Conclui-se, portanto, que embora
a modelagem da dependência espacial seja necessária para a correção do
viés de especificação, a variável \(X\) não é um determinante
estatisticamente relevante do fenômeno estudado, apresentando um efeito
total nulo (\(0.045\); \(p=0.831\)).

Você poderia gerar formatação em latex e colar no seu
\href{https://en.wikipedia.org/wiki/Overleaf}{overleaf} ou latex.

\begin{table}

\caption{\label{tbl-sar_latex}}

\centering{

\begin{Shaded}
\begin{Highlighting}[]
\FunctionTok{texreg}\NormalTok{(}
  \FunctionTok{list}\NormalTok{(mod\_ols, mod\_sar),}
  \AttributeTok{custom.model.names =} \FunctionTok{c}\NormalTok{(}\StringTok{"OLS (Clássico)"}\NormalTok{, }\StringTok{"SAR (Lag Espacial)"}\NormalTok{),}
  \AttributeTok{custom.coef.names =} \FunctionTok{c}\NormalTok{(}\StringTok{"Intercepto"}\NormalTok{, }\StringTok{"Variável X"}\NormalTok{, }\StringTok{"$}\SpecialCharTok{\textbackslash{}\textbackslash{}}\StringTok{rho$ (Autocorrelação)"}\NormalTok{),}
  \AttributeTok{caption =} \StringTok{"Comparação de Modelos: OLS vs SAR"}\NormalTok{,}
  \AttributeTok{digits =} \DecValTok{3}\NormalTok{,}
  \AttributeTok{booktabs =} \ConstantTok{TRUE}\NormalTok{, }
  \AttributeTok{dcolumn =} \ConstantTok{TRUE}
\NormalTok{)}
\end{Highlighting}
\end{Shaded}

\usepackage{booktabs}
\usepackage{dcolumn}

\begin{center}
\begin{tabular}{l D{.}{.}{3.6} D{.}{.}{5.6}}
\toprule
 & \multicolumn{1}{c}{OLS (Clássico)} & \multicolumn{1}{c}{SAR (Lag Espacial)} \\
\midrule
Intercepto              & 8.973^{***} & 5.751^{***} \\
                        & (0.182)     & (0.446)     \\
Variável X              & 0.058       & 0.029       \\
                        & (0.185)     & (0.176)     \\
$\rho$ (Autocorrelação) &             & 0.359^{***} \\
                        &             & (0.046)     \\
\midrule
R$^2$                   & 0.000       &             \\
Adj. R$^2$              & -0.001      &             \\
Num. obs.               & 853         & 853         \\
Parameters              &             & 4           \\
Log Likelihood          &             & -2602.463   \\
AIC (Linear model)      &             & 5275.028    \\
AIC (Spatial model)     &             & 5212.926    \\
LR test: statistic      &             & 64.102      \\
LR test: p-value        &             & 0.000       \\
\bottomrule
\multicolumn{3}{l}{\scriptsize{$^{***}p<0.001$; $^{**}p<0.01$; $^{*}p<0.05$}}
\end{tabular}
\caption{Comparação de Modelos: OLS vs SAR}
\label{table:coefficients}
\end{center}

}

\end{table}%

\begin{Shaded}
\begin{Highlighting}[]
\ControlFlowTok{if}\NormalTok{ (}\SpecialCharTok{!}\FunctionTok{require}\NormalTok{(}\StringTok{"pacman"}\NormalTok{)) }\FunctionTok{install.packages}\NormalTok{(}\StringTok{"pacman"}\NormalTok{)}
\NormalTok{pacman}\SpecialCharTok{::}\FunctionTok{p\_load}\NormalTok{(spatialreg, spdep, sf, ggplot2, dplyr, tidyr, patchwork, viridis, Matrix)}

\CommentTok{\# }
\ControlFlowTok{if}\NormalTok{ (}\SpecialCharTok{!}\FunctionTok{exists}\NormalTok{(}\StringTok{"mod\_sar"}\NormalTok{) }\SpecialCharTok{||} \SpecialCharTok{!}\FunctionTok{exists}\NormalTok{(}\StringTok{"mg\_dados"}\NormalTok{)) \{}
\NormalTok{  mg\_dados }\OtherTok{\textless{}{-}}\NormalTok{ geobr}\SpecialCharTok{::}\FunctionTok{read\_municipality}\NormalTok{(}\AttributeTok{code\_muni =} \StringTok{"MG"}\NormalTok{, }\AttributeTok{year =} \DecValTok{2020}\NormalTok{, }\AttributeTok{showProgress =} \ConstantTok{FALSE}\NormalTok{)}
\NormalTok{  coords }\OtherTok{\textless{}{-}} \FunctionTok{st\_coordinates}\NormalTok{(}\FunctionTok{st\_centroid}\NormalTok{(mg\_dados))}
  \FunctionTok{set.seed}\NormalTok{(}\DecValTok{123}\NormalTok{)}
\NormalTok{  mg\_dados}\SpecialCharTok{$}\NormalTok{taxa\_bruta }\OtherTok{\textless{}{-}}\NormalTok{ (}\SpecialCharTok{{-}}\NormalTok{coords[,}\DecValTok{2}\NormalTok{] }\SpecialCharTok{*} \DecValTok{10}\NormalTok{) }\SpecialCharTok{+} \FunctionTok{rnorm}\NormalTok{(}\FunctionTok{nrow}\NormalTok{(mg\_dados), }\DecValTok{0}\NormalTok{, }\DecValTok{5}\NormalTok{)}
\NormalTok{  mg\_dados}\SpecialCharTok{$}\NormalTok{variavel\_x }\OtherTok{\textless{}{-}} \FunctionTok{rnorm}\NormalTok{(}\FunctionTok{nrow}\NormalTok{(mg\_dados))}
  
\NormalTok{  nb }\OtherTok{\textless{}{-}} \FunctionTok{poly2nb}\NormalTok{(mg\_dados, }\AttributeTok{queen =} \ConstantTok{TRUE}\NormalTok{)}
\NormalTok{  lw }\OtherTok{\textless{}{-}} \FunctionTok{nb2listw}\NormalTok{(nb, }\AttributeTok{style =} \StringTok{"W"}\NormalTok{, }\AttributeTok{zero.policy =} \ConstantTok{TRUE}\NormalTok{)}
\NormalTok{  mod\_sar }\OtherTok{\textless{}{-}} \FunctionTok{lagsarlm}\NormalTok{(taxa\_bruta }\SpecialCharTok{\textasciitilde{}}\NormalTok{ variavel\_x, }\AttributeTok{data =}\NormalTok{ mg\_dados, }\AttributeTok{listw =}\NormalTok{ lw)}
\NormalTok{\}}

\CommentTok{\#Gráfico de Impactos (Direto vs Indireto Unificado)}

\FunctionTok{set.seed}\NormalTok{(}\DecValTok{123}\NormalTok{)}
\NormalTok{imp\_mc\_plot }\OtherTok{\textless{}{-}} \FunctionTok{impacts}\NormalTok{(mod\_sar, }\AttributeTok{listw =}\NormalTok{ lw, }\AttributeTok{R =} \DecValTok{1000}\NormalTok{)}

\ControlFlowTok{if}\NormalTok{ (}\FunctionTok{is.null}\NormalTok{(imp\_mc\_plot}\SpecialCharTok{$}\NormalTok{smat) }\SpecialCharTok{\&\&} \FunctionTok{is.null}\NormalTok{(imp\_mc\_plot}\SpecialCharTok{$}\NormalTok{res)) \{}
\NormalTok{   imp\_mc\_plot }\OtherTok{\textless{}{-}} \FunctionTok{impacts}\NormalTok{(mod\_sar, }\AttributeTok{listw =}\NormalTok{ lw, }\AttributeTok{R =} \DecValTok{1000}\NormalTok{, }\AttributeTok{zstats =} \ConstantTok{TRUE}\NormalTok{)}
\NormalTok{\}}

\CommentTok{\# Dataframe Direto e Indireto}
\NormalTok{df\_impactos }\OtherTok{\textless{}{-}} \FunctionTok{data.frame}\NormalTok{(}
  \AttributeTok{direct =}\NormalTok{ imp\_mc\_plot}\SpecialCharTok{$}\NormalTok{res}\SpecialCharTok{$}\NormalTok{direct,}
  \AttributeTok{indirect =}\NormalTok{ imp\_mc\_plot}\SpecialCharTok{$}\NormalTok{res}\SpecialCharTok{$}\NormalTok{indirect}
\NormalTok{) }\SpecialCharTok{\%\textgreater{}\%}
  \FunctionTok{pivot\_longer}\NormalTok{(}\AttributeTok{cols =} \FunctionTok{everything}\NormalTok{(), }\AttributeTok{names\_to =} \StringTok{"Tipo"}\NormalTok{, }\AttributeTok{values\_to =} \StringTok{"Valor"}\NormalTok{) }\SpecialCharTok{\%\textgreater{}\%}
  \FunctionTok{mutate}\NormalTok{(}\AttributeTok{Tipo =} \FunctionTok{factor}\NormalTok{(Tipo, }\AttributeTok{levels =} \FunctionTok{c}\NormalTok{(}\StringTok{"direct"}\NormalTok{, }\StringTok{"indirect"}\NormalTok{),}
                       \AttributeTok{labels =} \FunctionTok{c}\NormalTok{(}\StringTok{"Direto"}\NormalTok{, }\StringTok{"Indireto (Spillover)"}\NormalTok{)))}

\CommentTok{\# }
\NormalTok{g\_impactos }\OtherTok{\textless{}{-}} \FunctionTok{ggplot}\NormalTok{(df\_impactos, }\FunctionTok{aes}\NormalTok{(}\AttributeTok{x =}\NormalTok{ Tipo, }\AttributeTok{fill =}\NormalTok{ Tipo, }\AttributeTok{y=}\NormalTok{Valor)) }\SpecialCharTok{+}
  \FunctionTok{geom\_col}\NormalTok{(}\AttributeTok{width =} \FloatTok{0.2}\NormalTok{, }\AttributeTok{color =} \StringTok{"gray30"}\NormalTok{) }\SpecialCharTok{+}
  \FunctionTok{scale\_fill\_manual}\NormalTok{(}\AttributeTok{values =} \FunctionTok{c}\NormalTok{(}\StringTok{"Direto"} \OtherTok{=} \StringTok{"\#1b9e77"}\NormalTok{, }\StringTok{"Indireto (Spillover)"} \OtherTok{=} \StringTok{"\#d95f02"}\NormalTok{)) }\SpecialCharTok{+}
  \FunctionTok{labs}\NormalTok{(}\AttributeTok{title =} \StringTok{"A. Impactos: Direto vs. Indireto"}\NormalTok{, }
       \AttributeTok{y =} \StringTok{"Magnitude do efeito"}\NormalTok{, }\AttributeTok{x =} \ConstantTok{NULL}\NormalTok{) }\SpecialCharTok{+}
  \FunctionTok{theme\_minimal}\NormalTok{() }\SpecialCharTok{+} 
  \FunctionTok{theme}\NormalTok{(}\AttributeTok{legend.position =} \StringTok{"none"}\NormalTok{, }
        \AttributeTok{legend.title =} \FunctionTok{element\_blank}\NormalTok{())}

\CommentTok{\# Mapa dos Valores Ajustados}

\NormalTok{mg\_dados}\SpecialCharTok{$}\NormalTok{fitted\_sar }\OtherTok{\textless{}{-}} \FunctionTok{fitted}\NormalTok{(mod\_sar)}

\NormalTok{g\_fit }\OtherTok{\textless{}{-}} \FunctionTok{ggplot}\NormalTok{(mg\_dados) }\SpecialCharTok{+}
  \FunctionTok{geom\_sf}\NormalTok{(}\FunctionTok{aes}\NormalTok{(}\AttributeTok{fill =}\NormalTok{ fitted\_sar), }\AttributeTok{color =} \ConstantTok{NA}\NormalTok{) }\SpecialCharTok{+}
  \FunctionTok{scale\_fill\_viridis\_c}\NormalTok{(}\AttributeTok{option =} \StringTok{"turbo"}\NormalTok{, }\AttributeTok{name =} \StringTok{"Predito"}\NormalTok{) }\SpecialCharTok{+}
  \FunctionTok{labs}\NormalTok{(}\AttributeTok{title =} \StringTok{"B. Valores Preditos (SAR)"}\NormalTok{, }
       \AttributeTok{subtitle =} \StringTok{"Padrão espacial recuperado"}\NormalTok{)}\SpecialCharTok{+}
  \FunctionTok{theme\_minimal}\NormalTok{() }\SpecialCharTok{+} 
  
  \FunctionTok{annotation\_scale}\NormalTok{(}
    \AttributeTok{location =} \StringTok{"bl"}\NormalTok{,           }
    \AttributeTok{width\_hint =} \FloatTok{0.3}\NormalTok{,          }
    \AttributeTok{bar\_cols =} \FunctionTok{c}\NormalTok{(}\StringTok{"black"}\NormalTok{, }\StringTok{"white"}\NormalTok{), }
    \AttributeTok{text\_family =} \StringTok{"sans"}       
\NormalTok{  ) }\SpecialCharTok{+}
  
  \FunctionTok{annotation\_north\_arrow}\NormalTok{(}
    \AttributeTok{location =} \StringTok{"tl"}\NormalTok{,           }
    \AttributeTok{which\_north =} \StringTok{"true"}\NormalTok{,      }
    \AttributeTok{pad\_x =} \FunctionTok{unit}\NormalTok{(}\FloatTok{0.2}\NormalTok{, }\StringTok{"in"}\NormalTok{),   }
    \AttributeTok{pad\_y =} \FunctionTok{unit}\NormalTok{(}\FloatTok{0.2}\NormalTok{, }\StringTok{"in"}\NormalTok{),   }
    \AttributeTok{style =}\NormalTok{ north\_arrow\_fancy\_orienteering }
\NormalTok{  )}


\CommentTok{\# Diagnóstico dos Resíduos}
\NormalTok{mg\_dados}\SpecialCharTok{$}\NormalTok{resid\_sar }\OtherTok{\textless{}{-}} \FunctionTok{residuals}\NormalTok{(mod\_sar)}
\NormalTok{moran\_res }\OtherTok{\textless{}{-}} \FunctionTok{moran.test}\NormalTok{(mg\_dados}\SpecialCharTok{$}\NormalTok{resid\_sar, lw)}
\NormalTok{mg\_dados}\SpecialCharTok{$}\NormalTok{resid\_lag }\OtherTok{\textless{}{-}} \FunctionTok{lag.listw}\NormalTok{(lw, mg\_dados}\SpecialCharTok{$}\NormalTok{resid\_sar)}

\NormalTok{g\_resid\_scatter }\OtherTok{\textless{}{-}} \FunctionTok{ggplot}\NormalTok{(mg\_dados, }\FunctionTok{aes}\NormalTok{(}\AttributeTok{x =}\NormalTok{ resid\_sar, }\AttributeTok{y =}\NormalTok{ resid\_lag)) }\SpecialCharTok{+}
  \FunctionTok{geom\_hline}\NormalTok{(}\AttributeTok{yintercept =} \DecValTok{0}\NormalTok{, }\AttributeTok{linetype =} \StringTok{"dashed"}\NormalTok{, }\AttributeTok{color =} \StringTok{"gray"}\NormalTok{) }\SpecialCharTok{+}
  \FunctionTok{geom\_vline}\NormalTok{(}\AttributeTok{xintercept =} \DecValTok{0}\NormalTok{, }\AttributeTok{linetype =} \StringTok{"dashed"}\NormalTok{, }\AttributeTok{color =} \StringTok{"gray"}\NormalTok{) }\SpecialCharTok{+}
  \FunctionTok{geom\_point}\NormalTok{(}\AttributeTok{alpha =} \FloatTok{0.3}\NormalTok{) }\SpecialCharTok{+}
  \FunctionTok{geom\_smooth}\NormalTok{(}\AttributeTok{method =} \StringTok{"lm"}\NormalTok{, }\AttributeTok{se =} \ConstantTok{FALSE}\NormalTok{, }\AttributeTok{color =} \StringTok{"red"}\NormalTok{, }\AttributeTok{size =} \FloatTok{0.8}\NormalTok{) }\SpecialCharTok{+}
  \FunctionTok{labs}\NormalTok{(}\AttributeTok{title =} \StringTok{"C. Scatter de Moran (Resíduos)"}\NormalTok{, }
       \AttributeTok{subtitle =} \FunctionTok{paste0}\NormalTok{(}\StringTok{"I de Moran: "}\NormalTok{, }\FunctionTok{round}\NormalTok{(moran\_res}\SpecialCharTok{$}\NormalTok{estimate[}\DecValTok{1}\NormalTok{], }\DecValTok{3}\NormalTok{), }
                         \StringTok{" (p{-}valor: "}\NormalTok{, }\FunctionTok{round}\NormalTok{(moran\_res}\SpecialCharTok{$}\NormalTok{p.value, }\DecValTok{3}\NormalTok{), }\StringTok{")"}\NormalTok{),}
       \AttributeTok{x =} \StringTok{"Resíduos"}\NormalTok{, }\AttributeTok{y =} \StringTok{"Lag Espacial"}\NormalTok{) }\SpecialCharTok{+}
  \FunctionTok{theme\_minimal}\NormalTok{()}


\CommentTok{\# }
\NormalTok{(g\_impactos }\SpecialCharTok{|}\NormalTok{ g\_fit }\SpecialCharTok{|}\NormalTok{ g\_resid\_scatter)}
\end{Highlighting}
\end{Shaded}

\begin{figure}[H]

\centering{

\pandocbounded{\includegraphics[keepaspectratio]{lattice_data_files/figure-pdf/fig-sar-diagnostico-comparativo-1.pdf}}

}

\caption{\label{fig-sar-diagnostico-comparativo}Diagnóstico SAR: (A)
Comparação Direto vs Indireto, (B) Ajuste do Modelo e (C) Análise de
Resíduos.}

\end{figure}%

\textbf{Interpretação}

O mapa de valores preditos em
Figura~\ref{fig-sar-diagnostico-comparativo} (B) mostra que o modelo foi
capaz de recuperar a estrutura espacial do processo, reproduzindo a
heterogeneidade regional e os padrões de aglomeração observados na
variável dependente, uma contribuição atribuível quase exclusivamente ao
termo de autoregressão espacial (\(\rho\)) dada a nulidade da covariável
\(X\). A validade dessa especificação é confirmada na
Figura~\ref{fig-sar-diagnostico-comparativo} (C), onde o grafico de
dispersão para os resíduos de Moran não mostra nenhuma tendência. A
estatística de Moran (\(I = -0.015\)) com valor-p não significativo
(\(0.735\)) atesta a independência espacial dos resíduos.

\subsection{\texorpdfstring{Modelo de Erro Espacial (SEM - \emph{Spatial
Error
Model})}{Modelo de Erro Espacial (SEM - Spatial Error Model)}}\label{modelo-de-erro-espacial-sem---spatial-error-model}

O Modelo de Erro Espacial (SEM) é um caso particular do Modelo Espacial
Geral (GNS, Eq.~\ref{eq-GNS}) obtido ao se impor as restrições
paramétricas \(\rho = 0\) e \(\boldsymbol{\theta} = \mathbf{0}\). Esta
especificação pressupõe a ausência de interações espaciais diretas tanto
na variável dependente quanto nas variáveis explicativas, concentrando
toda a estrutura de dependência espacial no termo de erro estocástico.
Formalmente, o modelo é definido pelo sistema de equações (Anselin
1988):

\begin{equation}\phantomsection\label{eq-SEM}{
\mathbf{y} = \mathbf{X}\boldsymbol{\beta} + \mathbf{u}, \qquad 
\mathbf{u} = \lambda \mathbf{W}\mathbf{u} + \boldsymbol{\epsilon},
}\end{equation}

com inovações i.i.d.
\(\boldsymbol{\epsilon} \sim \mathcal{N}(\mathbf{0}, \sigma^2 \mathbf{I}_n)\).
A distribuição condicional de \(\mathbf{y}\) é, portanto, uma normal
multivariada:

\[
\mathbf{y} \, | \, \mathbf{X}, \mathbf{W} \sim \mathcal{N}\left( \mathbf{X}\boldsymbol{\beta}, \;
\sigma^2 (\mathbf{I}_n - \lambda \mathbf{W})^{-1} (\mathbf{I}_n - \lambda \mathbf{W})^{-\top} \right).
\]

Na forma escalar, para uma unidade de observação \(i\), o modelo se
expressa como:

\[
y_i = \sum_{k=1}^K x_{ik}\beta_k + u_i, \qquad 
u_i = \lambda \sum_{j=1}^N w_{ij} u_j + \epsilon_i, \qquad 
\epsilon_i \sim \mathcal{N}(0, \sigma^2).
\]

Conforme destacado por J. LeSage e Pace (2009), a utilidade do SEM
reside no tratamento da dependência espacial como um fenômeno residual,
frequentemente resultante de variáveis omitidas com padrão espacial ou
de erros de medição decorrentes de delimitações administrativas
arbitrárias que não coincidem com os verdadeiros limites econômicos.
Diferentemente dos modelos SAR (Eq.~\ref{eq-SAR}) e SDM
(Eq.~\ref{eq-SDM}), que capturam interações espaciais substantivas, a
estrutura do SEM implica que a dependência espacial é um ruído a ser
controlado para inferência válida.

A forma reduzida do processo gerador de dados é obtida substituindo a
equação do erro na equação principal:

\[
\mathbf{y} = \mathbf{X}\boldsymbol{\beta} + (\mathbf{I}_n - \lambda \mathbf{W})^{-1}\boldsymbol{\epsilon},
\]

o que revela que erros (choques) aleatórios \(\boldsymbol{\epsilon}\)
propagam-se globalmente através do multiplicador espacial inverso
\((\mathbf{I}_n - \lambda \mathbf{W})^{-1}\) (desde que este exista).
Contudo, a esperança condicional da variável dependente mantém-se não
espacial:

\[
\mathbb{E}[\mathbf{y} \, | \, \mathbf{X}, \mathbf{W}] = \mathbf{X}\boldsymbol{\beta}.
\]

No SEM, os coeficientes \(\boldsymbol{\beta}\) representam derivadas
parciais verdadeiras, isto é, os efeitos marginais diretos das variáveis
explicativas. Uma variação em \(x_{ik}\) afeta apenas \(y_i\), sem
desencadear \emph{feedback} ou \emph{spillovers} espaciais na média das
unidades vizinhas (Elhorst et al. 2014).

Do ponto de vista da estimação, a imposição incorreta de \(\lambda = 0\)
e o uso de Mínimos Quadrados Ordinários (MQO) não comprometem a
consistência dos estimadores pontuais de \(\boldsymbol{\beta}\), uma vez
que a média dos erros permanece zero e ortogonal aos regressores. No
entanto, essa estratégia produz estimadores ineficientes e, mais
criticamente, enviesa a estimação da matriz de covariância dos
parâmetros, invalidando a inferência baseada em testes \(t\) e \(F\)
padrão (Anselin 1988).

A estimação eficiente exige métodos que incorporem explicitamente a
estrutura não esférica da matriz de covariância dos erros:

\[
\operatorname{Cov}[\mathbf{y} \, | \, \mathbf{X}, \mathbf{W}] = \boldsymbol{\Omega} = 
\sigma^2 \left[ (\mathbf{I}_n - \lambda \mathbf{W})^{\top} (\mathbf{I}_n - \lambda \mathbf{W}) \right]^{-1}.
\]

As abordagens usuais para obter estimadores consistentes e eficientes
são a máxima verossimilhança e o Método dos Momentos Generalizados
(GMM), que permitem a recuperação adequada dos erros-padrão e a
realização de inferência válida (H. H. Kelejian e Prucha 2010).

\begin{table}

\caption{\label{tbl-sem_comparacao}Resultados da Estimação: Comparação
OLS, SAR e SEM. Estimativas {[}intervalo de confiança 95\%{]}.}

\centering{

\begin{Shaded}
\begin{Highlighting}[]
\ControlFlowTok{if}\NormalTok{ (}\SpecialCharTok{!}\FunctionTok{require}\NormalTok{(}\StringTok{"pacman"}\NormalTok{)) }\FunctionTok{install.packages}\NormalTok{(}\StringTok{"pacman"}\NormalTok{)}
\NormalTok{pacman}\SpecialCharTok{::}\FunctionTok{p\_load}\NormalTok{(spatialreg, spdep, sf, modelsummary, kableExtra, dplyr)}

\CommentTok{\# SEM (Spatial Error Model)}
\CommentTok{\# A função errorsarlm ajusta o SEM via Máxima Verossimilhança}
\NormalTok{mod\_sem }\OtherTok{\textless{}{-}} \FunctionTok{errorsarlm}\NormalTok{(taxa\_bruta }\SpecialCharTok{\textasciitilde{}}\NormalTok{ variavel\_x, }\AttributeTok{data =}\NormalTok{ mg\_dados, }\AttributeTok{listw =}\NormalTok{ lw)}

\CommentTok{\#Tabela }
\NormalTok{mapa\_vars }\OtherTok{\textless{}{-}} \FunctionTok{c}\NormalTok{(}
  \StringTok{"(Intercept)"} \OtherTok{=} \StringTok{"Intercepto"}\NormalTok{,}
  \StringTok{"variavel\_x"}  \OtherTok{=} \StringTok{"Variável X"}\NormalTok{,}
  \StringTok{"rho"}         \OtherTok{=} \StringTok{"$}\SpecialCharTok{\textbackslash{}\textbackslash{}}\StringTok{rho$ (Lag Espacial)"}\NormalTok{,  }\CommentTok{\# Parâmetro do SAR}
  \StringTok{"lambda"}      \OtherTok{=} \StringTok{"$}\SpecialCharTok{\textbackslash{}\textbackslash{}}\StringTok{lambda$ (Erro Espacial)"} \CommentTok{\# Parâmetro do SEM}
\NormalTok{)}

\NormalTok{mapa\_gof }\OtherTok{\textless{}{-}} \FunctionTok{list}\NormalTok{(}
  \FunctionTok{list}\NormalTok{(}\StringTok{"raw"} \OtherTok{=} \StringTok{"nobs"}\NormalTok{, }\StringTok{"clean"} \OtherTok{=} \StringTok{"N"}\NormalTok{, }\StringTok{"fmt"} \OtherTok{=} \DecValTok{0}\NormalTok{),}
  \FunctionTok{list}\NormalTok{(}\StringTok{"raw"} \OtherTok{=} \StringTok{"r.squared"}\NormalTok{, }\StringTok{"clean"} \OtherTok{=} \StringTok{"$R\^{}2$"}\NormalTok{, }\StringTok{"fmt"} \OtherTok{=} \DecValTok{3}\NormalTok{),}
  \FunctionTok{list}\NormalTok{(}\StringTok{"raw"} \OtherTok{=} \StringTok{"aic"}\NormalTok{, }\StringTok{"clean"} \OtherTok{=} \StringTok{"AIC"}\NormalTok{, }\StringTok{"fmt"} \OtherTok{=} \DecValTok{1}\NormalTok{),}
  \FunctionTok{list}\NormalTok{(}\StringTok{"raw"} \OtherTok{=} \StringTok{"logLik"}\NormalTok{, }\StringTok{"clean"} \OtherTok{=} \StringTok{"Log Likelihood"}\NormalTok{, }\StringTok{"fmt"} \OtherTok{=} \DecValTok{1}\NormalTok{)}
\NormalTok{)}

\CommentTok{\#Gerar a Tabela Comparativa (OLS, SAR, SEM)}
\FunctionTok{modelsummary}\NormalTok{(}
  \FunctionTok{list}\NormalTok{(}
    \StringTok{"OLS (Clássico)"} \OtherTok{=}\NormalTok{ mod\_ols, }
    \StringTok{"SAR"} \OtherTok{=}\NormalTok{ mod\_sar, }
    \StringTok{"SEM"} \OtherTok{=}\NormalTok{ mod\_sem}
\NormalTok{  ),}
  \AttributeTok{coef\_map =}\NormalTok{ mapa\_vars,      }
  \AttributeTok{gof\_map =}\NormalTok{ mapa\_gof,      }
  \AttributeTok{estimate =} \StringTok{"\{estimate\} [\{conf.low\}, \{conf.high\}]"}\NormalTok{,}
  \AttributeTok{statistic =} \ConstantTok{NULL}\NormalTok{, }
  \AttributeTok{stars =} \FunctionTok{c}\NormalTok{(}\StringTok{\textquotesingle{}*\textquotesingle{}} \OtherTok{=}\NormalTok{ .}\DecValTok{05}\NormalTok{, }\StringTok{\textquotesingle{}**\textquotesingle{}} \OtherTok{=}\NormalTok{ .}\DecValTok{01}\NormalTok{, }\StringTok{\textquotesingle{}***\textquotesingle{}} \OtherTok{=}\NormalTok{ .}\DecValTok{001}\NormalTok{),}
  \AttributeTok{title =} \ConstantTok{NULL}\NormalTok{,     }
  \AttributeTok{output =} \StringTok{"kableExtra"}\NormalTok{, }
  \AttributeTok{escape =} \ConstantTok{FALSE}  
\NormalTok{) }\SpecialCharTok{\%\textgreater{}\%}
  \FunctionTok{kable\_styling}\NormalTok{(}\AttributeTok{latex\_options =} \FunctionTok{c}\NormalTok{(}\StringTok{"HOLD\_position"}\NormalTok{), }
                \AttributeTok{full\_width =} \ConstantTok{FALSE}\NormalTok{, }
                \AttributeTok{position =} \StringTok{"center"}\NormalTok{) }\SpecialCharTok{\%\textgreater{}\%}
  \FunctionTok{row\_spec}\NormalTok{(}\FunctionTok{c}\NormalTok{(}\DecValTok{5}\NormalTok{, }\DecValTok{7}\NormalTok{), }\AttributeTok{bold =} \ConstantTok{TRUE}\NormalTok{) }\SpecialCharTok{\%\textgreater{}\%} 
  \FunctionTok{as.character}\NormalTok{() }\SpecialCharTok{\%\textgreater{}\%}
  \FunctionTok{cat}\NormalTok{()}
\end{Highlighting}
\end{Shaded}

\centering\centering
\begin{tabular}[t]{lccc}
\toprule
  & OLS (Clássico) & SAR & SEM\\
\midrule
Intercepto & \num{8.973} [\num{8.616}, \num{9.330}] & \num{5.751} [\num{4.876}, \num{6.626}] & \num{8.978} [\num{8.449}, \num{9.507}]\\
Variável X & \num{0.058} [\num{-0.306}, \num{0.422}] & \num{0.029} [\num{-0.316}, \num{0.374}] & \num{-0.001} [\num{-0.342}, \num{0.340}]\\
$\rho$ (Lag Espacial) &  & \num{0.359} [\num{0.269}, \num{0.449}] & \\
$\lambda$ (Erro Espacial) &  &  & \num{0.359} [\num{0.270}, \num{0.449}]\\
\midrule
\textbf{N} & \textbf{\num{853}} & \textbf{} & \textbf{}\\
$R^2$ & \num{0.000} &  & \\
\textbf{AIC} & \textbf{\num{5275.0}} & \textbf{\num{5212.9}} & \textbf{\num{5213.0}}\\
Log Likelihood & \num{-2634.5} &  & \\
\bottomrule
\multicolumn{4}{l}{\rule{0pt}{1em}* p $<$ 0.05, ** p $<$ 0.01, *** p $<$ 0.001}\\
\end{tabular}

}

\end{table}%

\textbf{Interprete}

A Tabela~\ref{tbl-sem_comparacao} apresenta a análise comparativa entre
os modelos de regressão linear (OLS), de defasagem espacial (SAR) e de
erro espacial (SEM), evidenciando a superioridade de ajuste das
especificações espaciais sobre a abordagem clássica. Observa-se uma
melhoria substancial na qualidade do ajuste ao incorporar a dependência
espacial, demonstrada pela redução do Critério de Informação de Akaike
(AIC), que declinou de \(5275.0\) no OLS para valores idênticos de
\(5212.9\) no SAR e \(5213.0\) no SEM. Essa equivalência entre os
modelos espaciais é reforçada pela magnitude dos parâmetros estimados:
tanto o coeficiente de defasagem espacial \(\rho\) (\(0.359\);
\(IC_{95\%} [0.269, 0.449]\)) quanto o coeficiente de erro espacial
\(\lambda\) (\(0.359\); \(IC_{95\%} [0.270, 0.449]\)) são
estatisticamente significativos e apresentam estimativas pontuais
coincidentes, indicando que a estrutura de autocorrelação
independentemente da especificação funcional adotada. Adicionalmente, no
modelo SEM o coeficiente estimado permaneceu estatisticamente nulo
(\(-0.001\); \(IC_{95\%} [-0.342, 0.340]\)), confirmando que a
variabilidade do fenômeno é explicada predominantemente pela estrutura
de dependência espacial e não pela variável preditora \(X\).

\begin{Shaded}
\begin{Highlighting}[]
\ControlFlowTok{if}\NormalTok{ (}\SpecialCharTok{!}\FunctionTok{require}\NormalTok{(}\StringTok{"pacman"}\NormalTok{)) }\FunctionTok{install.packages}\NormalTok{(}\StringTok{"pacman"}\NormalTok{)}
\NormalTok{pacman}\SpecialCharTok{::}\FunctionTok{p\_load}\NormalTok{(spatialreg, spdep, sf, ggplot2, dplyr, tidyr, patchwork, viridis, Matrix, ggspatial)}


\ControlFlowTok{if}\NormalTok{ (}\SpecialCharTok{!}\FunctionTok{exists}\NormalTok{(}\StringTok{"mod\_sem"}\NormalTok{) }\SpecialCharTok{||} \SpecialCharTok{!}\FunctionTok{exists}\NormalTok{(}\StringTok{"mg\_dados"}\NormalTok{)) \{}
\NormalTok{  mg\_dados }\OtherTok{\textless{}{-}}\NormalTok{ geobr}\SpecialCharTok{::}\FunctionTok{read\_municipality}\NormalTok{(}\AttributeTok{code\_muni =} \StringTok{"MG"}\NormalTok{, }\AttributeTok{year =} \DecValTok{2020}\NormalTok{, }\AttributeTok{showProgress =} \ConstantTok{FALSE}\NormalTok{)}
\NormalTok{  coords }\OtherTok{\textless{}{-}} \FunctionTok{st\_coordinates}\NormalTok{(}\FunctionTok{st\_centroid}\NormalTok{(mg\_dados))}
  \FunctionTok{set.seed}\NormalTok{(}\DecValTok{123}\NormalTok{)}
\NormalTok{  mg\_dados}\SpecialCharTok{$}\NormalTok{taxa\_bruta }\OtherTok{\textless{}{-}}\NormalTok{ (}\SpecialCharTok{{-}}\NormalTok{coords[,}\DecValTok{2}\NormalTok{] }\SpecialCharTok{*} \DecValTok{10}\NormalTok{) }\SpecialCharTok{+} \FunctionTok{rnorm}\NormalTok{(}\FunctionTok{nrow}\NormalTok{(mg\_dados), }\DecValTok{0}\NormalTok{, }\DecValTok{5}\NormalTok{)}
\NormalTok{  mg\_dados}\SpecialCharTok{$}\NormalTok{variavel\_x }\OtherTok{\textless{}{-}} \FunctionTok{rnorm}\NormalTok{(}\FunctionTok{nrow}\NormalTok{(mg\_dados))}
  
\NormalTok{  nb }\OtherTok{\textless{}{-}} \FunctionTok{poly2nb}\NormalTok{(mg\_dados, }\AttributeTok{queen =} \ConstantTok{TRUE}\NormalTok{)}
\NormalTok{  lw }\OtherTok{\textless{}{-}} \FunctionTok{nb2listw}\NormalTok{(nb, }\AttributeTok{style =} \StringTok{"W"}\NormalTok{, }\AttributeTok{zero.policy =} \ConstantTok{TRUE}\NormalTok{)}

\NormalTok{  mod\_sem }\OtherTok{\textless{}{-}} \FunctionTok{errorsarlm}\NormalTok{(taxa\_bruta }\SpecialCharTok{\textasciitilde{}}\NormalTok{ variavel\_x, }\AttributeTok{data =}\NormalTok{ mg\_dados, }\AttributeTok{listw =}\NormalTok{ lw)}
\NormalTok{\}}



\CommentTok{\# Mapa dos Valores Ajustados }
\NormalTok{mg\_dados}\SpecialCharTok{$}\NormalTok{fitted\_sem }\OtherTok{\textless{}{-}} \FunctionTok{fitted}\NormalTok{(mod\_sem)}

\NormalTok{g\_fit }\OtherTok{\textless{}{-}} \FunctionTok{ggplot}\NormalTok{(mg\_dados) }\SpecialCharTok{+}
  \FunctionTok{geom\_sf}\NormalTok{(}\FunctionTok{aes}\NormalTok{(}\AttributeTok{fill =}\NormalTok{ fitted\_sem), }\AttributeTok{color =} \ConstantTok{NA}\NormalTok{) }\SpecialCharTok{+}
  \FunctionTok{scale\_fill\_viridis\_c}\NormalTok{(}\AttributeTok{option =} \StringTok{"turbo"}\NormalTok{, }\AttributeTok{name =} \StringTok{"Predito"}\NormalTok{) }\SpecialCharTok{+}
  \FunctionTok{labs}\NormalTok{(}\AttributeTok{title =} \StringTok{"A. Valores Preditos (SEM)"}\NormalTok{, }
       \AttributeTok{subtitle =} \StringTok{"Padrão recuperado (Erro Corrigido)"}\NormalTok{) }\SpecialCharTok{+}
  \FunctionTok{theme\_minimal}\NormalTok{() }\SpecialCharTok{+} 
  \FunctionTok{annotation\_scale}\NormalTok{(}
    \AttributeTok{location =} \StringTok{"bl"}\NormalTok{,            }
    \AttributeTok{width\_hint =} \FloatTok{0.3}\NormalTok{,           }
    \AttributeTok{bar\_cols =} \FunctionTok{c}\NormalTok{(}\StringTok{"black"}\NormalTok{, }\StringTok{"white"}\NormalTok{), }
    \AttributeTok{text\_family =} \StringTok{"sans"}        
\NormalTok{  ) }\SpecialCharTok{+}
  \FunctionTok{annotation\_north\_arrow}\NormalTok{(}
    \AttributeTok{location =} \StringTok{"tl"}\NormalTok{,            }
    \AttributeTok{which\_north =} \StringTok{"true"}\NormalTok{,       }
    \AttributeTok{pad\_x =} \FunctionTok{unit}\NormalTok{(}\FloatTok{0.2}\NormalTok{, }\StringTok{"in"}\NormalTok{),    }
    \AttributeTok{pad\_y =} \FunctionTok{unit}\NormalTok{(}\FloatTok{0.2}\NormalTok{, }\StringTok{"in"}\NormalTok{),    }
    \AttributeTok{style =}\NormalTok{ north\_arrow\_fancy\_orienteering }
\NormalTok{  )}


\CommentTok{\# Diagnóstico dos Resíduos}
\NormalTok{mg\_dados}\SpecialCharTok{$}\NormalTok{resid\_sem }\OtherTok{\textless{}{-}} \FunctionTok{residuals}\NormalTok{(mod\_sem)}
\NormalTok{moran\_res }\OtherTok{\textless{}{-}} \FunctionTok{moran.test}\NormalTok{(mg\_dados}\SpecialCharTok{$}\NormalTok{resid\_sem, lw)}
\NormalTok{mg\_dados}\SpecialCharTok{$}\NormalTok{resid\_lag\_sem }\OtherTok{\textless{}{-}} \FunctionTok{lag.listw}\NormalTok{(lw, mg\_dados}\SpecialCharTok{$}\NormalTok{resid\_sem)}

\NormalTok{g\_resid\_scatter }\OtherTok{\textless{}{-}} \FunctionTok{ggplot}\NormalTok{(mg\_dados, }\FunctionTok{aes}\NormalTok{(}\AttributeTok{x =}\NormalTok{ resid\_sem, }\AttributeTok{y =}\NormalTok{ resid\_lag\_sem)) }\SpecialCharTok{+}
  \FunctionTok{geom\_hline}\NormalTok{(}\AttributeTok{yintercept =} \DecValTok{0}\NormalTok{, }\AttributeTok{linetype =} \StringTok{"dashed"}\NormalTok{, }\AttributeTok{color =} \StringTok{"gray"}\NormalTok{) }\SpecialCharTok{+}
  \FunctionTok{geom\_vline}\NormalTok{(}\AttributeTok{xintercept =} \DecValTok{0}\NormalTok{, }\AttributeTok{linetype =} \StringTok{"dashed"}\NormalTok{, }\AttributeTok{color =} \StringTok{"gray"}\NormalTok{) }\SpecialCharTok{+}
  \FunctionTok{geom\_point}\NormalTok{(}\AttributeTok{alpha =} \FloatTok{0.3}\NormalTok{) }\SpecialCharTok{+}
  \FunctionTok{geom\_smooth}\NormalTok{(}\AttributeTok{method =} \StringTok{"lm"}\NormalTok{, }\AttributeTok{se =} \ConstantTok{FALSE}\NormalTok{, }\AttributeTok{color =} \StringTok{"red"}\NormalTok{, }\AttributeTok{size =} \FloatTok{0.8}\NormalTok{) }\SpecialCharTok{+}
  \FunctionTok{labs}\NormalTok{(}\AttributeTok{title =} \StringTok{"B. Scatter de Moran (Resíduos SEM)"}\NormalTok{, }
       \AttributeTok{subtitle =} \FunctionTok{paste0}\NormalTok{(}\StringTok{"I de Moran: "}\NormalTok{, }\FunctionTok{round}\NormalTok{(moran\_res}\SpecialCharTok{$}\NormalTok{estimate[}\DecValTok{1}\NormalTok{], }\DecValTok{3}\NormalTok{), }
                         \StringTok{" (p{-}valor: "}\NormalTok{, }\FunctionTok{round}\NormalTok{(moran\_res}\SpecialCharTok{$}\NormalTok{p.value, }\DecValTok{3}\NormalTok{), }\StringTok{")"}\NormalTok{),}
       \AttributeTok{x =} \StringTok{"Resíduos"}\NormalTok{, }\AttributeTok{y =} \StringTok{"Lag Espacial"}\NormalTok{) }\SpecialCharTok{+}
  \FunctionTok{theme\_minimal}\NormalTok{()}

\NormalTok{(g\_fit }\SpecialCharTok{|}\NormalTok{ g\_resid\_scatter)}
\end{Highlighting}
\end{Shaded}

\begin{figure}[H]

\centering{

\pandocbounded{\includegraphics[keepaspectratio]{lattice_data_files/figure-pdf/fig-sem-diagnostico-comparativo-1.pdf}}

}

\caption{\label{fig-sem-diagnostico-comparativo}Diagnóstico SEM: (A)
Ajuste do Modelo e (B) Análise de Resíduos.}

\end{figure}%

\textbf{Interpretação}

O mapeamento dos valores preditos em
Figura~\ref{fig-sem-diagnostico-comparativo} (A) demonstra a capacidade
do modelo em capturar a variabilidade espacial latente, reproduzindo os
gradientes regionais distintivos do fenômeno através da correção via
termo \(\lambda\). A robustez desta especificação é atestada em
Figura~\ref{fig-sem-diagnostico-comparativo} (B), onde a distribuição
dos resíduos de Moran mostra ausência de associação espacial,
evidenciada pela inclinação nula da reta de regressão e dispersão
isotrópica. A estatística de I de Moran (\(-0.014\)) associada a um
valor-p não significativo (\(0.734\)) confirma que a autocorrelação
espacial foi integralmente absorvida pela estrutura de erro
especificada, garantindo que os resíduos remanescentes se comportem como
ruído branco estocástico.

\subsection{\texorpdfstring{Modelo de Defasagem Espacial de X (SLX --
\emph{Spatial Lag of
X})}{Modelo de Defasagem Espacial de X (SLX -- Spatial Lag of X)}}\label{modelo-de-defasagem-espacial-de-x-slx-spatial-lag-of-x}

O Modelo de Defasagem Espacial de X (SLX) representa a especificação
mais parcimoniosa para incorporação de efeitos de interação espacial.
Diferentemente dos modelos SAR (Eq.~\ref{eq-SAR}) e SEM
(Eq.~\ref{eq-SEM}), que modelam a dependência através da variável
resposta ou do termo de erro, o SLX postula que o resultado de uma
unidade é influenciado tanto por suas características próprias quanto
pelas características observáveis de suas unidades vizinhas.

Na taxonomia do Modelo Geral Espacial (GNS, Eq.~\ref{eq-GNS}), o SLX é
obtido ao impor as restrições paramétricas \(\rho = 0\) e
\(\lambda = 0\). O modelo é especificado pela equação estrutural:

\begin{equation}\phantomsection\label{eq-SLX}{
\mathbf{y} = \mathbf{X}\boldsymbol{\beta} + \mathbf{W}\mathbf{X}\boldsymbol{\theta} + \boldsymbol{\epsilon},
}\end{equation}

com erros independentes e identicamente distribuídos (i.i.d.)
\(\boldsymbol{\epsilon} \sim \mathcal{N}(\mathbf{0}, \sigma^2 \mathbf{I}_n)\).

Para uma unidade de observação \(i\), o modelo expressa-se como:

\[
y_i = \alpha + \sum_{k=1}^K x_{ik}\beta_k + \sum_{k=1}^K \theta_k \left( \sum_{j=1}^n w_{ij} x_{jk} \right) + \epsilon_i.
\]

onde: - \(\mathbf{W}\mathbf{X}\) denota a matriz de defasagens espaciais
das variáveis explicativas, capturando o contexto espacial das
covariáveis.

\begin{itemize}
\item
  \(\boldsymbol{\beta} \in \mathbb{R}^{K}\) é o vetor de parâmetros
  associado aos efeitos diretos das características da própria unidade
  \(i\).
\item
  \(\boldsymbol{\theta} \in \mathbb{R}^{K}\) é o vetor de parâmetros
  associado aos efeitos de transbordamento exógenos (\emph{spillovers}),
  capturando a influência das características das unidades vizinhas.
\end{itemize}

Halleck Vega e Elhorst (2015), posiciona o SLX como um modelo de
referência inicial para modelagem de dados espaciais (o primeiro a ser
testado antes de ir aos mais complexos). Sua justificativa teórica
reside na modelagem de
\href{https://pt.wikipedia.org/wiki/Efeito_transbordamento}{transbordamentos
locais} (ou de primeira ordem), em contraste com a dependência global
induzida pelo multiplicador espacial
\((\mathbf{I}_n - \rho \mathbf{W})^{-1}\) do modelo SAR
(Eq.~\ref{eq-SAR}) (J. LeSage e Pace 2009).

A derivada parcial da esperança condicional
\(\mathbb{E}[\mathbf{y} \, | \, \mathbf{X}, \mathbf{W}]\) em relação à
\(k\)-ésima variável explicativa revela uma separação clara dos efeitos:

\[
\frac{\partial \, \mathbb{E}[\mathbf{y} \, | \, \mathbf{X}, \mathbf{W}]}{\partial \, \mathbf{x}_k'} = \beta_k \mathbf{I}_n + \theta_k \mathbf{W}.
\]

Esta expressão implica:

\begin{enumerate}
\def\labelenumi{\arabic{enumi}.}
\item
  Efeito Direto: \(\frac{\partial y_i}{\partial x_{ik}} = \beta_k\). Uma
  mudança em \(x_{ik}\) afeta \(y_i\) diretamente, sem mecanismos de
  \href{https://pt.wikipedia.org/wiki/Retroalimenta\%C3\%A7\%C3\%A3o}{retroalimentação}
  (\emph{feedback}).
\item
  Efeito Indireto (Spillover Local):
  \(\frac{\partial y_i}{\partial x_{jk}} = \theta_k w_{ij}\). O impacto
  de uma mudança na unidade vizinha \(j\) sobre \(y_i\) é proporcional
  ao peso espacial \(w_{ij}\) e ao parâmetro \(\theta_k\). Se
  \(w_{ij} = 0\), o efeito é nulo, caracterizando a localidade do
  transbordamento.
\end{enumerate}

Uma limitação crítica dos modelos de dependência global, como o SAR, é
que a razão entre efeitos indiretos e diretos é determinada unicamente
pelo parâmetro \(\rho\), sendo idêntica para todas as variáveis
explicativas. O modelo SLX supera esta restrição, permitindo que a
magnitude e até o sinal dos efeitos de transbordamento (\(\theta_k\))
sejam estimados livremente para cada covariável \(k\). Esta
flexibilidade permite testar hipóteses substantivas complexas, como a
coexistência de externalidades positivas para uma variável e negativas
para outra.

Sob os pressupostos clássicos de linearidade, exogeneidade de
\(\mathbf{X}\) e \(\mathbf{W}\mathbf{X}\), e erros esféricos, o modelo
SLX apresenta propriedades estatísticas desejáveis:

\begin{enumerate}
\def\labelenumi{\arabic{enumi}.}
\item
  Dado que todas as variáveis do lado direito da equação (\(\mathbf{X}\)
  e \(\mathbf{W}\mathbf{X}\)) são
  \href{https://pt.wikipedia.org/wiki/Vari\%C3\%A1veis_ex\%C3\%B3genas_e_end\%C3\%B3genas}{exógenas},
  o estimador MQO de \(\boldsymbol{\beta}\) e \(\boldsymbol{\theta}\) é
  não viesado, consistente e eficiente (sob homocedasticidade). Esta é
  uma vantagem operacional, dispensando métodos computacionalmente
  intensivos como máxima verossimilhança (ML) (Anselin 1988).
\item
  O modelo SLX minimiza problemas de multicolinearidade severa que podem
  surgir em modelos mais gerais (como o SDM ou GNS) devido à alta
  correlação entre \(\mathbf{W}\mathbf{y}\) e \(\mathbf{W}\mathbf{X}\).
\item
  Conforme descrito por Halleck Vega e Elhorst (2015), a estrutura do
  SLX simplifica o tratamento estatístico da possível
  \href{https://www.reddit.com/r/econometrics/comments/dpsi3k/what_is_endogeneity/?tl=pt-br}{endogeneidade}
  das variáveis explicativas \(\mathbf{X}\). Métodos padrão de Variáveis
  Instrumentais (VI) ou Mínimos Quadrados em Dois Estágios (MQ2E) podem
  ser aplicados diretamente para instrumentar \(\mathbf{X}\) e
  \(\mathbf{W}\mathbf{X}\), sem a complexidade adicional introduzida
  pela endogeneidade de \(\mathbf{W}\mathbf{y}\) em modelos SAR.
\item
  A simplicidade da função objetivo (soma de quadrados dos resíduos) no
  SLX permite a estimação conjunta dos parâmetros do modelo
  (\(\boldsymbol{\beta}, \boldsymbol{\theta}\)) e de parâmetros que
  definem a matriz de pesos espaciais \(\mathbf{W}\). Por exemplo,
  pode-se especificar \(w_{ij} = d_{ij}^{-\gamma}\) e estimar o
  parâmetro de decaimento \(\gamma\) via Mínimos Quadrados Não Lineares
  (NLS), permitindo que a estrutura de interação espacial seja inferida
  diretamente dos dados.
\end{enumerate}

\begin{table}

\caption{\label{tbl-slx_comparacao}Resultados da Estimação: Comparação
OLS, SLX, SAR e SEM. Estimativas {[}intervalo de confiança 95\%{]}.}

\centering{

\begin{Shaded}
\begin{Highlighting}[]
\ControlFlowTok{if}\NormalTok{ (}\SpecialCharTok{!}\FunctionTok{require}\NormalTok{(}\StringTok{"pacman"}\NormalTok{)) }\FunctionTok{install.packages}\NormalTok{(}\StringTok{"pacman"}\NormalTok{)}
\NormalTok{pacman}\SpecialCharTok{::}\FunctionTok{p\_load}\NormalTok{(spatialreg, spdep, sf, modelsummary, kableExtra, dplyr, ggplot2, patchwork, viridis)}


\ControlFlowTok{if}\NormalTok{ (}\SpecialCharTok{!}\FunctionTok{exists}\NormalTok{(}\StringTok{"mg\_dados"}\NormalTok{)) \{}
\NormalTok{  mg\_dados }\OtherTok{\textless{}{-}}\NormalTok{ geobr}\SpecialCharTok{::}\FunctionTok{read\_municipality}\NormalTok{(}\AttributeTok{code\_muni =} \StringTok{"MG"}\NormalTok{, }\AttributeTok{year =} \DecValTok{2020}\NormalTok{, }\AttributeTok{showProgress =} \ConstantTok{FALSE}\NormalTok{)}
\NormalTok{  coords }\OtherTok{\textless{}{-}} \FunctionTok{st\_coordinates}\NormalTok{(}\FunctionTok{st\_centroid}\NormalTok{(mg\_dados))}
  \FunctionTok{set.seed}\NormalTok{(}\DecValTok{123}\NormalTok{)}
\NormalTok{  mg\_dados}\SpecialCharTok{$}\NormalTok{taxa\_bruta }\OtherTok{\textless{}{-}}\NormalTok{ (}\SpecialCharTok{{-}}\NormalTok{coords[,}\DecValTok{2}\NormalTok{] }\SpecialCharTok{*} \DecValTok{10}\NormalTok{) }\SpecialCharTok{+} \FunctionTok{rnorm}\NormalTok{(}\FunctionTok{nrow}\NormalTok{(mg\_dados), }\DecValTok{0}\NormalTok{, }\DecValTok{5}\NormalTok{)}
\NormalTok{  mg\_dados}\SpecialCharTok{$}\NormalTok{variavel\_x }\OtherTok{\textless{}{-}} \FunctionTok{rnorm}\NormalTok{(}\FunctionTok{nrow}\NormalTok{(mg\_dados))}
\NormalTok{\}}


\CommentTok{\# SLX (Spatial Lag of X)}
\CommentTok{\# lmSLX cria automaticamente as defasagens (lag.variavel\_x)}
\NormalTok{mod\_slx }\OtherTok{\textless{}{-}} \FunctionTok{lmSLX}\NormalTok{(taxa\_bruta }\SpecialCharTok{\textasciitilde{}}\NormalTok{ variavel\_x, }\AttributeTok{data =}\NormalTok{ mg\_dados, }\AttributeTok{listw =}\NormalTok{ lw)}


\NormalTok{mapa\_vars }\OtherTok{\textless{}{-}} \FunctionTok{c}\NormalTok{(}
  \StringTok{"(Intercept)"}    \OtherTok{=} \StringTok{"Intercepto"}\NormalTok{,}
  \StringTok{"variavel\_x"}     \OtherTok{=} \StringTok{"Variável X (Direto)"}\NormalTok{,}
  \StringTok{"lag.variavel\_x"} \OtherTok{=} \StringTok{"WX $}\SpecialCharTok{\textbackslash{}\textbackslash{}}\StringTok{theta$"}\NormalTok{, }\CommentTok{\# SLX}
  \StringTok{"rho"}            \OtherTok{=} \StringTok{"$}\SpecialCharTok{\textbackslash{}\textbackslash{}}\StringTok{rho$ (Lag Espacial)"}\NormalTok{,       }\CommentTok{\# SAR}
  \StringTok{"lambda"}         \OtherTok{=} \StringTok{"$}\SpecialCharTok{\textbackslash{}\textbackslash{}}\StringTok{lambda$ (Erro Espacial)"}    \CommentTok{\# SEM}
\NormalTok{)}

\NormalTok{mapa\_gof }\OtherTok{\textless{}{-}} \FunctionTok{list}\NormalTok{(}
  \FunctionTok{list}\NormalTok{(}\StringTok{"raw"} \OtherTok{=} \StringTok{"nobs"}\NormalTok{, }\StringTok{"clean"} \OtherTok{=} \StringTok{"N"}\NormalTok{, }\StringTok{"fmt"} \OtherTok{=} \DecValTok{0}\NormalTok{),}
  \FunctionTok{list}\NormalTok{(}\StringTok{"raw"} \OtherTok{=} \StringTok{"r.squared"}\NormalTok{, }\StringTok{"clean"} \OtherTok{=} \StringTok{"$}\SpecialCharTok{\textbackslash{}\textbackslash{}}\StringTok{ R\^{}2$"}\NormalTok{, }\StringTok{"fmt"} \OtherTok{=} \DecValTok{3}\NormalTok{),}
  \FunctionTok{list}\NormalTok{(}\StringTok{"raw"} \OtherTok{=} \StringTok{"aic"}\NormalTok{, }\StringTok{"clean"} \OtherTok{=} \StringTok{"AIC"}\NormalTok{, }\StringTok{"fmt"} \OtherTok{=} \DecValTok{1}\NormalTok{),}
  \FunctionTok{list}\NormalTok{(}\StringTok{"raw"} \OtherTok{=} \StringTok{"logLik"}\NormalTok{, }\StringTok{"clean"} \OtherTok{=} \StringTok{"Log Likelihood"}\NormalTok{, }\StringTok{"fmt"} \OtherTok{=} \DecValTok{1}\NormalTok{)}
\NormalTok{)}

\CommentTok{\# }
\FunctionTok{modelsummary}\NormalTok{(}
  \FunctionTok{list}\NormalTok{(}
    \StringTok{"OLS"} \OtherTok{=}\NormalTok{ mod\_ols, }
    \StringTok{"SLX"} \OtherTok{=}\NormalTok{ mod\_slx,}
    \StringTok{"SAR"} \OtherTok{=}\NormalTok{ mod\_sar, }
    \StringTok{"SEM"} \OtherTok{=}\NormalTok{ mod\_sem}
\NormalTok{  ),}
  \AttributeTok{coef\_map =}\NormalTok{ mapa\_vars,      }
  \AttributeTok{gof\_map =}\NormalTok{ mapa\_gof,      }
  \AttributeTok{estimate =} \StringTok{"\{estimate\} [\{conf.low\}, \{conf.high\}]"}\NormalTok{,}
  \AttributeTok{statistic =} \ConstantTok{NULL}\NormalTok{, }
  \AttributeTok{stars =} \FunctionTok{c}\NormalTok{(}\StringTok{\textquotesingle{}*\textquotesingle{}} \OtherTok{=}\NormalTok{ .}\DecValTok{05}\NormalTok{, }\StringTok{\textquotesingle{}**\textquotesingle{}} \OtherTok{=}\NormalTok{ .}\DecValTok{01}\NormalTok{, }\StringTok{\textquotesingle{}***\textquotesingle{}} \OtherTok{=}\NormalTok{ .}\DecValTok{001}\NormalTok{),}
  \AttributeTok{title =} \ConstantTok{NULL}\NormalTok{,     }
  \AttributeTok{output =} \StringTok{"kableExtra"}\NormalTok{,}
  \AttributeTok{escape =} \ConstantTok{FALSE} 
\NormalTok{) }\SpecialCharTok{\%\textgreater{}\%}
  \FunctionTok{kable\_styling}\NormalTok{(}\AttributeTok{latex\_options =} \FunctionTok{c}\NormalTok{(}\StringTok{"HOLD\_position"}\NormalTok{), }
                \AttributeTok{full\_width =} \ConstantTok{FALSE}\NormalTok{, }
                \AttributeTok{position =} \StringTok{"center"}\NormalTok{) }\SpecialCharTok{\%\textgreater{}\%}
  \FunctionTok{row\_spec}\NormalTok{(}\FunctionTok{c}\NormalTok{(}\DecValTok{5}\NormalTok{, }\DecValTok{7}\NormalTok{, }\DecValTok{9}\NormalTok{), }\AttributeTok{bold =} \ConstantTok{TRUE}\NormalTok{) }\SpecialCharTok{\%\textgreater{}\%} 
  \FunctionTok{as.character}\NormalTok{() }\SpecialCharTok{\%\textgreater{}\%}
  \FunctionTok{cat}\NormalTok{()}
\end{Highlighting}
\end{Shaded}

\centering\centering
\begin{tabular}[t]{lcccc}
\toprule
  & OLS & SLX & SAR & SEM\\
\midrule
Intercepto & \num{8.973} [\num{8.616}, \num{9.330}] & \num{8.970} [\num{8.613}, \num{9.327}] & \num{5.751} [\num{4.876}, \num{6.626}] & \num{8.978} [\num{8.449}, \num{9.507}]\\
Variável X (Direto) & \num{0.058} [\num{-0.306}, \num{0.422}] & \num{0.056} [\num{-0.308}, \num{0.420}] & \num{0.029} [\num{-0.316}, \num{0.374}] & \num{-0.001} [\num{-0.342}, \num{0.340}]\\
WX $\theta$ &  & \num{0.397} [\num{-0.396}, \num{1.190}] &  & \\
$\rho$ (Lag Espacial) &  &  & \num{0.359} [\num{0.269}, \num{0.449}] & \\
\textbf{$\lambda$ (Erro Espacial)} & \textbf{} & \textbf{} & \textbf{} & \textbf{\num{0.359} [\num{0.270}, \num{0.449}]}\\
\midrule
N & \num{853} & \num{853} &  & \\
\textbf{$\ R^2$} & \textbf{\num{0.000}} & \textbf{\num{0.001}} & \textbf{} & \textbf{}\\
AIC & \num{5275.0} & \num{5276.1} & \num{5212.9} & \num{5213.0}\\
\textbf{Log Likelihood} & \textbf{\num{-2634.5}} & \textbf{\num{-2634.0}} & \textbf{} & \textbf{}\\
\bottomrule
\multicolumn{5}{l}{\rule{0pt}{1em}* p $<$ 0.05, ** p $<$ 0.01, *** p $<$ 0.001}\\
\end{tabular}

}

\end{table}%

\begin{Shaded}
\begin{Highlighting}[]
\CommentTok{\# Extraindo coeficientes e intervalos de confiança do SLX}
\NormalTok{coefs }\OtherTok{\textless{}{-}} \FunctionTok{coef}\NormalTok{(mod\_slx)}
\NormalTok{cis   }\OtherTok{\textless{}{-}} \FunctionTok{confint}\NormalTok{(mod\_slx)}


\CommentTok{\# Mapa dos Valores Ajustados}
\NormalTok{mg\_dados}\SpecialCharTok{$}\NormalTok{fitted\_slx }\OtherTok{\textless{}{-}} \FunctionTok{fitted}\NormalTok{(mod\_slx)}

\NormalTok{g\_fit }\OtherTok{\textless{}{-}} \FunctionTok{ggplot}\NormalTok{(mg\_dados) }\SpecialCharTok{+}
  \FunctionTok{geom\_sf}\NormalTok{(}\FunctionTok{aes}\NormalTok{(}\AttributeTok{fill =}\NormalTok{ fitted\_slx), }\AttributeTok{color =} \ConstantTok{NA}\NormalTok{) }\SpecialCharTok{+}
  \FunctionTok{scale\_fill\_viridis\_c}\NormalTok{(}\AttributeTok{option =} \StringTok{"turbo"}\NormalTok{, }\AttributeTok{name =} \StringTok{"Predito"}\NormalTok{) }\SpecialCharTok{+}
  \FunctionTok{labs}\NormalTok{(}\AttributeTok{title =} \StringTok{"A. Valores Preditos (SLX)"}\NormalTok{, }
       \AttributeTok{subtitle =} \StringTok{"Ajuste com defasagens de X"}\NormalTok{) }\SpecialCharTok{+}
  \FunctionTok{theme\_minimal}\NormalTok{() }\SpecialCharTok{+} 
  \FunctionTok{annotation\_scale}\NormalTok{(}\AttributeTok{location =} \StringTok{"bl"}\NormalTok{, }\AttributeTok{width\_hint =} \FloatTok{0.3}\NormalTok{) }\SpecialCharTok{+}
  \FunctionTok{annotation\_north\_arrow}\NormalTok{(}\AttributeTok{location =} \StringTok{"tl"}\NormalTok{, }\AttributeTok{style =}\NormalTok{ north\_arrow\_fancy\_orienteering,}
                         \AttributeTok{pad\_x =} \FunctionTok{unit}\NormalTok{(}\FloatTok{0.1}\NormalTok{, }\StringTok{"in"}\NormalTok{), }\AttributeTok{pad\_y =} \FunctionTok{unit}\NormalTok{(}\FloatTok{0.1}\NormalTok{, }\StringTok{"in"}\NormalTok{))}

\CommentTok{\# Diagnóstico dos Resíduos (Scatter de Moran)}
\NormalTok{mg\_dados}\SpecialCharTok{$}\NormalTok{resid\_slx }\OtherTok{\textless{}{-}} \FunctionTok{residuals}\NormalTok{(mod\_slx)}
\NormalTok{moran\_slx }\OtherTok{\textless{}{-}} \FunctionTok{moran.test}\NormalTok{(mg\_dados}\SpecialCharTok{$}\NormalTok{resid\_slx, lw)}
\NormalTok{mg\_dados}\SpecialCharTok{$}\NormalTok{resid\_lag\_slx }\OtherTok{\textless{}{-}} \FunctionTok{lag.listw}\NormalTok{(lw, mg\_dados}\SpecialCharTok{$}\NormalTok{resid\_slx)}

\NormalTok{g\_resid\_scatter }\OtherTok{\textless{}{-}} \FunctionTok{ggplot}\NormalTok{(mg\_dados, }\FunctionTok{aes}\NormalTok{(}\AttributeTok{x =}\NormalTok{ resid\_slx, }\AttributeTok{y =}\NormalTok{ resid\_lag\_slx)) }\SpecialCharTok{+}
  \FunctionTok{geom\_hline}\NormalTok{(}\AttributeTok{yintercept =} \DecValTok{0}\NormalTok{, }\AttributeTok{linetype =} \StringTok{"dashed"}\NormalTok{, }\AttributeTok{color =} \StringTok{"gray"}\NormalTok{) }\SpecialCharTok{+}
  \FunctionTok{geom\_vline}\NormalTok{(}\AttributeTok{xintercept =} \DecValTok{0}\NormalTok{, }\AttributeTok{linetype =} \StringTok{"dashed"}\NormalTok{, }\AttributeTok{color =} \StringTok{"gray"}\NormalTok{) }\SpecialCharTok{+}
  \FunctionTok{geom\_point}\NormalTok{(}\AttributeTok{alpha =} \FloatTok{0.3}\NormalTok{) }\SpecialCharTok{+}
  \FunctionTok{geom\_smooth}\NormalTok{(}\AttributeTok{method =} \StringTok{"lm"}\NormalTok{, }\AttributeTok{se =} \ConstantTok{FALSE}\NormalTok{, }\AttributeTok{color =} \StringTok{"red"}\NormalTok{, }\AttributeTok{size =} \FloatTok{0.8}\NormalTok{) }\SpecialCharTok{+}
  \FunctionTok{labs}\NormalTok{(}\AttributeTok{title =} \StringTok{"B. Scatter de Moran (Resíduos SLX)"}\NormalTok{, }
       \AttributeTok{subtitle =} \FunctionTok{paste0}\NormalTok{(}\StringTok{"I de Moran: "}\NormalTok{, }\FunctionTok{round}\NormalTok{(moran\_slx}\SpecialCharTok{$}\NormalTok{estimate[}\DecValTok{1}\NormalTok{], }\DecValTok{3}\NormalTok{), }
                         \StringTok{" (p{-}valor: "}\NormalTok{, }\FunctionTok{round}\NormalTok{(moran\_slx}\SpecialCharTok{$}\NormalTok{p.value, }\DecValTok{3}\NormalTok{), }\StringTok{")"}\NormalTok{),}
       \AttributeTok{x =} \StringTok{"Resíduos"}\NormalTok{, }\AttributeTok{y =} \StringTok{"Lag Espacial"}\NormalTok{) }\SpecialCharTok{+}
  \FunctionTok{theme\_minimal}\NormalTok{()}


\NormalTok{( g\_fit }\SpecialCharTok{|}\NormalTok{ g\_resid\_scatter)}
\end{Highlighting}
\end{Shaded}

\begin{figure}[H]

\centering{

\pandocbounded{\includegraphics[keepaspectratio]{lattice_data_files/figure-pdf/fig-slx-diagnostico-visual-1.pdf}}

}

\caption{\label{fig-slx-diagnostico-visual}Diagnóstico SLX: (A) Ajuste
do Modelo e (B) Análise de Resíduos.}

\end{figure}%

\textbf{Interpretação}

Ao contrário dos modelos globais (SAR e SEM), o mapa de valores preditos
em Figura~\ref{fig-slx-diagnostico-visual} (A) exibe uma superfície de
predição com variabilidade atenuada, falhando em reproduzir a
heterogeneidade e os clusters de valores altos observados nos dados
originais. A inadequação do ajuste é demonstrada definitivamente em
Figura~\ref{fig-slx-diagnostico-visual} (C), onde o Scatter de Moran dos
resíduos mostra tendência linear positiva, indicando que o modelo não
foi capaz de capturar a estrutura espacial dos dados. A estatística de I
de Moran (\(I = 0.198\)) com valor-p significativo (\(p \approx 0\))
confirma a persistência de autocorrelação espacial positiva nos erros,
violando o pressuposto de independência estocástica e evidenciando que a
simples inclusão de defasagens exógenas de \(X\) é ineficaz para
capturar a dependência espacial presente no fenômeno.

\subsection{\texorpdfstring{Modelo Espacial de Durbin (SDM --
\emph{Spatial Durbin
Model})}{Modelo Espacial de Durbin (SDM -- Spatial Durbin Model)}}\label{modelo-espacial-de-durbin-sdm-spatial-durbin-model}

O Modelo Espacial de Durbin (SDM) constitui uma especificação geral que
unifica os mecanismos de dependência espacial endógena e efeitos
contextuais exógenos. Dentro da família do Modelo Geral Espacial (GNS,
Eq.~\ref{eq-GNS}), o SDM é obtido ao impor a restrição \(\lambda = 0\),
mantendo os parâmetros \(\rho\) e \(\boldsymbol{\theta}\) livres. Esta
estrutura postula que o valor observado para uma unidade \(i\) é uma
função simultânea: (i) dos valores da variável resposta em suas unidades
vizinhas, (ii) de suas próprias características observadas, e (iii) das
características observadas de suas vizinhas.

Formalmente, o modelo é definido pela equação estrutural (Anselin 1988;
J. LeSage e Pace 2009):

\begin{equation}\phantomsection\label{eq-SDM}{
\mathbf{y} = \rho \mathbf{W}\mathbf{y} + \mathbf{X}\boldsymbol{\beta} + \mathbf{W}\mathbf{X}\boldsymbol{\theta} + \boldsymbol{\epsilon},
}\end{equation}

com o vetor de erros assumido como
\(\boldsymbol{\epsilon} \sim \mathcal{N}(\mathbf{0}, \sigma^2 \mathbf{I}_n)\).
Na sua forma escalar para uma unidade \(i\):

\[
y_i = \rho \sum_{j=1}^n w_{ij} y_j + \sum_{k=1}^K x_{ik}\beta_k + \sum_{k=1}^K \theta_k \left( \sum_{j=1}^n w_{ij} x_{jk} \right) + \epsilon_i.
\]

O SDM é frequentemente defendido como uma especificação de partida
robusta em modelagem espacial (J. LeSage e Pace 2009). Sua principal
justificativa reside em sua capacidade de mitigar potenciais tendências
de especificação
(\href{https://en.wikipedia.org/wiki/Omitted-variable_bias}{omitted
variable bias}). Se o processo gerador de dados subjacente envolver
fatores não observados que são espacialmente correlacionados e
relacionados a \(\mathbf{X}\), a omissão desses fatores induzirá
dependência espacial nos resíduos. A inclusão do termo
\(\mathbf{W}\mathbf{X}\) no SDM pode absorver parte desta correlação
espacial omitida, produzindo estimativas mais estáveis para
\(\boldsymbol{\beta}\).

Uma ligação fundamental na teoria dos modelos espaciais é que o modelo
SEM (Eq.~\ref{eq-SEM}) é um caso restrito do SDM. Esta relação é
estabelecida pela hipótese de fator comom
(\href{https://en.wikipedia.org/wiki/Common_factors_theory}{Common
Factor Hypothesis}) (Burridge 1981; Anselin 1988). A restrição
não-linear \(\boldsymbol{\theta} = -\rho \boldsymbol{\beta}\), quando
válida, reduz a forma reduzida do SDM à do SEM. Consequentemente,
estimar um SDM permite testar estatisticamente se a dependência espacial
é de natureza substancial (capturada por \(\rho\) e
\(\boldsymbol{\theta}\)) ou meramente residual (um artefato de erros
correlacionados, como no SEM).

A presença simultânea do termo autorregressivo \(\mathbf{W}\mathbf{y}\)
confere ao SDM uma estrutura de dependência global. A forma reduzida do
modelo, obtida isolando \(\mathbf{y}\), é:

\[
\mathbf{y} = (\mathbf{I}_n - \rho \mathbf{W})^{-1} (\mathbf{X}\boldsymbol{\beta} + \mathbf{W}\mathbf{X}\boldsymbol{\theta}) + (\mathbf{I}_n - \rho \mathbf{W})^{-1} \boldsymbol{\epsilon}.
\]

A matriz \((\mathbf{I}_n - \rho \mathbf{W})^{-1}\) atua como um
multiplicador espacial global, propagando choques e efeitos através de
toda a rede de interconexões (matriz de pesos / de vizinhança).

A interpretação direta dos coeficientes \(\boldsymbol{\beta}\) e
\(\boldsymbol{\theta}\) é limitada, pois não representam efeitos
marginais simples. Seguindo J. LeSage e Pace (2009), a matriz de
derivadas parciais da esperança condicional
\(\mathbb{E}[\mathbf{y} \, | \, \mathbf{X}, \mathbf{W}]\) em relação a
uma variável explicativa \(k\) fornece a decomposição completa dos
efeitos:

\[
\frac{\partial \, \mathbb{E}[\mathbf{y} \, | \, \mathbf{X}, \mathbf{W}]}{\partial \, \mathbf{x}_k'} = \mathbf{S}_k(\mathbf{W}) = (\mathbf{I}_n - \rho \mathbf{W})^{-1} (\beta_k \mathbf{I}_n + \theta_k \mathbf{W}).
\]

A partir desta matriz \(\mathbf{S}_k(\mathbf{W})\) de dimensão
\(n \times n\), calculam-se os seguintes impactos médios:

\begin{itemize}
\item
  Efeito direto médio, que é a média dos elementos da diagonal de
  \(\mathbf{S}_k(\mathbf{W})\). Mede o impacto esperado de uma mudança
  em \(x_{ik}\) sobre o próprio \(y_i\), incluindo todos os
  \emph{feedbacks} espaciais que retornam à unidade \(i\).
\item
  Efeito indireto médio (ou de transbordamento), que é média da soma dos
  elementos fora da diagonal para cada linha (ou coluna) de
  \(\mathbf{S}_k(\mathbf{W})\). Mede o impacto esperado de uma mudança
  em \(x_{ik}\) sobre todas as outras unidades \(y_j\) (\(j \neq i\)).
\item
  Efeito total médio que é a soma do efeito direto e do efeito indireto,
  equivalente ao impacto médio de uma mudança simultânea em \(x_{k}\)
  para todas as unidades.
\end{itemize}

Ao contrário do modelo SAR, onde a razão entre efeitos indiretos e
diretos é idêntica para todas as covariáveis, no SDM esta razão é única
para cada variável \(k\), determinada pela combinação específica de
\(\rho\), \(\beta_k\) e \(\theta_k\) (Halleck Vega e Elhorst 2015).

A presença da variável dependente defasada espacialmente
(\(\mathbf{W}\mathbf{y}\)) no lado direito da equação introduz
simultaneidade, tornando o estimador de Mínimos Quadrados Ordinários
(MQO) viesado e inconsistente. Métodos de estimação consistentes são
necessários:

\begin{enumerate}
\def\labelenumi{\arabic{enumi}.}
\item
  máxima verossimilhança, que maximiza a função de log-verossimilhança
  condicional, incorporando o termo
  \(\ln |\mathbf{I}_n - \rho \mathbf{W}|\) para corrigir a
  simultaneidade (Anselin 1988).
\item
  Abordagem Bayesiana (MCMC), na qual J. P. LeSage (1997) desenvolveu um
  procedimento de inferência via Monte Carlo via Cadeias de Markov
  (MCMC). Esta abordagem é particularmente útil para realizar inferência
  sobre os impactos médios (diretos, indiretos e totais), que são
  funções não-lineares dos parâmetros, permitindo a construção direta de
  intervalos de credibilidade.
\end{enumerate}

Assim, o SDM serve como um modelo geral que aninha especificações mais
simples: o SAR (\(\boldsymbol{\theta} = \mathbf{0}\)), o SLX
(\(\rho = 0\)), e o SEM (sob a restrição
\(\boldsymbol{\theta} = -\rho \boldsymbol{\beta}\)). Por esta razão, uma
estratégia de modelagem do geral para o específico, começando pelo SDM,
é frequentemente recomendada para a análise de dados espaciais (Elhorst
et al. 2014).

\begin{table}

\caption{\label{tbl-sdm_comparacao}Resultados da Estimação: Comparação
Geral (OLS, SLX, SAR, SEM, SDM). Estimativas {[}IC 95\%{]}.}

\centering{

\begin{Shaded}
\begin{Highlighting}[]
\ControlFlowTok{if}\NormalTok{ (}\SpecialCharTok{!}\FunctionTok{require}\NormalTok{(}\StringTok{"pacman"}\NormalTok{)) }\FunctionTok{install.packages}\NormalTok{(}\StringTok{"pacman"}\NormalTok{)}
\NormalTok{pacman}\SpecialCharTok{::}\FunctionTok{p\_load}\NormalTok{(spatialreg, spdep, sf, modelsummary, kableExtra, dplyr, ggplot2, patchwork, viridis, ggspatial, tidyr, Matrix)}

\CommentTok{\# 1. Preparação dos Dados}
\ControlFlowTok{if}\NormalTok{ (}\SpecialCharTok{!}\FunctionTok{exists}\NormalTok{(}\StringTok{"mg\_dados"}\NormalTok{)) \{}
\NormalTok{  mg\_dados }\OtherTok{\textless{}{-}}\NormalTok{ geobr}\SpecialCharTok{::}\FunctionTok{read\_municipality}\NormalTok{(}\AttributeTok{code\_muni =} \StringTok{"MG"}\NormalTok{, }\AttributeTok{year =} \DecValTok{2020}\NormalTok{, }\AttributeTok{showProgress =} \ConstantTok{FALSE}\NormalTok{)}
\NormalTok{  coords }\OtherTok{\textless{}{-}} \FunctionTok{st\_coordinates}\NormalTok{(}\FunctionTok{st\_centroid}\NormalTok{(mg\_dados))}
  \FunctionTok{set.seed}\NormalTok{(}\DecValTok{123}\NormalTok{)}
\NormalTok{  mg\_dados}\SpecialCharTok{$}\NormalTok{taxa\_bruta }\OtherTok{\textless{}{-}}\NormalTok{ (}\SpecialCharTok{{-}}\NormalTok{coords[,}\DecValTok{2}\NormalTok{] }\SpecialCharTok{*} \DecValTok{10}\NormalTok{) }\SpecialCharTok{+} \FunctionTok{rnorm}\NormalTok{(}\FunctionTok{nrow}\NormalTok{(mg\_dados), }\DecValTok{0}\NormalTok{, }\DecValTok{5}\NormalTok{)}
\NormalTok{  mg\_dados}\SpecialCharTok{$}\NormalTok{variavel\_x }\OtherTok{\textless{}{-}} \FunctionTok{rnorm}\NormalTok{(}\FunctionTok{nrow}\NormalTok{(mg\_dados))}
\NormalTok{\}}


\CommentTok{\# SDM (Spatial Durbin Model)}
\CommentTok{\# type = "mixed" inclui lag de Y (rho) e lag de X (theta)}
\NormalTok{mod\_sdm }\OtherTok{\textless{}{-}} \FunctionTok{lagsarlm}\NormalTok{(taxa\_bruta }\SpecialCharTok{\textasciitilde{}}\NormalTok{ variavel\_x, }\AttributeTok{data =}\NormalTok{ mg\_dados, }\AttributeTok{listw =}\NormalTok{ lw, }\AttributeTok{type =} \StringTok{"mixed"}\NormalTok{)}

\CommentTok{\#Tabela}
\NormalTok{mapa\_vars }\OtherTok{\textless{}{-}} \FunctionTok{c}\NormalTok{(}
  \StringTok{"(Intercept)"}    \OtherTok{=} \StringTok{"Intercepto"}\NormalTok{,}
  \StringTok{"variavel\_x"}     \OtherTok{=} \StringTok{"$}\SpecialCharTok{\textbackslash{}\textbackslash{}}\StringTok{beta$"}\NormalTok{,}
  \StringTok{"lag.variavel\_x"} \OtherTok{=} \StringTok{"WX $}\SpecialCharTok{\textbackslash{}\textbackslash{}}\StringTok{theta$"}\NormalTok{, }\CommentTok{\# Compartilhado por SLX e SDM}
  \StringTok{"rho"}            \OtherTok{=} \StringTok{"$}\SpecialCharTok{\textbackslash{}\textbackslash{}}\StringTok{rho$"}\NormalTok{,       }\CommentTok{\# Compartilhado por SAR e SDM}
  \StringTok{"lambda"}         \OtherTok{=} \StringTok{"$}\SpecialCharTok{\textbackslash{}\textbackslash{}}\StringTok{lambda$"}    \CommentTok{\# Exclusivo do SEM}
\NormalTok{)}

\NormalTok{mapa\_gof }\OtherTok{\textless{}{-}} \FunctionTok{list}\NormalTok{(}
  \FunctionTok{list}\NormalTok{(}\StringTok{"raw"} \OtherTok{=} \StringTok{"nobs"}\NormalTok{, }\StringTok{"clean"} \OtherTok{=} \StringTok{"N"}\NormalTok{, }\StringTok{"fmt"} \OtherTok{=} \DecValTok{0}\NormalTok{),}
  \FunctionTok{list}\NormalTok{(}\StringTok{"raw"} \OtherTok{=} \StringTok{"r.squared"}\NormalTok{, }\StringTok{"clean"} \OtherTok{=} \StringTok{"$R\^{}2$"}\NormalTok{, }\StringTok{"fmt"} \OtherTok{=} \DecValTok{3}\NormalTok{),}
  \FunctionTok{list}\NormalTok{(}\StringTok{"raw"} \OtherTok{=} \StringTok{"aic"}\NormalTok{, }\StringTok{"clean"} \OtherTok{=} \StringTok{"AIC"}\NormalTok{, }\StringTok{"fmt"} \OtherTok{=} \DecValTok{1}\NormalTok{),}
  \FunctionTok{list}\NormalTok{(}\StringTok{"raw"} \OtherTok{=} \StringTok{"logLik"}\NormalTok{, }\StringTok{"clean"} \OtherTok{=} \StringTok{"Log Likelihood"}\NormalTok{, }\StringTok{"fmt"} \OtherTok{=} \DecValTok{1}\NormalTok{)}
\NormalTok{)}

\CommentTok{\# Tabela Unificada}
\FunctionTok{modelsummary}\NormalTok{(}
  \FunctionTok{list}\NormalTok{(}
    \StringTok{"OLS"} \OtherTok{=}\NormalTok{ mod\_ols, }
    \StringTok{"SLX"} \OtherTok{=}\NormalTok{ mod\_slx,}
    \StringTok{"SAR"} \OtherTok{=}\NormalTok{ mod\_sar, }
    \StringTok{"SEM"} \OtherTok{=}\NormalTok{ mod\_sem,}
    \StringTok{"SDM"} \OtherTok{=}\NormalTok{ mod\_sdm}
\NormalTok{  ),}
  \AttributeTok{coef\_map =}\NormalTok{ mapa\_vars,      }
  \AttributeTok{gof\_map =}\NormalTok{ mapa\_gof,      }
  \AttributeTok{estimate =} \StringTok{"\{estimate\} [\{conf.low\}, \{conf.high\}]"}\NormalTok{,}
  \AttributeTok{statistic =} \ConstantTok{NULL}\NormalTok{, }
  \AttributeTok{stars =} \FunctionTok{c}\NormalTok{(}\StringTok{\textquotesingle{}*\textquotesingle{}} \OtherTok{=}\NormalTok{ .}\DecValTok{05}\NormalTok{, }\StringTok{\textquotesingle{}**\textquotesingle{}} \OtherTok{=}\NormalTok{ .}\DecValTok{01}\NormalTok{, }\StringTok{\textquotesingle{}***\textquotesingle{}} \OtherTok{=}\NormalTok{ .}\DecValTok{001}\NormalTok{),}
  \AttributeTok{title =} \ConstantTok{NULL}\NormalTok{,     }
  \AttributeTok{output =} \StringTok{"kableExtra"}\NormalTok{,}
  \AttributeTok{escape =} \ConstantTok{FALSE}
\NormalTok{) }\SpecialCharTok{\%\textgreater{}\%}
  \FunctionTok{kable\_styling}\NormalTok{(}\AttributeTok{latex\_options =} \FunctionTok{c}\NormalTok{(}\StringTok{"HOLD\_position"}\NormalTok{), }
                \AttributeTok{full\_width =} \ConstantTok{FALSE}\NormalTok{, }
                \AttributeTok{position =} \StringTok{"center"}\NormalTok{) }\SpecialCharTok{\%\textgreater{}\%}
  \FunctionTok{row\_spec}\NormalTok{(}\FunctionTok{c}\NormalTok{(}\DecValTok{5}\NormalTok{, }\DecValTok{7}\NormalTok{, }\DecValTok{9}\NormalTok{), }\AttributeTok{bold =} \ConstantTok{TRUE}\NormalTok{) }\SpecialCharTok{\%\textgreater{}\%} 
  \FunctionTok{as.character}\NormalTok{() }\SpecialCharTok{\%\textgreater{}\%}
  \FunctionTok{cat}\NormalTok{()}
\end{Highlighting}
\end{Shaded}

\centering\centering
\begin{tabular}[t]{lccccc}
\toprule
  & OLS & SLX & SAR & SEM & SDM\\
\midrule
Intercepto & \num{8.973} [\num{8.616}, \num{9.330}] & \num{8.970} [\num{8.613}, \num{9.327}] & \num{5.751} [\num{4.876}, \num{6.626}] & \num{8.978} [\num{8.449}, \num{9.507}] & \num{5.748} [\num{4.873}, \num{6.622}]\\
$\beta$ & \num{0.058} [\num{-0.306}, \num{0.422}] & \num{0.056} [\num{-0.308}, \num{0.420}] & \num{0.029} [\num{-0.316}, \num{0.374}] & \num{-0.001} [\num{-0.342}, \num{0.340}] & \num{0.027} [\num{-0.318}, \num{0.372}]\\
WX $\theta$ &  & \num{0.397} [\num{-0.396}, \num{1.190}] &  &  & \num{0.391} [\num{-0.361}, \num{1.143}]\\
$\rho$ &  &  & \num{0.359} [\num{0.269}, \num{0.449}] &  & \num{0.359} [\num{0.270}, \num{0.449}]\\
\textbf{$\lambda$} & \textbf{} & \textbf{} & \textbf{} & \textbf{\num{0.359} [\num{0.270}, \num{0.449}]} & \textbf{}\\
\midrule
N & \num{853} & \num{853} &  &  & \\
\textbf{$R^2$} & \textbf{\num{0.000}} & \textbf{\num{0.001}} & \textbf{} & \textbf{} & \textbf{}\\
AIC & \num{5275.0} & \num{5276.1} & \num{5212.9} & \num{5213.0} & \num{5213.9}\\
\textbf{Log Likelihood} & \textbf{\num{-2634.5}} & \textbf{\num{-2634.0}} & \textbf{} & \textbf{} & \textbf{}\\
\bottomrule
\multicolumn{6}{l}{\rule{0pt}{1em}* p $<$ 0.05, ** p $<$ 0.01, *** p $<$ 0.001}\\
\end{tabular}

}

\end{table}%

\begin{Shaded}
\begin{Highlighting}[]
\CommentTok{\# Mapa dos Valores Ajustados}
\NormalTok{mg\_dados}\SpecialCharTok{$}\NormalTok{fitted\_sdm }\OtherTok{\textless{}{-}} \FunctionTok{fitted}\NormalTok{(mod\_sdm)}

\NormalTok{g\_fit }\OtherTok{\textless{}{-}} \FunctionTok{ggplot}\NormalTok{(mg\_dados) }\SpecialCharTok{+}
  \FunctionTok{geom\_sf}\NormalTok{(}\FunctionTok{aes}\NormalTok{(}\AttributeTok{fill =}\NormalTok{ fitted\_sdm), }\AttributeTok{color =} \ConstantTok{NA}\NormalTok{) }\SpecialCharTok{+}
  \FunctionTok{scale\_fill\_viridis\_c}\NormalTok{(}\AttributeTok{option =} \StringTok{"turbo"}\NormalTok{, }\AttributeTok{name =} \StringTok{"Predito"}\NormalTok{) }\SpecialCharTok{+}
  \FunctionTok{labs}\NormalTok{(}\AttributeTok{title =} \StringTok{"A. Valores Preditos (SDM)"}\NormalTok{, }
       \AttributeTok{subtitle =} \StringTok{"Ajuste considerando Wy e WX"}\NormalTok{) }\SpecialCharTok{+}
  \FunctionTok{theme\_minimal}\NormalTok{() }\SpecialCharTok{+} 
  \FunctionTok{annotation\_scale}\NormalTok{(}\AttributeTok{location =} \StringTok{"bl"}\NormalTok{, }\AttributeTok{width\_hint =} \FloatTok{0.3}\NormalTok{) }\SpecialCharTok{+}
  \FunctionTok{annotation\_north\_arrow}\NormalTok{(}\AttributeTok{location =} \StringTok{"tl"}\NormalTok{, }\AttributeTok{style =}\NormalTok{ north\_arrow\_fancy\_orienteering,}
                         \AttributeTok{pad\_x =} \FunctionTok{unit}\NormalTok{(}\FloatTok{0.1}\NormalTok{, }\StringTok{"in"}\NormalTok{), }\AttributeTok{pad\_y =} \FunctionTok{unit}\NormalTok{(}\FloatTok{0.1}\NormalTok{, }\StringTok{"in"}\NormalTok{))}

\CommentTok{\#Diagnóstico dos Resíduos}
\NormalTok{mg\_dados}\SpecialCharTok{$}\NormalTok{resid\_sdm }\OtherTok{\textless{}{-}} \FunctionTok{residuals}\NormalTok{(mod\_sdm)}
\NormalTok{moran\_sdm }\OtherTok{\textless{}{-}} \FunctionTok{moran.test}\NormalTok{(mg\_dados}\SpecialCharTok{$}\NormalTok{resid\_sdm, lw)}
\NormalTok{mg\_dados}\SpecialCharTok{$}\NormalTok{resid\_lag\_sdm }\OtherTok{\textless{}{-}} \FunctionTok{lag.listw}\NormalTok{(lw, mg\_dados}\SpecialCharTok{$}\NormalTok{resid\_sdm)}

\NormalTok{g\_resid\_scatter }\OtherTok{\textless{}{-}} \FunctionTok{ggplot}\NormalTok{(mg\_dados, }\FunctionTok{aes}\NormalTok{(}\AttributeTok{x =}\NormalTok{ resid\_sdm, }\AttributeTok{y =}\NormalTok{ resid\_lag\_sdm)) }\SpecialCharTok{+}
  \FunctionTok{geom\_hline}\NormalTok{(}\AttributeTok{yintercept =} \DecValTok{0}\NormalTok{, }\AttributeTok{linetype =} \StringTok{"dashed"}\NormalTok{, }\AttributeTok{color =} \StringTok{"gray"}\NormalTok{) }\SpecialCharTok{+}
  \FunctionTok{geom\_vline}\NormalTok{(}\AttributeTok{xintercept =} \DecValTok{0}\NormalTok{, }\AttributeTok{linetype =} \StringTok{"dashed"}\NormalTok{, }\AttributeTok{color =} \StringTok{"gray"}\NormalTok{) }\SpecialCharTok{+}
  \FunctionTok{geom\_point}\NormalTok{(}\AttributeTok{alpha =} \FloatTok{0.3}\NormalTok{) }\SpecialCharTok{+}
  \FunctionTok{geom\_smooth}\NormalTok{(}\AttributeTok{method =} \StringTok{"lm"}\NormalTok{, }\AttributeTok{se =} \ConstantTok{FALSE}\NormalTok{, }\AttributeTok{color =} \StringTok{"red"}\NormalTok{, }\AttributeTok{size =} \FloatTok{0.8}\NormalTok{) }\SpecialCharTok{+}
  \FunctionTok{labs}\NormalTok{(}\AttributeTok{title =} \StringTok{"C. Scatter de Moran (Resíduos SDM)"}\NormalTok{, }
       \AttributeTok{subtitle =} \FunctionTok{paste0}\NormalTok{(}\StringTok{"I de Moran: "}\NormalTok{, }\FunctionTok{round}\NormalTok{(moran\_sdm}\SpecialCharTok{$}\NormalTok{estimate[}\DecValTok{1}\NormalTok{], }\DecValTok{3}\NormalTok{), }
                         \StringTok{" (p{-}valor: "}\NormalTok{, }\FunctionTok{round}\NormalTok{(moran\_sdm}\SpecialCharTok{$}\NormalTok{p.value, }\DecValTok{3}\NormalTok{), }\StringTok{")"}\NormalTok{),}
       \AttributeTok{x =} \StringTok{"Resíduos"}\NormalTok{, }\AttributeTok{y =} \StringTok{"Lag Espacial"}\NormalTok{) }\SpecialCharTok{+}
  \FunctionTok{theme\_minimal}\NormalTok{()}

\NormalTok{g\_fit }\SpecialCharTok{|}\NormalTok{ g\_resid\_scatter}
\end{Highlighting}
\end{Shaded}

\begin{figure}[H]

\centering{

\pandocbounded{\includegraphics[keepaspectratio]{lattice_data_files/figure-pdf/fig-sdm-diagnostico-1.pdf}}

}

\caption{\label{fig-sdm-diagnostico}Diagnóstico SDM: (A) Densidade dos
Impactos Globais, (B) Ajuste e (C) Resíduos.}

\end{figure}%

\subsection{\texorpdfstring{Modelo Espacial de Durbin com Erro (SDEM --
\emph{Spatial Durbin Error
Model})}{Modelo Espacial de Durbin com Erro (SDEM -- Spatial Durbin Error Model)}}\label{modelo-espacial-de-durbin-com-erro-sdem-spatial-durbin-error-model}

O Modelo Espacial de Durbin com Erro (SDEM) é uma especificação híbrida
que combina a modelagem de efeitos contextuais exógenos locais com a
correção para dependência espacial residual de natureza global. No
contexto do Modelo Geral Espacial (GNS, Eq.~\ref{eq-GNS}), o SDEM é
obtido ao impor a restrição paramétrica \(\rho = 0\), mantendo livres os
parâmetros \(\boldsymbol{\theta}\) e \(\lambda\).

Esta estrutura pressupõe que o valor da variável dependente em uma
unidade \(i\) é explicado por: (1) suas próprias características
observadas, (2) as características observadas de suas unidades vizinhas,
e (3) um termo de erro que exibe autocorrelação espacial. O modelo é
definido pelo seguinte sistema de equações (Anselin 1988; Elhorst et al.
2014):

\begin{equation}\phantomsection\label{eq-SDEM}{
\mathbf{y} = \mathbf{X}\boldsymbol{\beta} + \mathbf{W}\mathbf{X}\boldsymbol{\theta} + \mathbf{u}, \quad \mathbf{u} = \lambda \mathbf{W}\mathbf{u} + \boldsymbol{\epsilon},
}\end{equation} com erros independentes e identicamente distribuídos
\(\boldsymbol{\epsilon} \sim \mathcal{N}(\mathbf{0}, \sigma^2 \mathbf{I}_n)\).

Na sua forma escalar, para uma unidade \(i\):

\[
y_i = \sum_{k=1}^K x_{ik}\beta_k + \sum_{k=1}^K \theta_k \left( \sum_{j=1}^n w_{ij} x_{jk} \right) + u_i, \quad u_i = \lambda \sum_{j=1}^n w_{ij} u_j + \epsilon_i.
\]

O SDEM tem sido destacado na literatura como uma alternativa robusta ao
Modelo de Durbin Espacial (SDM), especialmente quando a teoria
subjacente sugere que os mecanismos de interação substantiva têm alcance
geográfico limitado (Halleck Vega e Elhorst 2015). A especificação
separa duas fontes de dependência espacial:

\begin{enumerate}
\def\labelenumi{\arabic{enumi}.}
\item
  A ausência do termo autorregressivo \(\rho \mathbf{W}\mathbf{y}\)
  implica que os \emph{spillovers} substantivos são estritamente locais.
  Uma mudança nas características \(\mathbf{x}_j\) de uma unidade \(j\)
  afeta a unidade \(i\) apenas se \(w_{ij} \neq 0\), ou seja, se as
  unidades forem vizinhas diretas na estrutura definida por
  \(\mathbf{W}\). Não há mecanismos de \emph{feedback} ou propagação
  global destes efeitos através da rede (matriz de pesos / de
  vizinhança).
\item
  O parâmetro \(\lambda\) captura a autocorrelação espacial nos
  resíduos, tipicamente atribuída a variáveis omitidas com padrão
  espacial ou a choques não observados que se difundem regionalmente.
  Diferentemente dos efeitos na média, estes choques propagam-se
  globalmente através do multiplicador espacial
  \((\mathbf{I}_n - \lambda \mathbf{W})^{-1}\) no termo de erro.
\end{enumerate}

Esta separação é teoricamente atraente em aplicações onde se espera que
externalidades de política ou características de vizinhança
(\(\mathbf{W}\mathbf{X}\)) tenham alcance limitado (ex.: poluição
sonora, sombreamento), enquanto fatores de confusão não observados
(ex.:, normas culturais, clima) exibam um padrão de correlação espacial
de longo alcance (J. LeSage e Pace 2009).

A forma reduzida do modelo é obtida ao substituir a estrutura do erro na
Eq.~\ref{eq-SDEM}:

\[
\mathbf{y} = \mathbf{X}\boldsymbol{\beta} + \mathbf{W}\mathbf{X}\boldsymbol{\theta} + (\mathbf{I}_n - \lambda \mathbf{W})^{-1} \boldsymbol{\epsilon}.
\]

A esperança condicional da variável dependente,
\(\mathbb{E}[\mathbf{y} \, | \, \mathbf{X}, \mathbf{W}]\), é dada por:

\[
\mathbb{E}[\mathbf{y} \, | \, \mathbf{X}, \mathbf{W}] = \mathbf{X}\boldsymbol{\beta} + \mathbf{W}\mathbf{X}\boldsymbol{\theta}.
\]

Crucialmente, a ausência do multiplicador global
\((\mathbf{I}_n - \rho \mathbf{W})^{-1}\) na parte determinística
simplifica radicalmente a interpretação dos coeficientes. A matriz de
derivadas parciais é linear e diretamente obtida dos parâmetros
estimados:

\[
\frac{\partial \, \mathbb{E}[\mathbf{y} \, | \, \mathbf{X}, \mathbf{W}]}{\partial \, \mathbf{x}_k'} = \beta_k \mathbf{I}_n + \theta_k \mathbf{W}.
\]

Disto decorre uma decomposição dos efeitos:

\begin{itemize}
\item
  Efeito direto, onde \(\frac{\partial y_i}{\partial x_{ik}} = \beta_k\)
  representa o impacto de uma mudança na característica própria
  \(x_{ik}\) sobre o resultado \(y_i\). Este efeito é idêntico ao de uma
  regressão clássica, sem \emph{feedback} espacial.
\item
  Efeito indireto (ou de \emph{Spillover} local), onde
  \(\frac{\partial y_i}{\partial x_{jk}} = \theta_k w_{ij}\) representa
  o impacto de uma mudança na característica de uma unidade vizinha
  \(j\) sobre o resultado da unidade \(i\). Se \(w_{ij} = 0\) (unidades
  não conectadas), o efeito é nulo.
\end{itemize}

Embora a equação da média contenha apenas regressores exógenos
(\(\mathbf{X}\) e \(\mathbf{W}\mathbf{X}\)), a presença de
autocorrelação espacial no erro (\(\lambda \neq 0\)) viola o pressuposto
de esfericidade. Consequentemente, o estimador de Mínimos Quadrados
Ordinários (MQO) permanece não viesado e consistente para
\(\boldsymbol{\beta}\) e \(\boldsymbol{\theta}\), mas se torna
ineficiente. Mais criticamente, a estimativa da matriz de covariância
dos parâmetros e, portanto, os erros-padrão e testes de hipótese padrão
(testes \(t\), \(F\)) tornam-se inválidos (Anselin 1988).

A estimação eficiente e a inferência válida exigem métodos que
incorporem explicitamente a estrutura de covariância não esférica dos
erros,
\(\operatorname{Cov}(\mathbf{u}) = \sigma^2 [(\mathbf{I}_n - \lambda \mathbf{W})^{\top}(\mathbf{I}_n - \lambda \mathbf{W})]^{-1}\).
As abordagens padrão são:

\begin{enumerate}
\def\labelenumi{\arabic{enumi}.}
\item
  máxima verossimilhança, que maximiza a função de log-verossimilhança
  que inclui o termo \(\ln |\mathbf{I}_n - \lambda \mathbf{W}|\)
  associado à transformação do erro.
\item
  Método dos Momentos Generalizados (GMM) que utiliza os momentos dos
  resíduos para estimar \(\lambda\) de forma consistente, conforme
  proposto por H. H. Kelejian e Prucha (1999) para modelos com erros
  espaciais.
\end{enumerate}

O SDEM é, portanto, uma ferramenta valiosa quando a teoria apoia a
existência de \emph{spillovers} contextuais locais, mas é necessário
controlar rigorosamente a heterogeneidade espacial não observada de
longo alcance. Conforme recomendado por Elhorst et al. (2014), este
modelo é preferível ao SDM quando testes de especificação rejeitam a
dependência espacial na variável dependente (rejeitam \(\rho \neq 0\) em
favor de \(\rho = 0\)), mas indicam a presença simultânea de efeitos
contextuais (\(\boldsymbol{\theta} \neq \mathbf{0}\)) e dependência
residual nos erros (\(\lambda \neq 0\)).

\begin{table}

\caption{\label{tbl-sdem_comparacao}Resultados da Estimação: Comparação
Geral (OLS, SLX, SAR, SEM, SDM, SDEM). Estimativas {[}IC 95\%{]}.}

\centering{

\begin{Shaded}
\begin{Highlighting}[]
\ControlFlowTok{if}\NormalTok{ (}\SpecialCharTok{!}\FunctionTok{require}\NormalTok{(}\StringTok{"pacman"}\NormalTok{)) }\FunctionTok{install.packages}\NormalTok{(}\StringTok{"pacman"}\NormalTok{)}
\NormalTok{pacman}\SpecialCharTok{::}\FunctionTok{p\_load}\NormalTok{(spatialreg, spdep, sf, modelsummary, kableExtra, dplyr, ggplot2, patchwork, viridis, ggspatial, tidyr, Matrix)}


\ControlFlowTok{if}\NormalTok{ (}\SpecialCharTok{!}\FunctionTok{exists}\NormalTok{(}\StringTok{"mg\_dados"}\NormalTok{)) \{}
\NormalTok{  mg\_dados }\OtherTok{\textless{}{-}}\NormalTok{ geobr}\SpecialCharTok{::}\FunctionTok{read\_municipality}\NormalTok{(}\AttributeTok{code\_muni =} \StringTok{"MG"}\NormalTok{, }\AttributeTok{year =} \DecValTok{2020}\NormalTok{, }\AttributeTok{showProgress =} \ConstantTok{FALSE}\NormalTok{)}
\NormalTok{  coords }\OtherTok{\textless{}{-}} \FunctionTok{st\_coordinates}\NormalTok{(}\FunctionTok{st\_centroid}\NormalTok{(mg\_dados))}
  \FunctionTok{set.seed}\NormalTok{(}\DecValTok{123}\NormalTok{)}
\NormalTok{  mg\_dados}\SpecialCharTok{$}\NormalTok{taxa\_bruta }\OtherTok{\textless{}{-}}\NormalTok{ (}\SpecialCharTok{{-}}\NormalTok{coords[,}\DecValTok{2}\NormalTok{] }\SpecialCharTok{*} \DecValTok{10}\NormalTok{) }\SpecialCharTok{+} \FunctionTok{rnorm}\NormalTok{(}\FunctionTok{nrow}\NormalTok{(mg\_dados), }\DecValTok{0}\NormalTok{, }\DecValTok{5}\NormalTok{)}
\NormalTok{  mg\_dados}\SpecialCharTok{$}\NormalTok{variavel\_x }\OtherTok{\textless{}{-}} \FunctionTok{rnorm}\NormalTok{(}\FunctionTok{nrow}\NormalTok{(mg\_dados))}
\NormalTok{\}}


\CommentTok{\# SDEM (Spatial Durbin Error Model)}
\CommentTok{\# errorsarlm com Durbin=TRUE inclui WX (theta) e lambda (erro)}
\NormalTok{mod\_sdem }\OtherTok{\textless{}{-}} \FunctionTok{errorsarlm}\NormalTok{(taxa\_bruta }\SpecialCharTok{\textasciitilde{}}\NormalTok{ variavel\_x, }\AttributeTok{data =}\NormalTok{ mg\_dados, }\AttributeTok{listw =}\NormalTok{ lw, }\AttributeTok{Durbin =} \ConstantTok{TRUE}\NormalTok{)}

\CommentTok{\#Configuração da Tabela}
\NormalTok{mapa\_vars }\OtherTok{\textless{}{-}} \FunctionTok{c}\NormalTok{(}
  \StringTok{"(Intercept)"}    \OtherTok{=} \StringTok{"Intercepto"}\NormalTok{,}
  \StringTok{"variavel\_x"}     \OtherTok{=} \StringTok{"$}\SpecialCharTok{\textbackslash{}\textbackslash{}}\StringTok{beta$"}\NormalTok{,}
  \StringTok{"lag.variavel\_x"} \OtherTok{=} \StringTok{"WX $}\SpecialCharTok{\textbackslash{}\textbackslash{}}\StringTok{theta$"}\NormalTok{,      }\CommentTok{\# Theta (SLX, SDM, SDEM)}
  \StringTok{"rho"}            \OtherTok{=} \StringTok{"$}\SpecialCharTok{\textbackslash{}\textbackslash{}}\StringTok{rho$"}\NormalTok{,      }\CommentTok{\# Rho (SAR, SDM)}
  \StringTok{"lambda"}         \OtherTok{=} \StringTok{"$}\SpecialCharTok{\textbackslash{}\textbackslash{}}\StringTok{lambda$"}     \CommentTok{\# Lambda (SEM, SDEM)}
\NormalTok{)}

\NormalTok{mapa\_gof }\OtherTok{\textless{}{-}} \FunctionTok{list}\NormalTok{(}
  \FunctionTok{list}\NormalTok{(}\StringTok{"raw"} \OtherTok{=} \StringTok{"nobs"}\NormalTok{, }\StringTok{"clean"} \OtherTok{=} \StringTok{"N"}\NormalTok{, }\StringTok{"fmt"} \OtherTok{=} \DecValTok{0}\NormalTok{),}
  \FunctionTok{list}\NormalTok{(}\StringTok{"raw"} \OtherTok{=} \StringTok{"r.squared"}\NormalTok{, }\StringTok{"clean"} \OtherTok{=} \StringTok{"$R\^{}2$"}\NormalTok{, }\StringTok{"fmt"} \OtherTok{=} \DecValTok{3}\NormalTok{),}
  \FunctionTok{list}\NormalTok{(}\StringTok{"raw"} \OtherTok{=} \StringTok{"aic"}\NormalTok{, }\StringTok{"clean"} \OtherTok{=} \StringTok{"AIC"}\NormalTok{, }\StringTok{"fmt"} \OtherTok{=} \DecValTok{1}\NormalTok{),}
  \FunctionTok{list}\NormalTok{(}\StringTok{"raw"} \OtherTok{=} \StringTok{"logLik"}\NormalTok{, }\StringTok{"clean"} \OtherTok{=} \StringTok{"Log Likelihood"}\NormalTok{, }\StringTok{"fmt"} \OtherTok{=} \DecValTok{1}\NormalTok{)}
\NormalTok{)}

\CommentTok{\# Tabela Unificada (6 Modelos)}
\FunctionTok{modelsummary}\NormalTok{(}
  \FunctionTok{list}\NormalTok{(}
    \StringTok{"OLS"}  \OtherTok{=}\NormalTok{ mod\_ols, }
    \StringTok{"SLX"}  \OtherTok{=}\NormalTok{ mod\_slx,}
    \StringTok{"SAR"}  \OtherTok{=}\NormalTok{ mod\_sar, }
    \StringTok{"SEM"}  \OtherTok{=}\NormalTok{ mod\_sem,}
    \StringTok{"SDM"}  \OtherTok{=}\NormalTok{ mod\_sdm,}
    \StringTok{"SDEM"} \OtherTok{=}\NormalTok{ mod\_sdem}
\NormalTok{  ),}
  \AttributeTok{coef\_map =}\NormalTok{ mapa\_vars,      }
  \AttributeTok{gof\_map =}\NormalTok{ mapa\_gof,      }
  \AttributeTok{estimate =} \StringTok{"\{estimate\} [\{conf.low\}, \{conf.high\}]"}\NormalTok{,}
  \AttributeTok{statistic =} \ConstantTok{NULL}\NormalTok{, }
  \AttributeTok{stars =} \FunctionTok{c}\NormalTok{(}\StringTok{\textquotesingle{}*\textquotesingle{}} \OtherTok{=}\NormalTok{ .}\DecValTok{05}\NormalTok{, }\StringTok{\textquotesingle{}**\textquotesingle{}} \OtherTok{=}\NormalTok{ .}\DecValTok{01}\NormalTok{, }\StringTok{\textquotesingle{}***\textquotesingle{}} \OtherTok{=}\NormalTok{ .}\DecValTok{001}\NormalTok{),}
  \AttributeTok{title =} \ConstantTok{NULL}\NormalTok{,    }
  \AttributeTok{output =} \StringTok{"kableExtra"}\NormalTok{, }
  \AttributeTok{escape =} \ConstantTok{FALSE}
\NormalTok{) }\SpecialCharTok{\%\textgreater{}\%}
  \FunctionTok{kable\_styling}\NormalTok{(}\AttributeTok{latex\_options =} \FunctionTok{c}\NormalTok{(}\StringTok{"HOLD\_position"}\NormalTok{), }
                \AttributeTok{full\_width =} \ConstantTok{FALSE}\NormalTok{, }
                \AttributeTok{position =} \StringTok{"center"}\NormalTok{) }\SpecialCharTok{\%\textgreater{}\%}
  \FunctionTok{row\_spec}\NormalTok{(}\FunctionTok{c}\NormalTok{(}\DecValTok{5}\NormalTok{, }\DecValTok{7}\NormalTok{, }\DecValTok{9}\NormalTok{), }\AttributeTok{bold =} \ConstantTok{TRUE}\NormalTok{) }\SpecialCharTok{\%\textgreater{}\%} 
  \FunctionTok{as.character}\NormalTok{() }\SpecialCharTok{\%\textgreater{}\%}
  \FunctionTok{cat}\NormalTok{()}
\end{Highlighting}
\end{Shaded}

\centering\centering
\begin{tabular}[t]{lcccccc}
\toprule
  & OLS & SLX & SAR & SEM & SDM & SDEM\\
\midrule
Intercepto & \num{8.973} [\num{8.616}, \num{9.330}] & \num{8.970} [\num{8.613}, \num{9.327}] & \num{5.751} [\num{4.876}, \num{6.626}] & \num{8.978} [\num{8.449}, \num{9.507}] & \num{5.748} [\num{4.873}, \num{6.622}] & \num{8.974} [\num{8.444}, \num{9.503}]\\
$\beta$ & \num{0.058} [\num{-0.306}, \num{0.422}] & \num{0.056} [\num{-0.308}, \num{0.420}] & \num{0.029} [\num{-0.316}, \num{0.374}] & \num{-0.001} [\num{-0.342}, \num{0.340}] & \num{0.027} [\num{-0.318}, \num{0.372}] & \num{0.060} [\num{-0.297}, \num{0.417}]\\
WX $\theta$ &  & \num{0.397} [\num{-0.396}, \num{1.190}] &  &  & \num{0.391} [\num{-0.361}, \num{1.143}] & \num{0.496} [\num{-0.366}, \num{1.359}]\\
$\rho$ &  &  & \num{0.359} [\num{0.269}, \num{0.449}] &  & \num{0.359} [\num{0.270}, \num{0.449}] & \\
\textbf{$\lambda$} & \textbf{} & \textbf{} & \textbf{} & \textbf{\num{0.359} [\num{0.270}, \num{0.449}]} & \textbf{} & \textbf{\num{0.360} [\num{0.270}, \num{0.450}]}\\
\midrule
N & \num{853} & \num{853} &  &  &  & \\
\textbf{$R^2$} & \textbf{\num{0.000}} & \textbf{\num{0.001}} & \textbf{} & \textbf{} & \textbf{} & \textbf{}\\
AIC & \num{5275.0} & \num{5276.1} & \num{5212.9} & \num{5213.0} & \num{5213.9} & \num{5213.7}\\
\textbf{Log Likelihood} & \textbf{\num{-2634.5}} & \textbf{\num{-2634.0}} & \textbf{} & \textbf{} & \textbf{} & \textbf{}\\
\bottomrule
\multicolumn{7}{l}{\rule{0pt}{1em}* p $<$ 0.05, ** p $<$ 0.01, *** p $<$ 0.001}\\
\end{tabular}

}

\end{table}%

\begin{Shaded}
\begin{Highlighting}[]
\CommentTok{\# Mapa dos Valores Ajustados}
\NormalTok{mg\_dados}\SpecialCharTok{$}\NormalTok{fitted\_sdem }\OtherTok{\textless{}{-}} \FunctionTok{fitted}\NormalTok{(mod\_sdem)}

\NormalTok{g\_fit }\OtherTok{\textless{}{-}} \FunctionTok{ggplot}\NormalTok{(mg\_dados) }\SpecialCharTok{+}
  \FunctionTok{geom\_sf}\NormalTok{(}\FunctionTok{aes}\NormalTok{(}\AttributeTok{fill =}\NormalTok{ fitted\_sdem), }\AttributeTok{color =} \ConstantTok{NA}\NormalTok{) }\SpecialCharTok{+}
  \FunctionTok{scale\_fill\_viridis\_c}\NormalTok{(}\AttributeTok{option =} \StringTok{"turbo"}\NormalTok{, }\AttributeTok{name =} \StringTok{"Predito"}\NormalTok{) }\SpecialCharTok{+}
  \FunctionTok{labs}\NormalTok{(}\AttributeTok{title =} \StringTok{"A. Valores Preditos (SDEM)"}\NormalTok{, }
       \AttributeTok{subtitle =} \StringTok{"Padrão recuperado (WX + Erro)"}\NormalTok{) }\SpecialCharTok{+}
  \FunctionTok{theme\_minimal}\NormalTok{() }\SpecialCharTok{+} 
  \FunctionTok{annotation\_scale}\NormalTok{(}\AttributeTok{location =} \StringTok{"bl"}\NormalTok{, }\AttributeTok{width\_hint =} \FloatTok{0.3}\NormalTok{, }\AttributeTok{bar\_cols =} \FunctionTok{c}\NormalTok{(}\StringTok{"black"}\NormalTok{, }\StringTok{"white"}\NormalTok{)) }\SpecialCharTok{+}
  \FunctionTok{annotation\_north\_arrow}\NormalTok{(}\AttributeTok{location =} \StringTok{"tl"}\NormalTok{, }\AttributeTok{which\_north =} \StringTok{"true"}\NormalTok{, }
                         \AttributeTok{pad\_x =} \FunctionTok{unit}\NormalTok{(}\FloatTok{0.2}\NormalTok{, }\StringTok{"in"}\NormalTok{), }\AttributeTok{pad\_y =} \FunctionTok{unit}\NormalTok{(}\FloatTok{0.2}\NormalTok{, }\StringTok{"in"}\NormalTok{),}
                         \AttributeTok{style =}\NormalTok{ north\_arrow\_fancy\_orienteering)}

\CommentTok{\# Diagnóstico dos Resíduos}
\NormalTok{mg\_dados}\SpecialCharTok{$}\NormalTok{resid\_sdem }\OtherTok{\textless{}{-}} \FunctionTok{residuals}\NormalTok{(mod\_sdem)}
\NormalTok{moran\_sdem }\OtherTok{\textless{}{-}} \FunctionTok{moran.test}\NormalTok{(mg\_dados}\SpecialCharTok{$}\NormalTok{resid\_sdem, lw)}
\NormalTok{mg\_dados}\SpecialCharTok{$}\NormalTok{resid\_lag\_sdem }\OtherTok{\textless{}{-}} \FunctionTok{lag.listw}\NormalTok{(lw, mg\_dados}\SpecialCharTok{$}\NormalTok{resid\_sdem)}

\NormalTok{g\_resid\_scatter }\OtherTok{\textless{}{-}} \FunctionTok{ggplot}\NormalTok{(mg\_dados, }\FunctionTok{aes}\NormalTok{(}\AttributeTok{x =}\NormalTok{ resid\_sdem, }\AttributeTok{y =}\NormalTok{ resid\_lag\_sdem)) }\SpecialCharTok{+}
  \FunctionTok{geom\_hline}\NormalTok{(}\AttributeTok{yintercept =} \DecValTok{0}\NormalTok{, }\AttributeTok{linetype =} \StringTok{"dashed"}\NormalTok{, }\AttributeTok{color =} \StringTok{"gray"}\NormalTok{) }\SpecialCharTok{+}
  \FunctionTok{geom\_vline}\NormalTok{(}\AttributeTok{xintercept =} \DecValTok{0}\NormalTok{, }\AttributeTok{linetype =} \StringTok{"dashed"}\NormalTok{, }\AttributeTok{color =} \StringTok{"gray"}\NormalTok{) }\SpecialCharTok{+}
  \FunctionTok{geom\_point}\NormalTok{(}\AttributeTok{alpha =} \FloatTok{0.3}\NormalTok{) }\SpecialCharTok{+}
  \FunctionTok{geom\_smooth}\NormalTok{(}\AttributeTok{method =} \StringTok{"lm"}\NormalTok{, }\AttributeTok{se =} \ConstantTok{FALSE}\NormalTok{, }\AttributeTok{color =} \StringTok{"red"}\NormalTok{, }\AttributeTok{size =} \FloatTok{0.8}\NormalTok{) }\SpecialCharTok{+}
  \FunctionTok{labs}\NormalTok{(}\AttributeTok{title =} \StringTok{"B. Scatter de Moran (Resíduos SDEM)"}\NormalTok{, }
       \AttributeTok{subtitle =} \FunctionTok{paste0}\NormalTok{(}\StringTok{"I de Moran: "}\NormalTok{, }\FunctionTok{round}\NormalTok{(moran\_sdem}\SpecialCharTok{$}\NormalTok{estimate[}\DecValTok{1}\NormalTok{], }\DecValTok{3}\NormalTok{), }
                         \StringTok{" (p{-}valor: "}\NormalTok{, }\FunctionTok{round}\NormalTok{(moran\_sdem}\SpecialCharTok{$}\NormalTok{p.value, }\DecValTok{3}\NormalTok{), }\StringTok{")"}\NormalTok{),}
       \AttributeTok{x =} \StringTok{"Resíduos"}\NormalTok{, }\AttributeTok{y =} \StringTok{"Lag Espacial"}\NormalTok{) }\SpecialCharTok{+}
  \FunctionTok{theme\_minimal}\NormalTok{()}


\NormalTok{(g\_fit }\SpecialCharTok{|}\NormalTok{ g\_resid\_scatter)}
\end{Highlighting}
\end{Shaded}

\begin{figure}[H]

\centering{

\pandocbounded{\includegraphics[keepaspectratio]{lattice_data_files/figure-pdf/fig-sdem-diagnostico-1.pdf}}

}

\caption{\label{fig-sdem-diagnostico}Diagnóstico SDEM: (A) Impactos
Locais (Sem Feedback Global), (B) Ajuste e (C) Resíduos.}

\end{figure}%

\subsection{\texorpdfstring{Modelo Autorregressivo Espacial com Erros
Autorregressivos (SARAR-\emph{Spatial Autoregressive with Autoregressive
Disturbances})}{Modelo Autorregressivo Espacial com Erros Autorregressivos (SARAR-Spatial Autoregressive with Autoregressive Disturbances)}}\label{modelo-autorregressivo-espacial-com-erros-autorregressivos-sarar-spatial-autoregressive-with-autoregressive-disturbances}

O Modelo Autorregressivo Espacial com Erros Autorregressivos (SARAR),
também referido como modelo SAC (\emph{Spatial Autoregressive Combined})
ou modelo generalizado de Cliff-Ord, é uma especificação abrangente que
modela simultaneamente a dependência espacial na variável resposta e a
autocorrelação espacial no termo de erro.

No contexto do Modelo Geral Espacial (GNS, Eq.~\ref{eq-GNS}), o SARAR é
obtido ao impor a restrição \(\boldsymbol{\theta} = \mathbf{0}\),
mantendo os parâmetros \(\rho \neq 0\) e \(\lambda \neq 0\).
Formalmente, o modelo é definido pelo sistema de equações (Anselin 1988;
H. H. Kelejian e Prucha 1998):

\[
\begin{aligned}
\mathbf{y} &= \rho \mathbf{W}_1 \mathbf{y} + \mathbf{X}\boldsymbol{\beta} + \mathbf{u}, \quad
\mathbf{u} = \lambda \mathbf{W}_2 \mathbf{u} + \boldsymbol{\epsilon},
\end{aligned}
\]

com erros independentes e identicamente distribuídos
\(\boldsymbol{\epsilon} \sim (\mathbf{0}, \sigma^2 \mathbf{I}_n)\). As
matrizes de pesos espaciais \(\mathbf{W}_1\) e \(\mathbf{W}_2\) podem
ser distintas, refletindo diferentes estruturas de interação para a
variável dependente e para os erros, embora a especificação com
\(\mathbf{W}_1 = \mathbf{W}_2 = \mathbf{W}\) seja comum por parcimônia.

Na forma escalar, para uma unidade \(i\):

\[
y_i = \rho \sum_{j=1}^n w_{1,ij} y_j + \sum_{k=1}^K x_{ik}\beta_k + u_i, \quad u_i = \lambda \sum_{j=1}^n w_{2,ij} u_j + \epsilon_i.
\]

A especificação SARAR é motivada pela necessidade de distinguir e
controlar dois processos espaciais operando conjuntamente: 1) um
mecanismo de interação substantiva ou de \emph{feedback}, onde o
resultado de uma unidade é diretamente influenciado pelos resultados de
suas vizinhas (capturado por \(\rho\)); e 2) um mecanismo de dependência
residual, onde fatores não observados ou choques exógenos exibem padrão
espacial (capturado por \(\lambda\)).

A identificação conjunta dos parâmetros \(\rho\) e \(\lambda\) é
assegurada pela não-linearidade da função de verossimilhança, embora o
uso de matrizes de pesos distintas (\(\mathbf{W}_1 \neq \mathbf{W}_2\))
possa melhorar as propriedades de identificação ao introduzir variação
exógena adicional (Anselin 1988).

Assumindo que \((\mathbf{I}_n - \rho \mathbf{W}_1)\) e
\((\mathbf{I}_n - \lambda \mathbf{W}_2)\) são não singulares, a forma
reduzida do modelo é obtida por substituição:

\[
\mathbf{y} = (\mathbf{I}_n - \rho \mathbf{W}_1)^{-1}\mathbf{X}\boldsymbol{\beta} + (\mathbf{I}_n - \rho \mathbf{W}_1)^{-1}(\mathbf{I}_n - \lambda \mathbf{W}_2)^{-1}\boldsymbol{\epsilon}.
\]

Esta expressão revela a estrutura do processo gerador de dados:

\begin{itemize}
\item
  \(\mathbb{E}[\mathbf{y} \, | \, \mathbf{X}] = (\mathbf{I}_n - \rho \mathbf{W}_1)^{-1}\mathbf{X}\boldsymbol{\beta}\).
  A interpretação dos efeitos marginais das variáveis explicativas segue
  a mesma lógica do modelo SAR, com a decomposição em efeitos diretos,
  indiretos e totais via o multiplicador espacial
  \((\mathbf{I}_n - \rho \mathbf{W}_1)^{-1}\) (J. LeSage e Pace 2009).
\item
  \(\operatorname{Cov}(\mathbf{y} \, | \, \mathbf{X}) = \sigma^2 [(\mathbf{I}_n - \rho \mathbf{W}_1)^{\top}(\mathbf{I}_n - \lambda \mathbf{W}_2)^{\top}(\mathbf{I}_n - \lambda \mathbf{W}_2)(\mathbf{I}_n - \rho \mathbf{W}_1)]^{-1}\).
  A estrutura de dependência estocástica resulta de uma dupla filtragem
  espacial: os choques \(\boldsymbol{\epsilon}\) são primeiro filtrados
  pelo processo de erro (\(\lambda\)) e depois propagados pelo mecanismo
  autorregressivo da variável dependente (\(\rho\)).
\end{itemize}

A presença simultânea da variável dependente defasada
(\(\mathbf{W}_1\mathbf{y}\)) e da autocorrelação nos erros torna o
estimador de Mínimos Quadrados Ordinários (MQO) inconsistente e
ineficiente. Os métodos de estimação padrão são:

\begin{enumerate}
\def\labelenumi{\arabic{enumi}.}
\tightlist
\item
  máxima verossimilhança, onde Sob a suposição de normalidade dos erros,
  a função de log-verossimilhança é:
\end{enumerate}

\[
\begin{aligned}
\ln L(\boldsymbol{\beta}, \rho, \lambda, \sigma^2) &= C + \ln|\mathbf{I}_n - \rho \mathbf{W}_1| + \ln|\mathbf{I}_n - \lambda \mathbf{W}_2| - \frac{n}{2}\ln(\sigma^2) - \frac{1}{2\sigma^2} \boldsymbol{\epsilon}(\rho, \lambda)^{\top} \boldsymbol{\epsilon}(\rho, \lambda),
\end{aligned}
\] onde
\(\boldsymbol{\epsilon}(\rho, \lambda) = (\mathbf{I}_n - \lambda \mathbf{W}_2)[(\mathbf{I}_n - \rho \mathbf{W}_1)\mathbf{y} - \mathbf{X}\boldsymbol{\beta}]\).
A maximização requer o cálculo de dois
\href{https://pt.wikipedia.org/wiki/Matriz_jacobiana}{determinantes
Jacobianos}, o que pode ser computacionalmente intensivo para grandes
amostras.

\begin{enumerate}
\def\labelenumi{\arabic{enumi}.}
\setcounter{enumi}{1}
\item
  Método Generalizado dos Momentos (GMM) / Mínimos Quadrados em Dois
  Estágios Espaciais Generalizados (GS2SLS), desenvolvida por H. H.
  Kelejian e Prucha (1998) e H. H. Kelejian e Prucha (1999), não exige
  suposições distribucionais fortes e evita o cálculo de determinantes.
  O procedimento é iterativo:

  \begin{enumerate}
  \def\labelenumii{\alph{enumii}.}
  \item
    No primeiro estágio (2SLS) estima-se a equação estrutural ignorando
    inicialmente a autocorrelação do erro. Utilizam-se como instrumentos
    para \(\mathbf{W}_1\mathbf{y}\) as variáveis
    \(\mathbf{H} = [\mathbf{X}, \mathbf{W}_1\mathbf{X}, \mathbf{W}_1^2\mathbf{X}, \dots]\),
    obtendo estimativas consistentes de \(\rho\) e
    \(\boldsymbol{\beta}\) e os resíduos \(\hat{\mathbf{u}}\).
  \item
    Estimação de \(\lambda\) utilizando um estimador GMM baseado nos
    momentos dos resíduos \(\hat{\mathbf{u}}\) para obter uma estimativa
    consistente de \(\lambda\).
  \item
    Transformação e estimação, na qual aplica-se a
    \href{https://en.wikipedia.org/wiki/Cochrane\%E2\%80\%93Orcutt_estimation}{transformação
    de Cochrane-Orcutt} espacial aos dados para eliminar a
    autocorrelação, usando \(\hat{\lambda}\). A equação transformada é
    então reestimada via 2SLS, produzindo estimativas finais eficientes.
  \end{enumerate}
\end{enumerate}

O modelo SARAR é a especificação preferencial quando evidências
empíricas (testes de Multiplicador de Lagrange robustos) indicam a
presença conjunta de dependência espacial substantiva e dependência
residual. Ele oferece um controle robusto para a heterogeneidade
espacial não observada enquanto captura os mecanismos de interação de
interesse.

\begin{table}

\caption{\label{tbl-sarar_comparacao}Comparação Completa (OLS, SLX, SAR,
SEM, SDM, SDEM, SARAR). Estimativas {[}IC 95\%{]}.}

\centering{

\begin{Shaded}
\begin{Highlighting}[]
\ControlFlowTok{if}\NormalTok{ (}\SpecialCharTok{!}\FunctionTok{require}\NormalTok{(}\StringTok{"pacman"}\NormalTok{)) }\FunctionTok{install.packages}\NormalTok{(}\StringTok{"pacman"}\NormalTok{)}
\NormalTok{pacman}\SpecialCharTok{::}\FunctionTok{p\_load}\NormalTok{(spatialreg, spdep, sf, modelsummary, kableExtra, dplyr, ggplot2, patchwork, viridis, ggspatial, tidyr, Matrix)}

\CommentTok{\# Ajuste do SARAR (SAC)}
\CommentTok{\# Estima rho e lambda simultaneamente}
\NormalTok{mod\_sarar }\OtherTok{\textless{}{-}} \FunctionTok{sacsarlm}\NormalTok{(taxa\_bruta }\SpecialCharTok{\textasciitilde{}}\NormalTok{ variavel\_x, }\AttributeTok{data =}\NormalTok{ mg\_dados, }\AttributeTok{listw =}\NormalTok{ lw)}

\NormalTok{mapa\_vars }\OtherTok{\textless{}{-}} \FunctionTok{c}\NormalTok{(}
  \StringTok{"(Intercept)"}    \OtherTok{=} \StringTok{"Intercepto"}\NormalTok{,}
  \StringTok{"variavel\_x"}     \OtherTok{=} \StringTok{"$}\SpecialCharTok{\textbackslash{}\textbackslash{}}\StringTok{beta$"}\NormalTok{,}
  \StringTok{"lag.variavel\_x"} \OtherTok{=} \StringTok{"WX $}\SpecialCharTok{\textbackslash{}\textbackslash{}}\StringTok{theta$"}\NormalTok{,      }
  \StringTok{"rho"}            \OtherTok{=} \StringTok{"$}\SpecialCharTok{\textbackslash{}\textbackslash{}}\StringTok{rho$"}\NormalTok{,      }
  \StringTok{"lambda"}         \OtherTok{=} \StringTok{"$}\SpecialCharTok{\textbackslash{}\textbackslash{}}\StringTok{lambda$"}     
\NormalTok{)}

\NormalTok{mapa\_gof }\OtherTok{\textless{}{-}} \FunctionTok{list}\NormalTok{(}
  \FunctionTok{list}\NormalTok{(}\StringTok{"raw"} \OtherTok{=} \StringTok{"nobs"}\NormalTok{, }\StringTok{"clean"} \OtherTok{=} \StringTok{"N"}\NormalTok{, }\StringTok{"fmt"} \OtherTok{=} \DecValTok{0}\NormalTok{),}
  \FunctionTok{list}\NormalTok{(}\StringTok{"raw"} \OtherTok{=} \StringTok{"r.squared"}\NormalTok{, }\StringTok{"clean"} \OtherTok{=} \StringTok{"$R\^{}2$"}\NormalTok{, }\StringTok{"fmt"} \OtherTok{=} \DecValTok{3}\NormalTok{),}
  \FunctionTok{list}\NormalTok{(}\StringTok{"raw"} \OtherTok{=} \StringTok{"aic"}\NormalTok{, }\StringTok{"clean"} \OtherTok{=} \StringTok{"AIC"}\NormalTok{, }\StringTok{"fmt"} \OtherTok{=} \DecValTok{1}\NormalTok{),}
  \FunctionTok{list}\NormalTok{(}\StringTok{"raw"} \OtherTok{=} \StringTok{"logLik"}\NormalTok{, }\StringTok{"clean"} \OtherTok{=} \StringTok{"Log Likelihood"}\NormalTok{, }\StringTok{"fmt"} \OtherTok{=} \DecValTok{1}\NormalTok{)}
\NormalTok{)}

\FunctionTok{modelsummary}\NormalTok{(}
  \FunctionTok{list}\NormalTok{(}
    \StringTok{"OLS"}   \OtherTok{=}\NormalTok{ mod\_ols, }
    \StringTok{"SLX"}   \OtherTok{=}\NormalTok{ mod\_slx,}
    \StringTok{"SAR"}   \OtherTok{=}\NormalTok{ mod\_sar, }
    \StringTok{"SEM"}   \OtherTok{=}\NormalTok{ mod\_sem,}
    \StringTok{"SDM"}   \OtherTok{=}\NormalTok{ mod\_sdm,}
    \StringTok{"SDEM"}  \OtherTok{=}\NormalTok{ mod\_sdem,}
    \StringTok{"SARAR"} \OtherTok{=}\NormalTok{ mod\_sarar}
\NormalTok{  ),}
  \AttributeTok{coef\_map =}\NormalTok{ mapa\_vars,      }
  \AttributeTok{gof\_map =}\NormalTok{ mapa\_gof,      }
  \AttributeTok{estimate =} \StringTok{"\{estimate\} [\{conf.low\}, \{conf.high\}]"}\NormalTok{,}
  \AttributeTok{statistic =} \ConstantTok{NULL}\NormalTok{, }
  \AttributeTok{stars =} \FunctionTok{c}\NormalTok{(}\StringTok{\textquotesingle{}*\textquotesingle{}} \OtherTok{=}\NormalTok{ .}\DecValTok{05}\NormalTok{, }\StringTok{\textquotesingle{}**\textquotesingle{}} \OtherTok{=}\NormalTok{ .}\DecValTok{01}\NormalTok{, }\StringTok{\textquotesingle{}***\textquotesingle{}} \OtherTok{=}\NormalTok{ .}\DecValTok{001}\NormalTok{),}
  \AttributeTok{title =} \ConstantTok{NULL}\NormalTok{,     }
  \AttributeTok{output =} \StringTok{"kableExtra"}\NormalTok{, }
  \AttributeTok{escape =} \ConstantTok{FALSE}
\NormalTok{) }\SpecialCharTok{\%\textgreater{}\%}
  \FunctionTok{kable\_styling}\NormalTok{(}\AttributeTok{latex\_options =} \FunctionTok{c}\NormalTok{(}\StringTok{"HOLD\_position"}\NormalTok{), }
                \AttributeTok{full\_width =} \ConstantTok{FALSE}\NormalTok{, }
                \AttributeTok{position =} \StringTok{"center"}\NormalTok{) }\SpecialCharTok{\%\textgreater{}\%}
  \FunctionTok{row\_spec}\NormalTok{(}\FunctionTok{c}\NormalTok{(}\DecValTok{5}\NormalTok{, }\DecValTok{7}\NormalTok{, }\DecValTok{9}\NormalTok{), }\AttributeTok{bold =} \ConstantTok{TRUE}\NormalTok{) }\SpecialCharTok{\%\textgreater{}\%} 
  \FunctionTok{as.character}\NormalTok{() }\SpecialCharTok{\%\textgreater{}\%}
  \FunctionTok{cat}\NormalTok{()}
\end{Highlighting}
\end{Shaded}

\centering\centering
\begin{tabular}[t]{lccccccc}
\toprule
  & OLS & SLX & SAR & SEM & SDM & SDEM & SARAR\\
\midrule
Intercepto & \num{8.973} [\num{8.616}, \num{9.330}] & \num{8.970} [\num{8.613}, \num{9.327}] & \num{5.751} [\num{4.876}, \num{6.626}] & \num{8.978} [\num{8.449}, \num{9.507}] & \num{5.748} [\num{4.873}, \num{6.622}] & \num{8.974} [\num{8.444}, \num{9.503}] & \num{2.715} [\num{1.695}, \num{3.735}]\\
$\beta$ & \num{0.058} [\num{-0.306}, \num{0.422}] & \num{0.056} [\num{-0.308}, \num{0.420}] & \num{0.029} [\num{-0.316}, \num{0.374}] & \num{-0.001} [\num{-0.342}, \num{0.340}] & \num{0.027} [\num{-0.318}, \num{0.372}] & \num{0.060} [\num{-0.297}, \num{0.417}] & \num{0.065} [\num{-0.243}, \num{0.373}]\\
WX $\theta$ &  & \num{0.397} [\num{-0.396}, \num{1.190}] &  &  & \num{0.391} [\num{-0.361}, \num{1.143}] & \num{0.496} [\num{-0.366}, \num{1.359}] & \\
$\rho$ &  &  & \num{0.359} [\num{0.269}, \num{0.449}] &  & \num{0.359} [\num{0.270}, \num{0.449}] &  & \num{0.698} [\num{0.587}, \num{0.809}]\\
\textbf{$\lambda$} & \textbf{} & \textbf{} & \textbf{} & \textbf{\num{0.359} [\num{0.270}, \num{0.449}]} & \textbf{} & \textbf{\num{0.360} [\num{0.270}, \num{0.450}]} & \textbf{\num{-0.558} [\num{-0.767}, \num{-0.349}]}\\
\midrule
N & \num{853} & \num{853} &  &  &  &  & \\
\textbf{$R^2$} & \textbf{\num{0.000}} & \textbf{\num{0.001}} & \textbf{} & \textbf{} & \textbf{} & \textbf{} & \textbf{}\\
AIC & \num{5275.0} & \num{5276.1} & \num{5212.9} & \num{5213.0} & \num{5213.9} & \num{5213.7} & \num{5200.4}\\
\textbf{Log Likelihood} & \textbf{\num{-2634.5}} & \textbf{\num{-2634.0}} & \textbf{} & \textbf{} & \textbf{} & \textbf{} & \textbf{}\\
\bottomrule
\multicolumn{8}{l}{\rule{0pt}{1em}* p $<$ 0.05, ** p $<$ 0.01, *** p $<$ 0.001}\\
\end{tabular}

}

\end{table}%

Enquanto as especificações de dependência única (SAR, SEM) e suas
generalizações com defasagens exógenas (SDM, SDEM) atingiram um patamar
de ajuste estacionário, com AICs oscilando em torno de \(5213\)
(Tabela~\ref{tbl-sarar_comparacao}), o modelo SARAR
(Tabela~\ref{tbl-sarar_comparacao} ) reduziu o critério de informação
para \(5200.4\). Esta melhoria no ajuste é suportada pela significância
estatística simultânea de ambos os parâmetros de dependência: uma forte
autoregressão espacial positiva capturada por \(\rho\) (\(0.698\);
\(IC_{95\%} [0.587, 0.809]\)) e um ajuste corretivo nos resíduos
evidenciado por um \(\lambda\) negativo significante (\(-0.558\);
\(IC_{95\%} [-0.767, -0.349]\)). Adicionalmente, a tabela reforça a
robustez da inferência sobre a covariável \(X\): em todas as sete
especificações testadas desde o OLS básico até o complexo SARAR o
coeficiente \(\beta\) e seus transbordamentos locais (\(\theta\))
permaneceram estatisticamente nulos, consolidando a evidência de que a
estrutura dos dados é dominada por dinâmicas espaciais endógenas e
estocásticas, independentes da variável explicativa proposta.

\begin{Shaded}
\begin{Highlighting}[]
\FunctionTok{set.seed}\NormalTok{(}\DecValTok{123}\NormalTok{)}
\NormalTok{imp\_sarar }\OtherTok{\textless{}{-}} \FunctionTok{impacts}\NormalTok{(mod\_sarar, }\AttributeTok{listw =}\NormalTok{ lw, }\AttributeTok{R =} \DecValTok{1000}\NormalTok{)}

\ControlFlowTok{if}\NormalTok{ (}\FunctionTok{is.null}\NormalTok{(imp\_sarar}\SpecialCharTok{$}\NormalTok{res)) \{}
\NormalTok{  imp\_sarar }\OtherTok{\textless{}{-}} \FunctionTok{impacts}\NormalTok{(mod\_sarar, }\AttributeTok{listw =}\NormalTok{ lw, }\AttributeTok{R =} \DecValTok{1000}\NormalTok{, }\AttributeTok{zstats =} \ConstantTok{TRUE}\NormalTok{)}
\NormalTok{\}}

\NormalTok{df\_impactos }\OtherTok{\textless{}{-}} \FunctionTok{data.frame}\NormalTok{(}
  \AttributeTok{direct =}\NormalTok{ imp\_sarar}\SpecialCharTok{$}\NormalTok{res}\SpecialCharTok{$}\NormalTok{direct,}
  \AttributeTok{indirect =}\NormalTok{ imp\_sarar}\SpecialCharTok{$}\NormalTok{res}\SpecialCharTok{$}\NormalTok{indirect}
\NormalTok{) }\SpecialCharTok{\%\textgreater{}\%}
  \FunctionTok{pivot\_longer}\NormalTok{(}\AttributeTok{cols =} \FunctionTok{everything}\NormalTok{(), }\AttributeTok{names\_to =} \StringTok{"Tipo"}\NormalTok{, }\AttributeTok{values\_to =} \StringTok{"Valor"}\NormalTok{) }\SpecialCharTok{\%\textgreater{}\%}
  \FunctionTok{mutate}\NormalTok{(}\AttributeTok{Tipo =} \FunctionTok{factor}\NormalTok{(Tipo, }\AttributeTok{levels =} \FunctionTok{c}\NormalTok{(}\StringTok{"direct"}\NormalTok{, }\StringTok{"indirect"}\NormalTok{),}
                       \AttributeTok{labels =} \FunctionTok{c}\NormalTok{(}\StringTok{"Direto"}\NormalTok{, }\StringTok{"Indireto (Spillover Global)"}\NormalTok{)))}


\NormalTok{g\_impactos }\OtherTok{\textless{}{-}} \FunctionTok{ggplot}\NormalTok{(df\_impactos, }\FunctionTok{aes}\NormalTok{(}\AttributeTok{x =}\NormalTok{ Tipo, }\AttributeTok{fill =}\NormalTok{ Tipo, }\AttributeTok{y=}\NormalTok{Valor)) }\SpecialCharTok{+}
  \FunctionTok{geom\_col}\NormalTok{(}\AttributeTok{width =} \FloatTok{0.2}\NormalTok{, }\AttributeTok{color =} \StringTok{"gray30"}\NormalTok{) }\SpecialCharTok{+}
  \FunctionTok{scale\_fill\_manual}\NormalTok{(}\AttributeTok{values =} \FunctionTok{c}\NormalTok{(}\StringTok{"Direto"} \OtherTok{=} \StringTok{"\#1b9e77"}\NormalTok{, }\StringTok{"Indireto (Spillover)"} \OtherTok{=} \StringTok{"\#d95f02"}\NormalTok{)) }\SpecialCharTok{+}
  \FunctionTok{labs}\NormalTok{(}\AttributeTok{title =} \StringTok{"A. Impactos: Direto vs. Indireto"}\NormalTok{, }
       \AttributeTok{y =} \StringTok{"Magnitude do efeito"}\NormalTok{, }\AttributeTok{x =} \ConstantTok{NULL}\NormalTok{) }\SpecialCharTok{+}
  \FunctionTok{theme\_minimal}\NormalTok{() }\SpecialCharTok{+} 
  \FunctionTok{theme}\NormalTok{(}\AttributeTok{legend.position =} \StringTok{"none"}\NormalTok{, }
        \AttributeTok{legend.title =} \FunctionTok{element\_blank}\NormalTok{())}

\CommentTok{\# Mapa dos Valores Ajustados}
\NormalTok{mg\_dados}\SpecialCharTok{$}\NormalTok{fitted\_sarar }\OtherTok{\textless{}{-}} \FunctionTok{fitted}\NormalTok{(mod\_sarar)}

\NormalTok{g\_fit }\OtherTok{\textless{}{-}} \FunctionTok{ggplot}\NormalTok{(mg\_dados) }\SpecialCharTok{+}
  \FunctionTok{geom\_sf}\NormalTok{(}\FunctionTok{aes}\NormalTok{(}\AttributeTok{fill =}\NormalTok{ fitted\_sarar), }\AttributeTok{color =} \ConstantTok{NA}\NormalTok{) }\SpecialCharTok{+}
  \FunctionTok{scale\_fill\_viridis\_c}\NormalTok{(}\AttributeTok{option =} \StringTok{"turbo"}\NormalTok{, }\AttributeTok{name =} \StringTok{"Predito"}\NormalTok{) }\SpecialCharTok{+}
 \FunctionTok{labs}\NormalTok{(}
  \AttributeTok{title =} \StringTok{"B. Valores Preditos (SARAR)"}\NormalTok{, }
  \AttributeTok{subtitle =} \FunctionTok{expression}\NormalTok{(}\StringTok{"Ajuste simultâneo"} \SpecialCharTok{\textasciitilde{}}\NormalTok{ (rho }\SpecialCharTok{+}\NormalTok{ lambda))}
\NormalTok{) }\SpecialCharTok{+}
\FunctionTok{theme\_minimal}\NormalTok{() }\SpecialCharTok{+} 
  \FunctionTok{annotation\_scale}\NormalTok{(}\AttributeTok{location =} \StringTok{"bl"}\NormalTok{, }\AttributeTok{width\_hint =} \FloatTok{0.3}\NormalTok{, }\AttributeTok{bar\_cols =} \FunctionTok{c}\NormalTok{(}\StringTok{"black"}\NormalTok{, }\StringTok{"white"}\NormalTok{)) }\SpecialCharTok{+}
  \FunctionTok{annotation\_north\_arrow}\NormalTok{(}\AttributeTok{location =} \StringTok{"tl"}\NormalTok{, }\AttributeTok{style =}\NormalTok{ north\_arrow\_fancy\_orienteering,}
                         \AttributeTok{pad\_x =} \FunctionTok{unit}\NormalTok{(}\FloatTok{0.1}\NormalTok{, }\StringTok{"in"}\NormalTok{), }\AttributeTok{pad\_y =} \FunctionTok{unit}\NormalTok{(}\FloatTok{0.1}\NormalTok{, }\StringTok{"in"}\NormalTok{))}

\CommentTok{\# Diagnóstico dos Resíduos}
\NormalTok{mg\_dados}\SpecialCharTok{$}\NormalTok{resid\_sarar }\OtherTok{\textless{}{-}} \FunctionTok{residuals}\NormalTok{(mod\_sarar)}
\NormalTok{moran\_sarar }\OtherTok{\textless{}{-}} \FunctionTok{moran.test}\NormalTok{(mg\_dados}\SpecialCharTok{$}\NormalTok{resid\_sarar, lw)}
\NormalTok{mg\_dados}\SpecialCharTok{$}\NormalTok{resid\_lag\_sarar }\OtherTok{\textless{}{-}} \FunctionTok{lag.listw}\NormalTok{(lw, mg\_dados}\SpecialCharTok{$}\NormalTok{resid\_sarar)}

\NormalTok{g\_resid\_scatter }\OtherTok{\textless{}{-}} \FunctionTok{ggplot}\NormalTok{(mg\_dados, }\FunctionTok{aes}\NormalTok{(}\AttributeTok{x =}\NormalTok{ resid\_sarar, }\AttributeTok{y =}\NormalTok{ resid\_lag\_sarar)) }\SpecialCharTok{+}
  \FunctionTok{geom\_hline}\NormalTok{(}\AttributeTok{yintercept =} \DecValTok{0}\NormalTok{, }\AttributeTok{linetype =} \StringTok{"dashed"}\NormalTok{, }\AttributeTok{color =} \StringTok{"gray"}\NormalTok{) }\SpecialCharTok{+}
  \FunctionTok{geom\_vline}\NormalTok{(}\AttributeTok{xintercept =} \DecValTok{0}\NormalTok{, }\AttributeTok{linetype =} \StringTok{"dashed"}\NormalTok{, }\AttributeTok{color =} \StringTok{"gray"}\NormalTok{) }\SpecialCharTok{+}
  \FunctionTok{geom\_point}\NormalTok{(}\AttributeTok{alpha =} \FloatTok{0.3}\NormalTok{) }\SpecialCharTok{+}
  \FunctionTok{geom\_smooth}\NormalTok{(}\AttributeTok{method =} \StringTok{"lm"}\NormalTok{, }\AttributeTok{se =} \ConstantTok{FALSE}\NormalTok{, }\AttributeTok{color =} \StringTok{"red"}\NormalTok{, }\AttributeTok{size =} \FloatTok{0.8}\NormalTok{) }\SpecialCharTok{+}
  \FunctionTok{labs}\NormalTok{(}\AttributeTok{title =} \StringTok{"C. Scatter de Moran (Resíduos SARAR)"}\NormalTok{, }
       \AttributeTok{subtitle =} \FunctionTok{paste0}\NormalTok{(}\StringTok{"I de Moran: "}\NormalTok{, }\FunctionTok{round}\NormalTok{(moran\_sarar}\SpecialCharTok{$}\NormalTok{estimate[}\DecValTok{1}\NormalTok{], }\DecValTok{3}\NormalTok{), }
                         \StringTok{" (p{-}valor: "}\NormalTok{, }\FunctionTok{round}\NormalTok{(moran\_sarar}\SpecialCharTok{$}\NormalTok{p.value, }\DecValTok{3}\NormalTok{), }\StringTok{")"}\NormalTok{),}
       \AttributeTok{x =} \StringTok{"Resíduos"}\NormalTok{, }\AttributeTok{y =} \StringTok{"Lag Espacial"}\NormalTok{) }\SpecialCharTok{+}
  \FunctionTok{theme\_minimal}\NormalTok{()}


\NormalTok{(g\_impactos }\SpecialCharTok{|}\NormalTok{ g\_fit }\SpecialCharTok{|}\NormalTok{ g\_resid\_scatter)}
\end{Highlighting}
\end{Shaded}

\begin{figure}[H]

\centering{

\pandocbounded{\includegraphics[keepaspectratio]{lattice_data_files/figure-pdf/fig-sarar-diagnostico-1.pdf}}

}

\caption{\label{fig-sarar-diagnostico}Diagnóstico SARAR: (A) Impactos
Globais, (B) Ajuste e (C) Resíduos.}

\end{figure}%

\subsection{\texorpdfstring{Modelo de Média Móvel Espacial (SMA --
\emph{Spatial Moving
Average})}{Modelo de Média Móvel Espacial (SMA -- Spatial Moving Average)}}\label{modelo-de-muxe9dia-muxf3vel-espacial-sma-spatial-moving-average}

O modelo de Média Móvel Espacial (SMA) oferece uma abordagem alternativa
para a modelagem da dependência espacial no termo de erro. Enquanto o
modelo SEM especifica um processo autorregressivo que induz uma
dependência de longo alcance, o SMA postula que a dependência espacial é
um fenômeno estritamente local e de curto alcance.

Embora menos comum na literatura aplicada que os modelos SAR ou SEM, o
SMA é relevante para testar hipóteses sobre a extensão geográfica das
interdependências. O modelo é especificado pela seguinte estrutura
(Anselin 1988; R. P. Haining 2003):

\[
\mathbf{y} = \mathbf{X}\boldsymbol{\beta} + \mathbf{u}, \quad \mathbf{u} = \boldsymbol{\epsilon} + \lambda \mathbf{W}\boldsymbol{\epsilon},
\]

com erros independentes e identicamente distribuídos
\(\boldsymbol{\epsilon} \sim \mathcal{N}(\mathbf{0}, \sigma^2 \mathbf{I}_n)\).

Na forma escalar para uma unidade \(i\), o processo para o termo de erro
é dado por:

\[
u_i = \epsilon_i + \lambda \sum_{j=1}^n w_{ij} \epsilon_j.
\]

Onde \(\lambda\) é o coeficiente de média móvel espacial.

A distinção fundamental entre o SMA e os modelos autorregressivos (SAR,
SEM) reside na estrutura de propagação dos choques estocásticos.

\begin{enumerate}
\def\labelenumi{\arabic{enumi}.}
\item
  Nos modelos autorregressivos, a dependência é modelada através da
  inversa de uma matriz de filtro espacial, por exemplo,
  \((\mathbf{I}_n - \lambda \mathbf{W})^{-1}\). A expansão em
  \href{https://pt.wikipedia.org/wiki/S\%C3\%A9rie_de_Neumann}{série de
  Neumann} desta inversa,
  \((\mathbf{I}_n - \lambda \mathbf{W})^{-1} = \mathbf{I}_n + \lambda \mathbf{W} + \lambda^2 \mathbf{W}^2 + \cdots\),
  implica que um choque em uma unidade \(i\) afeta seus vizinhos diretos
  (\(\lambda \mathbf{W}\)), os vizinhos dos vizinhos
  (\(\lambda^2 \mathbf{W}^2\)), e assim sucessivamente, propagando-se
  por todo o sistema, ainda que com intensidade decrescente (Elhorst et
  al. 2014). Isso caracteriza uma dependência de longo alcance.
\item
  No modelo SMA, não há inversão de matriz na definição do erro
  \(\mathbf{u}\). Um choque \(\epsilon_i\) afeta diretamente a unidade
  \(i\) e, através do termo \(\lambda \mathbf{W}\boldsymbol{\epsilon}\),
  afeta apenas os vizinhos imediatos de \(i\) (aqueles para os quais
  \(w_{ij} \neq 0\)). O efeito é contido na primeira ordem de
  vizinhança; não há mecanismo de retransmissão para unidades mais
  distantes.
\end{enumerate}

A natureza local do SMA é explicitamente descrita por sua matriz de
covariância. Dado que
\(\mathbf{u} = (\mathbf{I}_n + \lambda \mathbf{W}) \boldsymbol{\epsilon}\),
a matriz de covariância de \(\mathbf{u}\) (condicional a \(\mathbf{X}\))
é:

\[
\operatorname{Cov}(\mathbf{u}) = \sigma^2 (\mathbf{I}_n + \lambda \mathbf{W})(\mathbf{I}_n + \lambda \mathbf{W})^{\top} = \sigma^2 (\mathbf{I}_n + \lambda \mathbf{W} + \lambda \mathbf{W}^{\top} + \lambda^2 \mathbf{W}\mathbf{W}^{\top}).
\]

Note que:

\begin{itemize}
\item
  A covariância entre \(u_i\) e \(u_j\) é não nula se \(w_{ij} \neq 0\)
  (são vizinhos diretos) ou se \(w_{ji} \neq 0\) (para \(\mathbf{W}\)
  não simétrica).
\item
  O elemento \((i, j)\) da matriz \(\mathbf{W}\mathbf{W}^{\top}\) é não
  nulo se existe pelo menos uma unidade \(k\) tal que \(w_{ik} \neq 0\)
  e \(w_{jk} \neq 0\). Ou seja, se as unidades \(i\) e \(j\)
  compartilham pelo menos um vizinho comum.
\end{itemize}

Portanto, no modelo SMA, a correlação espacial é efetivamente nula para
qualquer par de unidades que não sejam vizinhas diretas nem compartilhem
um vizinho comum (vizinhos de segunda ordem). Isto contrasta com o SEM,
onde a matriz de covariância é teoricamente densa, implicando correlação
não nula (embora pequena) entre todos os pares de unidades no sistema.

Assim como no SEM, o estimador de Mínimos Quadrados Ordinários (MQO)
para \(\boldsymbol{\beta}\) no modelo SMA é não viesado e consistente
sob a exogeneidade de \(\mathbf{X}\). No entanto, é ineficiente na
presença de autocorrelação espacial (\(\lambda \neq 0\)), e as
estimativas dos erros-padrão obtidas pelo procedimento MQO padrão são
viesadas, invalidando a inferência estatística habitual (Anselin 1988).

A estimação eficiente tipicamente requer o método de máxima
verossimilhança. A função de log-verossimilhança assume a forma:

\[
\ln L(\boldsymbol{\beta}, \lambda, \sigma^2) = C - \frac{1}{2} \ln |\sigma^2 \boldsymbol{\Omega}(\lambda)| - \frac{1}{2\sigma^2} (\mathbf{y} - \mathbf{X}\boldsymbol{\beta})^{\top} \boldsymbol{\Omega}(\lambda)^{-1} (\mathbf{y} - \mathbf{X}\boldsymbol{\beta}),
\]

onde
\(\boldsymbol{\Omega}(\lambda) = (\mathbf{I}_n + \lambda \mathbf{W})(\mathbf{I}_n + \lambda \mathbf{W})^{\top}\).
A complexidade computacional reside no cálculo do determinante e da
inversa de \(\boldsymbol{\Omega}(\lambda)\). Diferentemente dos modelos
autorregressivos, que envolvem o determinante de
\((\mathbf{I}_n - \lambda \mathbf{W})\), o SMA requer o cálculo do
determinante de \((\mathbf{I}_n + \lambda \mathbf{W})\).

Para garantir que o processo seja invertível e a matriz de covariância
seja positiva definida, são impostas restrições no espaço do parâmetro
\(\lambda\). Geralmente, exige-se que
\(|\lambda| < 1 / |\omega_{max}|\), onde \(\omega_{max}\) é o maior
autovalor de \(\mathbf{W}\) em módulo (Hepple 1979; Anselin 1988).

O uso do SMA é recomendado quando a análise exploratória dos dados (por
exemplo, correlogramas espaciais ou testes de dependência para
diferentes ordens de vizinhança) indica um decaimento abrupto da
correlação espacial após a primeira defasagem, em contraste com o
decaimento suave e prolongado característico dos processos
autorregressivos.

\begin{table}

\caption{\label{tbl-comparacao_final_8_modelos}Comparação entre ajuste
dos modelos OLS, SLX, SAR, SEM, SDM, SDEM, SARAR e SMA. Estimativas
{[}IC 95\%{]}.}

\centering{

\begin{Shaded}
\begin{Highlighting}[]
\ControlFlowTok{if}\NormalTok{ (}\SpecialCharTok{!}\FunctionTok{require}\NormalTok{(}\StringTok{"pacman"}\NormalTok{)) }\FunctionTok{install.packages}\NormalTok{(}\StringTok{"pacman"}\NormalTok{)}
\NormalTok{pacman}\SpecialCharTok{::}\FunctionTok{p\_load}\NormalTok{(spatialreg, spdep, sf, modelsummary, kableExtra, dplyr, ggplot2, patchwork, viridis, ggspatial, tidyr, Matrix)}


\ControlFlowTok{if}\NormalTok{ (}\SpecialCharTok{!}\FunctionTok{exists}\NormalTok{(}\StringTok{"mg\_dados"}\NormalTok{)) \{}
\NormalTok{  mg\_dados }\OtherTok{\textless{}{-}}\NormalTok{ geobr}\SpecialCharTok{::}\FunctionTok{read\_municipality}\NormalTok{(}\AttributeTok{code\_muni =} \StringTok{"MG"}\NormalTok{, }\AttributeTok{year =} \DecValTok{2020}\NormalTok{, }\AttributeTok{showProgress =} \ConstantTok{FALSE}\NormalTok{)}
\NormalTok{  coords }\OtherTok{\textless{}{-}} \FunctionTok{st\_coordinates}\NormalTok{(}\FunctionTok{st\_centroid}\NormalTok{(mg\_dados))}
  \FunctionTok{set.seed}\NormalTok{(}\DecValTok{123}\NormalTok{)}
\NormalTok{  mg\_dados}\SpecialCharTok{$}\NormalTok{taxa\_bruta }\OtherTok{\textless{}{-}}\NormalTok{ (}\SpecialCharTok{{-}}\NormalTok{coords[,}\DecValTok{2}\NormalTok{] }\SpecialCharTok{*} \DecValTok{10}\NormalTok{) }\SpecialCharTok{+} \FunctionTok{rnorm}\NormalTok{(}\FunctionTok{nrow}\NormalTok{(mg\_dados), }\DecValTok{0}\NormalTok{, }\DecValTok{5}\NormalTok{)}
\NormalTok{  mg\_dados}\SpecialCharTok{$}\NormalTok{variavel\_x }\OtherTok{\textless{}{-}} \FunctionTok{rnorm}\NormalTok{(}\FunctionTok{nrow}\NormalTok{(mg\_dados))}
\NormalTok{\}}

\ControlFlowTok{if}\NormalTok{ (}\SpecialCharTok{!}\FunctionTok{exists}\NormalTok{(}\StringTok{"lw"}\NormalTok{)) \{}
\NormalTok{   nb }\OtherTok{\textless{}{-}} \FunctionTok{poly2nb}\NormalTok{(mg\_dados, }\AttributeTok{queen =} \ConstantTok{TRUE}\NormalTok{)}
\NormalTok{   lw }\OtherTok{\textless{}{-}} \FunctionTok{nb2listw}\NormalTok{(nb, }\AttributeTok{style =} \StringTok{"W"}\NormalTok{, }\AttributeTok{zero.policy =} \ConstantTok{TRUE}\NormalTok{)}
\NormalTok{\}}


\NormalTok{mod\_sma }\OtherTok{\textless{}{-}} \FunctionTok{spautolm}\NormalTok{(taxa\_bruta }\SpecialCharTok{\textasciitilde{}}\NormalTok{ variavel\_x, }\AttributeTok{data =}\NormalTok{ mg\_dados, }\AttributeTok{listw =}\NormalTok{ lw, }\AttributeTok{family =} \StringTok{"SMA"}\NormalTok{)}

\NormalTok{sum\_sma }\OtherTok{\textless{}{-}} \FunctionTok{summary}\NormalTok{(mod\_sma)}

\CommentTok{\#Extrair Betas da matriz de coeficientes }
\NormalTok{coefs\_mat }\OtherTok{\textless{}{-}}\NormalTok{ sum\_sma}\SpecialCharTok{$}\NormalTok{Coef }\CommentTok{\# Matriz com Estimate, Std. Error, etc.}
\NormalTok{df\_tidy\_betas }\OtherTok{\textless{}{-}} \FunctionTok{data.frame}\NormalTok{(}
  \AttributeTok{term =} \FunctionTok{rownames}\NormalTok{(coefs\_mat),}
  \AttributeTok{estimate =}\NormalTok{ coefs\_mat[, }\StringTok{"Estimate"}\NormalTok{],}
  \AttributeTok{std.error =}\NormalTok{ coefs\_mat[, }\StringTok{"Std. Error"}\NormalTok{]}
\NormalTok{)}

\CommentTok{\# Extrair Lambda}
\NormalTok{df\_tidy\_lambda }\OtherTok{\textless{}{-}} \FunctionTok{data.frame}\NormalTok{(}
  \AttributeTok{term =} \StringTok{"lambda"}\NormalTok{,}
  \AttributeTok{estimate =}\NormalTok{ mod\_sma}\SpecialCharTok{$}\NormalTok{lambda,}
  \AttributeTok{std.error =}\NormalTok{ mod\_sma}\SpecialCharTok{$}\NormalTok{lambda.se}
\NormalTok{)}

\CommentTok{\#Unir tudo}
\NormalTok{df\_tidy\_sma }\OtherTok{\textless{}{-}} \FunctionTok{rbind}\NormalTok{(df\_tidy\_betas, df\_tidy\_lambda)}

\CommentTok{\#Calcular estatísticas finais}
\NormalTok{df\_tidy\_sma}\SpecialCharTok{$}\NormalTok{statistic }\OtherTok{\textless{}{-}}\NormalTok{ df\_tidy\_sma}\SpecialCharTok{$}\NormalTok{estimate }\SpecialCharTok{/}\NormalTok{ df\_tidy\_sma}\SpecialCharTok{$}\NormalTok{std.error}
\NormalTok{df\_tidy\_sma}\SpecialCharTok{$}\NormalTok{p.value }\OtherTok{\textless{}{-}} \DecValTok{2} \SpecialCharTok{*}\NormalTok{ (}\DecValTok{1} \SpecialCharTok{{-}} \FunctionTok{pnorm}\NormalTok{(}\FunctionTok{abs}\NormalTok{(df\_tidy\_sma}\SpecialCharTok{$}\NormalTok{statistic)))}
\NormalTok{df\_tidy\_sma}\SpecialCharTok{$}\NormalTok{conf.low }\OtherTok{\textless{}{-}}\NormalTok{ df\_tidy\_sma}\SpecialCharTok{$}\NormalTok{estimate }\SpecialCharTok{{-}}\NormalTok{ (}\FloatTok{1.96} \SpecialCharTok{*}\NormalTok{ df\_tidy\_sma}\SpecialCharTok{$}\NormalTok{std.error)}
\NormalTok{df\_tidy\_sma}\SpecialCharTok{$}\NormalTok{conf.high }\OtherTok{\textless{}{-}}\NormalTok{ df\_tidy\_sma}\SpecialCharTok{$}\NormalTok{estimate }\SpecialCharTok{+}\NormalTok{ (}\FloatTok{1.96} \SpecialCharTok{*}\NormalTok{ df\_tidy\_sma}\SpecialCharTok{$}\NormalTok{std.error)}

\CommentTok{\#}
\NormalTok{df\_glance\_sma }\OtherTok{\textless{}{-}} \FunctionTok{data.frame}\NormalTok{(}
  \AttributeTok{nobs =} \FunctionTok{length}\NormalTok{(}\FunctionTok{residuals}\NormalTok{(mod\_sma)),}
  \AttributeTok{logLik =} \FunctionTok{as.numeric}\NormalTok{(}\FunctionTok{logLik}\NormalTok{(mod\_sma)),}
  \AttributeTok{aic =} \FunctionTok{AIC}\NormalTok{(mod\_sma),}
  \AttributeTok{r.squared =} \ConstantTok{NA}
\NormalTok{)}

\CommentTok{\#}
\NormalTok{mod\_sma\_custom }\OtherTok{\textless{}{-}} \FunctionTok{list}\NormalTok{(}\AttributeTok{tidy =}\NormalTok{ df\_tidy\_sma, }\AttributeTok{glance =}\NormalTok{ df\_glance\_sma)}
\FunctionTok{class}\NormalTok{(mod\_sma\_custom) }\OtherTok{\textless{}{-}} \StringTok{"modelsummary\_list"}

\CommentTok{\#}
\NormalTok{mapa\_vars }\OtherTok{\textless{}{-}} \FunctionTok{c}\NormalTok{(}
  \StringTok{"(Intercept)"}    \OtherTok{=} \StringTok{"Intercepto"}\NormalTok{,}
  \StringTok{"variavel\_x"}     \OtherTok{=} \StringTok{"$}\SpecialCharTok{\textbackslash{}\textbackslash{}}\StringTok{beta$"}\NormalTok{,}
  \StringTok{"lag.variavel\_x"} \OtherTok{=} \StringTok{"WX $}\SpecialCharTok{\textbackslash{}\textbackslash{}}\StringTok{theta$"}\NormalTok{,      }
  \StringTok{"rho"}            \OtherTok{=} \StringTok{"$}\SpecialCharTok{\textbackslash{}\textbackslash{}}\StringTok{rho$"}\NormalTok{,      }
  \StringTok{"lambda"}         \OtherTok{=} \StringTok{"$}\SpecialCharTok{\textbackslash{}\textbackslash{}}\StringTok{lambda$"}     
\NormalTok{)}

\NormalTok{mapa\_gof }\OtherTok{\textless{}{-}} \FunctionTok{list}\NormalTok{(}
  \FunctionTok{list}\NormalTok{(}\StringTok{"raw"} \OtherTok{=} \StringTok{"nobs"}\NormalTok{, }\StringTok{"clean"} \OtherTok{=} \StringTok{"N"}\NormalTok{, }\StringTok{"fmt"} \OtherTok{=} \DecValTok{0}\NormalTok{),}
  \FunctionTok{list}\NormalTok{(}\StringTok{"raw"} \OtherTok{=} \StringTok{"r.squared"}\NormalTok{, }\StringTok{"clean"} \OtherTok{=} \StringTok{"$R\^{}2$"}\NormalTok{, }\StringTok{"fmt"} \OtherTok{=} \DecValTok{3}\NormalTok{),}
  \FunctionTok{list}\NormalTok{(}\StringTok{"raw"} \OtherTok{=} \StringTok{"aic"}\NormalTok{, }\StringTok{"clean"} \OtherTok{=} \StringTok{"AIC"}\NormalTok{, }\StringTok{"fmt"} \OtherTok{=} \DecValTok{1}\NormalTok{),}
  \FunctionTok{list}\NormalTok{(}\StringTok{"raw"} \OtherTok{=} \StringTok{"logLik"}\NormalTok{, }\StringTok{"clean"} \OtherTok{=} \StringTok{"Log Likelihood"}\NormalTok{, }\StringTok{"fmt"} \OtherTok{=} \DecValTok{1}\NormalTok{)}
\NormalTok{)}

\CommentTok{\# Tabela Final}
\FunctionTok{modelsummary}\NormalTok{(}
  \FunctionTok{list}\NormalTok{(}
    \StringTok{"OLS"}   \OtherTok{=}\NormalTok{ mod\_ols, }
    \StringTok{"SLX"}   \OtherTok{=}\NormalTok{ mod\_slx,}
    \StringTok{"SAR"}   \OtherTok{=}\NormalTok{ mod\_sar, }
    \StringTok{"SEM"}   \OtherTok{=}\NormalTok{ mod\_sem,}
    \StringTok{"SDM"}   \OtherTok{=}\NormalTok{ mod\_sdm,}
    \StringTok{"SDEM"}  \OtherTok{=}\NormalTok{ mod\_sdem,}
    \StringTok{"SARAR"} \OtherTok{=}\NormalTok{ mod\_sarar,}
    \StringTok{"SMA"}   \OtherTok{=}\NormalTok{ mod\_sma\_custom}
\NormalTok{  ),}
  \AttributeTok{coef\_map =}\NormalTok{ mapa\_vars,      }
  \AttributeTok{gof\_map =}\NormalTok{ mapa\_gof,      }
  \AttributeTok{estimate =} \StringTok{"\{estimate\} [\{conf.low\}, \{conf.high\}]"}\NormalTok{,}
  \AttributeTok{statistic =} \ConstantTok{NULL}\NormalTok{, }
  \AttributeTok{stars =} \FunctionTok{c}\NormalTok{(}\StringTok{\textquotesingle{}*\textquotesingle{}} \OtherTok{=}\NormalTok{ .}\DecValTok{05}\NormalTok{, }\StringTok{\textquotesingle{}**\textquotesingle{}} \OtherTok{=}\NormalTok{ .}\DecValTok{01}\NormalTok{, }\StringTok{\textquotesingle{}***\textquotesingle{}} \OtherTok{=}\NormalTok{ .}\DecValTok{001}\NormalTok{),}
  \AttributeTok{title =} \ConstantTok{NULL}\NormalTok{,     }
  \AttributeTok{output =} \StringTok{"kableExtra"}\NormalTok{, }
  \AttributeTok{escape =} \ConstantTok{FALSE}
\NormalTok{) }\SpecialCharTok{\%\textgreater{}\%}
  \FunctionTok{kable\_styling}\NormalTok{(}\AttributeTok{latex\_options =} \FunctionTok{c}\NormalTok{(}\StringTok{"HOLD\_position"}\NormalTok{, }\StringTok{"scale\_down"}\NormalTok{),}
                \AttributeTok{full\_width =} \ConstantTok{FALSE}\NormalTok{, }
                \AttributeTok{position =} \StringTok{"center"}\NormalTok{) }\SpecialCharTok{\%\textgreater{}\%}
  \FunctionTok{row\_spec}\NormalTok{(}\FunctionTok{c}\NormalTok{(}\DecValTok{5}\NormalTok{, }\DecValTok{7}\NormalTok{, }\DecValTok{9}\NormalTok{), }\AttributeTok{bold =} \ConstantTok{TRUE}\NormalTok{) }\SpecialCharTok{\%\textgreater{}\%} 
  \FunctionTok{as.character}\NormalTok{() }\SpecialCharTok{\%\textgreater{}\%}
  \FunctionTok{cat}\NormalTok{()}
\end{Highlighting}
\end{Shaded}

\centering\centering
\resizebox{\ifdim\width>\linewidth\linewidth\else\width\fi}{!}{
\begin{tabular}[t]{lcccccccc}
\toprule
  & OLS & SLX & SAR & SEM & SDM & SDEM & SARAR & SMA\\
\midrule
Intercepto & \num{8.973} [\num{8.616}, \num{9.330}] & \num{8.970} [\num{8.613}, \num{9.327}] & \num{5.751} [\num{4.876}, \num{6.626}] & \num{8.978} [\num{8.449}, \num{9.507}] & \num{5.748} [\num{4.873}, \num{6.622}] & \num{8.974} [\num{8.444}, \num{9.503}] & \num{2.715} [\num{1.695}, \num{3.735}] & \num{8.975} [\num{8.499}, \num{9.450}]\\
$\beta$ & \num{0.058} [\num{-0.306}, \num{0.422}] & \num{0.056} [\num{-0.308}, \num{0.420}] & \num{0.029} [\num{-0.316}, \num{0.374}] & \num{-0.001} [\num{-0.342}, \num{0.340}] & \num{0.027} [\num{-0.318}, \num{0.372}] & \num{0.060} [\num{-0.297}, \num{0.417}] & \num{0.065} [\num{-0.243}, \num{0.373}] & \num{0.001} [\num{-0.343}, \num{0.345}]\\
WX $\theta$ &  & \num{0.397} [\num{-0.396}, \num{1.190}] &  &  & \num{0.391} [\num{-0.361}, \num{1.143}] & \num{0.496} [\num{-0.366}, \num{1.359}] &  & \\
$\rho$ &  &  & \num{0.359} [\num{0.269}, \num{0.449}] &  & \num{0.359} [\num{0.270}, \num{0.449}] &  & \num{0.698} [\num{0.587}, \num{0.809}] & \\
\textbf{$\lambda$} & \textbf{} & \textbf{} & \textbf{} & \textbf{\num{0.359} [\num{0.270}, \num{0.449}]} & \textbf{} & \textbf{\num{0.360} [\num{0.270}, \num{0.450}]} & \textbf{\num{-0.558} [\num{-0.767}, \num{-0.349}]} & \textbf{\num{0.363}}\\
\midrule
N & \num{853} & \num{853} &  &  &  &  &  & \num{853}\\
\textbf{$R^2$} & \textbf{\num{0.000}} & \textbf{\num{0.001}} & \textbf{} & \textbf{} & \textbf{} & \textbf{} & \textbf{} & \textbf{}\\
AIC & \num{5275.0} & \num{5276.1} & \num{5212.9} & \num{5213.0} & \num{5213.9} & \num{5213.7} & \num{5200.4} & \num{5221.0}\\
\textbf{Log Likelihood} & \textbf{\num{-2634.5}} & \textbf{\num{-2634.0}} & \textbf{} & \textbf{} & \textbf{} & \textbf{} & \textbf{} & \textbf{\num{-2606.5}}\\
\bottomrule
\multicolumn{9}{l}{\rule{0pt}{1em}* p $<$ 0.05, ** p $<$ 0.01, *** p $<$ 0.001}\\
\end{tabular}}

}

\end{table}%

\begin{Shaded}
\begin{Highlighting}[]
\CommentTok{\# Mapa}
\NormalTok{mg\_dados}\SpecialCharTok{$}\NormalTok{fitted\_sma }\OtherTok{\textless{}{-}} \FunctionTok{fitted}\NormalTok{(mod\_sma)}
\NormalTok{g\_fit }\OtherTok{\textless{}{-}} \FunctionTok{ggplot}\NormalTok{(mg\_dados) }\SpecialCharTok{+}
  \FunctionTok{geom\_sf}\NormalTok{(}\FunctionTok{aes}\NormalTok{(}\AttributeTok{fill =}\NormalTok{ fitted\_sma), }\AttributeTok{color =} \ConstantTok{NA}\NormalTok{) }\SpecialCharTok{+}
  \FunctionTok{scale\_fill\_viridis\_c}\NormalTok{(}\AttributeTok{option =} \StringTok{"turbo"}\NormalTok{, }\AttributeTok{name =} \StringTok{"Predito"}\NormalTok{) }\SpecialCharTok{+}
  \FunctionTok{labs}\NormalTok{(}\AttributeTok{title =} \StringTok{"A. Valores Preditos (SMA)"}\NormalTok{) }\SpecialCharTok{+}
  \FunctionTok{theme\_minimal}\NormalTok{() }\SpecialCharTok{+} 
  \FunctionTok{annotation\_scale}\NormalTok{(}\AttributeTok{location =} \StringTok{"bl"}\NormalTok{, }\AttributeTok{width\_hint =} \FloatTok{0.3}\NormalTok{) }\SpecialCharTok{+}
  \FunctionTok{annotation\_north\_arrow}\NormalTok{(}\AttributeTok{location =} \StringTok{"tl"}\NormalTok{, }\AttributeTok{style =}\NormalTok{ north\_arrow\_fancy\_orienteering)}

\CommentTok{\#Resíduos}
\NormalTok{mg\_dados}\SpecialCharTok{$}\NormalTok{resid\_sma }\OtherTok{\textless{}{-}} \FunctionTok{residuals}\NormalTok{(mod\_sma)}
\NormalTok{moran\_sma }\OtherTok{\textless{}{-}} \FunctionTok{moran.test}\NormalTok{(mg\_dados}\SpecialCharTok{$}\NormalTok{resid\_sma, lw)}
\NormalTok{mg\_dados}\SpecialCharTok{$}\NormalTok{resid\_lag\_sma }\OtherTok{\textless{}{-}} \FunctionTok{lag.listw}\NormalTok{(lw, mg\_dados}\SpecialCharTok{$}\NormalTok{resid\_sma)}

\NormalTok{g\_resid\_scatter }\OtherTok{\textless{}{-}} \FunctionTok{ggplot}\NormalTok{(mg\_dados, }\FunctionTok{aes}\NormalTok{(}\AttributeTok{x =}\NormalTok{ resid\_sma, }\AttributeTok{y =}\NormalTok{ resid\_lag\_sma)) }\SpecialCharTok{+}
  \FunctionTok{geom\_hline}\NormalTok{(}\AttributeTok{yintercept =} \DecValTok{0}\NormalTok{, }\AttributeTok{linetype =} \StringTok{"dashed"}\NormalTok{, }\AttributeTok{color =} \StringTok{"gray"}\NormalTok{) }\SpecialCharTok{+}
  \FunctionTok{geom\_vline}\NormalTok{(}\AttributeTok{xintercept =} \DecValTok{0}\NormalTok{, }\AttributeTok{linetype =} \StringTok{"dashed"}\NormalTok{, }\AttributeTok{color =} \StringTok{"gray"}\NormalTok{) }\SpecialCharTok{+}
  \FunctionTok{geom\_point}\NormalTok{(}\AttributeTok{alpha =} \FloatTok{0.3}\NormalTok{) }\SpecialCharTok{+}
  \FunctionTok{geom\_smooth}\NormalTok{(}\AttributeTok{method =} \StringTok{"lm"}\NormalTok{, }\AttributeTok{se =} \ConstantTok{FALSE}\NormalTok{, }\AttributeTok{color =} \StringTok{"red"}\NormalTok{, }\AttributeTok{size =} \FloatTok{0.8}\NormalTok{) }\SpecialCharTok{+}
  \FunctionTok{labs}\NormalTok{(}\AttributeTok{title =} \StringTok{"B. Scatter de Moran (Resíduos SMA)"}\NormalTok{, }
       \AttributeTok{subtitle =} \FunctionTok{paste0}\NormalTok{(}\StringTok{"I de Moran: "}\NormalTok{, }\FunctionTok{round}\NormalTok{(moran\_sma}\SpecialCharTok{$}\NormalTok{estimate[}\DecValTok{1}\NormalTok{], }\DecValTok{3}\NormalTok{), }
                         \StringTok{" (p{-}valor: "}\NormalTok{, }\FunctionTok{round}\NormalTok{(moran\_sma}\SpecialCharTok{$}\NormalTok{p.value, }\DecValTok{3}\NormalTok{), }\StringTok{")"}\NormalTok{),}
       \AttributeTok{x =} \StringTok{"Resíduos"}\NormalTok{, }\AttributeTok{y =} \StringTok{"Lag Espacial"}\NormalTok{) }\SpecialCharTok{+}
  \FunctionTok{theme\_minimal}\NormalTok{()}

\NormalTok{( g\_fit }\SpecialCharTok{|}\NormalTok{ g\_resid\_scatter)}
\end{Highlighting}
\end{Shaded}

\begin{figure}[H]

\centering{

\pandocbounded{\includegraphics[keepaspectratio]{lattice_data_files/figure-pdf/fig-sma-diagnostico-1.pdf}}

}

\caption{\label{fig-sma-diagnostico}Diagnóstico SMA: (A) Ajuste e (B)
Resíduos.}

\end{figure}%

\section{Modelos espaciais para respostas limitadas e
discretas}\label{modelos-espaciais-para-respostas-limitadas-e-discretas}

\subsection{Modelo Probit Espacial}\label{modelo-probit-espacial}

O Modelo Probit Espacial é aplicado quando a variável dependente
observada \(y_i\) é binária (\(y_i \in \{0, 1\}\)) e existe
interdependência espacial entre as unidades de observação.

A formulação comum para o modelo Probit Espacial Autorregressivo (SAR
Probit) baseia-se numa variável latente contínua não observada,
\(y_i^*\). Assume-se que esta variável latente segue um processo
autorregressivo espacial:

\[
y_i^* = \rho \sum_{j=1}^n w_{ij} y_j^* + \mathbf{x}_i^{\top}\boldsymbol{\beta} + \epsilon_i, \quad \epsilon_i \sim \mathcal{N}(0, 1).
\]

A variável binária observada \(y_i\) é então determinada por um
mecanismo de limiar:

\[
y_i =
\begin{cases}
1, & \text{se } y_i^* > 0,\\
0, & \text{se } y_i^* \le 0.
\end{cases}
\]

Aqui, \(\mathbf{W}\) é a matriz de pesos espaciais, \(\rho\) é o
parâmetro de autocorrelação espacial, \(\mathbf{X}\) é a matriz de
covariáveis e \(\boldsymbol{\beta}\) o vetor de coeficientes. A
variância do erro \(\epsilon_i\) é fixada em 1 para identificação do
modelo Probit (J. LeSage e Pace 2009).

Resolvendo a equação estrutural para o vetor latente \(\mathbf{y}^*\),
obtém-se sua forma reduzida:

\[
\mathbf{y}^* = (\mathbf{I}_n - \rho \mathbf{W})^{-1}\mathbf{X}\boldsymbol{\beta} + (\mathbf{I}_n - \rho \mathbf{W})^{-1}\boldsymbol{\epsilon}.
\]

Define-se
\(\tilde{\boldsymbol{\epsilon}} = (\mathbf{I}_n - \rho \mathbf{W})^{-1}\boldsymbol{\epsilon}\).
A estrutura de covariância deste termo de erro composto é:

\[
\operatorname{Cov}(\tilde{\boldsymbol{\epsilon}}) = [(\mathbf{I}_n - \rho \mathbf{W})(\mathbf{I}_n - \rho \mathbf{W})^{\top}]^{-1}.
\]

A diagonal desta matriz de covariância não é constante. Cada elemento
diagonal, \(\tilde{\sigma}_i^2\), varia em função da posição da unidade
\(i\) na rede espacial, conforme definida por \(\mathbf{W}\).
Consequentemente, o modelo de variável latente exibe heterocedasticidade
induzida espacialmente. A aplicação de um estimador Probit padrão, que
assume homocedasticidade (\(\tilde{\sigma}_i^2 = 1\) para todo \(i\)), a
dados gerados por este processo, produz estimativas inconsistentes dos
parâmetros \(\rho\) e \(\boldsymbol{\beta}\) (McMillen 1992; Calabrese e
Elkink 2014).

A função de verossimilhança para o modelo Probit Espacial é a
probabilidade conjunta de observar a amostra
\(\mathbf{y} = (y_1, \ldots, y_n)^{\top}\). Esta probabilidade requer o
cálculo de uma integral \(n\)-dimensional sobre uma distribuição normal
multivariada restrita a ortantes definidos pelos valores observados
\(y_i\):

\[
L(\boldsymbol{\beta}, \rho | \mathbf{y}) = \int_{R_1} \cdots \int_{R_n} \phi_n(\mathbf{y}^* | \boldsymbol{\mu}, \boldsymbol{\Omega}) \, d\mathbf{y}^*,
\]

onde
\(\boldsymbol{\mu} = (\mathbf{I}_n - \rho \mathbf{W})^{-1}\mathbf{X}\boldsymbol{\beta}\),
\(\boldsymbol{\Omega} = [(\mathbf{I}_n - \rho \mathbf{W})(\mathbf{I}_n - \rho \mathbf{W})^{\top}]^{-1}\),
e \(R_i = (-\infty, 0]\) se \(y_i = 0\) e \(R_i = (0, \infty)\) se
\(y_i = 1\). Para amostras de tamanho moderado ou grande, o cálculo
numérico direto desta integral é computacionalmente proibitivo (Fleming
2004).

A literatura desenvolveu várias estratégias para superar esta
intratabilidade:

\begin{enumerate}
\def\labelenumi{\arabic{enumi}.}
\item
  Algoritmo EM
  (\href{https://en.wikipedia.org/wiki/Expectation\%E2\%80\%93maximization_algorithm}{Expectation-Maximization}):
  Proposto por McMillen (1992), este método trata o vetor latente
  \(\mathbf{y}^*\) como dados faltantes. O algoritmo itera entre um
  passo E, que calcula a esperança de \(\mathbf{y}^*\) condicional nos
  parâmetros atuais e em \(\mathbf{y}\), e um passo M, que maximiza a
  verossimilhança de um modelo espacial linear contínuo. Apesar de
  fornecer estimativas consistentes, a obtenção de erros-padrão válidos
  é complexa.
\item
  Método Generalizado dos Momentos (GMM): Pinkse e Slade (1998)
  desenvolveram um estimador GMM baseado nos resíduos generalizados do
  modelo Probit. Klier e McMillen (2008) propuseram uma versão
  linearizada (LGMM), computacionalmente eficiente e adequada para
  grandes conjuntos de dados, embora potencialmente sujeita a viés
  quando a dependência espacial \(\rho\) é forte (Calabrese e Elkink
  2014).
\item
  Simulação recursiva de importância (GHK): Beron e Vijverberg (2004)
  implementaram a estimação de máxima verossimilhança simulada usando o
  simulador GHK
  (\href{https://en.wikipedia.org/wiki/GHK_algorithm}{Geweke-Hajivassiliou-Keane})
  para aproximar a integral \(n\)-dimensional. Este método é preciso,
  mas seu custo computacional cresce consideravelmente com o tamanho da
  amostra.
\item
  Abordagem Bayesiana (MCMC): Consolidada por J. P. LeSage (2000), esta
  abordagem emprega métodos de Monte Carlo via Cadeias de Markov (MCMC).
  Utiliza-se uma estratégia de aumento de dados
  (\href{https://en.wikipedia.org/wiki/Data_augmentation}{data
  augmentation}), na qual o vetor latente \(\mathbf{y}^*\) é tratado
  como um parâmetro a ser amostrado sequencialmente de uma distribuição
  normal multivariada truncada, condicional aos valores observados
  \(\mathbf{y}\) e aos parâmetros do modelo. Esta é frequentemente
  considerada a abordagem mais precisa para amostras finitas, fornecendo
  a distribuição completa a posteriori dos parâmetros (Calabrese e
  Elkink 2014; Wilhelm e Godinho de Matos 2013).
\item
  Aproximações e Verossimilhança Parcial: Para grandes conjuntos de
  dados, métodos aproximados ganham relevância. Destacam-se a
  aproximação de verossimilhança via método de Mendell-Elston
  (Martinetti e Geniaux 2017), a verossimilhança parcial por pares
  (Billé e Leorato 2020), e técnicas de amostragem por importância
  eficiente (EIS) que exploram a esparsidade da matriz de precisão
  (Liesenfeld, Richard, e Vogler 2013).
\end{enumerate}

Em modelos Probit Espaciais, os coeficientes \(\boldsymbol{\beta}\) não
possuem uma interpretação direta como efeitos marginais sobre a
probabilidade \(P(y_i = 1)\), devido à não linearidade da função de
ligação Probit e à presença do multiplicador espacial global
\((\mathbf{I}_n - \rho \mathbf{W})^{-1}\).

A análise deve concentrar-se nos efeitos médios diretos, indiretos (de
transbordamento) e totais (J. LeSage e Pace 2009). Um cambio em uma
covariável \(x_{ik}\) não afeta apenas a probabilidade na unidade \(i\),
mas também se propaga através da rede, afetando as probabilidades em
todas as outras unidades \(j \neq i\). Estes efeitos são não lineares e
variam entre observações. Prática comum resume-os através das médias
amostrais dos efeitos calculados para cada unidade, fornecendo uma
medida sumária do impacto global de cada variável explicativa.

\begin{table}

\caption{\label{tbl-sarprobit_resumo}Modelo Probit Binário (Bayesiano) -
SP.}

\centering{

\begin{Shaded}
\begin{Highlighting}[]
\ControlFlowTok{if}\NormalTok{ (}\SpecialCharTok{!}\FunctionTok{require}\NormalTok{(}\StringTok{"pacman"}\NormalTok{)) }\FunctionTok{install.packages}\NormalTok{(}\StringTok{"pacman"}\NormalTok{)}
\NormalTok{pacman}\SpecialCharTok{::}\FunctionTok{p\_load}\NormalTok{(spatialprobit, spdep, sf, geobr, ggplot2, viridis, ggspatial, kableExtra, dplyr, Matrix, patchwork, scales)}

\CommentTok{\# Preparação e Simulação (IGNORE ISTO)}

\ControlFlowTok{if}\NormalTok{ (}\SpecialCharTok{!}\FunctionTok{exists}\NormalTok{(}\StringTok{"sp\_dados"}\NormalTok{) }\SpecialCharTok{||} \SpecialCharTok{!}\NormalTok{(}\StringTok{"y\_bin"} \SpecialCharTok{\%in\%} \FunctionTok{names}\NormalTok{(sp\_dados))) \{}
\NormalTok{  sp\_dados }\OtherTok{\textless{}{-}}\NormalTok{ geobr}\SpecialCharTok{::}\FunctionTok{read\_municipality}\NormalTok{(}\AttributeTok{code\_muni =} \StringTok{"SP"}\NormalTok{, }\AttributeTok{year =} \DecValTok{2020}\NormalTok{, }\AttributeTok{showProgress =} \ConstantTok{FALSE}\NormalTok{)}
\NormalTok{  coords }\OtherTok{\textless{}{-}} \FunctionTok{st\_coordinates}\NormalTok{(}\FunctionTok{st\_centroid}\NormalTok{(sp\_dados))}
  
\NormalTok{  knn }\OtherTok{\textless{}{-}} \FunctionTok{knearneigh}\NormalTok{(coords, }\AttributeTok{k =} \DecValTok{6}\NormalTok{)}
\NormalTok{  nb\_sp }\OtherTok{\textless{}{-}} \FunctionTok{knn2nb}\NormalTok{(knn)}
\NormalTok{  lw\_sp }\OtherTok{\textless{}{-}} \FunctionTok{nb2listw}\NormalTok{(nb\_sp, }\AttributeTok{style =} \StringTok{"W"}\NormalTok{)}
\NormalTok{  W\_mat }\OtherTok{\textless{}{-}} \FunctionTok{as}\NormalTok{(lw\_sp, }\StringTok{"CsparseMatrix"}\NormalTok{)}
  
  \FunctionTok{set.seed}\NormalTok{(}\DecValTok{999}\NormalTok{)}
\NormalTok{  n }\OtherTok{\textless{}{-}} \FunctionTok{nrow}\NormalTok{(sp\_dados)}
\NormalTok{  rho\_true }\OtherTok{\textless{}{-}} \FloatTok{0.65}
\NormalTok{  beta\_0 }\OtherTok{\textless{}{-}} \SpecialCharTok{{-}}\FloatTok{1.0}
\NormalTok{  beta\_1 }\OtherTok{\textless{}{-}} \FloatTok{1.5}
  
\NormalTok{  sp\_dados}\SpecialCharTok{$}\NormalTok{x\_var }\OtherTok{\textless{}{-}} \FunctionTok{rnorm}\NormalTok{(n, }\DecValTok{0}\NormalTok{, }\DecValTok{1}\NormalTok{)}
  
\NormalTok{  I\_n }\OtherTok{\textless{}{-}}\NormalTok{ Matrix}\SpecialCharTok{::}\FunctionTok{Diagonal}\NormalTok{(n)}
\NormalTok{  inv\_spatial }\OtherTok{\textless{}{-}} \FunctionTok{solve}\NormalTok{(I\_n }\SpecialCharTok{{-}}\NormalTok{ rho\_true }\SpecialCharTok{*}\NormalTok{ W\_mat)}
\NormalTok{  epsilon }\OtherTok{\textless{}{-}} \FunctionTok{rnorm}\NormalTok{(n, }\DecValTok{0}\NormalTok{, }\DecValTok{1}\NormalTok{)}
  
\NormalTok{  y\_latente }\OtherTok{\textless{}{-}} \FunctionTok{as.vector}\NormalTok{(inv\_spatial }\SpecialCharTok{\%*\%}\NormalTok{ (beta\_0 }\SpecialCharTok{+}\NormalTok{ beta\_1 }\SpecialCharTok{*}\NormalTok{ sp\_dados}\SpecialCharTok{$}\NormalTok{x\_var }\SpecialCharTok{+}\NormalTok{ epsilon))}
  
  \CommentTok{\# Y binário: 0 ou 1 (Corte em 0)}
\NormalTok{  sp\_dados}\SpecialCharTok{$}\NormalTok{y\_bin }\OtherTok{\textless{}{-}} \FunctionTok{ifelse}\NormalTok{(y\_latente }\SpecialCharTok{\textgreater{}} \DecValTok{0}\NormalTok{, }\DecValTok{1}\NormalTok{, }\DecValTok{0}\NormalTok{)}
  
\NormalTok{\} }\ControlFlowTok{else}\NormalTok{ \{}
    \ControlFlowTok{if}\NormalTok{ (}\SpecialCharTok{!}\FunctionTok{exists}\NormalTok{(}\StringTok{"W\_mat"}\NormalTok{)) \{}
\NormalTok{      knn }\OtherTok{\textless{}{-}} \FunctionTok{knearneigh}\NormalTok{(}\FunctionTok{st\_coordinates}\NormalTok{(}\FunctionTok{st\_centroid}\NormalTok{(sp\_dados)), }\AttributeTok{k =} \DecValTok{6}\NormalTok{)}
\NormalTok{      nb\_sp }\OtherTok{\textless{}{-}} \FunctionTok{knn2nb}\NormalTok{(knn)}
\NormalTok{      lw\_sp }\OtherTok{\textless{}{-}} \FunctionTok{nb2listw}\NormalTok{(nb\_sp, }\AttributeTok{style =} \StringTok{"W"}\NormalTok{)}
\NormalTok{      W\_mat }\OtherTok{\textless{}{-}} \FunctionTok{as}\NormalTok{(lw\_sp, }\StringTok{"CsparseMatrix"}\NormalTok{)}
\NormalTok{  \}}
\NormalTok{\}}

\CommentTok{\# Ajuste do Modelo PROBIT (Binário)}

\NormalTok{mod\_sar\_probit }\OtherTok{\textless{}{-}} \FunctionTok{sarprobit}\NormalTok{(y\_bin }\SpecialCharTok{\textasciitilde{}}\NormalTok{ x\_var, }
                             \AttributeTok{W =}\NormalTok{ W\_mat, }
                             \AttributeTok{data =}\NormalTok{ sp\_dados, }
                             \AttributeTok{ndraw =} \DecValTok{2000}\NormalTok{, }
                             \AttributeTok{burn.in =} \DecValTok{500}\NormalTok{, }
                             \AttributeTok{showProgress =} \ConstantTok{FALSE}\NormalTok{)}

\CommentTok{\# Tabela de Resultados}

\NormalTok{all\_draws1 }\OtherTok{\textless{}{-}} \FunctionTok{as.data.frame}\NormalTok{(mod\_sar\_probit}\SpecialCharTok{$}\NormalTok{B)}
\ControlFlowTok{if}\NormalTok{ (}\SpecialCharTok{!}\FunctionTok{is.null}\NormalTok{(mod\_sar\_probit}\SpecialCharTok{$}\NormalTok{names)) }\FunctionTok{colnames}\NormalTok{(all\_draws1) }\OtherTok{\textless{}{-}}\NormalTok{ mod\_sar\_probit}\SpecialCharTok{$}\NormalTok{names}
\ControlFlowTok{if}\NormalTok{ (}\SpecialCharTok{!}\FunctionTok{any}\NormalTok{(}\FunctionTok{grepl}\NormalTok{(}\StringTok{"rho"}\NormalTok{, }\FunctionTok{colnames}\NormalTok{(all\_draws1), }\AttributeTok{ignore.case =} \ConstantTok{TRUE}\NormalTok{))) all\_draws1}\SpecialCharTok{$}\NormalTok{rho }\OtherTok{\textless{}{-}} \FunctionTok{as.vector}\NormalTok{(mod\_sar\_probit}\SpecialCharTok{$}\NormalTok{rho)}

\NormalTok{resumo\_bayesiano1 }\OtherTok{\textless{}{-}} \FunctionTok{data.frame}\NormalTok{(}
  \AttributeTok{Parametro  =} \FunctionTok{names}\NormalTok{(all\_draws1),}
  \AttributeTok{Estimativa =} \FunctionTok{colMeans}\NormalTok{(all\_draws1),}
  \AttributeTok{IC\_Inf     =} \FunctionTok{apply}\NormalTok{(all\_draws1, }\DecValTok{2}\NormalTok{, quantile, }\AttributeTok{probs =} \FloatTok{0.025}\NormalTok{),}
  \AttributeTok{IC\_Sup     =} \FunctionTok{apply}\NormalTok{(all\_draws1, }\DecValTok{2}\NormalTok{, quantile, }\AttributeTok{probs =} \FloatTok{0.975}\NormalTok{)}
\NormalTok{)}

\CommentTok{\# Formatação da Tabela}
\NormalTok{resumo\_bayesiano1}\SpecialCharTok{$}\NormalTok{Resultado }\OtherTok{\textless{}{-}} \FunctionTok{sprintf}\NormalTok{(}\StringTok{"\%.3f [\%.3f, \%.3f]"}\NormalTok{, }
\NormalTok{                                      resumo\_bayesiano1}\SpecialCharTok{$}\NormalTok{Estimativa, }
\NormalTok{                                      resumo\_bayesiano1}\SpecialCharTok{$}\NormalTok{IC\_Inf, }
\NormalTok{                                      resumo\_bayesiano1}\SpecialCharTok{$}\NormalTok{IC\_Sup)}

\NormalTok{resumo\_bayesiano1}\SpecialCharTok{$}\NormalTok{Parametro }\OtherTok{\textless{}{-}}\NormalTok{ dplyr}\SpecialCharTok{::}\FunctionTok{case\_when}\NormalTok{(}
\NormalTok{  resumo\_bayesiano1}\SpecialCharTok{$}\NormalTok{Parametro }\SpecialCharTok{\%in\%} \FunctionTok{c}\NormalTok{(}\StringTok{"(Intercept)"}\NormalTok{, }\StringTok{"beta\_1"}\NormalTok{) }\SpecialCharTok{\textasciitilde{}} \StringTok{"Intercepto"}\NormalTok{,}
\NormalTok{  resumo\_bayesiano1}\SpecialCharTok{$}\NormalTok{Parametro }\SpecialCharTok{\%in\%} \FunctionTok{c}\NormalTok{(}\StringTok{"x\_var"}\NormalTok{, }\StringTok{"beta\_2"}\NormalTok{) }\SpecialCharTok{\textasciitilde{}} \StringTok{"Variável X"}\NormalTok{,}
  \FunctionTok{grepl}\NormalTok{(}\StringTok{"rho"}\NormalTok{, resumo\_bayesiano1}\SpecialCharTok{$}\NormalTok{Parametro, }\AttributeTok{ignore.case =} \ConstantTok{TRUE}\NormalTok{) }\SpecialCharTok{\textasciitilde{}} \StringTok{"$}\SpecialCharTok{\textbackslash{}\textbackslash{}}\StringTok{rho$ (Dependência)"}\NormalTok{,}
  \ConstantTok{TRUE} \SpecialCharTok{\textasciitilde{}}\NormalTok{ resumo\_bayesiano1}\SpecialCharTok{$}\NormalTok{Parametro}
\NormalTok{)}

\NormalTok{tabela\_final1 }\OtherTok{\textless{}{-}}\NormalTok{ resumo\_bayesiano1 }\SpecialCharTok{\%\textgreater{}\%}
  \FunctionTok{filter}\NormalTok{(}\SpecialCharTok{!}\FunctionTok{duplicated}\NormalTok{(Parametro)) }\SpecialCharTok{\%\textgreater{}\%}
\NormalTok{  dplyr}\SpecialCharTok{::}\FunctionTok{select}\NormalTok{(Parametro, Resultado)}
\FunctionTok{rownames}\NormalTok{(tabela\_final1) }\OtherTok{\textless{}{-}} \ConstantTok{NULL}

\FunctionTok{kbl}\NormalTok{(tabela\_final1, }
    \AttributeTok{format =} \StringTok{"latex"}\NormalTok{,}
    \AttributeTok{booktabs =} \ConstantTok{TRUE}\NormalTok{, }
    \AttributeTok{align =} \StringTok{"lc"}\NormalTok{,  }
    \AttributeTok{caption =} \StringTok{""}\NormalTok{, }
    \AttributeTok{escape =} \ConstantTok{FALSE}\NormalTok{) }\SpecialCharTok{\%\textgreater{}\%}
  \FunctionTok{kable\_styling}\NormalTok{(}\AttributeTok{latex\_options =} \FunctionTok{c}\NormalTok{(}\StringTok{"HOLD\_position"}\NormalTok{, }\StringTok{"striped"}\NormalTok{), }
                \AttributeTok{full\_width =} \ConstantTok{FALSE}\NormalTok{, }
                \AttributeTok{position =} \StringTok{"center"}\NormalTok{) }\SpecialCharTok{\%\textgreater{}\%}
  \FunctionTok{row\_spec}\NormalTok{(}\DecValTok{0}\NormalTok{, }\AttributeTok{bold =} \ConstantTok{TRUE}\NormalTok{) }
\end{Highlighting}
\end{Shaded}

\centering
\caption{\label{tab:tbl-sarprobit_resumo}}
\centering
\begin{tabular}[t]{lc}
\toprule
\textbf{Parametro} & \textbf{Resultado}\\
\midrule
\cellcolor{gray!10}{Intercepto} & \cellcolor{gray!10}{-1.208 [-1.675, -0.800]}\\
Variável X & 2.201 [1.636, 2.953]\\
\cellcolor{gray!10}{$\rho$ (Dependência)} & \cellcolor{gray!10}{0.707 [0.647, 0.761]}\\
\bottomrule
\end{tabular}

}

\end{table}%

\textbf{Interpretação}

A Tabela Tabela~\ref{tbl-sarprobit_resumo} apresenta as estimativas
estruturais modelo Probit ajustado. Observa-se que a Variável \(X\)
exerce um efeito positivo (\(2.274\); \(IC_{95\%} [1.677, 2.909]\)),
indicando que elevações nesta covariável aumentam significativamente a
probabilidade de classificação nos estratos superiores da resposta
ordinal. A dependência espacial, mensurada pelo parâmetro \(\rho\)
(\(0.707\); \(IC_{95\%} [0.649, 0.761]\)), demonstra que a categoria
observada em uma unidade espacial é fortemente condicionada pelo perfil
de sua vizinhança, validando a presença de feedback espacial no processo
gerador dos dados.

\begin{Shaded}
\begin{Highlighting}[]
\ControlFlowTok{if}\NormalTok{ (}\SpecialCharTok{!}\FunctionTok{require}\NormalTok{(}\StringTok{"pacman"}\NormalTok{)) }\FunctionTok{install.packages}\NormalTok{(}\StringTok{"pacman"}\NormalTok{)}
\NormalTok{pacman}\SpecialCharTok{::}\FunctionTok{p\_load}\NormalTok{(spatialprobit, spdep, sf, geobr, ggplot2, viridis, ggspatial, kableExtra, dplyr, Matrix, patchwork, scales)}


\CommentTok{\#}
\NormalTok{X\_mat }\OtherTok{\textless{}{-}} \FunctionTok{model.matrix}\NormalTok{(}\SpecialCharTok{\textasciitilde{}}\NormalTok{ x\_var, }\AttributeTok{data =}\NormalTok{ sp\_dados)}

\CommentTok{\#}
\NormalTok{all\_params }\OtherTok{\textless{}{-}} \FunctionTok{colMeans}\NormalTok{(mod\_sar\_probit}\SpecialCharTok{$}\NormalTok{B)}
\NormalTok{beta\_vec }\OtherTok{\textless{}{-}}\NormalTok{ all\_params[}\DecValTok{1}\SpecialCharTok{:}\FunctionTok{ncol}\NormalTok{(X\_mat)] }
\NormalTok{rho\_hat  }\OtherTok{\textless{}{-}} \FunctionTok{mean}\NormalTok{(mod\_sar\_probit}\SpecialCharTok{$}\NormalTok{rho)}

\CommentTok{\#Predição Linear (X * Beta)}
\NormalTok{xb }\OtherTok{\textless{}{-}}\NormalTok{ X\_mat }\SpecialCharTok{\%*\%}\NormalTok{ beta\_vec}

\CommentTok{\#(I {-} rho*W)\^{}{-}1}
\NormalTok{I\_n }\OtherTok{\textless{}{-}}\NormalTok{ Matrix}\SpecialCharTok{::}\FunctionTok{Diagonal}\NormalTok{(}\FunctionTok{nrow}\NormalTok{(sp\_dados))}
\NormalTok{S\_inv }\OtherTok{\textless{}{-}} \FunctionTok{solve}\NormalTok{(I\_n }\SpecialCharTok{{-}}\NormalTok{ rho\_hat }\SpecialCharTok{*}\NormalTok{ W\_mat)}

\CommentTok{\#y*\_pred = S\_inv * Xb}
\NormalTok{y\_star\_pred }\OtherTok{\textless{}{-}} \FunctionTok{as.vector}\NormalTok{(S\_inv }\SpecialCharTok{\%*\%}\NormalTok{ xb)}

\CommentTok{\#}
\NormalTok{y\_obs }\OtherTok{\textless{}{-}}\NormalTok{ sp\_dados}\SpecialCharTok{$}\NormalTok{y\_bin}
\NormalTok{lo }\OtherTok{\textless{}{-}} \FunctionTok{ifelse}\NormalTok{(y\_obs }\SpecialCharTok{==} \DecValTok{0}\NormalTok{, }\SpecialCharTok{{-}}\ConstantTok{Inf}\NormalTok{, }\DecValTok{0}\NormalTok{)}
\NormalTok{hi }\OtherTok{\textless{}{-}} \FunctionTok{ifelse}\NormalTok{(y\_obs }\SpecialCharTok{==} \DecValTok{0}\NormalTok{, }\DecValTok{0}\NormalTok{, }\ConstantTok{Inf}\NormalTok{)}

\CommentTok{\#}
\NormalTok{z\_lo }\OtherTok{\textless{}{-}}\NormalTok{ lo }\SpecialCharTok{{-}}\NormalTok{ y\_star\_pred}
\NormalTok{z\_hi }\OtherTok{\textless{}{-}}\NormalTok{ hi }\SpecialCharTok{{-}}\NormalTok{ y\_star\_pred}

\NormalTok{safe\_pnorm }\OtherTok{\textless{}{-}} \ControlFlowTok{function}\NormalTok{(q) }\FunctionTok{pnorm}\NormalTok{(q)}
\NormalTok{safe\_dnorm }\OtherTok{\textless{}{-}} \ControlFlowTok{function}\NormalTok{(x) }\FunctionTok{dnorm}\NormalTok{(x)}

\CommentTok{\#}
\NormalTok{diff\_cdf }\OtherTok{\textless{}{-}} \FunctionTok{safe\_pnorm}\NormalTok{(z\_hi) }\SpecialCharTok{{-}} \FunctionTok{safe\_pnorm}\NormalTok{(z\_lo)}
\NormalTok{diff\_cdf[diff\_cdf }\SpecialCharTok{\textless{}} \FloatTok{1e{-}10}\NormalTok{] }\OtherTok{\textless{}{-}} \FloatTok{1e{-}10} 

\CommentTok{\#}
\NormalTok{diff\_pdf }\OtherTok{\textless{}{-}} \FunctionTok{safe\_dnorm}\NormalTok{(z\_lo) }\SpecialCharTok{{-}} \FunctionTok{safe\_dnorm}\NormalTok{(z\_hi) }\CommentTok{\# pdf(lo) {-} pdf(hi)}

\CommentTok{\#}
\CommentTok{\# E[y* | y] = mu + sigma * (pdf\_lo {-} pdf\_hi) / (cdf\_hi {-} cdf\_lo)}
\NormalTok{sp\_dados}\SpecialCharTok{$}\NormalTok{y\_latente\_esperada }\OtherTok{\textless{}{-}}\NormalTok{ y\_star\_pred }\SpecialCharTok{+}\NormalTok{ (diff\_pdf }\SpecialCharTok{/}\NormalTok{ diff\_cdf)}

\CommentTok{\# u = (I {-} rho*W) * E[y*|y] {-} X*beta}
\NormalTok{A\_mat }\OtherTok{\textless{}{-}}\NormalTok{ (I\_n }\SpecialCharTok{{-}}\NormalTok{ rho\_hat }\SpecialCharTok{*}\NormalTok{ W\_mat)}
\NormalTok{term\_spatial\_removed }\OtherTok{\textless{}{-}} \FunctionTok{as.vector}\NormalTok{(A\_mat }\SpecialCharTok{\%*\%}\NormalTok{ sp\_dados}\SpecialCharTok{$}\NormalTok{y\_latente\_esperada)}

\NormalTok{sp\_dados}\SpecialCharTok{$}\NormalTok{resid\_generalized }\OtherTok{\textless{}{-}}\NormalTok{ term\_spatial\_removed }\SpecialCharTok{{-}} \FunctionTok{as.vector}\NormalTok{(xb)}

\CommentTok{\# Teste de Moran nos Resíduos Generalizados}
\NormalTok{moran\_resid }\OtherTok{\textless{}{-}} \FunctionTok{moran.test}\NormalTok{(sp\_dados}\SpecialCharTok{$}\NormalTok{resid\_generalized, lw\_sp)}
\NormalTok{label\_moran }\OtherTok{\textless{}{-}} \FunctionTok{paste0}\NormalTok{(}\StringTok{"Moran (Gen. Resid): "}\NormalTok{, }\FunctionTok{round}\NormalTok{(moran\_resid}\SpecialCharTok{$}\NormalTok{estimate[}\DecValTok{1}\NormalTok{], }\DecValTok{3}\NormalTok{), }
                      \StringTok{" (p: "}\NormalTok{, }\FunctionTok{round}\NormalTok{(moran\_resid}\SpecialCharTok{$}\NormalTok{p.value, }\DecValTok{3}\NormalTok{), }\StringTok{")"}\NormalTok{)}

\CommentTok{\# Impactos (Efeitos Marginais)}
\NormalTok{imp\_probit }\OtherTok{\textless{}{-}} \FunctionTok{impacts}\NormalTok{(mod\_sar\_probit)}
\end{Highlighting}
\end{Shaded}

--------Marginal Effects--------

\begin{enumerate}
\def\labelenumi{(\alph{enumi})}
\item
  Direct effects lower\_005 posterior\_mean upper\_095 x\_var 0.04418
  0.07616 0.111
\item
  Indirect effects lower\_005 posterior\_mean upper\_095 x\_var 0.1064
  0.1909 0.283
\item
  Total effects lower\_005 posterior\_mean upper\_095 x\_var 0.1499
  0.2671 0.394
\end{enumerate}

\begin{Shaded}
\begin{Highlighting}[]
\NormalTok{df\_imp }\OtherTok{\textless{}{-}} \FunctionTok{data.frame}\NormalTok{(}
  \AttributeTok{Tipo =} \FunctionTok{factor}\NormalTok{(}\FunctionTok{c}\NormalTok{(}\StringTok{"direct"}\NormalTok{, }\StringTok{"indirect"}\NormalTok{, }\StringTok{"total"}\NormalTok{), }
                \AttributeTok{levels =} \FunctionTok{c}\NormalTok{(}\StringTok{"direct"}\NormalTok{, }\StringTok{"indirect"}\NormalTok{, }\StringTok{"total"}\NormalTok{),}
                \AttributeTok{labels =} \FunctionTok{c}\NormalTok{(}\StringTok{"Direto"}\NormalTok{, }\StringTok{"Indireto"}\NormalTok{, }\StringTok{"Total"}\NormalTok{)),}
  \AttributeTok{Valor =} \FunctionTok{c}\NormalTok{(imp\_probit}\SpecialCharTok{$}\NormalTok{direct[, }\StringTok{"posterior\_mean"}\NormalTok{], }
\NormalTok{            imp\_probit}\SpecialCharTok{$}\NormalTok{indirect[, }\StringTok{"posterior\_mean"}\NormalTok{], }
\NormalTok{            imp\_probit}\SpecialCharTok{$}\NormalTok{total[, }\StringTok{"posterior\_mean"}\NormalTok{])}
\NormalTok{)}

\CommentTok{\# PREDIÇÃO E CLASSIFICAÇÃO }
\NormalTok{sp\_dados}\SpecialCharTok{$}\NormalTok{latente\_predita }\OtherTok{\textless{}{-}} \FunctionTok{as.vector}\NormalTok{(}\FunctionTok{fitted}\NormalTok{(mod\_sar\_probit))}

\CommentTok{\#Definição dos Cortes (Breaks): O vetor \textquotesingle{}phi\textquotesingle{} contém os limites: 0 (fixo)}
\NormalTok{breaks\_finais }\OtherTok{\textless{}{-}} \FunctionTok{c}\NormalTok{(}\SpecialCharTok{{-}}\ConstantTok{Inf}\NormalTok{, }\DecValTok{0}\NormalTok{, }\ConstantTok{Inf}\NormalTok{)}

\CommentTok{\#Classificação}
\NormalTok{sp\_dados}\SpecialCharTok{$}\NormalTok{cat\_predita }\OtherTok{\textless{}{-}} \FunctionTok{cut}\NormalTok{(sp\_dados}\SpecialCharTok{$}\NormalTok{latente\_predita, }
                                  \AttributeTok{breaks =}\NormalTok{ breaks\_finais, }
                                  \AttributeTok{labels =} \ConstantTok{FALSE}\NormalTok{)}
\NormalTok{sp\_dados}\SpecialCharTok{$}\NormalTok{cat\_predita }\OtherTok{\textless{}{-}}\NormalTok{ sp\_dados}\SpecialCharTok{$}\NormalTok{cat\_predita}\DecValTok{{-}1}

\CommentTok{\#}
\ControlFlowTok{if}\NormalTok{(}\FunctionTok{any}\NormalTok{(}\FunctionTok{is.na}\NormalTok{(sp\_dados}\SpecialCharTok{$}\NormalTok{cat\_predita))) \{}
\NormalTok{  sp\_dados}\SpecialCharTok{$}\NormalTok{cat\_predita\_class[}\FunctionTok{is.na}\NormalTok{(sp\_dados}\SpecialCharTok{$}\NormalTok{cat\_predita)] }\OtherTok{\textless{}{-}} \DecValTok{1}
\NormalTok{\}}


\CommentTok{\# Tema Customizado (Do seu exemplo)}
\NormalTok{theme\_map\_custom }\OtherTok{\textless{}{-}} \ControlFlowTok{function}\NormalTok{() \{}
  \FunctionTok{list}\NormalTok{(}
    \FunctionTok{theme\_void}\NormalTok{(),}
    \FunctionTok{theme}\NormalTok{(}
      \AttributeTok{legend.position =} \StringTok{"bottom"}\NormalTok{, }
      \AttributeTok{legend.box.spacing =} \FunctionTok{unit}\NormalTok{(}\DecValTok{5}\NormalTok{, }\StringTok{"pt"}\NormalTok{),}
      \AttributeTok{legend.title =} \FunctionTok{element\_text}\NormalTok{(}\AttributeTok{size=}\DecValTok{9}\NormalTok{, }\AttributeTok{face=}\StringTok{"bold"}\NormalTok{),}
      \AttributeTok{plot.title =} \FunctionTok{element\_text}\NormalTok{(}\AttributeTok{face=}\StringTok{"bold"}\NormalTok{, }\AttributeTok{size=}\DecValTok{12}\NormalTok{, }\AttributeTok{hjust =} \DecValTok{0}\NormalTok{),}
      \AttributeTok{plot.subtitle =} \FunctionTok{element\_text}\NormalTok{(}\AttributeTok{size=}\DecValTok{9}\NormalTok{, }\AttributeTok{color=}\StringTok{"grey30"}\NormalTok{)}
\NormalTok{    ),}
    \FunctionTok{annotation\_scale}\NormalTok{(}\AttributeTok{location =} \StringTok{"br"}\NormalTok{, }\AttributeTok{width\_hint =} \FloatTok{0.3}\NormalTok{),}
    \FunctionTok{annotation\_north\_arrow}\NormalTok{(}\AttributeTok{location =} \StringTok{"tr"}\NormalTok{, }\AttributeTok{height =} \FunctionTok{unit}\NormalTok{(}\DecValTok{1}\NormalTok{, }\StringTok{"cm"}\NormalTok{), }\AttributeTok{width =} \FunctionTok{unit}\NormalTok{(}\DecValTok{1}\NormalTok{, }\StringTok{"cm"}\NormalTok{),}
                           \AttributeTok{style =}\NormalTok{ north\_arrow\_fancy\_orienteering)}
\NormalTok{  )}
\NormalTok{\}}

\CommentTok{\# CORES PADRONIZADAS}
\NormalTok{cor\_zero }\OtherTok{\textless{}{-}} \StringTok{"white"}
\NormalTok{cor\_um   }\OtherTok{\textless{}{-}} \StringTok{"\#FDE725"}

\CommentTok{\# Mapa Observado}
\NormalTok{g\_obs }\OtherTok{\textless{}{-}} \FunctionTok{ggplot}\NormalTok{(sp\_dados) }\SpecialCharTok{+}
  \FunctionTok{geom\_sf}\NormalTok{(}\FunctionTok{aes}\NormalTok{(}\AttributeTok{fill =} \FunctionTok{as.factor}\NormalTok{(y\_bin)), }\AttributeTok{color =} \StringTok{"black"}\NormalTok{, }\AttributeTok{lwd =} \FloatTok{0.05}\NormalTok{) }\SpecialCharTok{+}
  \FunctionTok{scale\_fill\_manual}\NormalTok{(}\AttributeTok{values =} \FunctionTok{c}\NormalTok{(}\StringTok{"0"} \OtherTok{=}\NormalTok{ cor\_zero, }\StringTok{"1"} \OtherTok{=}\NormalTok{ cor\_um), }\AttributeTok{name =} \StringTok{"Observado"}\NormalTok{) }\SpecialCharTok{+}
  \FunctionTok{labs}\NormalTok{(}\AttributeTok{title =} \StringTok{"C. Observado (Binário)"}\NormalTok{, }\AttributeTok{subtitle =} \StringTok{"Valor Real"}\NormalTok{) }\SpecialCharTok{+}
  \FunctionTok{theme\_void}\NormalTok{() }\SpecialCharTok{+} 
  \FunctionTok{theme}\NormalTok{(}\AttributeTok{legend.position =} \StringTok{"bottom"}\NormalTok{, }\AttributeTok{legend.box.spacing =} \FunctionTok{unit}\NormalTok{(}\DecValTok{0}\NormalTok{, }\StringTok{"pt"}\NormalTok{)) }\SpecialCharTok{+}
  \FunctionTok{annotation\_scale}\NormalTok{(}\AttributeTok{location =} \StringTok{"br"}\NormalTok{, }\AttributeTok{width\_hint =} \FloatTok{0.3}\NormalTok{) }\SpecialCharTok{+}
  \FunctionTok{annotation\_north\_arrow}\NormalTok{(}\AttributeTok{location =} \StringTok{"tr"}\NormalTok{, }\AttributeTok{height =} \FunctionTok{unit}\NormalTok{(}\FloatTok{0.8}\NormalTok{, }\StringTok{"cm"}\NormalTok{), }\AttributeTok{width =} \FunctionTok{unit}\NormalTok{(}\FloatTok{0.8}\NormalTok{, }\StringTok{"cm"}\NormalTok{), }
                         \AttributeTok{style =}\NormalTok{ north\_arrow\_fancy\_orienteering)}

\CommentTok{\#Mapa Predito}
\NormalTok{g\_pred }\OtherTok{\textless{}{-}} \FunctionTok{ggplot}\NormalTok{(sp\_dados) }\SpecialCharTok{+}
  \FunctionTok{geom\_sf}\NormalTok{(}\FunctionTok{aes}\NormalTok{(}\AttributeTok{fill =} \FunctionTok{as.factor}\NormalTok{(cat\_predita)), }\AttributeTok{color =} \StringTok{"black"}\NormalTok{, }\AttributeTok{lwd =} \FloatTok{0.05}\NormalTok{) }\SpecialCharTok{+}
  \FunctionTok{scale\_fill\_manual}\NormalTok{(}\AttributeTok{values =} \FunctionTok{c}\NormalTok{(}\StringTok{"0"} \OtherTok{=}\NormalTok{ cor\_zero, }\StringTok{"1"} \OtherTok{=}\NormalTok{ cor\_um), }\AttributeTok{name =} \StringTok{"Predito"}\NormalTok{) }\SpecialCharTok{+}
  \FunctionTok{labs}\NormalTok{(}\AttributeTok{title =} \StringTok{"C. Predito pelo Modelo"}\NormalTok{) }\SpecialCharTok{+}
  \FunctionTok{theme\_void}\NormalTok{() }\SpecialCharTok{+} 
  \FunctionTok{theme}\NormalTok{(}\AttributeTok{legend.position =} \StringTok{"bottom"}\NormalTok{, }\AttributeTok{legend.box.spacing =} \FunctionTok{unit}\NormalTok{(}\DecValTok{0}\NormalTok{, }\StringTok{"pt"}\NormalTok{)) }\SpecialCharTok{+}
  \FunctionTok{annotation\_scale}\NormalTok{(}\AttributeTok{location =} \StringTok{"br"}\NormalTok{, }\AttributeTok{width\_hint =} \FloatTok{0.3}\NormalTok{) }\SpecialCharTok{+}
  \FunctionTok{annotation\_north\_arrow}\NormalTok{(}\AttributeTok{location =} \StringTok{"tr"}\NormalTok{, }\AttributeTok{height =} \FunctionTok{unit}\NormalTok{(}\FloatTok{0.8}\NormalTok{, }\StringTok{"cm"}\NormalTok{), }\AttributeTok{width =} \FunctionTok{unit}\NormalTok{(}\FloatTok{0.8}\NormalTok{, }\StringTok{"cm"}\NormalTok{), }
                         \AttributeTok{style =}\NormalTok{ north\_arrow\_fancy\_orienteering)}

\CommentTok{\#Gráfico de Impactos}
\NormalTok{g\_imp }\OtherTok{\textless{}{-}} \FunctionTok{ggplot}\NormalTok{(df\_imp, }\FunctionTok{aes}\NormalTok{(}\AttributeTok{x =}\NormalTok{ Tipo, }\AttributeTok{y =}\NormalTok{ Valor, }\AttributeTok{fill =}\NormalTok{ Tipo)) }\SpecialCharTok{+}
  \FunctionTok{geom\_col}\NormalTok{(}\AttributeTok{width =} \FloatTok{0.5}\NormalTok{, }\AttributeTok{color =} \StringTok{"black"}\NormalTok{, }\AttributeTok{alpha =} \FloatTok{0.9}\NormalTok{) }\SpecialCharTok{+}
  \FunctionTok{geom\_text}\NormalTok{(}\FunctionTok{aes}\NormalTok{(}\AttributeTok{label =} \FunctionTok{round}\NormalTok{(Valor, }\DecValTok{3}\NormalTok{)), }\AttributeTok{vjust =} \SpecialCharTok{{-}}\FloatTok{0.5}\NormalTok{, }\AttributeTok{fontface =} \StringTok{"bold"}\NormalTok{) }\SpecialCharTok{+}
  \FunctionTok{scale\_fill\_viridis\_d}\NormalTok{(}\AttributeTok{option =} \StringTok{"viridis"}\NormalTok{, }\AttributeTok{begin =} \FloatTok{0.3}\NormalTok{, }\AttributeTok{end =} \FloatTok{0.9}\NormalTok{) }\SpecialCharTok{+}
  \FunctionTok{labs}\NormalTok{(}\AttributeTok{title =} \StringTok{"A. Efeitos Marginais Médios"}\NormalTok{, }
       \AttributeTok{y =} \StringTok{"Mudança na Probabilidade"}\NormalTok{, }\AttributeTok{x =} \ConstantTok{NULL}\NormalTok{) }\SpecialCharTok{+}
  \FunctionTok{theme\_minimal}\NormalTok{() }\SpecialCharTok{+} \FunctionTok{theme}\NormalTok{(}\AttributeTok{legend.position =} \StringTok{"none"}\NormalTok{)}

\CommentTok{\#Mapa de Resíduos}
\NormalTok{g\_resid }\OtherTok{\textless{}{-}} \FunctionTok{ggplot}\NormalTok{(sp\_dados) }\SpecialCharTok{+}
  \FunctionTok{geom\_sf}\NormalTok{(}\FunctionTok{aes}\NormalTok{(}\AttributeTok{fill =}\NormalTok{ resid\_generalized), }\AttributeTok{color =} \StringTok{"black"}\NormalTok{, }\AttributeTok{lwd =} \FloatTok{0.02}\NormalTok{) }\SpecialCharTok{+}
  \FunctionTok{scale\_fill\_distiller}\NormalTok{(}\AttributeTok{palette =} \StringTok{"RdBu"}\NormalTok{, }\AttributeTok{direction =} \SpecialCharTok{{-}}\DecValTok{1}\NormalTok{,}
                       \AttributeTok{name =} \StringTok{"Resíduo"}\NormalTok{) }\SpecialCharTok{+}
  \FunctionTok{labs}\NormalTok{(}\AttributeTok{title =} \StringTok{"D. Resíduos"}\NormalTok{, }
       \AttributeTok{subtitle =} \FunctionTok{paste0}\NormalTok{(}\StringTok{"Autocorrelação:}\SpecialCharTok{\textbackslash{}n}\StringTok{"}\NormalTok{, label\_moran)) }\SpecialCharTok{+}
  \FunctionTok{theme\_void}\NormalTok{()}\SpecialCharTok{+}
  \FunctionTok{theme}\NormalTok{(}\AttributeTok{legend.box.spacing =} \FunctionTok{unit}\NormalTok{(}\DecValTok{0}\NormalTok{, }\StringTok{"pt"}\NormalTok{)) }\SpecialCharTok{+}
  \FunctionTok{annotation\_scale}\NormalTok{(}\AttributeTok{location =} \StringTok{"br"}\NormalTok{, }\AttributeTok{width\_hint =} \FloatTok{0.3}\NormalTok{) }\SpecialCharTok{+}
  \FunctionTok{annotation\_north\_arrow}\NormalTok{(}\AttributeTok{location =} \StringTok{"tr"}\NormalTok{, }\AttributeTok{height =} \FunctionTok{unit}\NormalTok{(}\FloatTok{0.8}\NormalTok{, }\StringTok{"cm"}\NormalTok{), }\AttributeTok{width =} \FunctionTok{unit}\NormalTok{(}\FloatTok{0.8}\NormalTok{, }\StringTok{"cm"}\NormalTok{),}
                         \AttributeTok{style =}\NormalTok{ north\_arrow\_fancy\_orienteering)}

\NormalTok{(g\_obs }\SpecialCharTok{|}\NormalTok{ g\_pred) }\SpecialCharTok{/}\NormalTok{ (g\_imp }\SpecialCharTok{|}\NormalTok{ g\_resid) }\SpecialCharTok{+} \FunctionTok{plot\_layout}\NormalTok{(}\AttributeTok{heights =} \FunctionTok{c}\NormalTok{(}\FloatTok{1.2}\NormalTok{, }\DecValTok{1}\NormalTok{))}
\end{Highlighting}
\end{Shaded}

\begin{figure}[H]

\centering{

\pandocbounded{\includegraphics[keepaspectratio]{lattice_data_files/figure-pdf/fig-sarprobit_completo_resid-1.pdf}}

}

\caption{\label{fig-sarprobit_completo_resid}(A) Efeitos Marginais, (B)
Resíduos Generalizados, (C) Observado e (D) Probabilidade Predita.}

\end{figure}%

\textbf{Interpretação}

A análise comparativa entre a distribuição dos eventos observados
(Figura~\ref{fig-sarprobit_completo_resid} C) e as probabilidades
preditas (Figura~\ref{fig-sarprobit_completo_resid} D) indica que o
modelo probit identificou as zonas de maior suscetibilidade, recuperando
o padrão geográfico do fenômeno. O efeito indireto (\(0.192\);
Figura~\ref{fig-sarprobit_completo_resid} A) supera o efeito direto
(\(0.073\); Figura~\ref{fig-sarprobit_completo_resid} A), sugerindo que
a alteração na probabilidade do evento local depende preponderantemente
das características das vizinhanças. Entretanto, o diagnóstico dos
resíduos apresentado em Figura~\ref{fig-sarprobit_completo_resid} (B)
evidencia uma violação do pressuposto de independência condicional. A
estatística de I de Moran global aplicada aos resíduos (\(I = 0.076\)),
estatisticamente significativa (\(p \approx 0\)), aponta para a
persistência de autocorrelação espacial positiva não capturada pelo
termo de defasagem, indicando que a especificação atual é insuficiente
para filtrar a totalidade da dependência espacial latente.

\subsection{Modelo Probit Ordenado
Espacial}\label{modelo-probit-ordenado-espacial}

O Modelo Probit Ordenado Espacial é uma extensão para dados categóricos
ordenados em contextos espaciais. Ele é adequado quando a variável
dependente observada, \(y_i\), assume valores em categorias ordenadas,
como classificações de intensidade, níveis de satisfação ou escalas de
severidade, e existe interdependência espacial entre as unidades de
observação.

A premissa central do modelo é que a variável categórica ordenada
observada é uma manifestação censurada de uma variável latente contínua
subjacente, \(y_i^*\), que representa uma propensão ou utilidade não
observada. A dependência espacial é modelada diretamente no processo
gerador desta variável latente.

Seguindo a estrutura autorregressiva espacial, a equação estrutural para
a variável latente é especificada como (J. LeSage e Pace 2009; X. Wang e
Kockelman 2009):

\[
\mathbf{y}^* = \rho \mathbf{W}\mathbf{y}^* + \mathbf{X}\boldsymbol{\beta} + \boldsymbol{\epsilon}, \quad \boldsymbol{\epsilon} \sim \mathcal{N}(\mathbf{0}, \mathbf{I}_n).
\]

A variável observada \(y_i\) está relacionada com sua contraparte
latente através de um conjunto de limiares ordenados,
\(\gamma = (\gamma_0, \gamma_1, \dots, \gamma_J, \gamma_{J+1})'\). Para
uma observação \(i\) com \(J+1\) categorias possíveis
(\(j = 0, 1, \dots, J\)):

\[
y_i = j \quad \text{se, e somente se,} \quad \gamma_j < y_i^* \le \gamma_{j+1}.
\]

Para identificação do modelo, fixam-se os valores
\(\gamma_0 = -\infty\), \(\gamma_1 = 0\) e \(\gamma_{J+1} = \infty\). Os
limiares restantes, \(\gamma_2, \dots, \gamma_J\), são parâmetros
desconhecidos a serem estimados, sujeitos à restrição de ordenamento
\(\gamma_2 < \gamma_3 < \dots < \gamma_J\). A variância do erro
\(\epsilon_i\) é fixada em 1, conforme convenção padrão em modelos
Probit.

Resolvendo a equação estrutural para o vetor latente \(\mathbf{y}^*\),
obtém-se sua forma reduzida:

\[
\mathbf{y}^* = (\mathbf{I}_n - \rho \mathbf{W})^{-1}\mathbf{X}\boldsymbol{\beta} + \mathbf{u}, \quad \text{onde} \quad \mathbf{u} = (\mathbf{I}_n - \rho \mathbf{W})^{-1}\boldsymbol{\epsilon}.
\]

A matriz de covariância do termo de erro composto \(\mathbf{u}\) é:

\[
\operatorname{Cov}(\mathbf{u}) = \boldsymbol{\Omega} = [(\mathbf{I}_n - \rho \mathbf{W})'(\mathbf{I}_n - \rho \mathbf{W})]^{-1}.
\]

Esta matriz é não diagonal e seus elementos diagonais,
\(\sigma_i^2 = \boldsymbol{\Omega}_{ii}\), não são constantes. Eles
variam em função da posição de cada unidade \(i\) na estrutura de
vizinhança definida por \(\mathbf{W}\). Consequentemente, o modelo exibe
heterocedasticidade induzida espacialmente. A aplicação de um estimador
Probit Ordenado padrão, que assume independência e homocedasticidade, a
dados gerados por este processo resulta em estimadores inconsistentes
para \(\boldsymbol{\beta}\), \(\rho\) e \(\boldsymbol{\gamma}\),
comprometendo a inferência (X. Wang e Kockelman 2009).

A função de verossimilhança do modelo envolve o cálculo de
probabilidades de uma distribuição normal multivariada restrita a
ortantes definidos pelos limiares, uma tarefa computacionalmente
intratável para amostras de tamanho moderado. A abordagem Bayesiana,
utilizando métodos de Monte Carlo via Cadeias de Markov (MCMC) com
aumento de dados (\emph{data augmentation}), supera esta dificuldade e
tornou-se o método padrão (J. LeSage e Pace 2009).

O algoritmo de amostragem de Gibbs itera entre os seguintes passos,
amostrando de cada distribuição condicional completa:

\begin{enumerate}
\def\labelenumi{\arabic{enumi}.}
\item
  Variável Latente (\(\mathbf{y}^*\)): Condicional aos parâmetros
  \((\boldsymbol{\beta}, \rho, \boldsymbol{\gamma})\) e aos dados
  observados \(\mathbf{y}\), cada \(y_i^*\) é amostrado de uma
  distribuição normal univariada truncada. O suporte de truncagem é o
  intervalo \((\gamma_{y_i}, \gamma_{y_i+1}]\), determinado pela
  categoria observada \(y_i\). A média e variância condicionais de
  \(y_i^*\) dependem dos valores latentes atuais das unidades vizinhas,
  requerendo algoritmos eficientes como o de Geweke para a amostragem.
\item
  Parâmetros de regressão e espacial (\(\boldsymbol{\beta}, \rho\)):
  Condicional ao vetor latente completo \(\mathbf{y}^*\), o modelo
  reduz-se a um modelo autorregressivo espacial (SAR) linear contínuo.
  Os parâmetros \(\boldsymbol{\beta}\) são amostrados de uma
  distribuição normal multivariada, e \(\rho\) é amostrado via um passo
  de Metropolis-Hastings, utilizando suas distribuições condicionais
  completas padrão.
\item
  Parâmetros de limiar (\(\boldsymbol{\gamma}\)): Condicional a
  \(\mathbf{y}^*\), os limiares \(\gamma_j\) (para \(j=2, \dots, J\))
  são amostrados de distribuições uniformes em intervalos restritos. O
  intervalo para \(\gamma_j\) é determinado pelos valores latentes das
  observações nas categorias adjacentes: \[
  \gamma_j \mid \mathbf{y}^*, \mathbf{y} \sim \mathcal{U}\left( \max_{\{i: y_i = j-1\}} y_i^*, \min_{\{i: y_i = j\}} y_i^* \right).
  \] Para garantir uma boa mistura da cadeia de Markov, emprega-se
  frequentemente o algoritmo de Cowles (1996) (Tobias 2024).
\end{enumerate}

A interpretação dos coeficientes \(\boldsymbol{\beta}\) no modelo Probit
Ordenado Espacial não é direta. O sinal de \(\beta_k\) indica a direção
do efeito da variável \(x_k\) sobre a variável latente \(y_i^*\), mas o
efeito sobre as probabilidades das categorias observadas \(P(y_i = j)\)
é ambíguo e depende dos limiares estimados \(\boldsymbol{\gamma}\)
(Greene 2003). Além disso, a presença do multiplicador espacial
\((\mathbf{I}_n - \rho \mathbf{W})^{-1}\) implica que uma mudança em uma
covariável para a unidade \(i\) gera efeitos de transbordamento
(\emph{spillovers}) sobre as probabilidades de todas as outras unidades.

Portanto, a análise deve concentrar-se nos efeitos marginais espaciais
para cada categoria \(j\). O efeito marginal de uma mudança na
covariável \(x_{rk}\) (da unidade \(r\)) sobre a probabilidade da
unidade \(i\) pertencer à categoria \(j\) é dado por:

\[
\frac{\partial P(y_i = j \mid \mathbf{X})}{\partial x_{rk}} = \left[ \phi(\gamma_j - \mu_i) - \phi(\gamma_{j+1} - \mu_i) \right] \cdot \frac{ \left[ (\mathbf{I}_n - \rho \mathbf{W})^{-1} \beta_k \right]_{ir} }{\sigma_i},
\]

onde \(\mu_i\) é o i-ésimo elemento do vetor
\((\mathbf{I}_n - \rho \mathbf{W})^{-1}\mathbf{X}\boldsymbol{\beta}\),
\(\sigma_i = \sqrt{\boldsymbol{\Omega}_{ii}}\), e \(\phi(\cdot)\) é a
função de densidade da normal padrão. O termo
\(\left[ (\mathbf{I}_n - \rho \mathbf{W})^{-1} \beta_k \right]_{ir}\)
representa o elemento \((i, r)\) da matriz de multiplicador espacial
ponderada por \(\beta_k\).

Estes efeitos são decompostos em:

\begin{itemize}
\item
  Efeito direto médio: O impacto médio de uma mudança em \(x_{ik}\)
  sobre a probabilidade da própria unidade \(i\) estar na categoria
  \(j\).
\item
  Efeito indireto (ou de Transbordamento) médio: O impacto médio
  cumulativo de uma mudança em \(x_{ik}\) sobre as probabilidades de
  todas as outras unidades \(r \neq i\) estarem na categoria \(j\).
\item
  Efeito total médio: A soma dos efeitos direto e indireto.
\end{itemize}

Dada a não linearidade do modelo, estes efeitos variam para cada
observação. A prática padrão é reportar as médias amostrais dos efeitos
diretos, indiretos e totais para cada categoria \(j\), juntamente com
medidas de incerteza (como intervalos de credibilidade) obtidas a partir
das amostras da cadeia MCMC.

\begin{table}

\caption{\label{tbl-sar_ordered_probitT}Estimativas do modelo Probit
Ordenado}

\centering{

\begin{Shaded}
\begin{Highlighting}[]
\CommentTok{\#}
\ControlFlowTok{if}\NormalTok{ (}\SpecialCharTok{!}\FunctionTok{require}\NormalTok{(}\StringTok{"pacman"}\NormalTok{)) }\FunctionTok{install.packages}\NormalTok{(}\StringTok{"pacman"}\NormalTok{)}
\NormalTok{pacman}\SpecialCharTok{::}\FunctionTok{p\_load}\NormalTok{(spatialprobit, spdep, sf, geobr, ggplot2, viridis, }
\NormalTok{               kableExtra, dplyr, Matrix, patchwork, ggspatial, scales, truncnorm)}

\CommentTok{\#Ignore esta parte}

\ControlFlowTok{if}\NormalTok{ (}\SpecialCharTok{!}\FunctionTok{exists}\NormalTok{(}\StringTok{"sp\_dados"}\NormalTok{) }\SpecialCharTok{||} \SpecialCharTok{!}\NormalTok{(}\StringTok{"y\_ordered"} \SpecialCharTok{\%in\%} \FunctionTok{names}\NormalTok{(sp\_dados))) \{}
  \FunctionTok{message}\NormalTok{(}\StringTok{"Baixando shapefile e simulando dados..."}\NormalTok{)}
\NormalTok{  sp\_dados }\OtherTok{\textless{}{-}}\NormalTok{ geobr}\SpecialCharTok{::}\FunctionTok{read\_municipality}\NormalTok{(}\AttributeTok{code\_muni =} \StringTok{"SP"}\NormalTok{, }\AttributeTok{year =} \DecValTok{2020}\NormalTok{, }\AttributeTok{showProgress =} \ConstantTok{FALSE}\NormalTok{)}

\NormalTok{  sp\_dados }\OtherTok{\textless{}{-}}\NormalTok{ sp\_dados[}\SpecialCharTok{!}\FunctionTok{is.na}\NormalTok{(}\FunctionTok{st\_dimension}\NormalTok{(sp\_dados)), ]}
\NormalTok{  coords }\OtherTok{\textless{}{-}} \FunctionTok{st\_coordinates}\NormalTok{(}\FunctionTok{st\_centroid}\NormalTok{(sp\_dados))}
  
  \CommentTok{\# Matriz de Vizinhança (k=6)}
\NormalTok{  knn }\OtherTok{\textless{}{-}} \FunctionTok{knearneigh}\NormalTok{(coords, }\AttributeTok{k =} \DecValTok{6}\NormalTok{)}
\NormalTok{  nb\_sp }\OtherTok{\textless{}{-}} \FunctionTok{knn2nb}\NormalTok{(knn)}
\NormalTok{  lw\_sp }\OtherTok{\textless{}{-}} \FunctionTok{nb2listw}\NormalTok{(nb\_sp, }\AttributeTok{style =} \StringTok{"W"}\NormalTok{)}
\NormalTok{  W\_mat }\OtherTok{\textless{}{-}} \FunctionTok{as}\NormalTok{(lw\_sp, }\StringTok{"CsparseMatrix"}\NormalTok{)}
  
  \FunctionTok{set.seed}\NormalTok{(}\DecValTok{123}\NormalTok{) }
\NormalTok{  n }\OtherTok{\textless{}{-}} \FunctionTok{nrow}\NormalTok{(sp\_dados)}
\NormalTok{  rho\_true }\OtherTok{\textless{}{-}} \FloatTok{0.60}
\NormalTok{  beta\_x }\OtherTok{\textless{}{-}} \FloatTok{2.0}
\NormalTok{  intercept\_true }\OtherTok{\textless{}{-}} \SpecialCharTok{{-}}\FloatTok{0.5}
  
\NormalTok{  sp\_dados}\SpecialCharTok{$}\NormalTok{x\_var }\OtherTok{\textless{}{-}} \FunctionTok{rnorm}\NormalTok{(n, }\DecValTok{0}\NormalTok{, }\DecValTok{1}\NormalTok{)}
  
  \CommentTok{\# SAR: y* = (I {-} rho W)\^{}{-}1 (Xb + e)}
\NormalTok{  I\_n }\OtherTok{\textless{}{-}}\NormalTok{ Matrix}\SpecialCharTok{::}\FunctionTok{Diagonal}\NormalTok{(n)}
\NormalTok{  inv\_spatial }\OtherTok{\textless{}{-}} \FunctionTok{solve}\NormalTok{(I\_n }\SpecialCharTok{{-}}\NormalTok{ rho\_true }\SpecialCharTok{*}\NormalTok{ W\_mat)}
\NormalTok{  epsilon }\OtherTok{\textless{}{-}} \FunctionTok{rnorm}\NormalTok{(n, }\DecValTok{0}\NormalTok{, }\DecValTok{1}\NormalTok{)}
  
\NormalTok{  xb }\OtherTok{\textless{}{-}}\NormalTok{ intercept\_true }\SpecialCharTok{+}\NormalTok{ (beta\_x }\SpecialCharTok{*}\NormalTok{ sp\_dados}\SpecialCharTok{$}\NormalTok{x\_var)}
\NormalTok{  y\_latente }\OtherTok{\textless{}{-}} \FunctionTok{as.vector}\NormalTok{(inv\_spatial }\SpecialCharTok{\%*\%}\NormalTok{ (xb }\SpecialCharTok{+}\NormalTok{ epsilon))}
  
\NormalTok{  cortes\_sim }\OtherTok{\textless{}{-}} \FunctionTok{quantile}\NormalTok{(y\_latente, }\AttributeTok{probs =} \FunctionTok{c}\NormalTok{(}\FloatTok{0.33}\NormalTok{, }\FloatTok{0.66}\NormalTok{))}
\NormalTok{  sp\_dados}\SpecialCharTok{$}\NormalTok{y\_ordered }\OtherTok{\textless{}{-}} \FunctionTok{cut}\NormalTok{(y\_latente, }
                            \AttributeTok{breaks =} \FunctionTok{c}\NormalTok{(}\SpecialCharTok{{-}}\ConstantTok{Inf}\NormalTok{, cortes\_sim, }\ConstantTok{Inf}\NormalTok{), }
                            \AttributeTok{labels =} \ConstantTok{FALSE}\NormalTok{)}
\NormalTok{\} }\ControlFlowTok{else}\NormalTok{ \{}

    \ControlFlowTok{if}\NormalTok{ (}\SpecialCharTok{!}\FunctionTok{exists}\NormalTok{(}\StringTok{"W\_mat"}\NormalTok{)) \{}
\NormalTok{    knn }\OtherTok{\textless{}{-}} \FunctionTok{knearneigh}\NormalTok{(}\FunctionTok{st\_coordinates}\NormalTok{(}\FunctionTok{st\_centroid}\NormalTok{(sp\_dados)), }\AttributeTok{k =} \DecValTok{6}\NormalTok{)}
\NormalTok{    nb\_sp }\OtherTok{\textless{}{-}} \FunctionTok{knn2nb}\NormalTok{(knn)}
\NormalTok{    lw\_sp }\OtherTok{\textless{}{-}} \FunctionTok{nb2listw}\NormalTok{(nb\_sp, }\AttributeTok{style =} \StringTok{"W"}\NormalTok{)}
\NormalTok{    W\_mat }\OtherTok{\textless{}{-}} \FunctionTok{as}\NormalTok{(lw\_sp, }\StringTok{"CsparseMatrix"}\NormalTok{)}
\NormalTok{  \}}
\NormalTok{\}}

\CommentTok{\# AJUSTE DO MODELO (ORDERED PROBIT)}
\NormalTok{mod\_sar\_ordered }\OtherTok{\textless{}{-}} \FunctionTok{sarorderedprobit}\NormalTok{(y\_ordered }\SpecialCharTok{\textasciitilde{}}\NormalTok{ x\_var, }
                                    \AttributeTok{W =}\NormalTok{ W\_mat, }
                                    \AttributeTok{data =}\NormalTok{ sp\_dados, }
                                    \AttributeTok{ndraw =} \DecValTok{2000}\NormalTok{, }
                                    \AttributeTok{burn.in =} \DecValTok{500}\NormalTok{, }
                                    \AttributeTok{showProgress =} \ConstantTok{FALSE}\NormalTok{)}

\CommentTok{\#TABELA DE RESULTADOS E IMPACTOS}

\NormalTok{all\_draws }\OtherTok{\textless{}{-}} \FunctionTok{as.data.frame}\NormalTok{(mod\_sar\_ordered}\SpecialCharTok{$}\NormalTok{B)}
\ControlFlowTok{if}\NormalTok{ (}\SpecialCharTok{!}\FunctionTok{is.null}\NormalTok{(mod\_sar\_ordered}\SpecialCharTok{$}\NormalTok{names)) }\FunctionTok{colnames}\NormalTok{(all\_draws) }\OtherTok{\textless{}{-}}\NormalTok{ mod\_sar\_ordered}\SpecialCharTok{$}\NormalTok{names}
\ControlFlowTok{if}\NormalTok{ (}\SpecialCharTok{!}\FunctionTok{any}\NormalTok{(}\FunctionTok{grepl}\NormalTok{(}\StringTok{"rho"}\NormalTok{, }\FunctionTok{colnames}\NormalTok{(all\_draws), }\AttributeTok{ignore.case =} \ConstantTok{TRUE}\NormalTok{))) all\_draws}\SpecialCharTok{$}\NormalTok{rho }\OtherTok{\textless{}{-}} \FunctionTok{as.vector}\NormalTok{(mod\_sar\_ordered}\SpecialCharTok{$}\NormalTok{rho)}

\NormalTok{resumo\_bayesiano }\OtherTok{\textless{}{-}} \FunctionTok{data.frame}\NormalTok{(}
  \AttributeTok{Parametro  =} \FunctionTok{names}\NormalTok{(all\_draws),}
  \AttributeTok{Estimativa =} \FunctionTok{colMeans}\NormalTok{(all\_draws),}
  \AttributeTok{IC\_Inf     =} \FunctionTok{apply}\NormalTok{(all\_draws, }\DecValTok{2}\NormalTok{, quantile, }\AttributeTok{probs =} \FloatTok{0.025}\NormalTok{),}
  \AttributeTok{IC\_Sup     =} \FunctionTok{apply}\NormalTok{(all\_draws, }\DecValTok{2}\NormalTok{, quantile, }\AttributeTok{probs =} \FloatTok{0.975}\NormalTok{)}
\NormalTok{)}

\CommentTok{\# Formatação da Tabela}
\NormalTok{resumo\_bayesiano}\SpecialCharTok{$}\NormalTok{Resultado }\OtherTok{\textless{}{-}} \FunctionTok{sprintf}\NormalTok{(}\StringTok{"\%.3f [\%.3f, \%.3f]"}\NormalTok{, }
\NormalTok{                                      resumo\_bayesiano}\SpecialCharTok{$}\NormalTok{Estimativa, }
\NormalTok{                                      resumo\_bayesiano}\SpecialCharTok{$}\NormalTok{IC\_Inf, }
\NormalTok{                                      resumo\_bayesiano}\SpecialCharTok{$}\NormalTok{IC\_Sup)}

\NormalTok{resumo\_bayesiano}\SpecialCharTok{$}\NormalTok{Parametro }\OtherTok{\textless{}{-}}\NormalTok{ dplyr}\SpecialCharTok{::}\FunctionTok{case\_when}\NormalTok{(}
\NormalTok{  resumo\_bayesiano}\SpecialCharTok{$}\NormalTok{Parametro }\SpecialCharTok{\%in\%} \FunctionTok{c}\NormalTok{(}\StringTok{"(Intercept)"}\NormalTok{, }\StringTok{"beta\_1"}\NormalTok{) }\SpecialCharTok{\textasciitilde{}} \StringTok{"Intercepto"}\NormalTok{,}
\NormalTok{  resumo\_bayesiano}\SpecialCharTok{$}\NormalTok{Parametro }\SpecialCharTok{\%in\%} \FunctionTok{c}\NormalTok{(}\StringTok{"x\_var"}\NormalTok{, }\StringTok{"beta\_2"}\NormalTok{) }\SpecialCharTok{\textasciitilde{}} \StringTok{"Variável X"}\NormalTok{,}
  \FunctionTok{grepl}\NormalTok{(}\StringTok{"rho"}\NormalTok{, resumo\_bayesiano}\SpecialCharTok{$}\NormalTok{Parametro, }\AttributeTok{ignore.case =} \ConstantTok{TRUE}\NormalTok{) }\SpecialCharTok{\textasciitilde{}} \StringTok{"$}\SpecialCharTok{\textbackslash{}\textbackslash{}}\StringTok{rho$ (Dependência)"}\NormalTok{,}
  \ConstantTok{TRUE} \SpecialCharTok{\textasciitilde{}}\NormalTok{ resumo\_bayesiano}\SpecialCharTok{$}\NormalTok{Parametro}
\NormalTok{)}

\NormalTok{tabela\_final }\OtherTok{\textless{}{-}}\NormalTok{ resumo\_bayesiano }\SpecialCharTok{\%\textgreater{}\%}
  \FunctionTok{filter}\NormalTok{(}\SpecialCharTok{!}\FunctionTok{duplicated}\NormalTok{(Parametro)) }\SpecialCharTok{\%\textgreater{}\%}
\NormalTok{  dplyr}\SpecialCharTok{::}\FunctionTok{select}\NormalTok{(Parametro, Resultado)}
\FunctionTok{rownames}\NormalTok{(tabela\_final) }\OtherTok{\textless{}{-}} \ConstantTok{NULL}

\FunctionTok{kbl}\NormalTok{(tabela\_final, }
    \AttributeTok{format =} \StringTok{"latex"}\NormalTok{, }
    \AttributeTok{booktabs =} \ConstantTok{TRUE}\NormalTok{, }
    \AttributeTok{caption =} \StringTok{""}\NormalTok{, }
    \AttributeTok{escape =} \ConstantTok{FALSE}\NormalTok{) }\SpecialCharTok{\%\textgreater{}\%}
  \FunctionTok{kable\_styling}\NormalTok{(}\AttributeTok{latex\_options =} \FunctionTok{c}\NormalTok{(}\StringTok{"HOLD\_position"}\NormalTok{, }\StringTok{"striped"}\NormalTok{), }
                \AttributeTok{full\_width =} \ConstantTok{FALSE}\NormalTok{, }
                \AttributeTok{position =} \StringTok{"center"}\NormalTok{) }\SpecialCharTok{\%\textgreater{}\%}
  \FunctionTok{row\_spec}\NormalTok{(}\DecValTok{0}\NormalTok{, }\AttributeTok{bold =} \ConstantTok{TRUE}\NormalTok{) }
\end{Highlighting}
\end{Shaded}

\centering
\caption{\label{tab:tbl-sar_ordered_probitT}}
\centering
\begin{tabular}[t]{ll}
\toprule
\textbf{Parametro} & \textbf{Resultado}\\
\midrule
\cellcolor{gray!10}{Intercepto} & \cellcolor{gray!10}{0.128 [0.039, 0.217]}\\
Variável X & 0.863 [0.762, 0.963]\\
\cellcolor{gray!10}{$\rho$ (Dependência)} & \cellcolor{gray!10}{0.437 [0.331, 0.537]}\\
y>=2 & 0.000 [0.000, 0.000]\\
\cellcolor{gray!10}{y>=3} & \cellcolor{gray!10}{0.865 [0.802, 0.941]}\\
\bottomrule
\end{tabular}

}

\end{table}%

\textbf{Interpretação}

A Tabela Tabela~\ref{tbl-sar_ordered_probitT} apresenta os parâmetros
estruturais do modelo. A covariável \(X\) exerce uma influência positiva
sobre a variável latente (\(0.863\); \(IC_{95\%} [0.762, 0.963]\)),
indicando que elevações nesta preditora aumentam a probabilidade de
classificação nos estratos ordinais superiores. Simultaneamente, a
dependência espacial é confirmada pela estimativa de \(\rho\)
(\(0.437\); \(IC_{95\%} [0.331, 0.537]\)), demonstrando que o nível
ordenado de uma observação é positivamente correlacionado com o status
de sua vizinhança geográfica. A calibração dos limiares
(\emph{thresholds}), especificamente o parâmetro de corte \(y \ge 3\)
(\(0.865\); \(IC_{95\%} [0.802, 0.941]\)), define as fronteiras
probabilísticas que segregam as categorias mais elevadas da
distribuição.

\begin{Shaded}
\begin{Highlighting}[]
\CommentTok{\#}
\ControlFlowTok{if}\NormalTok{ (}\SpecialCharTok{!}\FunctionTok{require}\NormalTok{(}\StringTok{"pacman"}\NormalTok{)) }\FunctionTok{install.packages}\NormalTok{(}\StringTok{"pacman"}\NormalTok{)}
\NormalTok{pacman}\SpecialCharTok{::}\FunctionTok{p\_load}\NormalTok{(spatialprobit, spdep, sf, geobr, ggplot2, viridis, }
\NormalTok{               kableExtra, dplyr, Matrix, patchwork, ggspatial, scales, truncnorm)}


\CommentTok{\# Cálculo dos Impactos (Médios)}
\NormalTok{rho\_medio }\OtherTok{\textless{}{-}} \FunctionTok{mean}\NormalTok{(mod\_sar\_ordered}\SpecialCharTok{$}\NormalTok{rho)}
\NormalTok{beta\_val }\OtherTok{\textless{}{-}}\NormalTok{ resumo\_bayesiano}\SpecialCharTok{$}\NormalTok{Estimativa[resumo\_bayesiano}\SpecialCharTok{$}\NormalTok{Parametro }\SpecialCharTok{==} \StringTok{"Variável X"}\NormalTok{]}

\NormalTok{impacto\_total  }\OtherTok{\textless{}{-}}\NormalTok{ beta\_val }\SpecialCharTok{/}\NormalTok{ (}\DecValTok{1} \SpecialCharTok{{-}}\NormalTok{ rho\_medio)}
\NormalTok{impacto\_direto }\OtherTok{\textless{}{-}}\NormalTok{ beta\_val }
\NormalTok{impacto\_indireto }\OtherTok{\textless{}{-}}\NormalTok{ impacto\_total }\SpecialCharTok{{-}}\NormalTok{ impacto\_direto}

\NormalTok{df\_imp }\OtherTok{\textless{}{-}} \FunctionTok{data.frame}\NormalTok{(}
  \AttributeTok{Tipo =} \FunctionTok{factor}\NormalTok{(}\FunctionTok{c}\NormalTok{(}\StringTok{"direct"}\NormalTok{, }\StringTok{"indirect"}\NormalTok{, }\StringTok{"total"}\NormalTok{), }
                \AttributeTok{levels =} \FunctionTok{c}\NormalTok{(}\StringTok{"direct"}\NormalTok{, }\StringTok{"indirect"}\NormalTok{, }\StringTok{"total"}\NormalTok{),}
                \AttributeTok{labels =} \FunctionTok{c}\NormalTok{(}\StringTok{"Direto"}\NormalTok{, }\StringTok{"Indireto"}\NormalTok{, }\StringTok{"Total"}\NormalTok{)),}
  \AttributeTok{Valor =} \FunctionTok{c}\NormalTok{(impacto\_direto, impacto\_indireto, impacto\_total)}
\NormalTok{)}



\CommentTok{\# PREDIÇÃO E CLASSIFICAÇÃO }
\NormalTok{sp\_dados}\SpecialCharTok{$}\NormalTok{latente\_predita }\OtherTok{\textless{}{-}} \FunctionTok{as.vector}\NormalTok{(}\FunctionTok{fitted}\NormalTok{(mod\_sar\_ordered))}

\CommentTok{\#Definição dos Cortes (Breaks): O vetor \textquotesingle{}phi\textquotesingle{} contém os limites: 0 (fixo) e o valor estimado}
\NormalTok{breaks\_finais }\OtherTok{\textless{}{-}} \FunctionTok{c}\NormalTok{(}\SpecialCharTok{{-}}\ConstantTok{Inf}\NormalTok{, mod\_sar\_ordered}\SpecialCharTok{$}\NormalTok{phi, }\ConstantTok{Inf}\NormalTok{)}

\CommentTok{\#Classificação}
\NormalTok{sp\_dados}\SpecialCharTok{$}\NormalTok{cat\_predita\_class }\OtherTok{\textless{}{-}} \FunctionTok{cut}\NormalTok{(sp\_dados}\SpecialCharTok{$}\NormalTok{latente\_predita, }
                                  \AttributeTok{breaks =}\NormalTok{ breaks\_finais, }
                                  \AttributeTok{labels =} \ConstantTok{FALSE}\NormalTok{)}

\CommentTok{\#}
\ControlFlowTok{if}\NormalTok{(}\FunctionTok{any}\NormalTok{(}\FunctionTok{is.na}\NormalTok{(sp\_dados}\SpecialCharTok{$}\NormalTok{cat\_predita\_class))) \{}
\NormalTok{  sp\_dados}\SpecialCharTok{$}\NormalTok{cat\_predita\_class[}\FunctionTok{is.na}\NormalTok{(sp\_dados}\SpecialCharTok{$}\NormalTok{cat\_predita\_class)] }\OtherTok{\textless{}{-}} \DecValTok{1}
\NormalTok{\}}


\CommentTok{\#CÁLCULO DOS RESÍDUOS GENERALIZADOS}

\CommentTok{\#Chesher, A. and Irish, M., 1987. Residual analysis in the grouped and censored normal linear model. Journal of Econometrics, 34(1{-}2), pp.33{-}61.}

\CommentTok{\#Gourieroux, C., Monfort, A., Renault, E. and Trognon, A., 1987. Generalised residuals. Journal of econometrics, 34(1{-}2), pp.5{-}32.}


\NormalTok{beta\_hat }\OtherTok{\textless{}{-}}\NormalTok{ resumo\_bayesiano}\SpecialCharTok{$}\NormalTok{Estimativa[resumo\_bayesiano}\SpecialCharTok{$}\NormalTok{Parametro }\SpecialCharTok{==} \StringTok{"Variável X"}\NormalTok{] }
\NormalTok{intercepto }\OtherTok{\textless{}{-}}\NormalTok{ resumo\_bayesiano}\SpecialCharTok{$}\NormalTok{Estimativa[resumo\_bayesiano}\SpecialCharTok{$}\NormalTok{Parametro }\SpecialCharTok{==} \StringTok{"Intercepto"}\NormalTok{]}
\NormalTok{rho\_hat }\OtherTok{\textless{}{-}} \FunctionTok{mean}\NormalTok{(mod\_sar\_ordered}\SpecialCharTok{$}\NormalTok{rho)}
\NormalTok{cuts }\OtherTok{\textless{}{-}} \FunctionTok{c}\NormalTok{(}\SpecialCharTok{{-}}\ConstantTok{Inf}\NormalTok{, }\DecValTok{0}\NormalTok{, mod\_sar\_ordered}\SpecialCharTok{$}\NormalTok{phi, }\ConstantTok{Inf}\NormalTok{) }\CommentTok{\# Cuts: 0 é fixo no spatialprobit}

\CommentTok{\#y* = (I {-} rho*W)\^{}{-}1 * (X*beta)}

\NormalTok{X\_mat }\OtherTok{\textless{}{-}} \FunctionTok{model.matrix}\NormalTok{(}\SpecialCharTok{\textasciitilde{}}\NormalTok{ x\_var, }\AttributeTok{data =}\NormalTok{ sp\_dados) }
\NormalTok{betas\_vec }\OtherTok{\textless{}{-}} \FunctionTok{c}\NormalTok{(intercepto, beta\_hat) }\CommentTok{\# Ordem deve bater com X\_mat}

\CommentTok{\#X * Beta}
\NormalTok{xb }\OtherTok{\textless{}{-}}\NormalTok{ X\_mat }\SpecialCharTok{\%*\%}\NormalTok{ betas\_vec}

\CommentTok{\#(I {-} rho * W)\^{}{-}1}

\NormalTok{I\_n }\OtherTok{\textless{}{-}}\NormalTok{ Matrix}\SpecialCharTok{::}\FunctionTok{Diagonal}\NormalTok{(}\FunctionTok{nrow}\NormalTok{(sp\_dados)) }\CommentTok{\#I}
\NormalTok{S\_inv }\OtherTok{\textless{}{-}} \FunctionTok{solve}\NormalTok{(I\_n }\SpecialCharTok{{-}}\NormalTok{ rho\_hat }\SpecialCharTok{*}\NormalTok{ W\_mat) }\CommentTok{\#(I {-} rho * W)\^{}{-}1}
\NormalTok{y\_star\_pred }\OtherTok{\textless{}{-}} \FunctionTok{as.vector}\NormalTok{(S\_inv }\SpecialCharTok{\%*\%}\NormalTok{ xb)  }\CommentTok{\#(I {-} rho * W)\^{}{-}1 *xb}

\CommentTok{\#E[y* | y\_obs]: mu + sigma * (pdf(a) {-} pdf(b)) / (cdf(b) {-} cdf(a))}

\NormalTok{y\_obs }\OtherTok{\textless{}{-}} \FunctionTok{as.numeric}\NormalTok{(sp\_dados}\SpecialCharTok{$}\NormalTok{y\_ordered)}
\NormalTok{lo }\OtherTok{\textless{}{-}}\NormalTok{ cuts[y\_obs]     }\CommentTok{\# Limite inferior da categoria observada}
\NormalTok{hi }\OtherTok{\textless{}{-}}\NormalTok{ cuts[y\_obs }\SpecialCharTok{+} \DecValTok{1}\NormalTok{] }\CommentTok{\# Limite superior da categoria observada}

\NormalTok{z\_lo }\OtherTok{\textless{}{-}}\NormalTok{ lo }\SpecialCharTok{{-}}\NormalTok{ y\_star\_pred}
\NormalTok{z\_hi }\OtherTok{\textless{}{-}}\NormalTok{ hi }\SpecialCharTok{{-}}\NormalTok{ y\_star\_pred}

\NormalTok{safe\_pnorm }\OtherTok{\textless{}{-}} \ControlFlowTok{function}\NormalTok{(q) }\FunctionTok{pnorm}\NormalTok{(q)}
\NormalTok{safe\_dnorm }\OtherTok{\textless{}{-}} \ControlFlowTok{function}\NormalTok{(x) }\FunctionTok{dnorm}\NormalTok{(x)}

\NormalTok{diff\_cdf }\OtherTok{\textless{}{-}} \FunctionTok{safe\_pnorm}\NormalTok{(z\_hi) }\SpecialCharTok{{-}} \FunctionTok{safe\_pnorm}\NormalTok{(z\_lo)}
\NormalTok{diff\_cdf[diff\_cdf }\SpecialCharTok{\textless{}} \FloatTok{1e{-}10}\NormalTok{] }\OtherTok{\textless{}{-}} \FloatTok{1e{-}10} 

\NormalTok{diff\_pdf }\OtherTok{\textless{}{-}} \FunctionTok{safe\_dnorm}\NormalTok{(z\_lo) }\SpecialCharTok{{-}} \FunctionTok{safe\_dnorm}\NormalTok{(z\_hi) }\CommentTok{\# Note a ordem: pdf(lo) {-} pdf(hi)}

\CommentTok{\# E[y* | y] = mu + (phi(lo) {-} phi(hi)) / (Phi(hi) {-} Phi(lo))}
\NormalTok{sp\_dados}\SpecialCharTok{$}\NormalTok{y\_latente\_esperada }\OtherTok{\textless{}{-}}\NormalTok{ y\_star\_pred }\SpecialCharTok{+}\NormalTok{ (diff\_pdf }\SpecialCharTok{/}\NormalTok{ diff\_cdf)}

\CommentTok{\# u = (I {-} rho*W) * E[y*|y] {-} X*beta}
\NormalTok{A\_mat }\OtherTok{\textless{}{-}}\NormalTok{ (I\_n }\SpecialCharTok{{-}}\NormalTok{ rho\_hat }\SpecialCharTok{*}\NormalTok{ W\_mat)}
\NormalTok{term\_spatial\_removed }\OtherTok{\textless{}{-}} \FunctionTok{as.vector}\NormalTok{(A\_mat }\SpecialCharTok{\%*\%}\NormalTok{ sp\_dados}\SpecialCharTok{$}\NormalTok{y\_latente\_esperada)}

\NormalTok{sp\_dados}\SpecialCharTok{$}\NormalTok{resid\_generalized }\OtherTok{\textless{}{-}}\NormalTok{ term\_spatial\_removed }\SpecialCharTok{{-}} \FunctionTok{as.vector}\NormalTok{(xb)}

\CommentTok{\#Teste de Moran}
\NormalTok{moran\_resid }\OtherTok{\textless{}{-}} \FunctionTok{moran.test}\NormalTok{(sp\_dados}\SpecialCharTok{$}\NormalTok{resid\_generalized, lw\_sp)}

\NormalTok{label\_moran }\OtherTok{\textless{}{-}} \FunctionTok{paste0}\NormalTok{(}\StringTok{"Moran (Gen. Resid): "}\NormalTok{, }\FunctionTok{round}\NormalTok{(moran\_resid}\SpecialCharTok{$}\NormalTok{estimate[}\DecValTok{1}\NormalTok{], }\DecValTok{3}\NormalTok{), }
                      \StringTok{" (p: "}\NormalTok{, }\FunctionTok{round}\NormalTok{(moran\_resid}\SpecialCharTok{$}\NormalTok{p.value, }\DecValTok{3}\NormalTok{), }\StringTok{")"}\NormalTok{)}

\NormalTok{max\_res }\OtherTok{\textless{}{-}} \FunctionTok{max}\NormalTok{(}\FunctionTok{abs}\NormalTok{(sp\_dados}\SpecialCharTok{$}\NormalTok{resid\_generalized), }\AttributeTok{na.rm=}\ConstantTok{TRUE}\NormalTok{)}

\CommentTok{\# Graficos}
\NormalTok{theme\_map\_custom }\OtherTok{\textless{}{-}} \ControlFlowTok{function}\NormalTok{() \{}
  \FunctionTok{list}\NormalTok{(}
    \FunctionTok{theme\_void}\NormalTok{(),}
    \FunctionTok{theme}\NormalTok{(}
      \AttributeTok{legend.position =} \StringTok{"bottom"}\NormalTok{, }
      \AttributeTok{legend.box.spacing =} \FunctionTok{unit}\NormalTok{(}\DecValTok{5}\NormalTok{, }\StringTok{"pt"}\NormalTok{),}
      \AttributeTok{legend.title =} \FunctionTok{element\_text}\NormalTok{(}\AttributeTok{size=}\DecValTok{9}\NormalTok{, }\AttributeTok{face=}\StringTok{"bold"}\NormalTok{),}
      \AttributeTok{plot.title =} \FunctionTok{element\_text}\NormalTok{(}\AttributeTok{face=}\StringTok{"bold"}\NormalTok{, }\AttributeTok{size=}\DecValTok{12}\NormalTok{, }\AttributeTok{hjust =} \DecValTok{0}\NormalTok{),}
      \AttributeTok{plot.subtitle =} \FunctionTok{element\_text}\NormalTok{(}\AttributeTok{size=}\DecValTok{9}\NormalTok{, }\AttributeTok{color=}\StringTok{"grey30"}\NormalTok{)}
\NormalTok{    ),}
    \FunctionTok{annotation\_scale}\NormalTok{(}\AttributeTok{location =} \StringTok{"br"}\NormalTok{, }\AttributeTok{width\_hint =} \FloatTok{0.3}\NormalTok{),}
    \FunctionTok{annotation\_north\_arrow}\NormalTok{(}\AttributeTok{location =} \StringTok{"tr"}\NormalTok{, }\AttributeTok{height =} \FunctionTok{unit}\NormalTok{(}\DecValTok{1}\NormalTok{, }\StringTok{"cm"}\NormalTok{), }\AttributeTok{width =} \FunctionTok{unit}\NormalTok{(}\DecValTok{1}\NormalTok{, }\StringTok{"cm"}\NormalTok{),}
                           \AttributeTok{style =}\NormalTok{ north\_arrow\_fancy\_orienteering)}
\NormalTok{  )}
\NormalTok{\}}

\CommentTok{\#Observado}
\NormalTok{g\_obs }\OtherTok{\textless{}{-}} \FunctionTok{ggplot}\NormalTok{(sp\_dados) }\SpecialCharTok{+}
  \FunctionTok{geom\_sf}\NormalTok{(}\FunctionTok{aes}\NormalTok{(}\AttributeTok{fill =} \FunctionTok{factor}\NormalTok{(y\_ordered, }\AttributeTok{levels =} \DecValTok{1}\SpecialCharTok{:}\DecValTok{3}\NormalTok{)), }\AttributeTok{color =} \StringTok{"white"}\NormalTok{, }\AttributeTok{lwd =} \FloatTok{0.02}\NormalTok{) }\SpecialCharTok{+}
  \FunctionTok{scale\_fill\_viridis\_d}\NormalTok{(}\AttributeTok{option =} \StringTok{"viridis"}\NormalTok{, }\AttributeTok{name =} \StringTok{"Observed"}\NormalTok{, }\AttributeTok{drop =} \ConstantTok{FALSE}\NormalTok{) }\SpecialCharTok{+}
  \FunctionTok{labs}\NormalTok{(}\AttributeTok{title =} \StringTok{"A. Dados Observados"}\NormalTok{, }\AttributeTok{subtitle =} \StringTok{"Variável Dependente Real"}\NormalTok{) }\SpecialCharTok{+}
  \FunctionTok{theme\_map\_custom}\NormalTok{()}

\CommentTok{\#Predito}
\NormalTok{g\_pred }\OtherTok{\textless{}{-}} \FunctionTok{ggplot}\NormalTok{(sp\_dados) }\SpecialCharTok{+}
  \FunctionTok{geom\_sf}\NormalTok{(}\FunctionTok{aes}\NormalTok{(}\AttributeTok{fill =} \FunctionTok{factor}\NormalTok{(cat\_predita\_class, }\AttributeTok{levels =} \DecValTok{1}\SpecialCharTok{:}\DecValTok{3}\NormalTok{)), }\AttributeTok{color =} \StringTok{"white"}\NormalTok{, }\AttributeTok{lwd =} \FloatTok{0.02}\NormalTok{) }\SpecialCharTok{+}
  \FunctionTok{scale\_fill\_viridis\_d}\NormalTok{(}\AttributeTok{option =} \StringTok{"viridis"}\NormalTok{, }\AttributeTok{name =} \StringTok{"Predicted"}\NormalTok{, }\AttributeTok{drop =} \ConstantTok{FALSE}\NormalTok{) }\SpecialCharTok{+}
  \FunctionTok{labs}\NormalTok{(}\AttributeTok{title =} \StringTok{"B. Predição do Modelo"}\NormalTok{) }\SpecialCharTok{+}
  \FunctionTok{theme\_map\_custom}\NormalTok{()}

\CommentTok{\#Impactos}
\NormalTok{g\_imp }\OtherTok{\textless{}{-}} \FunctionTok{ggplot}\NormalTok{(df\_imp, }\FunctionTok{aes}\NormalTok{(}\AttributeTok{x =}\NormalTok{ Tipo, }\AttributeTok{y =}\NormalTok{ Valor, }\AttributeTok{fill =}\NormalTok{ Tipo)) }\SpecialCharTok{+}
  \FunctionTok{geom\_col}\NormalTok{(}\AttributeTok{width =} \FloatTok{0.6}\NormalTok{, }\AttributeTok{color =} \StringTok{"black"}\NormalTok{, }\AttributeTok{alpha =} \FloatTok{0.8}\NormalTok{) }\SpecialCharTok{+}
  \FunctionTok{geom\_text}\NormalTok{(}\FunctionTok{aes}\NormalTok{(}\AttributeTok{label =} \FunctionTok{round}\NormalTok{(Valor, }\DecValTok{2}\NormalTok{)), }\AttributeTok{vjust =} \SpecialCharTok{{-}}\FloatTok{0.5}\NormalTok{, }\AttributeTok{size=}\DecValTok{4}\NormalTok{, }\AttributeTok{fontface =} \StringTok{"bold"}\NormalTok{) }\SpecialCharTok{+}
  \FunctionTok{scale\_fill\_viridis\_d}\NormalTok{(}\AttributeTok{option =} \StringTok{"cividis"}\NormalTok{, }\AttributeTok{begin =} \FloatTok{0.2}\NormalTok{, }\AttributeTok{end =} \FloatTok{0.8}\NormalTok{) }\SpecialCharTok{+}
  \FunctionTok{labs}\NormalTok{(}\AttributeTok{title =} \StringTok{"C. Decomposição de Impactos"}\NormalTok{, }\AttributeTok{y =} \StringTok{"Magnitude"}\NormalTok{, }\AttributeTok{x =} \ConstantTok{NULL}\NormalTok{) }\SpecialCharTok{+}
  \FunctionTok{theme\_minimal}\NormalTok{() }\SpecialCharTok{+} 
  \FunctionTok{theme}\NormalTok{(}\AttributeTok{legend.position =} \StringTok{"none"}\NormalTok{, }\AttributeTok{panel.grid.minor =} \FunctionTok{element\_blank}\NormalTok{())}

\CommentTok{\#Resíduos}
\NormalTok{max\_res }\OtherTok{\textless{}{-}} \FunctionTok{max}\NormalTok{(}\FunctionTok{abs}\NormalTok{(sp\_dados}\SpecialCharTok{$}\NormalTok{resid\_generalized), }\AttributeTok{na.rm=}\ConstantTok{TRUE}\NormalTok{)}
\NormalTok{g\_resid }\OtherTok{\textless{}{-}} \FunctionTok{ggplot}\NormalTok{(sp\_dados) }\SpecialCharTok{+}
  \FunctionTok{geom\_sf}\NormalTok{(}\FunctionTok{aes}\NormalTok{(}\AttributeTok{fill =}\NormalTok{ resid\_generalized), }\AttributeTok{color =} \StringTok{"white"}\NormalTok{, }\AttributeTok{lwd =} \FloatTok{0.02}\NormalTok{) }\SpecialCharTok{+}
  \FunctionTok{scale\_fill\_distiller}\NormalTok{(}\AttributeTok{palette =} \StringTok{"RdBu"}\NormalTok{, }\AttributeTok{direction =} \SpecialCharTok{{-}}\DecValTok{1}\NormalTok{, }
                       \AttributeTok{limits =} \FunctionTok{c}\NormalTok{(}\SpecialCharTok{{-}}\NormalTok{max\_res, max\_res),}
                       \AttributeTok{name =} \StringTok{"Resíduo"}\NormalTok{) }\SpecialCharTok{+}
  \FunctionTok{labs}\NormalTok{(}\AttributeTok{title =} \StringTok{"D. Resíduos"}\NormalTok{, }
       \AttributeTok{subtitle =} \FunctionTok{paste0}\NormalTok{(}\StringTok{"Autocorrelação:}\SpecialCharTok{\textbackslash{}n}\StringTok{"}\NormalTok{, label\_moran)) }\SpecialCharTok{+}
  \FunctionTok{theme\_map\_custom}\NormalTok{()}


\NormalTok{g\_obs}\SpecialCharTok{+}\NormalTok{g\_pred}\SpecialCharTok{+}\NormalTok{g\_imp}\SpecialCharTok{+}\NormalTok{g\_resid}
\end{Highlighting}
\end{Shaded}

\begin{figure}[H]

\centering{

\pandocbounded{\includegraphics[keepaspectratio]{lattice_data_files/figure-pdf/fig-sar_ordered_probit-1.pdf}}

}

\caption{\label{fig-sar_ordered_probit}(A) Observado, (B) Predito, (C)
Impactos e (D) Resíduos Generalizados (Estimativa do erro estrutural).}

\end{figure}%

\textbf{Interpretação}

Os padrões observados em Figura~\ref{fig-sar_ordered_probit} (A) e as
estimativas do modelo em Figura~\ref{fig-sar_ordered_probit} (B),
demonstram a eficácia da especificação modelo probit ordinal na
recuperação da estrutura espacial das categorias, replicando com
fidelidade a heterogeneidade e as aglomerações regionais. A decomposição
dos impactos ilustrada em Figura~\ref{fig-sar_ordered_probit} (C) revela
que a dinâmica do fenômeno é governada preponderantemente por fatores
intra-regionais, uma vez que a magnitude do efeito direto (\(0.86\))
supera a do efeito indireto (\(0.67\)), embora a influência do
transbordamento espacial permaneça substantiva. A validade estatística
das inferências é assegurada em Figura~\ref{fig-sar_ordered_probit} (D),
onde a análise dos resíduos generalizados atesta a ausência de padrões
espaciais sistemáticos; a estatística de I de Moran (\(-0.077\))
associada ao p-valor unitário (\(p=1\)) confirma que o modelo filtrou
adequadamente a autocorrelação espacial, garantindo a independência
estocástica dos erros.

\begin{tcolorbox}[enhanced jigsaw, left=2mm, toptitle=1mm, colback=white, colframe=quarto-callout-note-color-frame, colbacktitle=quarto-callout-note-color!10!white, opacityback=0, rightrule=.15mm, bottomtitle=1mm, arc=.35mm, title=\textcolor{quarto-callout-note-color}{\faInfo}\hspace{0.5em}{Estimativas do Modelo SAR Probit Ordenado}, titlerule=0mm, bottomrule=.15mm, leftrule=.75mm, coltitle=black, toprule=.15mm, breakable, opacitybacktitle=0.6]

Para ajustar o modelo certifique-se de que \(\mathbf{y}^*\) (seja
discreta), \(\mathbf{y}^* \in \{1, \dots, J\}\). Após ajustar o modelo,
os resultados não serão categóricos, e sim quantitativos contínuos. A
transição da escala contínua para as categorias discretas observadas
\(\mathbf{y}^* \in \{1, \dots, J\}\) é determinada pelo vetor de
limiares \(\boldsymbol{\gamma}\) (referenciado internamente no software
como phi), onde o primeiro corte \(\gamma_1\) é restrito a zero por
definição. Consequentemente, os valores ajustados para
\texttt{beta,\ rho\ e\ phi} representam as médias a posteriori das
distribuições dos parâmetros estimados via MCMC. Para fins de
diagnóstico e inferência estatística, o objeto (modelo ajustado)
armazena não apenas essas médias pontuais e os valores ajustados, mas
também as cadeias completas de simulação
(\texttt{bdraw,\ pdraw,\ phidraw}) e os metadados da matriz de pesos
\(\mathbf{W}\), permitindo ao pesquisador avaliar a convergência e a
incerteza associada aos parâmetros espaciais e aos cortes.

\end{tcolorbox}

\subsection{\texorpdfstring{Modelo Tobit Espacial (\emph{Spatial Tobit
Model})}{Modelo Tobit Espacial (Spatial Tobit Model)}}\label{modelo-tobit-espacial-spatial-tobit-model}

O Modelo Tobit Espacial é a extensão para processos espaciais onde a
variável dependente observada é contínua, mas sujeita a censura. Esta
situação ocorre frequentemente quando a variável de interesse assume um
valor limite (comumente zero) para uma proporção substantiva das
observações, enquanto apresenta variação contínua acima desse limite.
Exemplos incluem dados de despesas, fluxos comerciais ou níveis de
poluição, onde muitas unidades registram zero, mas os valores positivos
são contínuos (J. LeSage e Pace 2009).

A aplicação de um modelo Tobit padrão, que assume independência entre as
observações, ou de um modelo linear espacial, que ignora a censura, a
dados com tais características produz estimadores viesados e
inconsistentes dos parâmetros de interesse.

O modelo é formulado utilizando uma variável latente contínua não
observada, \(y_i^*\), que segue um processo autorregressivo espacial.
Para o Modelo Tobit Espacial Autorregressivo (SAR Tobit), a equação
estrutural é (J. LeSage e Pace 2009):

\[
y_i^* = \rho \sum_{j=1}^n w_{ij} y_j^* + \mathbf{x}_i^{\top}\boldsymbol{\beta} + \epsilon_i, \quad \epsilon_i \sim \mathcal{N}(0, \sigma^2).
\]

A variável observada \(y_i\) relaciona-se com sua contraparte latente
através da seguinte regra de
\href{https://en.wikipedia.org/wiki/Censoring_(statistics)}{censura} à
esquerda no limiar zero:

\[
y_i = \max(0, y_i^*).
\]

Aqui, \(\mathbf{W}\) é a matriz de pesos espaciais, \(\rho\) o parâmetro
de autocorrelação espacial, \(\mathbf{X}\) a matriz de covariáveis e
\(\boldsymbol{\beta}\) o vetor de coeficientes.

A introdução da dependência espacial na variável latente tem implicações
importantes na estrutura do erro na forma reduzida. Resolvendo a equação
estrutural para o vetor latente \(\mathbf{y}^*\), obtém-se:

\[
\mathbf{y}^* = (\mathbf{I}_n - \rho \mathbf{W})^{-1}\mathbf{X}\boldsymbol{\beta} + (\mathbf{I}_n - \rho \mathbf{W})^{-1}\boldsymbol{\epsilon}.
\]

Definindo o termo de erro composto como
\(\tilde{\boldsymbol{\epsilon}} = (\mathbf{I}_n - \rho \mathbf{W})^{-1}\boldsymbol{\epsilon}\),
sua matriz de covariância é:

\[
\operatorname{Cov}(\tilde{\boldsymbol{\epsilon}}) = \sigma^2 [(\mathbf{I}_n - \rho \mathbf{W})(\mathbf{I}_n - \rho \mathbf{W})^{\top}]^{-1}.
\]

Conforme demonstrado por McMillen (1992), os elementos da diagonal
principal desta matriz não são constantes. Cada variância
\(\tilde{\sigma}_i^2\) é uma função da localização da unidade \(i\) na
rede definida por \(\mathbf{W}\). Esta heterocedasticidade induzida
espacialmente na forma reduzida do modelo é uma consequência direta da
dependência espacial. Em modelos não lineares como o Tobit, a violação
do pressuposto de homocedasticidade leva à inconsistência do estimador
de máxima verossimilhança padrão, não apenas a uma perda de eficiência
(Billé e Arbia 2019; H. Kelejian e Piras 2017).

A função de verossimilhança para o modelo Tobit Espacial envolve a
probabilidade conjunta de observar os valores \(\mathbf{y}\), o que
requer a integração sobre uma distribuição normal multivariada truncada,
com a dimensão da integral igual ao número de observações censuradas.
Este cálculo é computacionalmente intratável para amostras de tamanho
moderado ou grande.

A literatura propõe várias estratégias para superar este obstáculo:

\begin{enumerate}
\def\labelenumi{\arabic{enumi}.}
\item
  Algoritmo EM (\emph{Expectation-Maximization}): McMillen (1992)
  adaptou o algoritmo EM para este contexto. O método itera entre um
  passo E, que imputa os valores esperados da variável latente \(y_i^*\)
  para as observações censuradas (condicional nos parâmetros atuais), e
  um passo M, que maximiza a verossimilhança de um modelo espacial
  linear contínuo utilizando os dados latentes completos. Embora produza
  estimativas consistentes, a obtenção de erros-padrão válidos é não
  trivial.
\item
  Abordagem Bayesiana (MCMC com Aumento de Dados): Esta é uma abordagem
  robusta e amplamente utilizada, detalhada por J. P. LeSage (2000) e J.
  LeSage e Pace (2009). O método emprega amostragem de Gibbs e uma
  estratégia de aumento de dados (\emph{data augmentation}), tratando os
  valores latentes das observações censuradas como parâmetros a serem
  estimados. O algoritmo amostra sequencialmente das seguintes
  distribuições condicionais completas:

  \begin{itemize}
  \item
    Os parâmetros \((\boldsymbol{\beta}, \rho, \sigma^2)\) condicionais
    ao vetor latente completo \(\mathbf{y}^*\). Dado \(\mathbf{y}^*\), o
    problema reduz-se à estimação de um modelo SAR Bayesiano padrão.
  \item
    A variável latente \(\mathbf{y}^*\) condicional aos parâmetros e aos
    dados observados \(\mathbf{y}\). Para uma observação não censurada
    (\(y_i > 0\)), temos \(y_i^* = y_i\).
  \end{itemize}
\end{enumerate}

Para uma observação censurada (\(y_i = 0\)), amostra-se \(y_i^*\) de uma
distribuição normal truncada à esquerda em zero,
\(y_i^* | \cdot \sim \mathcal{N}_{(-\infty, 0]}(E[y_i^* | \cdot], \operatorname{Var}(y_i^* | \cdot))\),
onde a média e a variância condicionais incorporam a dependência
espacial dos vizinhos.

Esta abordagem fornece a distribuição a posteriori completa dos
parâmetros, tratando adequadamente a incerteza associada aos valores
censurados e à estrutura de dependência.

\begin{enumerate}
\def\labelenumi{\arabic{enumi}.}
\setcounter{enumi}{2}
\item
  Métodos de Simulação (GHK): Técnicas de simulação, como o simulador
  GHK, podem ser empregadas para aproximar a integral da verossimilhança
  (Fleming 2004). No entanto, o custo computacional pode tornar-se
  proibitivo para amostras grandes com alta proporção de censura.
\item
  Antes de uma estimação de um modelo complexo, é recomendável testar a
  presença de dependência espacial nos resíduos de um modelo Tobit
  padrão (teste de especificação). H. Kelejian e Piras (2017) propõem
  versões generalizadas do teste I de Moran adaptadas para resíduos de
  modelos Tobit.
\end{enumerate}

No modelo Tobit Espacial, a interpretação dos coeficientes
\(\boldsymbol{\beta}\) não é direta. O efeito de uma mudança em uma
covariável \(x_{ik}\) sobre o valor esperado da variável observada
\(E[y_i | \mathbf{X}]\) é não linear e depende do multiplicador espacial
global \((\mathbf{I}_n - \rho \mathbf{W})^{-1}\) e da probabilidade de a
observação não ser censurada.

Seguindo a decomposição de J. LeSage e Pace (2009), é necessário
calcular os efeitos médios diretos, indiretos (de transbordamento) e
totais.

\begin{itemize}
\item
  O efeito direto médio captura o impacto esperado de uma mudança em
  \(x_{ik}\) sobre \(y_i\), incluindo os \emph{feedbacks} espaciais que
  retornam à unidade \(i\).
\item
  O efeito indireto médio captura o impacto esperado da mudança em
  \(x_{ik}\) sobre todos os outros \(y_j\) (\(j \neq i\)), ou seja, os
  \emph{spillovers} espaciais.
\item
  O efeito total médio é a soma dos dois.
\end{itemize}

Estes efeitos são calculados a partir das derivadas parciais de
\(E[\mathbf{y} | \mathbf{X}]\) em relação a \(\mathbf{x}_k\), que
envolvem a matriz \((\mathbf{I}_n - \rho \mathbf{W})^{-1}\) e os termos
da função de distribuição normal associados à probabilidade de censura.
Como variam entre observações, a prática padrão é reportar suas médias
amostrais.

\begin{table}

\caption{\label{tbl-sar_tobitT}Resultados da Estimação: Modelo SAR Tobit
(Bayesiano) - SP.}

\centering{

\begin{Shaded}
\begin{Highlighting}[]
\ControlFlowTok{if}\NormalTok{ (}\SpecialCharTok{!}\FunctionTok{require}\NormalTok{(}\StringTok{"pacman"}\NormalTok{)) }\FunctionTok{install.packages}\NormalTok{(}\StringTok{"pacman"}\NormalTok{)}
\NormalTok{pacman}\SpecialCharTok{::}\FunctionTok{p\_load}\NormalTok{(spatialprobit, spdep, sf, geobr, ggplot2, viridis, kableExtra, dplyr, Matrix, patchwork, ggspatial, scales)}

\CommentTok{\# Preparação e Simulação (DADOS TOBIT {-} CENSURA EM 0)}

\ControlFlowTok{if}\NormalTok{ (}\SpecialCharTok{!}\FunctionTok{exists}\NormalTok{(}\StringTok{"sp\_dados"}\NormalTok{) }\SpecialCharTok{||} \SpecialCharTok{!}\NormalTok{(}\StringTok{"y\_tobit"} \SpecialCharTok{\%in\%} \FunctionTok{names}\NormalTok{(sp\_dados))) \{}
\NormalTok{  sp\_dados }\OtherTok{\textless{}{-}}\NormalTok{ geobr}\SpecialCharTok{::}\FunctionTok{read\_municipality}\NormalTok{(}\AttributeTok{code\_muni =} \StringTok{"SP"}\NormalTok{, }\AttributeTok{year =} \DecValTok{2020}\NormalTok{, }\AttributeTok{showProgress =} \ConstantTok{FALSE}\NormalTok{)}
\NormalTok{  coords }\OtherTok{\textless{}{-}} \FunctionTok{st\_coordinates}\NormalTok{(}\FunctionTok{st\_centroid}\NormalTok{(sp\_dados))}
  
\NormalTok{  knn }\OtherTok{\textless{}{-}} \FunctionTok{knearneigh}\NormalTok{(coords, }\AttributeTok{k =} \DecValTok{6}\NormalTok{)}
\NormalTok{  nb\_sp }\OtherTok{\textless{}{-}} \FunctionTok{knn2nb}\NormalTok{(knn)}
\NormalTok{  lw\_sp }\OtherTok{\textless{}{-}} \FunctionTok{nb2listw}\NormalTok{(nb\_sp, }\AttributeTok{style =} \StringTok{"W"}\NormalTok{)}
\NormalTok{  W\_mat }\OtherTok{\textless{}{-}} \FunctionTok{as}\NormalTok{(lw\_sp, }\StringTok{"CsparseMatrix"}\NormalTok{)}
  
  \FunctionTok{set.seed}\NormalTok{(}\DecValTok{123}\NormalTok{)}
\NormalTok{  n }\OtherTok{\textless{}{-}} \FunctionTok{nrow}\NormalTok{(sp\_dados)}
\NormalTok{  rho\_true }\OtherTok{\textless{}{-}} \FloatTok{0.60}
\NormalTok{  beta\_x }\OtherTok{\textless{}{-}} \FloatTok{2.0}
\NormalTok{  sigma\_true }\OtherTok{\textless{}{-}} \FloatTok{1.5} 
  
\NormalTok{  sp\_dados}\SpecialCharTok{$}\NormalTok{x\_var }\OtherTok{\textless{}{-}} \FunctionTok{rnorm}\NormalTok{(n, }\DecValTok{0}\NormalTok{, }\DecValTok{1}\NormalTok{)}
  
\NormalTok{  I\_n }\OtherTok{\textless{}{-}}\NormalTok{ Matrix}\SpecialCharTok{::}\FunctionTok{Diagonal}\NormalTok{(n)}
\NormalTok{  inv\_spatial }\OtherTok{\textless{}{-}} \FunctionTok{solve}\NormalTok{(I\_n }\SpecialCharTok{{-}}\NormalTok{ rho\_true }\SpecialCharTok{*}\NormalTok{ W\_mat)}
\NormalTok{  epsilon }\OtherTok{\textless{}{-}} \FunctionTok{rnorm}\NormalTok{(n, }\DecValTok{0}\NormalTok{, sigma\_true)}
  
\NormalTok{  y\_latente }\OtherTok{\textless{}{-}} \FunctionTok{as.vector}\NormalTok{(inv\_spatial }\SpecialCharTok{\%*\%}\NormalTok{ (}\SpecialCharTok{{-}}\DecValTok{1} \SpecialCharTok{+}\NormalTok{ beta\_x }\SpecialCharTok{*}\NormalTok{ sp\_dados}\SpecialCharTok{$}\NormalTok{x\_var }\SpecialCharTok{+}\NormalTok{ epsilon))}
  
\NormalTok{  sp\_dados}\SpecialCharTok{$}\NormalTok{y\_tobit }\OtherTok{\textless{}{-}} \FunctionTok{pmax}\NormalTok{(}\DecValTok{0}\NormalTok{, y\_latente)}
  
\NormalTok{\} }\ControlFlowTok{else}\NormalTok{ \{}
    \ControlFlowTok{if}\NormalTok{ (}\SpecialCharTok{!}\FunctionTok{exists}\NormalTok{(}\StringTok{"W\_mat"}\NormalTok{)) \{}
\NormalTok{      knn }\OtherTok{\textless{}{-}} \FunctionTok{knearneigh}\NormalTok{(}\FunctionTok{st\_coordinates}\NormalTok{(}\FunctionTok{st\_centroid}\NormalTok{(sp\_dados)), }\AttributeTok{k =} \DecValTok{6}\NormalTok{)}
\NormalTok{      nb\_sp }\OtherTok{\textless{}{-}} \FunctionTok{knn2nb}\NormalTok{(knn)}
\NormalTok{      lw\_sp }\OtherTok{\textless{}{-}} \FunctionTok{nb2listw}\NormalTok{(nb\_sp, }\AttributeTok{style =} \StringTok{"W"}\NormalTok{)}
\NormalTok{      W\_mat }\OtherTok{\textless{}{-}} \FunctionTok{as}\NormalTok{(lw\_sp, }\StringTok{"CsparseMatrix"}\NormalTok{)}
\NormalTok{  \}}
\NormalTok{\}}

\CommentTok{\# Ajuste do Modelo (SAR TOBIT)}
\NormalTok{mod\_sar\_tobit }\OtherTok{\textless{}{-}} \FunctionTok{sartobit}\NormalTok{(y\_tobit }\SpecialCharTok{\textasciitilde{}}\NormalTok{ x\_var, }
                          \AttributeTok{W =}\NormalTok{ W\_mat, }
                          \AttributeTok{data =}\NormalTok{ sp\_dados, }
                          \AttributeTok{ndraw =} \DecValTok{1000}\NormalTok{, }
                          \AttributeTok{burn.in =} \DecValTok{200}\NormalTok{, }
                          \AttributeTok{showProgress =} \ConstantTok{FALSE}\NormalTok{)}

\CommentTok{\# Tabela de Resultados }

\NormalTok{draws\_beta }\OtherTok{\textless{}{-}} \FunctionTok{as.data.frame}\NormalTok{(mod\_sar\_tobit}\SpecialCharTok{$}\NormalTok{B)}
\ControlFlowTok{if}\NormalTok{ (}\SpecialCharTok{!}\FunctionTok{is.null}\NormalTok{(mod\_sar\_tobit}\SpecialCharTok{$}\NormalTok{names) }\SpecialCharTok{\&\&} \FunctionTok{length}\NormalTok{(mod\_sar\_tobit}\SpecialCharTok{$}\NormalTok{names) }\SpecialCharTok{==} \FunctionTok{ncol}\NormalTok{(draws\_beta)) \{}
  \FunctionTok{colnames}\NormalTok{(draws\_beta) }\OtherTok{\textless{}{-}}\NormalTok{ mod\_sar\_tobit}\SpecialCharTok{$}\NormalTok{names}
\NormalTok{\}}

\CommentTok{\#}
\ControlFlowTok{if}\NormalTok{ (}\SpecialCharTok{!}\FunctionTok{is.null}\NormalTok{(mod\_sar\_tobit}\SpecialCharTok{$}\NormalTok{pdraw)) \{}
\NormalTok{  draws\_rho }\OtherTok{\textless{}{-}} \FunctionTok{data.frame}\NormalTok{(}\AttributeTok{rho =} \FunctionTok{as.vector}\NormalTok{(mod\_sar\_tobit}\SpecialCharTok{$}\NormalTok{pdraw))}
\NormalTok{\} }\ControlFlowTok{else}\NormalTok{ \{}
\NormalTok{  draws\_rho }\OtherTok{\textless{}{-}} \FunctionTok{data.frame}\NormalTok{(}\AttributeTok{rho =} \FunctionTok{as.vector}\NormalTok{(mod\_sar\_tobit}\SpecialCharTok{$}\NormalTok{rho))}
\NormalTok{\}}

\CommentTok{\#}
\NormalTok{draws\_sigma }\OtherTok{\textless{}{-}} \FunctionTok{data.frame}\NormalTok{(}\AttributeTok{sigma2 =} \FunctionTok{as.vector}\NormalTok{(mod\_sar\_tobit}\SpecialCharTok{$}\NormalTok{sdraw))}

\CommentTok{\#}
\ControlFlowTok{if}\NormalTok{ (}\StringTok{"rho"} \SpecialCharTok{\%in\%} \FunctionTok{colnames}\NormalTok{(draws\_beta)) \{}
\NormalTok{  draws\_beta }\OtherTok{\textless{}{-}}\NormalTok{ draws\_beta[, }\SpecialCharTok{!}\FunctionTok{colnames}\NormalTok{(draws\_beta) }\SpecialCharTok{\%in\%} \StringTok{"rho"}\NormalTok{]}
\NormalTok{\}}

\NormalTok{all\_draws }\OtherTok{\textless{}{-}} \FunctionTok{cbind}\NormalTok{(draws\_beta, draws\_rho, draws\_sigma)}

\CommentTok{\# Estatísticas}
\NormalTok{resumo\_bayesiano }\OtherTok{\textless{}{-}} \FunctionTok{data.frame}\NormalTok{(}
  \AttributeTok{Parametro  =} \FunctionTok{names}\NormalTok{(all\_draws),}
  \AttributeTok{Estimativa =} \FunctionTok{colMeans}\NormalTok{(all\_draws),}
  \AttributeTok{IC\_Inf     =} \FunctionTok{apply}\NormalTok{(all\_draws, }\DecValTok{2}\NormalTok{, quantile, }\AttributeTok{probs =} \FloatTok{0.025}\NormalTok{),}
  \AttributeTok{IC\_Sup     =} \FunctionTok{apply}\NormalTok{(all\_draws, }\DecValTok{2}\NormalTok{, quantile, }\AttributeTok{probs =} \FloatTok{0.975}\NormalTok{)}
\NormalTok{)}

\NormalTok{resumo\_bayesiano}\SpecialCharTok{$}\NormalTok{Resultado }\OtherTok{\textless{}{-}} \FunctionTok{sprintf}\NormalTok{(}\StringTok{"\%.3f [\%.3f, \%.3f]"}\NormalTok{, }
\NormalTok{                                      resumo\_bayesiano}\SpecialCharTok{$}\NormalTok{Estimativa, }
\NormalTok{                                      resumo\_bayesiano}\SpecialCharTok{$}\NormalTok{IC\_Inf, }
\NormalTok{                                      resumo\_bayesiano}\SpecialCharTok{$}\NormalTok{IC\_Sup)}

\CommentTok{\# Renomear}
\NormalTok{resumo\_bayesiano}\SpecialCharTok{$}\NormalTok{Parametro }\OtherTok{\textless{}{-}}\NormalTok{ dplyr}\SpecialCharTok{::}\FunctionTok{case\_when}\NormalTok{(}
\NormalTok{  resumo\_bayesiano}\SpecialCharTok{$}\NormalTok{Parametro }\SpecialCharTok{\%in\%} \FunctionTok{c}\NormalTok{(}\StringTok{"(Intercept)"}\NormalTok{, }\StringTok{"beta\_1"}\NormalTok{) }\SpecialCharTok{\textasciitilde{}} \StringTok{"Intercepto"}\NormalTok{,}
\NormalTok{  resumo\_bayesiano}\SpecialCharTok{$}\NormalTok{Parametro }\SpecialCharTok{\%in\%} \FunctionTok{c}\NormalTok{(}\StringTok{"x\_var"}\NormalTok{, }\StringTok{"beta\_2"}\NormalTok{) }\SpecialCharTok{\textasciitilde{}} \StringTok{"Variável X"}\NormalTok{,}
  \FunctionTok{grepl}\NormalTok{(}\StringTok{"rho"}\NormalTok{, resumo\_bayesiano}\SpecialCharTok{$}\NormalTok{Parametro, }\AttributeTok{ignore.case =} \ConstantTok{TRUE}\NormalTok{) }\SpecialCharTok{\textasciitilde{}} \StringTok{"$}\SpecialCharTok{\textbackslash{}\textbackslash{}}\StringTok{rho$ (Dependência)"}\NormalTok{,}
  \FunctionTok{grepl}\NormalTok{(}\StringTok{"sigma"}\NormalTok{, resumo\_bayesiano}\SpecialCharTok{$}\NormalTok{Parametro, }\AttributeTok{ignore.case =} \ConstantTok{TRUE}\NormalTok{) }\SpecialCharTok{\textasciitilde{}} \StringTok{"$}\SpecialCharTok{\textbackslash{}\textbackslash{}}\StringTok{sigma\^{}2$ (Variância)"}\NormalTok{,}
  \ConstantTok{TRUE} \SpecialCharTok{\textasciitilde{}}\NormalTok{ resumo\_bayesiano}\SpecialCharTok{$}\NormalTok{Parametro}
\NormalTok{)}

\NormalTok{tabela\_final }\OtherTok{\textless{}{-}}\NormalTok{ resumo\_bayesiano }\SpecialCharTok{\%\textgreater{}\%}\NormalTok{ dplyr}\SpecialCharTok{::}\FunctionTok{select}\NormalTok{(Parametro, Resultado)}
\FunctionTok{rownames}\NormalTok{(tabela\_final) }\OtherTok{\textless{}{-}} \ConstantTok{NULL}

\FunctionTok{kbl}\NormalTok{(tabela\_final, }
    \AttributeTok{format =} \StringTok{"latex"}\NormalTok{, }
    \AttributeTok{booktabs =} \ConstantTok{TRUE}\NormalTok{, }
    \AttributeTok{caption =} \StringTok{""}\NormalTok{, }
    \AttributeTok{escape =} \ConstantTok{FALSE}\NormalTok{) }\SpecialCharTok{\%\textgreater{}\%}
  \FunctionTok{kable\_styling}\NormalTok{(}\AttributeTok{latex\_options =} \FunctionTok{c}\NormalTok{(}\StringTok{"HOLD\_position"}\NormalTok{, }\StringTok{"striped"}\NormalTok{), }
                \AttributeTok{full\_width =} \ConstantTok{FALSE}\NormalTok{, }
                \AttributeTok{position =} \StringTok{"center"}\NormalTok{) }\SpecialCharTok{\%\textgreater{}\%}
  \FunctionTok{row\_spec}\NormalTok{(}\DecValTok{0}\NormalTok{, }\AttributeTok{bold =} \ConstantTok{TRUE}\NormalTok{) }\SpecialCharTok{\%\textgreater{}\%}
  \FunctionTok{footnote}\NormalTok{(}\AttributeTok{general =} \StringTok{"Estimativas: Média a Posteriori [Intervalo de Credibilidade 95\%]."}\NormalTok{) }
\end{Highlighting}
\end{Shaded}

\centering
\caption{\label{tab:tbl-sar_tobitT}}
\centering
\begin{tabular}[t]{ll}
\toprule
\textbf{Parametro} & \textbf{Resultado}\\
\midrule
\cellcolor{gray!10}{Intercepto} & \cellcolor{gray!10}{-2.284 [-3.023, -1.601]}\\
Variável X & 3.389 [2.639, 4.315]\\
\cellcolor{gray!10}{sige} & \cellcolor{gray!10}{7.708 [4.551, 11.878]}\\
$\rho$ (Dependência) & 0.538 [0.413, 0.644]\\
\cellcolor{gray!10}{$\sigma^2$ (Variância)} & \cellcolor{gray!10}{7.708 [4.551, 11.878]}\\
\bottomrule
\multicolumn{2}{l}{\rule{0pt}{1em}\textit{Note: }}\\
\multicolumn{2}{l}{\rule{0pt}{1em}Estimativas: Média a Posteriori [Intervalo de Credibilidade 95\%].}\\
\end{tabular}

}

\end{table}%

\begin{Shaded}
\begin{Highlighting}[]
\ControlFlowTok{if}\NormalTok{ (}\SpecialCharTok{!}\FunctionTok{require}\NormalTok{(}\StringTok{"pacman"}\NormalTok{)) }\FunctionTok{install.packages}\NormalTok{(}\StringTok{"pacman"}\NormalTok{)}
\NormalTok{pacman}\SpecialCharTok{::}\FunctionTok{p\_load}\NormalTok{(spatialprobit, spdep, sf, geobr, ggplot2, viridis, kableExtra, dplyr, Matrix, patchwork, ggspatial, scales)}


\CommentTok{\# A. Impactos (Marginais na Latente)}
\NormalTok{beta\_val }\OtherTok{\textless{}{-}}\NormalTok{ resumo\_bayesiano}\SpecialCharTok{$}\NormalTok{Estimativa[resumo\_bayesiano}\SpecialCharTok{$}\NormalTok{Parametro }\SpecialCharTok{==} \StringTok{"Variável X"}\NormalTok{]}
\NormalTok{rho\_val  }\OtherTok{\textless{}{-}} \FunctionTok{mean}\NormalTok{(draws\_rho}\SpecialCharTok{$}\NormalTok{rho, }\AttributeTok{na.rm =} \ConstantTok{TRUE}\NormalTok{)}

\CommentTok{\#Cálculo}
\NormalTok{impacto\_total\_latente  }\OtherTok{\textless{}{-}}\NormalTok{ beta\_val }\SpecialCharTok{/}\NormalTok{ (}\DecValTok{1} \SpecialCharTok{{-}}\NormalTok{ rho\_val)}
\NormalTok{impacto\_direto\_latente }\OtherTok{\textless{}{-}}\NormalTok{ beta\_val}
\NormalTok{impacto\_indireto\_latente }\OtherTok{\textless{}{-}}\NormalTok{ impacto\_total\_latente }\SpecialCharTok{{-}}\NormalTok{ impacto\_direto\_latente}

\CommentTok{\#}
\NormalTok{df\_imp }\OtherTok{\textless{}{-}} \FunctionTok{data.frame}\NormalTok{(}
  \AttributeTok{Tipo =} \FunctionTok{factor}\NormalTok{(}\FunctionTok{c}\NormalTok{(}\StringTok{"direct"}\NormalTok{, }\StringTok{"indirect"}\NormalTok{, }\StringTok{"total"}\NormalTok{), }
                \AttributeTok{levels =} \FunctionTok{c}\NormalTok{(}\StringTok{"direct"}\NormalTok{, }\StringTok{"indirect"}\NormalTok{, }\StringTok{"total"}\NormalTok{),}
                \AttributeTok{labels =} \FunctionTok{c}\NormalTok{(}\StringTok{"Direto"}\NormalTok{, }\StringTok{"Indireto"}\NormalTok{, }\StringTok{"Total"}\NormalTok{)),}
  \AttributeTok{Valor =} \FunctionTok{c}\NormalTok{(impacto\_direto\_latente, }
\NormalTok{            impacto\_indireto\_latente, }
\NormalTok{            impacto\_total\_latente)}
\NormalTok{)}


\CommentTok{\# Resíduos Generalizados (Chesher \& Irish, 1987)}
\NormalTok{beta\_hat }\OtherTok{\textless{}{-}} \FunctionTok{colMeans}\NormalTok{(draws\_beta)}
\NormalTok{rho\_hat  }\OtherTok{\textless{}{-}} \FunctionTok{mean}\NormalTok{(draws\_rho}\SpecialCharTok{$}\NormalTok{rho)}
\NormalTok{sigma\_hat }\OtherTok{\textless{}{-}} \FunctionTok{sqrt}\NormalTok{(}\FunctionTok{mean}\NormalTok{(draws\_sigma}\SpecialCharTok{$}\NormalTok{sigma2)) }

\CommentTok{\# Matrizes}
\NormalTok{X\_mat }\OtherTok{\textless{}{-}} \FunctionTok{model.matrix}\NormalTok{(}\SpecialCharTok{\textasciitilde{}}\NormalTok{ x\_var, }\AttributeTok{data =}\NormalTok{ sp\_dados)}

\NormalTok{beta\_hat }\OtherTok{\textless{}{-}}\NormalTok{ beta\_hat[}\FunctionTok{colnames}\NormalTok{(draws\_beta) }\SpecialCharTok{\%in\%} \FunctionTok{colnames}\NormalTok{(X\_mat) }\SpecialCharTok{|} \FunctionTok{colnames}\NormalTok{(draws\_beta) }\SpecialCharTok{==} \StringTok{"(Intercept)"}\NormalTok{]}

\NormalTok{xb }\OtherTok{\textless{}{-}}\NormalTok{ X\_mat }\SpecialCharTok{\%*\%}\NormalTok{ beta\_hat}

\NormalTok{I\_n }\OtherTok{\textless{}{-}}\NormalTok{ Matrix}\SpecialCharTok{::}\FunctionTok{Diagonal}\NormalTok{(}\FunctionTok{nrow}\NormalTok{(sp\_dados))}
\NormalTok{S\_inv }\OtherTok{\textless{}{-}} \FunctionTok{solve}\NormalTok{(I\_n }\SpecialCharTok{{-}}\NormalTok{ rho\_hat }\SpecialCharTok{*}\NormalTok{ W\_mat)}

\CommentTok{\# Média Latente (Sem censura): mu = (I {-} rho W)\^{}{-}1 Xb}
\NormalTok{y\_star\_mu }\OtherTok{\textless{}{-}} \FunctionTok{as.vector}\NormalTok{(S\_inv }\SpecialCharTok{\%*\%}\NormalTok{ xb)}

\CommentTok{\# E[y* | y]: }
\CommentTok{\# Se y \textgreater{} 0: E = y\_obs}
\CommentTok{\# Se y = 0: E = mu {-} sigma * (pdf(z)/cdf(z)), onde z = (0 {-} mu)/sigma }

\NormalTok{y\_obs }\OtherTok{\textless{}{-}}\NormalTok{ sp\_dados}\SpecialCharTok{$}\NormalTok{y\_tobit}
\NormalTok{z\_score }\OtherTok{\textless{}{-}}\NormalTok{ (}\DecValTok{0} \SpecialCharTok{{-}}\NormalTok{ y\_star\_mu) }\SpecialCharTok{/}\NormalTok{ sigma\_hat}
\NormalTok{mills\_ratio }\OtherTok{\textless{}{-}} \FunctionTok{dnorm}\NormalTok{(z\_score) }\SpecialCharTok{/} \FunctionTok{pnorm}\NormalTok{(z\_score)}

\NormalTok{y\_latente\_generalized }\OtherTok{\textless{}{-}}\NormalTok{ y\_obs}
\NormalTok{censurados }\OtherTok{\textless{}{-}}\NormalTok{ (y\_obs }\SpecialCharTok{==} \DecValTok{0}\NormalTok{)}

\CommentTok{\# E[y* | y* \textless{} 0] é mu {-} sigma * lambda({-}z\_score)}
\NormalTok{y\_latente\_generalized[censurados] }\OtherTok{\textless{}{-}}\NormalTok{ y\_star\_mu[censurados] }\SpecialCharTok{{-}}\NormalTok{ (sigma\_hat }\SpecialCharTok{*}\NormalTok{ mills\_ratio[censurados])}

\CommentTok{\# u = (I {-} rho W) * y\_generalized {-} Xb}
\NormalTok{term\_spatial\_removed }\OtherTok{\textless{}{-}} \FunctionTok{as.vector}\NormalTok{((I\_n }\SpecialCharTok{{-}}\NormalTok{ rho\_hat }\SpecialCharTok{*}\NormalTok{ W\_mat) }\SpecialCharTok{\%*\%}\NormalTok{ y\_latente\_generalized)}
\NormalTok{sp\_dados}\SpecialCharTok{$}\NormalTok{resid\_generalized }\OtherTok{\textless{}{-}}\NormalTok{ term\_spatial\_removed }\SpecialCharTok{{-}} \FunctionTok{as.vector}\NormalTok{(xb)}

\CommentTok{\# Moran}
\NormalTok{moran\_resid }\OtherTok{\textless{}{-}} \FunctionTok{moran.test}\NormalTok{(sp\_dados}\SpecialCharTok{$}\NormalTok{resid\_generalized, lw\_sp)}
\NormalTok{label\_moran }\OtherTok{\textless{}{-}} \FunctionTok{paste0}\NormalTok{(}\StringTok{"I de Moran: "}\NormalTok{, }\FunctionTok{round}\NormalTok{(moran\_resid}\SpecialCharTok{$}\NormalTok{estimate[}\DecValTok{1}\NormalTok{], }\DecValTok{3}\NormalTok{), }
                      \StringTok{" (p: "}\NormalTok{, }\FunctionTok{round}\NormalTok{(moran\_resid}\SpecialCharTok{$}\NormalTok{p.value, }\DecValTok{3}\NormalTok{), }\StringTok{")"}\NormalTok{)}

\CommentTok{\# Cores}
\NormalTok{cor\_zero }\OtherTok{\textless{}{-}} \StringTok{"white"}
\NormalTok{cor\_um   }\OtherTok{\textless{}{-}} \StringTok{"\#FDE725"}

\CommentTok{\# PLOTS}
\NormalTok{sp\_dados}\SpecialCharTok{$}\NormalTok{y\_pred\_censurado }\OtherTok{\textless{}{-}} \FunctionTok{pmax}\NormalTok{(}\DecValTok{0}\NormalTok{, y\_star\_mu) }

\CommentTok{\# C. Mapa Observado }
\NormalTok{g\_obs }\OtherTok{\textless{}{-}} \FunctionTok{ggplot}\NormalTok{(sp\_dados) }\SpecialCharTok{+}
  \FunctionTok{geom\_sf}\NormalTok{(}\FunctionTok{aes}\NormalTok{(}\AttributeTok{fill =}\NormalTok{ y\_tobit), }\AttributeTok{color =} \StringTok{"black"}\NormalTok{, }\AttributeTok{lwd =} \FloatTok{0.05}\NormalTok{) }\SpecialCharTok{+}
  \FunctionTok{scale\_fill\_viridis\_c}\NormalTok{(}\AttributeTok{option =} \StringTok{"magma"}\NormalTok{, }\AttributeTok{direction =} \SpecialCharTok{{-}}\DecValTok{1}\NormalTok{, }\AttributeTok{name =} \StringTok{"Real"}\NormalTok{) }\SpecialCharTok{+}
  \FunctionTok{labs}\NormalTok{(}\AttributeTok{title =} \StringTok{"C. Observado (y)"}\NormalTok{, }\AttributeTok{subtitle =} \StringTok{"Valores Reais (Censurados em 0)"}\NormalTok{) }\SpecialCharTok{+}
  \FunctionTok{theme\_void}\NormalTok{() }\SpecialCharTok{+} 
  \FunctionTok{theme}\NormalTok{(}\AttributeTok{legend.position =} \StringTok{"bottom"}\NormalTok{, }\AttributeTok{legend.box.spacing =} \FunctionTok{unit}\NormalTok{(}\DecValTok{0}\NormalTok{, }\StringTok{"pt"}\NormalTok{)) }\SpecialCharTok{+}
  \FunctionTok{annotation\_scale}\NormalTok{(}\AttributeTok{location =} \StringTok{"br"}\NormalTok{, }\AttributeTok{width\_hint =} \FloatTok{0.3}\NormalTok{) }\SpecialCharTok{+}
  \FunctionTok{annotation\_north\_arrow}\NormalTok{(}\AttributeTok{location =} \StringTok{"tr"}\NormalTok{, }\AttributeTok{height =} \FunctionTok{unit}\NormalTok{(}\FloatTok{0.8}\NormalTok{, }\StringTok{"cm"}\NormalTok{), }\AttributeTok{width =} \FunctionTok{unit}\NormalTok{(}\FloatTok{0.8}\NormalTok{, }\StringTok{"cm"}\NormalTok{), }
                         \AttributeTok{style =}\NormalTok{ north\_arrow\_fancy\_orienteering)}

\CommentTok{\# Mapa Predito}
\NormalTok{g\_pred }\OtherTok{\textless{}{-}} \FunctionTok{ggplot}\NormalTok{(sp\_dados) }\SpecialCharTok{+}
  \FunctionTok{geom\_sf}\NormalTok{(}\FunctionTok{aes}\NormalTok{(}\AttributeTok{fill =}\NormalTok{ y\_pred\_censurado), }\AttributeTok{color =} \StringTok{"black"}\NormalTok{, }\AttributeTok{lwd =} \FloatTok{0.05}\NormalTok{) }\SpecialCharTok{+}
  \FunctionTok{scale\_fill\_viridis\_c}\NormalTok{(}\AttributeTok{option =} \StringTok{"magma"}\NormalTok{, }\AttributeTok{direction =} \SpecialCharTok{{-}}\DecValTok{1}\NormalTok{, }\AttributeTok{name =} \StringTok{"Predito"}\NormalTok{,}
                       \AttributeTok{limits =} \FunctionTok{c}\NormalTok{(}\DecValTok{0}\NormalTok{, }\FunctionTok{max}\NormalTok{(sp\_dados}\SpecialCharTok{$}\NormalTok{y\_tobit))) }\SpecialCharTok{+}
  \FunctionTok{labs}\NormalTok{(}\AttributeTok{title =} \StringTok{"D. Predição de Valores (y)"}\NormalTok{, }\AttributeTok{subtitle =} \StringTok{"Expectativa dos Valores Observáveis"}\NormalTok{) }\SpecialCharTok{+}
  \FunctionTok{theme\_void}\NormalTok{() }\SpecialCharTok{+} 
  \FunctionTok{theme}\NormalTok{(}\AttributeTok{legend.position =} \StringTok{"bottom"}\NormalTok{, }\AttributeTok{legend.box.spacing =} \FunctionTok{unit}\NormalTok{(}\DecValTok{0}\NormalTok{, }\StringTok{"pt"}\NormalTok{)) }\SpecialCharTok{+}
  \FunctionTok{annotation\_scale}\NormalTok{(}\AttributeTok{location =} \StringTok{"br"}\NormalTok{, }\AttributeTok{width\_hint =} \FloatTok{0.3}\NormalTok{) }\SpecialCharTok{+}
  \FunctionTok{annotation\_north\_arrow}\NormalTok{(}\AttributeTok{location =} \StringTok{"tr"}\NormalTok{, }\AttributeTok{height =} \FunctionTok{unit}\NormalTok{(}\FloatTok{0.8}\NormalTok{, }\StringTok{"cm"}\NormalTok{), }\AttributeTok{width =} \FunctionTok{unit}\NormalTok{(}\FloatTok{0.8}\NormalTok{, }\StringTok{"cm"}\NormalTok{),}
                         \AttributeTok{style =}\NormalTok{ north\_arrow\_fancy\_orienteering)}

\CommentTok{\#Impactos}
\NormalTok{g\_imp }\OtherTok{\textless{}{-}} \FunctionTok{ggplot}\NormalTok{(df\_imp, }\FunctionTok{aes}\NormalTok{(}\AttributeTok{x =}\NormalTok{ Tipo, }\AttributeTok{y =}\NormalTok{ Valor, }\AttributeTok{fill =}\NormalTok{ Tipo)) }\SpecialCharTok{+}
  \FunctionTok{geom\_col}\NormalTok{(}\AttributeTok{width =} \FloatTok{0.5}\NormalTok{, }\AttributeTok{color =} \StringTok{"black"}\NormalTok{, }\AttributeTok{alpha =} \FloatTok{0.9}\NormalTok{) }\SpecialCharTok{+}
  \FunctionTok{geom\_text}\NormalTok{(}\FunctionTok{aes}\NormalTok{(}\AttributeTok{label =} \FunctionTok{round}\NormalTok{(Valor, }\DecValTok{3}\NormalTok{)), }\AttributeTok{vjust =} \SpecialCharTok{{-}}\FloatTok{0.5}\NormalTok{, }\AttributeTok{fontface =} \StringTok{"bold"}\NormalTok{) }\SpecialCharTok{+}
  \FunctionTok{scale\_fill\_viridis\_d}\NormalTok{(}\AttributeTok{option =} \StringTok{"viridis"}\NormalTok{, }\AttributeTok{begin =} \FloatTok{0.3}\NormalTok{, }\AttributeTok{end =} \FloatTok{0.9}\NormalTok{) }\SpecialCharTok{+}
  \FunctionTok{labs}\NormalTok{(}\AttributeTok{title =} \StringTok{"A. Impactos na Latente (y*)"}\NormalTok{, }\AttributeTok{y =} \StringTok{"Mudança"}\NormalTok{, }\AttributeTok{x =} \ConstantTok{NULL}\NormalTok{) }\SpecialCharTok{+}
  \FunctionTok{theme\_minimal}\NormalTok{() }\SpecialCharTok{+} \FunctionTok{theme}\NormalTok{(}\AttributeTok{legend.position =} \StringTok{"none"}\NormalTok{)}

\CommentTok{\# Mapa de Resíduos}
\NormalTok{g\_resid }\OtherTok{\textless{}{-}} \FunctionTok{ggplot}\NormalTok{(sp\_dados) }\SpecialCharTok{+}
  \FunctionTok{geom\_sf}\NormalTok{(}\FunctionTok{aes}\NormalTok{(}\AttributeTok{fill =}\NormalTok{ resid\_generalized), }\AttributeTok{color =} \StringTok{"black"}\NormalTok{, }\AttributeTok{lwd =} \FloatTok{0.05}\NormalTok{) }\SpecialCharTok{+}
  \FunctionTok{scale\_fill\_gradient2}\NormalTok{(}\AttributeTok{low =} \StringTok{"\#440154"}\NormalTok{, }\AttributeTok{mid =} \StringTok{"white"}\NormalTok{, }\AttributeTok{high =} \StringTok{"\#FDE725"}\NormalTok{, }\AttributeTok{midpoint =} \DecValTok{0}\NormalTok{,}
                       \AttributeTok{limits =} \FunctionTok{c}\NormalTok{(}\SpecialCharTok{{-}}\NormalTok{max\_res, max\_res), }\AttributeTok{name =} \StringTok{"Resíduo"}\NormalTok{) }\SpecialCharTok{+}
  \FunctionTok{labs}\NormalTok{(}\AttributeTok{title =} \StringTok{"B. Resíduos Generalizados"}\NormalTok{, }\AttributeTok{subtitle =}\NormalTok{ label\_moran) }\SpecialCharTok{+}
  \FunctionTok{theme\_void}\NormalTok{() }\SpecialCharTok{+} 
  \FunctionTok{theme}\NormalTok{(}\AttributeTok{legend.position =} \StringTok{"bottom"}\NormalTok{, }\AttributeTok{legend.box.spacing =} \FunctionTok{unit}\NormalTok{(}\DecValTok{0}\NormalTok{, }\StringTok{"pt"}\NormalTok{)) }\SpecialCharTok{+}
  \FunctionTok{annotation\_scale}\NormalTok{(}\AttributeTok{location =} \StringTok{"br"}\NormalTok{, }\AttributeTok{width\_hint =} \FloatTok{0.3}\NormalTok{) }\SpecialCharTok{+}
  \FunctionTok{annotation\_north\_arrow}\NormalTok{(}\AttributeTok{location =} \StringTok{"tr"}\NormalTok{, }\AttributeTok{style =}\NormalTok{ north\_arrow\_fancy\_orienteering)}

\NormalTok{(g\_obs }\SpecialCharTok{|}\NormalTok{ g\_pred) }\SpecialCharTok{/}\NormalTok{ (g\_imp }\SpecialCharTok{|}\NormalTok{ g\_resid) }\SpecialCharTok{+} \FunctionTok{plot\_layout}\NormalTok{(}\AttributeTok{heights =} \FunctionTok{c}\NormalTok{(}\FloatTok{1.2}\NormalTok{, }\DecValTok{1}\NormalTok{))}
\end{Highlighting}
\end{Shaded}

\begin{figure}[H]

\centering{

\pandocbounded{\includegraphics[keepaspectratio]{lattice_data_files/figure-pdf/fig-sar_tobit-1.pdf}}

}

\caption{\label{fig-sar_tobit}Diagnóstico Tobit: (A) Observado
(Censurado), (B) Predito (Latente Esperado), (C) Impactos e (D) Resíduos
Generalizados.}

\end{figure}%

\section{Modelos locais e não
estacionários}\label{modelos-locais-e-nuxe3o-estacionuxe1rios}

Os modelos vistos até aqui (CAR, SAR, GNS e seus casos particulares)
fundamentam-se na premissa de estacionariedade espacial, assumindo que a
relação funcional entre a variável dependente e as variáveis
explicativas permanece invariante em todo o domínio geográfico. Por
exemplo, ao modelar o preço de imóveis no município de São Paulo, esses
modelos pressupõem que o impacto de uma variável (como a área
construída) é idêntico em todas as regiões. Contudo, em fenômenos
geográficos, essa premissa é frequentemente violada, caracterizando a
não estacionariedade ou heterogeneidade espacial (Brunsdon,
Fotheringham, e Charlton 1996). Na prática, bairros periféricos, como os
da Zona Leste (por exemplo, Guaianases, Jardim Ângela e São Mateus),
podem apresentar uma valorização marginal menor por metro quadrado,
enquanto bairros nobres, como Vila Madalena, Pinheiros, Vila Mariana,
Higienópolis, Jardim Paulista e Moema, podem apresentar uma valorização
superior devido a fatores de localização e infraestrutura.

Sachdeva, Fotheringham, e Li (2022) demonstram, utilizando modelagem
local, que é possível decompor o preço de um imóvel em componentes
estruturais e um valor intrínseco da localização (capturado pelo
intercepto local), permitindo quantificar quanto se paga apenas por
estar em um determinado lugar,
\href{https://pt.wikipedia.org/wiki/Ceteris_paribus}{ceteris paribus}. A
aplicação de modelos globais a processos espacialmente heterogêneos
tende a produzir estimativas de parâmetros que representam médias
espaciais enganosas, mascarando variações locais relevantes e induzindo
a erros de especificação (Binbin Lu et al. 2014; Fotheringham, Yang, e
Kang 2017).

Para abordar esta limitação, desenvolveram-se abordagens de modelagem
local, entre as quais a Regressão Geograficamente Ponderada (GWR) e a
Regressão Geograficamente Ponderada Multiescalar (MGWR).

\subsection{Regressão Geograficamente Ponderada
(GWR)}\label{regressuxe3o-geograficamente-ponderada-gwr}

A Regressão Geograficamente Ponderada (GWR) é uma técnica de análise
espacial local que estende o modelo de regressão linear clássico,
permitindo que os coeficientes variem continuamente no espaço. Em vez de
estimar um único conjunto de parâmetros globais \(\boldsymbol{\beta}\),
a GWR estima um conjunto distinto de parâmetros
\(\boldsymbol{\beta}(u_i, v_i)\) para cada localização \(i\) da amostra
(Binbin Lu et al. 2014).

O modelo GWR para uma observação na localização \(i\), com coordenadas
\((u_i, v_i)\), é expresso por:

\[y_i = \beta_0(u_i, v_i) + \sum_{k=1}^{p} \beta_k(u_i, v_i) x_{ik} + \epsilon_i\]

onde:

\begin{itemize}
\item
  \(y_i\) é o valor da variável dependente na localização \(i\);
\item
  \(x_{ik}\) é o valor da \(k\)-ésima variável independente na
  localização \(i\);
\item
  \(\beta_k(u_i, v_i)\) é o coeficiente de regressão local para a
  \(k\)-ésima variável independente na localização \(i\);
\item
  \(\epsilon_i\) é o termo de erro estocástico, tipicamente assumido
  como \(\epsilon_i \sim \mathcal{N}(0, \sigma^2)\).
\end{itemize}

A estimação dos parâmetros locais \(\hat{\boldsymbol{\beta}}(u_i, v_i)\)
é realizada através do método de Mínimos Quadrados Ponderados
(\href{https://en.wikipedia.org/wiki/Weighted_least_squares}{WLS}), onde
a ponderação é função da proximidade espacial. O estimador para a
localização \(i\) é:

\[\hat{\boldsymbol{\beta}}(u_i, v_i) = \left( \mathbf{X}^\top \mathbf{W}(i) \mathbf{X} \right)^{-1} \mathbf{X}^\top \mathbf{W}(i) \mathbf{y}.\]

A matriz \(\mathbf{W}(i)\) é uma matriz diagonal \(n \times n\) de pesos
espaciais específica para a localização de calibração \(i\):

\[
\mathbf{W}(i) = \text{diag}\left(w_{i1}, w_{i2}, \dots, w_{in}\right) =
\begin{bmatrix}
w_{i1} & 0 & \cdots & 0 \\
0 & w_{i2} & \cdots & 0 \\
\vdots & \vdots & \ddots & \vdots \\
0 & 0 & \cdots & w_{in}
\end{bmatrix},
\]

Cada elemento diagonal \(w_{ij}\) representa o peso atribuído à
observação \(j\) quando o modelo é calibrado para a localização \(i\).
Estes pesos são determinados por uma função kernel. Embora o kernel
Gaussiano seja comum, Gollini et al. (2015) destacam a importância de
experimentar diferentes funções, como o kernel Bi-quadrado ou Box-car. O
kernel Bi-quadrado, por exemplo, oferece eficiência computacional e um
corte claro de influência, sendo definido como:

\[
w_{ij} = \begin{cases} 
\left[1 - (d_{ij}/b)^2\right]^2 & \text{se } d_{ij} < b \\
0 & \text{caso contrário}
\end{cases}
\] onde \(b > 0\) é o parâmetro de largura de banda (ver outras em
(Gollini et al. 2015)), que controla o decaimento espacial da
influência. Para \(i = j\), \(d_{ii} = 0\), resultando no peso máximo
\(w_{ii} = 1\). Conforme \(d_{ij}\) aumenta, \(w_{ij}\) tende
assintoticamente a zero, implementando a primeira lei da Geografia de
Tobler.

A seleção da largura de banda ótima \(b\) é fundamental, representando
um compromisso entre viés e variância. Valores pequenos de \(b\)
produzem estimativas locais de alta variância (\emph{overfitting}),
enquanto valores grandes introduzem viés, aproximando o modelo de uma
regressão global.

A seleção da largura de banda ótima \(b\) pode ser definida de duas
formas principais: Fixa (uma distância constante para todas as unidades)
ou Adaptativa (um número fixo de \(N\) vizinhos mais próximos). Guo, Ma,
e Zhang (2008) demonstram empiricamente que, em dados com agrupamento
espacial, kernels adaptativos tendem a capturar melhor a heterogeneidade
local do que kernels fixos, que podem suavizar excessivamente os padrões
em áreas densas e sofrer com escassez de dados em áreas dispersas.

Mais do que um parâmetro técnico de ajuste, Fotheringham et al. (2022)
argumentam que a largura de banda deve ser interpretada como um
indicador da escala do processo espacial. Segundo os autores, o valor
ótimo de \(b\) é determinado por três características do processo:

\begin{enumerate}
\def\labelenumi{\arabic{enumi}.}
\item
  Variabilidade do parâmetro: Processos altamente heterogêneos exigem
  larguras de banda pequenas.
\item
  Dependência espacial: Processos com forte autocorrelação espacial
  tendem a resultar em larguras de banda menores.
\item
  Força do processo (ruído): Relações fracas ou com alto nível de erro
  (\(\sigma^2\)) tendem a resultar em larguras de banda maiores, pois o
  modelo precisa ``emprestar'' mais dados para reduzir a incerteza da
  estimativa.
\end{enumerate}

A otimização de \(b\) geralmente busca minimizar critérios como a
Validação Cruzada (CV) ou o Critério de Informação de Akaike corrigido
(AICc) (Hurvich, Simonoff, e Tsai 1998; Binbin Lu et al. 2014):

\[
\text{AICc} = 2n \ln(\hat{\sigma}) + n \ln(2\pi) + n \frac{n + \text{tr}(\mathbf{S})}{n - 2 - \text{tr}(\mathbf{S})}.
\]

Nesta expressão, \(\text{tr}(\mathbf{S})\) é o traço da matriz chapéu
(hat matrix) \(\mathbf{S}\), que mapeia os valores observados
\(\mathbf{y}\) para os preditos \(\hat{\mathbf{y}}\), representando o
número efetivo de parâmetros.

Na GWR a matriz \(\mathbf{S}\) é construída linha por linha, pois cada
observação \(i\) possui sua própria matriz de pesos \(\mathbf{W}(i)\).

Conforme derivado por Hanchen Yu et al. (2020), o valor predito
\(\hat{y}_i\) é obtido multiplicando-se o vetor de covariáveis da
observação \(i\), denotado por \(\mathbf{x}_i\) (um vetor linha), pelos
parâmetros estimados localmente:

\[\hat{y}_i = \mathbf{x}_i \hat{\boldsymbol{\beta}}(u_i, v_i) = \mathbf{x}_i \left( \mathbf{X}^\top \mathbf{W}(i) \mathbf{X} \right)^{-1} \mathbf{X}^\top \mathbf{W}(i) \mathbf{y}.\]

Definindo o vetor linha
\(\mathbf{r}_i = \mathbf{x}_i (\mathbf{X}^\top \mathbf{W}(i) \mathbf{X})^{-1} \mathbf{X}^\top \mathbf{W}(i)\),
a matriz completa \(\mathbf{S}\) de dimensão \(n \times n\) é formada
pelo empilhamento destes vetores:

\[\mathbf{S} = \begin{bmatrix} \mathbf{r}_1 \\ \mathbf{r}_2 \\ \vdots \\ \mathbf{r}_n \end{bmatrix}.\]

Assim, temos a relação matricial
\(\hat{\mathbf{y}} = \mathbf{S}\mathbf{y}\). O traço desta matriz,
\(\text{tr}(\mathbf{S})\), utilizado no denominador do AICc, representa
a soma das influências de cada observação sobre o seu próprio valor
predito, fornecendo uma medida da complexidade do modelo equivalente aos
graus de liberdade em modelos lineares generalizados (Binbin Lu et al.
2014).

Embora a minimização do AICc ou a Validação Cruzada (CV) sejam
procedimentos padrão, a literatura recente aponta limitações nestes
métodos automáticos. Guo, Ma, e Zhang (2008) alertam que o critério AICc
tende a ser conservador, selecionando frequentemente larguras de banda
maiores que suavizam excessivamente os padrões espaciais, obscurecendo
heterogeneidades locais biologicamente ou socialmente relevantes em
favor de um ajuste global mais estável.

Adicionalmente, da Silva e Mendes (2018) demonstram que a função
objetivo de Validação Cruzada em modelos GWR com kernels adaptativos
frequentemente não é estritamente convexa, apresentando múltiplos
mínimos locais. O uso de algoritmos de otimização padrão, como a Busca
de Seção Áurea
(\href{https://en.wikipedia.org/wiki/Golden-section_search}{Golden
Section Search}), pode convergir para soluções sub-ótimas, sugerindo a
necessidade de algoritmos mais robustos como o \emph{Lightning Search
Algorithm} (ver Shareef, Ibrahim, e Mutlag (2015)) ou abordagens
híbridas de divisão de intervalo para garantir a identificação do mínimo
global.

Alternativamente, Koç (2022) propõe a substituição dos critérios
clássicos por critérios de Complexidade de Informação (ICOMP). Ao
penalizar não apenas o número de parâmetros, mas também a
interdependência (estrutura de covariância) entre as estimativas dos
parâmetros, o ICOMP demonstrou, em estudos de simulação e aplicações
reais, selecionar larguras de banda que produzem modelos com maior
precisão preditiva e melhor equilíbrio entre viés e variância do que o
AICc ou CV.

A inferência estatística na GWR, como o cálculo de intervalos de
confiança e testes de hipóteses para os parâmetros locais, requer a
estimação da variância dos coeficientes locais. Reformulando a GWR como
um modelo aditivo generalizado (GAM), Hanchen Yu et al. (2020) fornecem
a expressão para a matriz de covariância dos estimadores:

\[\text{Var}(\hat{\boldsymbol{\beta}}_j) = \text{diag}(\mathbf{C}\mathbf{C}^\top \hat{\sigma}^2)\]

onde \(\mathbf{C} = [\text{diag}(\mathbf{X}_j)]^{-1}\mathbf{R}_j\) e
\(\mathbf{R}_j\) é a matriz de projeção específica da covariável. Isso
permite o cálculo de erros-padrão locais precisos e o ajuste dos valores
críticos da distribuição \(t\) para evitar falsos positivos, garantindo
que a heterogeneidade espacial detectada seja estatisticamente
significante e não apenas ruído aleatório.

Na reformulação do modelo GWR em modelo GAM, consistiu em não olhar o
GWR como uma soma de produtos entre coeficientes variáveis e
covariáveis. O vetor de resposta \(\mathbf{y}\) é modelado como uma soma
de funções suaves
(\href{https://en.wikipedia.org/wiki/Smoothness}{smooth functions}) mais
um termo de erro:

\[\mathbf{y} = \sum_{j=0}^{p} \mathbf{f}_j + \boldsymbol{\epsilon}, \quad \boldsymbol{\epsilon} \sim \mathcal{N}(\mathbf{0}, \sigma^2 \mathbf{I}_n)\]

No contexto da GWR, cada termo aditivo \(\mathbf{f}_j\) representa o
componente espacial da \(j\)-ésima variável explicativa. Este termo é
definido como o produto elemento a elemento entre a covariável e seu
coeficiente espacialmente variável:

\[\mathbf{f}_j = \text{diag}(\mathbf{X}_j) \boldsymbol{\beta}_j\]

onde \(\mathbf{X}_j\) é o vetor \(n \times 1\) contendo as observações
da \(j\)-ésima variável independente, \(\text{diag}(\mathbf{X}_j)\) é
uma matriz diagonal com esses valores, e \(\boldsymbol{\beta}_j\) é o
vetor \(n \times 1\) dos parâmetros locais para a variável \(j\) em
todas as localizações.

A calibração do modelo gera uma matriz de projeção específica
\(\mathbf{R}_j\) para cada covariável, tal que o valor ajustado para o
componente \(j\) é:

\[\hat{\mathbf{f}}_j = \mathbf{R}_j \mathbf{y}\]

Combinando as definições, temos que
\(\text{diag}(\mathbf{X}_j) \hat{\boldsymbol{\beta}}_j = \mathbf{R}_j \mathbf{y}\).
Isolando o vetor de parâmetros estimados \(\hat{\boldsymbol{\beta}}_j\),
obtemos uma expressão linear em relação a \(\mathbf{y}\):

\[\hat{\boldsymbol{\beta}}_j = [\text{diag}(\mathbf{X}_j)]^{-1} \mathbf{R}_j \mathbf{y} = \mathbf{C}_j \mathbf{y}\]

onde definimos a matriz linear transformadora como
\(\mathbf{C}_j = [\text{diag}(\mathbf{X}_j)]^{-1} \mathbf{R}_j\).

Com o estimador \(\hat{\boldsymbol{\beta}}_j\) expresso como uma
combinação linear da variável resposta (\(\mathbf{C}_j \mathbf{y}\)), a
derivação de sua variância torna-se direta, aplicando as propriedades de
variância de operadores lineares
(\(\text{Var}(\mathbf{A}\mathbf{y}) = \mathbf{A}\text{Var}(\mathbf{y})\mathbf{A}^\top\)):

\[\text{Var}(\hat{\boldsymbol{\beta}}_j) = \text{Var}(\mathbf{C}_j \mathbf{y}) = \mathbf{C}_j \text{Var}(\boldsymbol{\epsilon}) \mathbf{C}_j^\top = \mathbf{C}_j (\sigma^2 \mathbf{I}) \mathbf{C}_j^\top\]

Portanto, a matriz de covariância para as estimativas dos parâmetros
locais da \(j\)-ésima variável é dada por (Hanchen Yu et al. 2020):

\[\text{Var}(\hat{\boldsymbol{\beta}}_j) = \sigma^2 \mathbf{C}_j \mathbf{C}_j^\top\]

Os erros-padrão locais para a localização \(i\) e variável \(j\) são
obtidos simplesmente pela raiz quadrada dos elementos diagonais desta
matriz:

\[SE(\hat{\beta}_{ij}) = \sqrt{\left( \text{Var}(\hat{\boldsymbol{\beta}}_j) \right)_{ii}}\]

Este formalismo permite a construção de estatísticas \(t\) locais
(\(t_{ij} = \hat{\beta}_{ij} / SE(\hat{\beta}_{ij})\)). Contudo, a
realização de testes individuais para cada localização incorre no
problema de comparações múltiplas. Para mitigar o aumento da taxa de
erro tipo I sem a severidade excessiva da correção de Bonferroni, da
Silva e Fotheringham (2016) propõem o ajuste do nível de significância
\(\alpha\) baseando-se no número efetivo de parâmetros
(\(\text{ENP}_j = \text{tr}(\mathbf{R}_j)\)) específico de cada
covariável, garantindo que a heterogeneidade espacial detectada seja
estatisticamente significativa.

Uma limitação inerente do modelo GWR é a suposição de uma única escala
espacial para todos os processos, uma vez que utiliza uma largura de
banda \(b\) comum a todas as variáveis independentes. Como notado por
(Wenbai Yang, Fotheringham, e Harris 2011; Wenbo Yang 2014) e
formalizado por Fotheringham, Yang, e Kang (2017), diferentes processos
(ex: renda, clima, topografia) operam em escalas distintas. Esta
restrição motivou o desenvolvimento da Regressão Geograficamente
Ponderada Multiescalar (MGWR).

\subsection{Regressão Geograficamente Ponderada Multiescalar
(MGWR)}\label{regressuxe3o-geograficamente-ponderada-multiescalar-mgwr}

A aplicação da GWR impõe uma restrição: a suposição de que todos os
processos modelados operam na mesma escala espacial. Isto decorre da
estimação de uma única largura de banda ótima \(b\) para todas as
covariáveis simultaneamente. Contudo, em sistemas geográficos complexos,
é intuitivo e frequentemente observado que diferentes preditores
influenciem a variável resposta em escalas distintas (Fotheringham,
Yang, e Kang 2017; Wenbo Yang 2014).

Por exemplo, em um modelo de preços imobiliários, a influência da
proximidade a uma centralidade metropolitana pode variar suavemente em
uma escala regional (processo de larga escala), enquanto o efeito da
qualidade da infraestrutura local (como pavimentação) pode mudar
abruptamente entre quarteirões adjacentes (processo de escala local). A
imposição de uma largura de banda única na GWR introduz viés nas
estimativas de processos localmente heterogêneos (ao suavizá-los
excessivamente) e aumenta a variância das estimativas de processos
globalmente estáveis (ao torná-las desnecessariamente ruidosas) (Binbin
Lu et al. 2014).

Adicionalmente, estudos demonstram que o uso de uma largura de banda
única pode exacerbar problemas de multicolinearidade local e
concurvidade (colinearidade funcional em modelos não paramétricos),
levando a coeficientes instáveis e de difícil interpretação (Wheeler e
Tiefelsdorf 2005). Oshan, Smith, e Fotheringham (2020) evidenciam, em um
estudo sobre determinantes espaciais da obesidade, que a GWR consome
excessivos graus de liberdade ao forçar a modelagem local de processos
que são, na verdade, globais. Isto resulta em sobreajuste e perda de
parcimônia. A abordagem multiescalar, ao permitir que processos globais
sejam modelados como tal (com larguras de banda tendendo ao infinito),
produz modelos mais parcimoniosos, com menor AICc e diagnósticos de
colinearidade mais robustos.

A MGWR relaxa a suposição de homogeneidade de escala da GWR, permitindo
que cada covariável possua sua própria largura de banda ótima \(bw_j\).
Isto transforma a largura de banda de um parâmetro de suavização em um
indicador empírico das propriedades espaciais intrínsecas de cada
relação causal investigada (Fotheringham, Yang, e Kang 2017).

O modelo MGWR para uma observação na localização \(i\) é formalmente
especificado como:

\[
y_i = \beta_0(u_i, v_i; bw_0) + \sum_{j=1}^{p} \beta_j(u_i, v_i; bw_j) \, x_{ij} + \epsilon_i, \quad \epsilon_i \sim \mathcal{N}(0, \sigma^2),
\]

onde a notação \(\beta_j(\cdot; bw_j)\) enfatiza que a superfície do
coeficiente associado à \(j\)-ésima covariável é estimada utilizando uma
função kernel com largura de banda específica \(bw_j\). O intercepto
\(\beta_0\) também possui sua própria escala, \(bw_0\).

Diferentemente da GWR, que possui uma solução em forma fechada via
Mínimos Quadrados Ponderados para cada localização, a MGWR não tem uma
solução analítica direta devido à dependência mútua das superfícies de
coeficientes estimadas com diferentes larguras de banda. A estratégia
padrão de estimação reformula o problema dentro da estrutura dos Modelos
Aditivos Generalizados (GAMs) (Hanchen Yu et al. 2020).

O modelo é reescrito como uma soma de funções suaves:

\[
\mathbf{y} = \sum_{j=0}^{p} \mathbf{f}_j + \boldsymbol{\epsilon},
\]

onde cada termo aditivo \(\mathbf{f}_j\) é um vetor \(n \times 1\) que
representa a contribuição espacialmente suave da \(j\)-ésima covariável.
No contexto da MGWR,
\(\mathbf{f}_j = \text{diag}(\mathbf{x}_j) \, \boldsymbol{\beta}_j\),
com \(\boldsymbol{\beta}_j\) sendo o vetor de coeficientes locais para a
covariável \(j\).

A estimação é realizada através de um \textbf{algoritmo iterativo de
\emph{back-fitting}}, inspirado na estimação de GAMs (Hastie e
Tibshirani 1990). O procedimento, detalhado por Fotheringham, Yang, e
Kang (2017) e Hanchen Yu et al. (2020), segue os seguintes passos:

\begin{enumerate}
\def\labelenumi{\arabic{enumi}.}
\item
  Inicialização: As funções \(\hat{\mathbf{f}}_j^{(0)}\) são
  inicializadas, por exemplo, com as estimativas de um modelo de
  regressão linear ou de uma GWR com banda única.
\item
  Iteração (\emph{Back-fitting}): Para cada covariável
  \(j = 0, 1, \dots, p\) na iteração \([k+1]\):

  \begin{enumerate}
  \def\labelenumii{\alph{enumii}.}
  \item
    Calcula-se o resíduo parcial removendo a contribuição atual de todas
    as outras covariáveis: \[
    \mathbf{r}_j^{[k]} = \mathbf{y} - \sum_{l < j} \hat{\mathbf{f}}_l^{[k+1]} - \sum_{l > j} \hat{\mathbf{f}}_l^{[k]}.
    \]
  \item
    Ajusta-se um modelo GWR univariado do resíduo parcial
    \(\mathbf{r}_j^{[k]}\) contra a covariável \(\mathbf{x}_j\). Nesta
    etapa, otimiza-se a largura de banda específica \(bw_j\) que
    minimiza um critério de informação (usualmente o AICc) para este
    submodelo.
  \item
    Atualiza-se a estimativa da função:
    \(\hat{\mathbf{f}}_j^{[k+1]} = \text{diag}(\mathbf{x}_j) \, \hat{\boldsymbol{\beta}}_j^{[k+1]}\),
    onde \(\hat{\boldsymbol{\beta}}_j^{[k+1]}\) é o vetor de
    coeficientes obtido do modelo GWR univariado calibrado com
    \(bw_j^{[k+1]}\).
  \end{enumerate}
\item
  Convergência: O algoritmo itera até que a mudança nas estimativas dos
  coeficientes ou na soma dos quadrados dos resíduos entre iterações
  sucessivas seja inferior a um limiar pré-definido (ex: \(10^{-5}\)).
  Wenbo Yang (2014) introduz o conceito de \emph{Score of Change (SOC)}
  para monitorar esta convergência.
\end{enumerate}

A estabilidade deste processo depende do algoritmo de otimização
utilizado para encontrar cada \(bw_j\). Conforme destacado por da Silva
e Mendes (2018), a superfície do critério de seleção (AICc ou CV) para
larguras de banda adaptativos é frequentemente não convexa e multimodal.
Portanto, métodos robustos como a busca exaustiva em grade
(\href{https://en.wikipedia.org/wiki/Hyperparameter_optimization}{grid
search}) ou algoritmos heurísticos são preferíveis ao método padrão de
Busca da Seção Áurea para evitar mínimos locais subóptimos.

A principal inovação da MGWR é a capacidade de interpretar as larguras
de banda \(bw_j\) como métricas da escala espacial de cada processo.
Baseando-se na teoria desenvolvida por Fotheringham et al. (2022),
pode-se inferir que:

\begin{itemize}
\item
  \(bw_j\) pequeno: Indica um processo localmente heterogêneo, onde a
  relação entre \(x_j\) e \(y\) muda rapidamente no espaço. Isto pode
  ser causado por alta variabilidade do parâmetro verdadeiro ou por
  forte dependência espacial de curto alcance.
\item
  \(bw_j\) grande (próximo de \(n\)): Indica um processo regional ou
  globalmente estável, onde a relação é aproximadamente constante no
  espaço. Quando \(bw_j \to \infty\), o coeficiente \(\beta_j\) converge
  para uma constante, recuperando a estacionariedade do modelo OLS.
\end{itemize}

A análise dessas escalas não deve ser estática. Bo Lu et al. (2023)
demonstram, em um estudo longitudinal do mercado imobiliário, que as
larguras de banda ótimas podem evoluir temporalmente, refletindo
mudanças na estrutura subjacente do fenômeno (ex.: a influência de áreas
verdes tornar-se mais localizada após a implementação de políticas
urbanas específicas).

Para realizar inferência estatística na MGWR, é necessário estimar a
variância das superfícies de coeficientes. Hanchen Yu et al. (2020)
estendem o arcabouço de inferência da GWR para o caso multiescalar.

O modelo MGWR completo pode ser representado por uma única matriz de
projeção (ou \emph{hat matrix}) \(\mathbf{S}_{MGWR}\), tal que
\(\hat{\mathbf{y}} = \mathbf{S}_{MGWR} \, \mathbf{y}\). Esta matriz é
complexa de derivar analiticamente, mas pode ser entendida como o
resultado da composição das operações de \emph{back-fitting}. A
contribuição de cada covariável é associada a uma matriz de suavização
\(\mathbf{S}_j(bw_j)\).

A variância do estimador para o vetor de coeficientes da covariável
\(j\), \(\hat{\boldsymbol{\beta}}_j\), é dada por:

\[
\text{Var}(\hat{\boldsymbol{\beta}}_j) = \sigma^2 \, \mathbf{C}_j \mathbf{C}_j^\top, \quad \text{onde} \quad \mathbf{C}_j = [\text{diag}(\mathbf{x}_j)]^{-1} \mathbf{S}_j(bw_j).
\]

Os erros-padrão locais para \(\hat{\beta}_j(u_i, v_i)\) são as raízes
quadradas dos elementos diagonais correspondentes de
\(\text{Var}(\hat{\boldsymbol{\beta}}_j)\). Estatísticas \(t\) locais
podem então ser construídas para testar hipóteses pontuais (ex.:
\(\beta_j(u_i, v_i) = 0\)).

Contudo, a realização de testes simultâneos em centenas ou milhares de
localizações incorre no problema de comparações múltiplas. da Silva e
Fotheringham (2016) propõem uma correção baseada no Número Efetivo de
Parâmetros (ENP) específico de cada superfície. O nível de significância
\(\alpha\) é ajustado utilizando
\(ENP_j = \text{tr}(\mathbf{S}_j(bw_j))\) como uma medida dos graus de
liberdade consumidos pela covariável \(j\), oferecendo um controle mais
adequado da taxa de erro tipo I do que a correção de Bonferroni
excessivamente conservadora.

A lógica multiescalar foi estendida para incorporar explicitamente a
dimensão temporal, resultando no modelo MGWR Espaço-Temporal (MGTWR) (Wu
et al. 2019; Huili Yu et al. 2020). Nesta formulação, cada covariável
possui uma largura de banda espacial (\(h_S\)) e uma largura de banda
temporal (\(h_T\)) específicas, definidas por um kernel espaço-temporal
(ex.: produto de kernels separáveis).

O modelo MGTWR permite discriminar processos que são (1) espacialmente
locais mas temporalmente estáveis (ex.: efeito de uma escola no preço da
habitação); (2) espacialmente globais mas temporalmente voláteis (ex.:
impacto de uma política monetária nacional).

A calibração envolve a otimização conjunta de \(h_S\) e \(h_T\) para
cada termo, aumentando a complexidade computacional, mas oferecendo uma
representação mais rica da dinâmica espaço-temporal.

A flexibilidade da MGWR tem facilitado sua integração com fontes de
dados não tradicionais e de alta dimensão. Exemplos incluem:

\begin{itemize}
\item
  Imagens de \emph{Street View} e Visão Computacional: He et al. (2022)
  utilizaram a MGWR para modelar a relação entre características visuais
  extraídas do \emph{Google Street View} (vegetação, manutenção de
  edifícios) e taxas de criminalidade em Nova York, revelando que a
  influência de certas características varia espacialmente em escalas
  diferentes.
\item
  Dados de mobilidade e redes sociais: Liu, Chau, e Bao (2023) aplicaram
  a MGWR para analisar como diferentes fatores socioeconômicos, em
  diferentes escalas espaciais, explicam padrões de uso do transporte
  público derivados de dados de cartões inteligentes.
\end{itemize}

Nestes contextos, a MGWR frequentemente supera modelos tradicionais
(OLS, SLM, SEM, GWR padrão) em medidas de ajuste (\(R^2\) ajustado,
AICc) e capacidade explicativa, por sua aptidão em capturar a
multiescalaridade inerente a processos urbanos e sociais complexos.

\part{Processos Pontuais}

\chapter{Processos Pontuais}\label{sec-proc_pont}

Constituem conjunto de dados que fornecem as localizações espaciais de
fenômenos ou eventos observados (Peter J. Diggle 2013; Adrian Baddeley,
Rubak, e Turner 2015; Scalon 2024). São modelos matemáticos que
descrevem a disposição de eventos que estão irregular ou aleatoriamente
distribuídos no plano, ou no espaço (J. Illian et al. 2008). São
conjuntos contáveis de eventos espaciais que surgem como realizações de
processos estocásticos (aleatórios ou probabilísticos) de eventos
espaciais tomando valores em uma região plana
\(B\subset \mathbb{R}^{2}\) (Moller e Waagepetersen 2003; Mateus 2013;
Leininger 2014; Moraga 2023).

Seja \(S \subseteq \mathbb{R}^{d}\) um espaço métrico, onde
\(\mathbb{R}^{d}\) denota um espaço euclidiano \(d\)-dimensional, onde
os eventos ocorrem. Um processo pontual espacial em
\(B\subseteq \mathbb{R}^{d}\), é uma coleção de variáveis aleatórias
\{\(n(S): S \subseteq B\)\}, onde \(n(S)\) representa o número de
eventos ocorridos em um conjunto mensurável \(s\) de \(B\).

Processos pontuais referem-se a uma área da estatística espacial que
estuda a distribuição de eventos em uma área geográfica específica
delimitada, onde as coordenadas geográficas são a própria informação.
Isto é, os eventos são geralmente representados por pontos e descritos
utilizando coordenadas geográficas. São exemplos de processos pontuais,
coordenadas geográficas da localização de: árvores, estabelecimentos
comerciais, acidentes rodoviários, epicentros de terremotos,
crimes/raptos, ninhos de pássaros, Universidades, etc.

Dependendo do número de tipos diferentes de eventos considerados, um
processo pontual pode ser classificado como univariado, quando envolve
um único tipo de evento ou mais de um tipo de evento sendo estudado cada
um separadamente, ou seja, sem investigar a relação entre diferentes
eventos; e multivariado, quando envolve mais de um tipo de evento e o
objetivo é investigar a relação entre diferentes eventos (dependência
interespecífica). Esta pesquisa se concentra exclusivamente em processos
pontuais univariados.

\section{Elementos de processos
pontuais}\label{elementos-de-processos-pontuais}

Nesta seção, apresentamos os principais elementos que compõem os
processos pontuais, incluindo as marcas e as covariáveis.

\textbf{Marcas}

Um processo pontual espacial pode envolver não apenas a localização dos
eventos no espaço, mas também informações qualitativas ou quantitativas
adicionais associadas a cada evento, ou localização \(s_{i}=(x, y)\), em
que \(x\) e \(y\) são as respectivas coordenadas. Essas informações
adicionais são denominadas marcas, e são representadas por \(m(s_{i})\)
(J. B. Illian 2019; Adrian Baddeley, Rubak, e Turner 2015; Scalon 2024).
Assim, considerando \(B \subseteq \mathbb{R}^{d}\) e
\(C \subseteq \mathbb{R}\), em que \(B\) representa a região em que
ocorrem os eventos ou fenômenos \(s_{i}=(x, y)\) de interesse e \(C\) as
informações adicionais para cada evento ou fenômeno \(s_{i}=(x, y)\), um
processo pontual marcado \(M(B \times C)\), pode ser representado:

\begin{equation}\phantomsection\label{eq-1}{
M(B \times C) = \{[s_{i}, m(s_{i})]\}= \{[s_{1}, m(s_{1})],[s_{2}, m(s_{2})], \ldots, [s_{n}, m(s_{n})]\},\,s_{i} \in B,\,m(s_{i}) \in C.
}\end{equation}

Em um processo pontual onde os eventos são localizações de árvores em
uma floresta, exemplos de marcas incluem a altura das árvores, tipo de
espécie florestal, diâmetro à altura do peito (DAP), entre outros. Em um
processo pontual com eventos sendo acidentes de trânsito em uma cidade,
as marcas podem representar a gravidade do acidente (ferimentos leves,
graves ou fatalidades). Em um processo pontual onde os eventos são
estabelecimentos comerciais em uma cidade, as marcas podem representar o
tipo de estabelecimento (restaurantes, lojas de roupas, supermercados).

\textbf{Covariáveis}

Além das marcas, um processo pontual pode incluir variáveis
explicativas, também conhecidas como covariáveis. Essas covariáveis são
variáveis adicionais que auxiliam na compreensão dos motivos pelos quais
os eventos ocorrem em determinados locais ou por que ocorrem de maneira
desigual.

Adrian Baddeley, Rubak, e Turner (2015) definem covariáveis como
quaisquer dados tratados como explicativos e não como resposta, tomando
como exemplo, uma função espacial \(z(s_{i})\) que descreve a altitude
no evento \(s_{i}\), em uma floresta. Em um processo pontual onde os
eventos são crimes em uma cidade, as covariáveis podem incluir densidade
populacional, níveis de desemprego e acesso a serviços policiais.

\section{Propriedades fundamentais de processos
pontuais}\label{sec-2.3.2}

\textbf{Estacionaridade e Isotropia}

Um processo pontual é estacionário se a distribuição espacial dos
eventos é invariante sob condição de translação (J. Illian et al. 2008;
Adrian Baddeley, Rubak, e Turner 2015). Seja
\(N=\{s_{1}, s_{2}, \ldots , s_{n}\}\) um processo pontual e
\(N_{t} = \{s_{1} + t, s_{2}+t, \ldots , s_{n} +t\}\) a sua translação,
este será estacionário se para qualquer \(t\in \mathbb{R}^{d}\), \(N\) e
\(N_{t}\) apresentarem a mesma distribuição espacial dos eventos, ou
seja,

\begin{equation}\phantomsection\label{eq-2}{
\forall t \in \mathbb{R}^{d}, N \stackrel{D}{=} N_{t} \Leftrightarrow \{s_{1}, s_{2}, \ldots , s_{n}\} \stackrel{D}{=} \{s_{1} + t, s_{2}+t, \ldots, s_{n} +t\},
}\end{equation}

em que \(t\) indica as unidades transladadas. No caso de um processo
pontual marcado, onde \(m(s_{i})\) é a marca associada ao evento, a
expressão Eq.~\ref{eq-2} fica representada por:

\begin{equation}\phantomsection\label{eq-3}{
\forall t \in \mathbb{R}^{d}, N_{[s_{i}, m(s_{i})]} \stackrel{D}{=} N_{\{[s_{i}+t, m(s_{i})]\}} 
\Leftrightarrow \{[s_{1}, m(s_{i})], \ldots , [s_{n},m(s_{n})]\} \stackrel{D}{=} \{[s_{1}+t, m(s_{i})], \ldots , [s_{n}+t,m(s_{n})]\} .
}\end{equation}

Um processo pontual é isotrópico se a distribuição espacial dos eventos
é invariante sob rotação em torno da origem (Moller e Waagepetersen
2003; J. Illian et al. 2008; Wiegand e Moloney 2014; Scalon 2024).

Seja \(N=\{s_{1}, s_{2}, \ldots , s_{n}\}\) um processo pontual e
\(R_{\alpha} N = \{R_{\alpha}s_{1}, R_{\alpha}s_{2}, \ldots , R_{\alpha}s_{n}\}\)
a sua rotação, onde \(\alpha \in [0^{o}, 360^{o}]\). Este processo será
isotrópico se a distribuição espacial dos eventos \(N\) e
\(R_{\alpha} N\) for a mesma, ou seja,

\begin{equation}\phantomsection\label{eq-4}{
\forall \alpha \in [0^{o}, 360^{o}], N \stackrel{D}{=} R_{\alpha} N \Leftrightarrow \{s_{1}, s_{2}, \ldots , s_{n}\} \stackrel{D}{=} \{R_{\alpha}s_{1}, R_{\alpha}s_{2}, \ldots , R_{\alpha}s_{n}\}.
}\end{equation}

No caso do processo pontual ser marcado, onde \(m(s_{i})\) é a marca
associada a cada evento, a expressão Eq.~\ref{eq-4} fica representada
por:

\begin{equation}\phantomsection\label{eq-5}{
\forall \alpha \in [0^{o}, 360^{o}], N_{[s_{i}, m(s_{i})]} \stackrel{D}{=} R_{\alpha}N_{\{[s_{i}, m(s_{i})]\}}
\Leftrightarrow \{[s_{1}, m(s_{1})], \ldots , [s_{n},m(s_{n})]\} \stackrel{D}{=} \{R_{\alpha}[s_{1}, m(s_{1})], \ldots , R_{\alpha}[s_{n},m(s_{n})]\}.
}\end{equation}

Um processo pontual pode ser isotrópico sem ser estacionário, mas o
contrário não é verdadeiro (Adrian Baddeley, Rubak, e Turner 2015).

\textbf{Homogeneidade, Independência e Tendência Espacial}

Um processo pontual é considerado homogêneo quando os eventos ocorrem em
proporções iguais na área de estudo. Isso significa que a probabilidade
de encontrar um evento em qualquer sub-região da área de estudo onde
ocorrem os eventos, é constante e igual. Em contraste, se essa
probabilidade não for constante e variar, indicando que certas áreas são
mais propensas a ter um maior ou menor número de eventos, ou se os
eventos ocorrerem apenas em algumas sub-regiões específicas da área de
estudo, diz-se que o processo pontual apresenta tendência espacial.

Pela teoria de probabilidade sabe-se que dois eventos \(s\) e \(Z\) são
independentes se e somente se \(P(S\cap Z) = P(S)P(Z)\), ou seja, eles
serão independentes se e somente se sua probabilidade conjunta
(ocorrência simultânea) for igual ao produto de suas probabilidades
marginais (ocorrências individuais), como evidenciado por (Ferreira
2020).

Da mesma forma, pode-se aplicar essa lógica ao conceito de independência
de dois ou mais eventos em processos pontuais. Em um processo pontual,
dois ou mais eventos são considerados independentes se a ocorrência de
um evento em um determinado local ou sub-região da área de estudo não
influencia a ocorrência de outros eventos.

\section{Principais distribuições de probabilidade em processos
pontuais}\label{principais-distribuiuxe7uxf5es-de-probabilidade-em-processos-pontuais}

Nesta seção, apresentamos as principais distribuições de probabilidade,
incluindo distribuição uniforme, binomial e Poisson.

\textbf{Distribuição Uniforme}

Seja \(B\) a região de estudo e \(|B|\) a sua área ou volume
correspondente. Diz-se que os eventos de um processo pontual qualquer
estão uniformemente distribuídos em \(B\), se a probabilidade de
ocorrência for constante na região \(B\) e zero fora dela, ou seja,

\begin{equation}\phantomsection\label{eq-6}{
f(s_{i}) = \left\{\begin{array}{rcl}
\frac{1}{|B|} & \text {se} & (x,y) \in B \\
0 & \text{ se } & (x,y) \notin B .
\end{array}\right.
}\end{equation}

Considerando \(S \subseteq B\), como uma sub-região de \(B\), onde
\(|S|\) e \(|B|\) são as áreas ou volumes correspondentes, a
probabilidade dos eventos \(s_{i} \in s\) ocorrerem na sub-região \(s\)
é,

\[
P(s \in S) = \int f(x,y) dx dy = \frac{1}{|B|} \int_{S} 1 dx dy = \frac{|S|}{|B|} ,
\] em que \(s_{i}=(x, y)\).

\textbf{Distribuição Binomial}

Seja \(S \subseteq B\) e \(n(S \cap B) \in B\) representando os eventos
no processo pontual \(s\) na região de estudo \(B\), onde \(|S|\) e
\(|B|\) são as áreas ou volumes correspondentes. Diz-se que os eventos
aleatórios \(n(S \cap B)\) do processo pontual \(s\) seguem distribuição
binomial, se resultam de ensaios independentes e idênticos, com apenas
dois resultados possíveis, sucesso (\(p\)) ou fracasso (\(1-p\)),
constantes em cada ensaio, ou seja,

\begin{equation}\phantomsection\label{eq-8}{
P(N(S)=\kappa) = \binom{n}{\kappa} p^{\kappa} (1-p)^{n-\kappa}; \quad \kappa= 0,1, \ldots, n; \quad p = \frac{|S|}{|B|} .
}\end{equation}

Sabe-se da teoria de probabilidade que o valor esperado (média) de uma
distribuição binomial \(X\) é dado por
\(\mu = \mathbb{E}(X) = n\times p\), onde \(n\) é o número de ensaios
realizados. Substituindo \(p = \frac{|S|}{|B|}\) e \(X\) por
\(n(S\cap B)\), obtém-se o número médio de eventos de um processo
pontual dado por,

\begin{equation}\phantomsection\label{eq-10}{
\mu = \mathbb{E}[n(S\cap B)] = n\times \frac{|S|}{|B|} \Leftrightarrow \frac{ \mathbb{E}[n(S\cap B)]}{|S|} =\frac{n}{|B|} \Rightarrow \hat{\lambda} (s) =\frac{n}{|B|},
}\end{equation}

onde \(\hat{\lambda}(s)\) é denominado estimador da intensidade
\(\lambda (s)\), e corresponde ao número médio de eventos por unidade de
área ou volume sob condição de homogeneidade.

\textbf{Distribuição Poisson}

A distribuição Poisson é comumente empregada para modelar a ocorrência
de eventos raros e independentes em uma área de estudo. Assim, um
processo pontual que segue a distribuição Poisson descreve o número de
eventos raros que ocorrem em uma determinada área de estudo, e sua
distribuição de probabilidade é dada por,

\begin{equation}\phantomsection\label{eq-11}{
P(n(S)=K) =\frac{\mu^{k} \times e^{- \mu }}{k!} = \frac{\left(\lambda (s)|S| \right)^{k} \times e^{ - \lambda (s)|S|} }{k!} .
}\end{equation}

Um processo pontual que segue uma distribuição Poisson e é homogêneo, é
denominado processo pontual Poisson homogêneo. Este processo é
caracterizado por ser estacionário e isotrópico, razão pela qual é
também denominado processo de aleatoriedade espacial completa (não exibe
dependência espacial), e se a região \(B\) de estudo é dividida em \(s\)
sub-regiões diferentes, a intensidade \(\lambda (s)\) nessas \(s\)
sub-regiões, estimada por \(\hat{\lambda} (s)\) será constante (não
variará), ou seja,
\(\forall s_{i}, s_{i^{'}} \in B: \lambda (s_{i}) \equiv \lambda (s_{i^{'}})\).
Isso decorre do fato de que a distribuição Poisson possui média e
variância iguais, e sua relação com a distribuição Binomial é
demonstrada em Lima (2005) e Mateus (2013). Caso o processo pontual seja
Poisson, mas apresente tendência, ele é denominado processo pontual
Poisson não homogêneo. Nesse contexto, embora o processo não exiba
dependência espacial, a intensidade \(\lambda (s)\) nas \(s\)
sub-regiões não é constante, ou seja,
\(\exists \, s_{i} \in B: \lambda (s_{i}) \neq \lambda (s_{i^{'}})\) e,
o número médio de eventos nessa situação é dado por,
\(\mu = \int_{B} \lambda (s)\).

\section{Propriedade de primeira e segunda ordem}\label{sec-2.2.4}

\textbf{Propriedade de primeira ordem}

Sejam \(S_{i}\) e \(S_{i^{'}}\) sub-regiões de \(B\) contendo os eventos
\(s_{i}=(x,y)\) e \(s_{i^{'}}=(x^{'},y^{'})\), onde o número de eventos
é dado por \(n(s_{i})\) e \(n(s_{i^{'}})\) e sua área dada por
\(|S_{i}|\) e \(|S_{i^{'}}|\) respectivamente. A propriedade de primeira
ordem, também designada intensidade \(\lambda(s_{i})\), corresponde o
número de eventos esperados por unidade de área, ou seja,\\
\(\lambda (s_{i}) = \lim_{|S_{i}| \to 0} \left\{\frac{ \mathbb{E}[n(s_{i})]}{|S_{i}|}\right\}\),
conforme descrito por Mateus (2013), E. Pebesma e Bivand (2023) e Moraga
(2023).

Se o processo pontual for homogêneo, a intensidade \(\lambda (s_{i})\) é
constante, e dado por,
\(\lambda (s_{i}) = \lambda =\frac{ \mathbb{E}[n(s_{i})]}{|S_{i}|}, \,\hat{\lambda}= \frac{n(s_{i})}{|S_{i}|}\).

Para ilustrar estes conceitos, vamos simular um Processo Pontual de
Poisson Não-Homogêneo, onde a intensidade dos eventos aumenta conforme a
coordenada \(X\) aumenta (um gradiente).

\begin{Shaded}
\begin{Highlighting}[]
\NormalTok{pacman}\SpecialCharTok{::}\FunctionTok{p\_load}\NormalTok{(spatstat,ggplot2,patchwork)}

\NormalTok{janela }\OtherTok{\textless{}{-}} \FunctionTok{owin}\NormalTok{(}\FunctionTok{c}\NormalTok{(}\DecValTok{0}\NormalTok{, }\DecValTok{10}\NormalTok{), }\FunctionTok{c}\NormalTok{(}\DecValTok{0}\NormalTok{, }\DecValTok{10}\NormalTok{))}

\NormalTok{funcao\_intensidade }\OtherTok{\textless{}{-}} \ControlFlowTok{function}\NormalTok{(x, y) \{ }\DecValTok{5} \SpecialCharTok{*}\NormalTok{ x \}}

\FunctionTok{set.seed}\NormalTok{(}\DecValTok{123}\NormalTok{)}
\NormalTok{pp\_simulado }\OtherTok{\textless{}{-}} \FunctionTok{rpoispp}\NormalTok{(funcao\_intensidade, }\AttributeTok{win =}\NormalTok{ janela)}

\FunctionTok{plot}\NormalTok{(pp\_simulado, }\AttributeTok{main =} \StringTok{"Tendência em X"}\NormalTok{, }
     \AttributeTok{pch =} \DecValTok{19}\NormalTok{, }\AttributeTok{cex =} \FloatTok{0.4}\NormalTok{)}
\end{Highlighting}
\end{Shaded}

\pandocbounded{\includegraphics[keepaspectratio]{point_process_files/figure-pdf/simulacao-dados1-1.pdf}}

Uma maneira de verificar se a intensidade é constante consiste em
dividir a área de estudo em sub-regiões ou \emph{quadrats} do mesmo
tamanho (área), contar os eventos em cada sub-região ou \emph{quadrats}
e dividir pela respectiva área da sub-região ou \emph{quadrats}. No
entanto, este procedimento resulta em uma intensidade não suavizada (
Figura~\ref{fig-estimadores-intensidade} (a)).

Outra alternativa é usar uma janela móvel de tamanho fixo, centrada em
vários locais da área de estudo. Ao mover essa janela pela região, os
eventos são contados e divididos pela área da janela, proporcionando uma
estimativa mais suave e interpretável da variação da intensidade
\(\lambda (s_{i})\) ( Figura~\ref{fig-estimadores-intensidade} (b)). No
entanto, em cada uma das estimativas de intensidade (com janela móvel ou
sub-regiões do mesmo tamanho), não se considera a localização relativa
dos eventos dentro janela/sub-região (se os eventos estão próximos ou
afastados) e a escolha de um tamanho de janela adequado não é clara
(Gatrell et al. 1996).

Vale ressaltar que a variação na intensidade é mais significativa em
processos pontuais não homogêneos. Assim, no caso do processo pontual
não homogêneo, uma maneira de estimar a função intensidade
\(\lambda (s_{i})\), é usando os estimadores não paramétricos de kernel,
onde a janela ou \emph{quadrats} outrora descrita (o), é substituída por
uma função (Kernel) que descreve um objeto tridimensional móvel, e que
pondera os eventos dentro de sua esfera de influência consoante a
distância do ponto onde a intensidade está sendo estimada (Gatrell et
al. 1996) ( Figura~\ref{fig-estimadores-intensidade} (c)).
Essencialmente, neste método estima-se a função de densidade de
probabilidade \(f(z)\) e não função intensidade \(\lambda(s_{i})\).

A função de densidade de probabilidade descreve a probabilidade de
observar um evento em uma determinada região de estudo e esta função é
não-negativa e integra para 1, enquanto a função intensidade
\(\lambda (s_{i})\) não é uma medida de probabilidade, mas sim uma
medida da taxa de ocorrência de eventos em uma determinada área, que
embora seja não-negativa, não precisa integrar para 1, pois ela não
descreve uma distribuição de probabilidade (Moraga 2023).

Embora a função de densidade de probabilidade \(f(z)\) e a função
intensidade \(\lambda(s_{i})\) não sejam iguais, estas são
proporcionais, o que significa que locais com maior taxa de ocorrências
dos eventos (intensidade), terão também maior densidade de probabilidade
\(f(z)\). Assim, a função densidade de probabilidade e função
intensidade são dadas por,

\begin{equation}\phantomsection\label{eq-13..}{
\lambda (z) = f(z) \int_{B} \lambda (s_{i})ds_{i}, \: \hat{f}(z)=\frac{1}{n}\sum_{i=1}^{n} \frac{1}{h^{2}}K\left(\frac{z-s_{i}}{h}\right) \: \text{ e } \: \hat{\lambda}(z)=\sum_{i=1}^{n} \frac{1}{h^{2}}K\left(\frac{z-s_{i}}{h}\right),
}\end{equation}

onde \(\int_{B} \lambda (s_{i})ds_{i}\) é o número esperado de eventos
na região \(S_{i}\),\(\hat{f}(z)\) e \(\hat{\lambda}(z)\) são
respectivamente estimadores Kernel não paramétricos da função densidade
e intensidade, no local \(z\) da região \(B\) em estudo, com base nos
eventos \(s_{i}\).\(K(s)\) é uma função densidade de probabilidade
simétrica tal que \(\forall s, \, K(s)\geq 0\), com \(\int_{B} K(s)=1\),
designada função Kernel e \(h\) é um parâmetro de suavização ou largura
da banda ( Figura~\ref{fig-estimadores-intensidade} (c)).

\begin{figure}

\centering{

\pandocbounded{\includegraphics[keepaspectratio]{point_process_files/figure-pdf/fig-estimadores-intensidade-1.pdf}}

}

\caption{\label{fig-estimadores-intensidade}Comparação de Estimadores de
Intensidade: (a) Quadrats (Não suavizada), (b) Janela Móvel (Disco
Uniforme) e (c) Estimativa de Kernel (Gaussiano).}

\end{figure}%

Conforme descrito por Gatrell et al. (1996) e Gelfand et al. (2010), o
estimador de Kernel \(K(s)\), é sensível à escolha da largura de banda
\(h\), de tal modo que, um valor maior de \(h\) resulta em maior
suavização na variação espacial da intensidade (viés aumenta e a
variância diminui), enquanto menor valor de \(h\) resulta em menor
suavização (viés diminui e a variância aumenta). Portanto, testar várias
larguras de banda é uma opção viável para selecionar o valor ideal do
parâmetro de suavização, mas também existem métodos computacionais que
permitem uma escolha mais eficiente, como a validação cruzada por
verossimilhança ou outros descritos por Cronie e Van Lieshout (2016).

As escolhas comuns para o Kernel \(k(s)\) incluem
\(K(s)=\frac{1}{\sqrt{2\pi}}\exp\left(-\frac{s^{2}}{2}\right)\),\(K(s)=\frac{3}{4} (1-s^{2}) \mathbb{I} (|s|<1)\),\(K(s)=\frac{15}{16} (1-s^{2})^{2}  \mathbb{I} (|s|<1)\),\(K(s)=\frac{1}{2} \mathbb{I} (|s|<1)\),
que correspondem a kernel Gaussiano, Kernel Epanechnikov, Kernel
Quadrático e Kernel Uniforme, respectivamente. Além destas, existem
outras na literatura e algumas são apresentadas na
Figura~\ref{fig-kernels-comparison}.

\begin{Shaded}
\begin{Highlighting}[]
\NormalTok{pacman}\SpecialCharTok{::}\FunctionTok{p\_load}\NormalTok{(ggplot2, dplyr,tidyr,spatstat, patchwork)}

\CommentTok{\#PARTE 1: Curvas Matemáticas 1D}
\NormalTok{s }\OtherTok{\textless{}{-}} \FunctionTok{seq}\NormalTok{(}\SpecialCharTok{{-}}\FloatTok{1.5}\NormalTok{, }\FloatTok{1.5}\NormalTok{, }\AttributeTok{length.out =} \DecValTok{500}\NormalTok{)}

\NormalTok{df\_curves }\OtherTok{\textless{}{-}} \FunctionTok{data.frame}\NormalTok{(}\AttributeTok{s =}\NormalTok{ s) }\SpecialCharTok{\%\textgreater{}\%}
  \FunctionTok{mutate}\NormalTok{(}
    \AttributeTok{Gaussiano =} \FunctionTok{dnorm}\NormalTok{(s), }\CommentTok{\# Aproximação padrão}
    \AttributeTok{Epanechnikov =} \FunctionTok{ifelse}\NormalTok{(}\FunctionTok{abs}\NormalTok{(s) }\SpecialCharTok{\textless{}} \DecValTok{1}\NormalTok{, }\FloatTok{0.75} \SpecialCharTok{*}\NormalTok{ (}\DecValTok{1} \SpecialCharTok{{-}}\NormalTok{ s}\SpecialCharTok{\^{}}\DecValTok{2}\NormalTok{), }\DecValTok{0}\NormalTok{),}
    \AttributeTok{Quadratico =} \FunctionTok{ifelse}\NormalTok{(}\FunctionTok{abs}\NormalTok{(s) }\SpecialCharTok{\textless{}} \DecValTok{1}\NormalTok{, (}\DecValTok{15}\SpecialCharTok{/}\DecValTok{16}\NormalTok{) }\SpecialCharTok{*}\NormalTok{ (}\DecValTok{1} \SpecialCharTok{{-}}\NormalTok{ s}\SpecialCharTok{\^{}}\DecValTok{2}\NormalTok{)}\SpecialCharTok{\^{}}\DecValTok{2}\NormalTok{, }\DecValTok{0}\NormalTok{),}
    \AttributeTok{Uniforme =} \FunctionTok{ifelse}\NormalTok{(}\FunctionTok{abs}\NormalTok{(s) }\SpecialCharTok{\textless{}} \DecValTok{1}\NormalTok{, }\FloatTok{0.5}\NormalTok{, }\DecValTok{0}\NormalTok{),}
    \CommentTok{\# Adicionando um extra comum para comparação}
    \AttributeTok{Triangular =} \FunctionTok{ifelse}\NormalTok{(}\FunctionTok{abs}\NormalTok{(s) }\SpecialCharTok{\textless{}} \DecValTok{1}\NormalTok{, }\DecValTok{1} \SpecialCharTok{{-}} \FunctionTok{abs}\NormalTok{(s), }\DecValTok{0}\NormalTok{)}
\NormalTok{  ) }\SpecialCharTok{\%\textgreater{}\%}
  \FunctionTok{pivot\_longer}\NormalTok{(}\AttributeTok{cols =} \SpecialCharTok{{-}}\NormalTok{s, }\AttributeTok{names\_to =} \StringTok{"Kernel"}\NormalTok{, }\AttributeTok{values\_to =} \StringTok{"Peso"}\NormalTok{)}

\CommentTok{\# Gráfico das Curvas}
\NormalTok{p\_curves }\OtherTok{\textless{}{-}} \FunctionTok{ggplot}\NormalTok{(df\_curves, }\FunctionTok{aes}\NormalTok{(}\AttributeTok{x =}\NormalTok{ s, }\AttributeTok{y =}\NormalTok{ Peso, }\AttributeTok{color =}\NormalTok{ Kernel)) }\SpecialCharTok{+}
  \FunctionTok{geom\_line}\NormalTok{(}\AttributeTok{size =}\NormalTok{ .}\DecValTok{5}\NormalTok{) }\SpecialCharTok{+}
  \FunctionTok{facet\_wrap}\NormalTok{(}\SpecialCharTok{\textasciitilde{}}\NormalTok{Kernel, }\AttributeTok{nrow =} \DecValTok{1}\NormalTok{) }\SpecialCharTok{+}
  \FunctionTok{theme\_minimal}\NormalTok{() }\SpecialCharTok{+}
  \FunctionTok{labs}\NormalTok{(}\AttributeTok{title =} \StringTok{""}\NormalTok{, }\AttributeTok{x =} \StringTok{"Distância (s)"}\NormalTok{, }\AttributeTok{y =} \StringTok{"Peso K(s)"}\NormalTok{) }\SpecialCharTok{+}
  \FunctionTok{theme}\NormalTok{(}\AttributeTok{legend.position =} \StringTok{"none"}\NormalTok{, }
        \AttributeTok{plot.title =} \FunctionTok{element\_text}\NormalTok{(}\AttributeTok{hjust =} \FloatTok{0.5}\NormalTok{))}


\CommentTok{\#PARTE 2: Visualização Espacial 2D }
\NormalTok{pp\_unico }\OtherTok{\textless{}{-}} \FunctionTok{ppp}\NormalTok{(}\AttributeTok{x =} \DecValTok{5}\NormalTok{, }\AttributeTok{y =} \DecValTok{5}\NormalTok{, }\AttributeTok{window =} \FunctionTok{owin}\NormalTok{(}\FunctionTok{c}\NormalTok{(}\DecValTok{0}\NormalTok{, }\DecValTok{10}\NormalTok{), }\FunctionTok{c}\NormalTok{(}\DecValTok{0}\NormalTok{, }\DecValTok{10}\NormalTok{)))}

\CommentTok{\# Função auxiliar para gerar o raster do spatstat e converter para data.frame}
\NormalTok{get\_kernel\_raster }\OtherTok{\textless{}{-}} \ControlFlowTok{function}\NormalTok{(pp, kernel\_name, }\AttributeTok{sigma=}\FloatTok{1.5}\NormalTok{) \{}
\NormalTok{  k\_spatstat }\OtherTok{\textless{}{-}} \ControlFlowTok{switch}\NormalTok{(kernel\_name,}
                       \StringTok{"Gaussiano"} \OtherTok{=} \StringTok{"gaussian"}\NormalTok{,}
                       \StringTok{"Epanechnikov"} \OtherTok{=} \StringTok{"epanechnikov"}\NormalTok{,}
                       \StringTok{"Quadratico"} \OtherTok{=} \StringTok{"quartic"}\NormalTok{, }
                       \StringTok{"Uniforme"} \OtherTok{=} \StringTok{"disc"}\NormalTok{)     }
  
  \ControlFlowTok{if}\NormalTok{(}\FunctionTok{is.null}\NormalTok{(k\_spatstat)) }\FunctionTok{return}\NormalTok{(}\ConstantTok{NULL}\NormalTok{)}
\NormalTok{  dens }\OtherTok{\textless{}{-}} \FunctionTok{density}\NormalTok{(pp, }\AttributeTok{sigma =}\NormalTok{ sigma, }\AttributeTok{kernel =}\NormalTok{ k\_spatstat)}
\NormalTok{  df }\OtherTok{\textless{}{-}} \FunctionTok{as.data.frame}\NormalTok{(dens)}
\NormalTok{  df}\SpecialCharTok{$}\NormalTok{Kernel }\OtherTok{\textless{}{-}}\NormalTok{ kernel\_name}
  \FunctionTok{return}\NormalTok{(df)}
\NormalTok{\}}

\CommentTok{\#}
\NormalTok{lista\_dfs }\OtherTok{\textless{}{-}} \FunctionTok{lapply}\NormalTok{(}\FunctionTok{c}\NormalTok{(}\StringTok{"Gaussiano"}\NormalTok{, }\StringTok{"Epanechnikov"}\NormalTok{, }\StringTok{"Quadratico"}\NormalTok{, }\StringTok{"Uniforme"}\NormalTok{), }
                    \ControlFlowTok{function}\NormalTok{(k) }\FunctionTok{get\_kernel\_raster}\NormalTok{(pp\_unico, k))}
\NormalTok{df\_raster }\OtherTok{\textless{}{-}} \FunctionTok{do.call}\NormalTok{(rbind, lista\_dfs)}


\NormalTok{p\_raster }\OtherTok{\textless{}{-}} \FunctionTok{ggplot}\NormalTok{(df\_raster, }\FunctionTok{aes}\NormalTok{(x, y, }\AttributeTok{fill =}\NormalTok{ value)) }\SpecialCharTok{+}
  \FunctionTok{geom\_raster}\NormalTok{() }\SpecialCharTok{+}
  \FunctionTok{facet\_wrap}\NormalTok{(}\SpecialCharTok{\textasciitilde{}}\NormalTok{Kernel, }\AttributeTok{nrow =} \DecValTok{1}\NormalTok{) }\SpecialCharTok{+}
  \FunctionTok{scale\_fill\_viridis\_c}\NormalTok{(}\AttributeTok{option =} \StringTok{"mako"}\NormalTok{) }\SpecialCharTok{+}
  \FunctionTok{coord\_fixed}\NormalTok{() }\SpecialCharTok{+}
  \FunctionTok{theme\_void}\NormalTok{() }\SpecialCharTok{+}
  \FunctionTok{labs}\NormalTok{(}\AttributeTok{title =} \StringTok{"Efeito no Espaço (2D)"}\NormalTok{ ) }\SpecialCharTok{+}
  \FunctionTok{theme}\NormalTok{(}\AttributeTok{legend.position =} \StringTok{"none"}\NormalTok{,}
        \AttributeTok{strip.text =} \FunctionTok{element\_text}\NormalTok{(}\AttributeTok{size =} \DecValTok{11}\NormalTok{, }\AttributeTok{margin =} \FunctionTok{margin}\NormalTok{(}\AttributeTok{b=}\DecValTok{5}\NormalTok{)),}
        \AttributeTok{plot.title =} \FunctionTok{element\_text}\NormalTok{(}\AttributeTok{hjust =} \FloatTok{0.5}\NormalTok{, }\AttributeTok{margin =} \FunctionTok{margin}\NormalTok{(}\AttributeTok{t=}\DecValTok{15}\NormalTok{, }\AttributeTok{b=}\DecValTok{5}\NormalTok{)))}


\NormalTok{p\_curves }\SpecialCharTok{/}\NormalTok{ p\_raster}
\end{Highlighting}
\end{Shaded}

\begin{figure}[H]

\centering{

\pandocbounded{\includegraphics[keepaspectratio]{point_process_files/figure-pdf/fig-kernels-comparison-1.pdf}}

}

\caption{\label{fig-kernels-comparison}Funções de Kernel}

\end{figure}%

A escolha do Kernel dependerá da pesquisa e do pesquisador, tomando em
consideração que o Kernel Uniforme atribui peso igual a todos os eventos
em uma distância fixa do evento de referência, sendo útil quando a
distribuição dos eventos é uniforme, embora seja menos sensível a
variações na densidade dos dados. O Kernel Gaussiano atribui pesos
baseados na distribuição gaussiana (normal), capturando padrões suaves e
contínuos nos dados e sendo adequado para distribuições não uniformes,
embora mais sensível a variações locais. O Kernel Epanechnikov, com sua
forma parabólica e suporte compacto, é mais eficiente que o Gaussiano,
equilibrando características locais e evitando sensibilidade excessiva
ao ruído. O Kernel Quadrático, também parabólico, oferece um compromisso
entre os Kernels Uniforme e Gaussiano, sendo menos sensível a outliers
que o Gaussiano, mas mais sensível que o Uniforme, ideal para equilibrar
robustez e suavidade (García-Portugués 2024; Scalon 2024).

Conforme descrito por Moraga (2023) e Scalon (2024), os efeitos de borda
tendem a distorcer as estimativas do Kernel, próximo à fronteira
(margens) da região de estudo, uma vez que os eventos próximos à
fronteira (margens) têm menos vizinhos locais do que os eventos no
interior. Uma maneira de lidar com esse problema é modificar a
estimativa do Kernel dividindo-a pelo seguinte termo de correção de
borda \(e_{h} (s)=\int_{B} h^{-2}K(\frac{z-s}{h})dx\), que representa o
volume sob o Kernel centrado em \(z\) que está na região de estudo \(B\)
(Gatrell et al. 1996).

Conforme descrito por Adrian Baddeley, Rubak, e Turner (2015), dentre os
estimadores de Kernel usuais da função de intensidade
destacam-se,\(\hat{\lambda}^{0}(z)=\sum_{i=1}^{n}K(z-s_{i})\),\(\hat{\lambda}^{U}(z)=\frac{1}{e(z)}\sum_{i=1}^{n}K(z-s_{i})\)
e \(\hat{\lambda}^{D}(z)=\sum_{i=1}^{n} \frac{1}{e(s_{i})} K(z-s_{i})\),
e correspondem a estimador não corrigido, estimador uniformemente
corrigido e estimador com correção de Diggle, respectivamente. Nestes
estimadores,\(K(\cdot)\) é função densidade de probabilidade Kernel e
\(e(s_{i})\) a correção da borda, dada por
\(e(s_{i})=\int_{B} K(z-s)dx\).

Além do estimador não paramétricos de kernel, outra maneira de estimar a
função de intensidade \(\lambda (s_{i})\), é usar estimadores
paramétricos (Seção~\ref{sec-2.7}).

\section{Propriedade de segunda
ordem}\label{propriedade-de-segunda-ordem}

Segundo Mateus (2013) e E. Pebesma e Bivand (2023), a propriedade de
segunda ordem, também conhecida como intensidade de segunda ordem ou
densidade, descreve a forma como os eventos estão distribuídos no
espaço, podendo indicar se estão distribuídos independentemente uns dos
outros (aleatoriedade espacial completa), se tendem a se agrupar
(agrupamento) ou se repelem mutuamente (distribuição mais regular do que
sob aleatoriedade espacial completa) e, é representada por

\[
\lambda (s_{i},s_{i^{'}})= \lim_{\substack{|S_{i}|\to 0 \\ |S_{i^{'}}|\to 0}} \left\{\frac{ \mathbb{E}[n(s_{i}) n(s_{i^{'}})]}{|S_{i}||S_{i^{'}}|}\right\}
\]

e estimada usando as funções correlação par \(g(r)\), função \(L(r)\) e
função \(K (r)\) ou variantes não homogêneas dessas funções.

Segundo Gatrell et al. (1996) e Peter J. Diggle (2013), se o processo
pontual for estacionário, a propriedade de segunda ordem dependerá
apenas da diferença vetorial \(d\) (direção e distância), entre
\(s_{i} \, \text{ e } \, s_{i^{'}}\), e não de suas localizações
absolutas, ou seja,
\(\lambda (s_{i},s_{i^{'}})\equiv \lambda (||s_{i}-s_{i^{'}}||)\).
Todavia, se for estacionário e isotrópico, a intensidade de segunda
ordem dependerá apenas da distância \(d(s_{i},s_{i^{'}})\) e não da sua
orientação ou direção.

Além da intensidade de primeira e segunda ordem, existe a chamada
intensidade condicional.

Segundo Peter J. Diggle (2013), a intensidade condicional corresponde à
intensidade do evento \(s_{i}\) condicionada à (dada) informação do
evento \(s_{i^{'}}\), ou seja,\\
\(\lambda (s_{i}|s_{i^{'}}) = \frac{\lambda (s_{i},s_{i^{'}})}{\lambda(s_{i^{'}})}\).
Pela propriedade de independência entre os eventos, descrita na seção
Seção~\ref{sec-2.3.2}, se os eventos \(s_{i}\) e \(s_{i^{'}}\) forem
independentes, a intensidade condicional \(\lambda (s_{i}|s_{i^{'}})\)
será dada por
\(\lambda (s_{i}|s_{i^{'}}) =\frac{ \lambda (s_{i},s_{i^{'}})}{\lambda(s_{i^{'}})} =\lambda (s_{i})\).

\section{Padrões de distribuição espacial e sua
identificação}\label{sec-2.6.1.3}

Um processo pontual pode exibir três padrões espaciais distintos (
Figura~\ref{fig-padroes-espaciais}), que são, o aleatório, onde os
eventos (representados por ``o'') estão distribuídos de forma aleatória,
com média e variância iguais; regular, quando os eventos estão espaçados
de maneira uniforme em comparação com um padrão aleatório, resultando em
uma variância menor que a média (subdispersão); e agrupado, quando os
eventos estão mais próximos uns dos outros do que seria esperado em um
padrão aleatório, resultando em uma variância maior que a média
(superdispersão) (Lima 2005; Scalon 2024).

\begin{Shaded}
\begin{Highlighting}[]
\FunctionTok{set.seed}\NormalTok{(}\DecValTok{42}\NormalTok{) }\CommentTok{\# Para reprodutibilidade}
\NormalTok{janela }\OtherTok{\textless{}{-}} \FunctionTok{owin}\NormalTok{(}\FunctionTok{c}\NormalTok{(}\DecValTok{0}\NormalTok{, }\DecValTok{1}\NormalTok{), }\FunctionTok{c}\NormalTok{(}\DecValTok{0}\NormalTok{, }\DecValTok{1}\NormalTok{))}
\NormalTok{n\_pontos }\OtherTok{\textless{}{-}} \DecValTok{60} \CommentTok{\# Número alvo aproximado de pontos}

\CommentTok{\#Padrão Aleatório (Poisson Homogêneo {-} CSR)}
\NormalTok{pp\_aleatorio }\OtherTok{\textless{}{-}} \FunctionTok{rpoispp}\NormalTok{(}\AttributeTok{lambda =}\NormalTok{ n\_pontos, }\AttributeTok{win =}\NormalTok{ janela)}
\NormalTok{df\_rand }\OtherTok{\textless{}{-}} \FunctionTok{as.data.frame}\NormalTok{(pp\_aleatorio)}
\NormalTok{df\_rand}\SpecialCharTok{$}\NormalTok{Tipo }\OtherTok{\textless{}{-}} \StringTok{"(a) Aleatório"}

\CommentTok{\#Padrão Regular (Processo de Inibição {-} Hard Core)}
\NormalTok{pp\_regular }\OtherTok{\textless{}{-}} \FunctionTok{rSSI}\NormalTok{(}\AttributeTok{r =} \FloatTok{0.09}\NormalTok{, }\AttributeTok{n =}\NormalTok{ n\_pontos, }\AttributeTok{win =}\NormalTok{ janela)}
\NormalTok{df\_reg }\OtherTok{\textless{}{-}} \FunctionTok{as.data.frame}\NormalTok{(pp\_regular)}
\NormalTok{df\_reg}\SpecialCharTok{$}\NormalTok{Tipo }\OtherTok{\textless{}{-}} \StringTok{"(b) Regular"}

\CommentTok{\#Padrão Agrupado (Processo de Thomas {-} Cluster)}
\NormalTok{pp\_agrupado }\OtherTok{\textless{}{-}} \FunctionTok{rThomas}\NormalTok{(}\AttributeTok{kappa =} \DecValTok{5}\NormalTok{, }\AttributeTok{scale =} \FloatTok{0.04}\NormalTok{, }\AttributeTok{mu =} \DecValTok{12}\NormalTok{, }\AttributeTok{win =}\NormalTok{ janela)}
\NormalTok{df\_clus }\OtherTok{\textless{}{-}} \FunctionTok{as.data.frame}\NormalTok{(pp\_agrupado)}
\NormalTok{df\_clus}\SpecialCharTok{$}\NormalTok{Tipo }\OtherTok{\textless{}{-}} \StringTok{"(c) Agrupado"}

\CommentTok{\#}
\NormalTok{plot\_padrao }\OtherTok{\textless{}{-}} \ControlFlowTok{function}\NormalTok{(df, titulo, cor) \{}
  \FunctionTok{ggplot}\NormalTok{(df, }\FunctionTok{aes}\NormalTok{(x, y)) }\SpecialCharTok{+}
    \FunctionTok{geom\_point}\NormalTok{(}\AttributeTok{shape =} \DecValTok{1}\NormalTok{, }\AttributeTok{size =} \DecValTok{2}\NormalTok{, }\AttributeTok{stroke =} \DecValTok{1}\NormalTok{, }\AttributeTok{color =}\NormalTok{ cor) }\SpecialCharTok{+}
    \FunctionTok{coord\_fixed}\NormalTok{(}\AttributeTok{xlim =} \FunctionTok{c}\NormalTok{(}\DecValTok{0}\NormalTok{, }\DecValTok{1}\NormalTok{), }\AttributeTok{ylim =} \FunctionTok{c}\NormalTok{(}\DecValTok{0}\NormalTok{, }\DecValTok{1}\NormalTok{)) }\SpecialCharTok{+}
    \FunctionTok{theme\_bw}\NormalTok{() }\SpecialCharTok{+}
    \FunctionTok{labs}\NormalTok{(}\AttributeTok{title =}\NormalTok{ titulo, }\AttributeTok{x =} \StringTok{""}\NormalTok{, }\AttributeTok{y =} \StringTok{""}\NormalTok{) }\SpecialCharTok{+}
    \FunctionTok{theme}\NormalTok{(}
      \AttributeTok{plot.title =} \FunctionTok{element\_text}\NormalTok{(}\AttributeTok{hjust =} \FloatTok{0.5}\NormalTok{),}
      \AttributeTok{axis.text =} \FunctionTok{element\_blank}\NormalTok{(),}
      \AttributeTok{axis.ticks =} \FunctionTok{element\_blank}\NormalTok{(),}
      \AttributeTok{panel.grid =} \FunctionTok{element\_blank}\NormalTok{()}
\NormalTok{    )}
\NormalTok{\}}


\NormalTok{g1 }\OtherTok{\textless{}{-}} \FunctionTok{plot\_padrao}\NormalTok{(df\_rand, }\StringTok{"(a) Aleatório"}\NormalTok{, }\StringTok{"black"}\NormalTok{)}
\NormalTok{g2 }\OtherTok{\textless{}{-}} \FunctionTok{plot\_padrao}\NormalTok{(df\_reg,  }\StringTok{"(b) Regular"}\NormalTok{,   }\StringTok{"blue"}\NormalTok{)}
\NormalTok{g3 }\OtherTok{\textless{}{-}} \FunctionTok{plot\_padrao}\NormalTok{(df\_clus, }\StringTok{"(c) Agrupado"}\NormalTok{,  }\StringTok{"red"}\NormalTok{)}


\NormalTok{g1 }\SpecialCharTok{+}\NormalTok{ g2 }\SpecialCharTok{+}\NormalTok{ g3}
\end{Highlighting}
\end{Shaded}

\begin{figure}[H]

\centering{

\pandocbounded{\includegraphics[keepaspectratio]{point_process_files/figure-pdf/fig-padroes-espaciais-1.pdf}}

}

\caption{\label{fig-padroes-espaciais}Padrões de Distribuição Espacial:
(a) Aleatório (Poisson), (b) Regular (Inibição) e (c) Agrupado
(Thomas).}

\end{figure}%

Para identificar os padrões apresentados na
Figura~\ref{fig-padroes-espaciais}, há várias abordagens, que vão desde
estatísticas simples como o índice de Morisita (1959), o índice de Clark
e Evans (1954), o índice de Hopkins e Skellam (1954), até o uso de
funções (Scalon 2024).

Neste estudo, prioriza-se o uso de funções, pois estas permitem analisar
os padrões em diferentes distâncias (r) e podem ser visualizadas
graficamente, o que não é possível com simples estatísticas. Entre as
funções, destacam-se as funções sumárias homogêneas de
distância,\(F(r)\),\(G(r)\),\(J(r)\), bem como as funções sumárias
homogêneas de segunda ordem \(K(r)\),\(L(r)\) e \(g(r)\), ou suas
extensões não homogêneas. Essas funções são comparadas com um processo
pontual Poisson homogêneo
\(F_{pois}(r)= G_{pois}(r) = 1- e^{-\pi \lambda r^{2}}\) e
\(J_{pois}(r) = 1\) ou \(K_{pois}(r)= \pi r^{2}\),\(L_{pois}(r)= r\) e
\(g_{pois}(r)= 1\). Se forem equivalentes, diz-se que o padrão de
distribuição espacial é de aleatoriedade espacial completa (AEC), caso
contrário, pode ser regular (Inibição) ou agregado (Atração) conforme
descrito no Quadro Tabela~\ref{tbl-1}.

\section{\texorpdfstring{Funções
\(F(r),G(r),J(r),K(r),L(r)\, e \,g(r)\)}{Funções F(r),G(r),J(r),K(r),L(r)\textbackslash, e \textbackslash,g(r)}}\label{sec-2.6.1}

Nesta seção, apresentamos as funções
\(F(r),G(r),J(r),K(r),L(r)\, e \,g(r)\) homogêneas, úteis para
identificação do padrão de distribuição (regular, aleatório ou agrupado)
dos processos pontuais.

\textbf{Função F (r)}

Em um processo pontual \(s\), a distância
\(d(z, s_{i}) = min{||z-s_{i}|| : s_{i} \in S}\) representa a menor
distância de quaisquer pontos fixos imaginários \(z \in \mathbb{R}^{2}\)
para os \(i\)-ésimos eventos \(s_{i}\) de estudo mais próximos, conforme
ilustrado na Figura~\ref{fig-conceito-gf} (b). Esta distância é
comumente referida como a distância do espaço vazio e a função que
representa todas as possíveis distâncias (função cumulativa) é
denominada função \(F(r)\) e é definida como:

\begin{equation}\phantomsection\label{eq-16}{
F(r)= P\{d(z, s_{i})\leq r\} \text{ e } \hat{F}(r) = \frac{1}{n(z)} \sum_{k=1}^{K} \mathbb{I}\left\{d(z_{k}, s_{i})\leq r\right\}, \forall r \geq 0,
}\end{equation}

onde \(\hat{F}(r)\) é seu estimador, \(\mathbb{I}\) é função indicadora
e \(d(z_{k}, s_{i})\) a distância real entre \(k\)-ésimos pontos
\(z_{k}\) (não eventos) e \(i\)-ésimos eventos \(s_{i}\) mais próximos
(Gelfand et al. 2010; Peter John Diggle 2003; Adrian Baddeley, Rubak, e
Turner 2015; Scalon 2024).

Se a curva \(\hat{F}(r)\) estiver acima da curva \(F_{pois}(r)\), ou
seja, \(\hat{F}(r)>F_{pois}(r)\), indica que a distância entre quaisquer
pontos \(z\) até o evento mais próximo \(s_{i}\) é menor do que seria
sob aleatoriedade espacial completa. Isto sugere que existe menor espaço
vazio (não ocupado pelos eventos) e, consequentemente, o padrão espacial
dos eventos é regular ( Figura~\ref{fig-funcao-f} (a)). Se
\(\hat{F}(r)\) é equivalente a
\(F_{pois}(r)\)\((\hat{F}(r) \equiv F_{pois}(r))\), o padrão é
considerado aleatório, também designado aleatoriedade espacial completa
(AEC) ( Figura~\ref{fig-funcao-f} (b)). Se a curva \(\hat{F}(r)\)
estiver abaixo da curva \(F_{pois}(r)\)\((\hat{F}(r) < F_{pois}(r))\),
significa que a distância entre quaisquer pontos \(z\) até o evento mais
próximo \(s_{i}\) é maior do que seria sob aleatoriedade espacial
completa. Nesse caso, há mais espaço vazio e, consequentemente, o padrão
espacial dos eventos é agrupado ( Figura~\ref{fig-funcao-f} (c)).

\begin{Shaded}
\begin{Highlighting}[]
\FunctionTok{set.seed}\NormalTok{(}\DecValTok{100}\NormalTok{) }
\NormalTok{janela }\OtherTok{\textless{}{-}} \FunctionTok{owin}\NormalTok{(}\FunctionTok{c}\NormalTok{(}\DecValTok{0}\NormalTok{, }\DecValTok{1}\NormalTok{), }\FunctionTok{c}\NormalTok{(}\DecValTok{0}\NormalTok{, }\DecValTok{1}\NormalTok{))}

\CommentTok{\# (b) Aleatório (Referência)}
\NormalTok{pp\_aleatorio }\OtherTok{\textless{}{-}} \FunctionTok{rpoispp}\NormalTok{(}\AttributeTok{lambda =} \DecValTok{50}\NormalTok{, }\AttributeTok{win =}\NormalTok{ janela)}

\CommentTok{\# (a) Regular}
\NormalTok{pp\_regular }\OtherTok{\textless{}{-}} \FunctionTok{rSSI}\NormalTok{(}\AttributeTok{r =} \FloatTok{0.1}\NormalTok{, }\AttributeTok{n =} \DecValTok{50}\NormalTok{, }\AttributeTok{win =}\NormalTok{ janela)}

\CommentTok{\# (c) Agrupado}
\NormalTok{pp\_agrupado }\OtherTok{\textless{}{-}} \FunctionTok{rThomas}\NormalTok{(}\AttributeTok{kappa =} \DecValTok{5}\NormalTok{, }\AttributeTok{scale =} \FloatTok{0.05}\NormalTok{, }\AttributeTok{mu =} \DecValTok{10}\NormalTok{, }\AttributeTok{win =}\NormalTok{ janela)}

\CommentTok{\# Função auxiliar para calcular F(r) e plotar}
\NormalTok{plot\_f\_function }\OtherTok{\textless{}{-}} \ControlFlowTok{function}\NormalTok{(pp, titulo, tipo\_padrao) \{}
  \CommentTok{\#correção de borda \textquotesingle{}km\textquotesingle{} (Kaplan{-}Meier)}
\NormalTok{  F\_calc }\OtherTok{\textless{}{-}} \FunctionTok{Fest}\NormalTok{(pp, }\AttributeTok{correction =} \StringTok{"km"}\NormalTok{)}
\NormalTok{  df\_F }\OtherTok{\textless{}{-}} \FunctionTok{as.data.frame}\NormalTok{(F\_calc)}
\NormalTok{  df\_F }\OtherTok{\textless{}{-}}\NormalTok{ df\_F[df\_F}\SpecialCharTok{$}\NormalTok{r }\SpecialCharTok{\textless{}=} \FloatTok{0.25}\NormalTok{, ]}
  
  \FunctionTok{ggplot}\NormalTok{(df\_F, }\FunctionTok{aes}\NormalTok{(}\AttributeTok{x =}\NormalTok{ r)) }\SpecialCharTok{+}
    \FunctionTok{geom\_line}\NormalTok{(}\FunctionTok{aes}\NormalTok{(}\AttributeTok{y =}\NormalTok{ theo, }\AttributeTok{linetype =} \StringTok{"Teórico (Poisson)"}\NormalTok{), }\AttributeTok{color =} \StringTok{"red"}\NormalTok{, }\AttributeTok{size =} \FloatTok{0.8}\NormalTok{) }\SpecialCharTok{+}
    \FunctionTok{geom\_line}\NormalTok{(}\FunctionTok{aes}\NormalTok{(}\AttributeTok{y =}\NormalTok{ km, }\AttributeTok{linetype =} \StringTok{"Observado"}\NormalTok{), }\AttributeTok{color =} \StringTok{"black"}\NormalTok{, }\AttributeTok{size =} \DecValTok{1}\NormalTok{) }\SpecialCharTok{+}
    \FunctionTok{scale\_linetype\_manual}\NormalTok{(}\AttributeTok{name =} \StringTok{""}\NormalTok{, }\AttributeTok{values =} \FunctionTok{c}\NormalTok{(}\StringTok{"Observado"} \OtherTok{=} \StringTok{"solid"}\NormalTok{, }\StringTok{"Teórico (Poisson)"} \OtherTok{=} \StringTok{"dashed"}\NormalTok{)) }\SpecialCharTok{+}
    \FunctionTok{labs}\NormalTok{(}\AttributeTok{title =}\NormalTok{ titulo, }
         \AttributeTok{subtitle =}\NormalTok{ tipo\_padrao,}
         \AttributeTok{x =} \StringTok{"Distância (r)"}\NormalTok{, }\AttributeTok{y =} \StringTok{"F(r)"}\NormalTok{) }\SpecialCharTok{+}
    \FunctionTok{theme\_minimal}\NormalTok{() }\SpecialCharTok{+}
    \FunctionTok{theme}\NormalTok{(}\AttributeTok{legend.position =} \StringTok{"bottom"}\NormalTok{,}
          \AttributeTok{plot.title =} \FunctionTok{element\_text}\NormalTok{(}\AttributeTok{size =} \DecValTok{11}\NormalTok{))}
\NormalTok{\}}


\NormalTok{p1 }\OtherTok{\textless{}{-}} \FunctionTok{plot\_f\_function}\NormalTok{(pp\_regular, }\StringTok{"(a) Regular"}\NormalTok{, }\StringTok{"Obs \textgreater{} Teórico)"}\NormalTok{)}
\NormalTok{p2 }\OtherTok{\textless{}{-}} \FunctionTok{plot\_f\_function}\NormalTok{(pp\_aleatorio, }\StringTok{"(b) Aleatório"}\NormalTok{, }\StringTok{"Obs = Teórico (AEC)"}\NormalTok{)}
\NormalTok{p3 }\OtherTok{\textless{}{-}} \FunctionTok{plot\_f\_function}\NormalTok{(pp\_agrupado, }\StringTok{"(c) Agrupado"}\NormalTok{, }\StringTok{"Obs \textless{} Teórico"}\NormalTok{)}


\NormalTok{p1 }\SpecialCharTok{+}\NormalTok{ p2 }\SpecialCharTok{+}\NormalTok{ p3}
\end{Highlighting}
\end{Shaded}

\begin{figure}[H]

\centering{

\pandocbounded{\includegraphics[keepaspectratio]{point_process_files/figure-pdf/fig-funcao-f-1.pdf}}

}

\caption{\label{fig-funcao-f}Comportamento da Função F(r): (a) Regular
(Acima da teórica), (b) Aleatório (Sobreposta) e (c) Agrupado (Abaixo da
teórica).}

\end{figure}%

Considerando a região de estudo \(S\), uma sub-região de uma área
infinita, o estimador \(\hat{F}(r)\), representado pela expressão
Eq.~\ref{eq-16}, apresentará viés. Esse viés ocorre porque a distância
\(d(z_{k}, s_{i})\) de um ponto \(z_{k}\) localizado na borda (limite da
área de estudo) até o evento mais próximo \(s_{i}\) pode parecer maior
do que a distância \(r\), porque na sua determinação, não são
considerados os eventos fora da sub-região em estudo (Scalon 2024).

No entanto, se todos os eventos existentes na sub-região fora da área de
estudo fossem levados em consideração, essa distância seria menor ou
igual a \(r\) (Peter John Diggle 2003; Leininger 2014). Esse viés é
conhecido como o ``efeito da borda''. A correção para esse viés é
denominada ``correção da borda'' (\(e_{k}\)), que corresponde à
distância de um ponto \(z_{k}\) até o limite da área de estudo \(B\)
(borda). Portanto, um estimador que não apresenta viés para essa
situação é,

\begin{equation}\phantomsection\label{eq-16.}{
\hat{F}_{bord}(r)=\frac{\sum_{k} \mathbb{I}\left\{d\left(z_{k}, s_{i}\right) \leq r\right\} \mathbb{I}\left\{e_{k}>r\right\}}{\sum_{k} \mathbb{I}\left\{e_{k}>r\right\}}, \text{ onde } e_{k}=d(z_{k}, S \cap B), \; r \geq 0 ,
}\end{equation}

onde \(d\left(z_{k}, S \cap B\right)\) representa a distância entre
quaisquer pontos amostrais (não eventos) \(z_{k}\) para o evento
\(s_{i}\in s\) mais próximo, considerando a existência dos efeitos borda
na região \(B\) de estudo.

\textbf{Função G (r)}

Diferentemente da função \(F(r)\), que mede a distância entre um ponto e
um evento, a função \(G(r)\) mede a distância entre eventos (
Figura~\ref{fig-conceito-gf} (a)). Seja \(s_{i}\) um evento de um
processo pontual \(s\), a distância até o vizinho mais próximo
\(d_{\min}(s_{i}, s_{i^{'}}) = \min_{i^{'}\neq i}||s_{i^{'}} - s_{i}||\)
pode ser expressa por
\(d_{\min}(s_{i}, s_{i^{'}}) = d(s_{i}, s_{i^{'}} \setminus s_{i})\),
que corresponde à distância mínima de um evento \(s_{i}\) até os outros
eventos \(s_{i^{'}}\) diferentes de \(s_{i}\) (Adrian Baddeley, Rubak, e
Turner 2015; Scalon 2024).

Conforme mencionado por Peter John Diggle (2003), Leininger (2014) e
Scalon (2024), considerando \(n(s)\) como o número de eventos \(s\) em
uma região de estudo \(B\) e
\(d_{\min}(s_{i}, s_{i^{'}}) = d(s_{i}, s_{i^{'}} \setminus s_{i})\)
como a distância do \(i\)-ésimo evento \(s_{i}\) até o evento mais
próximo \(s_{i} \setminus s_{i^{'}}\),\(d_{\min}(s_{i}, s_{i^{'}})\) é
designada como a distância do vizinho mais próximo. Esta medida inclui
distâncias repetidas em pares de vizinhos mais próximos recíprocos
\{\(d(s_{i}, s_{i^{'}} \setminus s_{i})\) e
\(d(s_{i^{'}} \setminus s_{i}, s_{i})\)\}, ou seja,

\begin{equation}\phantomsection\label{eq-17}{
G(r)=P\{d(s_{i}, s_{i^{'}} \setminus s_{i})\leq r \} \quad e \quad \hat{G} (r)=\frac{1}{n(s)} \sum_{i} \mathbb{I}\left\{d\left(s_{i},s_{i^{'}}\setminus s_{i}\right)\leq r\right\}, \quad r \geq 0,
}\end{equation}

onde \(\hat{G}(r)\) é respectivo estimador.

Se a curva \(\hat{G}(r)\) estiver abaixo da curva \(G_{pois}(r)\), ou
seja, \(\hat{G}(r)<G_{pois}(r)\), indica que a distância entre quaisquer
eventos \(s_{i}\) até os eventos \(s_{i^{'}}\) mais próximo é maior do
que seria sob aleatoriedade espacial completa. Nesse caso, há maior
distância de separação entre os eventos e, consequentemente, o padrão
espacial dos eventos é regular ( Figura~\ref{fig-funcao-g} (a)). Se
\(\hat{G}(r)\) é equivalente a
\(G_{pois}(r)\)\((\hat{G}(r) \equiv G_{pois}(r))\), o padrão é
considerado aleatório, também designado aleatoriedade espacial completa
(AEC) ( Figura~\ref{fig-funcao-g} (b)). Se a curva \(\hat{G}(r)\)
estiver acima da curva \(G_{pois}(r)\)\((\hat{G}(r) > G_{pois}(r))\),
significa que a distância entre quaisquer eventos \(s_{i}\) até os
eventos \(s_{i^{'}}\) mais próximo é menor do que seria sob
aleatoriedade espacial completa. Isto sugere que existe menor distância
de separação entre os eventos e, consequentemente, o padrão espacial dos
eventos é agrupado ( Figura~\ref{fig-funcao-g} (c)).

\begin{equation}\phantomsection\label{eq-17.}{
\hat{G}_{bord}(r)=\frac{\sum_{i} \mathbb{I}\left\{e_{i}\geq r \, \cap \, d(s_{i}, s_{i^{'}} \setminus s_{i})\leq r \right\}}{\sum_{i} \mathbb{I}\left\{e_{i}\geq r\right\}}, \, \text{ onde }\; e_{i}= d(s_{i},S \cap B)\quad r \geq 0.
}\end{equation}

\begin{Shaded}
\begin{Highlighting}[]
\FunctionTok{set.seed}\NormalTok{(}\DecValTok{42}\NormalTok{)}
\NormalTok{janela }\OtherTok{\textless{}{-}} \FunctionTok{owin}\NormalTok{(}\FunctionTok{c}\NormalTok{(}\DecValTok{0}\NormalTok{, }\DecValTok{1}\NormalTok{), }\FunctionTok{c}\NormalTok{(}\DecValTok{0}\NormalTok{, }\DecValTok{1}\NormalTok{))}

\CommentTok{\#(b) Aleatório (Referência)}
\NormalTok{pp\_aleatorio }\OtherTok{\textless{}{-}} \FunctionTok{rpoispp}\NormalTok{(}\AttributeTok{lambda =} \DecValTok{60}\NormalTok{, }\AttributeTok{win =}\NormalTok{ janela)}

\CommentTok{\# (a) Regular (Inibição forte {-} Vizinhos afastados)}
\NormalTok{pp\_regular }\OtherTok{\textless{}{-}} \FunctionTok{rSSI}\NormalTok{(}\AttributeTok{r =} \FloatTok{0.08}\NormalTok{, }\AttributeTok{n =} \DecValTok{60}\NormalTok{, }\AttributeTok{win =}\NormalTok{ janela)}

\CommentTok{\# (c) Agrupado (Atração forte {-} Vizinhos muito próximos)}
\NormalTok{pp\_agrupado }\OtherTok{\textless{}{-}} \FunctionTok{rThomas}\NormalTok{(}\AttributeTok{kappa =} \DecValTok{5}\NormalTok{, }\AttributeTok{scale =} \FloatTok{0.03}\NormalTok{, }\AttributeTok{mu =} \DecValTok{12}\NormalTok{, }\AttributeTok{win =}\NormalTok{ janela)}

\CommentTok{\# Função auxiliar para calcular G(r) e plotar}
\NormalTok{plot\_g\_function }\OtherTok{\textless{}{-}} \ControlFlowTok{function}\NormalTok{(pp, titulo, anotacao) \{}
\NormalTok{  G\_calc }\OtherTok{\textless{}{-}} \FunctionTok{Gest}\NormalTok{(pp, }\AttributeTok{correction =} \StringTok{"km"}\NormalTok{)}
\NormalTok{  df\_G }\OtherTok{\textless{}{-}} \FunctionTok{as.data.frame}\NormalTok{(G\_calc)}
  
  \CommentTok{\# Limitar o eixo X para focar onde a ação acontece}
\NormalTok{  df\_G }\OtherTok{\textless{}{-}}\NormalTok{ df\_G[df\_G}\SpecialCharTok{$}\NormalTok{r }\SpecialCharTok{\textless{}=} \FloatTok{0.20}\NormalTok{, ]}
  
  \FunctionTok{ggplot}\NormalTok{(df\_G, }\FunctionTok{aes}\NormalTok{(}\AttributeTok{x =}\NormalTok{ r)) }\SpecialCharTok{+}
    \FunctionTok{geom\_line}\NormalTok{(}\FunctionTok{aes}\NormalTok{(}\AttributeTok{y =}\NormalTok{ theo, }\AttributeTok{linetype =} \StringTok{"Teórico (Poisson)"}\NormalTok{), }\AttributeTok{color =} \StringTok{"red"}\NormalTok{, }\AttributeTok{size =} \FloatTok{0.8}\NormalTok{) }\SpecialCharTok{+}
    \FunctionTok{geom\_line}\NormalTok{(}\FunctionTok{aes}\NormalTok{(}\AttributeTok{y =}\NormalTok{ km, }\AttributeTok{linetype =} \StringTok{"Observado"}\NormalTok{), }\AttributeTok{color =} \StringTok{"black"}\NormalTok{, }\AttributeTok{size =} \DecValTok{1}\NormalTok{) }\SpecialCharTok{+}
    \FunctionTok{scale\_linetype\_manual}\NormalTok{(}\AttributeTok{name =} \StringTok{""}\NormalTok{, }\AttributeTok{values =} \FunctionTok{c}\NormalTok{(}\StringTok{"Observado"} \OtherTok{=} \StringTok{"solid"}\NormalTok{, }\StringTok{"Teórico (Poisson)"} \OtherTok{=} \StringTok{"dashed"}\NormalTok{)) }\SpecialCharTok{+}
    \FunctionTok{labs}\NormalTok{(}\AttributeTok{title =}\NormalTok{ titulo, }
         \AttributeTok{subtitle =}\NormalTok{ anotacao,}
         \AttributeTok{x =} \StringTok{"Distância (r)"}\NormalTok{, }\AttributeTok{y =} \StringTok{"G(r)"}\NormalTok{) }\SpecialCharTok{+}
    \FunctionTok{theme\_minimal}\NormalTok{() }\SpecialCharTok{+}
    \FunctionTok{theme}\NormalTok{(}\AttributeTok{legend.position =} \StringTok{"bottom"}\NormalTok{,}
          \AttributeTok{plot.title =} \FunctionTok{element\_text}\NormalTok{(}\AttributeTok{size =} \DecValTok{11}\NormalTok{))}
\NormalTok{\}}

\NormalTok{p1 }\OtherTok{\textless{}{-}} \FunctionTok{plot\_g\_function}\NormalTok{(pp\_regular, }\StringTok{"(a) Regular"}\NormalTok{, }\StringTok{"Obs \textless{} Teórico"}\NormalTok{)}

\NormalTok{p2 }\OtherTok{\textless{}{-}} \FunctionTok{plot\_g\_function}\NormalTok{(pp\_aleatorio, }\StringTok{"(b) Aleatório"}\NormalTok{, }\StringTok{"Obs = Teórico (AEC)"}\NormalTok{)}

\NormalTok{p3 }\OtherTok{\textless{}{-}} \FunctionTok{plot\_g\_function}\NormalTok{(pp\_agrupado, }\StringTok{"(c) Agrupado"}\NormalTok{, }\StringTok{"Obs \textgreater{} Teórico"}\NormalTok{)}


\NormalTok{p1 }\SpecialCharTok{+}\NormalTok{ p2 }\SpecialCharTok{+}\NormalTok{ p3}
\end{Highlighting}
\end{Shaded}

\begin{figure}[H]

\centering{

\pandocbounded{\includegraphics[keepaspectratio]{point_process_files/figure-pdf/fig-funcao-g-1.pdf}}

}

\caption{\label{fig-funcao-g}Comportamento da Função G(r): (a) Regular
(Abaixo da teórica), (b) Aleatório (Sobreposta) e (c) Agrupado (Acima da
teórica).}

\end{figure}%

Considerando a região de estudo \(s\), uma sub-área de uma infinita área
\(B\), o estimador \(\hat{G}(r)\) representado pela expressão
Eq.~\ref{eq-17}, apresentará viés, pois a distância
\(d(s_{i},s_{i^{'}}\setminus s_{i})\) de um evento \(s_{i}\) que esteja
na borda para o evento \(s_{i^{'}}\) mais próximo pode aparentemente ser
maior que a distância \(r\), enquanto na realidade, se fossem
considerados os eventos que estão fora da área de estudo, esta distância
seria menor ou igual a \(r\) ( Figura~\ref{fig-efeito-borda-conceito}
(c)). Este viés ocorre porque a distância
\(d(s_{i},s_{i^{'}}\setminus s_{i})\) de qualquer evento em \(B\), para
o evento mais próximo que esteja fora da área de estudo não é
considerada, resultando em um viés. Portanto, um estimador que não
apresentara viés para esta situação em concreto é,

\begin{equation}\phantomsection\label{eq-17.}{
\hat{G}_{bord}(r)=\frac{\sum_{i} \mathbb{I}\left\{e_{i}\geq r \, \cap \, d(s_{i}, s_{i^{'}} \setminus s_{i})\leq r \right\}}{\sum_{i} \mathbb{I}\left\{e_{i}\geq r\right\}}, \, \text{ onde }\; e_{i}= d(s_{i},S \cap B)\quad r \geq 0.
}\end{equation}

\begin{Shaded}
\begin{Highlighting}[]
\CommentTok{\# Função auxiliar para desenhar círculos}
\NormalTok{get\_circle }\OtherTok{\textless{}{-}} \ControlFlowTok{function}\NormalTok{(center\_x, center\_y, radius) \{}
\NormalTok{  angle }\OtherTok{\textless{}{-}} \FunctionTok{seq}\NormalTok{(}\DecValTok{0}\NormalTok{, }\DecValTok{2} \SpecialCharTok{*}\NormalTok{ pi, }\AttributeTok{length.out =} \DecValTok{100}\NormalTok{)}
  \FunctionTok{data.frame}\NormalTok{(}\AttributeTok{x =}\NormalTok{ center\_x }\SpecialCharTok{+}\NormalTok{ radius }\SpecialCharTok{*} \FunctionTok{cos}\NormalTok{(angle), }\AttributeTok{y =}\NormalTok{ center\_y }\SpecialCharTok{+}\NormalTok{ radius }\SpecialCharTok{*} \FunctionTok{sin}\NormalTok{(angle))}
\NormalTok{\}}

\CommentTok{\# Função G(r)}
\NormalTok{df\_pts\_g }\OtherTok{\textless{}{-}} \FunctionTok{data.frame}\NormalTok{(}\AttributeTok{x =} \FunctionTok{c}\NormalTok{(}\FloatTok{0.4}\NormalTok{, }\FloatTok{0.7}\NormalTok{), }\AttributeTok{y =} \FunctionTok{c}\NormalTok{(}\FloatTok{0.4}\NormalTok{, }\FloatTok{0.7}\NormalTok{), }\AttributeTok{type =} \FunctionTok{c}\NormalTok{(}\StringTok{"si"}\NormalTok{, }\StringTok{"sj"}\NormalTok{))}
\NormalTok{radius\_g }\OtherTok{\textless{}{-}} \FunctionTok{sqrt}\NormalTok{((}\FloatTok{0.7{-}0.4}\NormalTok{)}\SpecialCharTok{\^{}}\DecValTok{2} \SpecialCharTok{+}\NormalTok{ (}\FloatTok{0.7{-}0.4}\NormalTok{)}\SpecialCharTok{\^{}}\DecValTok{2}\NormalTok{) }\CommentTok{\# Distância euclidiana}
\NormalTok{circle\_g }\OtherTok{\textless{}{-}} \FunctionTok{get\_circle}\NormalTok{(}\FloatTok{0.4}\NormalTok{, }\FloatTok{0.4}\NormalTok{, radius\_g)}

\NormalTok{p1 }\OtherTok{\textless{}{-}} \FunctionTok{ggplot}\NormalTok{() }\SpecialCharTok{+}
  \CommentTok{\# Círculo de busca}
  \FunctionTok{geom\_path}\NormalTok{(}\AttributeTok{data =}\NormalTok{ circle\_g, }\FunctionTok{aes}\NormalTok{(x, y), }\AttributeTok{linetype =} \StringTok{"dashed"}\NormalTok{, }\AttributeTok{color =} \StringTok{"gray60"}\NormalTok{) }\SpecialCharTok{+}
  
  \CommentTok{\# Eventos (Pontos sólidos)}
  \FunctionTok{geom\_point}\NormalTok{(}\AttributeTok{data =}\NormalTok{ df\_pts\_g, }\FunctionTok{aes}\NormalTok{(x, y), }\AttributeTok{size =} \DecValTok{4}\NormalTok{) }\SpecialCharTok{+}
  
  \CommentTok{\# Seta de distância}
  \FunctionTok{geom\_segment}\NormalTok{(}\FunctionTok{aes}\NormalTok{(}\AttributeTok{x =} \FloatTok{0.4}\NormalTok{, }\AttributeTok{y =} \FloatTok{0.4}\NormalTok{, }\AttributeTok{xend =} \FloatTok{0.7}\NormalTok{, }\AttributeTok{yend =} \FloatTok{0.7}\NormalTok{), }
               \AttributeTok{arrow =} \FunctionTok{arrow}\NormalTok{(}\AttributeTok{length =} \FunctionTok{unit}\NormalTok{(}\FloatTok{0.3}\NormalTok{, }\StringTok{"cm"}\NormalTok{)), }\AttributeTok{color =} \StringTok{"blue"}\NormalTok{) }\SpecialCharTok{+}
  
  \CommentTok{\# Labels}
  \FunctionTok{geom\_text}\NormalTok{(}\FunctionTok{aes}\NormalTok{(}\AttributeTok{x =} \FloatTok{0.4}\NormalTok{, }\AttributeTok{y =} \FloatTok{0.4}\NormalTok{, }\AttributeTok{label =} \StringTok{"s[i]"}\NormalTok{), }\AttributeTok{parse =} \ConstantTok{TRUE}\NormalTok{, }\AttributeTok{vjust =} \DecValTok{2}\NormalTok{, }\AttributeTok{size =} \DecValTok{6}\NormalTok{) }\SpecialCharTok{+}
  \FunctionTok{geom\_text}\NormalTok{(}\FunctionTok{aes}\NormalTok{(}\AttributeTok{x =} \FloatTok{0.7}\NormalTok{, }\AttributeTok{y =} \FloatTok{0.7}\NormalTok{, }\AttributeTok{label =} \StringTok{"s[j]"}\NormalTok{), }\AttributeTok{parse =} \ConstantTok{TRUE}\NormalTok{, }\AttributeTok{vjust =} \SpecialCharTok{{-}}\DecValTok{1}\NormalTok{, }\AttributeTok{size =} \DecValTok{6}\NormalTok{) }\SpecialCharTok{+}
  \FunctionTok{annotate}\NormalTok{(}\StringTok{"text"}\NormalTok{, }\AttributeTok{x =} \FloatTok{0.5}\NormalTok{, }\AttributeTok{y =} \FloatTok{0.6}\NormalTok{, }\AttributeTok{label =} \StringTok{"d(s[i], s[j])"}\NormalTok{, }\AttributeTok{parse =} \ConstantTok{TRUE}\NormalTok{, }\AttributeTok{color =} \StringTok{"blue"}\NormalTok{) }\SpecialCharTok{+}
  
  \FunctionTok{coord\_fixed}\NormalTok{(}\AttributeTok{xlim =} \FunctionTok{c}\NormalTok{(}\DecValTok{0}\NormalTok{, }\DecValTok{1}\NormalTok{), }\AttributeTok{ylim =} \FunctionTok{c}\NormalTok{(}\DecValTok{0}\NormalTok{, }\DecValTok{1}\NormalTok{)) }\SpecialCharTok{+}
  \FunctionTok{theme\_void}\NormalTok{() }\SpecialCharTok{+}
  \FunctionTok{geom\_rect}\NormalTok{(}\FunctionTok{aes}\NormalTok{(}\AttributeTok{xmin=}\DecValTok{0}\NormalTok{, }\AttributeTok{xmax=}\DecValTok{1}\NormalTok{, }\AttributeTok{ymin=}\DecValTok{0}\NormalTok{, }\AttributeTok{ymax=}\DecValTok{1}\NormalTok{), }\AttributeTok{fill=}\ConstantTok{NA}\NormalTok{, }\AttributeTok{color=}\StringTok{"black"}\NormalTok{) }\SpecialCharTok{+}
  \FunctionTok{labs}\NormalTok{(}\AttributeTok{title =} \StringTok{"(a) Função G(r)}\SpecialCharTok{\textbackslash{}n}\StringTok{(Evento para Evento)"}\NormalTok{) }\SpecialCharTok{+}
  \FunctionTok{theme}\NormalTok{(}\AttributeTok{plot.title =} \FunctionTok{element\_text}\NormalTok{(}\AttributeTok{hjust =} \FloatTok{0.5}\NormalTok{))}

\CommentTok{\# Função F(r)}
\NormalTok{df\_evt\_f }\OtherTok{\textless{}{-}} \FunctionTok{data.frame}\NormalTok{(}\AttributeTok{x =} \FloatTok{0.7}\NormalTok{, }\AttributeTok{y =} \FloatTok{0.6}\NormalTok{) }\CommentTok{\# Evento}
\NormalTok{df\_pt\_z }\OtherTok{\textless{}{-}} \FunctionTok{data.frame}\NormalTok{(}\AttributeTok{x =} \FloatTok{0.3}\NormalTok{, }\AttributeTok{y =} \FloatTok{0.4}\NormalTok{)  }\CommentTok{\# Ponto Z}
\NormalTok{radius\_f }\OtherTok{\textless{}{-}} \FunctionTok{sqrt}\NormalTok{((}\FloatTok{0.7{-}0.3}\NormalTok{)}\SpecialCharTok{\^{}}\DecValTok{2} \SpecialCharTok{+}\NormalTok{ (}\FloatTok{0.6{-}0.4}\NormalTok{)}\SpecialCharTok{\^{}}\DecValTok{2}\NormalTok{)}
\NormalTok{circle\_f }\OtherTok{\textless{}{-}} \FunctionTok{get\_circle}\NormalTok{(}\FloatTok{0.3}\NormalTok{, }\FloatTok{0.4}\NormalTok{, radius\_f)}

\NormalTok{p2 }\OtherTok{\textless{}{-}} \FunctionTok{ggplot}\NormalTok{() }\SpecialCharTok{+}
  \CommentTok{\# Círculo de busca}
  \FunctionTok{geom\_path}\NormalTok{(}\AttributeTok{data =}\NormalTok{ circle\_f, }\FunctionTok{aes}\NormalTok{(x, y), }\AttributeTok{linetype =} \StringTok{"dashed"}\NormalTok{, }\AttributeTok{color =} \StringTok{"gray60"}\NormalTok{) }\SpecialCharTok{+}
  
  \CommentTok{\# Evento (Ponto sólido)}
  \FunctionTok{geom\_point}\NormalTok{(}\AttributeTok{data =}\NormalTok{ df\_evt\_f, }\FunctionTok{aes}\NormalTok{(x, y), }\AttributeTok{size =} \DecValTok{4}\NormalTok{) }\SpecialCharTok{+}
  
  \CommentTok{\# Ponto Arbitrário Z (Cruz)}
  \FunctionTok{geom\_point}\NormalTok{(}\AttributeTok{data =}\NormalTok{ df\_pt\_z, }\FunctionTok{aes}\NormalTok{(x, y), }\AttributeTok{shape =} \DecValTok{4}\NormalTok{, }\AttributeTok{size =} \DecValTok{5}\NormalTok{, }\AttributeTok{stroke =} \DecValTok{2}\NormalTok{) }\SpecialCharTok{+}
  
  \CommentTok{\# Seta de distância}
  \FunctionTok{geom\_segment}\NormalTok{(}\FunctionTok{aes}\NormalTok{(}\AttributeTok{x =} \FloatTok{0.3}\NormalTok{, }\AttributeTok{y =} \FloatTok{0.4}\NormalTok{, }\AttributeTok{xend =} \FloatTok{0.7}\NormalTok{, }\AttributeTok{yend =} \FloatTok{0.6}\NormalTok{), }
               \AttributeTok{arrow =} \FunctionTok{arrow}\NormalTok{(}\AttributeTok{length =} \FunctionTok{unit}\NormalTok{(}\FloatTok{0.3}\NormalTok{, }\StringTok{"cm"}\NormalTok{)), }\AttributeTok{color =} \StringTok{"red"}\NormalTok{) }\SpecialCharTok{+}
  
  \CommentTok{\# Labels}
  \FunctionTok{geom\_text}\NormalTok{(}\FunctionTok{aes}\NormalTok{(}\AttributeTok{x =} \FloatTok{0.7}\NormalTok{, }\AttributeTok{y =} \FloatTok{0.6}\NormalTok{, }\AttributeTok{label =} \StringTok{"s[i]"}\NormalTok{), }\AttributeTok{parse =} \ConstantTok{TRUE}\NormalTok{, }\AttributeTok{vjust =} \SpecialCharTok{{-}}\DecValTok{1}\NormalTok{, }\AttributeTok{size =} \DecValTok{6}\NormalTok{) }\SpecialCharTok{+}
  \FunctionTok{geom\_text}\NormalTok{(}\FunctionTok{aes}\NormalTok{(}\AttributeTok{x =} \FloatTok{0.3}\NormalTok{, }\AttributeTok{y =} \FloatTok{0.4}\NormalTok{, }\AttributeTok{label =} \StringTok{"z"}\NormalTok{), }\AttributeTok{vjust =} \DecValTok{2}\NormalTok{, }\AttributeTok{size =} \DecValTok{6}\NormalTok{, }\AttributeTok{fontface =} \StringTok{"italic"}\NormalTok{) }\SpecialCharTok{+}
  \FunctionTok{annotate}\NormalTok{(}\StringTok{"text"}\NormalTok{, }\AttributeTok{x =} \FloatTok{0.5}\NormalTok{, }\AttributeTok{y =} \FloatTok{0.55}\NormalTok{, }\AttributeTok{label =} \StringTok{"d(z, s[i])"}\NormalTok{, }\AttributeTok{parse =} \ConstantTok{TRUE}\NormalTok{, }\AttributeTok{color =} \StringTok{"red"}\NormalTok{) }\SpecialCharTok{+}
  
  \FunctionTok{coord\_fixed}\NormalTok{(}\AttributeTok{xlim =} \FunctionTok{c}\NormalTok{(}\DecValTok{0}\NormalTok{, }\DecValTok{1}\NormalTok{), }\AttributeTok{ylim =} \FunctionTok{c}\NormalTok{(}\DecValTok{0}\NormalTok{, }\DecValTok{1}\NormalTok{)) }\SpecialCharTok{+}
  \FunctionTok{theme\_void}\NormalTok{() }\SpecialCharTok{+}
  \FunctionTok{geom\_rect}\NormalTok{(}\FunctionTok{aes}\NormalTok{(}\AttributeTok{xmin=}\DecValTok{0}\NormalTok{, }\AttributeTok{xmax=}\DecValTok{1}\NormalTok{, }\AttributeTok{ymin=}\DecValTok{0}\NormalTok{, }\AttributeTok{ymax=}\DecValTok{1}\NormalTok{), }\AttributeTok{fill=}\ConstantTok{NA}\NormalTok{, }\AttributeTok{color=}\StringTok{"black"}\NormalTok{) }\SpecialCharTok{+}
  \FunctionTok{labs}\NormalTok{(}\AttributeTok{title =} \StringTok{"(b) Função F(r)}\SpecialCharTok{\textbackslash{}n}\StringTok{(Ponto Vazio para Evento)"}\NormalTok{) }\SpecialCharTok{+}
  \FunctionTok{theme}\NormalTok{(}\AttributeTok{plot.title =} \FunctionTok{element\_text}\NormalTok{(}\AttributeTok{hjust =} \FloatTok{0.5}\NormalTok{))}

\NormalTok{p1 }\SpecialCharTok{+}\NormalTok{ p2}
\end{Highlighting}
\end{Shaded}

\begin{figure}[H]

\centering{

\pandocbounded{\includegraphics[keepaspectratio]{point_process_files/figure-pdf/fig-conceito-gf-1.pdf}}

}

\caption{\label{fig-conceito-gf}Esquema Conceitual: (a) Função G mede a
distância entre eventos; (b) Função F mede a distância de um ponto vazio
(z) até um evento.}

\end{figure}%

\begin{Shaded}
\begin{Highlighting}[]
\NormalTok{get\_circle }\OtherTok{\textless{}{-}} \ControlFlowTok{function}\NormalTok{(center\_x, center\_y, radius, }\AttributeTok{n\_points =} \DecValTok{100}\NormalTok{) \{}
\NormalTok{  angle }\OtherTok{\textless{}{-}} \FunctionTok{seq}\NormalTok{(}\DecValTok{0}\NormalTok{, }\DecValTok{2} \SpecialCharTok{*}\NormalTok{ pi, }\AttributeTok{length.out =}\NormalTok{ n\_points)}
  \FunctionTok{data.frame}\NormalTok{(}
    \AttributeTok{x =}\NormalTok{ center\_x }\SpecialCharTok{+}\NormalTok{ radius }\SpecialCharTok{*} \FunctionTok{cos}\NormalTok{(angle),}
    \AttributeTok{y =}\NormalTok{ center\_y }\SpecialCharTok{+}\NormalTok{ radius }\SpecialCharTok{*} \FunctionTok{sin}\NormalTok{(angle)}
\NormalTok{  )}
\NormalTok{\}}

\CommentTok{\#}
\NormalTok{df\_g\_pts }\OtherTok{\textless{}{-}} \FunctionTok{data.frame}\NormalTok{(}\AttributeTok{x =} \FunctionTok{c}\NormalTok{(}\FloatTok{0.3}\NormalTok{, }\FloatTok{0.7}\NormalTok{), }\AttributeTok{y =} \FunctionTok{c}\NormalTok{(}\FloatTok{0.5}\NormalTok{, }\FloatTok{0.7}\NormalTok{), }\AttributeTok{label =} \FunctionTok{c}\NormalTok{(}\StringTok{"si"}\NormalTok{, }\StringTok{"sj"}\NormalTok{))}
\NormalTok{circle\_g }\OtherTok{\textless{}{-}} \FunctionTok{get\_circle}\NormalTok{(}\FloatTok{0.3}\NormalTok{, }\FloatTok{0.5}\NormalTok{, }\AttributeTok{radius =} \FunctionTok{sqrt}\NormalTok{((}\FloatTok{0.7{-}0.3}\NormalTok{)}\SpecialCharTok{\^{}}\DecValTok{2} \SpecialCharTok{+}\NormalTok{ (}\FloatTok{0.7{-}0.5}\NormalTok{)}\SpecialCharTok{\^{}}\DecValTok{2}\NormalTok{))}

\NormalTok{p1 }\OtherTok{\textless{}{-}} \FunctionTok{ggplot}\NormalTok{() }\SpecialCharTok{+}
  \FunctionTok{geom\_path}\NormalTok{(}\AttributeTok{data =}\NormalTok{ circle\_g, }\FunctionTok{aes}\NormalTok{(x, y), }\AttributeTok{linetype =} \StringTok{"dashed"}\NormalTok{, }\AttributeTok{color =} \StringTok{"grey50"}\NormalTok{) }\SpecialCharTok{+}
  \FunctionTok{geom\_point}\NormalTok{(}\AttributeTok{data =}\NormalTok{ df\_g\_pts, }\FunctionTok{aes}\NormalTok{(x, y), }\AttributeTok{size =} \DecValTok{3}\NormalTok{) }\SpecialCharTok{+}
  \FunctionTok{geom\_segment}\NormalTok{(}\FunctionTok{aes}\NormalTok{(}\AttributeTok{x =} \FloatTok{0.3}\NormalTok{, }\AttributeTok{y =} \FloatTok{0.5}\NormalTok{, }\AttributeTok{xend =} \FloatTok{0.7}\NormalTok{, }\AttributeTok{yend =} \FloatTok{0.7}\NormalTok{), }\AttributeTok{arrow =} \FunctionTok{arrow}\NormalTok{(}\AttributeTok{length =} \FunctionTok{unit}\NormalTok{(}\FloatTok{0.2}\NormalTok{, }\StringTok{"cm"}\NormalTok{))) }\SpecialCharTok{+}
  \FunctionTok{geom\_text}\NormalTok{(}\AttributeTok{data =}\NormalTok{ df\_g\_pts, }\FunctionTok{aes}\NormalTok{(x, y, }\AttributeTok{label =}\NormalTok{ label), }\AttributeTok{vjust =} \SpecialCharTok{{-}}\FloatTok{1.5}\NormalTok{, }\AttributeTok{size =} \DecValTok{5}\NormalTok{) }\SpecialCharTok{+}
  \FunctionTok{coord\_fixed}\NormalTok{(}\AttributeTok{xlim =} \FunctionTok{c}\NormalTok{(}\DecValTok{0}\NormalTok{, }\DecValTok{1}\NormalTok{), }\AttributeTok{ylim =} \FunctionTok{c}\NormalTok{(}\DecValTok{0}\NormalTok{, }\DecValTok{1}\NormalTok{)) }\SpecialCharTok{+}
  \FunctionTok{theme\_void}\NormalTok{() }\SpecialCharTok{+}
  \FunctionTok{geom\_rect}\NormalTok{(}\FunctionTok{aes}\NormalTok{(}\AttributeTok{xmin=}\DecValTok{0}\NormalTok{, }\AttributeTok{xmax=}\DecValTok{1}\NormalTok{, }\AttributeTok{ymin=}\DecValTok{0}\NormalTok{, }\AttributeTok{ymax=}\DecValTok{1}\NormalTok{), }\AttributeTok{fill=}\ConstantTok{NA}\NormalTok{, }\AttributeTok{color=}\StringTok{"black"}\NormalTok{) }\SpecialCharTok{+}
  \FunctionTok{labs}\NormalTok{(}\AttributeTok{title =} \StringTok{"(a) Função G(r)}\SpecialCharTok{\textbackslash{}n}\StringTok{(Evento para Evento)"}\NormalTok{) }\SpecialCharTok{+}
  \FunctionTok{theme}\NormalTok{(}\AttributeTok{plot.title =} \FunctionTok{element\_text}\NormalTok{(}\AttributeTok{hjust =} \FloatTok{0.5}\NormalTok{))}

\CommentTok{\# {-}{-}{-} Plot B: Função F (Ponto Vazio para Evento) {-}{-}{-}}
\NormalTok{df\_f\_evt }\OtherTok{\textless{}{-}} \FunctionTok{data.frame}\NormalTok{(}\AttributeTok{x =} \FloatTok{0.6}\NormalTok{, }\AttributeTok{y =} \FloatTok{0.6}\NormalTok{, }\AttributeTok{label =} \StringTok{"si"}\NormalTok{)}
\NormalTok{df\_f\_pt }\OtherTok{\textless{}{-}} \FunctionTok{data.frame}\NormalTok{(}\AttributeTok{x =} \FloatTok{0.4}\NormalTok{, }\AttributeTok{y =} \FloatTok{0.4}\NormalTok{, }\AttributeTok{label =} \StringTok{"z"}\NormalTok{)}
\NormalTok{circle\_f }\OtherTok{\textless{}{-}} \FunctionTok{get\_circle}\NormalTok{(}\FloatTok{0.4}\NormalTok{, }\FloatTok{0.4}\NormalTok{, }\AttributeTok{radius =} \FunctionTok{sqrt}\NormalTok{((}\FloatTok{0.6{-}0.4}\NormalTok{)}\SpecialCharTok{\^{}}\DecValTok{2} \SpecialCharTok{+}\NormalTok{ (}\FloatTok{0.6{-}0.4}\NormalTok{)}\SpecialCharTok{\^{}}\DecValTok{2}\NormalTok{))}

\NormalTok{p2 }\OtherTok{\textless{}{-}} \FunctionTok{ggplot}\NormalTok{() }\SpecialCharTok{+}
  \FunctionTok{geom\_path}\NormalTok{(}\AttributeTok{data =}\NormalTok{ circle\_f, }\FunctionTok{aes}\NormalTok{(x, y), }\AttributeTok{linetype =} \StringTok{"dashed"}\NormalTok{, }\AttributeTok{color =} \StringTok{"grey50"}\NormalTok{) }\SpecialCharTok{+}
  \FunctionTok{geom\_point}\NormalTok{(}\AttributeTok{data =}\NormalTok{ df\_f\_evt, }\FunctionTok{aes}\NormalTok{(x, y), }\AttributeTok{size =} \DecValTok{3}\NormalTok{) }\SpecialCharTok{+} \CommentTok{\# Evento (bolinha)}
  \FunctionTok{geom\_point}\NormalTok{(}\AttributeTok{data =}\NormalTok{ df\_f\_pt, }\FunctionTok{aes}\NormalTok{(x, y), }\AttributeTok{shape =} \DecValTok{4}\NormalTok{, }\AttributeTok{size =} \DecValTok{4}\NormalTok{, }\AttributeTok{stroke =} \DecValTok{2}\NormalTok{) }\SpecialCharTok{+} \CommentTok{\# Ponto Z (X)}
  \FunctionTok{geom\_segment}\NormalTok{(}\FunctionTok{aes}\NormalTok{(}\AttributeTok{x =} \FloatTok{0.4}\NormalTok{, }\AttributeTok{y =} \FloatTok{0.4}\NormalTok{, }\AttributeTok{xend =} \FloatTok{0.6}\NormalTok{, }\AttributeTok{yend =} \FloatTok{0.6}\NormalTok{), }\AttributeTok{arrow =} \FunctionTok{arrow}\NormalTok{(}\AttributeTok{length =} \FunctionTok{unit}\NormalTok{(}\FloatTok{0.2}\NormalTok{, }\StringTok{"cm"}\NormalTok{))) }\SpecialCharTok{+}
  \FunctionTok{geom\_text}\NormalTok{(}\FunctionTok{aes}\NormalTok{(}\AttributeTok{x =} \FloatTok{0.6}\NormalTok{, }\AttributeTok{y =} \FloatTok{0.6}\NormalTok{, }\AttributeTok{label =} \StringTok{"si"}\NormalTok{), }\AttributeTok{vjust =} \SpecialCharTok{{-}}\FloatTok{1.5}\NormalTok{, }\AttributeTok{size =} \DecValTok{5}\NormalTok{) }\SpecialCharTok{+}
  \FunctionTok{geom\_text}\NormalTok{(}\FunctionTok{aes}\NormalTok{(}\AttributeTok{x =} \FloatTok{0.4}\NormalTok{, }\AttributeTok{y =} \FloatTok{0.4}\NormalTok{, }\AttributeTok{label =} \StringTok{"z"}\NormalTok{), }\AttributeTok{vjust =} \FloatTok{1.5}\NormalTok{, }\AttributeTok{size =} \DecValTok{5}\NormalTok{) }\SpecialCharTok{+}
  \FunctionTok{coord\_fixed}\NormalTok{(}\AttributeTok{xlim =} \FunctionTok{c}\NormalTok{(}\DecValTok{0}\NormalTok{, }\DecValTok{1}\NormalTok{), }\AttributeTok{ylim =} \FunctionTok{c}\NormalTok{(}\DecValTok{0}\NormalTok{, }\DecValTok{1}\NormalTok{)) }\SpecialCharTok{+}
  \FunctionTok{theme\_void}\NormalTok{() }\SpecialCharTok{+}
  \FunctionTok{geom\_rect}\NormalTok{(}\FunctionTok{aes}\NormalTok{(}\AttributeTok{xmin=}\DecValTok{0}\NormalTok{, }\AttributeTok{xmax=}\DecValTok{1}\NormalTok{, }\AttributeTok{ymin=}\DecValTok{0}\NormalTok{, }\AttributeTok{ymax=}\DecValTok{1}\NormalTok{), }\AttributeTok{fill=}\ConstantTok{NA}\NormalTok{, }\AttributeTok{color=}\StringTok{"black"}\NormalTok{) }\SpecialCharTok{+}
  \FunctionTok{labs}\NormalTok{(}\AttributeTok{title =} \StringTok{"(b) Função F(r)}\SpecialCharTok{\textbackslash{}n}\StringTok{(Ponto para Evento)"}\NormalTok{) }\SpecialCharTok{+}
  \FunctionTok{theme}\NormalTok{(}\AttributeTok{plot.title =} \FunctionTok{element\_text}\NormalTok{(}\AttributeTok{hjust =} \FloatTok{0.5}\NormalTok{))}

\CommentTok{\#}
\NormalTok{center\_x }\OtherTok{\textless{}{-}} \FloatTok{0.8}\NormalTok{; center\_y }\OtherTok{\textless{}{-}} \FloatTok{0.5}
\NormalTok{obs\_neigh\_x }\OtherTok{\textless{}{-}} \FloatTok{0.4}\NormalTok{; obs\_neigh\_y }\OtherTok{\textless{}{-}} \FloatTok{0.5} \CommentTok{\# Distância 0.4}
\NormalTok{radius }\OtherTok{\textless{}{-}} \FloatTok{0.4}
\NormalTok{circle\_edge }\OtherTok{\textless{}{-}} \FunctionTok{get\_circle}\NormalTok{(center\_x, center\_y, radius)}

\NormalTok{p3 }\OtherTok{\textless{}{-}} \FunctionTok{ggplot}\NormalTok{() }\SpecialCharTok{+}
  \CommentTok{\# Área "Fora" do estudo}
  \FunctionTok{geom\_rect}\NormalTok{(}\FunctionTok{aes}\NormalTok{(}\AttributeTok{xmin=}\DecValTok{1}\NormalTok{, }\AttributeTok{xmax=}\FloatTok{1.3}\NormalTok{, }\AttributeTok{ymin=}\DecValTok{0}\NormalTok{, }\AttributeTok{ymax=}\DecValTok{1}\NormalTok{), }\AttributeTok{fill=}\StringTok{"gray90"}\NormalTok{, }\AttributeTok{alpha=}\FloatTok{0.5}\NormalTok{) }\SpecialCharTok{+}
  \FunctionTok{geom\_rect}\NormalTok{(}\FunctionTok{aes}\NormalTok{(}\AttributeTok{xmin=}\DecValTok{0}\NormalTok{, }\AttributeTok{xmax=}\DecValTok{1}\NormalTok{, }\AttributeTok{ymin=}\DecValTok{0}\NormalTok{, }\AttributeTok{ymax=}\DecValTok{1}\NormalTok{), }\AttributeTok{fill=}\ConstantTok{NA}\NormalTok{, }\AttributeTok{color=}\StringTok{"black"}\NormalTok{, }\AttributeTok{size=}\DecValTok{1}\NormalTok{) }\SpecialCharTok{+}
  
  \CommentTok{\# Círculo de busca que "vaza"}
  \FunctionTok{geom\_path}\NormalTok{(}\AttributeTok{data =}\NormalTok{ circle\_edge, }\FunctionTok{aes}\NormalTok{(x, y), }\AttributeTok{linetype =} \StringTok{"dotted"}\NormalTok{, }\AttributeTok{color =} \StringTok{"red"}\NormalTok{, }\AttributeTok{size=}\DecValTok{1}\NormalTok{) }\SpecialCharTok{+}
  
  \CommentTok{\# Evento Focal}
  \FunctionTok{geom\_point}\NormalTok{(}\FunctionTok{aes}\NormalTok{(}\AttributeTok{x=}\NormalTok{center\_x, }\AttributeTok{y=}\NormalTok{center\_y), }\AttributeTok{size=}\DecValTok{3}\NormalTok{) }\SpecialCharTok{+}
  \FunctionTok{geom\_text}\NormalTok{(}\FunctionTok{aes}\NormalTok{(}\AttributeTok{x=}\NormalTok{center\_x, }\AttributeTok{y=}\NormalTok{center\_y, }\AttributeTok{label=}\StringTok{"si"}\NormalTok{), }\AttributeTok{vjust=}\SpecialCharTok{{-}}\FloatTok{1.5}\NormalTok{, }\AttributeTok{size=}\DecValTok{5}\NormalTok{) }\SpecialCharTok{+}
  
  \CommentTok{\# Vizinho Observado (Longe)}
  \FunctionTok{geom\_point}\NormalTok{(}\FunctionTok{aes}\NormalTok{(}\AttributeTok{x=}\NormalTok{obs\_neigh\_x, }\AttributeTok{y=}\NormalTok{obs\_neigh\_y), }\AttributeTok{size=}\DecValTok{3}\NormalTok{) }\SpecialCharTok{+}
  \FunctionTok{geom\_text}\NormalTok{(}\FunctionTok{aes}\NormalTok{(}\AttributeTok{x=}\NormalTok{obs\_neigh\_x, }\AttributeTok{y=}\NormalTok{obs\_neigh\_y, }\AttributeTok{label=}\StringTok{"sj}\SpecialCharTok{\textbackslash{}n}\StringTok{(Observado)"}\NormalTok{), }\AttributeTok{vjust=}\DecValTok{2}\NormalTok{, }\AttributeTok{size=}\DecValTok{3}\NormalTok{) }\SpecialCharTok{+}
  
  \FunctionTok{geom\_point}\NormalTok{(}\FunctionTok{aes}\NormalTok{(}\AttributeTok{x=}\FloatTok{1.1}\NormalTok{, }\AttributeTok{y=}\FloatTok{0.55}\NormalTok{), }\AttributeTok{shape=}\DecValTok{1}\NormalTok{, }\AttributeTok{color=}\StringTok{"red"}\NormalTok{, }\AttributeTok{size=}\DecValTok{3}\NormalTok{) }\SpecialCharTok{+}
  \FunctionTok{geom\_text}\NormalTok{(}\FunctionTok{aes}\NormalTok{(}\AttributeTok{x=}\FloatTok{1.1}\NormalTok{, }\AttributeTok{y=}\FloatTok{0.55}\NormalTok{, }\AttributeTok{label=}\StringTok{"?}\SpecialCharTok{\textbackslash{}n}\StringTok{(Real?)"}\NormalTok{), }\AttributeTok{vjust=}\FloatTok{1.5}\NormalTok{, }\AttributeTok{color=}\StringTok{"red"}\NormalTok{, }\AttributeTok{size=}\DecValTok{3}\NormalTok{) }\SpecialCharTok{+}
  
  \CommentTok{\# Setas indicando o problema}
  \FunctionTok{geom\_segment}\NormalTok{(}\FunctionTok{aes}\NormalTok{(}\AttributeTok{x=}\NormalTok{center\_x, }\AttributeTok{y=}\NormalTok{center\_y, }\AttributeTok{xend=}\NormalTok{obs\_neigh\_x, }\AttributeTok{yend=}\NormalTok{obs\_neigh\_y), }\AttributeTok{arrow=}\FunctionTok{arrow}\NormalTok{(}\AttributeTok{length=}\FunctionTok{unit}\NormalTok{(}\FloatTok{0.2}\NormalTok{,}\StringTok{"cm"}\NormalTok{))) }\SpecialCharTok{+}
  \FunctionTok{geom\_segment}\NormalTok{(}\FunctionTok{aes}\NormalTok{(}\AttributeTok{x=}\NormalTok{center\_x, }\AttributeTok{y=}\NormalTok{center\_y, }\AttributeTok{xend=}\DecValTok{1}\NormalTok{, }\AttributeTok{yend=}\FloatTok{0.5}\NormalTok{), }\AttributeTok{linetype=}\StringTok{"solid"}\NormalTok{, }\AttributeTok{color=}\StringTok{"blue"}\NormalTok{, }\AttributeTok{size=}\DecValTok{1}\NormalTok{) }\SpecialCharTok{+}
  
  \FunctionTok{annotate}\NormalTok{(}\StringTok{"text"}\NormalTok{, }\AttributeTok{x=}\FloatTok{0.9}\NormalTok{, }\AttributeTok{y=}\FloatTok{0.45}\NormalTok{, }\AttributeTok{label=}\StringTok{"Dist. Borda"}\NormalTok{, }\AttributeTok{color=}\StringTok{"blue"}\NormalTok{, }\AttributeTok{size=}\DecValTok{3}\NormalTok{) }\SpecialCharTok{+}
  
  \FunctionTok{coord\_fixed}\NormalTok{(}\AttributeTok{xlim =} \FunctionTok{c}\NormalTok{(}\DecValTok{0}\NormalTok{, }\FloatTok{1.3}\NormalTok{), }\AttributeTok{ylim =} \FunctionTok{c}\NormalTok{(}\DecValTok{0}\NormalTok{, }\DecValTok{1}\NormalTok{)) }\SpecialCharTok{+}
  \FunctionTok{theme\_void}\NormalTok{() }\SpecialCharTok{+}
  \FunctionTok{labs}\NormalTok{(}\AttributeTok{title =} \StringTok{"(c) Efeito de Borda}\SpecialCharTok{\textbackslash{}n}\StringTok{(Círculo vaza a fronteira)"}\NormalTok{) }\SpecialCharTok{+}
  \FunctionTok{theme}\NormalTok{(}\AttributeTok{plot.title =} \FunctionTok{element\_text}\NormalTok{(}\AttributeTok{hjust =} \FloatTok{0.5}\NormalTok{, }\AttributeTok{color=}\StringTok{"red"}\NormalTok{))}

\NormalTok{p1 }\SpecialCharTok{+}\NormalTok{ p2 }\SpecialCharTok{+}\NormalTok{ p3}
\end{Highlighting}
\end{Shaded}

\begin{figure}[H]

\centering{

\pandocbounded{\includegraphics[keepaspectratio]{point_process_files/figure-pdf/fig-efeito-borda-conceito-1.pdf}}

}

\caption{\label{fig-efeito-borda-conceito}Esquema Conceitual: (a) Função
G (Evento-Evento), (b) Função F (Ponto-Evento) e (c) O Problema do
Efeito de Borda (Censura).}

\end{figure}%

\textbf{Função J(r)}

A função \(J(r)\) é uma função que não apresenta viés e resulta da razão
entre as funções \(G(r)\) e \(F(r)\) acima descritas, ou seja,

\begin{equation}\phantomsection\label{eq-18}{
J(r) = \frac{1-G(r)}{1-F(r)} \text{ e } \hat{J}(r) =\frac{1-\hat{G}(r)}{1-\hat{F}(r)}= \frac{1-\frac{1}{n(s)} \sum_{i}  \mathbb{I}\left\{d\left(s_{i},s_{i^{'}}\setminus s_{i}\right)\right\}\leq r)}{1-\frac{1}{J} \sum_{j=1}^{J} \mathbb{I}\left\{d(z_{j}, s_{i^{'}})\leq r\right\}},\quad \, F(r), \hat{F} (r)< 1,
}\end{equation}

onde \(\hat{J}(r)\) é o respectivo estimador (Van Lieshout e Baddeley
1996; Scalon 2024).

Como
\(F_{pois}(r)= G_{pois}(r) = 1- e^{-\pi \lambda r^{2}}\),\(J_{pois} (r) = \frac{1-G_{pois}(r)}{1-F_{pois}(r)}= \frac{1-(1- e^{-\pi \lambda r^{2}})}{1-(1- e^{-\pi \lambda r^{2}})}=1\),
consequentemente, se \(\hat{J}(r) > 1\), o processo pontual apresenta um
padrão regular ( Figura~\ref{fig-funcao-j} (a)). Se
\(\hat{J}(r)\equiv 1\) o processo pontual apresenta um padrão aleatório
( Figura~\ref{fig-funcao-j} (b)). Se \(\hat{J}(r)<1\) o processo pontual
apresenta um padrão agrupado ( Figura~\ref{fig-funcao-j} (c)).

\begin{Shaded}
\begin{Highlighting}[]
\FunctionTok{set.seed}\NormalTok{(}\DecValTok{888}\NormalTok{) }
\NormalTok{janela }\OtherTok{\textless{}{-}} \FunctionTok{owin}\NormalTok{(}\FunctionTok{c}\NormalTok{(}\DecValTok{0}\NormalTok{, }\DecValTok{1}\NormalTok{), }\FunctionTok{c}\NormalTok{(}\DecValTok{0}\NormalTok{, }\DecValTok{1}\NormalTok{))}

\CommentTok{\#  (b) Aleatório (Referência {-} Poisson)}
\NormalTok{pp\_aleatorio }\OtherTok{\textless{}{-}} \FunctionTok{rpoispp}\NormalTok{(}\AttributeTok{lambda =} \DecValTok{50}\NormalTok{, }\AttributeTok{win =}\NormalTok{ janela)}

\CommentTok{\# (a) Regular (Inibição {-} Hard Core)}
\NormalTok{pp\_regular }\OtherTok{\textless{}{-}} \FunctionTok{rSSI}\NormalTok{(}\AttributeTok{r =} \FloatTok{0.09}\NormalTok{, }\AttributeTok{n =} \DecValTok{50}\NormalTok{, }\AttributeTok{win =}\NormalTok{ janela)}

\CommentTok{\# (c) Agrupado (Atração {-} Thomas)}
\NormalTok{pp\_agrupado }\OtherTok{\textless{}{-}} \FunctionTok{rThomas}\NormalTok{(}\AttributeTok{kappa =} \DecValTok{5}\NormalTok{, }\AttributeTok{scale =} \FloatTok{0.04}\NormalTok{, }\AttributeTok{mu =} \DecValTok{10}\NormalTok{, }\AttributeTok{win =}\NormalTok{ janela)}

\CommentTok{\# Função auxiliar para calcular J(r) e plotar}
\NormalTok{plot\_j\_function }\OtherTok{\textless{}{-}} \ControlFlowTok{function}\NormalTok{(pp, titulo, anotacao) \{}
  \CommentTok{\# Calcular J(r)}
\NormalTok{  J\_calc }\OtherTok{\textless{}{-}} \FunctionTok{Jest}\NormalTok{(pp, }\AttributeTok{correction =} \StringTok{"km"}\NormalTok{)}
\NormalTok{  df\_J }\OtherTok{\textless{}{-}} \FunctionTok{as.data.frame}\NormalTok{(J\_calc)}
  
  \CommentTok{\# Limitar o eixo X para onde a interação é relevante}
\NormalTok{  df\_J }\OtherTok{\textless{}{-}}\NormalTok{ df\_J[df\_J}\SpecialCharTok{$}\NormalTok{r }\SpecialCharTok{\textless{}=} \FloatTok{0.22} \SpecialCharTok{\&} \FunctionTok{is.finite}\NormalTok{(df\_J}\SpecialCharTok{$}\NormalTok{km), ]}
  
  \FunctionTok{ggplot}\NormalTok{(df\_J, }\FunctionTok{aes}\NormalTok{(}\AttributeTok{x =}\NormalTok{ r)) }\SpecialCharTok{+}
    \FunctionTok{geom\_hline}\NormalTok{(}\FunctionTok{aes}\NormalTok{(}\AttributeTok{yintercept =} \DecValTok{1}\NormalTok{, }\AttributeTok{linetype =} \StringTok{"Teórico (Poisson)"}\NormalTok{), }
               \AttributeTok{color =} \StringTok{"red"}\NormalTok{, }\AttributeTok{size =} \FloatTok{0.8}\NormalTok{) }\SpecialCharTok{+}
    \CommentTok{\# Linha Observada}
    \FunctionTok{geom\_line}\NormalTok{(}\FunctionTok{aes}\NormalTok{(}\AttributeTok{y =}\NormalTok{ km, }\AttributeTok{linetype =} \StringTok{"Observado"}\NormalTok{), }\AttributeTok{color =} \StringTok{"black"}\NormalTok{, }\AttributeTok{size =} \DecValTok{1}\NormalTok{) }\SpecialCharTok{+}
    
    \FunctionTok{scale\_linetype\_manual}\NormalTok{(}\AttributeTok{name =} \StringTok{""}\NormalTok{, }\AttributeTok{values =} \FunctionTok{c}\NormalTok{(}\StringTok{"Observado"} \OtherTok{=} \StringTok{"solid"}\NormalTok{, }\StringTok{"Teórico (Poisson)"} \OtherTok{=} \StringTok{"dashed"}\NormalTok{)) }\SpecialCharTok{+}
    \FunctionTok{coord\_cartesian}\NormalTok{(}\AttributeTok{ylim =} \FunctionTok{c}\NormalTok{(}\DecValTok{0}\NormalTok{, }\FloatTok{2.5}\NormalTok{)) }\SpecialCharTok{+}
    
    \FunctionTok{labs}\NormalTok{(}\AttributeTok{title =}\NormalTok{ titulo, }
         \AttributeTok{subtitle =}\NormalTok{ anotacao,}
         \AttributeTok{x =} \StringTok{"Distância (r)"}\NormalTok{, }\AttributeTok{y =} \StringTok{"J(r)"}\NormalTok{) }\SpecialCharTok{+}
    \FunctionTok{theme\_minimal}\NormalTok{() }\SpecialCharTok{+}
    \FunctionTok{theme}\NormalTok{(}\AttributeTok{legend.position =} \StringTok{"bottom"}\NormalTok{,}
          \AttributeTok{plot.title =} \FunctionTok{element\_text}\NormalTok{(}\AttributeTok{size =} \DecValTok{11}\NormalTok{))}
\NormalTok{\}}


\NormalTok{p1 }\OtherTok{\textless{}{-}} \FunctionTok{plot\_j\_function}\NormalTok{(pp\_regular, }\StringTok{"(a) Regular"}\NormalTok{, }\StringTok{"J(r) \textgreater{} 1 (Inibição)"}\NormalTok{)}
\NormalTok{p2 }\OtherTok{\textless{}{-}} \FunctionTok{plot\_j\_function}\NormalTok{(pp\_aleatorio, }\StringTok{"(b) Aleatório"}\NormalTok{, }\StringTok{"J(r) = 1 (Independência)"}\NormalTok{)}
\NormalTok{p3 }\OtherTok{\textless{}{-}} \FunctionTok{plot\_j\_function}\NormalTok{(pp\_agrupado, }\StringTok{"(c) Agrupado"}\NormalTok{, }\StringTok{"J(r) \textless{} 1 (Atração)"}\NormalTok{)}


\NormalTok{p1 }\SpecialCharTok{+}\NormalTok{ p2 }\SpecialCharTok{+}\NormalTok{ p3}
\end{Highlighting}
\end{Shaded}

\begin{figure}[H]

\centering{

\pandocbounded{\includegraphics[keepaspectratio]{point_process_files/figure-pdf/fig-funcao-j-1.pdf}}

}

\caption{\label{fig-funcao-j}Comportamento da Função J(r): (a) Regular
(J \textgreater{} 1), (b) Aleatório (J = 1) e (c) Agrupado (J
\textless{} 1).}

\end{figure}%

\textbf{Função \(K (r)\) ou função \(\kappa\) de Ripley}

A função de \(K(r)\) ou função \(\kappa\) de Ripley estima o número
esperado de \(r\)-ésimos vizinhos de um evento \(s\), dividido pela
intensidade \(\lambda (s)\). Em outras palavras, esta função estima o
número médio de eventos
\(( \mathbb{E}[d(s_{i^{'}},r,s_{i})\vert s_{i^{'}} \in S])\) que estão
contidos em um círculo de raio \(r\), centrado em um evento \(s_{i}\) de
referência, sem contar o próprio evento \(s_{i}\) (\(r\) é a distância
entre o evento de referência \(s_{i}\) até outro evento \(s_{i^{'}}\)).
Em seguida, número médio de eventos estimados é dividido pela
intensidade \(\lambda (s)\) ( Figura~\ref{fig-k-ripley-conceito}), ou
seja,

\begin{equation}\phantomsection\label{eq-19.}{
\begin{aligned}
K(r)=&\frac{ \mathbb{E}[d(s_{i},r,s_{i^{'}})\leq r\vert s_{i^{'}} \in S]}{\lambda (s_{i}) \lambda (s_{i^{'}})} = \frac{ \mathbb{E}[\sum_{i^{'}\neq i}  \mathbb{I} \{0 < ||s_{i}-s_{i^{'}}|| \leq r \} \vert s_{i^{'}} \in S]}{\lambda (s_{i}) \lambda (s_{i^{'}})}, \lambda (s) > 0, r \geq 0, \\
\hat{K}(r)&=\frac{1}{|B|}\sum_{i=1} \sum_{i^{'}\neq i} \frac{ \mathbb{I} \left[||s_{i}-s_{i^{'}}||\leq r \right]}{\hat{\lambda} (s_{i}) \hat{\lambda} (s_{i^{'}})} \text{ ou } \hat{K}_{bord}(r)=\frac{1}{|B|}\sum_{i=1} \sum_{i^{'}\neq i} \frac{ \mathbb{I} \left[||s_{i}-s_{i^{'}}||\leq r \right]}{\hat{\lambda} (s_{i}) \hat{\lambda} (s_{i^{'}})e_{ii^{'}}},
\end{aligned}
}\end{equation}

onde \(\hat{K}(r)\) e \(\hat{K}_{bord}(r)\) são respectivos estimadores
na ausência e na presença dos efeitos da borda
respectivamente.\(e_{ii^{'}}\) é correção da borda,
\(\mathbb{I}[\cdot]\) função indicadora, \(|B|\) área de região em
estudo e \(\hat{\lambda} (s)=n/|B|\) é intensidade (N. Cressie 1993;
Gatrell et al. 1996; Peter J. Diggle 2013; Leininger 2014; A González e
Moraga 2023; Scalon 2024).

\begin{Shaded}
\begin{Highlighting}[]
\FunctionTok{library}\NormalTok{(ggplot2)}

\CommentTok{\# 1. Configuração do Cenário}
\FunctionTok{set.seed}\NormalTok{(}\DecValTok{123}\NormalTok{)}
\NormalTok{centro\_x }\OtherTok{\textless{}{-}} \FloatTok{0.5}
\NormalTok{centro\_y }\OtherTok{\textless{}{-}} \FloatTok{0.5}
\NormalTok{raio\_r }\OtherTok{\textless{}{-}} \FloatTok{0.25}

\CommentTok{\# 2. Gerar Eventos Aleatórios}
\NormalTok{df\_pontos }\OtherTok{\textless{}{-}} \FunctionTok{data.frame}\NormalTok{(}
  \AttributeTok{x =} \FunctionTok{runif}\NormalTok{(}\DecValTok{20}\NormalTok{, }\DecValTok{0}\NormalTok{, }\DecValTok{1}\NormalTok{),}
  \AttributeTok{y =} \FunctionTok{runif}\NormalTok{(}\DecValTok{20}\NormalTok{, }\DecValTok{0}\NormalTok{, }\DecValTok{1}\NormalTok{)}
\NormalTok{)}

\CommentTok{\# Forçar o evento de referência (si) no centro}
\NormalTok{df\_pontos }\OtherTok{\textless{}{-}} \FunctionTok{rbind}\NormalTok{(df\_pontos, }\FunctionTok{data.frame}\NormalTok{(}\AttributeTok{x =}\NormalTok{ centro\_x, }\AttributeTok{y =}\NormalTok{ centro\_y))}

\CommentTok{\# 3. Calcular distâncias e classificar}
\NormalTok{df\_pontos}\SpecialCharTok{$}\NormalTok{dist }\OtherTok{\textless{}{-}} \FunctionTok{sqrt}\NormalTok{((df\_pontos}\SpecialCharTok{$}\NormalTok{x }\SpecialCharTok{{-}}\NormalTok{ centro\_x)}\SpecialCharTok{\^{}}\DecValTok{2} \SpecialCharTok{+}\NormalTok{ (df\_pontos}\SpecialCharTok{$}\NormalTok{y }\SpecialCharTok{{-}}\NormalTok{ centro\_y)}\SpecialCharTok{\^{}}\DecValTok{2}\NormalTok{)}

\NormalTok{df\_pontos}\SpecialCharTok{$}\NormalTok{tipo }\OtherTok{\textless{}{-}}\NormalTok{ dplyr}\SpecialCharTok{::}\FunctionTok{case\_when}\NormalTok{(}
\NormalTok{  df\_pontos}\SpecialCharTok{$}\NormalTok{dist }\SpecialCharTok{==} \DecValTok{0} \SpecialCharTok{\textasciitilde{}} \StringTok{"Referencia (si)"}\NormalTok{,}
\NormalTok{  df\_pontos}\SpecialCharTok{$}\NormalTok{dist }\SpecialCharTok{\textless{}=}\NormalTok{ raio\_r }\SpecialCharTok{\textasciitilde{}} \StringTok{"Vizinho (sj)"}\NormalTok{,}
  \ConstantTok{TRUE} \SpecialCharTok{\textasciitilde{}} \StringTok{"Outros"}
\NormalTok{)}

\CommentTok{\# 4. Criar o Círculo para o plot}
\NormalTok{angulo }\OtherTok{\textless{}{-}} \FunctionTok{seq}\NormalTok{(}\DecValTok{0}\NormalTok{, }\DecValTok{2} \SpecialCharTok{*}\NormalTok{ pi, }\AttributeTok{length.out =} \DecValTok{100}\NormalTok{)}
\NormalTok{circulo }\OtherTok{\textless{}{-}} \FunctionTok{data.frame}\NormalTok{(}
  \AttributeTok{x =}\NormalTok{ centro\_x }\SpecialCharTok{+}\NormalTok{ raio\_r }\SpecialCharTok{*} \FunctionTok{cos}\NormalTok{(angulo),}
  \AttributeTok{y =}\NormalTok{ centro\_y }\SpecialCharTok{+}\NormalTok{ raio\_r }\SpecialCharTok{*} \FunctionTok{sin}\NormalTok{(angulo)}
\NormalTok{)}

\CommentTok{\# 5. Plotagem}
\FunctionTok{ggplot}\NormalTok{() }\SpecialCharTok{+}
  \CommentTok{\# Área de busca (Círculo)}
  \FunctionTok{geom\_polygon}\NormalTok{(}\AttributeTok{data =}\NormalTok{ circulo, }\FunctionTok{aes}\NormalTok{(x, y), }\AttributeTok{fill =} \StringTok{"blue"}\NormalTok{, }\AttributeTok{alpha =} \FloatTok{0.1}\NormalTok{) }\SpecialCharTok{+}
  \FunctionTok{geom\_path}\NormalTok{(}\AttributeTok{data =}\NormalTok{ circulo, }\FunctionTok{aes}\NormalTok{(x, y), }\AttributeTok{linetype =} \StringTok{"dashed"}\NormalTok{, }\AttributeTok{color =} \StringTok{"blue"}\NormalTok{) }\SpecialCharTok{+}
  
  \CommentTok{\# Raio r (Seta)}
  \FunctionTok{geom\_segment}\NormalTok{(}\FunctionTok{aes}\NormalTok{(}\AttributeTok{x =}\NormalTok{ centro\_x, }\AttributeTok{y =}\NormalTok{ centro\_y, }
                   \AttributeTok{xend =}\NormalTok{ centro\_x }\SpecialCharTok{+}\NormalTok{ raio\_r }\SpecialCharTok{*} \FunctionTok{cos}\NormalTok{(pi}\SpecialCharTok{/}\DecValTok{4}\NormalTok{), }
                   \AttributeTok{yend =}\NormalTok{ centro\_y }\SpecialCharTok{+}\NormalTok{ raio\_r }\SpecialCharTok{*} \FunctionTok{sin}\NormalTok{(pi}\SpecialCharTok{/}\DecValTok{4}\NormalTok{)), }
               \AttributeTok{arrow =} \FunctionTok{arrow}\NormalTok{(}\AttributeTok{length =} \FunctionTok{unit}\NormalTok{(}\FloatTok{0.2}\NormalTok{, }\StringTok{"cm"}\NormalTok{)), }\AttributeTok{color =} \StringTok{"black"}\NormalTok{) }\SpecialCharTok{+}
  \FunctionTok{annotate}\NormalTok{(}\StringTok{"text"}\NormalTok{, }\AttributeTok{x =}\NormalTok{ centro\_x }\SpecialCharTok{+} \FloatTok{0.1}\NormalTok{, }\AttributeTok{y =}\NormalTok{ centro\_y }\SpecialCharTok{+} \FloatTok{0.12}\NormalTok{, }\AttributeTok{label =} \StringTok{"r"}\NormalTok{, }\AttributeTok{size =} \DecValTok{6}\NormalTok{) }\SpecialCharTok{+}
  
  \CommentTok{\# Pontos (Eventos)}
  \FunctionTok{geom\_point}\NormalTok{(}\AttributeTok{data =}\NormalTok{ df\_pontos, }\FunctionTok{aes}\NormalTok{(x, y, }\AttributeTok{color =}\NormalTok{ tipo, }\AttributeTok{size =}\NormalTok{ tipo, }\AttributeTok{shape =}\NormalTok{ tipo)) }\SpecialCharTok{+}
  
  \CommentTok{\# Rótulos}
  \FunctionTok{annotate}\NormalTok{(}\StringTok{"text"}\NormalTok{, }\AttributeTok{x =}\NormalTok{ centro\_x, }\AttributeTok{y =}\NormalTok{ centro\_y }\SpecialCharTok{{-}} \FloatTok{0.04}\NormalTok{, }\AttributeTok{label =} \StringTok{"s[i]"}\NormalTok{, }\AttributeTok{parse =} \ConstantTok{TRUE}\NormalTok{, }\AttributeTok{color =} \StringTok{"red"}\NormalTok{, }\AttributeTok{size =} \DecValTok{5}\NormalTok{) }\SpecialCharTok{+}
  
  \CommentTok{\# Estilo}
  \FunctionTok{scale\_color\_manual}\NormalTok{(}\AttributeTok{values =} \FunctionTok{c}\NormalTok{(}\StringTok{"Referencia (si)"} \OtherTok{=} \StringTok{"red"}\NormalTok{, }\StringTok{"Vizinho (sj)"} \OtherTok{=} \StringTok{"blue"}\NormalTok{, }\StringTok{"Outros"} \OtherTok{=} \StringTok{"gray70"}\NormalTok{)) }\SpecialCharTok{+}
  \FunctionTok{scale\_size\_manual}\NormalTok{(}\AttributeTok{values =} \FunctionTok{c}\NormalTok{(}\StringTok{"Referencia (si)"} \OtherTok{=} \DecValTok{5}\NormalTok{, }\StringTok{"Vizinho (sj)"} \OtherTok{=} \DecValTok{3}\NormalTok{, }\StringTok{"Outros"} \OtherTok{=} \DecValTok{2}\NormalTok{)) }\SpecialCharTok{+}
  \FunctionTok{scale\_shape\_manual}\NormalTok{(}\AttributeTok{values =} \FunctionTok{c}\NormalTok{(}\StringTok{"Referencia (si)"} \OtherTok{=} \DecValTok{19}\NormalTok{, }\StringTok{"Vizinho (sj)"} \OtherTok{=} \DecValTok{19}\NormalTok{, }\StringTok{"Outros"} \OtherTok{=} \DecValTok{1}\NormalTok{)) }\SpecialCharTok{+}
  
  \FunctionTok{coord\_fixed}\NormalTok{(}\AttributeTok{xlim =} \FunctionTok{c}\NormalTok{(}\DecValTok{0}\NormalTok{, }\DecValTok{1}\NormalTok{), }\AttributeTok{ylim =} \FunctionTok{c}\NormalTok{(}\DecValTok{0}\NormalTok{, }\DecValTok{1}\NormalTok{)) }\SpecialCharTok{+}
  \FunctionTok{theme\_void}\NormalTok{() }\SpecialCharTok{+}
  \FunctionTok{theme}\NormalTok{(}\AttributeTok{legend.position =} \StringTok{"bottom"}\NormalTok{, }
        \AttributeTok{legend.title =} \FunctionTok{element\_blank}\NormalTok{(),}
        \AttributeTok{plot.margin =} \FunctionTok{margin}\NormalTok{(}\DecValTok{10}\NormalTok{, }\DecValTok{10}\NormalTok{, }\DecValTok{10}\NormalTok{, }\DecValTok{10}\NormalTok{)) }\SpecialCharTok{+}
  \FunctionTok{labs}\NormalTok{(}\AttributeTok{title =} \StringTok{"Contagem K(r): Quantos pontos azuis existem?"}\NormalTok{)}
\end{Highlighting}
\end{Shaded}

\begin{figure}[H]

\centering{

\pandocbounded{\includegraphics[keepaspectratio]{point_process_files/figure-pdf/fig-k-ripley-conceito-1.pdf}}

}

\caption{\label{fig-k-ripley-conceito}Intuição da Função K de Ripley:
Contagem de eventos vizinhos (azuis) dentro de um círculo de raio r
centrado no evento de referência si (vermelho).}

\end{figure}%

A interpretação da função K(r) é idêntica à função G(r). Se
\(K(r)<K_{pois}(r)\) (abaixo de), significa que na distância
considerada, há poucos eventos contidos no círculo do que seria esperado
se o padrão fosse aleatório. Consequentemente, o padrão é regular (
Figura~\ref{fig-funcao-k} (a)). Se \(K(r)\equiv K_{pois}\), o padrão é
aleatório ( Figura~\ref{fig-funcao-k} (b)). Se \(K(r)>K_{pois}\) (acima
de), significa que na distância (r) considerada, existem muitos eventos
contidos no círculo do que seria esperado se o padrão fosse aleatório.
Consequentemente, o padrão é agrupado ( Figura~\ref{fig-funcao-k} (c)).

\begin{Shaded}
\begin{Highlighting}[]
\FunctionTok{set.seed}\NormalTok{(}\DecValTok{500}\NormalTok{)}
\NormalTok{janela }\OtherTok{\textless{}{-}} \FunctionTok{owin}\NormalTok{(}\FunctionTok{c}\NormalTok{(}\DecValTok{0}\NormalTok{, }\DecValTok{1}\NormalTok{), }\FunctionTok{c}\NormalTok{(}\DecValTok{0}\NormalTok{, }\DecValTok{1}\NormalTok{))}

\CommentTok{\# (b) Aleatório (Referência)}
\NormalTok{pp\_aleatorio }\OtherTok{\textless{}{-}} \FunctionTok{rpoispp}\NormalTok{(}\AttributeTok{lambda =} \DecValTok{50}\NormalTok{, }\AttributeTok{win =}\NormalTok{ janela)}

\CommentTok{\# (a) Regular (Inibição {-} Vizinhos distantes reduzem o valor acumulado de K)}
\NormalTok{pp\_regular }\OtherTok{\textless{}{-}} \FunctionTok{rSSI}\NormalTok{(}\AttributeTok{r =} \FloatTok{0.1}\NormalTok{, }\AttributeTok{n =} \DecValTok{50}\NormalTok{, }\AttributeTok{win =}\NormalTok{ janela)}

\CommentTok{\# (c) Agrupado (Atração {-} Vizinhos próximos aumentam o valor acumulado de K)}
\NormalTok{pp\_agrupado }\OtherTok{\textless{}{-}} \FunctionTok{rThomas}\NormalTok{(}\AttributeTok{kappa =} \DecValTok{5}\NormalTok{, }\AttributeTok{scale =} \FloatTok{0.04}\NormalTok{, }\AttributeTok{mu =} \DecValTok{10}\NormalTok{, }\AttributeTok{win =}\NormalTok{ janela)}

\NormalTok{plot\_k\_function }\OtherTok{\textless{}{-}} \ControlFlowTok{function}\NormalTok{(pp, titulo, anotacao) \{}
\NormalTok{  K\_calc }\OtherTok{\textless{}{-}} \FunctionTok{Kest}\NormalTok{(pp, }\AttributeTok{correction =} \StringTok{"iso"}\NormalTok{)}
\NormalTok{  df\_K }\OtherTok{\textless{}{-}} \FunctionTok{as.data.frame}\NormalTok{(K\_calc)}
\NormalTok{  df\_K }\OtherTok{\textless{}{-}}\NormalTok{ df\_K[df\_K}\SpecialCharTok{$}\NormalTok{r }\SpecialCharTok{\textless{}=} \FloatTok{0.25}\NormalTok{, ]}
  
  \FunctionTok{ggplot}\NormalTok{(df\_K, }\FunctionTok{aes}\NormalTok{(}\AttributeTok{x =}\NormalTok{ r)) }\SpecialCharTok{+}
    \CommentTok{\# Linha Teórica (Poisson = pi * r\^{}2)}
    \FunctionTok{geom\_line}\NormalTok{(}\FunctionTok{aes}\NormalTok{(}\AttributeTok{y =}\NormalTok{ theo, }\AttributeTok{linetype =} \StringTok{"Teórico (Poisson)"}\NormalTok{), }
              \AttributeTok{color =} \StringTok{"red"}\NormalTok{, }\AttributeTok{size =} \FloatTok{0.8}\NormalTok{) }\SpecialCharTok{+}
    \FunctionTok{geom\_line}\NormalTok{(}\FunctionTok{aes}\NormalTok{(}\AttributeTok{y =}\NormalTok{ iso, }\AttributeTok{linetype =} \StringTok{"Observado"}\NormalTok{), }
              \AttributeTok{color =} \StringTok{"black"}\NormalTok{, }\AttributeTok{size =} \DecValTok{1}\NormalTok{) }\SpecialCharTok{+}
    
    \FunctionTok{scale\_linetype\_manual}\NormalTok{(}\AttributeTok{name =} \StringTok{""}\NormalTok{, }\AttributeTok{values =} \FunctionTok{c}\NormalTok{(}\StringTok{"Observado"} \OtherTok{=} \StringTok{"solid"}\NormalTok{, }\StringTok{"Teórico (Poisson)"} \OtherTok{=} \StringTok{"dashed"}\NormalTok{)) }\SpecialCharTok{+}
    
    \FunctionTok{labs}\NormalTok{(}\AttributeTok{title =}\NormalTok{ titulo, }
         \AttributeTok{subtitle =}\NormalTok{ anotacao,}
         \AttributeTok{x =} \StringTok{"Distância (r)"}\NormalTok{, }\AttributeTok{y =} \StringTok{"K(r)"}\NormalTok{) }\SpecialCharTok{+}
    \FunctionTok{theme\_minimal}\NormalTok{() }\SpecialCharTok{+}
    \FunctionTok{theme}\NormalTok{(}\AttributeTok{legend.position =} \StringTok{"bottom"}\NormalTok{,}
          \AttributeTok{plot.title =} \FunctionTok{element\_text}\NormalTok{(}\AttributeTok{size =} \DecValTok{11}\NormalTok{))}
\NormalTok{\}}


\NormalTok{p1 }\OtherTok{\textless{}{-}} \FunctionTok{plot\_k\_function}\NormalTok{(pp\_regular, }\StringTok{"(a) Regular"}\NormalTok{, }\StringTok{"Obs \textless{} Teórico (Crescimento Lento)"}\NormalTok{)}
\NormalTok{p2 }\OtherTok{\textless{}{-}} \FunctionTok{plot\_k\_function}\NormalTok{(pp\_aleatorio, }\StringTok{"(b) Aleatório"}\NormalTok{, }\StringTok{"Obs = Teórico (Parábola)"}\NormalTok{)}
\NormalTok{p3 }\OtherTok{\textless{}{-}} \FunctionTok{plot\_k\_function}\NormalTok{(pp\_agrupado, }\StringTok{"(c) Agrupado"}\NormalTok{, }\StringTok{"Obs \textgreater{} Teórico (Crescimento Rápido)"}\NormalTok{)}


\NormalTok{p1 }\SpecialCharTok{+}\NormalTok{ p2 }\SpecialCharTok{+}\NormalTok{ p3}
\end{Highlighting}
\end{Shaded}

\begin{figure}[H]

\centering{

\pandocbounded{\includegraphics[keepaspectratio]{point_process_files/figure-pdf/fig-funcao-k-1.pdf}}

}

\caption{\label{fig-funcao-k}Comportamento da Função K(r): (a) Regular
(Abaixo da teórica), (b) Aleatório (Sobreposta) e (c) Agrupado (Acima da
teórica).}

\end{figure}%

\textbf{Função L (r)}

Conforme descrito em Scalon (2024), a função L (r) é uma transformação
da função \(K(r)\) sugerida por J. Besag (1977), visando transformar o
modelo Poisson teórico (representada por ``- - -'') em uma linha reta,
bem como estabilizar a variância nos dados sob completa aleatoriedade
espacial ( Figura~\ref{fig-funcao-l}), ou seja,

\begin{equation}\phantomsection\label{eq-20}{L(r)=\sqrt{\frac{K(r)}{\pi}} \text{ e } \hat{L}(r)=\sqrt{\frac{\hat{K}(r)}{\pi}}}\end{equation}

\begin{Shaded}
\begin{Highlighting}[]
\FunctionTok{set.seed}\NormalTok{(}\DecValTok{300}\NormalTok{)}
\NormalTok{janela }\OtherTok{\textless{}{-}} \FunctionTok{owin}\NormalTok{(}\FunctionTok{c}\NormalTok{(}\DecValTok{0}\NormalTok{, }\DecValTok{1}\NormalTok{), }\FunctionTok{c}\NormalTok{(}\DecValTok{0}\NormalTok{, }\DecValTok{1}\NormalTok{))}

\CommentTok{\# 1. Simulação dos Processos}
\CommentTok{\# (b) Aleatório (Referência)}
\NormalTok{pp\_aleatorio }\OtherTok{\textless{}{-}} \FunctionTok{rpoispp}\NormalTok{(}\AttributeTok{lambda =} \DecValTok{60}\NormalTok{, }\AttributeTok{win =}\NormalTok{ janela)}

\CommentTok{\# (a) Regular (Inibição forte)}
\NormalTok{pp\_regular }\OtherTok{\textless{}{-}} \FunctionTok{rSSI}\NormalTok{(}\AttributeTok{r =} \FloatTok{0.09}\NormalTok{, }\AttributeTok{n =} \DecValTok{60}\NormalTok{, }\AttributeTok{win =}\NormalTok{ janela)}

\CommentTok{\# (c) Agrupado (Atração forte)}
\NormalTok{pp\_agrupado }\OtherTok{\textless{}{-}} \FunctionTok{rThomas}\NormalTok{(}\AttributeTok{kappa =} \DecValTok{5}\NormalTok{, }\AttributeTok{scale =} \FloatTok{0.04}\NormalTok{, }\AttributeTok{mu =} \DecValTok{12}\NormalTok{, }\AttributeTok{win =}\NormalTok{ janela)}

\NormalTok{plot\_l\_function }\OtherTok{\textless{}{-}} \ControlFlowTok{function}\NormalTok{(pp, titulo, anotacao) \{}
  \CommentTok{\# L(r) = sqrt(K(r)/pi)}
\NormalTok{  L\_calc }\OtherTok{\textless{}{-}} \FunctionTok{Lest}\NormalTok{(pp, }\AttributeTok{correction =} \StringTok{"iso"}\NormalTok{)}
\NormalTok{  df\_L }\OtherTok{\textless{}{-}} \FunctionTok{as.data.frame}\NormalTok{(L\_calc)}
  
\NormalTok{  df\_L }\OtherTok{\textless{}{-}}\NormalTok{ df\_L[df\_L}\SpecialCharTok{$}\NormalTok{r }\SpecialCharTok{\textless{}=} \FloatTok{0.25}\NormalTok{, ]}
  
  \FunctionTok{ggplot}\NormalTok{(df\_L, }\FunctionTok{aes}\NormalTok{(}\AttributeTok{x =}\NormalTok{ r)) }\SpecialCharTok{+}
    \FunctionTok{geom\_line}\NormalTok{(}\FunctionTok{aes}\NormalTok{(}\AttributeTok{y =}\NormalTok{ theo, }\AttributeTok{linetype =} \StringTok{"Teórico (Poisson)"}\NormalTok{), }
              \AttributeTok{color =} \StringTok{"red"}\NormalTok{, }\AttributeTok{size =} \FloatTok{0.8}\NormalTok{) }\SpecialCharTok{+}
    \FunctionTok{geom\_line}\NormalTok{(}\FunctionTok{aes}\NormalTok{(}\AttributeTok{y =}\NormalTok{ iso, }\AttributeTok{linetype =} \StringTok{"Observado"}\NormalTok{), }
              \AttributeTok{color =} \StringTok{"black"}\NormalTok{, }\AttributeTok{size =} \DecValTok{1}\NormalTok{) }\SpecialCharTok{+}
    
    \FunctionTok{scale\_linetype\_manual}\NormalTok{(}\AttributeTok{name =} \StringTok{""}\NormalTok{, }\AttributeTok{values =} \FunctionTok{c}\NormalTok{(}\StringTok{"Observado"} \OtherTok{=} \StringTok{"solid"}\NormalTok{, }\StringTok{"Teórico (Poisson)"} \OtherTok{=} \StringTok{"dashed"}\NormalTok{)) }\SpecialCharTok{+}
    
    \FunctionTok{labs}\NormalTok{(}\AttributeTok{title =}\NormalTok{ titulo, }
         \AttributeTok{subtitle =}\NormalTok{ anotacao,}
         \AttributeTok{x =} \StringTok{"Distância (r)"}\NormalTok{, }\AttributeTok{y =} \StringTok{"L(r)"}\NormalTok{) }\SpecialCharTok{+}
    \FunctionTok{theme\_minimal}\NormalTok{() }\SpecialCharTok{+}
    \FunctionTok{theme}\NormalTok{(}\AttributeTok{legend.position =} \StringTok{"bottom"}\NormalTok{,}
          \AttributeTok{plot.title =} \FunctionTok{element\_text}\NormalTok{(}\AttributeTok{size =} \DecValTok{11}\NormalTok{))}
\NormalTok{\}}

\NormalTok{p1 }\OtherTok{\textless{}{-}} \FunctionTok{plot\_l\_function}\NormalTok{(pp\_regular, }\StringTok{"(a) Regular"}\NormalTok{, }\StringTok{"L(r) \textless{} r (Abaixo da reta)"}\NormalTok{)}
\NormalTok{p2 }\OtherTok{\textless{}{-}} \FunctionTok{plot\_l\_function}\NormalTok{(pp\_aleatorio, }\StringTok{"(b) Aleatório"}\NormalTok{, }\StringTok{"L(r) = r (Linha Reta)"}\NormalTok{)}
\NormalTok{p3 }\OtherTok{\textless{}{-}} \FunctionTok{plot\_l\_function}\NormalTok{(pp\_agrupado, }\StringTok{"(c) Agrupado"}\NormalTok{, }\StringTok{"L(r) \textgreater{} r (Acima da reta)"}\NormalTok{)}


\NormalTok{p1 }\SpecialCharTok{+}\NormalTok{ p2 }\SpecialCharTok{+}\NormalTok{ p3}
\end{Highlighting}
\end{Shaded}

\begin{figure}[H]

\centering{

\pandocbounded{\includegraphics[keepaspectratio]{point_process_files/figure-pdf/fig-funcao-l-1.pdf}}

}

\caption{\label{fig-funcao-l}Comportamento da Função L(r): (a) Regular
(Abaixo da diagonal), (b) Aleatório (Sobre a diagonal) e (c) Agrupado
(Acima da diagonal).}

\end{figure}%

Como \(L_{pois}(r) = r\), que corresponde a uma linha reta de inclinação
1, é comum subtrair \(r\) de \(L(r)\), resultando em
\(L_{pois}(r) - r = 0\), o que permite estudar o padrão espacial em
torno de zero, conforme será detalhado na seção dos resultados.

Em qualquer uma das possibilidades aqui apresentadas, se
\(\hat{L}(r) < L_{pois}(r)\), o processo pontual apresenta um padrão
regular ( Figura~\ref{fig-funcao-l} (a)). Se
\(\hat{L}(r)\equiv L_{pois}(r)\) o processo pontual apresenta um padrão
aleatório ( Figura~\ref{fig-funcao-l} (b)). Se
\(\hat{L}(r) > L_{pois}(r)\) o processo pontual apresenta um padrão
agrupado ( Figura~\ref{fig-funcao-l} (c)).

\textbf{Função \(g(r)\)}

A função g(r), também conhecida como função correlação par, é uma função
de densidade de probabilidade padronizada que descreve a ocorrência
conjunta dos eventos \(s_{i}\) e \(s_{i^{'}}\) em uma determinada região
de estudo e, diferentemente de todas as funções já apresentadas, não é
cumulativa ( Figura~\ref{fig-funcao-g-pair}), ou seja,\\
\(g(r)=\frac{\lambda(s_{i},s_{i^{'}})}{\lambda(s_{i})\lambda(s_{i^{'}})} \text{ e } \hat{g}_{bord}(r)=\frac{1}{2\pi r |B|} \sum_{i=1}^{n}\sum_{j} \frac{K_{h}\left(||s_{i}-s_{i^{'}}||-r\right)}{\hat{\lambda}(s_{i})\hat{\lambda}(s_{i^{'}})e_{ii^{'}}}\),
onde \(g(r)\neq G(r)\),\(\hat{g}_{bord}(r)\) é seu estimador na presença
dos efeitos da borda,\(K_{h}\) é função Kernel,\(h\) é largura da banda
e \(e_{ii^{'}}\) correção da borda. Na ausência destes,\(\hat{g}(r)\) é
obtido omitindo \(e_{ii^{'}}\) em \(\hat{g}_{bord}(r)\) (Peter J. Diggle
2010; A González e Moraga 2023; Scalon 2024).

Segundo Waagepetersen e Guan (2009) e Scalon (2024), se o processo
pontual for isotrópico, a função \(g(r)\) é dada pela razão entre a
derivada da função \(K(r)\) e \(2\pi r\), ou seja,\\
\(g(r) = \frac{K^{'}(r)}{2\pi r} \text{ e } \hat{g}(r)= \frac{\hat{K}^{'}(r)}{2\pi r}\).

\begin{Shaded}
\begin{Highlighting}[]
\FunctionTok{set.seed}\NormalTok{(}\DecValTok{42}\NormalTok{)}
\NormalTok{janela }\OtherTok{\textless{}{-}} \FunctionTok{owin}\NormalTok{(}\FunctionTok{c}\NormalTok{(}\DecValTok{0}\NormalTok{, }\DecValTok{1}\NormalTok{), }\FunctionTok{c}\NormalTok{(}\DecValTok{0}\NormalTok{, }\DecValTok{1}\NormalTok{))}
\NormalTok{n\_pts }\OtherTok{\textless{}{-}} \DecValTok{100} \CommentTok{\# Aumentei para 100 para suavizar as curvas}


\CommentTok{\# (b) Aleatório: Poisson Homogêneo}
\NormalTok{pp\_aleatorio }\OtherTok{\textless{}{-}} \FunctionTok{rpoispp}\NormalTok{(}\AttributeTok{lambda =}\NormalTok{ n\_pts, }\AttributeTok{win =}\NormalTok{ janela)}

\CommentTok{\# (a) Regular: Inibição (Hard Core)}
\NormalTok{pp\_regular }\OtherTok{\textless{}{-}} \FunctionTok{rSSI}\NormalTok{(}\AttributeTok{r =} \FloatTok{0.07}\NormalTok{, }\AttributeTok{n =}\NormalTok{ n\_pts, }\AttributeTok{win =}\NormalTok{ janela)}

\CommentTok{\# (c) Agrupado: Processo de Thomas}
\CommentTok{\# kappa=10 (pais), scale=0.02 (filhos bem pertinho {-}\textgreater{} pico alto no g(r))}
\NormalTok{pp\_agrupado }\OtherTok{\textless{}{-}} \FunctionTok{rThomas}\NormalTok{(}\AttributeTok{kappa =} \DecValTok{10}\NormalTok{, }\AttributeTok{scale =} \FloatTok{0.02}\NormalTok{, }\AttributeTok{mu =} \DecValTok{10}\NormalTok{, }\AttributeTok{win =}\NormalTok{ janela)}

\NormalTok{plot\_pcf\_custom }\OtherTok{\textless{}{-}} \ControlFlowTok{function}\NormalTok{(pp, titulo, anotacao, }\AttributeTok{y\_max =} \ConstantTok{NULL}\NormalTok{) \{}
\NormalTok{  g\_calc }\OtherTok{\textless{}{-}} \FunctionTok{pcf}\NormalTok{(pp, }\AttributeTok{correction =} \StringTok{"best"}\NormalTok{, }\AttributeTok{divisor =} \StringTok{"d"}\NormalTok{) }
\NormalTok{  df\_g }\OtherTok{\textless{}{-}} \FunctionTok{as.data.frame}\NormalTok{(g\_calc)}
\NormalTok{  df\_g }\OtherTok{\textless{}{-}}\NormalTok{ df\_g[df\_g}\SpecialCharTok{$}\NormalTok{r }\SpecialCharTok{\textless{}=} \FloatTok{0.20}\NormalTok{, ]}
  
\NormalTok{  p }\OtherTok{\textless{}{-}} \FunctionTok{ggplot}\NormalTok{(df\_g, }\FunctionTok{aes}\NormalTok{(}\AttributeTok{x =}\NormalTok{ r)) }\SpecialCharTok{+}
    \FunctionTok{geom\_hline}\NormalTok{(}\AttributeTok{yintercept =} \DecValTok{1}\NormalTok{, }\AttributeTok{linetype =} \StringTok{"dashed"}\NormalTok{, }\AttributeTok{color =} \StringTok{"red"}\NormalTok{, }\AttributeTok{size =} \FloatTok{0.8}\NormalTok{) }\SpecialCharTok{+}
    \FunctionTok{geom\_line}\NormalTok{(}\FunctionTok{aes}\NormalTok{(}\AttributeTok{y =}\NormalTok{ iso), }\AttributeTok{color =} \StringTok{"black"}\NormalTok{, }\AttributeTok{size =} \FloatTok{1.2}\NormalTok{) }\SpecialCharTok{+}
    
    \FunctionTok{labs}\NormalTok{(}\AttributeTok{title =}\NormalTok{ titulo, }
         \AttributeTok{subtitle =}\NormalTok{ anotacao,}
         \AttributeTok{x =} \StringTok{"Distância (r)"}\NormalTok{, }\AttributeTok{y =} \StringTok{"g(r)"}\NormalTok{) }\SpecialCharTok{+}
    \FunctionTok{theme\_light}\NormalTok{() }\SpecialCharTok{+}
    \FunctionTok{theme}\NormalTok{(}\AttributeTok{plot.title =} \FunctionTok{element\_text}\NormalTok{( }\AttributeTok{size =} \DecValTok{12}\NormalTok{),}
          \AttributeTok{plot.subtitle =} \FunctionTok{element\_text}\NormalTok{(}\AttributeTok{size =} \DecValTok{10}\NormalTok{))}
    \ControlFlowTok{if}\NormalTok{ (}\SpecialCharTok{!}\FunctionTok{is.null}\NormalTok{(y\_max)) \{}
\NormalTok{    p }\OtherTok{\textless{}{-}}\NormalTok{ p }\SpecialCharTok{+} \FunctionTok{coord\_cartesian}\NormalTok{(}\AttributeTok{ylim =} \FunctionTok{c}\NormalTok{(}\DecValTok{0}\NormalTok{, y\_max))}
\NormalTok{  \}}
  
  \FunctionTok{return}\NormalTok{(p)}
\NormalTok{\}}

\NormalTok{p1 }\OtherTok{\textless{}{-}} \FunctionTok{plot\_pcf\_custom}\NormalTok{(pp\_regular, }
                      \StringTok{"(a) Regular"}\NormalTok{, }
                      \StringTok{"Inicia em ZERO (Inibição) Sobe para 1"}\NormalTok{, }
                      \AttributeTok{y\_max =} \FloatTok{2.5}\NormalTok{)}

\NormalTok{p2 }\OtherTok{\textless{}{-}} \FunctionTok{plot\_pcf\_custom}\NormalTok{(pp\_aleatorio, }
                      \StringTok{"(b) Aleatório"}\NormalTok{, }
                      \StringTok{"Oscila em torno de 1 (Linha Vermelha)"}\NormalTok{, }
                      \AttributeTok{y\_max =} \FloatTok{2.5}\NormalTok{)}

\NormalTok{p3 }\OtherTok{\textless{}{-}} \FunctionTok{plot\_pcf\_custom}\NormalTok{(pp\_agrupado, }
                      \StringTok{"(c) Agrupado"}\NormalTok{, }
                      \StringTok{"Pico muito alto no início (\textgreater{}1) Decai rapidamente"}\NormalTok{, }
                      \AttributeTok{y\_max =} \ConstantTok{NULL}\NormalTok{) }\CommentTok{\# Deixa o ggplot decidir a altura do pico}


\NormalTok{p1 }\SpecialCharTok{+}\NormalTok{ p2 }\SpecialCharTok{+}\NormalTok{ p3}
\end{Highlighting}
\end{Shaded}

\begin{figure}[H]

\centering{

\pandocbounded{\includegraphics[keepaspectratio]{point_process_files/figure-pdf/fig-funcao-g-pair-1.pdf}}

}

\caption{\label{fig-funcao-g-pair}Comportamento da Função g(r): (a)
Regular (g \textless{} 1), (b) Aleatório (g ≈ 1) e (c) Agrupado (g
\textgreater{} 1).}

\end{figure}%

Se \(\hat{g}(r) < 1\), o processo pontual apresenta um padrão regular (
Figura~\ref{fig-funcao-g-pair} (a)). Se \(\hat{g}(r)\equiv 1\) o
processo pontual apresenta um padrão aleatório (
Figura~\ref{fig-funcao-g-pair} (b)). Se \(\hat{g}(r) > 1\) o processo
pontual apresenta um padrão agrupado ( Figura~\ref{fig-funcao-g-pair}
(c)).

\section{\texorpdfstring{Funções
\(F_{inhom} (r), G_{inhom} (r), J_{inhom} (r), K_{inhom} (r), L_{inhom} (r)\, e \,g_{inhom} (r)\)}{Funções F\_\{inhom\} (r), G\_\{inhom\} (r), J\_\{inhom\} (r), K\_\{inhom\} (r), L\_\{inhom\} (r)\textbackslash, e \textbackslash,g\_\{inhom\} (r)}}\label{sec-2.2.5.2}

As funções \(F(r), G(r), J(r), K(r), L(r) \, e \, g(r)\) apresentadas na
seção Seção~\ref{sec-2.6.1} são úteis quando a condição de
estacionariedade ou homogeneidade é satisfeita. Caso contrário, elas têm
sua extensão para processos pontuais não homogêneos, sendo,

\[
F_{inhom} (r) = 1- \mathbb{E}\left[\prod_{s_{i}} \left(1-\frac{\lambda_{\min}}{\lambda (s_{i})}\right)\right], \\
\]

\[
\hat{F}_{inhom} (r) = 1-\frac{\sum_{i}\left( \mathbb{I}\left(w_{i}>r\right)\left[\prod_{i}  \mathbb{I}\left(||s_{i}-z_{i}||\leq r\right)\left(1-\frac{\hat{\lambda}_{\min}}{\hat{\lambda} (s_{i})}\right)\right]\right)}{\sum_{i} \mathbb{I}\left(w_{i}>r\right)} ,
\text{para a função F(r) não homogênea.}
\]

\[
G_{inhom} (r) = 1- \mathbb{E}\left[\prod_{s_{i}} \left(1-\frac{\lambda_{\min}}{\lambda (s_{i})}\right) \middle\lvert z\in S \right],
\]

\[
\hat{G}_{inhom} (r) = 1-\frac{\sum_{i}\left( \mathbb{I}\left(e_{i}>r\right)\left[\prod  \mathbb{I}\left(||s_{i}-s_{i^{'}}||\leq r\right)\left(1-\frac{\hat{\lambda}_{\min}}{\hat{\lambda} (s_{i})}\right)\right]\right)}{\sum_{i} \mathbb{I}\left(e_{i}>r\right)},
\]

\[
\text{para a função G(r) não homogênea} (G_{inhom} (r)).
\]

\[
J_{inhom}(r)= \frac{1-G_{inhom}(r)}{1-F_{inhom} (r)}, r \geq 0 , F_{inhom} (r) > 1,
\]

\[
\text{para a função J (r) não homogênea } (J_{inhom} (r)).
\]

\[
K_{inhom} (r) = \frac{1}{|B|} \mathbb{E} \left[\sum_{s_{i}}^{n}\sum_{s_{i^{'}} \setminus \{s_{i}\}} \frac{ \mathbb{I}\left(||s_{i}-s_{i^{'}}||\leq r\right)}{\lambda (s_{i}) \lambda (s_{i^{'}})} \right] \text{ e }
\]

\[
\hat{K}_{inhom} (r) = \frac{1}{|B|}\sum_{i}\sum_{i^{'}} \frac{ \mathbb{I}\left(||s_{i}-s_{i^{'}}||\leq r\right)}{\hat{\lambda} (s_{i})\hat{\lambda} (s_{i^{'}})e_{ii^{'}}},
\]

\[
\text{para a função K(r) não homogênea } (K_{inhom} (r)).
\] \[
L_{inhom}(r)=\sqrt{\frac{K_{inhom}(r)}{\pi}},\,\hat{L}_{inhom}(r)=\sqrt{\frac{\hat{K}_{inhom}(r)}{\pi}}, \text{para a função L(r) não homogênea }  (L_{inhom} (r)) e,
\]

\[
g_{inhom}(r) = \frac{K_{inhom}^{'}(r)}{2\pi r} \text{ e } \hat{g}_{inhom}(r) = \frac{\hat{K}_{inhom}^{'}(r)}{2\pi r}, \text{ para a função g(r) não homogênea} (g_{inhom } (r)).
\]

É comum encontrar na literatura, como em A. J. Baddeley, Møller, e
Waagepetersen (2000) e Adrian Baddeley, Rubak, e Turner (2015), a
correção de borda \(e_{ii'}\) no numerador, pois apenas eventos dentro
de uma certa distância da borda são considerados. Por outro lado, em
algumas literaturas, como as de Peter J. Diggle (2010), Leininger (2014)
e A González e Moraga (2023), a correção de borda é colocada no
denominador, pois o ajuste é feito normalizando a estatística de teste,
ou seja, \(e_{ii'}\) é igual a 1 quando o círculo centrado em \(s_{i'}\)
ou \(z_{k}\), passando pelo evento \(s_{i}\), está completamente contido
na área de estudo. Se parte do círculo estiver fora da área de estudo,
então \(e_{ii'}\) é calculada como a proporção do círculo contida dentro
da área de estudo. Assim, a escolha da correção de borda dependerá da
pesquisa e da preferência do pesquisador.

Apesar das funções homogêneas e não homogêneas apresentarem expressões
matemáticas diferentes, a sua interpretação é igual (Quadro
Tabela~\ref{tbl-1}).

\begin{longtable}[]{@{}
  >{\raggedright\arraybackslash}p{(\linewidth - 12\tabcolsep) * \real{0.1429}}
  >{\raggedright\arraybackslash}p{(\linewidth - 12\tabcolsep) * \real{0.1429}}
  >{\raggedright\arraybackslash}p{(\linewidth - 12\tabcolsep) * \real{0.1429}}
  >{\raggedright\arraybackslash}p{(\linewidth - 12\tabcolsep) * \real{0.1429}}
  >{\raggedright\arraybackslash}p{(\linewidth - 12\tabcolsep) * \real{0.1429}}
  >{\raggedright\arraybackslash}p{(\linewidth - 12\tabcolsep) * \real{0.1429}}
  >{\raggedright\arraybackslash}p{(\linewidth - 12\tabcolsep) * \real{0.1429}}@{}}
\caption{Resumo da interpretação gráfica das funções que capturam o
padrão de distribuição espacial.}\label{tbl-1}\tabularnewline
\toprule\noalign{}
\begin{minipage}[b]{\linewidth}\raggedright
\(\downarrow\) Padrão
\end{minipage} & \begin{minipage}[b]{\linewidth}\raggedright
\(F(r)\)
\end{minipage} & \begin{minipage}[b]{\linewidth}\raggedright
\(G(r)\)
\end{minipage} & \begin{minipage}[b]{\linewidth}\raggedright
\(J(r)\)
\end{minipage} & \begin{minipage}[b]{\linewidth}\raggedright
\(K(r)\)
\end{minipage} & \begin{minipage}[b]{\linewidth}\raggedright
\(L(r)\)
\end{minipage} & \begin{minipage}[b]{\linewidth}\raggedright
\(g(r)\)
\end{minipage} \\
\midrule\noalign{}
\endfirsthead
\toprule\noalign{}
\begin{minipage}[b]{\linewidth}\raggedright
\(\downarrow\) Padrão
\end{minipage} & \begin{minipage}[b]{\linewidth}\raggedright
\(F(r)\)
\end{minipage} & \begin{minipage}[b]{\linewidth}\raggedright
\(G(r)\)
\end{minipage} & \begin{minipage}[b]{\linewidth}\raggedright
\(J(r)\)
\end{minipage} & \begin{minipage}[b]{\linewidth}\raggedright
\(K(r)\)
\end{minipage} & \begin{minipage}[b]{\linewidth}\raggedright
\(L(r)\)
\end{minipage} & \begin{minipage}[b]{\linewidth}\raggedright
\(g(r)\)
\end{minipage} \\
\midrule\noalign{}
\endhead
\bottomrule\noalign{}
\endlastfoot
AEC & \(F(r)\equiv F_{pois}(r)\) & \(G(r) \equiv G_{pois}(r)\) &
\(J(r) \equiv 1\) & \(K(r) \equiv K_{pois}(r)\) &
\(L(r) \equiv L_{pois}(r)\) & \(g(r) \equiv 1\) \\
Agrupado & \(F(r)<F_{pois}(r)\) & \(G(r)>G_{pois}(r)\) & \(J(r)<1\) &
\(K(r)>K_{pois}(r)\) & \(L(r)>L_{pois}(r)\) & \(g(r)>1\) \\
Regular & \(F(r)>F_{pois}(r)\) & \(G(r)<G_{pois}(r)\) & \(J(r)>1\) &
\(K(r)<K_{pois}(r)\) & \(L(r)<L_{pois}(r)\) & \(g(r)<1\) \\
Padrão \(\uparrow\) & \(F_{inhom}(r)\) & \(G_{inhom}(r)\) &
\(J_{inhom}(r)\) & \(K_{inhom}(r)\) & \(L_{inhom}(r)\) &
\(g_{inhom}(r)\) \\
\end{longtable}

Em um processo pontual, a presença de tendência pode levar as funções
apresentadas na seção Seção~\ref{sec-2.6.1} a identificar dependência
espacial entre eventos (padrão regular ou agrupado), mesmo que essa
dependência não seja devido à afinidade entre os eventos ou à influência
de um evento \(s_{i}\) sobre a presença de outro evento \(s_{j}\). A
dependência espacial identificada pode resultar da competição por
recursos (como nutrientes no solo), e na ausência desses recursos, os
eventos não apresentariam dependência espacial. Uma forma eficiente de
investigar o padrão de distribuição espacial é a inclusão de envelopes
de simulação nas funções \(F(r)\), \(G(r)\), \(J(r)\), \(K(r)\),
\(L(r)\) e \(g(r)\) ou suas extensões não homogêneas. Os envelopes
representam um intervalo mínimo e máximo dentro do qual é aceitável
assumir a existência ou não de dependência espacial. Se as funções
\(F(r)\), \(G(r)\), \(J(r)\), \(K(r)\), \(L(r)\) e \(g(r)\) ou suas
extensões não homogêneas estiverem totalmente dentro do envelope, mas
acima ou abaixo das funções \(F_{pois}(r)\), \(G_{pois}(r)\),
\(J_{pois}(r)\), \(K_{pois}(r)\), \(L_{pois}(r)\) e \(g_{pois}(r)\),
isso indicará a presença de tendência (
Figura~\ref{fig-envelopes-simulacao} (a)). No entanto, se pelo menos uma
parte dessas funções estiver fora do envelope (
Figura~\ref{fig-envelopes-simulacao} (b)), mantêm-se todas as regras de
interpretação apresentadas no Quadro Tabela~\ref{tbl-1}.

\begin{Shaded}
\begin{Highlighting}[]
\FunctionTok{set.seed}\NormalTok{(}\DecValTok{123}\NormalTok{)}
\NormalTok{janela }\OtherTok{\textless{}{-}} \FunctionTok{owin}\NormalTok{(}\FunctionTok{c}\NormalTok{(}\DecValTok{0}\NormalTok{, }\DecValTok{1}\NormalTok{), }\FunctionTok{c}\NormalTok{(}\DecValTok{0}\NormalTok{, }\DecValTok{1}\NormalTok{))}


\NormalTok{pp\_trend }\OtherTok{\textless{}{-}} \FunctionTok{rpoispp}\NormalTok{(}\AttributeTok{lambda =} \ControlFlowTok{function}\NormalTok{(x,y) \{ }\DecValTok{100} \SpecialCharTok{*}\NormalTok{ x \}, }\AttributeTok{win =}\NormalTok{ janela)}

\CommentTok{\# (b) Dependência Espacial (Padrão Agrupado)}
\NormalTok{pp\_cluster }\OtherTok{\textless{}{-}} \FunctionTok{rThomas}\NormalTok{(}\AttributeTok{kappa =} \DecValTok{10}\NormalTok{, }\AttributeTok{scale =} \FloatTok{0.05}\NormalTok{, }\AttributeTok{mu =} \DecValTok{5}\NormalTok{, }\AttributeTok{win =}\NormalTok{ janela)}

\CommentTok{\#Cálculo dos Envelopes (Monte Carlo)}

\NormalTok{calc\_envelope }\OtherTok{\textless{}{-}} \ControlFlowTok{function}\NormalTok{(pp) \{}
\NormalTok{  env }\OtherTok{\textless{}{-}} \FunctionTok{envelope}\NormalTok{(pp, Lest, }\AttributeTok{nsim =} \DecValTok{39}\NormalTok{, }\AttributeTok{rank =} \DecValTok{1}\NormalTok{, }
                  \AttributeTok{correction =} \StringTok{"iso"}\NormalTok{, }\AttributeTok{global =} \ConstantTok{FALSE}\NormalTok{, }
                  \AttributeTok{verbose =} \ConstantTok{FALSE}\NormalTok{, }\AttributeTok{savefuns =} \ConstantTok{FALSE}\NormalTok{)}
  \FunctionTok{as.data.frame}\NormalTok{(env)}
\NormalTok{\}}

\NormalTok{df\_env\_trend }\OtherTok{\textless{}{-}} \FunctionTok{calc\_envelope}\NormalTok{(pp\_trend)}
\NormalTok{df\_env\_cluster }\OtherTok{\textless{}{-}} \FunctionTok{calc\_envelope}\NormalTok{(pp\_cluster)}

\NormalTok{plot\_env\_custom }\OtherTok{\textless{}{-}} \ControlFlowTok{function}\NormalTok{(df, titulo, subtitulo) \{}
\NormalTok{  df }\OtherTok{\textless{}{-}}\NormalTok{ df[df}\SpecialCharTok{$}\NormalTok{r }\SpecialCharTok{\textless{}=} \FloatTok{0.25}\NormalTok{, ]}
  
  \FunctionTok{ggplot}\NormalTok{(df, }\FunctionTok{aes}\NormalTok{(}\AttributeTok{x =}\NormalTok{ r)) }\SpecialCharTok{+}
    \FunctionTok{geom\_ribbon}\NormalTok{(}\FunctionTok{aes}\NormalTok{(}\AttributeTok{ymin =}\NormalTok{ lo, }\AttributeTok{ymax =}\NormalTok{ hi), }\AttributeTok{fill =} \StringTok{"grey70"}\NormalTok{, }\AttributeTok{alpha =} \FloatTok{0.5}\NormalTok{) }\SpecialCharTok{+}
    \FunctionTok{geom\_line}\NormalTok{(}\FunctionTok{aes}\NormalTok{(}\AttributeTok{y =}\NormalTok{ theo, }\AttributeTok{linetype =} \StringTok{"Teórico (CSR)"}\NormalTok{), }
              \AttributeTok{color =} \StringTok{"red"}\NormalTok{, }\AttributeTok{size =} \FloatTok{0.8}\NormalTok{) }\SpecialCharTok{+}
    \FunctionTok{geom\_line}\NormalTok{(}\FunctionTok{aes}\NormalTok{(}\AttributeTok{y =}\NormalTok{ obs, }\AttributeTok{linetype =} \StringTok{"Observado"}\NormalTok{), }
              \AttributeTok{color =} \StringTok{"black"}\NormalTok{, }\AttributeTok{size =} \DecValTok{1}\NormalTok{) }\SpecialCharTok{+}
    
    \FunctionTok{scale\_linetype\_manual}\NormalTok{(}\AttributeTok{name =} \StringTok{""}\NormalTok{, }
                          \AttributeTok{values =} \FunctionTok{c}\NormalTok{(}\StringTok{"Observado"} \OtherTok{=} \StringTok{"solid"}\NormalTok{, }\StringTok{"Teórico (CSR)"} \OtherTok{=} \StringTok{"dashed"}\NormalTok{)) }\SpecialCharTok{+}
    
    \FunctionTok{labs}\NormalTok{(}\AttributeTok{title =}\NormalTok{ titulo, }
         \AttributeTok{subtitle =}\NormalTok{ subtitulo,}
         \AttributeTok{x =} \StringTok{"Distância (r)"}\NormalTok{, }\AttributeTok{y =} \StringTok{"L(r)"}\NormalTok{) }\SpecialCharTok{+}
    \FunctionTok{theme\_minimal}\NormalTok{() }\SpecialCharTok{+}
    \FunctionTok{theme}\NormalTok{(}\AttributeTok{legend.position =} \StringTok{"bottom"}\NormalTok{,}
          \AttributeTok{plot.title =} \FunctionTok{element\_text}\NormalTok{(}\AttributeTok{size =} \DecValTok{11}\NormalTok{, }\AttributeTok{hjust =} \FloatTok{0.5}\NormalTok{))}
\NormalTok{\}}


\NormalTok{p1 }\OtherTok{\textless{}{-}} \FunctionTok{plot\_env\_custom}\NormalTok{(df\_env\_trend, }
                      \StringTok{"(a) Sem Dependência (Tendência)"}\NormalTok{, }
                      \StringTok{"Padrão Poisson Não{-}Homogêneo}\SpecialCharTok{\textbackslash{}n}\StringTok{(Mantém{-}se no envelope ou desvio suave)"}\NormalTok{)}

\NormalTok{p2 }\OtherTok{\textless{}{-}} \FunctionTok{plot\_env\_custom}\NormalTok{(df\_env\_cluster, }
                      \StringTok{"(b) Com Dependência (Agrupado)"}\NormalTok{, }
                      \StringTok{"Padrão Cluster (Thomas)}\SpecialCharTok{\textbackslash{}n}\StringTok{(Rompe o envelope significativamente)"}\NormalTok{)}

\NormalTok{p1 }\SpecialCharTok{+}\NormalTok{ p2}
\end{Highlighting}
\end{Shaded}

\begin{figure}[H]

\centering{

\pandocbounded{\includegraphics[keepaspectratio]{point_process_files/figure-pdf/fig-envelopes-simulacao-1.pdf}}

}

\caption{\label{fig-envelopes-simulacao}Diagnóstico com Envelopes de
Simulação (L-function): (a) Tendência (Poisson Não-Homogêneo) - curva
contida ou limítrofe; (b) Dependência (Cluster) - curva rompe o
envelope.}

\end{figure}%

Como pode ser observado na Figura~\ref{fig-envelopes-simulacao} (a), a
função \(\hat{J}(r)\) está abaixo de \(J_{pois}(r)\) e, se não houvesse
envelope (área cinza), dir-se-ia que existe dependência espacial e o
padrão seria agrupado, conforme descrito na seção
Seção~\ref{sec-2.6.1.3} e no Quadro Tabela~\ref{tbl-1}. No entanto,
apesar de \(\hat{J}(r)\) estar abaixo de \(J_{pois}(r)\), como está
totalmente dentro do envelope, isso significa que os eventos não
apresentam dependência espacial (não possuem afinidade natural), mas sim
tendência, ou seja, o padrão observado sem considerar o envelope é
resultado da tendência, provavelmente porque os eventos (por exemplo,
espécies florestais) estão competindo por um recurso que está restrito a
uma sub-região.

Na Figura~\ref{fig-envelopes-simulacao} (b), se não houvesse envelope
(área cinza), haveria dependência espacial e o padrão seria agrupado.
Com a incorporação do envelope, a dependência espacial permanece e o
padrão continua agrupado, indicando que os eventos possuem uma afinidade
natural que não é influenciada por fatores externos (como recursos).

\section{Modelos de processos pontuais}\label{sec-2.7}

Identificar o padrão de distribuição (estrutura de dependência) ou
tendência espacial dos eventos em um processo pontual é importante, mas
por si só não fornece uma análise completa dos possíveis fatores
(variáveis) que influenciam essa estrutura observada. É necessário
modelar essa distribuição dos eventos, tomando (ou não) como covariáveis
os possíveis fatores que podem ajudar a explicar cada padrão observado.
Assim, em processos pontuais, existem três classes de modelos mais
utilizadas, que são: modelos Poisson, modelos Cox e modelos Gibbs.

\subsection{Modelos Poisson}\label{sec-2.7.1}

Os modelos Poisson são utilizados quando não há dependência espacial
entre os eventos, ou seja, o padrão não é regular nem agrupado, como
ocorre na presença de tendência ou padrão aleatório. Os modelos Poisson
podem ser divididos em duas categorias principais, que são: modelo
Poisson homogêneo e modelo Poisson não homogêneo (Isham 2010; Scalon
2024).

\textbf{Modelo Poisson homogêneo}

O modelo Poisson homogêneo é utilizado quando não existem locais com
mais eventos em relação aos outros na região de estudo (intensidade
constante), ou seja, é um modelo utilizado quando o padrão é aleatório,
sinônimo da não existência da tendência.

No modelo Poisson homogêneo, os eventos são independentes e
identicamente distribuídos, com função intensidade denotada por
\(\lambda (s) = \beta\) e função densidade de probabilidade por

\begin{equation}\phantomsection\label{eq-26.}{
f(s)= \beta^{n(s)}\exp\{\left(1-\beta\right)|B|\},
}\end{equation}

onde,\(\beta\) é uma constante que representa a intensidade
(\(\lambda\)), \(n(s)\) )é o número de eventos existentes, B é a região
de estudo e \(|B|\) é a respectiva área (Adrian Baddeley, Rubak, e
Turner 2015; Scalon 2024).

\textbf{Modelo Poisson não homogêneo}

O modelo Poisson não homogêneo é uma extensão do modelo Poisson
homogêneo e é utilizado quando existe tendência, ou seja, quando existem
locais com uma maior concentração de eventos em relação a outros na
região de estudo (Adrian Baddeley, Rubak, e Turner 2015; J. B. Illian
2019; Scalon 2024). Neste modelo, a intensidade não é constante, mas é
uma função determinística (fixa) e pode ser modelada utilizando diversas
covariáveis que podem explicar o padrão observado, ou seja,

\begin{equation}\phantomsection\label{eq-36}{
\lambda(s) = \exp \left[\alpha + \theta^{\top} Z(s)\right] = \exp \left[\alpha + \theta_{1} Z_{1}(s) + \theta_{2} Z_{2}(s) + \ldots + \theta_{n} Z_{n}(s)\right],
}\end{equation}

onde \(\alpha\) é intercepto, \(Z(s)\) é um vetor de covariáveis e
\(\theta^{\top}\) são parâmetros (coeficientes)
\(\theta_{1}, \theta_{2} , \ldots  , \theta_{n}\), associados às
covariáveis. Exemplos de covariáveis incluem características do solo,
altitude, precipitação, temperatura, diâmetro à altura do peito de
árvores, entre outras.

Na ausência de covariáveis, é possível utilizar as coordenadas
geográficas (latitude e longitude) ou transformá-las em coordenadas
cartesianas (x, y), conforme descrito por Adrian Baddeley, Rubak, e
Turner (2015) e Scalon (2024). Conforme descrito por Scalon (2024), o
uso das coordenadas cartesianas faz com que a função intensidade assuma
várias formas, que são, Linear:
\(\lambda_{\theta} (x, y) = \exp \left(\theta_{0} + \theta_{1} x + \theta_{2} y \right)\),
onde, \(\lambda_{\theta} (x, y)\) corresponde a função intensidade,
variando em função das coordenadas \((x, y)\) (possíveis covariáveis não
conhecidas) e \(\theta_{i=\{0,1,2\}}\) parâmetros a serem estimados.

Quadrática:
\(\lambda_{\theta} (x, y) = \exp\left(\theta_{0} + \theta_{1} x + \theta_{2} y + \theta_{3} xy + \theta_{4} x^{2} + \theta_{5} y^{2}\right)\).

Cúbica:
\(\lambda_{\theta} (x, y) = \exp\left(\theta_{0} + \theta_{1} x + \theta_{2} y + \theta_{3} xy + \theta_{4} x^{2} + \theta_{5} y^{2} + \theta_{6} x^{2}y +\theta_{7} xy^{2} + \theta_{8} x^{3} + \theta_{9} y^{3}\right )\),
ou n-ésima forma de maneira geral dada por

\begin{equation}\phantomsection\label{eq-32}{
\lambda_{\theta} (x, y) = \exp\left(\sum_{i=0}^{n} \sum_{i^{'}=0}^{n} \theta_{ii^{'}} x^{i}y^{i^{'}} \right ).
}\end{equation}

Esta classe de modelo Poisson apresenta a seguinte função densidade de
probabilidade,

\begin{equation}\phantomsection\label{eq-2.7.}{
f(s)=\beta(s_{1})\ldots \beta(s_{n})\exp{\left[\int_{B} \left(1-\beta(s)\right)dz\right]} = \prod_{i=1}^{n}\beta(s_{i})\exp{\left[\int_{B} \left(1-\beta(s)\right)dz\right]},
}\end{equation}

onde \(\beta=\lambda\), outrora designada intensidade dos eventos
\(s_{i}\) e \(B\) é a região de estudo.

\subsection{Modelos Cox}\label{sec-2.7.2}

Conforme descrito na seção Seção~\ref{sec-2.7.1}, a disposição dos
eventos na área de estudo pode ser influenciada por diversos fatores
(covariáveis), como os fatores ambientais. A escassez desses fatores em
certas partes da região de estudo pode resultar em uma menor propensão à
ocorrência dos eventos, criando uma tendência na distribuição espacial.
Como essa tendência é atribuída a fatores ambientais, é possível, por
meio do modelo Poisson não homogêneo, modelar a distribuição dos eventos
por área (intensidade) levando em consideração essas variáveis
ambientais. O modelo resultante é capaz de capturar (explicar) a
variabilidade na distribuição dos eventos observados.

No entanto, é comum que as covariáveis disponíveis não consigam explicar
completamente o padrão observado, porque, além da influência das
covariáveis, os eventos apresentam uma afinidade natural (dependência
espacial). Essa afinidade natural faz com que, ao ajustar um modelo
Poisson não homogêneo, este, não seja capaz de explicar completamente
variabilidade na distribuição dos eventos observados.

Considerando essa limitação, é possível, em um processo pontual, modelar
a função de intensidade como uma função aleatória \(\Lambda(s)\), que
além de considerar a influência das covariáveis, pressupõe a existência
de dependência espacial entre os eventos. Esse tipo de modelagem resulta
em um processo pontual duplamente estocástico (aleatoriedade na
intensidade e nos eventos), conhecido como modelo Cox.

Os modelos Cox são uma extensão dos modelos Poisson, nos quais a
intensidade originalmente denotada por \(\lambda(s)\) é expandida para
incorporar um campo aleatório \(\Psi(s)\), conforme apresentado na
Figura~\ref{fig-1ca}. Este campo aleatório torna a função de intensidade
aleatória e denotada por \(\Lambda(s)\), onde
\(\{\Lambda = \Lambda(s) : s \in  \mathbb{R}^2\}\).

Estes modelos são frequentemente utilizados para modelar padrões
pontuais que apresentam dependência espacial do tipo padrão agrupado (N.
Cressie 1993; Scalon 2024).

\begin{figure}

\centering{

\includegraphics[width=0.8\linewidth,height=\textheight,keepaspectratio]{Figures/Campoaleatório.png}

}

\caption{\label{fig-1ca}Ilustração do campo aleatório
(superfície)\(\Psi (s)\), que controla a abundância e distribuição dos
eventos em um modelo Cox. Fonte: Jalilian, Safari, e Sohrabi (2020)}

\end{figure}%

Na Figura~\ref{fig-1ca}, o plano representa a região de estudo onde os
eventos estão distribuídos. Os pontos no plano indicam a localização dos
eventos. Abaixo do plano, há um campo aleatório que influencia o padrão
de distribuição dos eventos no plano. As variações neste campo aleatório
refletem áreas com maiores ou menores quantidades da variável aleatória.
\%É importante observar que, nos modelos Cox, a dependência espacial
(agrupamento) pressupõe-se que ocorra exclusivamente entre os
descendentes do mesmo progenitor A. J. Baddeley, Van Lieshout, e Møller
(1996). Dentre as subclasses dos modelos Cox, incluem-se o modelo
Log-Cox Gaussiano e os modelos de Neyman-Scott, que podem ser Matérn,
Thomas, Cauchy e Variância gama.

\textbf{Modelo Log Cox Gaussiano}

O papel do campo aleatório apresentado na Figura~\ref{fig-1ca} é de
explicar a configuração espacial dos eventos, que não é explicada pelas
covariáveis (J. B. Illian 2019). Seja, \(\Psi (s)\) um campo aleatório
Gaussiano de média \(\mu\) e função covariância \(\gamma(r)\), ou seja,
uma função cujo valor em qualquer ponto é uma variável aleatória, que
explica a variabilidade espacial não explicada pelas covariáveis. Em
outras palavras, um campo aleatório \(\Psi (s)\) é gaussiano se, para
qualquer coleção finita de eventos \(s_{1}, s_{2}, \ldots , s_{n}\),
qualquer combinação linear
\(\beta_{1} \Psi_{s_{1}} +\beta_{2} \Psi_{s_{2}} + \ldots +\beta_{n} \Psi_{s_{n}}\),
com \(\beta_{i} \in  \mathbb{R}\), tem em uma distribuição normal
unidimensional (J. Illian et al. 2008).

Assim, sob essa condição, pode-se definir uma função intensidade
aleatória \(\Lambda(s)\), que além das covariáveis, incorpora um campo
aleatório Gaussiano, como,

\begin{equation}\phantomsection\label{eq-37}{
\Lambda(s)= Z(s)\beta^{T} + \Psi (s) .
}\end{equation}

No entanto, a expressão Eq.~\ref{eq-37} não pode ser diretamente
utilizada como função de intensidade em um modelo de Cox, pois o campo
aleatório \(\Psi(s)\) sendo Gaussiano pode assumir valores negativos, o
que não é admissível em processos pontuais (a intensidade deve ser
não-negativa) (Møller, Syversveen, e Waagepetersen 1998; Møller e
Waagepetersen 2007). Para resolver essa questão, uma transformação
apropriada é aplicar o logaritmo neperiano (ln), resultando no modelo
conhecido como Log-Cox Gaussiano, dado por,

\begin{equation}\phantomsection\label{eq-38}{
ln \left[\Lambda(s)\right] = Z(s)\beta^{T} + \Psi (s) \Leftrightarrow \Lambda(s)= \exp \{Z(s)\beta^{T} + \Psi (s)\}.
}\end{equation}

Em síntese, um modelo Cox será Log Cox Gaussiano se o campo aleatório
\(\Psi(s)\) for Gaussiano (possuir distribuição normal).

\subsubsection{Modelos de Neyman-Scott: Modelo Matérn, Thomas,
Variância-Gama e
Cauchy}\label{modelos-de-neyman-scott-modelo-matuxe9rn-thomas-variuxe2ncia-gama-e-cauchy}

Desenvolvidos por Neyman e Scott (1958), os modelos Neyman-Scott
constituem uma classe de modelos Cox utilizados para descrever padrões
de agrupamento quando se presume que os eventos observados são
descendentes de um ou mais eventos progenitores (Moller e Waagepetersen
2003; Adrian Baddeley, Rubak, e Turner 2015; Scalon 2024).

Os modelos de Neyman-Scott são compostos por duas etapas distintas. Na
primeira etapa, os eventos \(s_{i}\) (progenitores) são gerados, como
por exemplo árvores. Na segunda etapa, cada evento \(s_{i}\)
(progenitor) gera eventos \(s_{i^{'}}\) descendentes, como por exemplo
mudas. O padrão de eventos que consiste exclusivamente nos eventos
descendentes (mudas), independentemente de seus progenitores, forma
realização do processo de agrupamento Neyman-Scott. Dependendo da
distribuição espacial dos eventos descendentes em relação aos seus
progenitores, os modelos podem ser do tipo Thomas, Matérn,
variância-gama ou Cauchy.

\textbf{Modelo Matérn}

Um modelo Matérn é caracterizado pela distribuição de probabilidade
\(h(s_{i^{'}})\) dos descendentes em relação aos seus progenitores, onde
os descendentes, em média por progenitor, estão distribuídos
uniformemente em um círculo de raio r centrado no progenitor, ou seja,
\(h(s_{i^{'}}) = \frac{ \mathbb{I}[||s_{i}-s_{i^{'}}|| \leq r]}{\omega^{2} r^{2}}\),
onde \(\omega\) é um parâmetro (escala) de agrupamento dos descendentes
em relação aos progenitores \(s_{i}\) (Scalon 2024).

Para J. Illian et al. (2008), no modelo Matérn, a função intensidade
\(\Lambda (s_{i^{'}})\) pode ser denotada por\\
\(\Lambda (s_{i^{'}}) = \lambda (s_{i^{'}}) \sum_{s_{i^{'}}\in S}  \mathbb{I}_{b(s_{i^{'}}, \:r)} (s)\),
onde \(\lambda (s_{i^{'}})\) é a intensidade do \(i\)-ésimo aglomerado
formado pelos
descendentes,\(\sum_{\in s_{i^{'}}}  \mathbb{I}_{b(s_{i^{'}}, \:r)}(s)\)
é a soma sobre todos os eventos \(s_{i^{'}}\) no processo Poisson
homogêneo \(s\), \(\mathbb{I}_{b(s_{i^{'}}, \:r)} (s)\) é uma função
indicadora que é igual a 1 se o evento \(s_{i^{'}}\) estiver no círculo
de raio \(r\) centrado no progenitor, e 0 caso contrário.

Adrian Baddeley, Rubak, e Turner (2015) para a mesma função intensidade
propuseram a seguinte expressão,
\(\Lambda (s_{i^{'}}) = \sum_{i} h(s_{i^{'}}), \, h(s_{i^{'}}) = \left\{\begin{array}{rcl} \frac{s_{i^{'}}}{\pi r^{2}} & \text{ se } & ||s_{i^{'}}|| \leq r \\ 0 & \text{ se } & ||s_{i^{'}}|| > r \end{array} \right.\),
onde \(||\cdot||\) representa a norma ou distância entre os eventos
\(s_{{i}^{'}}\) e \(r\) distância de referência.

\textbf{Modelo Thomas}

Neste modelo, os eventos descendentes \(s_{{i}^{'}}\) estão distribuídos
ao redor dos eventos principais (progenitores) seguindo uma distribuição
normal simétrica, ou seja,

\begin{equation}\phantomsection\label{eq-42}{
h(s_{{i}^{'}}) = \frac{1}{\sqrt{2\pi \sigma^{2}}} e^{\frac{-||s_{{i}^{'}} ||^{2}}{2\sigma^{2}}}, \,s_{{i}^{'}} \sim N (0, \sigma^{2}),
}\end{equation}

onde \(\sigma\) é um parâmetro mede o padrão de agrupamento dos
descendentes em relação aos progenitores e \(||\cdot||\) representa a
norma ou distância entre os eventos \(s_{{i}^{'}}\) (Adrian Baddeley,
Rubak, e Turner 2015; Scalon 2024).

\textbf{Modelo Cauchy}

Neste modelo, a densidade de probabilidade \(h(s_{i^{'}})\) dos
descendentes é uma distribuição bivariada de Cauchy, dada por
\(h(s_{i^{'}})= \frac{1}{2\pi \omega^{2}} \left[1 + \frac{||s_{i^{'}}||^{2}}{\omega^{2}}\right]^{\frac{-3}{2}}\),
onde \(\omega\) é um parâmetro que mede o agrupamento (Jalilian, Guan, e
Waagepetersen 2013; Adrian Baddeley, Rubak, e Turner 2015; J. B. Illian
2019).

\textbf{Modelo Variância Gama}

Segundo Jalilian, Guan, e Waagepetersen (2013) e Adrian Baddeley, Rubak,
e Turner (2015), no modelo variância-gama, a densidade de probabilidade
dos descendentes é variância gama, dada por

\begin{equation}\phantomsection\label{eq-44}{
h(s_{{i}^{'}}) = \frac{1}{2^{\nu+1} \pi \eta^{2} \Gamma(\nu+1)} \frac{||s_{{i}^{'}}||^{\nu}}{\eta^{\nu}} Y_{\nu} \left(\frac{||s_{{i}^{'}}||}{\nu}\right),
}\end{equation}

onde \(\eta\) é o parâmetro de escala e \(\nu\) é um parâmetro adicional
que controla a forma da densidade e deve satisfazer
\(\nu>\frac{1}{2}\).\(\Gamma\) é função gama e \(Y_{\nu}\) é função
Bessel modificada de segunda espécie e ordem \(\nu\) que pode ser
encontrada no livro de Bell (2004).

Para o modelo variância gama, J. Illian et al. (2008), propõe uma
possível representação da função intensidade \(\Lambda (s_{{i}^{'}})\),
dada por
\(\Lambda (s_{{i}^{'}}) = \sum_{s_{{i}^{'}}\in S} \omega_{i} k_{z} (z-s_{{i}^{'}})\),
onde \(s\) representa o processo Poisson que gera uma coleção de eventos
\(s_{{i}^{'}}\) no espaço \(d\), \(\omega_{i}\) representa uma sequência
de variáveis aleatórias independentes e identicamente distribuídas
(iid), com distribuição gama. \(k_{z}\) representa a função de densidade
de probabilidade (Kernel) que modela a influência ou interação dos
eventos.

\subsection{Modelos Gibbs (Markov)}\label{modelos-gibbs-markov}

Os modelos apresentados nas seções Seção~\ref{sec-2.7.1} e
Seção~\ref{sec-2.7.2} são úteis para eventos que não apresentam
dependência espacial e para eventos que apresentam dependência espacial
do tipo padrão agrupado, respectivamente. Para eventos que apresentam
dependência espacial do tipo regular, é necessária uma classe diferente
de modelos denominada Gibbs.

Os modelos Gibbs podem ser subdivididos em modelos de interação
par-a-par e modelos de interação por área.

\textbf{Modelos de interação par a par}

Os modelos de interação par-a-par são representados pela densidade de
probabilidade dada por

\begin{equation}\phantomsection\label{eq-10}{
f(s)= \alpha \left[\prod_{i=1}^{n(s)} b(s_{i})\right]\left[\prod_{i<i^{'}} c(s_{i}, s_{i^{'}})\right] \quad \text{ou} \quad g(s) \propto \prod_{s_{i} \in s} \phi \left(s_{i}\right) \prod_{\{s_{i}, \,\eta\} \subseteq s} \phi \left(\{s_{i}, \eta\}\right),
}\end{equation}

onde \(\propto\) indica proporcionalidade e \(\alpha\) constante de
normalização (J. Illian et al. 2008; Peter J. Diggle 2013; Adrian
Baddeley, Rubak, e Turner 2015; Scalon et al. 2022; Scalon 2024). As
funções \(b(s_{i})\) e \(\phi (s_{i})\) desempenham o mesmo papel,
representando estatísticas de primeira ordem (\(\lambda\)) enquanto
\(c(s_{i}, s_{i^{'}})\) e \(\phi\left({s_{i}, \eta}\right)\), que também
poderiam ser expressos como \(\phi\left(||s_{i} -s_{i^{'}}||\right)\),
representam funções de interação par-a-par (Moller e Waagepetersen 2003;
Adrian Baddeley, Rubak, e Turner 2015; Scalon 2024).

Conforme J. Illian et al. (2008), a função de interação
\(c(s_{i}, s_{i^{'}})\), também conhecida como potencial par, um termo
originário da física, mede a ``energia potencial'' gerada pela interação
entre pares de eventos \((s_{i}, s_{i^{'}})\) em função de sua distância
\(||s_{i} - s_{i^{'}}||\).

Segundo Scalon (2024), os modelos Poisson descritos nas seções
Seção~\ref{sec-2.7.1} e Seção~\ref{sec-2.7.2}, apesar de não
apresentarem dependência espacial (interação), fazem parte dos modelos
Gibbs, pois, se \(b(s_{i})\equiv \lambda\) e \(c(s_{i}, s_{i^{'}})=1\)
na equação Eq.~\ref{eq-10}, obtém-se a função densidade de probabilidade
do modelo Poisson homogêneo, onde \(b(s_{i})=\beta\) e
\(\alpha=\exp\{\left(1-\beta\right)|B|\}\). De forma análoga, se
\(b(s_{i})\equiv \lambda (s_{i})\) e \(c(s_{i}, s_{i^{'}})=1\) na
equação Eq.~\ref{eq-10}, obtém-se a função densidade de probabilidade do
modelo Poisson não homogêneo.

Na prática, os modelos da classe Gibbs são frequentemente apresentados
com base na intensidade condicional \(\lambda(s_{i}|s_{i^{'}})\), que
pode ser interpretada como o número esperado de eventos \(s_{i}\) por
unidade de área (infinitesimalmente pequena) em uma localização, dada a
existência do evento \(s_{i^{'}}\).

No entanto, Scalon (2024) mostra que as funções densidade de
probabilidade \(f(s)\) e intensidade condicional
\(\lambda(s_{i}|s_{i^{'}})\) são relacionadas,

\begin{equation}\phantomsection\label{eq-2.7}{
\lambda(s_{i}|s_{i^{'}})=\frac{f\left(s \cup s_{i^{'}}\right)}{f(s_{i^{'}})}=b(s_{i})\left[\prod_{i\neq i^{'}}^{n} c(s_{i}, s_{i^{'}})\right]
}\end{equation}

e esta intensidade apresenta relação com a intensidade dos modelos
Poisson, se \(\prod_{i}^{n}c(s_{i}, s_{i^{'}})=1\) e \(b(s_{i})=\beta\),
obtém-se a intensidade do modelo Poisson homogêneo.

Embora modelos Gibbs sejam adequados para padrões regulares, podem
identificar padrões de atração (agrupamento) com baixa eficácia e sem
aplicação prática significativa (Adrian Baddeley, Rubak, e Turner 2015).

Em processos pontuais, ao utilizar modelos Gibbs de interação par-a-par,
diferentes modelos Gibbs podem ser obtidos dependendo da forma que a
função de interação \(c(s_i, s_{i'})\) assumir e alguns desses modelos
incluem modelos Hard Core (núcleo duro), modelo Strauss, Strauss Hard
Core, etc.

\textbf{Modelo Hard core}

Neste tipo de modelo da classe Gibbs considera-se que não existem
eventos que estejam mais próximos do que a distância mínima \(h\) e o
modelo só é útil na situação em que os eventos são centros de partículas
não elásticas esféricas ou circulares do mesmo tamanho, e \(h\) é o
diâmetro dessas partículas, que deve ser o mesmo para todos os eventos
(J. Illian et al. 2008; Adrian Baddeley, Rubak, e Turner 2015; Scalon
2024). Este modelo, é obtido considerando \(b(s) \equiv \beta\) e

\[c(s_{i}, s_{i^{'}})= \left\{\begin{array}{rcl} 1 & \text{se} & ||s_{i}-s_{i^{'}}||> h \\ 0 & \text{se} & ||s_{i}-s_{i^{'}}|| \leq h \end{array}\right.\],
na equação Eq.~\ref{eq-2.7}., o que resulta na função densidade de
probabilidade

\[f(x)= \left\{\begin{array}{rcl} \alpha \beta^{n(x)} & \text{se} & ||s_{i}-s_{i^{'}}||> h \quad \forall i\neq i^{'}\\ 0 \quad &\text{se} & \text{caso contrário} \end{array}\right.\]
e intensidade condicional

\[\lambda (s_{i}|s_{i^{'}}\setminus s_{i}) = \left \{\begin{array}{rcl} \beta & \text{se} & ||s_{i}- s_{i^{'}}||> h \quad \forall i\neq i^{'}\\ 0 &\text{se} & \text{caso contrário} \end{array}\right.\]

\textbf{Modelo de nascimento e morte}

É um modelo mais adequado para a sucessão florestal que envolve a morte
de árvores existentes em momentos aleatórios e a germinação de novas
mudas em lugares e tempos aleatórios (N. Cressie 1993; Adrian Baddeley,
Rubak, e Turner 2015).

Supondo que em cada intervalo de tempo \(\Delta t\), cada árvore
existente tenha probabilidade \(m \Delta t\) de morrer,
independentemente das outras árvores, onde \(m\) é taxa de mortalidade
por árvore e por unidade de tempo. Durante o mesmo intervalo, em uma
região \(\Delta a\), nova árvore germina com probabilidade
\(g \Delta a \Delta t\), se estiver a mais de \(d\) unidades de
distância da árvore mais próxima e considerando \(g\) a taxa de
germinação por unidade de tempo. Independentemente do estado inicial da
floresta, ao longo de tempo, este processo de nascimento e morte
espacial alcançará um equilíbrio no qual qualquer evento da floresta é
uma realização do processo de Hard Core com parâmetro
\(\beta = \frac{g}{m}\) e diâmetro de Hard Core \(h\).

Este modelo, é obtido considerando \(b(s) \equiv \beta=\frac{g}{m}\) e a
função densidade de probabilidade
\(f(x)= \left\{\begin{array}{rcl} \alpha (\frac{g}{m})^{n(s)} & \text{se} & ||s_{i}-s_{i^{'}}||> h\\ 0 \, &\text{se} & \text{caso contrário} \end{array}\right.\)
e
\(\lambda (s_{i}|s_{i^{'}}) = \left \{\begin{array}{rcl} (\frac{g}{m}) & \text{se} & ||s_{i}-s_{i^{'}}||> h\\ 0 &\text{se} & \text{caso contrário} \end{array}\right.\).

\textbf{Modelo Strauss}

Segundo Adrian Baddeley, Rubak, e Turner (2015) e Scalon (2024), o
modelo Hard Core é apropriado quando é fisicamente impossível que dois
eventos estejam a uma distância menor que \(d\). Se não é impossível que
dois ou mais eventos estejam próximos, mas é improvável, um modelo
apropriado é o modelo Strauss, obtido considerando
\(c(s_{i}, s_{i^{'}})= \left\{\begin{array}{rcl} 1 & \text{se} & ||s_{i}-s_{i^{'}}||> d \\ \gamma & \text{se} & ||s_{i}-s_{i^{'}}|| \leq d \end{array}\right.\)
e
\(f(s)= \alpha \beta^{n(s)} \gamma^{n(s, d)} \, \text{e} \, \lambda (s_{i}|s_{i^{'}})=\beta \gamma^{n(s_{i}, d,s_{i^{'}})}\),
onde \(n(s,d)\) é o número de pares não ordenados de eventos em \(s\)
que estão mais próximos do que a distância de interação \(d\);
\(n(s_{i}, d,s_{i^{'}})=n(s,d)-n(s_{i^{'}},d)\) é o número de eventos
vizinhos de \(s_{i}\) excluindo de \(s_{i^{'}}\) e \(\gamma\) um
parâmetro de interação.

Se \(0<\gamma<1\) o modelo caracteriza inibição; se \(\gamma =0\) e não
existirem eventos próximos do que a distância \(d\), o modelo passa a
ser de Hard Core; se \(\gamma=1\), o modelo é Poisson homogêneo e o
modelo não está definido se \(\gamma>1\) pois, a função densidade
\(f(s)\) não é integrável.

\textbf{Modelo Strauss Hard Core}

Neste modelo, é impossível que dois ou mais eventos estejam a uma
distância menor que a distância de Hard Core (\(h\)), mas eles podem
apresentar interação a uma distância \(d > h\), onde \(d\) é a distância
de interação de Strauss. A função de interação é expressa por:

\(c(s_{i}, s_{i^{'}})= \left\{\begin{array}{rcl} 0 & \text{se} & ||s_{i}-s_{i^{'}}||\leq h \\ \gamma & \text{se} & h<||s_{i}-s_{i^{'}}|| \leq d \\ 1 & \text{se} & ||s_{i}-s_{i^{'}}||>d \end{array}\right.\).

\textbf{Modelo de interação soft-core}

Segundo Okabe e Tanemura (2006) observamos frequentemente padrões
regulares de eventos no mundo natural e nesse padrão, ocorre um
espaçamento entre os \(i\)-ésimos eventos, que pode ser devido à
competição entre os eventos por territórios, nutrientes e assim por
diante. Para representar o alcance e a suavidade das interações, os
chamados potenciais Soft-Core são os mais convenientes. Neste modelo, a
interação entre os eventos é suave e diminui gradualmente à medida que a
distância \(d\) entre os eventos aumenta. Em contraste com o modelo de
Hard core, onde a interação abruptamente se torna zero após uma certa
distância, e no modelo de Strauss, onde a interação é constante em uma
certa distância e zero fora dela. A função interação é dado por:
\(c(s_{i}, s_{i^{'}})= \left(\frac{\sigma}{||s_{i}-s_{i^{'}}||}\right)^{\frac{2}{\kappa}}, \;\sigma>0 \; e\;0<\kappa<1\).

\textbf{Modelo de interação Diggle-Gates-Stibbard}

O modelo proposto por Peter J. Diggle, Gates, e Stibbard (1987) é um
modelo de processo pontual com interação par a par, útil para modelar
padrões onde a interação entre os pontos é influenciada pela distância
entre eles. Este modelo é especialmente adequado quando há evidências de
que a intensidade da interação entre os pontos diminui à medida que a
distância entre eles aumenta. A função interação é dado por:
\(c(s_{i}, s_{i^{'}})= \left\{\begin{array}{rcl} \sin\left(\frac{\pi||s_{i}-s_{i^{'}}||}{2d}\right)^{2} & \text{se} & ||s_{i}-s_{i^{'}}||\leq d \\ 1 \quad \quad \quad \quad & \text{se} & ||s_{i}-s_{i^{'}}||>d \end{array}\right.\).

\textbf{Modelo de interação Diggle-Gratton}

Proposto por Peter J. Diggle e Gratton (1984), o modelo Diggle-Gratton é
um tipo de modelo estatístico utilizado para descrever a distribuição
espacial de eventos pontuais. A função de potencial de par nesse modelo
é definida em termos da distância entre pares de pontos, e quantifica
como a presença de um evento afeta a probabilidade de encontrar outro
ponto em sua vizinhança. Os parâmetros do modelo incluem a distância
mínima de interação \(h\) (também conhecida como ``hard core''), a
distância máxima de interação \(d\), e um parâmetro \(\kappa\) que
controla a força da interação. A função interação é dado por:

\(c(s_{i}, s_{i^{'}})= \left\{\begin{array}{rcl} 0 \quad \quad \quad & \text{se} & ||s_{i}-s_{i^{'}}||\leq h \\ \left(\frac{||s_{i}-s_{i^{'}}||- h}{d-h}\right)^{\kappa} & \text{se} & h<||s_{i}-s_{i^{'}}|| \leq d \\ 1 \quad \quad \quad & \text{se} & ||s_{i}-s_{i^{'}}||>d \end{array}\right.\).

\subsection{Modelos de interação por
área}\label{modelos-de-interauxe7uxe3o-por-uxe1rea}

Até agora, os modelos Gibbs (Markov) que foram discutidos são conhecidos
como modelos de interação par-a-par. Eles descrevem interações
intraespecíficas (eventos do mesmo tipo) no caso univariado e
interespecíficos espécies (eventos diferentes) no caso multivariado. No
entanto, além desses modelos, existem aqueles que consideram a interação
por área, os quais são mais adequados quando se considera que os eventos
podem competir por recursos, como nutrientes ou alimentos, em uma
determinada região.

Conforme descrito em Peter J. Diggle (2013), Adrian Baddeley, Rubak, e
Turner (2015) e Nightingale et al. (2019), os modelos de interação por
área, também conhecidos como modelos de esfera penetrável de equilíbrio
líquido-vapor de Widom-Rowlinson, foram propostos por A. J. Baddeley e
Van Lieshout (1995), visando identificar eventos que possam exibir tanto
inibição quanto agregação.

Segundo A. J. Baddeley e Van Lieshout (1995), a distinção fundamental
entre os modelos de interação par-a-par e de área reside na
especificação da função de interação para cada tipo de modelo.

Nos modelos de interação par-a-par, a função de interação é definida
como uma função da distância euclidiana \(\left(||\cdot||\right)\) entre
cada par de eventos no padrão em estudo. Em contrapartida, a função de
interação em um modelo de interação de área é descrita como a área da
união de círculos associados a cada evento em um processo pontual.

A função densidade de probabilidade para este modelo é expressa, por

\[
f(x)=\alpha\beta^{n(s)}\gamma^{-A(s, r)}, \; \text{onde} \; A(s, r)=\left| B \cap \bigcup_{i=1}^{n(s)} b(s_{i},r) \right|,
\]

onde \(\alpha\) é uma constante,\(\beta >0\) é um parâmetro de
intensidade,\(\gamma>0\) é um parâmetro de interação, e \(A(s, r)\)
denota a área da região obtida ao desenhar um círculo de raio \(r\)
centrado em cada evento \(s_{i}\) e unir esses círculos.

É importante observar que a função de densidade para o modelo de
interação por área é integrável para todos os valores de \(\gamma\), ao
passo que, para modelos de interação par-a-par, como o modelo Strauss,
para \(\gamma>1\) a função não é integrável. Porém, para ambos modelos,
quando \(\gamma =1\), temos um modelo Poisson. Se \(\gamma<1\), os
eventos apresentam inibição ou regularidade, enquanto para \(\gamma>1\),
os eventos exibem aglomeração ou atração nos modelos de interação por
área.

Além das subclasses dos modelos Gibbs mencionadas anteriormente, existem
outros que surgem da combinação desses modelos em uma única classe,
denominada modelos híbridos. Além disso, existem modelos conhecidos como
modelos de interação tripla, nos quais três eventos interagem
simultaneamente. Outra possibilidade é o modelo de saturação de Geyer,
uma variação do modelo Strauss, porém com uma restrição \(\gamma<1\). No
entanto, esses modelos não estão no escopo destas notas de aula, podendo
consultar Adrian Baddeley, Rubak, e Turner (2015) para mais detalhes.

\section{Diagnóstico e Validação de
Modelos}\label{diagnuxf3stico-e-validauxe7uxe3o-de-modelos}

\subsection{Diagnóstico}\label{diagnuxf3stico}

O diagnóstico dos modelos pode ser feito utilizando diversos tipos de
testes estatísticos, a saber:

\begin{enumerate}
\def\labelenumi{\alph{enumi})}
\tightlist
\item
  \textbf{Teste da razão de verossimilhança} - este teste, é útil na
  presença de covariáveis e avalia se há ou não influência das
  covariáveis no padrão de distribuição dos eventos em estudo. Suas
  hipóteses nulas e alternativas são, respectivamente,
  \(H_{0}: \varphi = 0\) e \(H_{a}: \varphi \neq 0\), onde \(\varphi\) é
  o vetor de covariáveis. A estatística do teste é dada por
\end{enumerate}

\begin{equation}\phantomsection\label{eq-101}{\Gamma = 2 \log \frac{L_{1}}{L_{0}} = 2 \left( \log L_{1} - \log L_{0} \right),}\end{equation}

onde \(L_{0}\) e \(L_{1}\) são os valores máximos da verossimilhança sob
as hipóteses nula e alternativa, respectivamente. Para decidir sobre a
aceitação ou rejeição da hipótese nula de não influência das covariáveis
no padrão observado, utiliza-se um teste \(\chi^{2}\) com \(n\) graus de
liberdade, onde \(n\) é a dimensão do vetor das covariáveis
(\(\varphi\)).

\begin{enumerate}
\def\labelenumi{\alph{enumi})}
\setcounter{enumi}{1}
\tightlist
\item
  \textbf{Teste Escore} - também designado teste de multiplicação de
  Lagrange, requer e apenas um modelo considerado de referência, podendo
  ser Poisson homogêneo \(k(r)=\pi r^{2}\), representando aleatoriedade
  espacial completa, com parâmetros \(\theta\) associados a cada
  coordenada x, y. O teste de hipóteses para os parâmetros é:
  \(H_{0}: \theta = \theta_{0}\) e \(H_{a}: \theta \neq \theta_{0}\),
  cuja estatística de teste é:
\end{enumerate}

\[S=U(\theta_{0}) I_{\theta}^{-1}U(\theta_{0}), \: \text{ se } \theta = \psi \text{ ou } S=U( \theta)_{\varphi}^{\top} \left( I_{\theta}^{-1}\right)_{\varphi \varphi}U( \theta_{0})_{\varphi}, \text{ se } \theta=(\phi, \varphi)\],

onde \(U(\theta_{0})\) é vetor escore e no caso uniparamétrico é
designada função escore, \(I_{\theta}\) é informação de
Fisher\footnote{A informação de Fisher é uma medida estatística que
  quantifica a quantidade de informação que uma variável aleatória
  contém sobre um parâmetro desconhecido de uma distribuição, sendo
  definida como a variância do escore, ou seja, a derivada parcial do
  logaritmo natural da função de verossimilhança em relação ao parâmetro
  e indica o quão bem podemos estimar um parâmetro com base nos dados
  observados (Casella e Berger 2001).} e os subscritos \(\varphi\) e
\(\varphi \varphi\), são componentes do vetor score e matriz inversa de
informação de Fisher respectivamente.

\begin{enumerate}
\def\labelenumi{\alph{enumi})}
\setcounter{enumi}{2}
\tightlist
\item
  \textbf{Teste \(\chi^{2}\)} - o teste \(\chi^{2}\) pode ser utilizado
  tanto para testar a hipótese nula de aleatoriedade espacial completa
  quanto para avaliar a qualidade de ajuste de um modelo. No contexto de
  ajuste de modelos, o teste avalia a hipótese nula de que os eventos
  seguem um modelo espacial específico ajustado em relação à hipótese de
  que não o seguem. Aqui, ao utilizar o teste qui-quadrado
  \(\chi^{2}\),\(m-p\) serão os graus de liberdade, onde \(p\) é número
  de parâmetros do modelo ajustado e \(m\) número de \emph{quadrats} e a
  estatística de teste é,
\end{enumerate}

\begin{equation}\phantomsection\label{eq-104}{
\chi^{2}=\sum_{j} \frac{\left(n_{j}-\hat{\mu}_{j}\right)^{2}}{\hat{\mu}_{j}}, \, \hat{\mu}_{j} = \int_{B_{j}} \hat{\lambda} (s) ds ,
}\end{equation}

onde \(n_{j}\) e \(\hat{\mu}_{j}\) representam o número observado e
esperado de eventos no local \(j\) da área de estudo \(B_{j}\).

\begin{enumerate}
\def\labelenumi{\alph{enumi})}
\setcounter{enumi}{3}
\tightlist
\item
  \textbf{Simulações de Monte Carlo} - realizadas de forma análoga ao
  teste contra a hipótese de completa aleatoriedade espacial, entretanto
  os eventos em estudo são simulados múltiplas vezes usando o modelo
  ajustado ou um modelo de referência. Se for difícil distinguir
  visualmente a diferença entre os eventos observados e simulados, o
  modelo é considerado ajustado ao padrão dos eventos observados (Scalon
  2024).
\end{enumerate}

Para evitar equívocos entre tendência e dependência espacial, é
importante incorporar envelopes de simulação. Segundo Adrian Baddeley,
Rubak, e Turner (2015), para a incorporação dos envelopes de simulação,
primeiro calcula-se a estimativa \(s\) para os dados observados
\(\hat{S}_{obs}\), onde \(s\) representa qualquer modelo tomado como
referência. Em seguida, usando a função \(s\) como referência, gera-se
\(m\) padrões de eventos por simulação de Monte Carlo, e obtêm-se as
suas estimativas
\(\{\hat{S}_{1} (r), \hat{S}_{2} (r), \ldots, \hat{S}_{m} (r)\}\), na
distância \(r\) tomada como referência. Considerando \(L_{inf}\) e
\(L_{sup}\) como os limites inferior e superior das estatísticas
associadas aos padrões simulados nas \(r\)-ésimas distâncias, rejeita-se
a hipótese nula de que os eventos observados seguem o padrão do modelo
tomado como referência se as estimativas de \(\hat{S}_{obs}\) para as
mesmas distâncias estiverem totalmente ou parcialmente acima do limite
superior, ou abaixo do limite inferior, ou seja,

\[
\begin{aligned}
L_{\text{inf}}(r) &= {\underset{j}{\max}}\, S_{j}(r) = \max\{S_{1}(r), \ldots, S_{m}(r)\} \\
L_{\text{sup}}(r) &= {\underset{j}{\min}}\, S_{j}(r) =\min\{S_{1}(r), \ldots, S_{m}(r)\} .
\end{aligned}
\]

\subsection{Validação}\label{validauxe7uxe3o}

Muitos dos métodos de validação de modelos usados na estatística
clássica podem ser utilizados na validação de modelos para configurações
pontuais, incluindo a análise de resíduos e a avaliação da independência
por meio do gráfico dos quantis dos resíduos (Q-Q plot). Estes métodos
são aplicáveis a todos os modelos apresentados, exceto aos modelos de
Cox, que não permitem a análise dos resíduos.

Para modelos Poisson, os resíduos são dados pela seguinte expressão:

\[
R(B) = n(S\cap B) - \int_{B}\hat{\lambda}(s)ds,
\]

onde \(n(S\cap B)\) representa o número de eventos existentes em toda
área de estudo e \(\int_{B}\hat{\lambda}(s)ds\) representa o número
esperado de eventos na região \(B\), sendo \(\hat{\lambda}(s)\) a
respectiva intensidade estimada pelo modelo ajustado ou função
intensidade (modelo ajustado). Assim, se \(R(B)\) for negativo, o modelo
superestima a intensidade. Se \(R(B)\) for positivo, o modelo subestima
e se for zero, ou aproximadamente, o modelo se ajusta ao padrão dos
eventos em estudo. Os resíduos podem ser estimados considerando subáreas
\(i\) (\emph{quadrats} \(i\)) da região de estudo, resultando nos
conhecidos resíduos de Pearson, dados por:

\[
R_{i}^{P} (B)=\frac{n(S\cap B)-\int_{B}\hat{\lambda}(s)ds}{\sqrt{\int_{B}\hat{\lambda}(s)ds}},
\]

conforme descrito em Adrian Baddeley et al. (2005), Adrian Baddeley
(2007), Adrian Baddeley et al. (2008), Adrian Baddeley (2010), Adrian
Baddeley, Chang, et al. (2013), AJ Baddeley et al. (2019) e Adrian
Baddeley, Rubak, e Turner (2015), podendo ser suavizados ou cumulativos.

Para modelos de Cox, como descrito por Adrian Baddeley (2010), uma vez
que não é possível testar os resíduos, uma alternativa é recorrer ao
método dos momentos e a funções que capturam a estrutura de dependência
(F, G, J, K, L, g), juntamente com envelopes de simulação.

Segundo Adrian Baddeley et al. (2022), os métodos existentes para
ajustar modelos de processos de agrupamento (modelos Cox) a dados de
processos pontuais frequentemente enfrentam dificuldades em convergir,
convergem para valores implausíveis dos parâmetros ou apresentam
instabilidade numérica. Como resultado, vários métodos foram
desenvolvidos para lidar com esses problemas, mas as dificuldades
persistem. Alguns autores, como Waagepetersen e Guan (2009) e Adrian
Baddeley, Rubak, e Turner (2015), sugerem que o problema está
relacionado à fraca estrutura de agrupamento dos dados analisados. No
entanto, este problema pode ocorrer tanto em situações de forte
agrupamento quanto de agrupamento fraco (Adrian Baddeley et al. 2022).

A validação dos modelos Gibbs segue a mesma lógica dos Poisson, porém
substituindo a intensidade pela intensidade condicional e mantendo todas
regras, ou seja,

\[
R(B) = n(S\cap B) - \int_{B}\hat{\lambda}(s_{i}|s_{j})ds.
\]

Fora os resíduos, envelopes de simulação e funções que ajudam a
identificar a estrutura de dependência espacial, pode-se utilizar o
Critério de Informação de Akaike, dado por:

\[
AIC=-2 log L_{max}+2p ,
\]

onde \(L_{max}=L(\hat{\theta})\) é função de máxima verossimilhança para
o modelo em questão, e \(p\) é o número de parâmetros para este modelo.
Para AIC, o melhor modelo é aquele que tiver valor mais baixo de AIC.

\section{Pacote spatstat}\label{pacote-spatstat}

O pacote \texttt{spatstat} (Adrian Baddeley e Turner 2005; Adrian
Baddeley, Turner, et al. 2013; Adrian Baddeley, Rubak, e Turner 2015) é
o maior se não o principal pacote utilizado para análise estatística de
padrões pontuais espaciais. De autoria do professor
\href{https://scholar.google.com/citations?user=xM69PQ0AAAAJ&hl=en}{Adrian
Baddeley}, seu desenvolvimento é acompanhado pelo livro de referência
Adrian Baddeley, Rubak, e Turner (2015), que detalha os fundamentos
teóricos da área com aplicações práticas em R.

Atualmente, o \texttt{spatstat} é estruturado como um ecossistema de
subpacotes especializados:

\begin{itemize}
\item
  \texttt{spatstat.data:} Contém os conjuntos de dados.
\item
  \texttt{spatstat.geom:} Define as estruturas fundamentais de geometria
  espacial, como as janelas de observação (\texttt{owin}), padrões de
  pontos (\texttt{ppp}) e padrões de segmentos de linha (\texttt{psp}).
  É a base sobre a qual todos os outros subpacotes operam.
\item
  \texttt{spatstat.explore:} Focado na Análise Exploratória de Dados
  Espaciais (\texttt{ESDA}). Inclui funções para estimativa de densidade
  de kernel (intensidade de primeira ordem) e funções sumárias de
  segunda ordem, como as funções \(F, G, J, K, L\) e a função de
  correlação par (\(g\)).
\item
  \texttt{spatstat.model:} Contém funções para ajuste de modelos.
  Permite a modelagem da intensidade via Processos de Poisson
  Não-Homogêneos e a modelagem de interações via Processos de Gibbs
  (como os modelos de Strauss e Hard Core). Inclui também diagnósticos
  de resíduos e validação de modelos.
\item
  \texttt{spatstat.random:} Para simulação estocástica de processos
  pontuais. É essencial para a geração de envelopes de simulação (Testes
  de Monte Carlo) para validar se um padrão observado se desvia da
  aleatoriedade espacial completa (AEC).
\item
  \texttt{spatstat.linnet:} Dedicado à análise de padrões de pontos que
  ocorrem sobre redes lineares, como acidentes em rodovias ou crimes em
  redes de ruas urbanas.
\item
  \texttt{spatstat.univar:} Fornece ferramentas auxiliares para
  distribuições de probabilidade univariadas e técnicas de estimativa de
  densidade.
\item
  \texttt{spatstat.sparse:} Lida com computação matricial esparsa,
  otimizando o desempenho interno do pacote em cálculos complexos.
\item
  \texttt{spatstat.utils:} Conjunto de funções utilitárias internas para
  manipulação de dados, strings e diagnósticos de sistema.
\end{itemize}

\textbf{Instalação do pacote e importação}

\begin{Shaded}
\begin{Highlighting}[]
\NormalTok{pacman}\SpecialCharTok{::}\FunctionTok{p\_load}\NormalTok{(spatstat, spatstat.model,spatstat.data )}
\end{Highlighting}
\end{Shaded}

\section{\texorpdfstring{Criação, Manipulação e Geometria
(\texttt{spatstat.geom})}{Criação, Manipulação e Geometria (spatstat.geom)}}\label{criauxe7uxe3o-manipulauxe7uxe3o-e-geometria-spatstat.geom}

Esta seção cobre a construção dos objetos fundamentais. Sem a definição
correta da geometria, a estatística é impossível.

\textbf{Padrões de Pontos (\texttt{ppp}) e Janelas (`owin```)}

O objeto \texttt{ppp} requer coordenadas e uma janela \texttt{owin}.

\begin{itemize}
\tightlist
\item
  \textbf{Criação do objeto ppp}:
\end{itemize}

\begin{enumerate}
\def\labelenumi{\arabic{enumi}.}
\item
  \texttt{ppp(x,\ y,\ window,\ marks):} Construtor principal.
\item
  \texttt{as.ppp(X)}: Converte data.frames/matrizes.
\item
  \texttt{clickppp(n)}: (Interativo) Permite clicar no gráfico para
  adicionar pontos.
\end{enumerate}

\begin{itemize}
\tightlist
\item
  \textbf{Criação Janelas (owin)}:
\end{itemize}

\begin{enumerate}
\def\labelenumi{\arabic{enumi}.}
\item
  \texttt{owin(xrange,\ yrange,\ mask):} Cria retângulos (polígonos) ou
  máscaras binárias.
\item
  \texttt{square(s),\ disc(r,\ c),\ ellipse(a,\ b):} para a geração
  rápida de primitivas geométricas elementares (quadrados, discos e
  elipses).
\item
  \texttt{ripras(x,\ y):} Estimador de Ripley-Rasson. Realiza a
  inferência estatística da fronteira do domínio a partir da
  distribuição dos eventos, assumindo que a região original é convexa.
\item
  \texttt{convexhull(x,\ y):} Calcula o fecho convexo matemático,
  determinando o menor polígono convexo possível que engloba todos os
  eventos observados.
\end{enumerate}

\begin{itemize}
\tightlist
\item
  \textbf{Operações de Conjunto (Janelas)}
\end{itemize}

\begin{enumerate}
\def\labelenumi{\arabic{enumi}.}
\item
  \texttt{intersect.owin(A,\ B),\ union.owin(A,\ B),\ setminus.owin(A,\ B):}
  úteis para, respetivamente, realizar a interseção (extrair a área
  comum), a união (combinar as áreas totais) e a diferença setorial
  (excluir de uma janela a área sobreposta por outra) de domínios
  espaciais.
\item
  \texttt{inside.owin(x,\ y,\ w):} Teste de pertinência de ponto.
\item
  \texttt{erosion(w,\ r),\ dilation(w,\ r):} Encolhe ou expande a janela
  por raio \(r\).
\item
  \texttt{opening(w,\ r),\ closing(w,\ r):} Suaviza contornos
  (abertura/fechamento).
\item
  \texttt{marks(X)\ e\ marks(X)\ \textless{}-\ value}: Funções para a
  manipulação de marcas (covariáveis ou metadados). Permitem,
  respetivamente, extrair e atribuir informações adicionais a cada ponto
  do padrão espacial (ex: altura da árvore, tipo de espécie, diâmetro),
  transformando um processo puramente geométrico em um Processo Pontual
  Marcado.
\item
  \texttt{unmark(X):} Remove marcas.
\item
  \texttt{subset(X,\ subset):} Filtra pontos por lógica booleana.
\item
  \texttt{rotate(X,\ angle),\ shift(X,\ vec):} Rotação e Translação.
\item
  \texttt{flipxy(X),\ reflect(X):} Espelhamento.
\item
  \texttt{affine(X,\ mat):} Realiza uma transformação afim arbitrária
  sobre o objeto espacial X. Ao contrário das funções simples de rotação
  ou translação, esta função utiliza uma matriz linear \texttt{mat} para
  aplicar distorções geométricas complexas, como cisalhamento
  (\texttt{shearing}), escala anisotrópica (esticamento desigual entre
  eixos) e reflexão, preservando a colinearidade e as proporções
  relativas das distâncias ao longo de linhas paralelas.
\end{enumerate}

\begin{Shaded}
\begin{Highlighting}[]
\CommentTok{\# Construção de Janelas (Geometria Construtiva)}

\NormalTok{W\_ret }\OtherTok{\textless{}{-}} \FunctionTok{owin}\NormalTok{(}\AttributeTok{xrange =} \FunctionTok{c}\NormalTok{(}\DecValTok{0}\NormalTok{, }\DecValTok{10}\NormalTok{), }\AttributeTok{yrange =} \FunctionTok{c}\NormalTok{(}\DecValTok{0}\NormalTok{, }\DecValTok{10}\NormalTok{))}

\NormalTok{W\_circ }\OtherTok{\textless{}{-}} \FunctionTok{disc}\NormalTok{(}\AttributeTok{radius =} \DecValTok{3}\NormalTok{, }\AttributeTok{centre =} \FunctionTok{c}\NormalTok{(}\DecValTok{5}\NormalTok{, }\DecValTok{5}\NormalTok{))}

\NormalTok{W\_complexa }\OtherTok{\textless{}{-}} \FunctionTok{setminus.owin}\NormalTok{(W\_ret, W\_circ) }\CommentTok{\# Retângulo menos o furo circular}

\CommentTok{\#Janelas inferidas dos dados}

\NormalTok{x\_raw }\OtherTok{\textless{}{-}} \FunctionTok{runif}\NormalTok{(}\DecValTok{20}\NormalTok{, }\DecValTok{0}\NormalTok{, }\DecValTok{10}\NormalTok{); y\_raw }\OtherTok{\textless{}{-}} \FunctionTok{runif}\NormalTok{(}\DecValTok{20}\NormalTok{, }\DecValTok{0}\NormalTok{, }\DecValTok{10}\NormalTok{)}

\NormalTok{W\_ripley }\OtherTok{\textless{}{-}} \FunctionTok{ripras}\NormalTok{(x\_raw, y\_raw)     }\CommentTok{\# Estimador estatístico da janela}

\NormalTok{W\_hull }\OtherTok{\textless{}{-}} \FunctionTok{convexhull.xy}\NormalTok{(}\AttributeTok{x =}\NormalTok{ x\_raw, }\AttributeTok{y =}\NormalTok{ y\_raw) }\CommentTok{\# Fecho convexo estrito}

\CommentTok{\#Criação do objeto PPP}

\NormalTok{meu\_ppp }\OtherTok{\textless{}{-}} \FunctionTok{ppp}\NormalTok{(}\AttributeTok{x =}\NormalTok{ x\_raw, }\AttributeTok{y =}\NormalTok{ y\_raw, }\AttributeTok{window =}\NormalTok{ W\_ret)}

\CommentTok{\#Manipulação de Marcas}

\FunctionTok{marks}\NormalTok{(meu\_ppp) }\OtherTok{\textless{}{-}} \FunctionTok{sample}\NormalTok{(}\FunctionTok{c}\NormalTok{(}\StringTok{"A"}\NormalTok{, }\StringTok{"B"}\NormalTok{), }\DecValTok{20}\NormalTok{, }\AttributeTok{replace =} \ConstantTok{TRUE}\NormalTok{) }\CommentTok{\# Categórica}

\FunctionTok{marks}\NormalTok{(meu\_ppp) }\OtherTok{\textless{}{-}} \FunctionTok{runif}\NormalTok{(}\DecValTok{20}\NormalTok{) }\CommentTok{\# Numérica (sobrescreve)}

\NormalTok{sem\_marcas }\OtherTok{\textless{}{-}} \FunctionTok{unmark}\NormalTok{(meu\_ppp)}

\CommentTok{\#Transformações}

\NormalTok{ppp\_rot }\OtherTok{\textless{}{-}} \FunctionTok{rotate}\NormalTok{(meu\_ppp, }\AttributeTok{angle =}\NormalTok{ pi}\SpecialCharTok{/}\DecValTok{4}\NormalTok{)}

\NormalTok{ppp\_shift }\OtherTok{\textless{}{-}} \FunctionTok{shift}\NormalTok{(meu\_ppp, }\AttributeTok{vec =} \FunctionTok{c}\NormalTok{(}\DecValTok{2}\NormalTok{, }\DecValTok{0}\NormalTok{))}

\NormalTok{ppp\_flip }\OtherTok{\textless{}{-}} \FunctionTok{flipxy}\NormalTok{(meu\_ppp)}

\CommentTok{\#Morfologia}

\NormalTok{W\_erodida }\OtherTok{\textless{}{-}} \FunctionTok{erosion}\NormalTok{(W\_complexa, }\AttributeTok{r =} \FloatTok{0.5}\NormalTok{)}



\FunctionTok{par}\NormalTok{(}\AttributeTok{mfrow=}\FunctionTok{c}\NormalTok{(}\DecValTok{2}\NormalTok{,}\DecValTok{2}\NormalTok{), }\AttributeTok{mar=}\FunctionTok{c}\NormalTok{(}\DecValTok{1}\NormalTok{,}\DecValTok{1}\NormalTok{,}\DecValTok{1}\NormalTok{,}\DecValTok{1}\NormalTok{))}
\FunctionTok{plot}\NormalTok{(meu\_ppp, }\AttributeTok{main=}\StringTok{"PPP Original"}\NormalTok{)}
\FunctionTok{plot}\NormalTok{(W\_ripley, }\AttributeTok{main=}\StringTok{"Estimador Ripley{-}Rasson"}\NormalTok{)}
\FunctionTok{plot}\NormalTok{(ppp\_rot, }\AttributeTok{main=}\StringTok{"Rotacionado"}\NormalTok{)}
\FunctionTok{plot}\NormalTok{(W\_erodida, }\AttributeTok{main=}\StringTok{"Erosão da Janela"}\NormalTok{)}
\end{Highlighting}
\end{Shaded}

\pandocbounded{\includegraphics[keepaspectratio]{point_process_files/figure-pdf/unnamed-chunk-2-1.pdf}}

\section{Imagens, Segmentos, Tesselações e
3D}\label{imagens-segmentos-tesselauxe7uxf5es-e-3d}

\begin{itemize}
\item
  Imagens (\texttt{im}): Dados raster/grid.
\item
  \texttt{im(mat),\ as.im(X):} Criação/Conversão.
\item
  \texttt{blur(X,\ sigma):} Suavização Gaussiana de imagem.
\item
  \texttt{contour.im(X):} Extrai linhas de contorno (vetorial).
\item
  \texttt{Segmentos\ (psp):} Linhas no espaço (falhas geológicas,
  agulhas).
\item
  \texttt{psp(x0,\ y0,\ x1,\ y1,\ window):} Cria segmentos.
\item
  \texttt{lengths\_psp(X),\ angles.psp(X):} Geometria dos segmentos.
\item
  \texttt{crossing.psp(A,\ B):} Detecta onde segmentos se cruzam.
\item
  \texttt{Tesselações\ (tess):} Divisão do espaço em mosaicos.
\item
  \texttt{tess(images),\ quadrats(X)}: Cria tesselações.
\item
  \texttt{dirichlet(X):} Polígonos de Voronoi.
\item
  \texttt{delaunay(X):} Triangulação de Delaunay.
\item
  \texttt{pp3(x,\ y,\ z,\ box3()):} Padrão de pontos em 3D.
\end{itemize}

\begin{Shaded}
\begin{Highlighting}[]
\ControlFlowTok{if}\NormalTok{ (}\SpecialCharTok{!}\FunctionTok{require}\NormalTok{(}\StringTok{"pacman"}\NormalTok{)) }\FunctionTok{install.packages}\NormalTok{(}\StringTok{"pacman"}\NormalTok{)}
\NormalTok{pacman}\SpecialCharTok{::}\FunctionTok{p\_load}\NormalTok{(spatstat)}


\FunctionTok{par}\NormalTok{(}\AttributeTok{mfrow =} \FunctionTok{c}\NormalTok{(}\DecValTok{2}\NormalTok{, }\DecValTok{2}\NormalTok{), }\AttributeTok{mar =} \FunctionTok{c}\NormalTok{(}\DecValTok{1}\NormalTok{, }\DecValTok{1}\NormalTok{, }\DecValTok{1}\NormalTok{, }\DecValTok{1}\NormalTok{))}

\CommentTok{\# Imagens (Pixel)}
\NormalTok{W\_ret }\OtherTok{\textless{}{-}} \FunctionTok{owin}\NormalTok{(}\FunctionTok{c}\NormalTok{(}\DecValTok{0}\NormalTok{, }\DecValTok{1}\NormalTok{), }\FunctionTok{c}\NormalTok{(}\DecValTok{0}\NormalTok{, }\DecValTok{1}\NormalTok{))}
\NormalTok{Z }\OtherTok{\textless{}{-}} \FunctionTok{as.im}\NormalTok{(}\ControlFlowTok{function}\NormalTok{(x,y) \{ x}\SpecialCharTok{\^{}}\DecValTok{2} \SpecialCharTok{+}\NormalTok{ y}\SpecialCharTok{\^{}}\DecValTok{2}\NormalTok{ \}, W\_ret)}

\FunctionTok{plot}\NormalTok{(Z, }\AttributeTok{main =} \StringTok{"Imagem (Pixel)"}\NormalTok{)}

\CommentTok{\# Segmentos de Linha (psp) e Cruzamentos}
\FunctionTok{set.seed}\NormalTok{(}\DecValTok{42}\NormalTok{)}
\CommentTok{\# Criar segmentos aleatórios}
\NormalTok{seg }\OtherTok{\textless{}{-}} \FunctionTok{psp}\NormalTok{(}\AttributeTok{x0=}\FunctionTok{runif}\NormalTok{(}\DecValTok{10}\NormalTok{), }\AttributeTok{y0=}\FunctionTok{runif}\NormalTok{(}\DecValTok{10}\NormalTok{), }\AttributeTok{x1=}\FunctionTok{runif}\NormalTok{(}\DecValTok{10}\NormalTok{), }\AttributeTok{y1=}\FunctionTok{runif}\NormalTok{(}\DecValTok{10}\NormalTok{), }\AttributeTok{window=}\FunctionTok{owin}\NormalTok{())}

\CommentTok{\#auto{-}intersecção}
\NormalTok{cruzamentos }\OtherTok{\textless{}{-}} \FunctionTok{selfcrossing.psp}\NormalTok{(seg)}

\FunctionTok{plot}\NormalTok{(seg, }\AttributeTok{main =} \StringTok{"Segmentos e Cruzamentos"}\NormalTok{, }\AttributeTok{col =} \StringTok{"blue"}\NormalTok{)}
\FunctionTok{plot}\NormalTok{(cruzamentos, }\AttributeTok{add =} \ConstantTok{TRUE}\NormalTok{, }\AttributeTok{col =} \StringTok{"red"}\NormalTok{, }\AttributeTok{pch =} \DecValTok{19}\NormalTok{, }\AttributeTok{cex =} \FloatTok{1.5}\NormalTok{) }

\CommentTok{\#Tesselação de Voronoi}

\FunctionTok{data}\NormalTok{(cells)}

\CommentTok{\# Calcular Voronoi (Dirichlet)}
\NormalTok{voro }\OtherTok{\textless{}{-}} \FunctionTok{dirichlet}\NormalTok{(cells)}

\CommentTok{\# Plotagem}
\FunctionTok{plot}\NormalTok{(voro, }\AttributeTok{main =} \StringTok{"Voronoi"}\NormalTok{, }\AttributeTok{border =} \StringTok{"blue"}\NormalTok{, }\AttributeTok{col =} \FunctionTok{gray.colors}\NormalTok{(}\FunctionTok{ntiles}\NormalTok{(voro), }\AttributeTok{alpha=}\FloatTok{0.3}\NormalTok{))}
\FunctionTok{plot}\NormalTok{(cells, }\AttributeTok{add =} \ConstantTok{TRUE}\NormalTok{, }\AttributeTok{pch =} \DecValTok{16}\NormalTok{, }\AttributeTok{col =} \StringTok{"red"}\NormalTok{, }\AttributeTok{cex =} \FloatTok{0.8}\NormalTok{)}

\CommentTok{\# riangulação de Delaunay}
\NormalTok{dela }\OtherTok{\textless{}{-}} \FunctionTok{delaunay}\NormalTok{(cells)}

\FunctionTok{plot}\NormalTok{(dela, }\AttributeTok{main =} \StringTok{"Delaunay"}\NormalTok{, }\AttributeTok{border =} \StringTok{"darkgreen"}\NormalTok{)}
\FunctionTok{plot}\NormalTok{(cells, }\AttributeTok{add =} \ConstantTok{TRUE}\NormalTok{, }\AttributeTok{pch =} \DecValTok{16}\NormalTok{, }\AttributeTok{col =} \StringTok{"red"}\NormalTok{, }\AttributeTok{cex =} \FloatTok{0.8}\NormalTok{)}
\end{Highlighting}
\end{Shaded}

\pandocbounded{\includegraphics[keepaspectratio]{point_process_files/figure-pdf/unnamed-chunk-3-1.pdf}}

\begin{Shaded}
\begin{Highlighting}[]
\FunctionTok{par}\NormalTok{(}\AttributeTok{mfrow =} \FunctionTok{c}\NormalTok{(}\DecValTok{1}\NormalTok{, }\DecValTok{1}\NormalTok{))}
\end{Highlighting}
\end{Shaded}

\section{\texorpdfstring{Análise Exploratória de Dados
(\texttt{spatstat.explore})}{Análise Exploratória de Dados (spatstat.explore)}}\label{anuxe1lise-exploratuxf3ria-de-dados-spatstat.explore}

Focada em estatísticas descritivas, intensidade e testes de hipótese.

\textbf{Intensidade (\(\lambda\)) e Covariáveis}

\begin{itemize}
\item
  \texttt{summary(X):} Visão geral.
\item
  \texttt{clarkevans(X):} Índice R (agrupamento vs regularidade).
\end{itemize}

\textbf{Estimativa de Densidade (Kernel)}

\begin{itemize}
\item
  \texttt{density(X):} Kernel fixo.
\item
  \texttt{densityAdaptiveKernel(X):} Largura de banda variável (melhor
  para intensidade heterogênea).
\item
  \texttt{densityVoronoi(X):} Estimador constante por polígono de
  Voronoi.
\end{itemize}

\textbf{Seleção de Largura de Banda (Bandwidth)}

\begin{itemize}
\item
  \texttt{bw.diggle(X):} Minimização do erro quadrático médio (para
  Cox).
\item
  \texttt{bw.ppl(X):} Verossimilhança (para Poisson).
\item
  \texttt{bw.scott(X):} Regra de Scott (normal).
\end{itemize}

\textbf{Dependência de Covariáveis}

\begin{itemize}
\item
  \texttt{rhohat(X,\ Z):} Estima \(\rho(z)\) não parametricamente.
\item
  \texttt{rho2hat(X,\ Z1,\ Z2):} Intensidade conjunta de duas
  covariáveis.
\item
  \texttt{roc(X,\ Z),\ auc(X,\ Z):} Curva ROC e Área sob a curva para
  capacidade preditiva da covariável.
\end{itemize}

\begin{Shaded}
\begin{Highlighting}[]
\FunctionTok{par}\NormalTok{(}\AttributeTok{mfrow=}\FunctionTok{c}\NormalTok{(}\DecValTok{2}\NormalTok{,}\DecValTok{2}\NormalTok{), }\AttributeTok{mar=}\FunctionTok{c}\NormalTok{(}\DecValTok{1}\NormalTok{,}\DecValTok{1}\NormalTok{,}\DecValTok{1}\NormalTok{,}\DecValTok{1}\NormalTok{))}
\FunctionTok{data}\NormalTok{(bei); }\FunctionTok{data}\NormalTok{(bei.extra) }\CommentTok{\# Árvores e gradiente/elevação}

\CommentTok{\# Seleção de Banda e Densidade}
\NormalTok{sig }\OtherTok{\textless{}{-}} \FunctionTok{bw.ppl}\NormalTok{(bei)}

\NormalTok{dens\_fixa }\OtherTok{\textless{}{-}} \FunctionTok{density}\NormalTok{(bei, }\AttributeTok{sigma =}\NormalTok{ sig)}

\NormalTok{dens\_adap }\OtherTok{\textless{}{-}} \FunctionTok{densityAdaptiveKernel}\NormalTok{(bei) }\CommentTok{\# Mais robusto}

\CommentTok{\# Dependência de Covariável}
\NormalTok{rho\_elev }\OtherTok{\textless{}{-}} \FunctionTok{rhohat}\NormalTok{(bei, bei.extra}\SpecialCharTok{$}\NormalTok{elev) }\CommentTok{\# Intensidade vs Elevação}

\FunctionTok{plot}\NormalTok{(rho\_elev, }\AttributeTok{main=}\StringTok{"Rhohat: Efeito da Elevação"}\NormalTok{)}

\CommentTok{\# Capacidade Preditiva (ROC/AUC)}
\NormalTok{roc\_curve }\OtherTok{\textless{}{-}} \FunctionTok{roc}\NormalTok{(bei, bei.extra}\SpecialCharTok{$}\NormalTok{grad)}

\NormalTok{area\_roc  }\OtherTok{\textless{}{-}} \FunctionTok{auc}\NormalTok{(bei, bei.extra}\SpecialCharTok{$}\NormalTok{grad)}

\FunctionTok{plot}\NormalTok{(roc\_curve, }\AttributeTok{main=}\FunctionTok{paste}\NormalTok{(}\StringTok{"ROC Gradiente (AUC ="}\NormalTok{, }\FunctionTok{round}\NormalTok{(area\_roc,}\DecValTok{2}\NormalTok{), }\StringTok{")"}\NormalTok{))}
\end{Highlighting}
\end{Shaded}

\pandocbounded{\includegraphics[keepaspectratio]{point_process_files/figure-pdf/unnamed-chunk-4-1.pdf}}

\textbf{Funções de Distância e Interação}

\begin{itemize}
\item
  \texttt{Fest(X):} Espaço vazio (``Empty Space'').
\item
  \texttt{Gest(X):} Vizinho mais próximo (``Nearest Neighbor'').
\item
  \texttt{Kest(X):} Ripley's K (cumulativa).
\item
  \texttt{Lest(X):} Besag's L (transformação de K para linearizar
  Poisson).
\item
  \texttt{pcf(X):} Correlação de Pares (\(g(r)\)).
\item
  \texttt{Jest(X):} Função J (combinação de F e G).
\end{itemize}

\textbf{Versões Inhomogêneas (para \(\lambda(u)\) variável):}

\begin{itemize}
\tightlist
\item
  \texttt{Kinhom,\ Linhom,\ pcfinhom,\ Finhom,\ Ginhom.}
\end{itemize}

\textbf{Envelopes de simulação}

\texttt{envelope(X,\ fun,\ nsim):} Gera bandas de confiança via Monte
Carlo para rejeitar CSR.

\begin{Shaded}
\begin{Highlighting}[]
\CommentTok{\#Funções de Distância (F, G, J)}

\FunctionTok{par}\NormalTok{(}\AttributeTok{mfrow =} \FunctionTok{c}\NormalTok{(}\DecValTok{2}\NormalTok{, }\DecValTok{4}\NormalTok{), }\AttributeTok{mar =} \FunctionTok{c}\NormalTok{(}\DecValTok{1}\NormalTok{, }\DecValTok{1}\NormalTok{, }\DecValTok{3}\NormalTok{, }\DecValTok{1}\NormalTok{))}

\CommentTok{\# Função F (Espaço Vazio)}
\FunctionTok{plot}\NormalTok{(}\FunctionTok{Fest}\NormalTok{(bei), }\AttributeTok{main =} \StringTok{"Função F (Vazio)"}\NormalTok{, }\AttributeTok{legend =} \ConstantTok{FALSE}\NormalTok{)}

\CommentTok{\# Função G (Vizinho Mais Próximo)}
\FunctionTok{plot}\NormalTok{(}\FunctionTok{Gest}\NormalTok{(bei), }\AttributeTok{main =} \StringTok{"Função G (Vizinho)"}\NormalTok{, }\AttributeTok{legend =} \ConstantTok{FALSE}\NormalTok{)}

\CommentTok{\# Função J (Interação)}
\FunctionTok{plot}\NormalTok{(}\FunctionTok{Jest}\NormalTok{(bei), }\AttributeTok{main =} \StringTok{"Função J"}\NormalTok{, }\AttributeTok{legend =} \ConstantTok{FALSE}\NormalTok{)}


\CommentTok{\# Funções de Segunda Ordem (K, L, pcf)}
\CommentTok{\# Função K de Ripley}
\FunctionTok{plot}\NormalTok{(}\FunctionTok{Kest}\NormalTok{(bei), }\AttributeTok{main =} \StringTok{"Função K"}\NormalTok{, }\AttributeTok{legend =} \ConstantTok{FALSE}\NormalTok{)}

\CommentTok{\# Função L (Linearizada), Plotando "L(r) {-} r" para ficar horizontal}
\FunctionTok{plot}\NormalTok{(}\FunctionTok{Lest}\NormalTok{(bei), . }\SpecialCharTok{{-}}\NormalTok{ r }\SpecialCharTok{\textasciitilde{}}\NormalTok{ r, }\AttributeTok{main =} \StringTok{"Função L"}\NormalTok{, }\AttributeTok{legend =} \ConstantTok{FALSE}\NormalTok{)}

\CommentTok{\# Função g (Correlação de Pares) {-} pcf}
\CommentTok{\# divisor="d" corrige viés em distâncias curtas}
\FunctionTok{plot}\NormalTok{(}\FunctionTok{pcf}\NormalTok{(bei, }\AttributeTok{divisor =} \StringTok{"d"}\NormalTok{), }\AttributeTok{main =} \StringTok{"Função g (PCF)"}\NormalTok{, }\AttributeTok{legend =} \ConstantTok{FALSE}\NormalTok{)}


\CommentTok{\# Análise Inhomogênea e Envelopes}

\CommentTok{\# Estimar a intensidade (tendência)}
\NormalTok{lambda\_bei }\OtherTok{\textless{}{-}} \FunctionTok{density}\NormalTok{(bei, }\AttributeTok{sigma =} \FunctionTok{bw.scott}\NormalTok{(bei))}

\CommentTok{\#Calcular L Inhomogêneo}
\NormalTok{L\_inhom }\OtherTok{\textless{}{-}} \FunctionTok{Linhom}\NormalTok{(bei, }\AttributeTok{lambda =}\NormalTok{ lambda\_bei)}

\FunctionTok{plot}\NormalTok{(L\_inhom, . }\SpecialCharTok{{-}}\NormalTok{ r }\SpecialCharTok{\textasciitilde{}}\NormalTok{ r, }
     \AttributeTok{main =} \StringTok{"L Inhomogêneo"}\NormalTok{, }
     \AttributeTok{legend =} \ConstantTok{FALSE}\NormalTok{)}

\CommentTok{\#Envelopes de Simulação (Teste de Monte Carlo)}
\NormalTok{env\_L }\OtherTok{\textless{}{-}} \FunctionTok{envelope}\NormalTok{(bei, Linhom, }
                  \AttributeTok{sigma =} \FunctionTok{bw.scott}\NormalTok{(bei), }
                  \AttributeTok{nsim =} \DecValTok{19}\NormalTok{, }\AttributeTok{rank =} \DecValTok{1}\NormalTok{, }\AttributeTok{global =} \ConstantTok{TRUE}\NormalTok{, }
                  \AttributeTok{simulate =} \FunctionTok{expression}\NormalTok{(}\FunctionTok{rpoispp}\NormalTok{(lambda\_bei)))}
\end{Highlighting}
\end{Shaded}

\begin{verbatim}
Generating 38 simulations by evaluating expression (19 to estimate the mean and 
19 to calculate envelopes) ...
1, 2, 3, 4, 5, 6, 7, 8, 9, 10, 11, 12, 13, 14, 15, 16, 17, 18, 19, 20,
21, 22, 23, 24, 25, 26, 27, 28, 29, 30, 31, 32, 33, 34, 35, 36, 37, 
38.

Done.
\end{verbatim}

\begin{Shaded}
\begin{Highlighting}[]
\FunctionTok{plot}\NormalTok{(env\_L, . }\SpecialCharTok{{-}}\NormalTok{ r }\SpecialCharTok{\textasciitilde{}}\NormalTok{ r, }\AttributeTok{main =} \StringTok{"Envelopes"}\NormalTok{, }\AttributeTok{legend =} \ConstantTok{FALSE}\NormalTok{)}
\end{Highlighting}
\end{Shaded}

\pandocbounded{\includegraphics[keepaspectratio]{point_process_files/figure-pdf/unnamed-chunk-5-1.pdf}}

\begin{Shaded}
\begin{Highlighting}[]
\FunctionTok{par}\NormalTok{(}\AttributeTok{mfrow =} \FunctionTok{c}\NormalTok{(}\DecValTok{1}\NormalTok{, }\DecValTok{1}\NormalTok{))}
\end{Highlighting}
\end{Shaded}

\section{Análise Multitype e Gráficos de
Diagnóstico}\label{anuxe1lise-multitype-e-gruxe1ficos-de-diagnuxf3stico}

Para dados com marcas ou diagnósticos visuais de estrutura.

\begin{itemize}
\item
  \texttt{Kcross(X,\ i,\ j),\ Lcross:} Interação Tipo \(i\) vs Tipo
  \(j\).
\item
  \texttt{Kdot(X,\ i),\ Ldot:} Tipo \(i\) vs Qualquer outro.
\item
  \texttt{markcorr(X):} Correlação para marcas contínuas.
\item
  \texttt{markvario(X):} Variograma de marcas.
\item
  \texttt{relrisk(X):} Probabilidade espacial relativa de tipos.
\end{itemize}

\textbf{Diagnóstico Visual}

\begin{itemize}
\item
  \texttt{fryplot(X):} Vetores entre todos os pares (detecta
  anisotropia/reticulado).
\item
  \texttt{miplot(X):} Índice de Morisita.
\end{itemize}

\textbf{Estatísticas Locais (LISA)}

\begin{itemize}
\item
  \texttt{localK(X),\ localL(X):} Contribuição individual de cada ponto.
\item
  \texttt{clusterset(X):} Método Allard-Fraley para isolar clusters
  densos.
\item
  \texttt{nnclean(X):} Remove ruído (Byers-Raftery).
\item
  \texttt{sharpen(X)}: Intensifica pontos para destacar estruturas.
\end{itemize}

\begin{Shaded}
\begin{Highlighting}[]
\FunctionTok{par}\NormalTok{(}\AttributeTok{mfrow=}\FunctionTok{c}\NormalTok{(}\DecValTok{1}\NormalTok{,}\DecValTok{1}\NormalTok{), }\AttributeTok{mar=}\FunctionTok{c}\NormalTok{(}\DecValTok{2}\NormalTok{,}\DecValTok{3}\NormalTok{,}\DecValTok{3}\NormalTok{,}\DecValTok{4}\NormalTok{))}

\FunctionTok{data}\NormalTok{(amacrine) }\CommentTok{\# Células On/Off\# Risco Relativo (Probabilidade de ser "On" no espaço)}

\NormalTok{rr }\OtherTok{\textless{}{-}} \FunctionTok{relrisk}\NormalTok{(amacrine)}
\FunctionTok{plot}\NormalTok{(rr, }\AttributeTok{main=}\StringTok{"Risco Relativo"}\NormalTok{)}\CommentTok{\# Correlação Cruzada}
\end{Highlighting}
\end{Shaded}

\pandocbounded{\includegraphics[keepaspectratio]{point_process_files/figure-pdf/unnamed-chunk-6-1.pdf}}

\begin{Shaded}
\begin{Highlighting}[]
\NormalTok{L\_cr }\OtherTok{\textless{}{-}} \FunctionTok{Lcross}\NormalTok{(amacrine, }\StringTok{"on"}\NormalTok{, }\StringTok{"off"}\NormalTok{)}
\FunctionTok{plot}\NormalTok{(L\_cr, }\AttributeTok{main=}\StringTok{"L Cross"}\NormalTok{)}\CommentTok{\# Diagnósticos Visuais}
\end{Highlighting}
\end{Shaded}

\pandocbounded{\includegraphics[keepaspectratio]{point_process_files/figure-pdf/unnamed-chunk-6-2.pdf}}

\begin{Shaded}
\begin{Highlighting}[]
\FunctionTok{fryplot}\NormalTok{(amacrine, }\AttributeTok{main=}\StringTok{"Fry Plot (Anisotropia)"}\NormalTok{, }\AttributeTok{cex=}\FloatTok{0.1}\NormalTok{)}
\end{Highlighting}
\end{Shaded}

\pandocbounded{\includegraphics[keepaspectratio]{point_process_files/figure-pdf/unnamed-chunk-6-3.pdf}}

\begin{Shaded}
\begin{Highlighting}[]
\NormalTok{limpos }\OtherTok{\textless{}{-}} \FunctionTok{nnclean}\NormalTok{(amacrine, }\AttributeTok{k=}\DecValTok{5}\NormalTok{) }\CommentTok{\# Detectar outliers}
\end{Highlighting}
\end{Shaded}

\begin{verbatim}
Iteration 1     logLik = 8240.77670246102   p = 0.5031 
Iteration 2     logLik = 8235.95198819664   p = 0.5027 
Estimated parameters:
p [cluster] = 0.50269 
lambda [cluster] = 148.74 
lambda [noise]   = 136.08 
\end{verbatim}

\begin{Shaded}
\begin{Highlighting}[]
\FunctionTok{par}\NormalTok{(}\AttributeTok{mfrow=}\FunctionTok{c}\NormalTok{(}\DecValTok{1}\NormalTok{,}\DecValTok{1}\NormalTok{), }\AttributeTok{mar=}\FunctionTok{c}\NormalTok{(}\DecValTok{1}\NormalTok{,}\DecValTok{1}\NormalTok{,}\DecValTok{1}\NormalTok{,}\DecValTok{1}\NormalTok{))}
\FunctionTok{plot}\NormalTok{(limpos, }\AttributeTok{main=}\StringTok{"Classificação Ruído vs Feature"}\NormalTok{, }\AttributeTok{size=}\FloatTok{0.2}\NormalTok{)}
\end{Highlighting}
\end{Shaded}

\pandocbounded{\includegraphics[keepaspectratio]{point_process_files/figure-pdf/unnamed-chunk-6-4.pdf}}

\section{\texorpdfstring{Ajuste de Modelos
(\texttt{spatstat.model})}{Ajuste de Modelos (spatstat.model)}}\label{ajuste-de-modelos-spatstat.model}

Ajuste de modelos paramétricos (Poisson, Gibbs, Cox, Cluster).

\textbf{Processos de Poisson e Gibbs (\texttt{ppm})}

Ajuste via Máxima Pseudoverossimilhança e a função principal é:

\texttt{ppm(formula,\ interaction,\ ...)}

\begin{enumerate}
\def\labelenumi{\arabic{enumi}.}
\tightlist
\item
  Especificação de Tendência (Fórmula):
\end{enumerate}

\begin{itemize}
\item
  \texttt{\textasciitilde{}1:} Homogêneo.
\item
  \texttt{\textasciitilde{}x\ +\ y:} Log-linear.
\item
  \texttt{\textasciitilde{}polynom(x,\ y,\ 2):} Quadrática.
\item
  \texttt{\textasciitilde{}Z:} Covariável imagem \(Z\).
\end{itemize}

\textbf{Interações (Família Gibbs):}

\begin{itemize}
\item
  \texttt{Poisson():} Sem interação.
\item
  \texttt{Strauss(r):} Repulsão hard/soft até raio \(r\).
\item
  \texttt{Hardcore(h):} Exclusão absoluta até \(h\).
\item
  \texttt{Geyer(r,\ sat):} Processo de saturação (cluster ou repulsão).
\item
  \texttt{AreaInter(r):} Interação de área (mais suave que Strauss).
\item
  \texttt{Pairwise(),\ SatPiece(),\ Saturated():} Interações
  personalizadas ou por partes.
\end{itemize}

\section{Seleção e Validação}\label{seleuxe7uxe3o-e-validauxe7uxe3o}

\begin{itemize}
\item
  \texttt{AIC(modelo\_ajustado):} Critério de Akaike.
\item
  \texttt{anova.ppm(fit1,\ fit2):} Teste de Razão de Verossimilhança
  (para aninhados).
\end{itemize}

\textbf{Processos de Cluster e Cox (\texttt{kppm})}

Modelos onde pontos se agrupam em torno de ``pais'' não observados.
Método de Contraste Mínimo ou Palm Likelihood.

\begin{itemize}
\item
  Função Principal: kppm(formula, clusters)
\item
  Tipos de Cluster

  \begin{enumerate}
  \def\labelenumi{\arabic{enumi}.}
  \item
    ``Thomas'': Dispersão Gaussiana (Thomas Process).
  \item
    ``MatClust'': Dispersão Uniforme em disco (Matérn Cluster).
  \item
    ``LGCP'': Log-Gaussian Cox Process.
  \item
    ``Cauchy'', ``VarGamma'': Clusters com caudas longas.
  \end{enumerate}
\end{itemize}

\textbf{Métodos Auxiliares}

\begin{itemize}
\item
  \texttt{mincontrast():} Função de baixo nível para ajuste via
  \texttt{K\ ou\ pcf}.
\item
  \texttt{thomas.estK(),\ matclust.estK():} Ajustes diretos específicos.
\end{itemize}

\textbf{Outros Modelos (SLRM, DPPM)}

\begin{itemize}
\item
  SLRM (slrm): Spatial Logistic Regression. Aproximação discreta
  (pixelada). Rápido para grandes dados.
\item
  \texttt{slrm(formula):} Ajuste.
\item
  \texttt{DPPM\ (dppm):} Determinantal Point Process. Para modelar
  regularidade forte (repulsão) de forma tratável.
\item
  \texttt{dppm(formula,\ family):} Ex: família Gaussian ou Bessel.
\end{itemize}

\begin{Shaded}
\begin{Highlighting}[]
\ControlFlowTok{if}\NormalTok{(}\SpecialCharTok{!}\FunctionTok{require}\NormalTok{(pacman)) }\FunctionTok{install.packages}\NormalTok{(}\StringTok{"pacman"}\NormalTok{)}
\NormalTok{pacman}\SpecialCharTok{::}\FunctionTok{p\_load}\NormalTok{(spatstat)}

\CommentTok{\# Dados}
\FunctionTok{data}\NormalTok{(cells)   }\CommentTok{\# Para modelos de repulsão (Gibbs/DPPM)}
\FunctionTok{data}\NormalTok{(redwood) }\CommentTok{\# Para modelos de cluster (Cox/kppm)}

\NormalTok{Z\_im }\OtherTok{\textless{}{-}} \FunctionTok{as.im}\NormalTok{(}\ControlFlowTok{function}\NormalTok{(x,y) x }\SpecialCharTok{+}\NormalTok{ y, }\FunctionTok{Window}\NormalTok{(cells)) }\CommentTok{\# Covariável sintética}

\CommentTok{\#AJUSTE DE MODELOS PPM (Poisson e Gibbs)}

\CommentTok{\#Especificação de Tendência}
\NormalTok{fit\_homo }\OtherTok{\textless{}{-}} \FunctionTok{ppm}\NormalTok{(cells }\SpecialCharTok{\textasciitilde{}} \DecValTok{1}\NormalTok{)                    }\CommentTok{\# Homogêneo}
\NormalTok{fit\_lin  }\OtherTok{\textless{}{-}} \FunctionTok{ppm}\NormalTok{(cells }\SpecialCharTok{\textasciitilde{}}\NormalTok{ x }\SpecialCharTok{+}\NormalTok{ y)                }\CommentTok{\# Log{-}linear}
\NormalTok{fit\_poly }\OtherTok{\textless{}{-}} \FunctionTok{ppm}\NormalTok{(cells }\SpecialCharTok{\textasciitilde{}} \FunctionTok{polynom}\NormalTok{(x, y, }\DecValTok{2}\NormalTok{))     }\CommentTok{\# Quadrática}
\NormalTok{fit\_cov  }\OtherTok{\textless{}{-}} \FunctionTok{ppm}\NormalTok{(cells }\SpecialCharTok{\textasciitilde{}}\NormalTok{ Z, }\AttributeTok{data=}\FunctionTok{list}\NormalTok{(}\AttributeTok{Z=}\NormalTok{Z\_im)) }\CommentTok{\# Covariável Imagem}

\CommentTok{\# Interações (Família Gibbs)}
\CommentTok{\# Nota: O raio \textquotesingle{}r\textquotesingle{} deve ser escolhido com base na análise exploratória (ex: pcf)}
\NormalTok{r\_int }\OtherTok{\textless{}{-}} \FloatTok{0.08} 

\NormalTok{fit\_pois    }\OtherTok{\textless{}{-}} \FunctionTok{ppm}\NormalTok{(cells }\SpecialCharTok{\textasciitilde{}} \DecValTok{1}\NormalTok{, }\FunctionTok{Poisson}\NormalTok{())}
\NormalTok{fit\_strauss }\OtherTok{\textless{}{-}} \FunctionTok{ppm}\NormalTok{(cells }\SpecialCharTok{\textasciitilde{}} \DecValTok{1}\NormalTok{, }\FunctionTok{Strauss}\NormalTok{(}\AttributeTok{r=}\NormalTok{r\_int))}
\NormalTok{fit\_hard    }\OtherTok{\textless{}{-}} \FunctionTok{ppm}\NormalTok{(cells }\SpecialCharTok{\textasciitilde{}} \DecValTok{1}\NormalTok{, }\FunctionTok{Hardcore}\NormalTok{(}\AttributeTok{h=}\FloatTok{0.05}\NormalTok{))}
\NormalTok{fit\_geyer   }\OtherTok{\textless{}{-}} \FunctionTok{ppm}\NormalTok{(cells }\SpecialCharTok{\textasciitilde{}} \DecValTok{1}\NormalTok{, }\FunctionTok{Geyer}\NormalTok{(}\AttributeTok{r=}\NormalTok{r\_int, }\AttributeTok{sat=}\DecValTok{2}\NormalTok{))}
\NormalTok{fit\_area    }\OtherTok{\textless{}{-}} \FunctionTok{ppm}\NormalTok{(cells }\SpecialCharTok{\textasciitilde{}} \DecValTok{1}\NormalTok{, }\FunctionTok{AreaInter}\NormalTok{(}\AttributeTok{r=}\NormalTok{r\_int))}
\CommentTok{\# fit\_pair  \textless{}{-} ppm(cells \textasciitilde{} 1, Pairwise(pot=function(d) \{ ... \})) \# Personalizado}

\CommentTok{\#SELEÇÃO E VALIDAÇÃO (PPM)}

\CommentTok{\# Critério de Akaike (menor é melhor)}
\FunctionTok{AIC}\NormalTok{(fit\_homo)}
\end{Highlighting}
\end{Shaded}

\begin{verbatim}
[1] -227.9642
\end{verbatim}

\begin{Shaded}
\begin{Highlighting}[]
\FunctionTok{AIC}\NormalTok{(fit\_strauss)}
\end{Highlighting}
\end{Shaded}

\begin{verbatim}
[1] -295.2359
\end{verbatim}

\begin{Shaded}
\begin{Highlighting}[]
\CommentTok{\# Teste de Razão de Verossimilhança (Modelos aninhados)}
\FunctionTok{anova}\NormalTok{(fit\_homo, fit\_lin, }\AttributeTok{test=}\StringTok{"LRT"}\NormalTok{)}
\end{Highlighting}
\end{Shaded}

\begin{verbatim}
Analysis of Deviance Table

Model 1: ~1      Poisson
Model 2: ~x + y      Poisson
  Npar Df Deviance Pr(>Chi)
1    1                     
2    3  2  0.30444   0.8588
\end{verbatim}

\begin{Shaded}
\begin{Highlighting}[]
\CommentTok{\# Diagnóstico de Resíduos (Gráfico)}
\FunctionTok{par}\NormalTok{(}\AttributeTok{mfrow=}\FunctionTok{c}\NormalTok{(}\DecValTok{1}\NormalTok{,}\DecValTok{2}\NormalTok{))}
\FunctionTok{diagnose.ppm}\NormalTok{(fit\_homo, }\AttributeTok{which=}\StringTok{"smooth"}\NormalTok{, }\AttributeTok{main=}\StringTok{"Resíduos Poisson"}\NormalTok{)}
\end{Highlighting}
\end{Shaded}

\begin{verbatim}
Model diagnostics (raw residuals)
Diagnostics available:
    smoothed residual field
range of smoothed field =  [-28.15, 15.33]
\end{verbatim}

\begin{Shaded}
\begin{Highlighting}[]
\FunctionTok{diagnose.ppm}\NormalTok{(fit\_strauss, }\AttributeTok{which=}\StringTok{"smooth"}\NormalTok{, }\AttributeTok{main=}\StringTok{"Resíduos Strauss"}\NormalTok{)}
\end{Highlighting}
\end{Shaded}

\pandocbounded{\includegraphics[keepaspectratio]{point_process_files/figure-pdf/unnamed-chunk-7-1.pdf}}

\begin{verbatim}
Model diagnostics (raw residuals)
Diagnostics available:
    smoothed residual field
range of smoothed field =  [-22.6, 36]
\end{verbatim}

\begin{Shaded}
\begin{Highlighting}[]
\CommentTok{\#PROCESSOS DE CLUSTER E COX (KPPM)}

\CommentTok{\# Ajuste via Min Contrast ou Palm Likelihood}
\NormalTok{fit\_thomas   }\OtherTok{\textless{}{-}} \FunctionTok{kppm}\NormalTok{(redwood }\SpecialCharTok{\textasciitilde{}} \DecValTok{1}\NormalTok{, }\AttributeTok{clusters=}\StringTok{"Thomas"}\NormalTok{)}
\NormalTok{fit\_matclust }\OtherTok{\textless{}{-}} \FunctionTok{kppm}\NormalTok{(redwood }\SpecialCharTok{\textasciitilde{}} \DecValTok{1}\NormalTok{, }\AttributeTok{clusters=}\StringTok{"MatClust"}\NormalTok{)}
\NormalTok{fit\_lgcp     }\OtherTok{\textless{}{-}} \FunctionTok{kppm}\NormalTok{(redwood }\SpecialCharTok{\textasciitilde{}} \DecValTok{1}\NormalTok{, }\AttributeTok{clusters=}\StringTok{"LGCP"}\NormalTok{)}
\NormalTok{fit\_cauchy   }\OtherTok{\textless{}{-}} \FunctionTok{kppm}\NormalTok{(redwood }\SpecialCharTok{\textasciitilde{}} \DecValTok{1}\NormalTok{, }\AttributeTok{clusters=}\StringTok{"Cauchy"}\NormalTok{)}
\NormalTok{fit\_vargam   }\OtherTok{\textless{}{-}} \FunctionTok{kppm}\NormalTok{(redwood }\SpecialCharTok{\textasciitilde{}} \DecValTok{1}\NormalTok{, }\AttributeTok{clusters=}\StringTok{"VarGamma"}\NormalTok{)}

\CommentTok{\# Métodos Auxiliares (Baixo nível / Ajuste direto no K)}
\NormalTok{K\_obs }\OtherTok{\textless{}{-}} \FunctionTok{Kest}\NormalTok{(redwood)}
\NormalTok{fit\_mincon\_t }\OtherTok{\textless{}{-}} \FunctionTok{thomas.estK}\NormalTok{(K\_obs)}
\NormalTok{fit\_mincon\_m }\OtherTok{\textless{}{-}} \FunctionTok{matclust.estK}\NormalTok{(K\_obs)}

\CommentTok{\# Diagnóstico KPPM (Envelopes de Simulação)}
\FunctionTok{par}\NormalTok{(}\AttributeTok{mfrow=}\FunctionTok{c}\NormalTok{(}\DecValTok{1}\NormalTok{,}\DecValTok{1}\NormalTok{))}
\FunctionTok{plot}\NormalTok{(}\FunctionTok{envelope}\NormalTok{(fit\_thomas, Lest, }\AttributeTok{nsim=}\DecValTok{19}\NormalTok{, }\AttributeTok{global=}\ConstantTok{TRUE}\NormalTok{), }
     \AttributeTok{main=}\StringTok{"Diagnóstico Thomas (Envelopes)"}\NormalTok{)}
\end{Highlighting}
\end{Shaded}

\begin{verbatim}
Generating 38 simulated realisations of fitted cluster model (19 to estimate 
the mean and 19 to calculate envelopes) ...
1, 2, 3, 4, 5, 6, 7, 8, 9, 10, 11, 12, 13, 14, 15, 16, 17, 18, 19, 20,
21, 22, 23, 24, 25, 26, 27, 28, 29, 30, 31, 32, 33, 34, 35, 36, 37, 
38.

Done.
\end{verbatim}

\pandocbounded{\includegraphics[keepaspectratio]{point_process_files/figure-pdf/unnamed-chunk-7-2.pdf}}

\begin{Shaded}
\begin{Highlighting}[]
\CommentTok{\# OUTROS MODELOS (SLRM e DPPM)}

\CommentTok{\# SLRM {-} Spatial Logistic Regression (Aproximação pixelada)}
\NormalTok{fit\_slrm }\OtherTok{\textless{}{-}} \FunctionTok{slrm}\NormalTok{(redwood }\SpecialCharTok{\textasciitilde{}}\NormalTok{ x }\SpecialCharTok{+}\NormalTok{ y)}
\FunctionTok{plot}\NormalTok{(fit\_slrm, }\AttributeTok{main=}\StringTok{"Predição SLRM"}\NormalTok{)}
\end{Highlighting}
\end{Shaded}

\pandocbounded{\includegraphics[keepaspectratio]{point_process_files/figure-pdf/unnamed-chunk-7-3.pdf}}

\begin{Shaded}
\begin{Highlighting}[]
\CommentTok{\# DPPM {-} Determinantal Point Process (Repulsão forte)}
\NormalTok{fit\_dppm }\OtherTok{\textless{}{-}} \FunctionTok{dppm}\NormalTok{(cells }\SpecialCharTok{\textasciitilde{}} \DecValTok{1}\NormalTok{, }\FunctionTok{dppGauss}\NormalTok{())}

\CommentTok{\# Visualização do modelo DPPM ajustado}
\FunctionTok{plot}\NormalTok{(fit\_dppm, }\AttributeTok{main=}\StringTok{"DPPM Gaussiano"}\NormalTok{)}
\end{Highlighting}
\end{Shaded}

\pandocbounded{\includegraphics[keepaspectratio]{point_process_files/figure-pdf/unnamed-chunk-7-4.pdf}}

\section{\texorpdfstring{Simulação
(\texttt{spatstat.random})}{Simulação (spatstat.random)}}\label{simulauxe7uxe3o-spatstat.random}

Gerar realizações estocásticas dos processos.

\begin{itemize}
\tightlist
\item
  Padrões Aleatórios Básicos:
\end{itemize}

\begin{enumerate}
\def\labelenumi{\arabic{enumi}.}
\item
  \texttt{runifpoint(n):\ Uniforme\ (CSR)} fixo \(n\).
\item
  \texttt{rpoispp(lambda):} Poisson (CSR) com intensidade \(\lambda\)
  (número aleatório de pontos).
\item
  \texttt{rpoint(n,\ f):} \(n\) pontos com densidade de probabilidade
  \(f\).
\end{enumerate}

\textbf{Processos de Gibbs (MCMC)}

\begin{itemize}
\item
  \texttt{rmh(model):} Metropolis-Hastings. Simula de um modelo
  \texttt{ppm} ou especificação manual.
\item
  \texttt{rStrauss(beta,\ gamma,\ r),\ rHardcore():} Wrappers diretos
  para simulação.
\item
  \texttt{rThomas(),\ rMatClust(),\ rVarGamma(),\ rCauchy():}Simulação
  direta dos processos Neyman-Scott.
\end{itemize}

\section{Reamostragem e
Perturbação}\label{reamostragem-e-perturbauxe7uxe3o}

\begin{itemize}
\item
  \texttt{rshift(X):} Deslocamento toroidal (preserva estrutura interna,
  quebra relação com geofísica).
\item
  \texttt{rjitter(X):} Adiciona ruído às coordenadas.
\item
  \texttt{simulate(fit):} Simula de qualquer objeto ajustado (ppm, kppm,
  slrm).
\end{itemize}

\begin{Shaded}
\begin{Highlighting}[]
\ControlFlowTok{if}\NormalTok{(}\SpecialCharTok{!}\FunctionTok{require}\NormalTok{(pacman)) }\FunctionTok{install.packages}\NormalTok{(}\StringTok{"pacman"}\NormalTok{)}
\NormalTok{pacman}\SpecialCharTok{::}\FunctionTok{p\_load}\NormalTok{(spatstat)}

\CommentTok{\# Janela}
\NormalTok{W }\OtherTok{\textless{}{-}} \FunctionTok{owin}\NormalTok{(}\FunctionTok{c}\NormalTok{(}\DecValTok{0}\NormalTok{, }\DecValTok{10}\NormalTok{), }\FunctionTok{c}\NormalTok{(}\DecValTok{0}\NormalTok{, }\DecValTok{10}\NormalTok{))}

\CommentTok{\#PADRÕES BÁSICOS (Leves)}

\CommentTok{\# Uniforme (CSR)}
\NormalTok{X\_unif }\OtherTok{\textless{}{-}} \FunctionTok{runifpoint}\NormalTok{(}\AttributeTok{n =} \DecValTok{50}\NormalTok{, }\AttributeTok{win =}\NormalTok{ W)}

\CommentTok{\# Poisson (CSR)}
\NormalTok{X\_pois }\OtherTok{\textless{}{-}} \FunctionTok{rpoispp}\NormalTok{(}\AttributeTok{lambda =} \DecValTok{2}\NormalTok{, }\AttributeTok{win =}\NormalTok{ W)}

\CommentTok{\# Não homogêneo}
\NormalTok{f\_dens }\OtherTok{\textless{}{-}} \ControlFlowTok{function}\NormalTok{(x,y) \{ x }\SpecialCharTok{+}\NormalTok{ y \}}

\NormalTok{X\_inhom }\OtherTok{\textless{}{-}} \FunctionTok{rpoint}\NormalTok{(}\AttributeTok{n =} \DecValTok{50}\NormalTok{, }\AttributeTok{f =}\NormalTok{ f\_dens, }\AttributeTok{win =}\NormalTok{ W, }\AttributeTok{nsim=}\DecValTok{1}\NormalTok{) }

\FunctionTok{par}\NormalTok{(}\AttributeTok{mfrow=}\FunctionTok{c}\NormalTok{(}\DecValTok{1}\NormalTok{,}\DecValTok{3}\NormalTok{), }\AttributeTok{mar=}\FunctionTok{c}\NormalTok{(}\DecValTok{1}\NormalTok{,}\DecValTok{1}\NormalTok{,}\DecValTok{1}\NormalTok{,}\DecValTok{1}\NormalTok{))}
\FunctionTok{plot}\NormalTok{(X\_unif, }\AttributeTok{main=}\StringTok{"Uniforme (n=50)"}\NormalTok{)}
\FunctionTok{plot}\NormalTok{(X\_pois, }\AttributeTok{main=}\StringTok{"Poisson (lambda=2)"}\NormalTok{)}
\FunctionTok{plot}\NormalTok{(X\_inhom, }\AttributeTok{main=}\StringTok{"Não homogêneo"}\NormalTok{)}
\end{Highlighting}
\end{Shaded}

\pandocbounded{\includegraphics[keepaspectratio]{point_process_files/figure-pdf/unnamed-chunk-8-1.pdf}}

\begin{Shaded}
\begin{Highlighting}[]
\CommentTok{\# PROCESSOS DE GIBBS}
\NormalTok{ctrl\_leve }\OtherTok{\textless{}{-}} \FunctionTok{rmhcontrol}\NormalTok{(}\AttributeTok{expand=}\DecValTok{1}\NormalTok{, }\AttributeTok{nrep=}\FloatTok{1e4}\NormalTok{) }

\CommentTok{\# Strauss (Repulsão suave)}
\NormalTok{mod\_strauss }\OtherTok{\textless{}{-}} \FunctionTok{list}\NormalTok{(}\AttributeTok{cif=}\StringTok{"strauss"}\NormalTok{, }\AttributeTok{par=}\FunctionTok{list}\NormalTok{(}\AttributeTok{beta=}\DecValTok{1}\NormalTok{, }\AttributeTok{gamma=}\FloatTok{0.5}\NormalTok{, }\AttributeTok{r=}\FloatTok{0.7}\NormalTok{), }\AttributeTok{w=}\NormalTok{W)}
\NormalTok{X\_strauss }\OtherTok{\textless{}{-}} \FunctionTok{rmh}\NormalTok{(}\AttributeTok{model=}\NormalTok{mod\_strauss, }\AttributeTok{start=}\FunctionTok{list}\NormalTok{(}\AttributeTok{n.start=}\DecValTok{40}\NormalTok{), }\AttributeTok{control=}\NormalTok{ctrl\_leve)}
\end{Highlighting}
\end{Shaded}

\begin{verbatim}
Checking arguments..determining simulation windows...Starting simulation.
Initial state...Ready to simulate. Generating proposal points...Running Metropolis-Hastings.
\end{verbatim}

\begin{Shaded}
\begin{Highlighting}[]
\CommentTok{\# Hardcore (Inibição estrita)}
\NormalTok{mod\_hard }\OtherTok{\textless{}{-}} \FunctionTok{list}\NormalTok{(}\AttributeTok{cif=}\StringTok{"hardcore"}\NormalTok{, }\AttributeTok{par=}\FunctionTok{list}\NormalTok{(}\AttributeTok{beta=}\DecValTok{2}\NormalTok{, }\AttributeTok{hc=}\FloatTok{0.3}\NormalTok{), }\AttributeTok{w=}\NormalTok{W)}
\NormalTok{X\_hard }\OtherTok{\textless{}{-}} \FunctionTok{rmh}\NormalTok{(}\AttributeTok{model=}\NormalTok{mod\_hard, }\AttributeTok{start=}\FunctionTok{list}\NormalTok{(}\AttributeTok{n.start=}\DecValTok{30}\NormalTok{), }\AttributeTok{control=}\NormalTok{ctrl\_leve)}
\end{Highlighting}
\end{Shaded}

\begin{verbatim}
Checking arguments..determining simulation windows...Starting simulation.
Initial state...Ready to simulate. Generating proposal points...Running Metropolis-Hastings.
\end{verbatim}

\begin{Shaded}
\begin{Highlighting}[]
\CommentTok{\# Simulação}
\NormalTok{mod\_manual }\OtherTok{\textless{}{-}} \FunctionTok{rmhmodel}\NormalTok{(}\AttributeTok{cif=}\StringTok{"strauss"}\NormalTok{, }\AttributeTok{par=}\FunctionTok{list}\NormalTok{(}\AttributeTok{beta=}\DecValTok{1}\NormalTok{, }\AttributeTok{gamma=}\FloatTok{0.2}\NormalTok{, }\AttributeTok{r=}\FloatTok{0.7}\NormalTok{), }\AttributeTok{w=}\NormalTok{W)}
\NormalTok{X\_rmh }\OtherTok{\textless{}{-}} \FunctionTok{rmh}\NormalTok{(}\AttributeTok{model =}\NormalTok{ mod\_manual, }\AttributeTok{start =} \FunctionTok{list}\NormalTok{(}\AttributeTok{n.start=}\DecValTok{40}\NormalTok{), }\AttributeTok{control =}\NormalTok{ ctrl\_leve)}
\end{Highlighting}
\end{Shaded}

\begin{verbatim}
Checking arguments..determining simulation windows...Starting simulation.
Initial state...Ready to simulate. Generating proposal points...Running Metropolis-Hastings.
\end{verbatim}

\begin{Shaded}
\begin{Highlighting}[]
\FunctionTok{par}\NormalTok{(}\AttributeTok{mfrow=}\FunctionTok{c}\NormalTok{(}\DecValTok{1}\NormalTok{,}\DecValTok{3}\NormalTok{), }\AttributeTok{mar=}\FunctionTok{c}\NormalTok{(}\DecValTok{1}\NormalTok{,}\DecValTok{1}\NormalTok{,}\DecValTok{1}\NormalTok{,}\DecValTok{1}\NormalTok{))}
\FunctionTok{plot}\NormalTok{(X\_strauss, }\AttributeTok{main=}\StringTok{"Strauss"}\NormalTok{)}
\FunctionTok{plot}\NormalTok{(X\_hard, }\AttributeTok{main=}\StringTok{"Hardcore"}\NormalTok{)}
\FunctionTok{plot}\NormalTok{(X\_rmh, }\AttributeTok{main=}\StringTok{"rmh()"}\NormalTok{)}
\end{Highlighting}
\end{Shaded}

\pandocbounded{\includegraphics[keepaspectratio]{point_process_files/figure-pdf/unnamed-chunk-8-2.pdf}}

\begin{Shaded}
\begin{Highlighting}[]
\CommentTok{\#PROCESSOS DE CLUSTER (Neyman{-}Scott)}

\CommentTok{\# Thomas (Dispersão Gaussiana)}
\NormalTok{X\_thomas }\OtherTok{\textless{}{-}} \FunctionTok{rThomas}\NormalTok{(}\AttributeTok{kappa =} \FloatTok{0.5}\NormalTok{, }\AttributeTok{scale =} \FloatTok{0.2}\NormalTok{, }\AttributeTok{mu =} \DecValTok{10}\NormalTok{, }\AttributeTok{win =}\NormalTok{ W)}

\CommentTok{\# Matérn Cluster (Dispersão Uniforme)}
\NormalTok{X\_mat }\OtherTok{\textless{}{-}} \FunctionTok{rMatClust}\NormalTok{(}\AttributeTok{kappa =} \FloatTok{0.5}\NormalTok{, }\AttributeTok{scale =} \FloatTok{0.5}\NormalTok{, }\AttributeTok{mu =} \DecValTok{10}\NormalTok{, }\AttributeTok{win =}\NormalTok{ W)}

\CommentTok{\# Variância{-}Gama}
\NormalTok{X\_vargam }\OtherTok{\textless{}{-}} \FunctionTok{rVarGamma}\NormalTok{(}\AttributeTok{kappa =} \FloatTok{0.5}\NormalTok{, }\AttributeTok{scale =} \FloatTok{0.2}\NormalTok{, }\AttributeTok{mu =} \DecValTok{10}\NormalTok{, }\AttributeTok{win =}\NormalTok{ W)}

\FunctionTok{par}\NormalTok{(}\AttributeTok{mfrow=}\FunctionTok{c}\NormalTok{(}\DecValTok{1}\NormalTok{,}\DecValTok{3}\NormalTok{), }\AttributeTok{mar=}\FunctionTok{c}\NormalTok{(}\DecValTok{1}\NormalTok{,}\DecValTok{1}\NormalTok{,}\DecValTok{1}\NormalTok{,}\DecValTok{1}\NormalTok{))}
\FunctionTok{plot}\NormalTok{(X\_thomas, }\AttributeTok{main=}\StringTok{"Thomas"}\NormalTok{)}
\FunctionTok{plot}\NormalTok{(X\_mat, }\AttributeTok{main=}\StringTok{"Matérn Cluster"}\NormalTok{)}
\FunctionTok{plot}\NormalTok{(X\_vargam, }\AttributeTok{main=}\StringTok{"Var{-}Gamma"}\NormalTok{)}
\end{Highlighting}
\end{Shaded}

\pandocbounded{\includegraphics[keepaspectratio]{point_process_files/figure-pdf/unnamed-chunk-8-3.pdf}}

\begin{Shaded}
\begin{Highlighting}[]
\CommentTok{\#REAMOSTRAGEM E PERTURBAÇÃO}

\FunctionTok{data}\NormalTok{(cells)}

\CommentTok{\# Deslocamento}
\NormalTok{X\_shift }\OtherTok{\textless{}{-}} \FunctionTok{rshift}\NormalTok{(cells, }\AttributeTok{radius =} \FloatTok{0.1}\NormalTok{)}

\CommentTok{\# Jittering}
\NormalTok{X\_jitter }\OtherTok{\textless{}{-}} \FunctionTok{rjitter}\NormalTok{(cells, }\AttributeTok{radius =} \FloatTok{0.05}\NormalTok{)}

\FunctionTok{par}\NormalTok{(}\AttributeTok{mfrow=}\FunctionTok{c}\NormalTok{(}\DecValTok{1}\NormalTok{,}\DecValTok{3}\NormalTok{), }\AttributeTok{mar=}\FunctionTok{c}\NormalTok{(}\DecValTok{1}\NormalTok{,}\DecValTok{1}\NormalTok{,}\DecValTok{1}\NormalTok{,}\DecValTok{1}\NormalTok{))}
\FunctionTok{plot}\NormalTok{(cells, }\AttributeTok{main=}\StringTok{"Original"}\NormalTok{, }\AttributeTok{pch=}\DecValTok{16}\NormalTok{)}
\FunctionTok{plot}\NormalTok{(X\_shift, }\AttributeTok{main=}\StringTok{"Shift"}\NormalTok{, }\AttributeTok{pch=}\DecValTok{16}\NormalTok{, }\AttributeTok{col=}\StringTok{"blue"}\NormalTok{)}
\FunctionTok{plot}\NormalTok{(X\_jitter, }\AttributeTok{main=}\StringTok{"Jitter"}\NormalTok{, }\AttributeTok{pch=}\DecValTok{16}\NormalTok{, }\AttributeTok{col=}\StringTok{"red"}\NormalTok{)}
\end{Highlighting}
\end{Shaded}

\pandocbounded{\includegraphics[keepaspectratio]{point_process_files/figure-pdf/unnamed-chunk-8-4.pdf}}

\begin{Shaded}
\begin{Highlighting}[]
\CommentTok{\# SIMULAÇÃO DE MODELOS AJUSTADOS}
\NormalTok{fit\_strauss }\OtherTok{\textless{}{-}} \FunctionTok{ppm}\NormalTok{(cells }\SpecialCharTok{\textasciitilde{}} \DecValTok{1}\NormalTok{, }\FunctionTok{Strauss}\NormalTok{(}\AttributeTok{r=}\FloatTok{0.07}\NormalTok{))}

\NormalTok{sim\_fit }\OtherTok{\textless{}{-}} \FunctionTok{simulate}\NormalTok{(fit\_strauss, }\AttributeTok{nsim =} \DecValTok{2}\NormalTok{, }\AttributeTok{control=}\NormalTok{ctrl\_leve, }\AttributeTok{retry=}\DecValTok{0}\NormalTok{)}
\end{Highlighting}
\end{Shaded}

\begin{verbatim}
Generating 2 simulated patterns ...1, 
2.
\end{verbatim}

\begin{Shaded}
\begin{Highlighting}[]
\FunctionTok{par}\NormalTok{(}\AttributeTok{mfrow=}\FunctionTok{c}\NormalTok{(}\DecValTok{1}\NormalTok{,}\DecValTok{2}\NormalTok{), }\AttributeTok{mar=}\FunctionTok{c}\NormalTok{(}\DecValTok{1}\NormalTok{,}\DecValTok{1}\NormalTok{,}\DecValTok{1}\NormalTok{,}\DecValTok{1}\NormalTok{))}
\FunctionTok{plot}\NormalTok{(sim\_fit[[}\DecValTok{1}\NormalTok{]], }\AttributeTok{main=}\StringTok{"Simulação 1"}\NormalTok{)}
\FunctionTok{plot}\NormalTok{(sim\_fit[[}\DecValTok{2}\NormalTok{]], }\AttributeTok{main=}\StringTok{"Simulação 2"}\NormalTok{)}
\end{Highlighting}
\end{Shaded}

\pandocbounded{\includegraphics[keepaspectratio]{point_process_files/figure-pdf/unnamed-chunk-8-5.pdf}}

\section{Testes e Diagnósticos}\label{testes-e-diagnuxf3sticos}

Validação estatística rigorosa dos modelos.

\subsection{Testes de Bondade de Ajuste
(Goodness-of-Fit)}\label{testes-de-bondade-de-ajuste-goodness-of-fit}

*\texttt{quadrat.test(fit):} \(\chi^2\) para modelos ajustados.

\begin{itemize}
\item
  \texttt{cdf.test(fit,\ covariate):} Kolmogorov-Smirnov espacial.
  Compara distribuição da covariável nos pontos vs.~no mapa.
\item
  \texttt{dclf.test(X,\ model):} Diggle-Cressie-Loosmore-Ford. Teste de
  Monte Carlo comparando envelopes globais (L ou K).
\item
  \texttt{mad.test(X,\ model):} Desvio Absoluto Médio. Similar ao DCLF,
  mas usa norma \(L_1\).
\item
  \texttt{dg.test(X):} Teste de Dao-Genton.
\end{itemize}

\subsection{Diagnósticos de Resíduos
(PPM)}\label{diagnuxf3sticos-de-resuxedduos-ppm}

\begin{itemize}
\item
  \texttt{residuals.ppm(fit):} Resíduos brutos, Pearson, e
  ``Inverse-Lambda''.
\item
  \texttt{diagnose.ppm(fit):} Plota a tendência cumulativa vs resíduos.
  Mostra onde o modelo falha.
\end{itemize}

\subsection{Diagnósticos de Influência e
Sensibilidade}\label{diagnuxf3sticos-de-influuxeancia-e-sensibilidade}

\begin{itemize}
\item
  \texttt{leverage.ppm(fit):} Onde a adição de um ponto altera mais o
  modelo?
\item
  \texttt{influence.ppm(fit):} Influência global.
\item
  \texttt{dfbetas.ppm(fit):} Mudança nos coeficientes dos parâmetros.
\item
  \texttt{parres(fit,\ covariate):} Partial Residuals. Sugere
  transformação funcional para covariável.
\item
  \texttt{addvar(fit,\ covariate):} Added Variable Plot. A covariável
  adiciona informação nova?
\end{itemize}

\begin{Shaded}
\begin{Highlighting}[]
\ControlFlowTok{if}\NormalTok{(}\SpecialCharTok{!}\FunctionTok{require}\NormalTok{(pacman)) }\FunctionTok{install.packages}\NormalTok{(}\StringTok{"pacman"}\NormalTok{)}
\NormalTok{pacman}\SpecialCharTok{::}\FunctionTok{p\_load}\NormalTok{(spatstat)}
\FunctionTok{par}\NormalTok{(}\AttributeTok{mfrow =} \FunctionTok{c}\NormalTok{(}\DecValTok{1}\NormalTok{, }\DecValTok{1}\NormalTok{), }\AttributeTok{mar=}\FunctionTok{c}\NormalTok{(}\DecValTok{1}\NormalTok{,}\DecValTok{1}\NormalTok{,}\DecValTok{1}\NormalTok{,}\DecValTok{1}\NormalTok{))}
\CommentTok{\# Dados: Árvores (bei) e Covariáveis (bei.extra)}
\FunctionTok{data}\NormalTok{(bei)}
\NormalTok{grad }\OtherTok{\textless{}{-}}\NormalTok{ bei.extra}\SpecialCharTok{$}\NormalTok{grad }\CommentTok{\# Gradiente do terreno}
\NormalTok{elev }\OtherTok{\textless{}{-}}\NormalTok{ bei.extra}\SpecialCharTok{$}\NormalTok{elev }\CommentTok{\# Elevação}

\CommentTok{\# Ajuste de um Modelo de Poisson ñ nhomogêneo (dependente do gradiente)}
\NormalTok{fit }\OtherTok{\textless{}{-}} \FunctionTok{ppm}\NormalTok{(bei }\SpecialCharTok{\textasciitilde{}}\NormalTok{ grad)}

\CommentTok{\#TESTES DE BOM AJUSTE (Goodness{-}of{-}Fit)}


\CommentTok{\# Teste de Quadrats (Chi{-}quadrado espacial)}
\CommentTok{\# H0: O modelo ajustado é adequado}
\NormalTok{qt }\OtherTok{\textless{}{-}} \FunctionTok{quadrat.test}\NormalTok{(fit, }\AttributeTok{nx =} \DecValTok{4}\NormalTok{, }\AttributeTok{ny =} \DecValTok{2}\NormalTok{)}
\FunctionTok{plot}\NormalTok{(qt, }\AttributeTok{main =} \StringTok{"Quadrat Test (Resíduos Pearson)"}\NormalTok{)}
\end{Highlighting}
\end{Shaded}

\pandocbounded{\includegraphics[keepaspectratio]{point_process_files/figure-pdf/unnamed-chunk-9-1.pdf}}

\begin{Shaded}
\begin{Highlighting}[]
\CommentTok{\# Teste de Kolmogorov{-}Smirnov Espacial (CDF)}
\CommentTok{\# Verifica se a distribuição da covariável nos pontos difere do mapa geral}
\NormalTok{cdf\_t }\OtherTok{\textless{}{-}} \FunctionTok{cdf.test}\NormalTok{(fit, }\AttributeTok{covariate=}\NormalTok{grad)}
\FunctionTok{plot}\NormalTok{(cdf\_t, }\AttributeTok{main =} \StringTok{"Teste CDF (Covariável: Grad)"}\NormalTok{)}
\end{Highlighting}
\end{Shaded}

\pandocbounded{\includegraphics[keepaspectratio]{point_process_files/figure-pdf/unnamed-chunk-9-2.pdf}}

\begin{Shaded}
\begin{Highlighting}[]
\CommentTok{\# Testes de Monte Carlo (Comparação de Envelopes Globais)}
\CommentTok{\# H0: O padrão observado é gerado pelo modelo ajustado}

\NormalTok{dclf\_t }\OtherTok{\textless{}{-}} \FunctionTok{dclf.test}\NormalTok{(fit, Lest, }\AttributeTok{nsim =} \DecValTok{19}\NormalTok{)}
\end{Highlighting}
\end{Shaded}

\begin{verbatim}
Generating 19 simulated realisations of fitted Poisson model  ...
1, 2, 3, 4, 5, 6, 7, 8, 9, 10, 11, 12, 13, 14, 15, 16, 17, 18, 
19.

Done.
\end{verbatim}

\begin{Shaded}
\begin{Highlighting}[]
\FunctionTok{print}\NormalTok{(dclf\_t)}
\end{Highlighting}
\end{Shaded}

\begin{verbatim}

    Diggle-Cressie-Loosmore-Ford test of fitted Poisson model
    Monte Carlo test based on 19 simulations
    Summary function: L(r)
    Reference function: sample mean
    Alternative: two.sided
    Interval of distance values: [0, 125] metres
    Test statistic: Integral of squared absolute deviation
    Deviation = leave-one-out

data:  fit
u = 48139, rank = 1, p-value = 0.05
\end{verbatim}

\begin{Shaded}
\begin{Highlighting}[]
\CommentTok{\# MAD (Desvio Absoluto Médio)}
\NormalTok{mad\_t }\OtherTok{\textless{}{-}} \FunctionTok{mad.test}\NormalTok{(fit, Lest, }\AttributeTok{nsim =} \DecValTok{19}\NormalTok{)}
\end{Highlighting}
\end{Shaded}

\begin{verbatim}
Generating 19 simulated realisations of fitted Poisson model  ...
1, 2, 3, 4, 5, 6, 7, 8, 9, 10, 11, 12, 13, 14, 15, 16, 17, 18, 
19.

Done.
\end{verbatim}

\begin{Shaded}
\begin{Highlighting}[]
\FunctionTok{print}\NormalTok{(mad\_t)}
\end{Highlighting}
\end{Shaded}

\begin{verbatim}

    Maximum absolute deviation test of fitted Poisson model
    Monte Carlo test based on 19 simulations
    Summary function: L(r)
    Reference function: sample mean
    Alternative: two.sided
    Interval of distance values: [0, 125] metres
    Test statistic: Maximum absolute deviation
    Deviation = leave-one-out

data:  fit
mad = 23.396, rank = 1, p-value = 0.05
\end{verbatim}

\begin{Shaded}
\begin{Highlighting}[]
\CommentTok{\# Teste de Dao{-}Genton}
\NormalTok{dg\_t }\OtherTok{\textless{}{-}} \FunctionTok{dg.test}\NormalTok{(fit, Lest, }\AttributeTok{nsim =} \DecValTok{19}\NormalTok{)}
\end{Highlighting}
\end{Shaded}

\begin{verbatim}
Applying first-stage test to original data... Done.
Running second-stage tests on 19 simulated patterns... 1, 2, 3, 4, 5, 6, 7, 8, 9, 10, 11, 12, 13, 14, 15, 16, 17, 18, 
19.
\end{verbatim}

\begin{Shaded}
\begin{Highlighting}[]
\FunctionTok{print}\NormalTok{(dg\_t)}
\end{Highlighting}
\end{Shaded}

\begin{verbatim}

    Dao-Genton adjusted goodness-of-fit test
    based on Diggle-Cressie-Loosmore-Ford test of fitted Poisson model
    First stage: Monte Carlo test based on 19 simulations
    Summary function: L(r)
    Reference function: sample mean
    Alternative: two.sided
    Interval of distance values: [0, 125] metres
    Test statistic: Integral of squared absolute deviation
    Deviation = leave-one-out
    Second stage: nested, 18 simulations for each first-stage simulation

data:  X
p0 = 0.05, p-value = 0.05
\end{verbatim}

\begin{Shaded}
\begin{Highlighting}[]
\CommentTok{\# DIAGNÓSTICOS DE RESÍDUOS (PPM)}

\NormalTok{res }\OtherTok{\textless{}{-}} \FunctionTok{residuals}\NormalTok{(fit, }\AttributeTok{type =} \StringTok{"pearson"}\NormalTok{)}


\FunctionTok{diagnose.ppm}\NormalTok{(fit, }\AttributeTok{which =} \StringTok{"all"}\NormalTok{, }\AttributeTok{main =} \StringTok{"Diagnóstico Geral"}\NormalTok{)}
\end{Highlighting}
\end{Shaded}

\pandocbounded{\includegraphics[keepaspectratio]{point_process_files/figure-pdf/unnamed-chunk-9-3.pdf}}

\begin{verbatim}
Model diagnostics (raw residuals)
Diagnostics available:
    four-panel plot
    mark plot 
    smoothed residual field
    x cumulative residuals
    y cumulative residuals
    sum of all residuals
sum of raw residuals in entire window = -6.924e-09
area of entire window = 5e+05
quadrature area = 5e+05
range of smoothed field =  [-0.005587, 0.008917]
\end{verbatim}

\begin{Shaded}
\begin{Highlighting}[]
\CommentTok{\#DIAGNÓSTICOS DE INFLUÊNCIA E SENSIBILIDADE}

\CommentTok{\# Leverage: Onde a presença de um ponto afeta mais o ajuste?}
\NormalTok{lev }\OtherTok{\textless{}{-}} \FunctionTok{leverage}\NormalTok{(fit)}

\CommentTok{\# Influence: Medida global de influência (DFBetas espacializado)}
\NormalTok{infl }\OtherTok{\textless{}{-}} \FunctionTok{influence}\NormalTok{(fit)}

\CommentTok{\# DFBetas: Mudança nos coeficientes se removermos um ponto}
\NormalTok{dfb }\OtherTok{\textless{}{-}} \FunctionTok{dfbetas}\NormalTok{(fit)}



\FunctionTok{plot}\NormalTok{(lev, }\AttributeTok{main =} \StringTok{"Leverage"}\NormalTok{)}
\end{Highlighting}
\end{Shaded}

\pandocbounded{\includegraphics[keepaspectratio]{point_process_files/figure-pdf/unnamed-chunk-9-4.pdf}}

\begin{Shaded}
\begin{Highlighting}[]
\FunctionTok{plot}\NormalTok{(infl, }\AttributeTok{main =} \StringTok{"Influence"}\NormalTok{)}
\FunctionTok{points}\NormalTok{(bei, }\AttributeTok{pch =} \StringTok{"."}\NormalTok{, }\AttributeTok{col =} \StringTok{"white"}\NormalTok{) }\CommentTok{\# Sobrepor pontos}
\end{Highlighting}
\end{Shaded}

\pandocbounded{\includegraphics[keepaspectratio]{point_process_files/figure-pdf/unnamed-chunk-9-5.pdf}}

\begin{Shaded}
\begin{Highlighting}[]
\FunctionTok{plot}\NormalTok{(dfb, }\AttributeTok{main =} \StringTok{"DFBetas"}\NormalTok{)}
\end{Highlighting}
\end{Shaded}

\pandocbounded{\includegraphics[keepaspectratio]{point_process_files/figure-pdf/unnamed-chunk-9-6.pdf}}

\begin{Shaded}
\begin{Highlighting}[]
\CommentTok{\# Partial Residuals (Sugere transformação funcional)}
\CommentTok{\# Se a linha verde (alisada) desvia da reta tracejada, a relação não é linear}
\FunctionTok{parres}\NormalTok{(fit, }\StringTok{"grad"}\NormalTok{, }\AttributeTok{main =} \StringTok{"Resíduos Parciais (Grad)"}\NormalTok{)}
\end{Highlighting}
\end{Shaded}

\begin{verbatim}
Transformation diagnostic (class parres)
for the canonical covariate 'grad' in the fitted model  ppm.formula(bei ~ grad) 
Call: parres.ppm(fit, "grad", main = "Resíduos Parciais (Grad)") 
Actual smoothing bandwidth sigma = 0.010404 
Algorithm: effect + smooth(residual)

Function value object (class 'fv')
for the function grad -> h(grad)
.........................................................................
     Math.label Description                                              
grad grad       covariate 'grad'                                         
h    h(grad)    Smoothed partial residual                                
varh var(grad)  Variance                                                 
hi   hi(grad)   Upper limit of pointwise 5%% significance band (integral)
lo   lo(grad)   Lower limit of pointwise 5%% significance band (integral)
hiX  hiX(grad)  Upper limit of pointwise 5%% significance band (sum)     
loX  loX(grad)  Lower limit of pointwise 5%% significance band (sum)     
fit  fit(grad)  Parametric fitted effect of 'grad'                       
.........................................................................
Default plot formula:  .~grad
where "." stands for 'h', 'hi', 'lo', 'fit'
Columns 'hi' and 'lo' will be plotted as shading (by default)
Recommended range of argument grad: [0.00086634, 0.32848]
Available range of argument grad: [0.00086634, 0.32848]
\end{verbatim}

\begin{Shaded}
\begin{Highlighting}[]
\CommentTok{\# Added Variable Plot (A covariável \textquotesingle{}elev\textquotesingle{} adiciona informação nova?)}
\CommentTok{\# Testa se devemos adicionar \textquotesingle{}elev\textquotesingle{} ao modelo que já tem \textquotesingle{}grad\textquotesingle{}}
\FunctionTok{addvar}\NormalTok{(fit, }\AttributeTok{covariate=}\NormalTok{elev, }\AttributeTok{data =} \FunctionTok{list}\NormalTok{(}\AttributeTok{elev =}\NormalTok{ elev), }\AttributeTok{main =} \StringTok{"Added Variable (Elev)"}\NormalTok{)}
\end{Highlighting}
\end{Shaded}

\begin{verbatim}
Added variable plot diagnostic (class addvar)
for the covariate "elev" for the fitted model:  ppm.formula(bei ~ grad) 

Call: addvar(fit, covariate = elev, data = list(elev = elev), main = "Added Variable (Elev)") 
Actual smoothing bandwidth sigma = 0.082804 

Function value object (class 'fv')
for the function r(paste(elev,'|',grad)) -> r(paste(points,"|",grad))
................................................................................
     Math.label                         
rcov r(paste(elev,'|',grad))            
rpts r(paste(points,'|',grad))          
theo 0                                  
var  bold(var)~r(paste(points,'|',grad))
hi   r(paste(points,'|',grad))[hi]      
lo   r(paste(points,'|',grad))[lo]      
     Description                                      
rcov Pearson residual of covariate elev | grad        
rpts Smoothed Pearson residual of point process | grad
theo Null expected value of point process residual    
var  Null variance of point process residual          
hi   Upper limit of pointwise 5%% significance band   
lo   Lower limit of pointwise 5%% significance band   
................................................................................
Default plot formula:  .~rcov
where "." stands for 'rpts', 'theo', 'hi', 'lo'
Recommended range of argument rcov: [-2.1742, 2.0845]
Available range of argument rcov: [-2.1742, 2.0845]
\end{verbatim}

\section{Redes Lineares
(spatstat.linnet)}\label{redes-lineares-spatstat.linnet}

Para dados em ruas, rios ou grafos (geometria não-Euclidiana).

\textbf{Infraestrutura (linnet, lpp)}

\begin{itemize}
\item
  Rede (linnet):

  \begin{enumerate}
  \def\labelenumi{\arabic{enumi}.}
  \item
    \texttt{linnet(vertices,\ m,\ edges):} Criação manual.
  \item
    \texttt{as.linnet(psp):} Converte segmentos em rede.
  \item
    \texttt{thinNetwork(L,\ retain):} Remove arestas.
  \item
    \texttt{insertVertices(L):} Adiciona nós.
  \item
    \texttt{connected.linnet(L):} Acha componentes conexos.
  \item
    \texttt{vertices(L):} Extrai nós.
  \end{enumerate}
\item
  Pontos na Rede (lpp):

  \begin{enumerate}
  \def\labelenumi{\arabic{enumi}.}
  \item
    \texttt{lpp(coords,\ L):} Cria pontos na rede.
  \item
    \texttt{as.lpp(x,\ y,\ L):} Projeta pontos 2D na rede mais próxima.
  \item
    \texttt{clicklpp(L):} Adição interativa.
  \item
    \texttt{distmap(X):} Mapa de distância geodésica (shortest-path).
  \end{enumerate}
\end{itemize}

\subsection{Análise Exploratória na
Rede}\label{anuxe1lise-exploratuxf3ria-na-rede}

\textbf{Intensidade}

\begin{itemize}
\item
  \texttt{density.lpp(X,\ distance="path"):} Kernel usando ``calor'' na
  rede (respeita topologia).
\item
  \texttt{bw.lppl(X):} Validação cruzada para largura de banda na rede.
\end{itemize}

\textbf{Estatísticas K e PCF:}

\begin{itemize}
\item
  \texttt{linearK(X),\ linearKinhom(X):} K de Ripley com distância de
  caminho.
\item
  \texttt{linearpcf(X),\ linearpcfinhom(X):} Correlação de pares na
  rede.
\end{itemize}

\textbf{Tesselações e Imagens na Rede}

\begin{itemize}
\item
  \texttt{linfun(f,\ L):} Define função matemática na rede.
\item
  \texttt{linim(X):} Imagem pixelada restrita à rede.
\item
  \texttt{lintess(L):} Tesselação da rede.
\item
  \texttt{lineardirichlet(X):} Voronoi métrico na rede.
\item
  \texttt{divide.linnet(L):} Particionamento.
\end{itemize}

\textbf{Modelagem e Validação (lppm)}

\begin{itemize}
\tightlist
\item
  Modelo:
\end{itemize}

\begin{enumerate}
\def\labelenumi{\arabic{enumi}.}
\tightlist
\item
  \texttt{lppm(formula):} Ajusta Poisson/Gibbs na rede.
\end{enumerate}

\textbf{Validação e Diagnóstico}

\begin{itemize}
\item
  \texttt{cdf.test.lpp(fit):} KS na rede.
\item
  \texttt{berman.test.lpp(X):} Teste Z1/Z2 de Berman.
\item
  \texttt{quadrat.test.lpp(X):} Qui-quadrado na rede.
\item
  \texttt{residuals.lppm(fit):} Resíduos.
\item
  \texttt{lurking(fit):} Gráfico de variável oculta (lurking variable
  plot).
\item
  \texttt{roc.lppm(fit),\ auc.lppm(fit):} ROC para modelos de rede.
\end{itemize}

\textbf{Simulação}

\begin{itemize}
\item
  \texttt{rpoislpp(lambda,\ L):} Poisson na rede.
\item
  \texttt{rThomaslpp(L):} Cluster Thomas na rede.
\item
  \texttt{rjitter.lpp(X):} Perturbação sobre as linhas.
\item
  \texttt{simulate.lppm(fit):} Simulação do modelo.
\end{itemize}

\begin{Shaded}
\begin{Highlighting}[]
\ControlFlowTok{if}\NormalTok{(}\SpecialCharTok{!}\FunctionTok{require}\NormalTok{(pacman)) }\FunctionTok{install.packages}\NormalTok{(}\StringTok{"pacman"}\NormalTok{)}
\NormalTok{pacman}\SpecialCharTok{::}\FunctionTok{p\_load}\NormalTok{(spatstat)}

\CommentTok{\#INFRAESTRUTURA (Redes e Pontos)}

\FunctionTok{data}\NormalTok{(simplenet)}
\NormalTok{L }\OtherTok{\textless{}{-}}\NormalTok{ simplenet}

\CommentTok{\# Criação de pontos na rede}
\FunctionTok{set.seed}\NormalTok{(}\DecValTok{42}\NormalTok{)}
\NormalTok{X }\OtherTok{\textless{}{-}} \FunctionTok{rpoislpp}\NormalTok{(}\AttributeTok{lambda =} \DecValTok{50}\NormalTok{, }\AttributeTok{L =}\NormalTok{ L)}

\CommentTok{\# Projetar pontos 2D para a rede}
\NormalTok{pts\_2d }\OtherTok{\textless{}{-}} \FunctionTok{runifpoint}\NormalTok{(}\DecValTok{20}\NormalTok{, }\AttributeTok{win =} \FunctionTok{Window}\NormalTok{(L))}
\NormalTok{X\_proj }\OtherTok{\textless{}{-}} \FunctionTok{as.lpp}\NormalTok{(pts\_2d, }\AttributeTok{L =}\NormalTok{ L)}

\CommentTok{\# Mapa de Distância Geodésica}
\NormalTok{dist\_img }\OtherTok{\textless{}{-}} \FunctionTok{distmap}\NormalTok{(X)}

\FunctionTok{par}\NormalTok{(}\AttributeTok{mfrow=}\FunctionTok{c}\NormalTok{(}\DecValTok{1}\NormalTok{,}\DecValTok{3}\NormalTok{))}
\FunctionTok{plot}\NormalTok{(L, }\AttributeTok{main=}\StringTok{"Rede (simplenet)"}\NormalTok{)}
\FunctionTok{plot}\NormalTok{(X, }\AttributeTok{main=}\StringTok{"Pontos (lpp)"}\NormalTok{)}
\FunctionTok{plot}\NormalTok{(dist\_img, }\AttributeTok{main=}\StringTok{"Mapa de Distância"}\NormalTok{)}
\end{Highlighting}
\end{Shaded}

\pandocbounded{\includegraphics[keepaspectratio]{point_process_files/figure-pdf/unnamed-chunk-10-1.pdf}}

\begin{Shaded}
\begin{Highlighting}[]
\CommentTok{\# ANÁLISE EXPLORATÓRIA}

\NormalTok{den }\OtherTok{\textless{}{-}} \FunctionTok{density}\NormalTok{(X, }\AttributeTok{sigma=}\FloatTok{0.1}\NormalTok{, }\AttributeTok{distance=}\StringTok{"path"}\NormalTok{)}
\NormalTok{K\_lin }\OtherTok{\textless{}{-}} \FunctionTok{linearK}\NormalTok{(X)}
\NormalTok{g\_lin }\OtherTok{\textless{}{-}} \FunctionTok{linearpcf}\NormalTok{(X)}

\FunctionTok{par}\NormalTok{(}\AttributeTok{mfrow=}\FunctionTok{c}\NormalTok{(}\DecValTok{1}\NormalTok{,}\DecValTok{3}\NormalTok{))}
\FunctionTok{plot}\NormalTok{(den, }\AttributeTok{main=}\StringTok{"Densidade Kernel"}\NormalTok{)}
\FunctionTok{plot}\NormalTok{(K\_lin, }\AttributeTok{main=}\StringTok{"K Linear"}\NormalTok{)}
\FunctionTok{plot}\NormalTok{(g\_lin, }\AttributeTok{main=}\StringTok{"Pair Correlation (PCF)"}\NormalTok{)}
\end{Highlighting}
\end{Shaded}

\pandocbounded{\includegraphics[keepaspectratio]{point_process_files/figure-pdf/unnamed-chunk-10-2.pdf}}

\begin{Shaded}
\begin{Highlighting}[]
\CommentTok{\#  TESSELAÇÕES E IMAGENS}

\NormalTok{voro\_net }\OtherTok{\textless{}{-}} \FunctionTok{lineardirichlet}\NormalTok{(X)}
\NormalTok{f\_net }\OtherTok{\textless{}{-}} \FunctionTok{linfun}\NormalTok{(}\ControlFlowTok{function}\NormalTok{(x,y,seg,tp) \{ x }\SpecialCharTok{+}\NormalTok{ y \}, L)}
\NormalTok{im\_net }\OtherTok{\textless{}{-}} \FunctionTok{as.linim}\NormalTok{(f\_net) }

\FunctionTok{par}\NormalTok{(}\AttributeTok{mfrow=}\FunctionTok{c}\NormalTok{(}\DecValTok{1}\NormalTok{,}\DecValTok{2}\NormalTok{))}
\FunctionTok{plot}\NormalTok{(voro\_net, }\AttributeTok{main=}\StringTok{"Voronoi Linear"}\NormalTok{)}
\FunctionTok{plot}\NormalTok{(im\_net, }\AttributeTok{main=}\StringTok{"Imagem Linear"}\NormalTok{)}
\end{Highlighting}
\end{Shaded}

\pandocbounded{\includegraphics[keepaspectratio]{point_process_files/figure-pdf/unnamed-chunk-10-3.pdf}}

\begin{Shaded}
\begin{Highlighting}[]
\CommentTok{\#  MODELAGEM (lppm)}


\FunctionTok{data}\NormalTok{(chicago)}
\NormalTok{chicago\_clean }\OtherTok{\textless{}{-}} \FunctionTok{unmark}\NormalTok{(chicago)}

\CommentTok{\# Ajuste de Modelo Poisson}
\NormalTok{fit\_lppm }\OtherTok{\textless{}{-}} \FunctionTok{lppm}\NormalTok{(chicago\_clean }\SpecialCharTok{\textasciitilde{}}\NormalTok{ x }\SpecialCharTok{+}\NormalTok{ y)}

\FunctionTok{print}\NormalTok{(}\FunctionTok{AIC}\NormalTok{(fit\_lppm))}
\end{Highlighting}
\end{Shaded}

\begin{verbatim}
[1] 1498.881
\end{verbatim}

\begin{Shaded}
\begin{Highlighting}[]
\CommentTok{\# VALIDAÇÃO E DIAGNÓSTICO}

\CommentTok{\#Teste de Qui{-}quadrado}
\CommentTok{\# Apenas imprimimos o resultado estatístico. Plotar gera erro em redes lineares.}
\NormalTok{qt\_net }\OtherTok{\textless{}{-}} \FunctionTok{quadrat.test}\NormalTok{(fit\_lppm, }\AttributeTok{nx=}\DecValTok{3}\NormalTok{, }\AttributeTok{ny=}\DecValTok{3}\NormalTok{)}
\FunctionTok{print}\NormalTok{(qt\_net) }
\end{Highlighting}
\end{Shaded}

\begin{verbatim}

    Chi-squared test of fitted Poisson model 'fit_lppm' on network using
    quadrat counts

data:  data from fit_lppm
X2 = 11.875, df = 6, p-value = 0.1296
alternative hypothesis: two.sided

Quadrats: Tessellation on a linear network
9 tiles
\end{verbatim}

\begin{Shaded}
\begin{Highlighting}[]
\CommentTok{\#Visualização da Predição}
\NormalTok{pred }\OtherTok{\textless{}{-}} \FunctionTok{predict}\NormalTok{(fit\_lppm)}
\FunctionTok{par}\NormalTok{(}\AttributeTok{mfrow=}\FunctionTok{c}\NormalTok{(}\DecValTok{1}\NormalTok{,}\DecValTok{2}\NormalTok{))}
\FunctionTok{plot}\NormalTok{(pred, }\AttributeTok{main=}\StringTok{"Intensidade Ajustada (Modelo)"}\NormalTok{)}
\FunctionTok{plot}\NormalTok{(chicago\_clean, }\AttributeTok{add=}\ConstantTok{TRUE}\NormalTok{, }\AttributeTok{pch=}\StringTok{"."}\NormalTok{, }\AttributeTok{cols=}\StringTok{"white"}\NormalTok{) }\CommentTok{\# Sobrepor pontos}

\CommentTok{\# 3. Teste CDF}
\NormalTok{cdf\_net }\OtherTok{\textless{}{-}} \FunctionTok{cdf.test}\NormalTok{(fit\_lppm, }\StringTok{"x"}\NormalTok{) }
\FunctionTok{plot}\NormalTok{(cdf\_net, }\AttributeTok{main=}\StringTok{"CDF Test (Coord X)"}\NormalTok{)}
\end{Highlighting}
\end{Shaded}

\pandocbounded{\includegraphics[keepaspectratio]{point_process_files/figure-pdf/unnamed-chunk-10-4.pdf}}

\begin{Shaded}
\begin{Highlighting}[]
\CommentTok{\# 4. Curva ROC}
\FunctionTok{par}\NormalTok{(}\AttributeTok{mfrow=}\FunctionTok{c}\NormalTok{(}\DecValTok{1}\NormalTok{,}\DecValTok{1}\NormalTok{))}
\NormalTok{roc\_curve }\OtherTok{\textless{}{-}} \FunctionTok{roc}\NormalTok{(fit\_lppm)}
\FunctionTok{plot}\NormalTok{(roc\_curve, }\AttributeTok{main=}\StringTok{"Curva ROC"}\NormalTok{)}
\end{Highlighting}
\end{Shaded}

\pandocbounded{\includegraphics[keepaspectratio]{point_process_files/figure-pdf/unnamed-chunk-10-5.pdf}}

\begin{Shaded}
\begin{Highlighting}[]
\FunctionTok{print}\NormalTok{(}\FunctionTok{auc}\NormalTok{(fit\_lppm))}
\end{Highlighting}
\end{Shaded}

\begin{verbatim}
      obs      theo 
0.6598881 0.6598997 
\end{verbatim}

\begin{Shaded}
\begin{Highlighting}[]
\CommentTok{\# SIMULAÇÃO }

\CommentTok{\# Simular do modelo ajustado (lppm)}
\NormalTok{X\_sim\_fit }\OtherTok{\textless{}{-}} \FunctionTok{simulate}\NormalTok{(fit\_lppm, }\AttributeTok{nsim=}\DecValTok{1}\NormalTok{)[[}\DecValTok{1}\NormalTok{]]}

\CommentTok{\# Simular Cluster de Thomas na Rede (Método de Projeção)}
\NormalTok{Net\_Chi }\OtherTok{\textless{}{-}} \FunctionTok{domain}\NormalTok{(chicago\_clean) }

\CommentTok{\#Simulamos o processo Thomas em 2D (usando a janela da rede)}
\CommentTok{\# kappa: intensidade dos pais, scale: dispersão, mu: filhos}
\NormalTok{X\_thomas\_2d }\OtherTok{\textless{}{-}} \FunctionTok{rThomas}\NormalTok{(}\AttributeTok{kappa=}\FloatTok{0.01}\NormalTok{, }\AttributeTok{scale=}\DecValTok{100}\NormalTok{, }\AttributeTok{mu=}\DecValTok{5}\NormalTok{, }\AttributeTok{win=}\FunctionTok{Window}\NormalTok{(Net\_Chi))}

\CommentTok{\# Projetamos os pontos 2D para a rede linear}
\NormalTok{X\_thomas }\OtherTok{\textless{}{-}} \FunctionTok{as.lpp}\NormalTok{(X\_thomas\_2d, }\AttributeTok{L=}\NormalTok{Net\_Chi)}

\CommentTok{\# Jittering (Perturbação)}
\CommentTok{\# Usamos rjitter direto no objeto lpp}
\NormalTok{X\_jit }\OtherTok{\textless{}{-}} \FunctionTok{rjitter}\NormalTok{(chicago\_clean, }\AttributeTok{radius=}\DecValTok{30}\NormalTok{)}

\CommentTok{\# Visualização Comparativa}
\FunctionTok{par}\NormalTok{(}\AttributeTok{mfrow=}\FunctionTok{c}\NormalTok{(}\DecValTok{1}\NormalTok{,}\DecValTok{3}\NormalTok{))}
\FunctionTok{plot}\NormalTok{(X\_sim\_fit, }\AttributeTok{main=}\StringTok{"Simulação Modelo Poisson"}\NormalTok{, }\AttributeTok{cols=}\StringTok{"red"}\NormalTok{, }\AttributeTok{cex=}\FloatTok{0.5}\NormalTok{, }\AttributeTok{pch=}\DecValTok{16}\NormalTok{)}
\FunctionTok{plot}\NormalTok{(X\_thomas, }\AttributeTok{main=}\StringTok{"Cluster Thomas (Projetado)"}\NormalTok{, }\AttributeTok{cols=}\StringTok{"blue"}\NormalTok{, }\AttributeTok{cex=}\FloatTok{0.5}\NormalTok{, }\AttributeTok{pch=}\DecValTok{16}\NormalTok{)}
\FunctionTok{plot}\NormalTok{(X\_jit, }\AttributeTok{main=}\StringTok{"Chicago com Jitter"}\NormalTok{, }\AttributeTok{cols=}\StringTok{"green"}\NormalTok{, }\AttributeTok{cex=}\FloatTok{0.5}\NormalTok{, }\AttributeTok{pch=}\StringTok{"."}\NormalTok{)}
\end{Highlighting}
\end{Shaded}

\pandocbounded{\includegraphics[keepaspectratio]{point_process_files/figure-pdf/unnamed-chunk-10-6.pdf}}

\section{Integração da Geoestatística, GAMs e processos
pontuais}\label{integrauxe7uxe3o-da-geoestatuxedstica-gams-e-processos-pontuais}

\begin{enumerate}
\def\labelenumi{\arabic{enumi}.}
\item
  Geoestatística (\texttt{automap,\ gstat,\ geoR}): Frequentemente, as
  covariáveis (ex: pH do solo, temperatura) são medidas apenas em locais
  amostrais. Utilizamos a krigagem para gerar mapas contínuos (imagens
  raster) dessas variáveis em toda a janela de observação, conforme
  discutido no Capítulo~\ref{sec-geoest}.
\item
  Processos Pontuais (\texttt{spatstat}): Os mapas krigados são
  convertidos para objetos de imagem (im) e utilizados como preditores
  espaciais para o padrão de pontos (\texttt{ppp}).
\item
  Modelagem (mgcv): Em vez de assumir relações lineares simples,
  ajustamos a intensidade \(\lambda(u)\) usando \texttt{splines} de
  suavização (termos \texttt{s()}) para capturar respostas ecológicas
  complexas e não-lineares. O \texttt{spatstat} permite ajustar esses
  modelos através da função \texttt{ppm} com a opção `use.gam=TRUE```.
\item
  Visualização (\texttt{gratia}): Extraímos o objeto \texttt{GAM}
  subjacente e utilizamos o pacote \texttt{gratia} para visualizar as
  funções parciais suaves e seus intervalos de confiança, substituindo
  os gráficos padrão do R base.
\end{enumerate}

\begin{Shaded}
\begin{Highlighting}[]
\ControlFlowTok{if}\NormalTok{ (}\SpecialCharTok{!}\FunctionTok{require}\NormalTok{(}\StringTok{"pacman"}\NormalTok{)) }\FunctionTok{install.packages}\NormalTok{(}\StringTok{"pacman"}\NormalTok{)}
\NormalTok{pacman}\SpecialCharTok{::}\FunctionTok{p\_load}\NormalTok{(spatstat, gstat, automap, sf, stars, mgcv, gratia, ggplot2, sp)}

\CommentTok{\#CARREGAMENTO DOS DADOS (Amostragem de Solo)}
\FunctionTok{data}\NormalTok{(meuse)      }\CommentTok{\# Locais de coleta de solo (Rikken \& Van Rijn, 1993)}
\FunctionTok{data}\NormalTok{(meuse.grid) }\CommentTok{\# Grade de referência da planície de inundação}

\CommentTok{\# Pontos de Coleta para SF}
\NormalTok{pontos\_sf }\OtherTok{\textless{}{-}} \FunctionTok{st\_as\_sf}\NormalTok{(meuse, }\AttributeTok{coords =} \FunctionTok{c}\NormalTok{(}\StringTok{"x"}\NormalTok{, }\StringTok{"y"}\NormalTok{), }\AttributeTok{crs =} \DecValTok{28992}\NormalTok{)}
\NormalTok{grid\_sf   }\OtherTok{\textless{}{-}} \FunctionTok{st\_as\_sf}\NormalTok{(meuse.grid, }\AttributeTok{coords =} \FunctionTok{c}\NormalTok{(}\StringTok{"x"}\NormalTok{, }\StringTok{"y"}\NormalTok{), }\AttributeTok{crs =} \DecValTok{28992}\NormalTok{)}

\CommentTok{\# Converter para Spatial (Necessário para o automap)}
\NormalTok{input\_sp }\OtherTok{\textless{}{-}} \FunctionTok{as}\NormalTok{(pontos\_sf, }\StringTok{"Spatial"}\NormalTok{)}
\NormalTok{grid\_sp  }\OtherTok{\textless{}{-}} \FunctionTok{as}\NormalTok{(grid\_sf, }\StringTok{"Spatial"}\NormalTok{)}
\FunctionTok{gridded}\NormalTok{(grid\_sp) }\OtherTok{\textless{}{-}} \ConstantTok{TRUE} 

\CommentTok{\#PREPARAÇÃO DA COVARIÁVEL (KRIGAGEM)}
\CommentTok{\# Interpolação da concentração de Zinco para toda a área}

\NormalTok{krigagem }\OtherTok{\textless{}{-}} \FunctionTok{autoKrige}\NormalTok{(}\FunctionTok{log}\NormalTok{(zinc) }\SpecialCharTok{\textasciitilde{}} \DecValTok{1}\NormalTok{, }\AttributeTok{input\_data =}\NormalTok{ input\_sp, }\AttributeTok{new\_data =}\NormalTok{ grid\_sp, }\AttributeTok{debug.level =} \DecValTok{0}\NormalTok{)}

\CommentTok{\# Conversão para Imagem (Covariável preditora)}

\NormalTok{mapa\_zinco\_im }\OtherTok{\textless{}{-}} \FunctionTok{as.im}\NormalTok{(}\FunctionTok{st\_as\_stars}\NormalTok{(krigagem}\SpecialCharTok{$}\NormalTok{krige\_output)[}\StringTok{"var1.pred"}\NormalTok{])}

\CommentTok{\# CONFIGURAÇÃO DO PADRÃO DE PONTOS (PPM)}
\NormalTok{janela }\OtherTok{\textless{}{-}} \FunctionTok{Window}\NormalTok{(mapa\_zinco\_im)}
\NormalTok{coords }\OtherTok{\textless{}{-}} \FunctionTok{st\_coordinates}\NormalTok{(pontos\_sf)}

\CommentTok{\# Criação do objeto PPP representando os LOCAIS DE AMOSTRAGEM}
\NormalTok{pontos\_observados }\OtherTok{\textless{}{-}} \FunctionTok{ppp}\NormalTok{(}\AttributeTok{x =}\NormalTok{ coords[,}\DecValTok{1}\NormalTok{], }
                         \AttributeTok{y =}\NormalTok{ coords[,}\DecValTok{2}\NormalTok{], }
                         \AttributeTok{window =}\NormalTok{ janela, }
                         \AttributeTok{checkdup =} \ConstantTok{TRUE}\NormalTok{)}

\FunctionTok{par}\NormalTok{(}\AttributeTok{mfrow =} \FunctionTok{c}\NormalTok{(}\DecValTok{1}\NormalTok{, }\DecValTok{2}\NormalTok{), }\AttributeTok{mar=}\FunctionTok{c}\NormalTok{(}\DecValTok{1}\NormalTok{,}\DecValTok{1}\NormalTok{,}\DecValTok{3}\NormalTok{,}\DecValTok{1}\NormalTok{))}
\FunctionTok{plot}\NormalTok{(mapa\_zinco\_im, }\AttributeTok{main =} \StringTok{"Log(Zinco) no Solo"}\NormalTok{, }\AttributeTok{box =} \ConstantTok{FALSE}\NormalTok{)}

\FunctionTok{plot}\NormalTok{(pontos\_observados, }\AttributeTok{main =} \StringTok{"Locais de}\SpecialCharTok{\textbackslash{}n}\StringTok{Coleta (Meuse)"}\NormalTok{, }\AttributeTok{pch =} \DecValTok{16}\NormalTok{, }\AttributeTok{cex =} \FloatTok{0.5}\NormalTok{, }\AttributeTok{cols =} \StringTok{"black"}\NormalTok{, }\AttributeTok{col =} \FunctionTok{c}\NormalTok{(}\ConstantTok{NA}\NormalTok{, }\ConstantTok{NA}\NormalTok{), }
     \AttributeTok{box =} \ConstantTok{FALSE}\NormalTok{)}
\FunctionTok{par}\NormalTok{(}\AttributeTok{mfrow =} \FunctionTok{c}\NormalTok{(}\DecValTok{1}\NormalTok{, }\DecValTok{1}\NormalTok{))}

\CommentTok{\#MODELAGEM (Investigação do Viés de Coleta)}
\CommentTok{\# ?Pergunta: A escolha dos locais de coleta foi influenciada pelo nível de poluição?}
\CommentTok{\# Modelo: Intensidade\_de\_Amostragem \textasciitilde{} s(Zinco)}

\NormalTok{fit\_real }\OtherTok{\textless{}{-}} \FunctionTok{ppm}\NormalTok{(pontos\_observados }\SpecialCharTok{\textasciitilde{}} \FunctionTok{s}\NormalTok{(zinco), }
                \AttributeTok{covariates =} \FunctionTok{list}\NormalTok{(}\AttributeTok{zinco =}\NormalTok{ mapa\_zinco\_im), }
                \AttributeTok{use.gam =} \ConstantTok{TRUE}\NormalTok{) }

\CommentTok{\#RESULTADOS E INTERPRETAÇÃO {-}{-}{-}}

\NormalTok{modelo\_gam }\OtherTok{\textless{}{-}}\NormalTok{ fit\_real}\SpecialCharTok{$}\NormalTok{internal}\SpecialCharTok{$}\NormalTok{glmfit}

\FunctionTok{draw}\NormalTok{(modelo\_gam) }\SpecialCharTok{+}
\NormalTok{  ggplot2}\SpecialCharTok{::}\FunctionTok{labs}\NormalTok{(}
    \AttributeTok{title =} \StringTok{"Resposta da densidade de amostras ao gradiente de poluição"}\NormalTok{,}
    \AttributeTok{y =} \StringTok{"Efeito na Densidade de Coleta (s(zinco))"}\NormalTok{,}
    \AttributeTok{x =} \StringTok{"Log(Concentração de Zinco) [ppm]"}
\NormalTok{  ) }\SpecialCharTok{+}
\NormalTok{  ggplot2}\SpecialCharTok{::}\FunctionTok{theme\_bw}\NormalTok{() }\SpecialCharTok{+}
\NormalTok{  ggplot2}\SpecialCharTok{::}\FunctionTok{theme}\NormalTok{(}\AttributeTok{plot.title =} \FunctionTok{element\_text}\NormalTok{(}\AttributeTok{face =} \StringTok{"bold"}\NormalTok{))}
\CommentTok{\#Intensidade Predita (Onde o modelo "espera" que haja amostras)}
\NormalTok{mapa\_predito }\OtherTok{\textless{}{-}} \FunctionTok{predict}\NormalTok{(fit\_real, }\AttributeTok{type =} \StringTok{"trend"}\NormalTok{)}

\FunctionTok{plot}\NormalTok{(mapa\_predito, }\AttributeTok{main =} \StringTok{"Densidade de Amostragem Predita (PPM)"}\NormalTok{)}
\FunctionTok{plot}\NormalTok{(pontos\_observados, }\AttributeTok{add =} \ConstantTok{TRUE}\NormalTok{, }\AttributeTok{pch =} \StringTok{"."}\NormalTok{, }\AttributeTok{cols =} \StringTok{"white"}\NormalTok{)}
\CommentTok{\#Diagnóstico de Resíduos}
\NormalTok{residuos }\OtherTok{\textless{}{-}} \FunctionTok{residuals}\NormalTok{(fit\_real, }\AttributeTok{type =} \StringTok{"pearson"}\NormalTok{)}
\FunctionTok{plot}\NormalTok{(}\FunctionTok{Smooth}\NormalTok{(residuos, }\AttributeTok{sigma =} \DecValTok{50}\NormalTok{), }
     \AttributeTok{main =} \StringTok{"Resíduos Espaciais }\SpecialCharTok{\textbackslash{}n}\StringTok{(Vermelho = Amostragem }\SpecialCharTok{\textbackslash{}n}\StringTok{mais densa que o previsto)"}\NormalTok{)}
\end{Highlighting}
\end{Shaded}

\begin{figure}

\begin{minipage}{0.50\linewidth}

\begin{figure}[H]

{\centering \pandocbounded{\includegraphics[keepaspectratio]{point_process_files/figure-pdf/unnamed-chunk-11-1.pdf}}

}

\subcaption{Integração da Geoestatística, GAMs e processos pontuais}

\end{figure}%

\end{minipage}%
%
\begin{minipage}{0.50\linewidth}

\begin{figure}[H]

{\centering \pandocbounded{\includegraphics[keepaspectratio]{point_process_files/figure-pdf/unnamed-chunk-11-2.pdf}}

}

\subcaption{Integração da Geoestatística, GAMs e processos pontuais}

\end{figure}%

\end{minipage}%
\newline
\begin{minipage}{0.50\linewidth}

\begin{figure}[H]

{\centering \pandocbounded{\includegraphics[keepaspectratio]{point_process_files/figure-pdf/unnamed-chunk-11-3.pdf}}

}

\subcaption{Integração da Geoestatística, GAMs e processos pontuais}

\end{figure}%

\end{minipage}%
%
\begin{minipage}{0.50\linewidth}

\begin{figure}[H]

{\centering \pandocbounded{\includegraphics[keepaspectratio]{point_process_files/figure-pdf/unnamed-chunk-11-4.pdf}}

}

\subcaption{Integração da Geoestatística, GAMs e processos pontuais}

\end{figure}%

\end{minipage}%

\end{figure}%

O gráfico ilustra a relação direta e não linear entre o esforço de
amostragem realizado pelos pesquisadores e o gradiente de poluição por
zinco, onde o eixo horizontal apresenta o logaritmo da concentração do
metal em ppm e o eixo vertical indica o efeito parcial dessa variável na
densidade de coleta. A linha preta sólida central descreve o
comportamento médio dessa relação, revelando uma curva sigmoide que
permanece relativamente estável e próxima de zero em áreas de baixa a
média contaminação, mas que apresenta uma inclinação ascendente
acentuada a partir do valor 6.0, evidenciando que a estratégia de campo
privilegiou uma coleta muito mais intensiva nas zonas mais poluídas.

Envolvendo a curva de tendência, a faixa sombreada em cinza representa o
intervalo de confiança de 95\%, cuja variação de largura serve como um
indicador visual da incerteza estatística do modelo; nota-se que essa
faixa é estreita na região central do gráfico, denotando alta precisão,
mas se expande significativamente na extremidade direita (valores acima
de 7.0), refletindo a escassez de dados nessas condições extremas. Essa
distribuição dos dados é confirmada pelas pequenas barras verticais
pretas situadas na base da figura (rug plot), que aparecem densamente
agrupadas nos valores intermediários de zinco, mas tornam-se esparsas
nas concentrações mais elevadas, o que explica matematicamente o aumento
da incerteza e confirma visualmente o viés intencional do desenho
experimental voltado para as áreas de maior interesse ambiental.

\begin{itemize}
\item
  \texttt{as.im():} Função crucial de ``ponte''. Ela converte os
  resultados espaciais de outros pacotes (como o output do automap ou
  sf) para o formato de imagem de pixel (im) que o spatstat exige para
  covariáveis.
\item
  \texttt{ppm(...,\ use.gam=TRUE):} Esta é a função de ajuste de modelos
  de processos pontuais (ppm). Ao ativar use.gam=TRUE, instruímos o
  spatstat a não usar o algoritmo padrão de regressão linear (glm), mas
  sim invocar o mgcv::gam para ajustar os coeficientes da intensidade.
  Isso permite o uso de termos s() (splines penalizados) na fórmula,
  capturando tendências complexas que uma função linear ou quadrática
  simples perderia.
\item
  \texttt{gratia::draw():} Uma função moderna projetada especificamente
  para objetos gam. Ela produz diagnósticos visuais e gráficos de
  efeitos parciais (mostrando como a intensidade muda em função da
  covariável, mantendo o resto constante) prontos para publicação. Como
  o spatstat armazena o objeto do modelo ajustado internamente, a
  extração (fit\(internal\)glmfit) permite conectar esses dois mundos.
\end{itemize}

\part{Espaço-Temporal}

\chapter{Modelagem Espaço-Temporal}\label{sec-espaco_temporais}

Até aqui, focamos em dados que variam apenas no espaço
\(\{Y(s), s \subset D\}\). No entanto, é comum que os fenômenos variem
tanto no espaço quanto no tempo. Ou seja, observamos o mesmo processo
espacial em múltiplos instantes temporais, e esse efeito temporal é
importante para a análise.

A única mudança fundamental é a adição de uma segunda dimensão: em vez
de considerarmos apenas \(\{Y(s), s \subset D\}\), agora trabalhamos com
\(\{\mathbf{Y(s, t)}, \mathbf{s} \subset D; \mathbf{t} \subset T\}\).

Trata-se de uma extensão natural dos modelos puramente espaciais.
Portanto, se você compreendeu bem os conteúdos anteriores,
geoestatística (Capítulo~\ref{sec-geoest}), dados de área
(Capítulo~\ref{sec-dados_area}) e processos pontuais
(Capítulo~\ref{sec-proc_pont}) será capaz de entender facilmente a
extensão espaço-temporal. Para auxiliar nesse passo, apresentamos abaixo
uma seleção de referências e pacotes do \texttt{R} para aplicação
prática. A lista não é exaustiva, mas oferece um ponto de partida
sólido.

\section{Referências Gerais}\label{referuxeancias-gerais}

\begin{enumerate}
\def\labelenumi{\arabic{enumi}.}
\item
  Pebesma, E.\texttt{spacetime}: Spatio-temporal data in R.
  \emph{Journal of Statistical Software}, 51, 2012, pp.~1--30.\\
\item
  Bakar, K. S. \& Sahu, S. K.\texttt{spTimer}: Spatio-temporal Bayesian
  modeling using R. \emph{Journal of Statistical Software}, 63, 2015,
  pp.~1--32.
\item
  Wikle, C. K., Zammit-Mangion, A. \& Cressie, N. \emph{Spatio-temporal
  statistics with R}. Chapman and Hall/CRC, 2019. Disponível
  gratuitamente em:
  \href{https://spacetimewithr.org/}{spacetimewithr.org}.
\item
  Cressie, N. \& Wikle, C. K.\emph{Statistics for spatio-temporal data}.
  John Wiley \& Sons, 2011.
\end{enumerate}

\section{Modelagem Espaço-Temporal em Dados
Geoestatísticos}\label{modelagem-espauxe7o-temporal-em-dados-geoestatuxedsticos}

\begin{enumerate}
\def\labelenumi{\arabic{enumi}.}
\item
  Gräler, B., Pebesma, E. \& Heuvelink, G. Spatio-temporal interpolation
  using \texttt{gstat}, 2016.
\item
  Sahu, S. \emph{Bayesian modeling of spatio-temporal data with R}.
  Chapman and Hall/CRC, 2022.
\end{enumerate}

\section{Modelagem Espaço-Temporal em Processos
Pontuais}\label{modelagem-espauxe7o-temporal-em-processos-pontuais}

\begin{enumerate}
\def\labelenumi{\arabic{enumi}.}
\item
  González, J. A. et al.~Spatio-temporal point process statistics: a
  review. \emph{Spatial Statistics}, 18, 2016, pp.~505--544.
\item
  Diggle, P. J.\emph{Statistical analysis of spatial and spatio-temporal
  point patterns}. CRC Press, 2013.
\item
  D'Angelo, N. \& Adelfio, G. \texttt{stopp}: An R package for
  spatio-temporal point pattern analysis. \emph{Journal of Statistical
  Software}, 113, 2025, pp.~1--35.
\item
  Gabriel, E. et al.\texttt{stpp}: an R package for plotting, simulating
  and analyzing Spatio-Temporal Point Patterns. \emph{Journal of
  Statistical Software}, 53, 2013, pp.~1--29.
\end{enumerate}

\section{Modelagem Espaço-Temporal de Dados de
Área}\label{modelagem-espauxe7o-temporal-de-dados-de-uxe1rea}

\begin{enumerate}
\def\labelenumi{\arabic{enumi}.}
\tightlist
\item
  Blangiardo, M. \& Cameletti, M. \emph{Spatial and spatio-temporal
  Bayesian models with R-INLA}. John Wiley \& Sons, 2015.
\end{enumerate}

\section{Modelagem local}\label{modelagem-local}

\begin{itemize}
\tightlist
\item
  Wu, C., Ren, F., Hu, W. and Du, Q.\emph{Multiscale geographically and
  temporally weighted regression: Exploring the spatiotemporal
  determinants of housing prices.} International Journal of Geographical
  Information Science, 33(3), 2019, pp.489-511.
\end{itemize}

\bookmarksetup{startatroot}

\chapter*{Referências
Bibliográficas}\label{referuxeancias-bibliogruxe1ficas}
\addcontentsline{toc}{chapter}{Referências Bibliográficas}

\markboth{Referências Bibliográficas}{Referências Bibliográficas}

\phantomsection\label{refs}
\begin{CSLReferences}{1}{0}
\bibitem[\citeproctext]{ref-moragagonz2023non}
A González, Jonatan, e Paula Moraga. 2023. {``Non-Parametric Analysis of
Spatial and Spatio-Temporal Point Patterns''}.

\bibitem[\citeproctext]{ref-abdulah2023large}
Abdulah, Sameh, Yuxiao Li, Jian Cao, Hatem Ltaief, David E Keyes, Marc G
Genton, e Ying Sun. 2023. {``Large-scale environmental data science with
ExaGeoStatR''}. \emph{Environmetrics} 34 (1): e2770.

\bibitem[\citeproctext]{ref-anselin1988spatial}
Anselin, Luc. 1988. {``Spatial econometrics: methods and models''}.
\emph{Kluwer Academic Publishers google schola} 2: 283--91.

\bibitem[\citeproctext]{ref-anselin1995local}
---------. 1995. {``Local indicators of spatial association---LISA''}.
\emph{Geographical analysis} 27 (2): 93--115.

\bibitem[\citeproctext]{ref-anselin2001spatial}
---------. 2001. {``Spatial Econometrics''}. Em \emph{A Companion to
Theoretical Econometrics}, editado por Badi H. Baltagi, 310--30. Oxford:
Blackwell Publishing.

\bibitem[\citeproctext]{ref-anselin2002under}
---------. 2002. {``Under the hood issues in the specification and
interpretation of spatial regression models''}. \emph{Agricultural
economics} 27 (3): 247--67.

\bibitem[\citeproctext]{ref-anselin2010thirty}
---------. 2010. {``Thirty years of spatial econometrics''}.
\emph{Papers in regional science} 89 (1): 3--26.

\bibitem[\citeproctext]{ref-baddeley2007validation}
Baddeley, Adrian. 2007. {``Validation of statistical models for spatial
point patterns''}. Em \emph{Statistical Challenges in Modern Astronomy
IV}, 371:22.

\bibitem[\citeproctext]{ref-baddeley2008analysing}
Baddeley, Adrian et al. 2008. {``Analysing spatial point patterns in
R''}. Technical report, CSIRO, 2010. Version 4. Available at www. csiro.
au~\ldots.

\bibitem[\citeproctext]{ref-baddeley2010multivariate}
Baddeley, Adrian. 2010. {``Multivariate and marked point processes''}.
\emph{Handbook of spatial statistics}, 371--402.

\bibitem[\citeproctext]{ref-baddeley2000non}
Baddeley, Adrian J, Jesper Møller, e Rasmus Waagepetersen. 2000.
{``Non-and semi-parametric estimation of interaction in inhomogeneous
point patterns''}. \emph{Statistica Neerlandica} 54 (3): 329--50.

\bibitem[\citeproctext]{ref-baddeley1995area}
Baddeley, Adrian J, e MNM Van Lieshout. 1995. {``Area-interaction point
processes''}. \emph{Annals of the Institute of Statistical Mathematics}
47: 601--19.

\bibitem[\citeproctext]{ref-baddeley1996markov}
Baddeley, Adrian J, MNM Van Lieshout, e Jesper Møller. 1996. {``Markov
properties of cluster processes''}. \emph{Advances in Applied
Probability} 28 (2): 346--55.

\bibitem[\citeproctext]{ref-baddeley2013residual}
Baddeley, Adrian, Ya-Mei Chang, Yong Song, e Rolf Turner. 2013.
{``Residual diagnostics for covariate effects in spatial point process
models''}. \emph{Journal of Computational and Graphical Statistics} 22
(4): 886--905.

\bibitem[\citeproctext]{ref-baddeley2022fundamental}
Baddeley, Adrian, Tilman M Davies, Martin L Hazelton, Suman Rakshit, e
Rolf Turner. 2022. {``Fundamental problems in fitting spatial cluster
process models''}. \emph{Spatial Statistics} 52: 100709.

\bibitem[\citeproctext]{ref-baddeley2015}
Baddeley, Adrian, Ege Rubak, e Rolf Turner. 2015. \emph{Spatial Point
Patterns: Methodology and Applications with {R}}. London: Chapman;
Hall/CRC Press.
\url{https://www.routledge.com/Spatial-Point-Patterns-Methodology-and-Applications-with-R/Baddeley-Rubak-Turner/p/book/9781482210200/}.

\bibitem[\citeproctext]{ref-spatstat}
Baddeley, Adrian, e Rolf Turner. 2005. {``{spatstat}: An {R} Package for
Analyzing Spatial Point Patterns''}. \emph{Journal of Statistical
Software} 12 (6): 1--42. \url{https://doi.org/10.18637/jss.v012.i06}.

\bibitem[\citeproctext]{ref-Hybrids}
Baddeley, Adrian, Rolf Turner, Jorge Mateu, e Andrew Bevan. 2013.
{``Hybrids of Gibbs Point Process Models and Their Implementation''}.
\emph{Journal of Statistical Software} 55 (11): 1--43.
\url{https://doi.org/10.18637/jss.v055.i11}.

\bibitem[\citeproctext]{ref-baddeley2005residual}
Baddeley, Adrian, Rolf Turner, Jesper Møller, e Martin Hazelton. 2005.
{``Residual analysis for spatial point processes (with discussion)''}.
\emph{Journal of the Royal Statistical Society Series B: Statistical
Methodology} 67 (5): 617--66.

\bibitem[\citeproctext]{ref-baddeley2019residuals}
Baddeley, AJ, M Hazelton, J Møller, e R Turner. 2019. {``Residuals and
diagnostics for spatial point processes''}. Em. Citeseer.

\bibitem[\citeproctext]{ref-bakka2018spatial}
Bakka, Haakon, Håvard Rue, Geir-Arne Fuglstad, Andrea Riebler, David
Bolin, Janine Illian, Elias Krainski, Daniel Simpson, e Finn Lindgren.
2018. {``Spatial modeling with R-INLA: A review''}. \emph{Wiley
Interdisciplinary Reviews: Computational Statistics} 10 (6): e1443.

\bibitem[\citeproctext]{ref-bandyopadhyay2017test}
Bandyopadhyay, Soutir, e Suhasini Subba Rao. 2017. {``A test for
stationarity for irregularly spaced spatial data''}. \emph{Journal of
the Royal Statistical Society Series B: Statistical Methodology} 79 (1):
95--123.

\bibitem[\citeproctext]{ref-banerjee2016spatial}
Banerjee, Sudipto. 2016. {``Spatial data analysis''}. \emph{Annual
review of public health} 37 (1): 47--60.

\bibitem[\citeproctext]{ref-banerjee2003hierarchical}
Banerjee, Sudipto, Bradley P Carlin, e Alan E Gelfand. 2003.
\emph{Hierarchical modeling and analysis for spatial data}. Chapman;
Hall/CRC.

\bibitem[\citeproctext]{ref-bartholomew1995statistics}
Bartholomew, David J. 1995. {``What is statistics?''} \emph{Journal of
the Royal Statistical Society Series A: Statistics in Society} 158 (1):
1--20.

\bibitem[\citeproctext]{ref-bell2004special}
Bell, William Wallace. 2004. \emph{Special functions for scientists and
engineers}. Courier Corporation.

\bibitem[\citeproctext]{ref-beron2004probit}
Beron, Kurt J, e Wim PM Vijverberg. 2004. {``Probit in a spatial
context: a Monte Carlo analysis''}. Em \emph{Advances in spatial
econometrics: methodology, tools and applications}, 169--95. Springer.

\bibitem[\citeproctext]{ref-besag1977discussion}
Besag, J. 1977. {``Discussion of Dr Ripley's paper''}. \emph{Journal of
the Royal Statistical Society, Series B} 39 (2): 193--95.

\bibitem[\citeproctext]{ref-besag1974spatial}
Besag, Julian. 1974. {``Spatial interaction and the statistical analysis
of lattice systems''}. \emph{Journal of the Royal Statistical Society:
Series B (Methodological)} 36 (2): 192--225.

\bibitem[\citeproctext]{ref-besag1975statistical}
---------. 1975. {``Statistical analysis of non-lattice data''}.
\emph{Journal of the Royal Statistical Society Series D: The
Statistician} 24 (3): 179--95.

\bibitem[\citeproctext]{ref-besag1993spatial}
Besag, Julian, e Peter J Green. 1993. {``Spatial statistics and Bayesian
computation''}. \emph{Journal of the Royal Statistical Society Series B:
Statistical Methodology} 55 (1): 25--37.

\bibitem[\citeproctext]{ref-BesagKooperberg1995}
Besag, Julian, e Charles Kooperberg. 1995. {``{On Conditional and
Intrinsic Autoregression}''}. \emph{Biometrika} 82 (4): 733--46.
\url{https://doi.org/10.2307/2337341}.

\bibitem[\citeproctext]{ref-besag1991bayesian}
Besag, Julian, Jeremy York, e Annie Mollié. 1991. {``Bayesian image
restoration, with two applications in spatial statistics''}.
\emph{Annals of the institute of statistical mathematics} 43 (1): 1--20.

\bibitem[\citeproctext]{ref-bille2019spatial}
Billé, Anna Gloria, e Giuseppe Arbia. 2019. {``Spatial limited dependent
variable models: A review focused on specification, estimation, and
health economics applications''}. \emph{Journal of Economic Surveys} 33
(5): 1531--54.

\bibitem[\citeproctext]{ref-bille2020partial}
Billé, Anna Gloria, e Samantha Leorato. 2020. {``Partial ML estimation
for spatial autoregressive nonlinear probit models with autoregressive
disturbances''}. \emph{Econometric Reviews} 39 (5): 437--75.

\bibitem[\citeproctext]{ref-boschma2005proximity}
Boschma, Ron. 2005. {``Proximity and innovation: a critical
assessment''}. \emph{Regional studies} 39 (1): 61--74.

\bibitem[\citeproctext]{ref-brunsdon1996geographically}
Brunsdon, Chris, A Stewart Fotheringham, e Martin E Charlton. 1996.
{``Geographically weighted regression: a method for exploring spatial
nonstationarity''}. \emph{Geographical analysis} 28 (4): 281--98.

\bibitem[\citeproctext]{ref-burbano2024spatial}
Burbano-Moreno, Alvaro Alexander, e Vinı́cius Diniz Mayrink. 2024.
{``Spatial functional data analysis: Irregular spacing and bernstein
polynomials''}. \emph{Spatial Statistics} 60: 100832.

\bibitem[\citeproctext]{ref-burridge1980cliff}
Burridge, Peter. 1980. {``On the Cliff-Ord test for spatial
correlation''}. \emph{Journal of the Royal Statistical Society: Series B
(Methodological)} 42 (1): 107--8.

\bibitem[\citeproctext]{ref-burridge1981testing}
---------. 1981. {``Testing for a common factor in a spatial
autoregression model''}. \emph{Environment and planning A} 13 (7):
795--800.

\bibitem[\citeproctext]{ref-calabrese2014estimators}
Calabrese, Raffaella, e Johan A Elkink. 2014. {``Estimators of binary
spatial autoregressive models: A Monte Carlo study''}. \emph{Journal of
Regional Science} 54 (4): 664--87.

\bibitem[\citeproctext]{ref-carvalho2017overview}
Carvalho, Dhaniel, e CV Deutsch. 2017. {``An overview of multiple
indicator kriging''}. \emph{Geostatistics Lessons} 7.

\bibitem[\citeproctext]{ref-berger2001statistical}
Casella, George, e Roger L Berger. 2001. \emph{Statistical inference}.
Duxbury.

\bibitem[\citeproctext]{ref-chen2021space}
Chen, Wanfang, Marc G Genton, e Ying Sun. 2021. {``Space-time covariance
structures and models''}. \emph{Annual Review of Statistics and Its
Application} 8 (1): 191--215.

\bibitem[\citeproctext]{ref-leaflet}
Cheng, Joe, Barret Schloerke, Bhaskar Karambelkar, Yihui Xie, e Garrick
Aden-Buie. 2025. \emph{leaflet: Create Interactive Web Maps with the
JavaScript 'Leaflet' Library}.
\url{https://doi.org/10.32614/CRAN.package.leaflet}.

\bibitem[\citeproctext]{ref-chiles2012geostatistics}
Chiles, Jean-Paul, e Pierre Delfiner. 2012. \emph{Geostatistics:
modeling spatial uncertainty}. John Wiley \& Sons.

\bibitem[\citeproctext]{ref-chun2017}
Chun, Yongwan, e Daniel A. Griffith. 2017. {``Measuring Spatial
Dependence''}. \emph{International Encyclopedia of Geography}, março,
1--14. \url{https://doi.org/10.1002/9781118786352.wbieg0850}.

\bibitem[\citeproctext]{ref-clark1954distance}
Clark, Philip J, e Francis C Evans. 1954. {``Distance to nearest
neighbor as a measure of spatial relationships in populations''}.
\emph{Ecology} 35 (4): 445--53.

\bibitem[\citeproctext]{ref-cliff1981spatial}
Cliff, Andrew David, e J Keith Ord. 1981. {``Spatial processes: models
\& applications''}. \emph{(No Title)}.
https://doi.org/\url{https://doi.org/10.2307/2530324}.

\bibitem[\citeproctext]{ref-conte2023lecture02}
Conte, Bruno. 2023. {``Lecture 02: Spatial Data Theory and Tools (a.k.a.
GIS Tools Lab.)''}. Alma Mater Studiorum Università di Bologna; Lecture
slides.

\bibitem[\citeproctext]{ref-cox2011}
Cox, Simon Jonathan David. 2011. \emph{ISO 19156:2011 - Geographic
Information -- Observations and Measurements}. International
Organization for Standardization.
\url{https://doi.org/10.13140/2.1.1142.3042}.

\bibitem[\citeproctext]{ref-crawford2009}
Crawford, T. W. 2009. {``Scale Analytical''}. Em, 29--36. Elsevier.
\url{https://doi.org/10.1016/b978-008044910-4.00399-0}.

\bibitem[\citeproctext]{ref-cressie1993}
Cressie, N. 1993. \emph{Statistics for \text{S}patial \text{D}ata:
\text{W}iley \text{S}eries in \text{P}robability and \text{S}tatistics}.
Wiley-Interscience.

\bibitem[\citeproctext]{ref-cressie1985fitting}
Cressie, Noel. 1985. {``Fitting variogram models by weighted least
squares''}. \emph{Journal of the international Association for
mathematical Geology} 17 (5): 563--86.

\bibitem[\citeproctext]{ref-cressie1989geostatistics}
---------. 1989. {``Geostatistics''}. \emph{The American Statistician}
43 (4): 197--202.

\bibitem[\citeproctext]{ref-cressie1990origins}
---------. 1990. {``The origins of kriging''}. \emph{Mathematical
geology} 22 (3): 239--52.

\bibitem[\citeproctext]{ref-cressie1991geostatistical}
---------. 1991. {``Geostatistical analysis of spatial data''}.
\emph{Spatial statistics and digital image analysis} 1991: 87--108.

\bibitem[\citeproctext]{ref-cressie1993statistics}
---------. 1993. \emph{Statistics for spatial data}. John Wiley \& Sons.

\bibitem[\citeproctext]{ref-cressie1989spatial}
Cressie, Noel, e Ngai H Chan. 1989. {``Spatial modeling of regional
variables''}. \emph{Journal of the American Statistical Association} 84
(406): 393--401.

\bibitem[\citeproctext]{ref-cressie1980robust}
Cressie, Noel, e Douglas M Hawkins. 1980. {``Robust estimation of the
variogram: I''}. \emph{Journal of the international Association for
Mathematical Geology} 12 (2): 115--25.

\bibitem[\citeproctext]{ref-cressie2022spatial}
Cressie, Noel, e Matthew T Moores. 2022. {``Spatial statistics''}. Em
\emph{Encyclopedia of mathematical geosciences}, 1--11. Springer.

\bibitem[\citeproctext]{ref-cressie2022basis}
Cressie, Noel, Matthew Sainsbury-Dale, e Andrew Zammit-Mangion. 2022.
{``Basis-function models in spatial statistics''}. \emph{Annual Review
of Statistics and Its Application} 9 (1): 373--400.

\bibitem[\citeproctext]{ref-cressie2016multivariate}
Cressie, Noel, e Andrew Zammit-Mangion. 2016. {``Multivariate spatial
covariance models: a conditional approach''}. \emph{Biometrika} 103 (4):
915--35.

\bibitem[\citeproctext]{ref-cronie2016bandwidth}
Cronie, O, e MNM Van Lieshout. 2016. {``Bandwidth selection for kernel
estimators of the spatial intensity function''}. \emph{arXiv preprint
arXiv:1611.10221}.

\bibitem[\citeproctext]{ref-dasilva2016multiple}
da Silva, Anderson R., e A. Stewart Fotheringham. 2016. {``The multiple
testing issue in geographically weighted regression''}.
\emph{Geographical Analysis} 48 (3): 233--47.

\bibitem[\citeproctext]{ref-dasilva2018comparing}
da Silva, Anderson R., e Flávio F. Mendes. 2018. {``On comparing some
algorithms for finding the optimal bandwidth in geographically weighted
regression''}. \emph{Applied Soft Computing} 73: 943--57.

\bibitem[\citeproctext]{ref-delicado2010statistics}
Delicado, Pedro, Ramón Giraldo, Carlos Comas, e Jorge Mateu. 2010.
{``Statistics for spatial functional data: some recent contributions''}.
\emph{Environmetrics: The official journal of the International
Environmetrics Society} 21 (3-4): 224--39.

\bibitem[\citeproctext]{ref-DeutschJournel1997}
Deutsch, Clayton V., e André G. Journel. 1997. \emph{GSLIB:
Geostatistical Software Library and User's Guide}. 2º ed. New York:
Oxford University Press.

\bibitem[\citeproctext]{ref-diggle2010nonparametric}
Diggle, Peter J. 2010. \emph{Nonparametric methods}. Chapman \& Hall/CRC
Handb. Mod. Stat. Methods.

\bibitem[\citeproctext]{ref-diggle2013statistical}
---------. 2013. \emph{Statistical analysis of spatial and
spatio-temporal point patterns}. CRC press.

\bibitem[\citeproctext]{ref-diggle1987nonparametric}
Diggle, Peter J, David J Gates, e Alyson Stibbard. 1987. {``A
nonparametric estimator for pairwise-interaction point processes''}.
\emph{Biometrika} 74 (4): 763--70.

\bibitem[\citeproctext]{ref-diggle1984monte}
Diggle, Peter J, e Richard J Gratton. 1984. {``Monte Carlo methods of
inference for implicit statistical models''}. \emph{Journal of the Royal
Statistical Society Series B: Statistical Methodology} 46 (2): 193--212.

\bibitem[\citeproctext]{ref-Diggle2003StatisticalAO}
Diggle, Peter John. 2003. {``Statistical analysis of spatial point
patterns''}. Em, 2nd ed.

\bibitem[\citeproctext]{ref-diggle1998model}
Diggle, Peter J, Jonathan A Tawn, e Rana A Moyeed. 1998. {``Model-based
geostatistics''}. \emph{Journal of the Royal Statistical Society Series
C: Applied Statistics} 47 (3): 299--350.

\bibitem[\citeproctext]{ref-dreesman2001non}
Dreesman, Johannes M, e Gerhard Tutz. 2001. {``Non-Stationary
Conditional Models for Spatial Data Based on Varying Coefficients''}.
\emph{Journal of the Royal Statistical Society: Series D (The
Statistician)} 50 (1): 1--15.

\bibitem[\citeproctext]{ref-ggspatial}
Dunnington, Dewey. 2025. \emph{ggspatial: Spatial Data Framework for
ggplot2}. \url{https://doi.org/10.32614/CRAN.package.ggspatial}.

\bibitem[\citeproctext]{ref-ecker2003geostatistics}
Ecker, Mark D. 2003. {``Geostatistics: past, present and future''}.
\emph{Encyclopedia of Life Support Systems (EOLSS)}, 50614--506.

\bibitem[\citeproctext]{ref-elhorst2014spatial}
Elhorst, J Paul et al. 2014. \emph{Spatial econometrics: from
cross-sectional data to spatial panels}. Vol. 479. Springer.

\bibitem[\citeproctext]{ref-elhorst2022dynamic}
Elhorst, J Paul. 2022. {``The dynamic general nesting spatial
econometric model for spatial panels with common factors: Further
raising the bar''}. \emph{Review of Regional Research} 42 (3): 249--67.

\bibitem[\citeproctext]{ref-favero2003modelos}
Fávero, Luiz Paulo Lopes. 2003. {``Modelos de pre{ç}os hed{ô}nicos
aplicados a im{ó}veis residenciais em lan{ç}amento no munic{ı́}pio de
S{ã}o Paulo.''} Tese de doutorado, Universidade de S{ã}o Paulo.

\bibitem[\citeproctext]{ref-ferreira2020fundamentos}
Ferreira, Daniel Furtado. 2020. \emph{Fundamentos de probabilidade}.
Lavras: UFLA.

\bibitem[\citeproctext]{ref-fienberg2014statistics}
Fienberg, Stephen E. 2014. {``What is statistics?''} \emph{Annual review
of statistics and its application} 1 (1): 1--9.

\bibitem[\citeproctext]{ref-fleming2004techniques}
Fleming, Mark M. 2004. {``Techniques for estimating spatially dependent
discrete choice models''}. Em \emph{Advances in spatial econometrics:
methodology, tools and applications}, 145--68. Springer.

\bibitem[\citeproctext]{ref-fotheringham2017multiscale}
Fotheringham, A. Stewart, Wenbai Yang, e Wei Kang. 2017. {``Multiscale
Geographically Weighted Regression (MGWR)''}. \emph{Annals of the
American Association of Geographers} 107 (6): 1247--65.
\url{https://doi.org/10.1080/24694452.2017.1352480}.

\bibitem[\citeproctext]{ref-fotheringham2022notion}
Fotheringham, A. Stewart, Hanchen Yu, Levi John Wolf, Taylor M. Oshan, e
Ziqi Li. 2022. {``On the notion of {`bandwidth'} in geographically
weighted regression models of spatially varying processes''}.
\emph{International Journal of Geographical Information Science} 36 (8):
1485--1502. \url{https://doi.org/10.1080/13658816.2022.2034829}.

\bibitem[\citeproctext]{ref-Garcia-Portugues2024}
García-Portugués, E. 2024. \emph{Notes for Nonparametric Statistics}.
\url{https://bookdown.org/egarpor/NP-UC3M/}.

\bibitem[\citeproctext]{ref-gatrell1996spatial}
Gatrell, Anthony C, Trevor C Bailey, Peter J Diggle, e Barry S
Rowlingson. 1996. {``Spatial point pattern analysis and its application
in geographical epidemiology''}. \emph{Transactions of the Institute of
British geographers}, 256--74.

\bibitem[\citeproctext]{ref-gelfand2010handbook}
Gelfand, Alan E, Peter Diggle, Peter Guttorp, e Montserrat Fuentes.
2010. \emph{Handbook of spatial statistics}. CRC press.

\bibitem[\citeproctext]{ref-gelman2006prior}
Gelman, Andrew. 2006. {``Prior Distributions for Variance Parameters in
Hierarchical Models''}. \emph{Bayesian Analysis} 1 (3): 515--34.

\bibitem[\citeproctext]{ref-genton1998variogram}
Genton, Marc G. 1998. {``Variogram fitting by generalized least squares
using an explicit formula for the covariance structure''}.
\emph{Mathematical Geology} 30 (4): 323--45.

\bibitem[\citeproctext]{ref-getis1995cliff}
Getis, Arthur. 1995. {``Cliff, ad and ord, jk 1973: Spatial
autocorrelation. london: Pion''}. \emph{Progress in Human Geography} 19
(2): 245--49.

\bibitem[\citeproctext]{ref-getis1999spatial}
---------. 1999. {``Spatial statistics''}. \emph{Geographical
information systems} 1: 239--51.

\bibitem[\citeproctext]{ref-Getis2010SpatialAutocorrelation}
---------. 2010. {``Spatial Autocorrelation''}. Em \emph{Handbook of
Applied Spatial Analysis: Software Tools, Methods and Applications},
editado por Manfred M. Fischer e Arthur Getis, 255--78. Berlin,
Heidelberg: Springer.

\bibitem[\citeproctext]{ref-getis1992analysis}
Getis, Arthur, e J Keith Ord. 1992. {``The analysis of spatial
association by use of distance statistics''}. \emph{Geographical
analysis} 24 (3): 189--206.

\bibitem[\citeproctext]{ref-giraud2025mapsf}
Giraud, Timothée. 2025. {``mapsf: Thematic Cartography''}.

\bibitem[\citeproctext]{ref-gollini2015gwmodel}
Gollini, Isabella, Binbin Lu, Martin Charlton, Christopher Brunsdon, e
Paul Harris. 2015. {``GWmodel: an R package for exploring spatial
heterogeneity using geographically weighted models''}. \emph{Journal of
statistical software} 63: 1--50.

\bibitem[\citeproctext]{ref-goovaerts1997geostatistics}
Goovaerts, Pierre. 1997. \emph{Geostatistics for natural resources
evaluation}. Oxford university press.

\bibitem[\citeproctext]{ref-gorsich2000variogram}
Gorsich, David J, e Marc G Genton. 2000. {``Variogram model selection
via nonparametric derivative estimation''}. \emph{Mathematical geology}
32 (3): 249--70.

\bibitem[\citeproctext]{ref-graler2016spatio}
Gräler, Benedikt, Edzer Pebesma, e Gerard Heuvelink. 2016.
{``Spatio-temporal interpolation using gstat''}.

\bibitem[\citeproctext]{ref-Greene2003Econometric}
Greene, William H. 2003. \emph{Econometric Analysis}. 5º ed. Prentice
Hall.

\bibitem[\citeproctext]{ref-guimaraes2012use}
Guimaraes, Ricardo JPS, Corina C Freitas, Luciano V Dutra, Carlos A
Felgueiras, Sandra C Drummond, Sandra HC Tibiriçá, Guilherme Oliveira, e
Omar S Carvalho. 2012. {``Use of indicator kriging to investigate
schistosomiasis in minas gerais state, Brazil''}. \emph{Journal of
Tropical Medicine} 2012 (1): 837428.

\bibitem[\citeproctext]{ref-guo2008comparison}
Guo, L., Z. Ma, e L. Zhang. 2008. {``Comparison of bandwidth selection
in application of geographically weighted regression: a case study''}.
\emph{Canadian Journal of Forest Research} 38 (9): 2526--34.

\bibitem[\citeproctext]{ref-guttorp2006studies}
Guttorp, Peter, e Tilmann Gneiting. 2006. {``Studies in the history of
probability and statistics XLIX on the Mat{é}rn correlation family''}.
\emph{Biometrika} 93 (4): 989--95.

\bibitem[\citeproctext]{ref-haining1998exploratory}
Haining, Robert Haining, Stephen Wise, e Jingsheng Ma. 1998.
{``Exploratory spatial data analysis in a geographic information system
environment''}. \emph{Journal of the Royal Statistical Society Series D:
The Statistician} 47 (3): 457--69.

\bibitem[\citeproctext]{ref-haining2003spatial}
Haining, Robert P. 2003. \emph{Spatial data analysis: theory and
practice}. Cambridge university press.

\bibitem[\citeproctext]{ref-halleck2015slx}
Halleck Vega, Solmaria, e J Paul Elhorst. 2015. {``The SLX model''}.
\emph{Journal of Regional Science} 55 (3): 339--63.

\bibitem[\citeproctext]{ref-harris2011search}
Harris, Richard, John Moffat, e Victoria Kravtsova. 2011. {``In search
of {`W'}''}. \emph{Spatial Economic Analysis} 6 (3): 249--70.

\bibitem[\citeproctext]{ref-harville1977maximum}
Harville, David A. 1977. {``Maximum likelihood approaches to variance
component estimation and to related problems''}. \emph{Journal of the
American statistical association} 72 (358): 320--38.

\bibitem[\citeproctext]{ref-hastie1990generalized}
Hastie, Trevor, e Robert Tibshirani. 1990. \emph{Generalized Additive
Models}. Monographs on Statistics e Applied Probability. London:
Chapman; Hall.

\bibitem[\citeproctext]{ref-he2022multiscale}
He, Zhanjun, Zhe Wang, Zhiqiang Xie, Lin Wu, e Zhen Chen. 2022.
{``Multiscale analysis of the influence of street built environment on
crime occurrence using street-view images''}. \emph{Computers,
Environment and Urban Systems} 97: 101865.
\url{https://doi.org/10.1016/j.compenvurbsys.2022.101865}.

\bibitem[\citeproctext]{ref-held2010conditional}
Held, Leonhard, e Havard Rue. 2010. {``Conditional and intrinsic
autoregressions''}. \emph{Handbook of spatial statistics}, 201--16.

\bibitem[\citeproctext]{ref-hepple1979bayesian}
Hepple, Leslie W. 1979. {``Bayesian analysis of the linear model with
spatial dependence''}. Em \emph{Exploratory and explanatory statistical
analysis of spatial data}, 179--99. Springer.

\bibitem[\citeproctext]{ref-automap}
Hiemstra, P. H., E. J. Pebesma, C. J. W. Twenh"ofel, e G. B. M.
Heuvelink. 2008. {``Real-time automatic interpolation of ambient gamma
dose rates from the Dutch Radioactivity Monitoring Network''}.
\emph{Computers \& Geosciences}.

\bibitem[\citeproctext]{ref-geodata}
Hijmans, Robert J. 2025. \emph{geodata: Access Geographic Data}.
\url{https://doi.org/10.32614/CRAN.package.geodata}.

\bibitem[\citeproctext]{ref-hill1998comparison}
Hill, Donna. 1998. {``Comparison of median indicator kriging with full
indicator kriging in the analysis of spatial data''}.

\bibitem[\citeproctext]{ref-hodges2010adding}
Hodges, James S, e Brian J Reich. 2010. {``Adding spatially-correlated
errors can mess up the fixed effect you love''}. \emph{The American
Statistician} 64 (4): 325--34.

\bibitem[\citeproctext]{ref-hopkins1954new}
Hopkins, Brian, e John Gordon Skellam. 1954. {``A new method for
determining the type of distribution of plant individuals''}.
\emph{Annals of Botany} 18 (2): 213--27.

\bibitem[\citeproctext]{ref-hurvich1998regression}
Hurvich, Clifford M., Jeffrey S. Simonoff, e Chih-Ling Tsai. 1998.
{``Smoothing Parameter Selection in Nonparametric Regression Using an
Improved Akaike Information Criterion''}. \emph{Journal of the Royal
Statistical Society: Series B (Statistical Methodology)} 60 (2):
271--93.

\bibitem[\citeproctext]{ref-iliffe2000datums}
Iliffe, Jonathan. 2000. \emph{Datums and map projections for remote
sensing, GIS, and surveying}. CRC Press.

\bibitem[\citeproctext]{ref-illian2019spatial}
Illian, Janine B. 2019. {``Spatial and spatio-temporal point processes
in ecological applications''}. Em \emph{Handbook of environmental and
ecological statistics}, 97--131. Chapman; Hall/CRC.

\bibitem[\citeproctext]{ref-illian2008statistical}
Illian, Janine, Antti Penttinen, Helga Stoyan, e Dietrich Stoyan. 2008.
\emph{Statistical analysis and modelling of spatial point patterns}.
John Wiley \& Sons.

\bibitem[\citeproctext]{ref-isaaks1989applied}
Isaaks, Edward H, R Mohan Srivastava, et al. 1989. {``Applied
geostatistics''}.

\bibitem[\citeproctext]{ref-isham2010spatial}
Isham, Valerie. 2010. {``Spatial point process models''}. \emph{Handbook
of spatial statistics}, 283--98.

\bibitem[\citeproctext]{ref-jalilian2013decomposition}
Jalilian, Abdollah, Yongtao Guan, e Rasmus Waagepetersen. 2013.
{``Decomposition of variance for spatial Cox processes''}.
\emph{Scandinavian Journal of Statistics} 40 (1): 119--37.

\bibitem[\citeproctext]{ref-jalilian2020modeling}
Jalilian, Abdollah, Amir Safari, e Hormoz Sohrabi. 2020. {``Modeling
spatial patterns and species associations in a Hyrcanian forest using a
multivariate log-Gaussian Cox process''}. \emph{Journal of Statistical
Modelling: Theory and Applications} 1 (2): 59--76.

\bibitem[\citeproctext]{ref-janssen2009understanding}
Janssen, Volker. 2009. {``Understanding coordinate reference systems,
datums and transformations''}.

\bibitem[\citeproctext]{ref-ji2025ordered}
Ji, Guangjun, Zizhao Cai, Keyan Xiao, Yan Lu, e Qian Wang. 2025.
{``Ordered Indicator Kriging Interpolation Method with Field Variogram
Parameters for Discrete Variables in the Aquifers of Quaternary Loose
Sediments''}. \emph{Water} 17 (21): 3116.

\bibitem[\citeproctext]{ref-de1984extreme}
Jong, Peter de, C Sprenger, e Frans van Veen. 1984. {``On extreme values
of Moran's I and Geary's c''}. \emph{Geographical Analysis} 16 (1):
17--24.

\bibitem[\citeproctext]{ref-journel1986constrained}
Journel, Andre G. 1986. {``Constrained interpolation and qualitative
information---the soft kriging approach''}. \emph{Mathematical Geology}
18 (3): 269--86.

\bibitem[\citeproctext]{ref-journel1976mining}
Journel, Andre G, e Charles J Huijbregts. 1976. {``Mining
geostatistics''}.

\bibitem[\citeproctext]{ref-journel1983nonparametric}
Journel, André G. 1983. {``Nonparametric estimation of spatial
distributions''}. \emph{Journal of the International Association for
Mathematical Geology} 15 (3): 445--68.

\bibitem[\citeproctext]{ref-juang1998simple}
Juang, Kai-Wei, e Dar-Yuan Lee. 1998. {``Simple indicator kriging for
estimating the probability of incorrectly delineating hazardous areas in
a contaminated site''}. \emph{Environmental Science \& Technology} 32
(17): 2487--93.

\bibitem[\citeproctext]{ref-kaplan2017understanding}
Kaplan, Elliott D, e Christopher Hegarty. 2017. \emph{Understanding
GPS/GNSS: principles and applications}. Artech house.

\bibitem[\citeproctext]{ref-KeefeFerreiraFranck2018}
Keefe, Matthew J., Marcelo A. Ferreira, e Christopher T. Franck. 2018a.
{``On the formal specification of sum-zero constrained intrinsic
conditional autoregressive models''}. \emph{Spatial Statistics} 24:
54--65. \url{https://doi.org/10.1016/j.spasta.2018.02.005}.

\bibitem[\citeproctext]{ref-KeefeFerreiraFranck2019}
Keefe, Matthew J., Marco A. R. Ferreira, e Christopher T. Franck. 2019.
{``Objective Bayesian Analysis for Gaussian Hierarchical Models with
Intrinsic Conditional Autoregressive Priors''}. \emph{Bayesian Analysis}
14 (1): 181--209. \url{https://doi.org/10.1214/18-BA1107}.

\bibitem[\citeproctext]{ref-keefe2018formal}
Keefe, Matthew J, Marco AR Ferreira, e Christopher T Franck. 2018b.
{``On the formal specification of sum-zero constrained intrinsic
conditional autoregressive models''}. \emph{Spatial statistics} 24:
54--65.

\bibitem[\citeproctext]{ref-Keefe2019ObjectiveBayesianICAR}
Keefe, Michael J., Marco A. R. Ferreira, e Christopher T. Franck. 2019.
{``Objective Bayesian analysis for Gaussian hierarchical models with
intrinsic conditional autoregressive priors''}. \emph{Bayesian Analysis}
14 (1): 181--209. \url{https://doi.org/10.1214/18-BA1107}.

\bibitem[\citeproctext]{ref-kelejian1998generalized}
Kelejian, Harry H, e Ingmar R Prucha. 1998. {``A generalized spatial
two-stage least squares procedure for estimating a spatial
autoregressive model with autoregressive disturbances''}. \emph{The
journal of real estate finance and economics} 17 (1): 99--121.

\bibitem[\citeproctext]{ref-kelejian1999generalized}
---------. 1999. {``A generalized moments estimator for the
autoregressive parameter in a spatial model''}. \emph{International
economic review} 40 (2): 509--33.

\bibitem[\citeproctext]{ref-kelejian2010specification}
---------. 2010. {``Specification and estimation of spatial
autoregressive models with autoregressive and heteroskedastic
disturbances''}. \emph{Journal of econometrics} 157 (1): 53--67.

\bibitem[\citeproctext]{ref-kelejian2017spatial}
Kelejian, Harry, e Gianfranco Piras. 2017. \emph{Spatial econometrics}.
Academic Press.

\bibitem[\citeproctext]{ref-klier2008clustering}
Klier, Thomas, e Daniel P McMillen. 2008. {``Clustering of auto supplier
plants in the United States: generalized method of moments spatial logit
for large samples''}. \emph{Journal of Business \& Economic Statistics}
26 (4): 460--71.

\bibitem[\citeproctext]{ref-koc2022bandwidth}
Koç, Tufan. 2022. {``Bandwidth selection in geographically weighted
regression models via information complexity criteria''}. \emph{Journal
of Mathematics}.

\bibitem[\citeproctext]{ref-krige2005genesis}
Krige, Danie, e Wynand Kleingeld. 2005. {``The genesis of geostatistics
in gold and diamond industries''}. Em \emph{Space, Structure and
Randomness: Contributions in Honor of Georges Matheron in the Field of
Geostatistics, Random Sets and Mathematical Morphology}, 5--16.
Springer.

\bibitem[\citeproctext]{ref-langley1999dilution}
Langley, Richard B et al. 1999. {``Dilution of precision''}. \emph{GPS
world} 10 (5): 52--59.

\bibitem[\citeproctext]{ref-lark2000estimating}
Lark, RM. 2000. {``Estimating variograms of soil properties by the
method-of-moments and maximum likelihood''}. \emph{European Journal of
Soil Science} 51 (4): 717--28.

\bibitem[\citeproctext]{ref-laslett1994kriging}
Laslett, Geoffrey M. 1994. {``Kriging and splines: an empirical
comparison of their predictive performance in some applications''}.
\emph{Journal of the American Statistical Association} 89 (426):
391--400.

\bibitem[\citeproctext]{ref-leick2015gps}
Leick, Alfred, Lev Rapoport, e Dmitry Tatarnikov. 2015. \emph{GPS
satellite surveying}. John Wiley \& Sons.

\bibitem[\citeproctext]{ref-leininger2014bayesian}
Leininger, Thomas J. 2014. {``Bayesian analysis of spatial point
patterns''}. Tese de doutorado, Duke University.

\bibitem[\citeproctext]{ref-lesage1997bayesian}
LeSage, James P. 1997. {``Bayesian estimation of spatial autoregressive
models''}. \emph{International regional science review} 20 (1-2):
113--29.

\bibitem[\citeproctext]{ref-lesage2000bayesian}
---------. 2000. {``Bayesian estimation of limited dependent variable
spatial autoregressive models''}. \emph{Geographical Analysis} 32 (1):
19--35.

\bibitem[\citeproctext]{ref-lesage2009introduction}
LeSage, James, e Robert Kelley Pace. 2009. \emph{Introduction to spatial
econometrics}. Chapman; Hall/CRC.

\bibitem[\citeproctext]{ref-li1994simulation}
Li, Habin, e James F Reynolds. 1994. {``A simulation experiment to
quantify spatial heterogeneity in categorical maps''}. \emph{Ecology} 75
(8): 2446--55.

\bibitem[\citeproctext]{ref-liesenfeld2013analysis}
Liesenfeld, Roman, Jean-François Richard, e Jan Vogler. 2013.
{``Analysis of discrete dependent variable models with spatial
correlation''}. Economics Working Paper.

\bibitem[\citeproctext]{ref-de2005modelagem}
Lima, Renato Ribeiro de. 2005. {``Modelagem espa{ç}o temporal para dados
de incid{ê}ncia de doen{ç}as em plantas''}. Tese de doutorado, Tese
piracicaba-SP.

\bibitem[\citeproctext]{ref-lindgren2022spde}
Lindgren, Finn, David Bolin, e Håvard Rue. 2022. {``The SPDE approach
for Gaussian and non-Gaussian fields: 10 years and still running''}.
\emph{Spatial Statistics} 50: 100599.

\bibitem[\citeproctext]{ref-lindgren2015bayesian}
Lindgren, Finn, e Håvard Rue. 2015. {``Bayesian spatial modelling with
R-INLA''}. \emph{Journal of statistical software} 63: 1--25.

\bibitem[\citeproctext]{ref-lindgren2011explicit}
Lindgren, Finn, Håvard Rue, e Johan Lindström. 2011. {``An explicit link
between Gaussian fields and Gaussian Markov random fields: the
stochastic partial differential equation approach''}. \emph{Journal of
the Royal Statistical Society Series B: Statistical Methodology} 73 (4):
423--98.

\bibitem[\citeproctext]{ref-liu2023multiscale}
Liu, Jun, K. W. Chau, e Zhen Bao. 2023. {``Multiscale spatial analysis
of metro usage and its determinants for sustainable urban development in
Shenzhen, China''}. \emph{Tunnelling and Underground Space Technology}.
\url{https://doi.org/10.1016/j.tust.2022.104912}.

\bibitem[\citeproctext]{ref-loonis2018handbook}
Loonis, Vincent, e Marie-Pierre de Bellefon. 2018. {``Handbook of
spatial analysis: theory and application with R''}. \emph{Paris:
Eurostat, INSEE}, 394.

\bibitem[\citeproctext]{ref-fotheringham2002geographically}
Lu, Binbin, Martin Charlton, Paul Harris, e A Stewart Fotheringham.
2014. {``Geographically weighted regression with a non-Euclidean
distance metric: a case study using hedonic house price data''}.
\emph{International Journal of Geographical Information Science} 28 (4):
660--81.

\bibitem[\citeproctext]{ref-lu2023uncovering}
Lu, Bo, Yong Ge, Yeqiao Shi, Jing Zheng, e Paul Harris. 2023.
{``Uncovering drivers of community-level house price dynamics through
multiscale geographically weighted regression: A case study of Wuhan,
China''}. \emph{Spatial Statistics} 53: 100723.
\url{https://doi.org/10.1016/j.spasta.2022.100723}.

\bibitem[\citeproctext]{ref-mallows1995more}
Mallows, Cohn L. 1995. {``More comments on Cp''}. \emph{Technometrics}
37 (4): 362--72.

\bibitem[\citeproctext]{ref-Mallows1973}
Mallows, Colin L. 1973. {``Some Comments on {\(C_p\)}''}.
\emph{Technometrics} 15 (4): 661--75.
\url{https://doi.org/10.1080/00401706.1973.10489103}.

\bibitem[\citeproctext]{ref-marchant2007robust}
Marchant, BP, e RM Lark. 2007. {``Robust estimation of the variogram by
residual maximum likelihood''}. \emph{Geoderma} 140 (1-2): 62--72.

\bibitem[\citeproctext]{ref-martinetti2017approximate}
Martinetti, Davide, e Ghislain Geniaux. 2017. {``Approximate likelihood
estimation of spatial probit models''}. \emph{Regional Science and Urban
Economics} 64: 30--45.

\bibitem[\citeproctext]{ref-mateu2022introduction}
Mateu, Jorge, e Ramón Giraldo. 2022. {``Introduction to Geostatistical
Functional Data Analysis''}. \emph{Geostatistical Functional Data
Analysis}, 1--25.

\bibitem[\citeproctext]{ref-mateus2013}
Mateus, Ana Lúcia Souza Silva. 2013. {``Proposição de novas metodologias
para análise de aleatoriedade em processos pontuais no espaço-tempo''}.
Tese (Doutorado em Estatística e Experimentação Agropecuária), Lavras:
Universidade Federal de Lavras.

\bibitem[\citeproctext]{ref-matheron1963principles}
Matheron, G. 1963. {``" Principles of geostatistics", Economic Geology,
58, pp 1246-1266''}.

\bibitem[\citeproctext]{ref-matheron1971theory}
Matheron, George. 1971. {``The theory of regionalised variables and its
applications''}. \emph{Les Cahiers du Centre de Morphologie
Math{é}matique} 5: 212.

\bibitem[\citeproctext]{ref-mcbratney1986choosing}
McBratney, AB, e R Webster. 1986. {``Choosing functions for
semi-variograms of soil properties and fitting them to sampling
estimates''}. \emph{Journal of soil Science} 37 (4): 617--39.

\bibitem[\citeproctext]{ref-mcmillen1992probit}
McMillen, Daniel P. 1992. {``Probit with spatial autocorrelation''}.
\emph{Journal of Regional Science} 32 (3): 335--48.

\bibitem[\citeproctext]{ref-miao2025spatial}
Miao, Xin, Fang Fang, Xuening Zhu, e Hansheng Wang. 2025. {``Spatial
weights matrix selection and model averaging for multivariate spatial
autoregressive models''}. \emph{Econometric Reviews}, 1--31.

\bibitem[\citeproctext]{ref-mocnik2023we}
Mocnik, Franz-Benjamin. 2023. {``Why we can read maps''}.
\emph{Cartography and Geographic Information Science} 50 (1): 1--19.

\bibitem[\citeproctext]{ref-mohammadpour2019geochemical}
Mohammadpour, Mahyadin, Abbas Bahroudi, Maysam Abedi, Gholamreza
Rahimipour, Golnaz Jozanikohan, e Farzaneh Mami Khalifani. 2019.
{``Geochemical distribution mapping by combining number-size
multifractal model and multiple indicator kriging''}. \emph{Journal of
Geochemical Exploration} 200: 13--26.

\bibitem[\citeproctext]{ref-moller2003statistical}
Moller, Jesper, e Rasmus Plenge Waagepetersen. 2003. \emph{Statistical
inference and simulation for spatial point processes}. CRC press.

\bibitem[\citeproctext]{ref-moller1998log}
Møller, Jesper, Anne Randi Syversveen, e Rasmus Plenge Waagepetersen.
1998. {``Log gaussian cox processes''}. \emph{Scandinavian journal of
statistics} 25 (3): 451--82.

\bibitem[\citeproctext]{ref-moller2007modern}
Møller, Jesper, e Rasmus P Waagepetersen. 2007. {``Modern statistics for
spatial point processes''}. \emph{Scandinavian Journal of Statistics} 34
(4): 643--84.

\bibitem[\citeproctext]{ref-monmonier2005lying}
Monmonier, Mark. 2005. {``Lying with maps''}. \emph{Statistical
science}, 215--22.

\bibitem[\citeproctext]{ref-moraga2023spatial}
Moraga, Paula. 2023. \emph{Spatial Statistics for Data Science: Theory
and Practice with R}. CRC Press.

\bibitem[\citeproctext]{ref-moran1950notes}
Moran, Patrick AP. 1950. {``Notes on continuous stochastic phenomena''}.
\emph{Biometrika} 37 (1/2): 17--23.

\bibitem[\citeproctext]{ref-moreno2023functional}
Moreno, Alvaro Alexander Burbano et al. 2023. {``Functional data
analysis: spatial association of curves and irregular spacing''}.

\bibitem[\citeproctext]{ref-morisita1959measuring}
Morisita, Masaaki. 1959. {``Measuring of the dispersion of individuals
and analysis of the distributional patterns''}. \emph{Memoirs of the
Faculty of Science, E 2, Kyushu Univ. Ser.}

\bibitem[\citeproctext]{ref-myers1994spatial}
Myers, Donald E. 1994. {``Spatial interpolation: an overview''}.
\emph{Geoderma} 62 (1-3): 17--28.

\bibitem[\citeproctext]{ref-nhancololo2024processos}
Nhancololo, A. M. 2024a. {``Processos pontuais espaciais univariados
aplicados à distribuição de espécies arbóreas em florestas naturais''}.
Dissertação (Mestrado em Estatística e Experimentação Agropecuária),
Lavras: Universidade Federal de Lavras.

\bibitem[\citeproctext]{ref-nhancololo2024}
---------. 2024b. {``Processos pontuais espaciais univariados aplicados
à distribuição de espécies arbóreas em florestas naturais''}.
Dissertação (Mestrado em Estatística e Experimentação Agropecuária),
Lavras: Universidade Federal de Lavras.

\bibitem[\citeproctext]{ref-nhancololo2024comparison}
Nhancololo, A. M., Wélson A. Oliveira, Fernandes A. C. Pereira, Bruno
Montoani Silva, e João Domingos Scalon. 2024. {``Comparison between the
laboratory method and the Stolf penetrometer in soil density analysis: a
study using geostatistical approaches''}. \emph{Sigmae} 13 (1): 63--78.
\url{https://doi.org/10.29327/2520355.13.1-7}.

\bibitem[\citeproctext]{ref-nightingale2019area}
Nightingale, Glenna F, Janine B Illian, Ruth King, e Peter Nightingale.
2019. {``Area interaction point processes for bivariate point patterns
in a Bayesian context''}. \emph{Journal of Environmental Statistics} 9
(2).

\bibitem[\citeproctext]{ref-okabe2006bayesian}
Okabe, Masahilo, e Masaharu Tanemura. 2006. {``Bayesian Estimation of
Soft-Core Potential Models for Spatial Point Patterns''}. \emph{Journal
of the Japan Statistical Society} 36 (2): 121--47.

\bibitem[\citeproctext]{ref-oliver2014tutorial}
Oliver, MA, e R Webster. 2014. {``A tutorial guide to geostatistics:
Computing and modelling variograms and kriging''}. \emph{Catena} 113:
56--69.

\bibitem[\citeproctext]{ref-openshaw1984modifiable}
Openshaw, Stan. 1984. {``The modifiable areal unit problem''}.
\emph{Concepts and techniques in modern geography}.

\bibitem[\citeproctext]{ref-oshan2020targeting}
Oshan, Taylor M., James P. Smith, e A. Stewart Fotheringham. 2020.
{``Targeting the spatial context of obesity determinants via multiscale
geographically weighted regression''}. \emph{International Journal of
Health Geographics} 19 (1): 11.
\url{https://doi.org/10.1186/s12942-020-00204-8}.

\bibitem[\citeproctext]{ref-paradis2005r}
Paradis, Emmanuel. 2005. \emph{R for Beginners}.
\url{https://cran.r-project.org/doc/contrib/Paradis-rdebuts_en.pdf}.

\bibitem[\citeproctext]{ref-Paula2013DiagnosticsDGLM}
Paula, Gilberto A. 2013. {``On diagnostics in double generalized linear
models''}. \emph{Computational Statistics \& Data Analysis} 68: 44--51.
\url{https://doi.org/10.1016/j.csda.2013.07.003}.

\bibitem[\citeproctext]{ref-Paula2025ModelosRegressao}
---------. 2025. \emph{Modelos de Regress{ã}o e Aplica{ç}{õ}es}. S{ã}o
Paulo: Instituto de Matem{á}tica e Estat{ı́}stica, Universidade de S{ã}o
Paulo.

\bibitem[\citeproctext]{ref-pebesma2018simple}
Pebesma, Edzer. 2018. {``Simple features for R: standardized support for
spatial vector data''}.

\bibitem[\citeproctext]{ref-pebesma2004multivariable}
Pebesma, Edzer J. 2004. {``Multivariable geostatistics in S: the gstat
package''}. \emph{Computers \& geosciences} 30 (7): 683--91.

\bibitem[\citeproctext]{ref-pebesma2023spatial}
Pebesma, Edzer, e Roger Bivand. 2023. \emph{Spatial data science: With
applications in R}. Chapman; Hall/CRC.

\bibitem[\citeproctext]{ref-peng2024multiscale}
Peng, Zhan, e Ryo Inoue. 2024. {``Multiscale Continuous and Discrete
Spatial Heterogeneity Analysis: The Development of a Local Model
Combining Eigenvector Spatial Filters and Generalized Lasso
Penalties''}. \emph{Geographical Analysis} 56 (2): 303--27.

\bibitem[\citeproctext]{ref-pereira2024geobr}
Pereira, Rafael HM, e Caio Nogueira Goncalves. 2024. {``geobr: download
official spatial data sets of Brazil''}. \emph{R package version} 1 (0):
18.

\bibitem[\citeproctext]{ref-pinkse1998contracting}
Pinkse, Joris, e Margaret E Slade. 1998. {``Contracting in space: An
application of spatial statistics to discrete-choice models''}.
\emph{Journal of Econometrics} 85 (1): 125--54.

\bibitem[\citeproctext]{ref-prestby2025trust}
Prestby, Timothy J. 2025. {``Trust in maps: What we know and what we
need to know''}. \emph{Cartography and Geographic Information Science}
52 (1): 1--18.

\bibitem[\citeproctext]{ref-ramsay2005functional}
Ramsay, James O, e Bernard W Silverman. 2005. \emph{Functional data
analysis}. Springer.

\bibitem[\citeproctext]{ref-reich2006effects}
Reich, Brian J, James S Hodges, e Vesna Zadnik. 2006. {``Effects of
residual smoothing on the posterior of the fixed effects in
disease-mapping models''}. \emph{Biometrics} 62 (4): 1197--1206.

\bibitem[\citeproctext]{ref-ribeiro2006geor}
Ribeiro Jr, Paulo J, e Peter J Diggle. 2006. {``geoR: Package for
Geostatistical Data Analysis an illustrative session''}.
\emph{Artificial Intelligence} 1: 1--24.

\bibitem[\citeproctext]{ref-geoR}
Ribeiro Jr, Paulo Justiniano, e Peter Diggle. 2025. \emph{geoR: Analysis
of Geostatistical Data}.
\url{https://doi.org/10.32614/CRAN.package.geoR}.

\bibitem[\citeproctext]{ref-richardson1995robust}
Richardson, Alice M, e Alan H Welsh. 1995. {``Robust restricted maximum
likelihood in mixed linear models''}. \emph{Biometrics}, 1429--39.

\bibitem[\citeproctext]{ref-riebler2016intuitive}
Riebler, Andrea, Sigrunn H Sørbye, Daniel Simpson, e Håvard Rue. 2016.
{``An intuitive Bayesian spatial model for disease mapping that accounts
for scaling''}. \emph{Statistical methods in medical research} 25 (4):
1145--65.

\bibitem[\citeproctext]{ref-Rigby2019DistributionsGAMLSS}
Rigby, Robert A., Mikis D. Stasinopoulos, Gillian Z. Heller, e Frosso De
Bastiani. 2019. \emph{Distributions for Modeling Location, Scale, and
Shape: Using {GAMLSS} in {R}}. Boca Raton: Chapman; Hall/CRC.

\bibitem[\citeproctext]{ref-rue2005gaussian}
Rue, Havard, e Leonhard Held. 2005. \emph{Gaussian Markov random fields:
theory and applications}. Chapman; Hall/CRC.

\bibitem[\citeproctext]{ref-RueHeld2005}
Rue, Håvard, e Leonhard Held. 2005. \emph{Gaussian Markov Random Fields:
Theory and Applications}. Chapman \& Hall/CRC Monographs on Statistics
\& Applied Probability. Boca Raton: Chapman; Hall/CRC.

\bibitem[\citeproctext]{ref-rue2009approximate}
Rue, Håvard, Sara Martino, e Nicolas Chopin. 2009. {``Approximate
Bayesian inference for latent Gaussian models by using integrated nested
Laplace approximations''}. \emph{Journal of the Royal Statistical
Society Series B: Statistical Methodology} 71 (2): 319--92.

\bibitem[\citeproctext]{ref-sachdeva2022do}
Sachdeva, Manish, A. Stewart Fotheringham, e Zhen Li. 2022. {``Do places
have value? {Q}uantifying the intrinsic value of housing neighborhoods
using {MGWR}''}. \emph{Journal of Housing Research} 31 (1): 24--52.

\bibitem[\citeproctext]{ref-sahu2022bayesian}
Sahu, Sujit. 2022. \emph{Bayesian modeling of spatio-temporal data with
R}. Chapman; Hall/CRC.

\bibitem[\citeproctext]{ref-scalon2024analise}
Scalon, João Domingos. 2024. \emph{Análise de Dados Espaciais com
Aplicações em R}. Lavras: Ed. UFLA.

\bibitem[\citeproctext]{ref-scalon2022statistical}
Scalon, João Domingos, Victor Ferreira da Silva, Wélson Antônio de
Oliveira, e Mateus Santos Peixoto. 2022. {``Statistical characterization
of spatial and size distributions of particles in composite materials
used in the manufacturing of biomedical instruments''}. \emph{Brazilian
Journal of Biometrics} 40 (4): 428--41.

\bibitem[\citeproctext]{ref-schmidt2003bayesian}
Schmidt, Alexandra M, e Anthony O'Hagan. 2003. {``Bayesian inference for
non-stationary spatial covariance structure via spatial deformations''}.
\emph{Journal of the Royal Statistical Society Series B: Statistical
Methodology} 65 (3): 743--58.

\bibitem[\citeproctext]{ref-Shareef2015LightningSearch}
Shareef, Hitham, A. A. Ibrahim, e A. H. Mutlag. 2015. {``Lightning
search algorithm''}. \emph{Applied Soft Computing} 36: 315--33.

\bibitem[\citeproctext]{ref-simpson2017penalising}
Simpson, Daniel, Håvard Rue, Andrea Riebler, Thiago G Martins, e Sigrunn
H Sørbye. 2017. {``Penalising model component complexity: A principled,
practical approach to constructing priors''}.

\bibitem[\citeproctext]{ref-ggrepel}
Slowikowski, Kamil. 2024. \emph{ggrepel: Automatically Position
Non-Overlapping Text Labels with 'ggplot2'}.
\url{https://doi.org/10.32614/CRAN.package.ggrepel}.

\bibitem[\citeproctext]{ref-snyder1987map}
Snyder, John Parr. 1987. \emph{Map projections--A working manual}. Vol.
1395. US Government Printing Office.

\bibitem[\citeproctext]{ref-sorbye2017penalised}
Sørbye, Sigrunn Holbek, e Håvard Rue. 2017. {``Penalised complexity
priors for stationary autoregressive processes''}. \emph{Journal of Time
Series Analysis} 38 (6): 923--35.

\bibitem[\citeproctext]{ref-sorbye2014scaling}
Sørbye, Sigrunn H., e Håvard Rue. 2014. {``Scaling intrinsic Gaussian
Markov random field priors in spatial modelling''}. \emph{Spatial
Statistics} 8: 39--51.
\url{https://doi.org/10.1016/j.spasta.2013.06.004}.

\bibitem[\citeproctext]{ref-Stasinopoulos2024GAMLSS}
Stasinopoulos, Mikis D., Thomas Kneib, Nadja Klein, Andreas Mayr, e
Gillian Z. Heller. 2024. \emph{Generalized Additive Models for Location,
Scale and Shape: A Distributional Regression Approach, with
Applications}. Vol. 56. Cambridge: Cambridge University Press.

\bibitem[\citeproctext]{ref-Stasinopoulos2017GAMLSS}
Stasinopoulos, Mikis D., Robert A. Rigby, Gillian Z. Heller, Vlasios
Voudouris, e Frosso De Bastiani. 2017. \emph{Flexible Regression and
Smoothing: Using {GAMLSS} in {R}}. Chapman \& Hall/CRC Texts em
Statistical Science. Boca Raton: CRC Press.

\bibitem[\citeproctext]{ref-ggmapinset}
Suster, Carl. 2024. \emph{ggmapinset: Add Inset Panels to Maps}.
\url{https://doi.org/10.32614/CRAN.package.ggmapinset}.

\bibitem[\citeproctext]{ref-tiefelsdorf2000modelling}
Tiefelsdorf, Michael. 2000. \emph{Modelling spatial processes: the
identification and analysis of spatial relationships in regression
residuals by means of Moran's I}. Springer.

\bibitem[\citeproctext]{ref-tobias2024note}
Tobias, Justin L. 2024. {``A Note on the Cowles-EM Algorithm for
Bayesian Ordinal Probit Models''}. \emph{Research in Statistics} 3 (1):
2476409.

\bibitem[\citeproctext]{ref-tobler1970computer}
Tobler, Waldo R. 1970. {``A computer movie simulating urban growth in
the Detroit region''}. \emph{Economic geography} 46 (sup1): 234--40.

\bibitem[\citeproctext]{ref-ungaro2008arsenic}
Ungaro, F, F Ragazzi, R Cappellin, e P Giandon. 2008. {``Arsenic
concentration in the soils of the Brenta Plain (Northern Italy): mapping
the probability of exceeding contamination thresholds''}. \emph{Journal
of Geochemical Exploration} 96 (2-3): 117--31.

\bibitem[\citeproctext]{ref-unwin1974statistical}
Unwin, David J, e Leslie W Hepple. 1974. {``The statistical analysis of
spatial series''}. \emph{Journal of the Royal Statistical Society:
Series D (The Statistician)} 23 (3-4): 211--27.

\bibitem[\citeproctext]{ref-van1996nonparametric}
Van Lieshout, MNM, e AJ1422574 Baddeley. 1996. {``A nonparametric
measure of spatial interaction in point patterns''}. \emph{Statistica
Neerlandica} 50 (3): 344--61.

\bibitem[\citeproctext]{ref-vanicek2015geodesy}
Vanicek, Petr, e Edward J Krakiwsky. 2015. \emph{Geodesy: the concepts}.
Elsevier.

\bibitem[\citeproctext]{ref-venables2015notes}
Venables, W. N., D. M. Smith, e R Core Team. 2015. \emph{An Introduction
to R}. Vienna, Austria: R Foundation for Statistical Computing.
\url{https://cran.r-project.org/doc/manuals/r-release/R-intro.pdf}.

\bibitem[\citeproctext]{ref-ver1998constructing}
Ver Hoef, Jay M, e Ronald Paul Barry. 1998. {``Constructing and fitting
models for cokriging and multivariable spatial prediction''}.
\emph{Journal of Statistical Planning and Inference} 69 (2): 275--94.

\bibitem[\citeproctext]{ref-ver2018relationship}
Ver Hoef, Jay M, Ephraim M Hanks, e Mevin B Hooten. 2018. {``On the
relationship between conditional (CAR) and simultaneous (SAR)
autoregressive models''}. \emph{Spatial statistics} 25: 68--85.

\bibitem[\citeproctext]{ref-waagepetersen2009two}
Waagepetersen, Rasmus, e Yongtao Guan. 2009. {``Two-step estimation for
inhomogeneous spatial point processes''}. \emph{Journal of the Royal
Statistical Society Series B: Statistical Methodology} 71 (3): 685--702.

\bibitem[\citeproctext]{ref-wackernagel2003multivariate}
Wackernagel, Hans. 2003. \emph{Multivariate geostatistics: an
introduction with applications}. Springer Science \& Business Media.

\bibitem[\citeproctext]{ref-wagner2005spatial}
Wagner, Helene H, e Marie-Josée Fortin. 2005. {``Spatial analysis of
landscapes: concepts and statistics''}. \emph{Ecology} 86 (8): 1975--87.

\bibitem[\citeproctext]{ref-wall2004close}
Wall, Melanie M. 2004. {``A close look at the spatial structure implied
by the CAR and SAR models''}. \emph{Journal of statistical planning and
inference} 121 (2): 311--24.

\bibitem[\citeproctext]{ref-waller2024maps}
Waller, Lance A. 2024. {``Maps: A Statistical View''}. \emph{Annual
Review of Statistics and Its Application} 11.

\bibitem[\citeproctext]{ref-wang2016functional}
Wang, Jane-Ling, Jeng-Min Chiou, e Hans-Georg Müller. 2016.
{``Functional data analysis''}. \emph{Annual Review of Statistics and
its application} 3 (1): 257--95.

\bibitem[\citeproctext]{ref-wang2009baysian}
Wang, Xiaokun, e Kara M Kockelman. 2009. {``Baysian inference for
ordered response data with a dynamic spatial-ordered probit model''}.
\emph{Journal of Regional Science} 49 (5): 877--913.

\bibitem[\citeproctext]{ref-wheeler2005multicollinearity}
Wheeler, David, e Michael Tiefelsdorf. 2005. {``Multicollinearity and
correlation among local regression coefficients in geographically
weighted regression''}. \emph{Journal of Geographical Systems} 7 (2):
161--87. \url{https://doi.org/10.1007/s10109-005-0155-6}.

\bibitem[\citeproctext]{ref-whittle1954stationary}
Whittle, Peter. 1954. {``On stationary processes in the plane''}.
\emph{Biometrika}, 434--49.

\bibitem[\citeproctext]{ref-wickham2016ggplot2}
Wickham, Hadley. 2016. \emph{ggplot2: Elegant Graphics for Data
Analysis}. 2º ed. Springer-Verlag New York.

\bibitem[\citeproctext]{ref-wickham2019advanced}
---------. 2019. \emph{Advanced R}. 2º ed. CRC Press.

\bibitem[\citeproctext]{ref-wickham2023r}
Wickham, Hadley, Mine Çetinkaya-Rundel, e Garrett Grolemund. 2023.
\emph{R for Data Science: Import, Tidy, Transform, Visualize, and Model
Data}. 2º ed. O'Reilly Media.

\bibitem[\citeproctext]{ref-wiegand2014handbook}
Wiegand, Thorsten, e Kirk A Moloney. 2014. \emph{Handbook of spatial
point-pattern analysis in ecology}. CRC press.

\bibitem[\citeproctext]{ref-wild2017statistics}
Wild, Christopher J, Jessica M Utts, e Nicholas J Horton. 2017. {``What
is statistics?''} Em \emph{International handbook of research in
statistics education}, 5--36. Springer.

\bibitem[\citeproctext]{ref-wilhelm2013estimating}
Wilhelm, Stefan, e Miguel Godinho de Matos. 2013. {``Estimating spatial
probit models in R''}.

\bibitem[\citeproctext]{ref-wu2018multiscale}
Wu, Chengdong, Feng Ren, Wei Hu, e Qingyun Du. 2019. {``Multiscale
geographically and temporally weighted regression: Exploring the
spatiotemporal determinants of housing prices''}. \emph{International
Journal of Geographical Information Science} 33 (3): 489--511.
\url{https://doi.org/10.1080/13658816.2018.1528246}.

\bibitem[\citeproctext]{ref-yamamoto2013geoestatistica}
Yamamoto, Jorge Kazuo, e Paulo M. Barbosa Landim. 2013.
\emph{Geoestatística: conceitos e aplicações}. São Paulo: Oficina de
Textos.

\bibitem[\citeproctext]{ref-yang2011extension}
Yang, Wenbai, A. Stewart Fotheringham, e Paul Harris. 2011. {``An
Extension of Geographically Weighted Regression with Flexible
Bandwidths''}. Em \emph{Proceedings of an International Conference on
Spatial Analysis and Modelling}. St Andrews, Scotland, UK.

\bibitem[\citeproctext]{ref-Yang2014FlexibleBandwidthGWR}
Yang, Wenbo. 2014. {``An extension of geographically weighted regression
with flexible bandwidths''}. Tese de doutorado, University of St
Andrews.

\bibitem[\citeproctext]{ref-yu2019inference}
Yu, Hanchen, A. Stewart Fotheringham, Zhen Li, Taylor M. Oshan, Wei
Kang, e Levi J. Wolf. 2020. {``Inference in multiscale geographically
weighted regression''}. \emph{Geographical Analysis} 52 (1): 87--106.

\bibitem[\citeproctext]{ref-fotheringham2020}
Yu, Huili, A. Stewart Fotheringham, Zhenlong Li, Taylor M. Oshan, e Levi
J. Wolf. 2020. {``On the measurement of bias in geographically weighted
regression models''}. \emph{Spatial Statistics} 38: 100453.
\url{https://doi.org/10.1016/j.spasta.2020.100453}.

\bibitem[\citeproctext]{ref-zhang2018spatial}
Zhang, Xinyu, e Jihai Yu. 2018. {``Spatial weights matrix selection and
model averaging for spatial autoregressive models''}. \emph{Journal of
Econometrics} 203 (1): 1--18.

\end{CSLReferences}




\end{document}
